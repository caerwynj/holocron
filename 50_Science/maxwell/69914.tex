%%%%%%%%%%%%%%%%%%%%%%%%%%%%%%%%%%%%%%%%%%%%%%%%%%%%%%%%%%%%%%%%%%%%%%%%%%%%
%                                                                          %
% This is the LaTeX source of a Project Gutenberg eBook, online at         %
% https://www.gutenberg.org. Please find the license for this eBook at     %
% https://www.gutenberg.org/license                                        %
%                                                                          %
%%%%%%%%%%%%%%%%%%%%%%%%%%%%%%%%%%%%%%%%%%%%%%%%%%%%%%%%%%%%%%%%%%%%%%%%%%%%
%%%%%%%%%%%%%%%%%%%%%%%%%%%%%%%%%%%%%%%%%%%%%%%%%%%%%%%%%%%%%%%%%%%%%%%%%%%%
%                                                                          %
% Producer's Comments:                                                     %
%                                                                          %
% Use lualatex so we can have modern fonts.                                %
%                                                                          %
% Running twice is required for long table and hyperref                    %
% Or use latexmk -lualatex                                                 %
%                                                                          %
% No overfull hboxes                                                       %
% 3 underfull hboxes                                                       %
%                                                                          %
% To avoid many overfull hboxes in lines with in-line maths allow more     %
% inter-word spacing using emergencystretch 3em.                           %
% This results in many underfull hboxes, suppress the warnings for most    %
% of these using hbadness=5000. Only the worst remain.                     %
%                                                                          %
%%%%%%%%%%%%%%%%%%%%%%%%%%%%%%%%%%%%%%%%%%%%%%%%%%%%%%%%%%%%%%%%%%%%%%%%%%%%

\listfiles
\documentclass[12pt,oneside]{book}[2021/10/04]

\usepackage{wrapfig2}[2022-01-26] % Wrap text around figures
% wrapfig2 must be before math packages
\usepackage{unicode-math}[2020/01/31]
\usepackage{mathtools}[2021/02/02] %% for multlined and shortintertext

\setmainfont{XITS}
\setmathfont{XITS Math}

% font for polytonic greek
\newfontfamily\greekfont{GFS Didot}[Scale=MatchLowercase]
\newfontfamily\gfont{QTCloisteredMonk}
\newfontfamily\sansfont{DejaVu Sans}

\setlength{\paperwidth}{14.87cm}%
\setlength{\paperheight}{21.02cm}%

\usepackage[
  top=0.8cm,
  bottom=0.5cm,
  left=0.81cm,
  right=0.81cm,
  headheight=17pt, % as per the warning by fancyhdr
  includehead,includefoot,
  heightrounded, % to avoid spurious underfull messages
]{geometry}[2020/01/02]% Page Geometry

\usepackage{letterspace}[2021/12/10] % letterspacing -- \textls
\usepackage{siunitx}[2022-02-02] % decimal aligned table entries
\usepackage{enumitem}[2019/06/20] % Customized lists
\usepackage[perpage]{footmisc}[2011/06/06] % for footnotes anchors
\usepackage{multirow}[2021/03/15] % for tables with a cell spanning 2+ rows in other columns
\usepackage{bigstrut}[2021/03/15] % for multirow entries in table
\usepackage{caption}[2020/01/03] % for table captions
\usepackage{float}[2001/11/08] % for H option with table
% need parskip. It doesn't have effect here because skip=0 but it prevents
% interaction with titles when skip is changed in pg boilerplate
\usepackage[skip=0pt plus1pt, indent=1em]{parskip}[2021-03-14] % parskip adjustments
\usepackage{fancyhdr}[2019/01/31] % for page headers
\usepackage{indentfirst}[1995/11/23] % indent first paragraph in sections
\usepackage{flafter}[2021/07/31] % force floats to appear after declaration, not before
\usepackage[version=4]{mhchem}[2021/12/31] % chemical formulae
\usepackage{longtable}[2021-09-01] % for multi-page tables
\usepackage{graphicx}[2021/09/16] % for images and scalebox
\graphicspath{ {./images/} }
\usepackage{array}[2021/10/04] % Tabular extension package - for prefix notation in tabular
\usepackage{lettrine}[2020-03-14] % for drop caps
\usepackage{xparse}[2022-01-12] % document command parser - \NewDocumentCommand

\usepackage[
  pdfpagelayout=SinglePage,
  pdfdisplaydoctitle,
  colorlinks=true,
  linkcolor=blue,
]{hyperref}[2021-06-07] % Hypertext links for LaTeX

\setlength{\footnotemargin}{1em}
% add space between marker and footnote
\let\oldfootnote\footnote
\renewcommand\footnote[1]{%
\oldfootnote{\hspace{0.14em}#1}}
\let\oldfootnotetext\footnotetext
\renewcommand\footnotetext[1]{%
\oldfootnotetext{\hspace{0.18em}#1}}

% redefine as mathord instead of mathbin
\DeclareMathSymbol{·}{\mathord}{symbols}{"00B7}
\sisetup{output-decimal-marker = {·}, group-digits = none}

\newcommand{\ldot}{\,.\,}
\newcommand{\rD}{\mathrm{D}}

\renewcommand{\headrulewidth}{0pt}

\newcommand{\Heading}{\centering\normalfont}
\newcommand{\Chapter}[1]{\section*{\large\Heading #1}}
\newcommand{\Section}[1]{\subsection*{\normalsize\Heading\scshape #1}}
\newcommand{\Subsection}[1]{\subsection*{\normalsize\Heading\itshape #1}}

\newcommand{\AdvSection}[1]{\subsection*{\Heading\large\bfseries\textls{#1}}}

% no header on page with only floats (need p option infigure)
\newcommand{\Runhead}[1]{\fancyhead[C]{\iffloatpage{}{\small#1}}}
\newcommand{\Subheading}[1]{\begin{center}\small #1 \end{center}}

\newcommand{\artlabel}[1]{\phantomsection \label{art:#1}{}}
\newcommand{\article}[1]{\phantomsection \label{art:#1}{#1.]}}

\renewcommand{\thefootnote}{\fnsymbol{footnote}} % use * etc. for footnote anchors

\newcommand{\singleline}{
  \hspace*{0.33\textwidth}{\rule{0.33\textwidth}{0.4pt}}
}

\newcommand{\longdash}{\rule[.5ex]{2em}{1.3pt}
}

\newcommand{\shortrule}{\rule[1.4pt]{.07\textwidth}{.6pt}}
\newcommand{\decorline}{
  \begin{center}
  \vbox{\shortrule\tiny ⬥⬥\!\shortrule}
  \end{center}
}

\newcommand{\hangpara}[1]{\hangindent=2em {\itshape #1} \vspace{1ex}}

% strut for dfrac
\newcommand{\xp}{\rule[-2.0ex]{0pt}{5.2ex}}
% + extra at bottom
\newcommand{\xbp}{\rule[-2.4ex]{0pt}{5.6ex}}
% for frac inline
\newcommand{\tstrut}{\rule[-1.5ex]{0pt}{4.2ex}}

\newcommand{\wrapfig}[3]{
\begin{wrapfigure}{r}{#1\textwidth}
\centering
\includegraphics[width=#1\textwidth]{#2}
\caption*{\small #3}
\end{wrapfigure}}

\newcommand{\widefig}[3]{
\begin{figure}[ht!]
\centering
\includegraphics[width=#1\textwidth]{#2}
\caption*{\small #3}
\end{figure}}

\newcommand{\plate}[4]{
\newpage
\thispagestyle{empty}
\begin{figure}[htp!]
\centering
\caption*{#3}
\includegraphics[width=#1\textwidth]{#2}
\caption*{#4}
\end{figure}}

\newcommand{\plategeometry}{
\newgeometry{
  top=0cm,
  bottom=0cm,
  left=0.81cm,
  right=0.81cm,
  headheight=0.1pt,
  includehead,includefoot,
  heightrounded, % to avoid spurious underfull messages
}}

\newcommand{\lowditto}[1]{\makebox[\widthof{#1}][c]{„}}
\newcommand{\dittoexp}{\lowditto{Exp.}}
\newcommand{\dittopage}{\lowditto{Page}}

% the table of contents: for multi-line entries need article# at top and page# at bottom
% tabular and longtable can't do this easily. tabularray can but can only expand one macro and is very slow
% asterisks cause difficult alignment for article#, insert phantom 0s and align left
% use longtable with hack for multilines.
\newcommand{\stocentry}[2]{\hphantom{0}\hphantom{0}#1. & #2 \nolinebreak\dotfill & \pageref{art:#1}\\}
% for 2 figure article #s
\NewDocumentCommand{\dtocent}{mmo}{\hphantom{0}#1.\IfValueT{#3}{#3} & #2 \nolinebreak\dotfill & \pageref{art:#1}\\}
% for 3 figure
\NewDocumentCommand{\ttocent}{mmo}{#1.\IfValueT{#3}{#3} & #2 \nolinebreak\dotfill & \pageref{art:#1}\\}
% for no article number
\newcommand{\ntocent}[2]{& #2 \nolinebreak\dotfill & \pageref{art:#1}\\}

\newcommand{\struta}{\rule[-1.0ex]{0pt}{5.2ex}}
\newcommand{\strutb}{\rule[-1.5ex]{0pt}{4.2ex}}
\newcommand{\tocchap}[1]{&\multicolumn{1}{c}{\struta #1}&\\}
\newcommand{\tocsubtitle}[1]{&\multicolumn{1}{p{10.6cm}}{\strutb \centering \textls{\scriptsize #1}}&\\}
\newcommand{\toclistitem}[1]{& \multicolumn{2}{p{10.5cm}}{\hangindent=5em \hspace{2em} #1}\\}

\emergencystretch 3em
\captionsetup{aboveskip=3pt, belowskip=0pt}

\newenvironment{advlist}{
  \begin{description}[leftmargin=1em, parsep=0.2ex, listparindent=1em,]
}{\end{description}}

\newenvironment{vollist}{
\begin{description}[nosep, topsep=-1ex, itemindent=-1em, leftmargin=2em]
}{\end{description}}

\newcommand{\¬}{\hphantom{0}}
\newcommand{\newchapter}{\newpage\thispagestyle{empty}}
\hbadness=5000


\usepackage{fontspec}[2022/01/15] % for font selection
\usepackage{array}[2021/10/04] % for >{\raggedright\arraybackslash} in tabular
\usepackage{fancyhdr}[2019/01/31] % for page headers
\usepackage{graphicx}[2021/09/16] % for images and scalebox
\usepackage{hyperref}[2021-06-07] % Hypertext links for LaTeX

\graphicspath{ {./images/} }

% for PG text use a sans font similar size to stix2, some others are much bigger
\newfontfamily\pgfont{Alegreya Sans}
\newenvironment{PGtext}{\thispagestyle{empty}
\raggedright \pgfont
\setlength\parskip{5pt plus1pt}}{}

\newcommand{\PGLicense}{%
  \clearpage
  \pagestyle{fancy}
  \fancyhf{}
  \pagenumbering{Roman}
  \fancyhead[L]{}
  \fancyhead[C]{\small \pgfont License}
  \fancyhead[R]{\thepage}
  \renewcommand{\headrulewidth}{0pt}
}

\newcommand{\LicenseSection}[1]{\subsection*{\pgfont\normalsize\bfseries #1}}

\begin{document}
\hypersetup{
    pdftitle={An elementary treatise on electricity},
    pdfauthor={James Clerk Maxwell},
    pageanchor=false,
    pdfdisplaydoctitle=true,
    colorlinks=true,
    urlcolor=blue,
}

\newgeometry{ % let image fill page
  top=0cm,
  bottom=0cm,
  left=0cm,
  right=0cm,
}
\thispagestyle{empty}
\begin{figure}[tbp]
\includegraphics[width=0.999\textwidth,height=0.999\textheight,keepaspectratio]{cover.jpg}
\centering
\end{figure}
\restoregeometry

\newpage
\begin{PGtext}
\begin{center}
\large
\textbf{
The Project Gutenberg eBook of An elementary treatise on electricity by James Clerk Maxwell
}
\end{center}
\InputIfFileExists{pgheader.tex}{}{}
\renewcommand*{\arraystretch}{1.5}
\begin{tabular}{p{2.5cm}>{\raggedright\arraybackslash}p{\dimexpr \linewidth-3.5cm}}
Title: &        An elementary treatise on electricity\\
Author: &       James Clerk Maxwell\\
Release Date: & January 31, 2023 [eBook \#69914]\\
Language: &     English\\
Produced by: &  The Online Distributed Proofreading Team at https://www.pgdp.net (This file was produced from images generously made available by The Internet Archive)\\
\end{tabular}
\vfill
\begin{center}
*** START OF THE PROJECT GUTENBERG EBOOK AN ELEMENTARY TREATISE ON ELECTRICITY ***
\end{center}
\end{PGtext}
\pagestyle{fancy}
\fancyhf{}
\fancyhead[R]{\iffloatpage{}{\thepage}}

\frontmatter
\fontdimen2\font=0.75ex% inter word space
\textit{\fontdimen2\font=0.75ex}
{\small{\fontdimen2\font=0.75ex}}
%%-----File: 001.png-----%%
\thispagestyle{empty}
\vspace{3cm}
\begin{center}
{\Large\gfont \textls[80]{Clarendon Press Series}}\\[4cm]
AN\\[0.5cm]
\Large
\textls{ELEMENTARY TREATISE}\\[0.5cm]
\small
ON\\[0.5cm]
\Huge
\textls{ELECTRICITY}\\[0.9cm]
\large
\textit{\textls{MAXWELL}}
\end{center}
%%-----File: 002.png-----%%

\newchapter
\hspace{0pt}
\vfill
\begin{center}
{\gfont \textls{London}}\\[0.3cm]
\small
\textls{HENRY FROWDE}\\
\begin{figure}[ht!]
\centering
\includegraphics[width=0.22\textwidth]{002.png}
\end{figure}
\textsc{Oxford University Press Warehouse\\[0.3cm]
Amen Corner, E.C.}
\end{center}
\vfill
%%-----File: 003.png-----%%

\newchapter
\begin{center}
{\Large\gfont \textls[80]{Clarendon Press Series}}\\[8mm]
AN\\[0.5cm]
\LARGE
\textls{ELEMENTARY TREATISE}\\[0.6cm]
\small
ON\\[0.5cm]
\Huge
\textls{ELECTRICITY}\\[0.9cm]
\footnotesize
BY\\[0.5cm]
\Large
\textls{JAMES CLERK MAXWELL, M.A.}\\[0.2cm]
\scriptsize
\textls{LL.D. EDIN., D.C.L., F.R.SS. LONDON AND EDINBURGH\\[1mm]
HONORARY FELLOW OF TRINITY COLLEGE\\[1mm]
AND PROFESSOR OF EXPERIMENTAL PHYSICS IN THE UNIVERSITY OF CAMBRIDGE}\\[4mm]
EDITED BY\\[0.2cm]
\large
WILLIAM GARNETT, M.A.\\[0.2cm]
\scriptsize
FORMERLY FELLOW OF ST. JOHN'S COLLEGE, CAMBRIDGE\\[1cm]
\normalsize
\textit{SECOND EDITION}\\[1cm]
\Large
{\gfont \textls[80]{Oxford}}\\[0.1cm]
\normalsize
AT THE CLARENDON PRESS\\[0.1cm]
1888\\[0.2cm]
{[}\textit{All rights reserved}{]}
\end{center}
%%-----File: 004.png-----%%
%%-----File: 005.png-----%%
\newchapter
\normalsize
\Chapter{EDITOR'S PREFACE.}
\Runhead{EDITOR'S PREFACE.}

\lettrine[lines=2, findent=0pt, loversize=0.1]{M}{OST} of the following pages were written by the late Professor
Clerk Maxwell, about seven years ago, and some of
them were used by him as the text of a portion of his lectures
on Electricity at the Cavendish Laboratory. Very little appears
to have been added to the MS. during the last three
or four years of Professor Maxwell's life, with the exception
of a few fragmentary portions in the latter part of the work.
This was partly due to the very great amount of time and
thought which he expended upon editing the Cavendish papers,
nearly all of which were copied by his own hand, while the
experimental investigations which he undertook in order to
corroborate Cavendish's results, and the enquiries he made
for the purpose of clearing up every obscure allusion in
Cavendish's MS., involved an amount of labour which left
him very little leisure for other work.

When the MS. came into the hands of the present Editor,
the first eight chapters appeared to have been finished and
were carefully indexed and the Articles numbered. Chapters
IX and X were also provided with tables of contents, but the
Articles were not numbered, and several references, Tables, etc.,
were omitted as well as a few sentences in the text. At the
end of the table of contents of Chapter X three points to be
treated were mentioned, viz.:---the Passage of Electricity at the
surfaces of insulators; Conditions of spark, etc.; Electrification
by pressure, friction, rupture, etc.: no Articles corresponding
to these headings could be found in the text. Some portions
of Chapters IX and X formed separate bundles of MS., and
%%-----File: 006.png-----%%
there was no indication of the place which they were intended
to fill. This was the case with Arts.\ 174-181 and 187-192.
Arts.\ 194-196 and 200 also formed a separate MS. with no
table of contents and no indication of their intended position.

It was for some time under consideration by the friends of
Professor Maxwell, whether the MS. should be published in
its fragmentary form or whether it should be completed by
another hand, so as to carry out as far as possible the author's
original design; but before any decision had been arrived at
it was suggested that the book might be made to serve the
purposes of students by a selection of Articles from Professor
Maxwell's \textit{Electricity and Magnetism}, so as to make it in a
sense complete for the portion of the subject covered by the
first volume of the last-mentioned work. In accordance with
this suggestion, a number of Articles have been selected from
the larger book and reprinted. These are indicated by a *
after the number of the Article. Arts.\ 93-98 and 141 are
identical with Arts.\ 118-123 and 58 of the larger treatise, but
these have been reprinted in accordance with directions contained
in Professor Maxwell's MS.

In the arrangement of the Articles selected from the \textit{Electricity
and Magnetism} care has been taken to interfere as
little as possible with the continuity of the MS. of the present
work, and in some cases logical order has been sacrificed to
this object, so that some subjects which are treated briefly in
the earlier portions are reintroduced in the latter part of the
book. In Chapter XII some articles are introduced from the
larger treatise which may appear somewhat inconsistent with
the plan of this book; this has been for the sake of the practical
value of the results arrived at. The latter part of the
note starting on page \pageref{note:192} may be taken as Professor Maxwell's
own comment on the method proposed in \hyperref[art:186]{Art.\ 186}, written a few
years subsequently to that Article.

All references, for the accuracy of which Professor Maxwell
is not responsible, and all Tables, notes, or interpolations inserted
%%-----File: 007.png-----%%
by the Editor, are enclosed in square brackets. This
system has not been carried out in the table of contents, but
the portion of this contained in Professor Maxwell's MS. is
stated above.

Of the Author's Preface the portion here given is all that
has been found.

\hspace*{\fill}
W. G.
\hspace{1cm}

\small
\hspace{1cm}\textsc{Cambridge:}

\hspace{2cm}\textit{August}, 1881.
\normalsize

\vspace{1cm}
\singleline
\vspace{1cm}

\Chapter{PREFACE TO THE SECOND EDITION.}

\lettrine[lines=2, findent=0pt, loversize=0.1]{W}{HEN} it became necessary to reprint this work, it appeared
to some desirable that certain changes should be made,
and especially that the articles taken from the larger work of
Professor Clerk Maxwell should be omitted. When the first
edition was published, very few books on electrical measurements
were available to the student; but since that time, the literature
of the subject has developed enormously, and there is no longer
the same reason for extending this book beyond the limits of
the Author's MS\@. In addition to this some of the Articles taken
from the larger book assume a knowledge on the part of the
student which is not to be obtained from the chapters of this
work. On careful consideration it was, however, thought best
that no change should be made, and, except for a few slight corrections,
the present edition is simply a reprint of the former.

\hspace*{\fill}
W. G.
\hspace{1cm}

\small
\hspace{1cm}\textsc{Newcastle-upon-Tyne}:

\hspace{2cm}\textit{August}, 1888.
%%-----File: 008.png-----%%

\vspace{1cm}
\singleline
\vspace{1cm}
\normalsize

\Runhead{PREFACE TO SECOND EDITION.}
\Chapter{FRAGMENT OF AUTHOR'S PREFACE.}

\lettrine[lines=2, findent=0pt, loversize=0.1]{T}{HE} aim of the following treatise is different from that of my
larger treatise on electricity and magnetism. In the larger
treatise the reader is supposed to be familiar with the higher
mathematical methods which are not used in this book, and his
studies are so directed as to give him the power of dealing
mathematically with the various phenomena of the science. In
this smaller book I have endeavoured to present, in as compact
a form as I can, those phenomena which appear to throw light
on the theory of electricity, and to use them, each in its place,
for the development of electrical ideas in the mind of the reader.

In the larger treatise I sometimes made use of methods which
I do not think the best in themselves, but without which the
student cannot follow the investigations of the founders of the
Mathematical Theory of Electricity. I have since become more
convinced of the superiority of methods akin to those of Faraday,
and have therefore adopted them from the first.

In the first two chapters experiments are described which
demonstrate the principal facts relating to electric charge considered
as a quantity capable of being measured.

The third chapter, `on electric work and energy,' consists of
deductions from these facts. To those who have some acquaintance
with the elementary parts of mathematics, this chapter
may be useful as tending to make their knowledge more precise.
Those who are not so prepared may omit this chapter in their
first reading of the book.

The fourth chapter describes the electric field, or the region in
which electric phenomena are exhibited.
\Runhead{FRAGMENT OF AUTHOR'S PREFACE.}
%%-----File: 009.png-----%%

\newchapter
\begin{center}
\LARGE \textls{CONTENTS.}
\end{center}
\Runhead{CONTENTS.}
\decorline
\small
{\centering CHAPTER I. \par}
\begin{longtable}{b{.87cm} @{\enspace} >{\hangindent=1em}b{10.6cm} @{} r}
Art. & & Page\endhead
\stocentry{1}{Exp.\ I\@. Electrification by friction}
\stocentry{2}{\dittoexp\ II\@. Electrification of a conductor}
\stocentry{3}{\dittoexp\ III\@. Positive and negative electrification}
\stocentry{4}{\dittoexp\ IV\@. Electrophorus}
\stocentry{5}{Electromotive force}
\stocentry{6}{Potential}
\stocentry{7}{Potential of a conductor}
\stocentry{8}{Of metals in contact}
\stocentry{9}{Equipotential surfaces}
\dtocent{10}{Potential, pressure, and temperature}
\dtocent{11}{Exp.\ V\@. Gold-leaf electroscope}
\dtocent{12}{\dittoexp\ V\@. Gold-leaf electroscope---\textit{continued}}
\dtocent{13}{Quadrant electrometer}
\dtocent{14}{Idio- and Hetero-Static}
\dtocent{15}{Insulators}
\dtocent{16}{Apparatus}
\tocchap{CHAPTER II.}
\tocsubtitle{ON THE CHARGES OF ELECTRIFIED BODIES.}
\dtocent{17}{Exp.\ VI\@. Electrified body within a closed vessel}
\dtocent{18}{\dittoexp\ VII\@. Comparison of the charges of two bodies}
\dtocent{19}{\dittoexp\ VIII\@. \hangindent=4em Electrification of inside of closed vessel equal and opposite to that of enclosed body}[\newline]
\dtocent{20}{\dittoexp\ IX\@. To discharge a body completely}
\dtocent{21}{\dittoexp\ X\@. \hangindent=4em To charge a body with a given number of times a particular charge}[\newline]
\dtocent{22}{Five laws of Electrical phenomena}
\toclistitem{I\@. In insulated bodies.}
\toclistitem{II\@. In a system of bodies during conduction.}
\toclistitem{III\@. In a system of bodies during electrification.}
\toclistitem{IV\@. Electrification of the two electrodes of a dielectric equal and opposite.}
\toclistitem{V\@. No electrification on the internal surface of a conducting vessel.}
%%-----File: 010.png-----%%
\tocchap{CHAPTER III.}
\tocsubtitle{ON ELECTRICAL WORK AND ENERGY.}
\dtocent{23}{Definitions of work, of energy, of a conservative system}
\dtocent{24}{Principle of conservation of energy. Examples of the measurement of work}[\newline]
\dtocent{25}{Definition of electric potential}
\dtocent{26}{Relation of the electromotive force to the equipotential surfaces}
\dtocent{27}{Indicator diagram of electric work}
\dtocent{28}{Indicator diagram of electric work---\textit{continued}}
\dtocent{29}{Superposition of electric effects}
\dtocent{30}{Charges and potentials of a system of conductors}
\dtocent{31}{Energy of a system of electrified bodies}
\dtocent{32}{Work spent in passing from one electrical state to another}
\dtocent{33}{\(P = \xp\dfrac{dQ_e}{dE}\)}
\dtocent{34}{\(\sum(EP') = \sum(E'P)\);---Green's theorem}
\dtocent{35}{Increment of energy under increments of potentials}
\dtocent{36}{\(E = \xp\dfrac{dQ_p}{dP}\)}
\dtocent{37}{Reciprocity of potentials}
\dtocent{38}{Reciprocity of charges}
\dtocent{39}{Green's theorem on potentials and charges}
\dtocent{40}{Mechanical work during the displacement of an insulated system}
\dtocent{41}{Mechanical work during the displacement of a system the potentials of which are maintained}[\newline]
\tocchap{CHAPTER IV.}
\nopagebreak
\tocsubtitle{THE ELECTRIC FIELD.}
\nopagebreak
\dtocent{42}{Two conductors separated by an insulating medium}
\dtocent{43}{This medium called a dielectric medium, or, the electric field}
\multicolumn{3}{c}{\strutb \textls{\scriptsize EXPLORATION OF THE ELECTRIC FIELD.}}\\
\dtocent{44}{Exp.\ XI\@. By a small electrified body}
\dtocent{45}{Exp.\ XII\@. By two disks}
\dtocent{46}{Electric tension}
\dtocent{47}{Exp.\ XIII\@. Coulomb's proof plane}
\dtocent{48}{Exp.\ XIV\@. Electromotive force at a point}
\dtocent{49}{Exp.\ XV\@. Potential at any point in the field. Two spheres}
\dtocent{50}{Exp.\ XVI\@. One sphere}
\dtocent{51}{Equipotential surfaces}
\dtocent{52}{Reciprocal method. Exp.\ XVII.}
%%-----File: 011.png-----%%
\dtocent{53}{Exp.\ XVIII\@. Method founded on Theorem V.}
\dtocent{54}{Lines of electric force}

\tocchap{CHAPTER V.}
\tocsubtitle{FARADAY'S LAW OF LINES OF INDUCTION.}
\dtocent{55}{Faraday's Law}
\dtocent{56}{Hollow vessel}
\dtocent{57}{Lines of force}
\dtocent{58}{Properties of a tube of induction}
\dtocent{59}{Properties of a tube of induction---\textit{continued}}
\dtocent{60}{Cells}
\dtocent{61}{Energy}
\dtocent{62}{Displacement}
\dtocent{63}{Tension}
\dtocent{64}{Analogies}
\dtocent{65}{Analogies---\textit{continued}}
\dtocent{66}{Limitation}
\dtocent{67}{Faraday's cube}
\dtocent{68}{Faraday's cube---\textit{continued}}
\dtocent{69}{Current}
\dtocent{70}{Displacement}
\dtocent{71}{Theorems}
\dtocent{72}{Induction and force}
\dtocent{73}{\(+\) and \(-\) ends}
\dtocent{74}{Not cyclic}
\dtocent{75}{In the inside of a hollow conducting vessel not containing any
  electrified body the potential is uniform and there is no
  electrification}[\newline]
\dtocent{76}{In the inside of a hollow conducting vessel not containing any
  electrified body the potential is uniform and there is no
  electrification---\textit{continued}}[\newline\newline]
\dtocent{77}{Superposition}
\dtocent{78}{Thomson's theorem}
\dtocent{79}{Example}
\dtocent{80}{Induced electricity of 1st and 2nd species}

\tocchap{CHAPTER VI.}
\tocsubtitle{PARTICULAR CASES OF ELECTRIFICATION.}
\dtocent{81}{Concentric spheres}
\dtocent{82}{Unit of electricity. Law of force}
\dtocent{83}{Electromotive force at a point}
\dtocent{84}{Definition of electromotive force}
%%-----File: 012.png-----%%
\dtocent{85}{Coulomb's law}
\dtocent{86}{Value of the potential due to a uniformly electrified sphere}
\dtocent{87}{Capacity of a sphere}
\dtocent{88}{Two concentric spherical surfaces. Leyden Jar}
\dtocent{89}{Two parallel planes}
\dtocent{90}{Force between planes}
\dtocent{91}{Thomson's attracted disk electrometers}
\dtocent{92}{Inverse problem of electrostatics}
\dtocent{93}{Equipotential surfaces and lines of force for charges of 20 and 5 units (Plate I)}[\newline]
\dtocent{94}{Equipotential surfaces and lines of force for opposite charges in the ratio of 4 to \(-1\) (Plate II)}[\newline]
\dtocent{95}{Equipotential surfaces and lines of force for an electrified point in a uniform field of force (Plate III)}[\newline]
\dtocent{96}{Equipotential surfaces and lines of force for charges of three electrified points (Plate IV)}[\newline]
\dtocent{97}{Faraday's use of the conception of lines of force}
\dtocent{98}{Method employed in drawing the diagrams}

\tocchap{CHAPTER VII.}
\tocsubtitle{ELECTRICAL IMAGES.}
\dtocent{99}{Introductory}
\ttocent{100}{Idea of an image derived from optics}
\ttocent{101}{Electrical image at centre of sphere}
\ttocent{102}{External point and sphere}
\ttocent{103}{Two spheres}
\ttocent{104}{Calculation of potentials when charges are given}
\ttocent{105}{Surface density induced on a sphere by an electrified point}
\ttocent{106}{Surface density on two spheres and condition for a neutral line}

\tocchap{CHAPTER VIII.}
\tocsubtitle{CAPACITY.}
\ttocent{107}{Capacity of a condenser}
\ttocent{108}{Coefficients of condenser}
\ttocent{109}{Comparison of two condensers}
\ttocent{110}{Thomson's method with four condensers}
\ttocent{111}{Condition of null effect}

\tocchap{CHAPTER IX.}
\tocsubtitle{ELECTRIC CURRENT.}
\ttocent{112}{Convection current with pith ball}
\ttocent{113}{Conduction current in a wire}
%%-----File: 013.png-----%%
\ttocent{113}{No evidence as to the velocity of electricity in the current}
\ttocent{114}{Displacement and discharge}
\ttocent{115}{Classification of bodies through which electricity passes}
\ntocent{115a}{Definition of the conductor, its electrodes, anode, and cathode}
\ntocent{115b}{External electromotive force}
\ntocent{115c}{Metals, electrolytes, and dielectrics}

\multicolumn{3}{c}{\struta 1. \textit{Metals.}}\\

\ttocent{116}{Ohm's Law}
\ttocent{117}{Generation of heat}

\multicolumn{3}{c}{\struta 2. \textit{Electrolytes.}}\\

\ttocent{118}{Anion and cation}
\ntocent{118a}{Electrochemical equivalents}
\ttocent{119}{Faraday's Laws}
\ntocent{119a}{Force required for complete electrolysis}
\ttocent{120}{Polarization}
\ttocent{121}{Helmholtz's experiments}
\ttocent{122}{Supposed inaccuracy of Faraday's Law not confirmed}
\ttocent{123}{Measurement of resistance}
\ntocent{123a}{Ohm's Law true for electrolytes}
\ttocent{124}{Theory of Clausius}
\ttocent{125}{Theory of Clausius---\textit{continued}}
\ttocent{126}{Velocities of ions}
\ttocent{127}{Molecular conductivity of an electrolyte}
\ttocent{128}{Kohlrausch's experiments}
\ttocent{129}{Secondary actions}

\multicolumn{3}{c}{\struta 3. \textit{Dielectrics.}}\\

\ttocent{130}{Displacement}
\ttocent{131}{Dielectric capacity of solids, including crystals}
\ttocent{132}{Dielectric capacity of solids, liquids, and gases}
\ttocent{133}{Disruptive discharge. Mechanical and electrical analogies. Ultimate strength. Brittleness}[\newline]
\ttocent{134}{Residual charge}
\ttocent{135*}{Mechanical illustration}
\ttocent{136}{Electric strength of gases}
\ttocent{137}{Gases as insulators}
\ttocent{138}{Experiment}
\ttocent{139}{Mercury and sodium vapours}
\ttocent{140}{Kinetic theory of gases}
\ttocent{141}{Electric phenomena of Tourmaline}
%%-----File: 014.png-----%%

\ttocent{142}{Electric glow}
\ttocent{143}{Electric windmill}
\ntocent{143a}{Electrified air}
\ntocent{143b}{Motion of thunder-clouds not due to electricity}
\ttocent{144}{To detect the presence of electrified air}
\ttocent{145}{Difference between positive and negative electricity}
\ttocent{146}{Discharge by a point on a conductor electrified by induction}
\ttocent{147}{The electric brush}
\ttocent{148}{The electric spark}
\ttocent{149}{Spectroscopic investigation}
\ttocent{150*}{Description of the voltaic battery}
\ttocent{151*}{Electromotive force}
\ttocent{152*}{Production of a steady current}
\ttocent{153*}{Magnetic action of the current}
\ttocent{154*}{The galvanometer}
\ttocent{155*}{Linear conductors}
\ttocent{156*}{Ohm's law}
\ttocent{157*}{Linear conductors in series}
\ttocent{158*}{Linear conductors in multiple arc}
\ntocent{158a}{Kirchhoff's Laws}
\ttocent{159*}{Resistance of conductor of uniform section}

\tocchap{CHAPTER X.}
\tocsubtitle{PHENOMENA OF AN ELECTRIC CURRENT WHICH FLOWS THROUGH HETEROGENEOUS MEDIA.}
\ttocent{160}{Seebeck's discovery}
\ttocent{161}{Law of Magnus}
\ttocent{162}{Thermoelectric diagram and definition of thermoelectric power}
\ttocent{163}{Electromotive force measured by an area on the diagram}
\ttocent{164}{Cumming's discovery}
\ttocent{165}{Thermal effects of the current}
\ttocent{166}{Peltier's effect}
\ttocent{167}{Thomson's effect}
\ttocent{168}{Thomson's analogy with a fluid in a tube}
\ttocent{169}{Le Roux's experiments}
\ttocent{170}{Expression of Peltier's and Thomson's effects}
\ttocent{171}{Heat produced at a junction depends on its temperature}
\ttocent{172}{Application of the second law of thermodynamics}
\ttocent{173}{Complete interpretation of the diagram}
\ttocent{174}{Entropy in thermodynamics}
\ttocent{175}{Electric entropy}
%%-----File: 015.png-----%%
\ttocent{176}{Definition of entropy}
\ttocent{177}{Electric entropy equivalent to thermoelectric power}
\ttocent{178}{Thermoelectric diagram}
\ttocent{179}{Specific heat of electricity}
\ttocent{180}{Difference between iron and copper}
\ttocent{181}{Complete interpretation of the diagram}
\ttocent{182}{Thomson's method of finding the E. M. F. at a point in a circuit}
\ttocent{183}{Determination of the seat of electromotive force}
\ttocent{184}{E. M. F. between metal and electrolyte}
\ttocent{185}{Electrolysis. Deposition of metal. Solution of metal}
\ttocent{186}{Heat generated or absorbed at anode and cathode}
\ttocent{187}{On the conservation of energy in electrolysis}
\ttocent{188}{Joule's experiments}
\ttocent{189}{Loss of heat when current does external work}
\ttocent{190}{Electromotive force of electrochemical apparatus}
\ttocent{191}{Reversible and irreversible effects}
\ttocent{192}{Example from electrolysis of argentic chloride}
\ttocent{193*}{On constant voltaic elements. Daniell's cell}

\tocchap{CHAPTER XI.}
\tocsubtitle{METHODS OF MAINTAINING AN ELECTRIC CURRENT.}
\ttocent{194}{Enumeration of methods}
\ttocent{195}{The frictional electric machine}
\ttocent{196}{On what the current depends. Use of silk flaps}
\ttocent{197*}{Production of electrification by mechanical work. Nicholson's revolving doubler}[\newline]
\ttocent{198*}{Principle of Varley's and Thomson's electrical machines}
\ttocent{199*}{Thomson's water-dropping machine}
\ttocent{200}{Holtz's electrical machine}
\ttocent{201*}{Theory of regenerators applied to electrical machines}
\ttocent{202*}{Coulomb's torsion balance for measuring charges}
\ttocent{203*}{Electrometers for measuring potentials. Snow-Harris's and Thomson's}[\newline]
\ttocent{204*}{Principle of the guard-ring. Thomson's absolute electrometer}
\ttocent{205*}{Heterostatic method}
\ttocent{206*}{Measurement of the electric potential of a small body}
\ttocent{207*}{Measurement of the potential at a point in the air}
\ttocent{208*}{Measurement of the potential of a conductor without touching it}
%%-----File: 016.png-----%%

\tocchap{CHAPTER XII.}
\tocsubtitle{ON THE MEASUREMENT OF ELECTRIC RESISTANCE.}
\ttocent{209*}{Advantage of using material standards of resistance in electrical measurements}[\newline]
\ttocent{210*}{Different standards which have been used and different systems which have been proposed}[\newline]
\ttocent{211*}{The electromagnetic system of units}
\ttocent{212*}{Weber's unit, and the British Association unit or Ohm}
\ttocent{213*}{Professed value of the Ohm 10,000,000 metres per second}
\ttocent{214*}{Reproduction of standards}
\ttocent{215*}{Forms of resistance coils}
\ttocent{216*}{Coils of great resistance}
\ttocent{217*}{Arrangement of coils in series}
\ttocent{218*}{Arrangement in multiple arc}
\ttocent{219*}{On the comparison of resistances. (1) Ohm's method}
\ttocent{220*}{(2) By the differential galvanometer}
\ttocent{221*}{(3) By Wheatstone's Bridge}
\ttocent{222*}{Estimation of limits of error in the determination}
\ttocent{223*}{Best arrangement of the conductors to be compared}
\ttocent{224*}{On the use of Wheatstone's Bridge}
\ttocent{225*}{Thomson's method for the resistance of a galvanometer}
\ttocent{226*}{Mance's method of determining the resistance of a battery}
\ttocent{227*}{Comparison of electromotive forces}

\tocchap{CHAPTER XIII.}
\tocsubtitle{ON THE ELECTRIC RESISTANCE OF SUBSTANCES.}
\ttocent{228*}{Metals, electrolytes and dielectrics}
\ttocent{229*}{Resistance of metals}
\ttocent{230*}{Table of resistance of metals}
\ttocent{231*}{Resistance of electrolytes}
\ttocent{232*}{Experiments of Paalzow}
\ttocent{233*}{Experiments of Kohlrausch and Nippoldt}
\ttocent{234*}{Resistance of dielectrics}
\ttocent{235*}{Gutta-percha}
\ttocent{236*}{Glass}
\ttocent{237*}{Gases}
\ttocent{238*}{Experiments of Wiedemann and Rühlmann}
\ntocent{239}{Note on the determination of the current in the galvanometer
of Wheatstone's Bridge}
\end{longtable}
\vspace{1cm}
\singleline
%%-----File: 017.png-----%%
\mainmatter
\normalsize
\thispagestyle{empty}
\begin{center}
{\large AN ELEMENTARY TREATISE}\\[0.5cm]
{\footnotesize ON}\\[0.5cm]
\Large{ELECTRICITY.}
\end{center}
\decorline
\Chapter{CHAPTER I.}
\Section{Experiment I.}
\Subsection{Electrification by Friction.}
\Runhead{ELECTRIFICATION BY FRICTION.}

\article{1}
\textsc{Take} a stick of sealing-wax, rub it on woollen cloth or
flannel, and then bring it near to some shreds of paper strewed on
the table. The shreds of paper will move, the lighter ones will
raise themselves on one end, and some of them will leap up to the
sealing-wax. Those which leap up to the sealing-wax sometimes
stick to it for awhile, and then fly away from it suddenly. It
appears therefore that in the space between the sealing-wax and
the table is a region in which small bodies, such as shreds of paper,
are acted on by certain forces which cause them to assume particular
positions and to move sometimes from the table to the
sealing-wax, and sometimes from the sealing-wax to the table.

These phenomena, with others related to them, are called electric
phenomena, the bodies between which the forces are manifested are
said to be electrified, and the region in which the phenomena take
place is called the electric field.

Other substances may be used instead of the sealing-wax. A
piece of ebonite, gutta-percha, resin or shellac will do as well, and
so will amber, the substance in which these phenomena were first
noticed, and from the Greek name of which the word \textit{electric} is
derived.

The substance on which these bodies are rubbed may also be
varied, and it is found that the fur of a cat's skin excites them
better than flannel.

It is found that in this experiment only those parts of the
surface of the sealing-wax which were rubbed exhibit these phenomena,
%%-----File: 018.png-----%%
and that some parts of the rubbed surface are apparently
more active than others. In fact, the distribution of the electrification
over the surface depends on the previous history of the
sealing-wax, and this in a manner so complicated that it would be
very difficult to investigate it. There are other bodies, however,
which may be electrified, and over which the electrification is
always distributed in a definite manner. We prefer, therefore, in
our experiments, to make use of such bodies.

The fact that certain bodies after being rubbed appear to attract
other bodies was known to the ancients. In modern times many
other phenomena have been observed, which have been found to be
related to these phenomena of attraction. They have been classed
under the name of \textit{electric} phenomena, amber, {\greekfont \textit{ἤλεκτρον}}, having
been the substance in which they were first described.

Other bodies, particularly the loadstone and pieces of iron and
steel which have been subjected to certain processes, have also been
long known to exhibit phenomena of action at a distance. These
phenomena, with others related to them, were found to differ from
the electric phenomena, and have been classed under the name of
magnetic phenomena, the loadstone, {\greekfont \textit{μάγνης}}, being found in
Magnesia\footnote{
The name of Magnesia has been given to two districts, one in Lydia, the other in
Thessaly. Both seem to have been celebrated for their mineral products, and several
substances have been known by the name of magnesia besides that which modern
chemists know by that name. The loadstone, the touchstone, and meerschaum,
seem however to have been the principal substances which were called Magnesian
and magnetic, and these are generally understood to be Lydian stones.
}.

These two classes of phenomena have since been found to be
related to each other, and the relations between the various phenomena
of both classes, so far as they are known, constitute the
science of Electromagnetism.

\Section{Experiment II.}
\Subsection{Electrification of a Conductor.}

\article{2}
Take a metal plate of any kind (a tea-tray, turned upside
down, is convenient for this purpose) and support it on three dry
wine glasses. Now place on the table a plate of ebonite, a sheet
of thin gutta-percha, or a well-dried sheet of brown paper. Rub it
lightly with fur or flannel, lift it up from the table by its edges
and place it on the inverted tea-tray, taking care not to touch the
tray with your fingers.
%%-----File: 019.png-----%%

It will be found that the tray is now electrified. Shreds of
paper or gold-leaf placed below it will fly up to it, and if the
knuckle is brought near the edge of the tray a spark will pass
between the tray and the knuckle, a peculiar sensation will be felt,
and the tray will no longer exhibit electrical phenomena. It is
then said to be \textit{discharged}. If a metal rod, held in the hand, be
brought near the tray the phenomena will be nearly the same.
The spark will be seen and the tray will be discharged, but the
sensation will be slightly different.

If, however, instead of a metal rod or wire, a glass rod, or stick
of sealing-wax, or a piece of gutta-percha, be held in the hand and
brought up to the tray there will be no spark, no sensation, and
no discharge. The discharge, therefore, takes place through metals
and through the human body, but not through glass, sealing-wax,
or gutta-percha. Bodies may therefore be divided into two
classes: conductors, or those which transmit the discharge, and
non-conductors, through which the discharge does not take place.

In electrical experiments, those conductors, the charge of which
we wish to maintain constant, must be supported by non-conducting
materials. In the present experiment the tray was supported on
wine glasses in order to prevent it from becoming discharged.
Pillars of glass, ebonite, or gutta-percha may be used as supports,
or the conductor may be suspended by a white silk thread. Solid
non-conductors, when employed for this purpose, are called \textit{insulators}.
Copper wires are sometimes lapped with silk, and sometimes
enclosed in a sheath of gutta-percha, in order to prevent
them from being in electric communication with other bodies.
They are then said to be insulated.

The metals are good conductors; air, glass, resins, gutta-percha,
caoutchouc, ebonite, paraffin, \&c., are good insulators; but, as we
shall find afterwards, all substances resist the passage of electricity,
and all substances allow it to pass though in exceedingly different
degrees. For the present we shall consider only two classes of
bodies, good conductors, and good insulators.
\Runhead{ELECTRIFICATION OF A CONDUCTOR.}

\Section{Experiment III.}
\Subsection{Positive and Negative Electrification.}
\Runhead{POSITIVE AND NEGATIVE ELECTRIFICATION.}

\article{3}
Take another tray and insulate it as before, then after
discharging the first tray remove the electrified sheet from it and
place it on the second tray. It will be found that both trays are
%%-----File: 020.png-----%%
now electrified. If a small ball of elder pith suspended by a white
silk thread\footnote{
I find it convenient to fasten the other end of the thread to a rod of ebonite
about 3 mm.\ diameter. The ebonite is a much better insulator than the silk thread,
especially in damp weather.
} be made to touch the first tray, it will be immediately
repelled from it but attracted towards the second. If it is now
allowed to touch the second tray it will be repelled from it but
attracted towards the first. The electrifications of the two trays
are therefore of opposite kinds, since each attracts what the other
repels. If a metal wire, attached to an ebonite rod, be made to
touch both trays at once, both trays will be completely discharged.
If two pith balls be used, then if both have been made to touch
the same tray and then hung up near each other they are found
to repel each other, but if they have been made to touch different
trays they attract each other. Hence bodies when electrified in
the same way are repelled from each other, but when they are
electrified in opposite ways they are attracted to each other.

If we distinguish one kind of electrification by calling it \textit{positive},
we must call the other kind of electrification \textit{negative}. We have
no physical reason for assigning the name of positive to one kind
of electrification rather than to the other. All scientific men,
however, are in the habit of calling that kind of electrification
positive which the surface of polished glass exhibits after having
been rubbed with zinc amalgam spread on leather. This is a
matter of mere convention, but the convention is a useful one,
provided we remember that it is a convention as arbitrary as
that adopted in the diagrams of analytical geometry of calling
horizontal distances positive or negative according as they are
measured towards the right or towards the left of the point of
reference.

In our experiment with a sheet of gutta-percha excited by
flannel, the electrification of the sheet and of the tray on which
it is placed is negative; that of the flannel and of the tray from
which the gutta-percha has been removed is positive.

In whatever way electrification is produced it is one or other of
these two kinds.

\Section{Experiment IV.}
\Subsection{The Electrophorus of Volta.}

\article{4}
This instrument is very convenient for electrical experiments
and is much more compact than any other electrifying apparatus.
%%-----File: 021.png-----%%
It consists of two disks, and an insulating handle which can be
screwed to the back of either plate. One of these disks consists
of resin or of ebonite in front supported by a metal back. In the
centre of the disk is a metal pin\footnote{
[This was introduced by Professor Phillips to obviate the necessity of touching
the carrier plate while in contact with the ebonite.]
}, which is in connection with the
metal back, and just reaches to the level of the surface of the
ebonite. The surface of the ebonite is electrified by striking it
with a piece of flannel or cat's fur. The other disk, which is
entirely of metal, is then brought near the ebonite disk by means
of the insulating handle. When it comes within a certain distance
of the metal pin, a spark passes, and if the disks are now separated
the metal disk is found to be charged positively, and the disk of
ebonite and metal to be charged negatively.

In using the instrument one of the disks is kept in connection
with one conductor while the other is applied alternately to the
first disk and to the other conductor. By this process the two
conductors will become charged with equal quantities of electricity,
that to which the ebonite disk was applied becoming negative,
while that to which the plain metal disk was applied becomes
positive.
\Runhead{THE ELECTROPHORUS.}

\Section{Electromotive Force.}

\article{5}
Definition.---\textit{Whatever produces or tends to produce a transfer
of Electrification is called Electromotive Force.}

Thus when two electrified conductors are connected by a wire,
and when electrification is transferred along the wire from one
body to the other, the tendency to this transfer, which existed
before the introduction of the wire, and which, when the wire is
introduced, produces this transfer, is called the Electromotive
Force \textit{from} the one body \textit{to} the other \textit{along} the path marked out
by the wire.

To define completely the electromotive force from one point to
another, it is necessary, in general, to specify a particular path from
the one point to the other along which the electromotive force is
to be reckoned. In many cases, some of which will be described
when we come to electrolytic, thermoelectric, and electromagnetic
phenomena, the electromotive force from one point to another may
be different along different paths. If we restrict our attention,
%%-----File: 022.png-----%%
however, as we must do in this part of our subject, to the theory of
the equilibrium of electricity at rest, we shall find that the electromotive
force from one point to another is the same for all paths
drawn in air from the one point to the other.
\Runhead{ELECTROMOTIVE FORCE.}

\Section{Electric Potential.}

\article{6} The electromotive force from any point, along a path drawn
in air, to a certain point chosen as a point of reference, is called
the Electric Potential at that point.

Since electrical phenomena depend only on differences of potential,
it is of no consequence what point of reference we assume for
the zero of potential, provided that we do not change it during
the same series of measurements.

In mathematical treatises, the point of reference is taken at an
infinite distance from the electrified system under consideration.
The advantage of this is that the mathematical expression for the
potential due to a small electrified body is thus reduced to its
simplest form.

In experimental work it is more convenient to assume as a point
of reference some object in metallic connection with the earth, such
as any part of the system of metal pipes conveying the gas or
water of a town.

It is often convenient to assume that the walls, floor and ceiling
of the room in which the experiments are carried on has conducting
power sufficient to reduce the whole inner surface of the room to
the same potential. This potential may then be taken for zero.
When an instrument is enclosed in a metallic case the potential
of the case may be assumed to be zero.
\Runhead{ELECTRIC POTENTIAL.}

\Subsection{Potential of a Conductor.}

\article{7} If the potentials at different points of a uniform conductor
are different there will be an electric current from the places of
high to the places of low potential. The theory of such currents
will be explained afterwards (Chap.\ ix). At present we are dealing
with cases of electric equilibrium in which there are no currents.
Hence in the cases with which we have now to do the potential
at every point of the conductor must be the same. This potential
is called the potential of the conductor.

The potential of a conductor is usually defined as the potential
%%-----File: 023.png-----%%
of any point in the air infinitely close to the surface of the conductor.
Within a nearly closed cavity in the conductor the
potential at any point in the air is the same, and by making the
experimental determination of the potential within such a cavity
we get rid of the difficulty of dealing with points infinitely close
to the surface.

\article{8} It is found that when two different metals are in contact and
in electric equilibrium their potentials as thus defined are in general
different. Thus, if a copper cylinder and a zinc cylinder are held
in contact with one another, and if first the copper and then the
zinc cylinder is made to surround the flame of a spirit lamp, the
lamp being in connection with an electrometer, the lamp rapidly
acquires the potential of the air within the cylinder, and the
electrometer shews that the potential of the air at any point within
the zinc cylinder is higher than the potential of the air within the
copper cylinder. We shall return to this subject again, but at
present, to avoid ambiguity, we shall suppose that the conductors
with which we have to do consist all of the same metal at the same
temperature, and that the dielectric medium is air.
\Runhead{POTENTIAL OF A CONDUCTOR.}

\article{9} The region of space in which the potential is higher than
a certain value is divided from the region in which it is lower than
this value by a surface called an equipotential surface, at every
point of which the potential has the given value.

We may conceive a series of equipotential surfaces to be described,
corresponding to a series of potentials in arithmetical order.
The potential of any point will then be indicated by its position in
the series of equipotential surfaces.

No two different equipotential surfaces can cut one another, for
no point can have two different potentials.
\Runhead{EQUIPOTENTIAL SURFACES.}

\article{10} The idea of electric potential may be illustrated by comparing
it with pressure in the theory of fluids and with temperature
in the theory of heat.

If two vessels containing the same or different fluids are put into
communication by means of a pipe, fluid will flow from the vessel
in which the pressure is greater into that in which it is less till the
pressure is equalized. This however will not be the case if one
vessel is higher than the other, for gravity has a tendency to make
the fluid pass from the higher to the lower vessel.

Similarly when two electrified bodies are put into electric communication
by means of a wire, electrification will be transferred
from the body of higher potential to the body of lower potential,
%%-----File: 024.png-----%%
unless there is an electromotive force tending to urge electricity
from one of these bodies to the other, as in the case of zinc and
copper above mentioned.

Again, if two bodies at different temperatures are placed in
thermal communication either by actual contact or by radiation,
heat will be transferred from the body at the higher temperature
to the body at the lower temperature till the temperature of the
two bodies becomes equalized.

The analogy between temperature and potential must not be
assumed to extend to all parts of the phenomena of heat and
electricity. Indeed this analogy breaks down altogether when it is
applied to those cases in which heat is generated or destroyed.

We must also remember that temperature corresponds to a real
physical state, whereas potential is a mere mathematical quantity,
the value of which depends on the point of reference which we may
choose. To raise a body to a high temperature may melt or
volatilize it. To raise a body, together with the vessel which surrounds
it, to a high potential produces no physical effect whatever
on the body. Hence the only part of the phenomena of electricity
and heat which we may regard as analogous is the condition of the
transfer of heat or of electricity, according as the temperature or
the potential is higher in one body or in the other.

With respect to the other analogy---that between potential and
fluid pressure---we must remember that the only respect in which
electricity resembles a fluid is that it is capable of flowing along
conductors as a fluid flows in a pipe. And here we may introduce
once for all the common phrase \textit{The Electric Fluid} for the purpose
of warning our readers against it. It is one of those phrases,
which, having been at one time used to denote an observed fact,
was immediately taken up by the public to connote a whole system
of imaginary knowledge. As long as we do not know whether
positive electricity, or negative, or both, should be called a substance
or the absence of a substance, and as long as we do not
know whether the velocity of an electric current is to be measured
by hundreds of thousands of miles in a second or by an hundredth of
an inch in an hour, or even whether the current flows from positive
to negative or in the reverse direction, we must avoid speaking of
the electric fluid.
\Runhead{POTENTIAL, PRESSURE, AND TEMPERATURE.}
%%-----File: 025.png-----%%

\Section{On Electroscopes.}

\article{11} An electroscope is an instrument by means of which the
existence of electrification may be detected. All electroscopes are
capable of indicating with more or less accuracy not only the
existence of electrification, but its amount. Such indications, however,
though sometimes very useful in guiding the experimenter,
are not to be regarded as furnishing a numerical measurement of
the electrification. Instruments so constructed that their indications
afford data for the numerical measurement of electrical
quantities are called Electrometers.

An electrometer may of course be used as an electroscope if it is
sufficiently sensitive to indicate the electrification in question, and
an instrument intended for an electroscope may, if its indications
are sufficiently uniform and regular, be used as an electrometer.

The class of electroscopes of simplest construction is that in
which the indicating part of the instrument consists of two light
bodies suspended side by side, which, when electrified, repel each
other, and indicate their electrification by separating from each
other.

The suspended bodies may be balls of elder pith, gilt, and hung
up by fine linen threads (which are better conductors than silk or
cotton), or pieces of straw or strips of metal, and in the latter case
the metal may be tinfoil or gold-leaf, thicker or thinner according
to the amount of electrification to be measured.

\wrapfig{0.3}{025.png}{Fig. 1.}
We shall suppose that our electroscope is of the most delicate
kind, in which gold leaves are employed (see Fig. 1). The indicating
apparatus \(l\), \(l\), is generally fastened to one
end of a rod of metal \(L\), which passes through an
opening in the top of a glass vessel \(G\). It then
hangs within the vessel, and is protected from
currents of air which might produce a motion of
the suspended bodies liable to be mistaken for
that due to electrification.
\Runhead{GOLD-LEAF ELECTROSCOPE.}

To test the electrification of a body the electrified
body is brought near the disk \(L\) at the top of the
metal rod, when, if the electrification is strong
enough, the suspended bodies diverge from one
another.

The glass case, however, is liable, as Faraday pointed out, to
become itself electrified, and when glass is electrified it is very
%%-----File: 026.png-----%%
difficult to ascertain by experiment the amount and the distribution
of its electrification. There is thus introduced into the experiment
a new force, the nature and amount of which are unknown, and this
interferes with the other forces acting on the gold leaves, so that
their divergence can no longer be taken as a true indication of
their electrification.

The best method of getting rid of this uncertainty is to place
within the glass case a metal vessel which almost surrounds the
gold leaves, this vessel being connected with the earth. When the
gold leaves are electrified the inside of this vessel, it is true, becomes
oppositely electrified, and so increases the divergence of the gold
leaves, but the distribution of this electrification is always strictly
dependent on that of the gold leaves, so that the divergence of the
gold leaves is still a true indication of their actual electrical state.
A continuous metal vessel, however, is opaque, so that the gold
leaves cannot be seen from the outside. A wire cage, however,
may be used, and this is found quite sufficient to shield the gold
leaves from the action of the glass, while it does not prevent them
from being seen.

The disk, \(L\), and the upper part of the rod which supports the
gold leaves, and another piece of metal \(M\), which is connected with
the cage \(m\), \(m\), and extends beyond the case of the instrument, are
called the \textit{electrodes}, a name invented by Faraday to denote the \textit{ways}
by which the electricity gets to the vital parts of the instrument.

The divergence of the gold leaves indicates that the potential of
the gold leaves and its electrode is different from that of the surrounding
metal cage and its electrode. If the two electrodes are
connected by a wire the whole instrument may be electrified to any
extent, but the leaves will not diverge.

\Section{Experiment V.}

The divergence of the gold leaves does not of itself indicate
whether their potential is higher or lower than that of the cage;
it only shews that these potentials are different. To ascertain
which has the higher potential take a rubbed stick of sealing-wax,
or any other substance which we know to be negatively electrified,
and bring it near the electrode which carries the gold leaves. If
the gold leaves are negatively electrified they will diverge more as
the sealing-wax approaches the rod which carries them; but if they
are positively electrified they will tend to collapse. If the electrification
of the sealing-wax is considerable with respect to that of
%%-----File: 027.png-----%%
the gold leaves they will first collapse entirely, but will again
open out as the sealing-wax is brought nearer, shewing that they
are now negatively electrified. If the electrode \(M\) belonging to
the cage is insulated from the earth, and if the sealing-wax is
brought near it, the indications will be exactly reversed; the leaves,
if electrified positively, will diverge more, and if electrified negatively,
will tend to collapse.

If the testing body used in this experiment is positively electrified,
as when a glass tube rubbed with amalgam is employed, the
indications are all reversed.

By these methods it is easy to determine whether the gold leaves
are positively or negatively electrified, or, in other words, whether
their potential is higher or lower than that of the cage.

\article{12} If the electrification of the gold leaves is considerable the
electric force which acts on them becomes much greater than their
weight, and they stretch themselves out towards the cage as far as
they can. In this state an increase of electrification produces no
visible effect on the electroscope, for the gold leaves cannot diverge
more. If the electrification is still further increased it often happens
that the gold leaves are torn off from their support, and the instrument
is rendered useless\footnote{
[For the sake of safety the cage is often so arranged that the gold leaves touch
it and become discharged before diverging to their extreme limit.]
}. It is better, therefore, when we have to
deal with high degrees of electrification to use a less delicate instrument.
A pair of pith balls suspended by linen threads answers
very well; the threads answer sufficiently well as conductors of electricity,
and the balls are repelled from each other when electrified.

For very small differences of potential, electroscopes much more
sensitive than the ordinary gold-leaf electroscope may be used.

\Section{Thomson's Quadrant Electrometer.}
\article{13} In Sir William Thomson's Quadrant Electrometer the indicating part
consists of a thin flat strip of aluminium (see
Fig. 2) called the needle, attached to a vertical axle of stout
platinum wire. It is hung up by two silk fibres \(x\), \(y\), so as to
be capable of turning about a vertical axis under the action of
the electric force, while it always tends to return to a definite
position of equilibrium. The axis carries a concave mirror \(t\) by
which the image of a flame, and of a vertical wire bisecting the
flame, is thrown upon a graduated scale, so as to indicate the
motion of the needle round a vertical axis. The lower end of
%%-----File: 028.png-----%%
the axle dips into sulphuric acid contained in the lower part of
the glass case of the instrument, and thus puts the needle into
electrical connection with the acid. The lower end of the axle
also carries a piece of platinum, immersed in the acid which serves
to check the vibrations of the needle. The needle hangs within
a shallow cylindrical brass box, with circular apertures in the
centre of its top and bottom. This box
is divided into four quadrants, \(a\), \(b\), \(c\), \(d\),
which are separately mounted on glass
stems, and thus insulated from the case
and from one another. The quadrant \(b\)
is removed in the figure to shew the
needle. The position of the needle, when
in equilibrium, is such, that one end is
half in the quadrant \(a\) and half in \(c\),
while the other end is half in \(b\) and
half in \(d\).

\wrapfig{0.4}{028.png}{Fig. 2.}
The electrode \(l\) is connected with the
quadrant \(a\) and also with \(d\) through the
wire \(w\). The other electrode, \(m\), is connected
with the quadrants \(b\) and \(c\).

The needle, \(u\), is kept always at a
high potential, generally positive. To
test the difference of potential between
any body and the earth, one of the electrodes, say \(m\), is connected
to the earth, and the other, \(l\), to the body to be tested.
\Runhead{QUADRANT ELECTROMETER.}

The quadrants \(b\) and \(c\) are therefore at potential zero, the
quadrants \(a\) and \(d\) are at the potential to be tested, and the needle
\(u\) is at a high positive potential.

The whole surface of the needle is electrified positively, and
the whole inner surface of the quadrants is electrified negatively,
but the greatest electrification and the greatest attraction is, other
things being equal, where the difference of potentials is greatest.
If, therefore, the potential of the quadrants \(a\) and \(d\) is positive,
the needle will move from \(a\) and \(d\) towards \(b\) and \(c\) or in the
direction of the hands of a watch. If the potential of \(a\) and \(d\) is
negative, the needle will move towards these quadrants, or in the
opposite direction to that of the hands of a watch.

The higher the potential of the needle, the greater will be the
force tending to turn the needle, and the more distinct will be the
indications of the instrument.
%%-----File: 029.png-----%%

\Subsection{Idiostatic and Heterostatic Instruments.}

\article{14} In the gold-leaf electroscope, the only electrification in
the field is the electrification to be tested. In the Quadrant
Electrometer the needle is kept always charged. Instruments in
which the only electrification is that which we wish to test, are
called Idiostatic. Those in which there is electrification independent
of that to be tested are called Heterostatic. In an
idiostatic instrument, like the gold-leaf electroscope, the indications
are the same, whether the potential to be tested is positive or
negative, and the amount of the indication is, when very small,
nearly as the square of the difference of potential. In a heterostatic
instrument, like the quadrant electrometer, the indication
is to the one side or to the other, as the potential is positive or
negative, and the amount of the indication is proportional to the
difference of potentials, and not to the square of that difference.
Hence an instrument on the heterostatic principle, not only indicates
of itself whether the potential is positive or negative, but
when the potential is very small its motion for a small variation
of potential is as great as when the potential is large, whereas in
the gold-leaf electroscope a very small electrification does not cause
the gold leaves to separate sensibly.
\Runhead{IDIOSTATIC AND HETEROSTATIC INSTRUMENTS.}

In Thomson's Quadrant Electrometer there is a contrivance by
which the potential of the needle is adjusted to a constant value,
and there are other organs for special purposes, which are not
represented in the figure which is a mere diagram of the most
essential parts of the instrument.

\Section{On Insulators.}

\article{15} In electrical experiments it is often necessary to support
an electrified body in such a way that the electricity may not
escape. For this purpose, nothing is better than to set it on a
stand supported by a glass rod, provided the surface of the glass
is quite dry. But, except in the very driest weather, the surface
of the glass has always a little moisture condensed on it. For
this reason electrical apparatus is often placed before a fire, before
it is to be used, so that the moisture of the air may not condense
on the warmed surface of the glass. But if the glass is made
too warm, it loses its insulating power and becomes a good
conductor.
%%-----File: 030.png-----%%

Hence it is best to adopt a method by which the surface of the
glass may be kept dry without heating it.
\Runhead{INSULATORS.}

\wrapfig{0.3}{030.png}{Fig. 3.}
The insulating stand in the figure consists of a glass vessel \(C\),
with a boss rising up in the middle to which
is cemented the glass pillar \(a~a\). To the upper
part of this pillar is cemented the neck of the
bell glass \(B\), which is thus supported so that
its rim is within the vessel \(C\), but does not
touch it. The pillar \(a\) carries the stand \(A\) on
which the body to be insulated is placed.

In the vessel \(C\) is placed some strong sulphuric
acid \(c\), which fills a wide shallow moat
round the boss in the middle. The air within
the bell glass \(B\), in contact with the pillar \(a\), is
thus dried, and before any damp air can enter
this part of the instrument, it must pass down between \(C\) and \(B\)
and glide over the surface of the sulphuric acid, so that it is
thoroughly dried before it reaches the glass pillar. Such an insulating
stand is very valuable when delicate experiments have to
be performed. For rougher purposes insulating stands may be
made with pillars of glass varnished with shellac or of sealing-wax
or ebonite.

\article{16} For carrying about an electrified conductor, it is very
convenient to fasten it to the end of an ebonite rod. Ebonite,
however, is very easily electrified. The slightest touch with the
hand, or friction of any kind, is sufficient to render its surface so
electrical, that no conclusion can be drawn as to the electrification
of the conductor at the end of the rod.

The rod therefore must never be touched but must be carried
by means of a handle of metal, or of wood covered with tinfoil,
and before making any experiment the whole surface of the ebonite
must be freed from electrification by passing it rapidly through a
flame.

The sockets by which the conductors are fastened to the ebonite
rods, should not project outwards from the conductors, for the
electricity not only accumulates on the projecting parts, but creeps
over the surface of the ebonite, and remains there when the
electricity of the conductor is discharged. The sockets should
therefore be entirely within the outer surface of the conductors
as in Fig. 4.

It is convenient to have a brass ball (Fig. 4), one inch in
%%-----File: 031.png-----%%
diameter, a cylindrical metal vessel (Fig. 5) about three inches
in diameter and five or six inches deep, a pair of disks of tin
plate (Figs. 6, 7), two inches in diameter, and a thin wire about
a foot long (Fig. 8) to make connection between electrified bodies.
These should all be mounted on ebonite rods (penholders), one
eighth of an inch in diameter, with handles of metal or of wood
covered with tinfoil.
\Runhead{APPARATUS.}
\begin{figure}[ht!]
\includegraphics[width=0.95\textwidth]{031.png}
\centering
\end{figure}
%%-----File: 032.png-----%%

\newchapter
\Chapter{CHAPTER II.}
\Subheading{ON THE CHARGES OF ELECTRIFIED BODIES.}
\Section{Experiment VI.}

\article{17} \textsc{Take} any deep vessel of metal,---a pewter ice-pail was used
by Faraday,---a piece of wire gauze rolled into a cylinder and set
on a metal plate is very convenient, as it allows any object within
it to be seen. Set this vessel on an insulating stand, and place
an electroscope near it. Connect one electrode of the electroscope
permanently with the earth or the walls of the room, and the
other with the insulated vessel, either permanently by a wire
reaching from the one to the other, or occasionally by means of
a wire carried on an ebonite rod and made to touch the vessel
and the electrode at the same time. We shall generally suppose
the vessel in permanent connection with the electroscope.
The simplest way when a gold-leaf
electroscope is used is to set the vessel on the
top of it.

\wrapfig{0.27}{032.png}{Fig. 9.}
Take a metal ball at the end of an ebonite rod,
electrify it by means of the electrophorus, and
carrying it by the rod as a handle let it down into
the metal vessel without touching the sides.

As the electrified ball approaches the vessel the
indications of the electroscope continually increase,
but when the ball is fairly within the vessel, that
is when its depth below the opening of the vessel
becomes considerable compared with the diameter
of the opening, the indications of the electroscope no longer increase,
but remain unchanged in whatever way the ball is moved
about within the vessel.

This statement, which is approximately true for any deep vessel,
is rigorously true for a closed vessel. This may be shewn by
%%-----File: 033.png-----%%
closing the mouth of the vessel with a metal lid worked by means
of a silk thread. If the electrified ball be drawn up and let
down in the vessel by means of a silk thread passing through a
hole in the lid, the external electrification of the vessel as indicated
by the electrometer will remain unchanged, while the ball
changes its position within the vessel. The electrification
of the gold leaves when tested is found to be
of the \textit{same} kind as that of the ball.

\widefig{0.18}{033.png}{Fig. 10.}
Now touch the outside of the vessel with the finger,
so as to put it in electric communication with the
floor and walls of the room. The external electrification
of the vessel will be discharged, and the gold
leaves of the electroscope will collapse. If the ball be
now moved about within the vessel, the electroscope
will shew no signs of electrification; but if the ball
be taken out of the vessel without touching the sides,
the gold leaves will again diverge as much as they did
during the first part of the experiment. Their electrification
however will be found to be of the \textit{opposite} kind from
that of the ball.

\Section{Experiment VII.}

\Subsection{To compare the charges or total Electrification of two
electrified balls.}

\article{18} Since whatever be the position of the electrified bodies
within the vessel its external electrification is the same, it must
depend on the total electrification of the bodies within it, and
not on the distribution of that electrification. Hence, if two balls,
when alternately let down into the vessel, produce the same divergence
of the gold leaves, their charges must be equal. This may
be further tested by discharging the outside of the vessel when
the first ball is in it, and then removing it and letting the second
ball down into the vessel. If the charges are equal, the electroscope
will still indicate no electrification.
\Runhead{COMPARISON OF CHARGES.}

If we wish to ascertain whether the charges of two bodies,
oppositely electrified, are numerically equal, we may do so by
discharging the vessel and then letting down both bodies into
it. If the charges are equal and opposite, the electroscope will not
be affected.
%%-----File: 034.png-----%%

\Section{Experiment VIII.}

\hangpara{
When an electrified body is hung up within a closed metallic vessel,
the total electrification of the inner surface of the vessel is equal
and opposite to that of the body.
}

\article{19} After hanging the body within the vessel, discharge the
external electrification of the vessel, and hang up the whole within
a larger vessel connected with the electroscope. The electroscope
will indicate no electrification, and will remain unaffected even
if the electrified body be taken out of the smaller vessel and moved
about within the larger vessel. If, however, either the electrified
body or the smaller vessel be removed from the large vessel, the
electroscope will indicate positive or negative electrification.

When an electrified body is placed within a vessel free of charge,
the external electrification is equal to that of the body. This
follows from the fact already proved that the internal electrification
is equal and opposite to that of the body, and from the circumstance
that the total charge of the vessel is zero.

But it may also be proved experimentally by placing, first the
electrified body itself, and then the electrified body surrounded
by an uncharged vessel, within the larger vessel and observing
that the indications of the electroscope are the same in both
cases.

\Section{Experiment IX.}

\hangpara{
When an electrified body is placed within a closed vessel and then
put into electrical connection with the vessel, the body is completely
discharged.}

\article{20} In performing any of the former experiments bring the
electrified body into contact with the inside of the vessel, and
then take it out and test its charge by placing it within another
vessel connected with the electroscope. It will be found quite
free of charge. This is the case however highly the body may
have been originally electrified, and however highly the vessel
itself, the inside of which it is made to touch, may be electrified.
\Runhead{TO DISCHARGE A BODY COMPLETELY.}

If the vessel, during the experiment, is kept connected with the
electroscope, no alteration of the external electrification can be
detected at the moment at which the electrified body is made to
touch the inside of the vessel. This affords another proof that
the electrification of the interior surface is equal and opposite to
%%-----File: 035.png-----%%
that of the electrified body within it. It also shews that when
there is no electrified body within the surface every part of that
surface is free from charge.

\Section{Experiment X.}

\Subsection{To charge a vessel with any number of times the charge of a given
electrified body.}

\article{21} Place a smaller vessel within the given vessel so as to be
insulated from it. Place the electrified body within the inner
vessel, taking care not to discharge it. The exterior
charges of the inner and outer vessels will
now be equal to that of the body, and their interior
charges will be numerically equal but of
the opposite kind. Now make electric connection
between the two vessels. The exterior charge of
the inner vessel and the interior charge of the
outer vessel will neutralise each other, and the
outer vessel will now have a charge equal to that
of the body, and the inner vessel an equal and opposite
charge.

Now remove the electrified body; take out the
inner vessel and discharge it; then replace it;
place the electrified body within it; and make contact between the
vessels. The outer vessel has now received a double charge, and
by repeating this process any number of charges, each equal to
that of the electrified body, may be communicated to the outer
vessel.

\wrapfig{0.27}{035.png}{Fig. 11.}
To charge the outer vessel with electrification opposite to that
of the electrified body is still easier. We have only to place the
electrified body within the smaller vessel, to put this vessel for a
moment in connection with the walls of the room so as to discharge
the exterior electrification, then to remove the electrified
body and carry the vessel into the inside of the larger vessel and
bring it into contact with it so as to give the larger vessel its
negative charge, and then to remove the smaller vessel, and to
repeat this process the required number of times.

We have thus a method of comparing the electric charges of
different bodies without discharging them, of producing charges
equal to that of a given electrified body, and either of the same
%%-----File: 036.png-----%%
or of opposite signs, and of adding any number of such charges
together.
\Runhead{MULTIPLE OF A GIVEN CHARGE.}

\article{22} In this way we may illustrate and test the truth of the
following laws of electrical phenomena.

I\@. The total electrification or charge of a body or system of
bodies remains always the same, except in so far as it receives
electrification from, or gives electrification to other bodies.

In all electrical experiments the electrification of bodies is found
to change, but it is always found that this change arises from
defective insulation, and that as the means of insulation are improved,
the loss of electrification becomes less. We may therefore
assert that the electrification of a body cut off from electrical communication
with all other bodies by a perfectly insulating medium
would remain absolutely constant.

II\@. When one body electrifies another by conduction the total
electrification of the two bodies remains the same, that is, the one
loses as much positive or gains as much negative electrification as
the other gains of positive or loses of negative electrification.

For if the electric connection is made when both bodies are
enclosed in a metal vessel, no change of the total electrification is
observed at the instant of contact.

III\@. When electrification is produced by friction or by any other
known method, equal quantities of positive and of negative electricity
are produced.
\Runhead{LAWS OF ELECTRICAL PHENOMENA.}

For if the process of electrification is conducted within the
closed vessel, however intense the electrification of the parts of
the system may be, the electrification of the whole, as indicated by
the electroscope connected with the vessel, remains zero.

IV\@. If an electrified body or system of bodies be placed within
a closed conducting surface (which may consist of the floor, walls,
and ceiling of the room in which the experiment is made), the interior
electrification of this surface is equal and opposite to the
electrification of the body or system of bodies.

V\@. If no electrified body is placed within the hollow conducting
surface, the electrification of this surface is zero. This is true, not
only of the electrification of the surface as a whole, but of every
part of this surface.

For if a conductor be placed within the surface and in contact
with it, the surface of this conductor becomes electrically continuous
with the interior surface of the enclosing vessel, and it is found
that if the conductor is removed and tested, its electrification is
%%-----File: 037.png-----%%
always zero, shewing that the electrification of every part of an
interior surface within which there is no electrified body is zero.

By means of Thomson's Quadrant Electrometer it is easy to
measure the electrification of a body when it is a million times less
than when charged to an amount convenient for experiment.
Hence the experimental evidence for the above statements shews
that they cannot be erroneous to the extent of one-millionth of the
principal electrifications concerned.
%%-----File: 038.png-----%%

\newchapter
\Chapter{CHAPTER III.}
\Subheading{ON ELECTRICAL WORK AND ENERGY.}

\article{23} \textsc{Work} in general is the act of producing a change of configuration
in a material system in opposition to a force which
resists this change.

Energy is the capacity of doing work.

When the nature of a material system is such that if after the
system has undergone any series of changes it is brought back
in any manner to its original state, the whole work done by
external agents on the system is equal to the whole work done
by the system in overcoming external forces, the system is called
a Conservative system.

The progress of physical science has led to the investigation of
different forms of energy, and to the establishment of the doctrine,
that all material systems may be regarded as conservative systems
provided that all the different forms of energy are taken into account.
This doctrine, of course, considered as a deduction from experiment,
can assert no more than that no instance of a non-conservative
system has hitherto been discovered, but as a scientific or science-producing
doctrine it is always acquiring additional credibility
from the constantly increasing number of deductions which have
been drawn from it, which are found in all cases to be verified.
In fact, this doctrine is the one generalised statement which is
found to be consistent with fact, not in one physical science only,
but in all. When once apprehended it furnishes to the physical
enquirer a principle on which he may hang every known law
relating to physical actions, and by which he may be put in the
way to discover the relations of such actions in new branches of
science. For such reasons the doctrine is commonly called the
Principle of the Conservation of Energy.
%%-----File: 039.png-----%%

\Section{General Statement of the Conservation of Energy.}

\article{24} The total energy of any system of bodies is a quantity
which can neither be increased nor diminished by any mutual
action of those bodies, though it may be transformed into any of
the forms of which energy is susceptible.

If, by the action of some external agent, the configuration of the
system is changed, then, if the forces of the system are such as to
\textit{resist} this change of configuration, the external agent is said to do
work on the system. In this case the energy of the system is
\textit{increased}. If, on the contrary, the forces of the system tend to
\textit{produce} the change of configuration, so that the external agent has
only to \textit{allow} it to take place, the system is said to do work on
the external agent, and in this case the energy of the system is
diminished. Thus when a fish has swallowed the angler's hook
and swims off, the angler following him for fear his line should
break, the fish is doing work against the angler, but when the fish
becomes tired and the angler draws him to shore, the angler is
doing work against the fish.

Work is always measured by the product of the change of
configuration into the force which resists that change. Thus, when
a man lifts a heavy body, the change of configuration is measured
by the increase of distance between the body and the earth, and
the force which resists it is the weight of the body. The product
of these measures the work done by the man. If the man, instead
of lifting the heavy body vertically upwards, rolls it up an inclined
plane to the same height above the ground, the work done against
gravity is precisely the same; for though the heavy body is moved
a greater distance, it is only the vertical component of that
distance which coincides in direction with the force of gravity
acting on the body.

\article{25} If a body having a positive charge of electricity is carried
by a man from a place of low to a place of high potential, the
motion is opposed by the electric force, and the man does work on
the electric system, thereby increasing its energy. The amount
of work is measured by the product of the number of units of
electricity into the increase of potential in moving from the one
place to the other.
\Runhead{ELECTRIC POTENTIAL.}

We thus obtain the dynamical definition of electric potential.
%%-----File: 040.png-----%%

\textit{The electric potential at a given point of the field is measured by
the amount of work which must be done by an external agent in
carrying one unit of positive electricity from a place where the potential
is zero to the given point.}

This definition is consistent with the imperfect definition given
at \hyperref[art:6]{Art.\ 6}, for the work done in carrying a unit of electricity from
one place to another will be positive or negative according as the
displacement is from lower to higher or from higher to lower
potential. In the latter case the motion, if not prevented, will
take place, without any interference from without, in obedience to
the electric forces of the system. Hence the flow of electricity
along conductors is always from places of high to places of low
potential.

\article{26} We have already defined the electromotive force from one
place to another along a given path as the work done by the
electric forces of the system on a unit of electricity carried along
that path. It is therefore measured by the excess of the potential
at the beginning over that at the end of the path.
\Runhead{ELECTROMOTIVE FORCE.}

The electromotive force \textit{at a point} is the force with which the
electrified system would act on a small body electrified with a unit
of positive electricity, and placed at that point.

If the electrified body is moved in such a way as to remain on
the same equipotential surface, no work is done by the electric
forces or against them. Hence the direction of the electric force
acting on the small body is such that any displacement of the body
along any line drawn on the equipotential surface is at right angles
to the force. The direction of the electromotive force, therefore, is
at right angles to the equipotential surface.

The magnitude of this force, multiplied by the distance between
two neighbouring equipotential surfaces, gives the work done in
passing from the one equipotential surface to the other, that is to
say, the difference of their potentials.

Hence the magnitude of the electric force may be found by
dividing the difference of the potentials of two neighbouring equipotential
surfaces by the distance between them, the distance
being, of course, very small, and measured perpendicularly to
either surface. The direction of the force is that of the normal
to the equipotential surface through the given point, and is
reckoned in the direction in which the potential \textit{diminishes}.
%%-----File: 041.png-----%%

\Section{Indicator Diagram of Electric Work.}

\article{27} The indicator diagram, employed by Watt for measuring
the work done by a steam engine\footnote{
Maxwell's `Theory of Heat,' 4th ed., p.\ 102.}, may be made use of in investigating
the work done during the charging of a conductor with
electricity.

\widefig{0.65}{041.png}{Fig. 12.}
Let the charge of the conductor at any instant be represented by
a horizontal line \(OC\), drawn from the point \(O\), which is called the
\textit{origin} of the diagram, and let the potential of the conductor at
the same instant be represented by a vertical line \(CA\), drawn from
the extremity of the first line, then the position of the extremity
of the second line will indicate the electric state of the conductor,
both with respect to its charge, and also with respect to its
potential.

\Runhead{DIAGRAM OF WORK.}
If during any electrical operation this point moves along the
line \(AFGHB\), we know not only that the charge has been increased
from the value \(OC\) to the value \(OD\), and that the potential has
been increased from \(CA\) to \(DB\), but that the charge and the
potential at any instant, corresponding, say, to the point \(F\) of the
curve, are represented respectively by \(Ox\) and \(xF\).

\article{28} \textit{Theorem.} The work expended by an external agent in
bringing the increment of charge from the walls of the room to
the conductor is represented by the area enclosed by the base line
\(CD\), the two vertical lines \(CA\) and \(DB\), and the curve \(AFGHB\).

For let \(CD\), the increment of the charge, be divided into any
number of equal parts at the points, \(x\), \(y\), \(z\).
%%-----File: 042.png-----%%

The value of the potential just before the application of the
charge \(Cx\) is represented by \(AC\). Hence if the potential were to
remain constant during the application of the charge \(Cx\), the work
expended in charging the conductor would be represented by the
product of this potential and the charge, or by the area \(ACxQ\).

\Runhead{WORK DONE IN CHARGING A CONDUCTOR.}
As soon as the charge \(Cx\) has been applied the potential is \(xF\).
If this had been the value of the potential during the whole
process, the work expended would have been represented by
\(KCxF\). But we know that the potential rises gradually during
the application of the charge, and that during the whole process
it is never less than \(CA\) or greater than \(xF\). Hence the work
expended in charging the conductor is not less than \(ACxQ\), nor
greater than \(KCxF\).

In the same way we may determine the lower and higher
limits of the work done during the application of any other portion
of the entire charge.

We conclude, therefore, that the work expended in increasing
the charge from \(OC\) to \(OD\) is not less than the area of the figure
\(CAQFRGSHTD\), nor greater than \(CKFLGMHNBD\). The difference
between these two values is the sum of the parallelograms
\(KQ\), \(LR\), \(MS\), \(NT\), the breadths of which are equal, and their
united height is \(BV\). Their united area is therefore equal to that
of the parallelogram \(NvVB\).

By increasing without limit the number of equal parts into
which the charge is divided, the breadth of the parallelograms will
be diminished without limit. In the limit, therefore, the difference
of the two values of the work vanishes, and either value becomes
ultimately equal to the area \(CAFGHBD\), bounded by the curve,
the extreme ordinates, and the base line.

This method of proof is applicable to any case in which the
potential is always increasing or always diminishing as the charge
increases. When this is not the case, the process of charging
may be divided into a number of parts, in each of which the
potential is either always increasing or always diminishing, and
the proof applied separately to each of these parts.

\Section{Superposition of Electric Effects.}

\article{29} It appears from Experiment VII that several electrified bodies
placed in a hollow vessel produce each its own effect on the
electrification of the vessel, in whatever positions they are placed.
%%-----File: 043.png-----%%
From this it follows that one electric phenomenon at least, that
called electrification by induction, is such that the effect of the
whole electrification is the sum of the effects due to the different
parts of the electrification. The different electrical phenomena,
however, are so intimately connected with each other that we are
led to infer that all other electrical phenomena may be regarded
as composed of parts, each part being due to a corresponding part
of the electrification.

\Runhead{SUPERPOSITION OF ELECTRIC EFFECTS.}
Thus if a body \(A\), electrified in a definite manner, would produce
a given potential, \(P\), at a given point of the field, and if a body, \(B\),
also electrified in a definite manner, would produce a potential, \(Q\),
at the same point of the field, then when both bodies, still electrified
as before, are introduced simultaneously into their former
places in the field, the potential at the given point will be \(P + Q\).
This statement may be verified by direct experiment, but its most
satisfactory verification is founded on a comparison of its consequences
with actual phenomena.

As a particular case, let the electrification of every part of the
system be multiplied \(n\) fold. The potential at every point of the
system will also be multiplied by \(n\).

\article{30} Let us now suppose that the electrical system consists of a
number of conductors (which we shall call \(A_1\), \(A_2\), \&c.) insulated from
each other, and capable of being charged with electricity. Let the
charges of these conductors be denoted by \(E_1\), \(E_2\), \&c., and their
potentials by \(P_1\), \(P_2\), \&c.

\wrapfig{0.4}{043.png}{Fig. 13.}
If at first the conductors are all without charge, and at potential
zero, and if at a certain instant each conductor begins to be charged
with electricity, so that the charge
increases uniformly with the time,
and if this process is continued till
the charges simultaneously become
\(E_1\) for the first conductor, \(E_2\) for the
second, and so on, then since the increment
of the charge of any conductor
is the same for every equal
interval of time during the process,
the increment of the potential of
each conductor will also be the same
for every equal increment of time, so that the line which represents,
on the indicator diagram, the succession of states of a given conductor
with respect to charge and potential will be described with
%%-----File: 044.png-----%%
a velocity, the horizontal and vertical components of which remain
constant during the process. This line on the diagram is therefore
a straight line, drawn from the origin, which represents the initial
state of the system when devoid of charge and at potential zero, to
the point \(A_1\) which indicates the final state of the conductor when
its charge is \(E_1\), and its potential \(P_1\), and will represent the process
of charging the conductor \(A_1\). The work expended in charging
this conductor is represented by the area \(OCA\), or half the product
of the final charge \(E_1\) and the final potential \(P_1\).

\Section{Energy of a System of Electrified Bodies.}

\Runhead{ENERGY OF AN ELECTRIFIED SYSTEM.}
\article{31} When the relative positions of the conductors are fixed, the
work done in charging them is entirely transformed into electrical
energy. If they are charged in the manner just described, the
work expended in charging any one of them is \(\tstrut\frac{1}{2}EP\), where \(E\) represents
its final charge and \(P\) its final potential. Hence the work
expended in charging the whole system may be written
\[\tfrac{1}{2}E_1P_1 + \tfrac{1}{2}E_2P_2 + \text{\&c.,}\]
there being as many products as there are conductors in the system.

It is convenient to write the sum of such a series of terms in the
form
\[\tfrac{1}{2}\sum(EP),\]
where the symbol \(\sum\) (sigma) denotes that all the products of the
form \(EP\) are to be summed together, there being one such product
for each of the conductors of which the system consists.

Since an electrified system is subject to the law of Conservation
of Energy, the work expended in charging it is entirely stored up
in the system in the form of electrical energy. The value of this
energy is therefore equal to that of the work which produced it, or
\(\tstrut\frac{1}{2}\sum(EP)\). But the electrical energy of the system depends altogether
on its actual state, and not on its previous history. Hence

\Section{Theorem I.}

\hangpara{
The electrical energy of a system of conductors, in whatever way
they may have been charged, is half the sum of the products of the
charge into the potential of each conductor.
}

We shall denote the electric energy of the system by the symbol
\(Q\), where
\[Q = \tfrac{1}{2}\sum(EP)\text{.} \tag{1}\]
%%-----File: 045.png-----%%

\Subsection{Work done in altering the charges of the system.}

\Runhead{WORK DONE IN ALTERING CHARGES.}
\article{32} We shall next suppose that the conductors of the system,
instead of being originally without charge and at potential zero, are
originally charged with quantities \(E_1\), \(E_2\), \&c.\ of electricity, and are
at potentials \(P_1\), \(P_2\), \&c.\ respectively.

\wrapfig{0.5}{045.png}{Fig. 14.}
When in this state let the charges of the conductors be changed,
each at a uniform rate, the rate being, in general, different for
each conductor, and let this process go on uniformly, till the
charges have become \({E_1}'\), \({E_2}'\), \&c., and the potentials \({P_1}'\), \({P_2}'\), \&c.\
respectively.

By the principle of the superposition of electrical effects the increment
of the potential will vary as the increment of the charge,
and the potential of each conductor
will increase or diminish
at a uniform rate from \(P\) to \(P'\),
while its charge varies at a uniform
rate from \(E\) to \(E'\). Hence
the line \(AA'\), which represents
the varying state of the conductor
during the process, is the
straight line which joins \(A\), the
point which indicates its original
state, with \(A'\), which represents
its final state. The work spent
in producing this increment of
charge in the conductor is represented by the area \(ACC'A'\), or
\(\tstrut\frac{1}{2}CC'(CA + C'A')\), or \((E' - E)\tstrut\frac{1}{2}(P + P')\), or, in words, it is the product
of the increase of charge and the half sum of the potentials
at the beginning and end of the operation, and this will be true for
every conductor of the system.

As, during this process, the electric energy of the system changes
from \(Q\), its original, to \(Q'\), its final value, we may write
\[Q' = Q + \tfrac{1}{2}\sum\{(E' - E)(P' + P)\}, \tag{2}\]
hence,
\Section{Theorem II.}

\hangpara{
The increment of the energy of the system is half the sum of the
products of the increment of charge of each conductor into the
sum of its potentials at the beginning and the end of the process.
}

\article{33} If all the charges but one are maintained constant (by the
insulation of the conductors) the equation (2) is reduced to
%%-----File: 046.png-----%%
\begin{align*}
Q' - Q &= (E' - E)\tfrac{1}{2}(P' + P)\text{,}\\
\shortintertext{or}
\frac{Q' - Q}{E' - E} &= \tfrac{1}{2}(P' + P)\text{.} \tag{3}
\end{align*}

If the increment of the charge is taken always smaller and smaller
till it ultimately vanishes, \(P'\) becomes equal to \(P\) and the equation
may be interpreted thus:---

The \textit{rate} of increase of the electrical energy due to the increase
of the charge of one of the conductors at a rate unity is numerically
equal to the potential of that conductor.

In the notation of the differential calculus this result is expressed
by the equation
\[\frac{dQ_e}{dE} = P, \tag{4}\]
in which it is to be remembered that all the charges but one are
maintained constant.

\Runhead{GREEN'S THEOREM.}
\article{34} Returning to equation (2), we have already shewn that
\[Q = \tfrac{1}{2}\sum(EP) \quad\text{and}\quad Q' = \tfrac{1}{2}\sum(E'P'); \tag{5}\]
we may therefore write equation (2)
\[\tfrac{1}{2}\sum(E'P') = \tfrac{1}{2}\sum(EP) + \tfrac{1}{2}\sum(E'P' - EP + E'P - EP'). \tag{6}\]

Cutting out from the equation the terms which destroy each
other, we obtain
\[\sum(EP') = \sum(E'P), \tag{7}\]
or in words,
\Section{Theorem III.}
\hangpara{
\textit{In a fixed system of conductors the sum of the products of the original
charge and the final potential of each conductor is equal to the sum
of the products of the final charge and the original potential.}
}

This theorem corresponds, in the elementary treatment of electrostatics,
to Green's Theorem in the analytical theory. By properly
choosing the original and the final state of the system we may
deduce a number of results which we shall find useful in our after-work.

\article{35} In the first place we may write, as before,
\[\tfrac{1}{2}\sum\{(E' - E)(P' + P)\} = \tfrac{1}{2}\sum(E'P' - EP + E'P - EP'); \tag{8}\]
adding and subtracting the equal quantities of equation (7),
\[0 = \sum(EP' - E'P), \tag{9}\]
and the right-hand side becomes
\[\tfrac{1}{2}\sum(E'P' - EP - E'P + EP'), \tag{10}\]
or
\[\tfrac{1}{2}\sum\{(E' - E)(P' + P)\} = Q' - Q = \tfrac{1}{2}\sum\{(E' + E)(P' - P)\}, \tag{11}\]
or in words,
%%-----File: 047.png-----%%
\Section{Theorem IV.}
\hangpara{
\textit{The increment of the energy of a fixed system of conductors is equal
to half the sum of the products of the increment of the potential
of each conductor into the sum of the original and final charges
of that conductor.}
}

\article{36} If all the conductors but one are maintained at constant
potentials (which may be done by connecting them with voltaic
batteries of constant electromotive force), equation (11) is reduced to
\begin{align*}
Q' - Q &= \tfrac{1}{2}(E' + E)(P' - P), \tag{12}\\
\shortintertext{or}
\frac{Q' - Q}{P' - P} &= \tfrac{1}{2}(E' + E). \tag{13}
\end{align*}

If the increment of the potential is taken successively smaller and
smaller, till it ultimately vanishes, \(E'\) becomes at last equal to \(E\)
and the equation may be interpreted thus:---

The \textit{rate} of increase of the electrical energy due to the increase
of potential of one of the conductors at a rate unity is numerically
equal to the charge of that conductor.

In the notation of the differential calculus this result is expressed
by the equation
\[\frac{dQ_p}{dP} = E, \tag{14}\]
in which it is to be remembered that all the potentials but one are
maintained constant.

\article{37} We have next to point out some of the results which may
be deduced from Theorem III.
\Runhead{RECIPROCITY OF POTENTIALS.}

If any conductor, as \(A_t\), is insulated and without charge both in
the initial and the final state, then \(E_t = 0\) and \({E_t}' = 0\), and therefore
\[E_t{P_t}' = 0\quad\text{and}\quad {E_t}'P_t = 0, \tag{15}\]
so that the terms depending on \(A_t\) disappear from both members
of equation (7).

Again, if another conductor, say \(A_u\), be connected with the earth
both in the initial and in the final state, \(P_u = 0\) and \({P_u}' = 0\), so that
\[E_u{P_u}' = 0\quad\text{and}\quad {E_u}'P_u = 0;\]
so that, in this case also, the terms depending on \(A_u\) disappear from
both sides of equation (7).

If, therefore, all the conductors with the exception of two, say
\(A_r\) and \(A_s\), are either insulated and without charge, or else connected
with the earth, equation (7) is reduced to the form
\[E_r{P_r}' + E_s{P_s}' = {E_r}'P_r + {E_s}'P_s. \tag{16}\]
%%-----File: 048.png-----%%

Let us first suppose that in the initial state all the conductors
except \(A_r\) are without charge, and that in the final state all the conductors
except \(A_s\) are without charge. The equation then becomes
\begin{align*}
E_r{P_r}' &= {E_s}'P_s, \tag{17}\\
\shortintertext{or}
\frac{P_s}{E_r} &= \frac{{P_r}'}{{E_s}'},
\end{align*}

[If, therefore, \qquad \(E_r = {E_s}', \qquad P_s = {P_r}'\)],\\
or in words,
\Section{Theorem V.}
\hangpara{
In a system of fixed insulated conductors, the potential (\(P_s\)) produced
in \(A_s\) by a charge \(E\) communicated to \(A_r\) is equal to the potential
(\({P_r}'\)) produced in \(A_r\) by an equal charge \(E\) communicated to \(A_s\).}

This is the first instance we have met with of the \textit{reciprocal}
relation of two bodies. There are many such reciprocal relations.
They occur in every branch of science, and they often enable us
to deduce the solution of new electrical problems from those of
simpler problems with which we are already familiar. Thus, if
we know the potential which an electrified sphere produces at a
point in its neighbourhood, we can deduce the effect which a small
electrified body, placed at that point, would have in raising the
potential of the sphere.

\Runhead{RECIPROCITY OF POTENTIALS AND CHARGES.}
\article{38} Let us next suppose that the original potential of \(A_s\) is \(P_s\)
and that all the other conductors are kept at potential zero by
being connected with the walls of the room, and let the final
potential of \(A_r\) be \({P_r}'\), that of all the others being zero, then in
equation (7) all the terms involving zero potentials will vanish,
and we shall have in this case also
\[E_r{P_r}' = {E_s}'P_s. \tag{18}\]
If, therefore,
\[{P_r}' = P_s, \qquad E_r = {E_s}', \tag{19}\]
or in words,
\Section{Theorem VI.}
\hangpara{In a system of fixed conductors, connected, all but one, with the walls
of the room, the charge (\(E_r\)) induced on \(A_r\) when \(A_s\) is raised to
the potential \(P_s\) is equal to the charge (\({E_s}'\)) induced on \(A_s\) when
\(A_r\) is raised to an equal potential (\({P_r}'\)).}

\article{39} As a third case, let us suppose all the conductors insulated
and without charge, and that a charge is communicated to \(A_r\)
%%-----File: 049.png-----%%
which raises its potential to \(P_r\) and that of \(A_s\) to \(P_s\). Next, let \(A_s\)
be connected with the earth, and let a charge \({E_r}'\) on \(A_r\) induce the
charge \({E_s}'\) on \(A_s\).

In equation (16) we have \(E_r = 0\) and \({P_s}' = 0\), so that the left-hand
member vanishes and the equation becomes
\[0 = {E_r}'P_r + {E_s}'P_s, \tag{20}\]
or
\[\frac{P_s}{P_r} = -\frac{{E_r}'}{{E_s}'}.\]
Hence, if
\[P_s = nP_r, \qquad {E_r}' = -n{E_s}', \tag{21}\]
or in words,

\Section{Theorem VII.}
\hangpara{If in a system of fixed conductors insulated and originally without
charge a charge be communicated to \(A_r\) which raises its potential
to \(P_r\), unity, and that of \(A_s\) to \(n\), then if in the same system of
conductors a charge unity be communicated to \(A_s\) and \(A_r\) be
connected with the earth the charge induced on \(A_r\) will be \(-n\).}

If, instead of supposing the other conductors \(A_t\) \&c.\ to be all
insulated and without charge, we supposed some or all of them
to be connected with the earth, the theorem would still be true,
only the value of \(n\) would be different according to the arrangement
we adopted.

This is one of Green's theorems. As an example of its application,
let us suppose that we have ascertained the distribution of
electric charge induced on the various parts of the surface of a
conductor by a small electrified body in a given position with unit
charge. Then by means of this theorem we can solve the following
problem:---The potential at every point of a surface coinciding in
position with that of the conductor being given, determine the
potential at a point corresponding to the position of the small
electrified body.

Hence, if the potential is known at all points of any closed
surface, it may be determined for any point within that surface if
there be no electrified body within it, and for any point outside if
there be no electrified body outside.

\Subsection{Mechanical work done by the electric forces during the displacement
of a system of insulated electrified conductors.}
\Runhead{GREEN'S THEOREM ON POTENTIALS AND CHARGES.}

\article{40} Let \(A_1\), \(A_2\) \&c.\ be the conductors, \(E_1\), \(E_2\) \&c.\ their charges,
which, as the conductors are insulated, remain constant. Let \(P_1\),
\(P_2\) \&c.\ be their potentials before and \({P_1}'\), \({P_2}'\) \&c.\ their potentials
%%-----File: 050.png-----%%
after the displacement. The electrical energy of the system before
the displacement is
\[Q = \tfrac{1}{2}\sum(EP). \tag{22}\]

During the displacement the electric forces which act in the
same direction as the displacement perform an amount of work
equal to \(W\), and the energy remaining in the system is
\[Q' = \tfrac{1}{2}\sum(EP'). \tag{23}\]
The original energy, \(Q\), is thus transformed into the work \(W\) and
the final energy \(Q'\), so that the equation of energy is
\begin{align*}
Q &= W + Q'\text{,} \tag{24}\\
\shortintertext{or}
W &= \tfrac{1}{2}\sum[E(P - P')]. \tag{25}
\end{align*}
This expression gives the work done during any displacement,
small or large, of an insulated system. To find the force, we must
make the displacement so small that the configuration of the
system is not sensibly altered thereby. The ultimate value of the
quotient found by dividing the work by the displacement is the
value of the force resolved in the direction of the displacement.

\Subsection{Mechanical work done by the electric force during the displacement
of a system of conductors each of which is kept at a constant
potential.}

\article{41} Let us begin by supposing each conductor of the system
insulated, and that a \textit{small} displacement is given to the system, by
which the potential is changed from \(P\) to \(P_1\). The work done
during this displacement is, as we have shewn,
\[W = \tfrac{1}{2}\sum[E(P - P_1)]. \tag{26}\]
Next, let the conductors remain fixed while the charges of the conductors
are altered from \(E\) to \(E_1\), so as to bring back the value of
the potential from \(P_1\) to \(P\). Then we know by equation (7) that
\[\sum(EP - E_1P_1) = 0. \tag{27}\]
Hence, substituting for \(\sum(EP)\) in (26),
\[W = \tfrac{1}{2}\sum[(E_1 - E)P_1]. \tag{28}\]
Performing these two operations alternately for any number of
times, and distinguishing each pair of operations by a suffix, we
find the whole work
\begin{align}
W &= W_1 + W_2 + \text{\&c.} \tag{29}\\
&= \tfrac{1}{2}\sum[(E_1 - E)P_1] + \tfrac{1}{2}\sum[(E_2 - E_1)P_2] + \text{\&c.} \tag{30}
\end{align}
By making each of the partial displacements small enough, the
%%-----File: 051.png-----%%
values of \(P_1\), \(P_2\) \&c.\ may be made to approach without limit to \(P\),
the constant value of the potential, and the expression becomes
\[W = \tfrac{1}{2}\sum[(E_1 - E)P] + \tfrac{1}{2}\sum[(E_2 - E_1)P] + \text{\&c.} + \tfrac{1}{2}\sum[(E' - E_{n-1})P], \tag{31}\]
where \(E'\) is the value of \(E\) after the last operation. The final result
is therefore
\[W = \tfrac{1}{2}\sum[(E' - E)P], \tag{32}\]
which is an expression giving the work done during a displacement
of any magnitude of a system of conductors, the potential of each of
which is maintained constant during the displacement.
\Runhead{MECHANICAL WORK DURING DISPLACEMENT.}

We may write this result
\begin{align}
W &= \tfrac{1}{2}\sum(E'P) - \tfrac{1}{2}\sum(EP), \tag{33}\\
\shortintertext{or}
W &= Q'- Q\text{;} \tag{34}
\end{align}
or the work done by the electric forces is equal to the \textit{increase} of
the electric energy of the system during the displacement when
the \textit{potential} of each conductor is maintained constant. When the
\textit{charge} of each conductor was maintained constant, the work done
was equal to the \textit{decrease} of the energy of the system.

Hence, when the potential of each conductor is maintained constant
during a displacement in which a quantity of work, \(W\), is
done, the voltaic batteries which are employed to keep the potentials
constant must do an amount of work equal to \(2W\). Of this
energy supplied to the system, half is spent in increasing the
energy of the system, and the other half appears as mechanical
work.
%%-----File: 052.png-----%%

\newchapter
\Chapter{CHAPTER IV.}

\Subheading{THE ELECTRIC FIELD.}
\Runhead{EXPLORATION OF THE ELECTRIC FIELD.}
\article{42} \textsc{We} have seen that, when an electrified body is enclosed in
a conducting vessel, the total electrification of the interior surface
of the surrounding vessel is invariably equal in numerical value
but opposite in kind to that of the body. This is true, however
large this vessel may be. It may be a room of any size having
its floor, walls and ceiling of conducting matter. Its boundaries
may be removed further, and may consist of the surface of the
earth, of the branches of trees, of clouds, perhaps of the extreme
limits of the atmosphere or of the universe. In every case, wherever
we find an electrified insulated body, we are sure to find at
the boundaries of the insulating medium, wherever they may be,
an equal amount of electrification of the opposite kind.

This correspondence of properties between the two limits of
the insulating medium leads us to examine the state of this
medium itself, in order to discover the reason why the electrification
at its inner and outer boundaries should be thus related. In
thus directing our attention to the state of the insulating medium,
rather than confining it to the inner conductor and the outer surface,
we are following the path which led Faraday to many of his
electrical discoveries.

\article{43} To render our conceptions more definite, we shall begin by
supposing a conducting body electrified positively and insulated
within a hollow conducting vessel. The space between the body
and the vessel is filled with air or some other insulating medium.
We call it an \textit{insulating} medium when we regard it simply as
retaining the charge on the surface of the electrified body. When
we consider it as taking an important part in the manifestation
of electric phenomena we shall use Faraday's expression, and call
it a \textit{dielectric} medium. Finally, when we contemplate the region
%%-----File: 053.png-----%%
occupied by the medium as being a part of space in which electric
phenomena may be observed, we shall call this region the Electric
Field. By using this last expression we are not obliged to figure
to ourselves the mode in which the dielectric medium takes part in
the phenomena. If we afterwards wish to form a theory of the
action of the medium, we may find the term dielectric useful.

\Section{Exploration of the Electric Field.}

\Section{Experiment XI.}

\Subsection{\textup{(}\greekfont{α}\textup{)} Exploration by means of a small electrified body.}

\article{44} Electrify a small round body, a gilt pith ball, for example,
and carry it by means of a white silk thread into any part of the
field. If the ball is suspended in such a way that the tension of
the string exactly balances the weight of the ball, then when the
ball is placed in the electric field it will move under the action of a
new force developed by the action of the electrified ball on the
electric condition of the field. This new force tends to move the
ball in a certain direction, which is called the direction of the
electromotive force.

If the charge of the ball is varied, the force is sensibly proportional
to the charge, provided this charge is not sufficient to
produce a sensible disturbance of the state of electrification of the
system. If the charge is positive, the force which acts on the ball
is, on the whole, \textit{from} the positively electrified body, and \textit{towards}
the negatively electrified walls of the room. If the charge is
negative, the force acts in the opposite direction.

Since, therefore, the force which acts on the ball depends partly
on the charge of the ball and partly on its position and on the
electrification of the system, it is convenient to regard this force as
the product of two factors, one being the charge of the ball, and
the other \textit{the electromotive force at that point of the field which is
occupied by the centre of the ball}.

This electromotive force at the point is definite in magnitude
and direction. A positively charged body placed there tends to
move in the positive direction of the line representing the force. A
negatively charged body tends to move in the opposite direction.
%%-----File: 054.png-----%%

\Section{Experiment XII.}

\Subsection{\textup{(}\,\greekfont{β}\textup{)} Exploration by means of two disks.}

\article{45} But the electromotive force not only tends to move electrified
bodies, it also tends to transfer electrification from one part
of a body to another.

\wrapfig{0.55}{054.png}{Fig. 15.}
Take two small equal thin metal disks, fastened to handles of
shellac or ebonite; discharge them and place them face to face in
the electric field, with their planes perpendicular to the direction of
the electromotive force. Bring them into contact, then separate
them and remove them, and
test first one and then the
other by introducing them
into the hollow vessel of Experiment
VII\@. It will be
found that each is charged,
and that if the electromotive
force acts in the direction \(AB\),
the disk on the side of \(A\) will
be charged negatively, and
that on the side of \(B\) positively,
the numerical values of these charges being equal. This shews
that there has been an actual transference of electricity from the one
disk to the other, the direction of this transference being that of
the electromotive force. This experiment with two disks affords a
much more convenient method of measuring the electromotive force
at a point than the experiment with the charged pith ball. The
measurement of small forces is always a difficult operation, and
becomes almost impossible when the weight of the body acted on
forms a disturbing force and has to be got rid of by the adjustment
of counterpoises. The measurement of the charges of the
disks, on the other hand, is much more simple.

The two disks, when in contact, may be regarded as forming a
single disk, and the fact that when separated they are found to
have received equal and opposite charges, shews that while the
disks were in contact there was a distribution of electrification
between them, the electrification of each disk being opposite to
that of the body next to it, whether the insulated body, which is
charged positively, or the inner surface of the surrounding vessel,
which is charged negatively.
%%-----File: 055.png-----%%

\Subsection{Electric Tension.}

\article{46} The two disks, after being brought into contact, tend to
separate from each other, and to approach the oppositely electrified
surfaces to which they are opposed. The force with which they
tend to separate is proportional to the area of the disks, and it
increases as the electromotive force increases, not, however, in the
simple ratio of that force, but in the ratio of the square of the
electromotive force.

The electrification of each disk is proportional to the electromotive
force, and the mechanical force on the disk is proportional
to its electrification and the electromotive force conjointly, that is,
to the \textit{square} of the electromotive force.

This force may be accounted for if we suppose that at every
point of the dielectric at which electromotive force exists there is
a tension, like the tension of a stretched rope, acting in the direction
of the electromotive force, this tension being proportional to
the square of the electromotive force at the point. This tension
acts only on the outer side of each disk, and not on the side which
is turned towards the other disk, for in the space between the disks
there is no electromotive force, and consequently no tension.

The expression Electric Tension has been used by some writers
in different senses. In this treatise we shall always use it in the
sense explained above,---the tension of so many pounds', or grains',
weight on the square foot exerted by the air or other dielectric
medium in the direction of the electromotive force.
\Runhead{ELECTRIC TENSION.}

\Section{Experiment XIII.}

\Subsection{Coulomb's Proof Plane.}

\article{47} If one of these disks be placed with one of its flat surfaces
in contact with the surface of an electrified conductor and
then removed, it will be found to be charged. If the disk is
very thin, and if the electrified surface is so nearly flat that the
whole surface of the disk lies very close to it, the charge of the disk
will be nearly equal to that of the portion of the electrified surface
which it covered. If the disk is thick, or does not lie very close to
the electrified surface, its charge, when removed, will be somewhat
greater.

This method of ascertaining the density of electrification of a
surface was introduced by Coulomb, and the disk when used for
this purpose is called Coulomb's Proof Plane.
\Runhead{COULOMB'S PROOF PLANE.}
%%-----File: 056.png-----%%

The charge of the disk is by Experiment XII proportional to the
electromotive force at the electrified surface. Hence the electromotive
force close to a conducting surface is proportional to the
density of the electrification at that part of the surface.

Since the surface of the conductor is an equipotential surface, the
electromotive force is perpendicular to it. The fact that the electromotive
force at a point close to the surface of a conductor is
perpendicular to the surface and proportional to the density of the
electrification at that point was first established experimentally by
Coulomb, and it is generally referred to as Coulomb's Law.

To prove that when the proof plane coincides with the surface of
the conductor the charge of the proof plane when removed from
the electrified conductor is equal to the charge on the part of the
surface which it covers, we may make the following experiment.

A sphere whose radius is 5 units is placed on an insulating
stand. A segment of a thin spherical shell is fastened to an insulating
handle. The radius of the spherical surface of the shell
is 5, the diameter of the circular edge of the segment is 8, and the
height of the segment is 2. When applied to the sphere it covers
one-fifth part of its surface. A second sphere, whose radius is
also 5, is placed on an insulating handle.

The first sphere is electrified, the segment is then placed in
contact with it and removed. The second sphere is then made to
touch the first sphere, removed and discharged, and then made to
touch the first sphere again. The segment is then placed within
a conducting vessel, which is discharged to earth, and then insulated
and the segment removed. One of the spheres is then
made to touch the outside of the vessel, and is found to be perfectly
discharged.

Let \(e\) be the electrification of the first sphere, and let the charge
removed by the segment be \(ne\), then the charge remaining on the
sphere is \((1 - n)e\). The charge of the first sphere is then divided
with the second sphere, and becomes \(\tstrut\frac{1}{2}(1 - n)e\). The second sphere
is then discharged, and the charge is again divided, so that the
charge on either sphere is \(\tstrut\frac{1}{4}(1 - n)e\). The charge on the insulated
vessel is equal and opposite to that on the segment, and it is therefore
\(-ne\), and this is perfectly neutralized by the charge on one of
the spheres; hence
\begin{align*}
\tfrac{1}{4}(1 - n)e + (-ne) &= 0\text{,}\\
\shortintertext{from which we find}
n &= \tfrac{1}{5}\text{,}
\end{align*}
or the electricity removed by the segment covering one-fifth of the
surface of the sphere is one-fifth of the whole charge of the sphere.
%%-----File: 057.png-----%%

\Section{Experiment XIV.}

\Subsection{Direction of Electromotive Force at a Point.}

\article{48} A convenient way of determining the direction of the electromotive
force is to suspend a small elongated conductor with its
middle point at the given point of the field. The two ends of the
short conductor will become oppositely electrified, and will then be
drawn in opposite directions by the electromotive force, so that the
axis of the conductor will place itself in the direction of the force
at that point. A short piece of fine cotton or linen thread, through
the middle of which a fine silk fibre is passed, answers very well.
The silk fibre, held by both ends, serves to carry the piece of thread
into any desired position, and the thread then takes up the direction
of the electric force at that place.
\Runhead{ELECTROMOTIVE FORCE AND POTENTIAL.}

\Section{Experiment XV.}

\Subsection{Potential at any Point of the Field.}

\article{49} Suspend two small uncharged metal balls in the field by
silk threads, and then connect them by means of a fine metal wire
fastened to the end of an ebonite rod. Remove the wire and the
spheres separately, and then examine the charges of the spheres.

It will be found that the two spheres, if they have become
electrified, have received equal and opposite charges. If the potentials
at the points of the field occupied by the centres of the spheres
are different, positive electrification will be transferred from the
place of high to the place of low potential, and the sphere at the
place of high potential will thus become charged negatively, and
that at the place of low potential will become charged positively.
These charges may be shewn to be proportional to the difference of
potentials at the two places.

We have thus a method of determining points of the field at
which the potential is the same. Place one of the spheres at
a fixed point, and move the other about till, on connecting the
spheres with a wire as before, no charge is found on either sphere.
The potentials of the field at the points occupied by the centres of
the spheres must now be the same. In this way a number of
points may be found, the potential at each of which is equal to that
at a given point.
%%-----File: 058.png-----%%

All these points lie on a certain surface, which is called an equipotential
surface. On one side of this surface the potential is
higher, on the other it is lower, than at the surface itself.

We have seen that electricity has no tendency to flow from one
part of such a surface to another. An electrified body, if constrained
so as to be capable of moving only from one point of the
surface to another, would be in equilibrium, and the force acting on
such a body is therefore everywhere perpendicular to the equipotential
surface.

\Section{Experiment XVI.}

\article{50} We may use one sphere only, and after placing it with its
centre at any given point of the field we may touch it for a moment
with a wire connected to the earth. We may then remove the
sphere and determine its charge. The charge of the sphere is proportional
to the potential at the given point, a positive charge,
however, corresponding to a negative potential.
\Runhead{POTENTIAL DETERMINED BY ONE SPHERE.}

\Subsection{Equipotential Surfaces.}

\article{51} In this way the potential at any number of points in the
field may be ascertained, and equipotential surfaces may be supposed
drawn corresponding to values of the potential represented
by the numbers 1, 2, 3, \&c.

These surfaces will form a series, each, in general, lying within
the preceding surface and having the succeeding surface within it.
No two distinct surfaces can intersect each other, though a particular
equipotential surface may consist of two or more sheets,
intersecting each other at certain lines or points.

The surface of any conductor in electric equilibrium is an equipotential
surface. For if the potential at one point of the conductor
is different from that at another point, electricity will flow
from the place of higher potential to the place of lower potential
till the potentials are rendered equal.

\Section{Experiment XVII.}

\article{52} To make an experimental determination of the equipotential
surfaces belonging to an electrified system we may use a small
exploring sphere permanently connected by a fine wire with one
electrode of the electroscope, the other electrode being connected
with the earth. Place the centre of the sphere at a given point,
%%-----File: 059.png-----%%
and connect the electrodes together for an instant. The indication of
the electroscope will thus be reduced to zero. If the sphere is now
moved in such a manner that the indication of the electrometer
remains zero during the motion, the path of the centre of the
exploring sphere will lie on an equipotential surface. For if it
moves to a place of higher potential, electricity will flow from the
sphere to the electroscope, and if it moves to a place of lower
potential, electricity will flow from the electroscope to the sphere.

If the bodies belonging to the electrified system are not perfectly
insulated, their potentials and the potentials of the points of the
field will tend to approach zero. The path in which the centre of
the exploring sphere moves is such that its potential at any point
has a given value at the time when the centre of the sphere passes
it. The different points of the path are not therefore on a surface
which has the same potential at any one instant, for the potential
is diminishing everywhere, and the path must therefore pass from
surfaces of lower to surfaces of higher potential so as to make up
for this loss.
\Runhead{RECIPROCAL METHOD.}

\Section{Experiment XVIII.}

\article{53} The following method, founded on Theorem V, \hyperref[art:37]{Art.\ 37}, is
therefore
\wrapfig{0.6}{059.png}{Fig. 16.}
in many cases more convenient, as it is much easier to
secure good insulation
for the exploring sphere
on an insulating handle
than for a large electrified
conductor of irregular
form. Let it be required
to determine the
equipotential surfaces
due to the electrification
of the conductor \(C\) placed
on an insulating stand.
Connect the conductor
with one electrode of the
electroscope, the other
being connected with
the earth. Electrify the
exploring sphere, and,
carrying it by the insulating handle, bring its centre to a given
point. Connect the electrodes for an instant, and then move the
%%-----File: 060.png-----%%
sphere in such a path that the indication of the electroscope remains
zero. This path will lie on an equipotential surface.

For by Theorem V, the part of the potential of the conductor \(C\)
due to the presence of the charged exploring sphere with its centre
at a given point is equal to the potential at the given point due to
a charge on the conductor \(C\) equal to that of the exploring sphere.

By this method the potential of the conductor remains zero, or
very nearly zero, during the whole time of the experiment, so that
there is very little tendency to change of the charge of this body.
The exploring sphere, on the other hand, is at a high potential, but
as it is not connected by a wire with any other body, its insulation
may be made very good.

\Subsection{Lines of Electric Force.}

\article{54} If the direction of the electric force at various points of the
field be determined, and if a line be drawn so that its direction at
every point of its course coincides with the direction of the electric
force at that point, such a line is called a Line of Force. By
drawing a number of such lines, the direction of the force at
different parts of the field may be indicated.
\Runhead{LINES OF ELECTRIC FORCE.}

The lines of force and equipotential surfaces may be drawn, not in
the electric field itself, where the mechanical operation of drawing
them might produce disturbance, but in a model or plan of the
electric field. Drawings of this kind are given in Plates I to VI
at the end of the volume.

Since the electric force is everywhere perpendicular to the equipotential
surfaces, the lines of force cut these surfaces everywhere
at right angles. The lines of force which meet the surface of a
conductor are therefore at right angles to it. When they issue from
the surface the electrification is positive, and when they enter the
surface of the conductor the electrification is negative.

A line of force in every part of its course passes from places of
higher to places of lower potential.

The extremities of the same line of force are called \textit{corresponding}
points.

The beginning of the line is a point on a positively electrified
surface, and the end of the line is a corresponding point on a
negatively electrified surface.

The potential of the first of these surfaces must be higher than
that of the second.
%%-----File: 061.png-----%%

\newchapter
\Chapter{CHAPTER V.}
\Subheading{FARADAY'S LAW OF LINES OF INDUCTION.}

\article{55} \textsc{Faraday} in his electrical researches employs the lines of
force to indicate, not only the direction of the electric force at each
point of the field, but also the quantity of electrification on any
given portion of the electrified surface.
\Runhead{FARADAY'S LAW OF LINES OF INDUCTION.}

If we consider a portion of an electrified surface as cut off from
the rest by the bounding line which surrounds it, and if from every
point of this bounding line we draw a line of force, producing it
till it meets the surface of some other body in a point which is
said to \textit{correspond} to the point of the body from which the line was
drawn, these lines will form a tubular surface, and will cut off a
certain portion from the surface of the other body corresponding to
the portion of the surface of the first body, and the total electrifications
of the two corresponding portions are equal in numerical
magnitude but opposite in kind.

\article{56} A particular instance of Faraday's law is that which we
have already proved by experiment, namely, that the electrification
of the inner surface of a closed conducting vessel is equal and
opposite to that of an electrified body placed within it. Here
we have a relation between the whole electrification of the inner
surface and that of the opposed surface of the interior body.
Faraday's law asserts that, by drawing lines of force from the one
surface to the other, points corresponding to each other in the two
surfaces may be found; that corresponding lines are such that any
point of one has its corresponding point in the other; and that the
electrifications of the two portions of the opposed surfaces bounded
by such corresponding lines are equal and opposite.

\article{57} We have called these lines `lines of force' because we
began by defining them as lines whose \textit{direction} at every point
%%-----File: 062.png-----%%
coincides with that of the electric force. Every line of force
begins at a positively electrified surface and ends at a negatively
electrified surface, and the points of these surfaces at which it
begins and ends are called \textit{corresponding} points.

A system of lines of force forming a tubular surface closed at
the one end by a portion of the positively electrified surface and
at the other by the corresponding portion of the negative surface,
is called by Faraday a \textit{Tube of Induction}, because electric induction,
according to Faraday, is that condition of the dielectric by which
the electrifications of the opposed surfaces are placed in that
physical relation to one another, which we express by saying that
their electrifications are equal and opposite.

\Subsection{Properties of a Tube of Induction.}

\article{58} (1) The electrification of the portion of the positively
electrified surface from which the tube of induction proceeds is
equal in numerical value but opposite in sign to the negative
electrification of the portion of the surface at which the tube of
induction terminates.

By dividing the positive surface into portions, the electrification
of each of which is unity, and drawing tubes corresponding to
each portion, we obtain a system of \textit{unit} tubes, which will be very
convenient in describing electric phenomena. For in this case
the electrification of any surface is measured by the \textit{number} of
tubes which abut on it. If they proceed \textit{from} the surface, this
number is to be taken as representing the \textit{positive} electrification;
if the tubes terminate at the surface, the electrification is negative.

It is in this sense that Faraday so often speaks of the \textit{number} of
lines of force which fall on a given area.

If we suppose an imaginary surface drawn in the electric field,
then the quantity of electrostatic induction through this surface
is measured by the number of tubes of induction which pass
through it, and is reckoned positive or negative accordingly as
the tubes pass through it in the positive or negative direction.

\textit{Note.} By an imaginary surface is meant a surface which has
no physical existence, but which may be imagined to exist in
space without interfering with the physical properties of the substance
which occupies that space. Thus we may imagine a vertical
plane dividing a man's head longitudinally into two equal parts,
and by means of this imaginary surface we may render our ideas
%%-----File: 063.png-----%%
of the form of his head more precise, though any attempt to convert
this imaginary surface into a physical one would be criminal.
Imaginary quantities, such as are mentioned in treatises on
analytical geometry, have no place in physical science.

\article{59} In every part of the course of a line of electrostatic induction
it is passing from places of higher to places of lower
potential, and in a direction at right angles to the equipotential
surfaces which it cuts.
\Runhead{PROPERTIES OF A TUBE OF INDUCTION.}

We have seen that the electric field is divided by the equipotential
surfaces into a series of shells, like the coats of an onion,
the thickness of each shell at any point being inversely as the
electric force at that point.

We have now divided the electric field into a system of unit
tubes of induction, the section of each tube at any point varying
inversely as the intensity of the electric induction at that point.

Each of these tubes is cut by the equipotential surfaces into a
number of segments which we may call unit cells.

\article{60} If we take the simplest case---that of a single positively
electrified body placed within a closed conducting vessel, all the
tubes of induction begin at the positively electrified body and end
at the negatively electrified inner surface of the vessel. The
number of these tubes, since they are unit-tubes, is equal to the
number of electrical units in the charge of the electrified body.
Each of them cuts all the equipotential surfaces which enclose
the electrified body and are enclosed by the vessel. Each tube,
therefore, is divided into a number of cells representing the difference
of the potential of the electrified body from that of the vessel.
If \(e\) is the charge of the body and \(p\) its potential, \(E\) and \(P\) being the
charge and potential of the vessel, the whole number of cells is
\[
e(p - P)\text{,}
\]
or, since \(E = -e\), we may write this expression
\[
ep + EP\text{.}
\]

Now this is double of the expression which we formerly obtained
for the electrical energy of the system (see \hyperref[art:31]{Art.\ 31}). Hence in
this simple case the number of cells is double the number of units
of energy in the system.

If there are several electrified bodies, \(A\), \(B\), \(C\), \&c., the tubes
of induction proceeding from one of them, \(A\), may abut either
on the inner surface of the surrounding vessel or on one of the
other electrified bodies.
%%-----File: 064.png-----%%

Let \(E_1\), \(E_2\), \(E_3\) be the charges of \(A\), \(B\), \(C\) and \(P_1\), \(P_2\), \(P_3\) their
potentials, the charge and potential of the vessel being \(E_0\) and \(P_0\).

Let \(E_{AB}\), \(E_{AC}\), \(E_{AO}\) denote the number of tubes of induction
which pass from \(A\) to the conductors \(B\) and \(C\) and the vessel \(O\),
respectively. Then the whole number of cells will be
\begin{align*}
E_{AB}(P_1 - P_2) + E_{AC}(P_1 - P_3) &+ E_{AO}(P_1 - P_0)\text{,}\\
                  + E_{BC}(P_2 - P_3) &+ E_{BO}(P_2 - P_0)\text{,}\\
                                      &+ E_{CO}(P_3 - P_0)\text{.}
\end{align*}

By arranging the terms according to the potentials involved in
them, and remembering that since \(E_{AB}\) denotes the number of
tubes which pass from \(A\) to \(B\), \(E_{BA}\) must denote the number
which pass from \(B\) to \(A\) and therefore
\[E_{BA} = -E_{AB}\text{,}\]
the expression may be written
\begin{align*}
&P_1(E_{AB} + E_{AC} + E_{AO})\text{,}\\
+ &P_2(E_{BC} + E_{BO} + E_{BA})\text{,}\\
+ &P_3(E_{CO} + E_{CA} + E_{CB})\text{,}\\
+ &P_0(E_{OA} + E_{OB} + E_{OC})\text{.}
\end{align*}

Now \(E_{AB} + E_{AC} + E_{AO}\) is the whole number of tubes issuing from
\(A\) and this therefore is equal to \(E_1\), the charge of \(A\), and the coefficients
of the other potentials are also the charges of the bodies
to which they refer, so that the final expression is
\[P_0E_0 + P_1E_1 + P_2E_2 + P_3E_3\text{,}\]
and this is double the energy of the system.

Hence, whether there is one electrified body or several, the number
of cells is twice the number of units of electrical energy in the
system.

\Runhead{ENERGY OF AN ELECTRIFIED SYSTEM.}
\article{61} This remarkable correspondence between the number of cells
into which the tubes of induction are cut by the equipotential surfaces,
and the electrical energy of the system, leads us to enquire
whether the electrical energy may not have its true seat in the
dielectric medium which is thus cut up into cells, each cell being a
portion of the medium in which half a unit of energy is stored up.
We have only to suppose that the electromotive force, when it acts
on a dielectric, puts it into a certain state of constraint, from which
it is always endeavouring to relieve itself.

To make our conception of what takes place more precise, let us
%%-----File: 065.png-----%%
consider a single cell belonging to a tube of induction proceeding
from a positively electrified body, the cell being bounded by two
consecutive equipotential surfaces surrounding the body.

\wrapfig{0.47}{065.png}{Fig. 17.}
We know that there is an electromotive force acting outwards
from the electrified body. This force, if it acted on a conducting
medium, would produce a current of electricity which would last as
long as the force continued to
act. The medium however is
a non-conducting or dielectric
medium, and the effect of the
electromotive force is to produce
what we may call electric displacement,
that is to say, the
electricity is forced outwards in
the direction of the electromotive
force, but its condition when
so displaced is such that, as soon
as the electromotive force is removed, the electricity resumes the
position which it had before displacement.
\Runhead{ELECTRIC DISPLACEMENT.}

The amount of electric displacement is measured by the quantity
of electricity which crosses an imaginary fixed surface drawn parallel
to the equipotential surfaces.

We know absolutely nothing with respect to the distance through
which any particular portion of electricity is displaced from its
original position. The only thing we know is the quantity which
crosses a given surface. The greater the amount of electricity
which we suppose to exist, say, in a cubic inch, the smaller the
distance through which we must suppose it displaced in order that
a given quantity of electricity may be displaced across a square
inch of area fixed in the medium. It is probable that the actual
motion of displacement is exceedingly small, in which case we must
suppose the quantity of electricity in a cubic inch of the medium to
be exceedingly great. If this is really the case the actual velocity
of electricity in a telegraph wire may be very small, less, say, than
the hundredth of an inch in an hour, though the signals which it
transmits may be propagated with great velocity.

\article{62} The displacement across any section of a unit tube of induction
is one unit of electricity and the direction of the displacement
is that of the electromotive force, namely, from places of
higher to places of lower potential.

Besides the electric displacement within the cell we have to
%%-----File: 066.png-----%%
consider the state of the two ends of the cell which are formed by
the equipotential surfaces. We must suppose that in every cell the
end formed by the surface of higher potential is coated with one
unit of positive electricity, the opposite end, that formed by the
surface of lower potential, being coated with one unit of negative
electricity. In the interior of the medium where the positive end
of one cell is in contact with the negative end of the next, these
two electrifications exactly neutralise each other, but where the
dielectric medium is bounded by a conductor, the electrification is
no longer neutralised, but constitutes the observed electrification at
the surface of the conductor.

According to this view of electrification, we must regard electrification
as a property of the dielectric medium rather than of the
conductor which is bounded by it.

\article{63} If we further admit that in every part of a dielectric
medium through which electric induction is taking place there is
a tension, like that of a rope, in the direction of the lines of force,
and a pressure in all directions at right angles to the lines of force,
we may account for all the mechanical actions which take place
between electrified bodies.
\Runhead{ELECTRIC TENSION.}

The tension, referred to unit of surface, is proportional to the
square of the electromotive force at the point. The pressure has
the same numerical value, but is, of course, opposite in sign.

In my larger treatise on electricity a proof is given of the fact
that a system of stress such as is here described is consistent with
the equilibrium of a fluid dielectric medium, and that this state of
stress in the medium is mechanically equivalent to the attraction
or repulsion which electrified bodies manifest.

I have not, however, attempted, by any hypothesis as to the internal
constitution of the dielectric medium, to explain in what way the
electric displacement causes or is associated with this state of stress.

We have thus, by means of the tubes of induction and the
equipotential surfaces, constructed a geometrical model of the field
of electric force. Diagrams of particular cases are given in the
figures at the end of this book.

The direction and magnitude of the electric force at any point
may be indicated either by means of the equipotential surfaces or
by means of the tubes of induction. Hence, when it is expressed
in both ways, we may by the study of the relation between the
equipotential surfaces and the tubes of induction deduce important
theorems in the theory of electricity.
%%-----File: 067.png-----%%

\Subsection{On the use of Physical Analogies.}

\article{64} In many cases the relations of the phenomena in two
different physical questions have a certain similarity which enables
us, when we have solved one of these questions, to make use of our
solution in answering the other. The similarity which constitutes
the analogy is not between the phenomena themselves, but between
the relations of these phenomena.
\Runhead{ANALOGIES BETWEEN ELECTROSTATICS AND HEAT.}

To begin with a case of extreme simplicity;---a person slow at
arithmetic having to find the price of 52 yards of cotton at 7 pence
a yard, if he happened to remember that there are 52 weeks and a
day in a year of 365 days, might at once give the answer, 364
pence, without performing the calculation. Here there is no resemblance
whatever between the quantities themselves---the weeks
and the yards of cotton,---the sole resemblance is between the arithmetical
relations of these quantities to others in the same question.

The analogy between electrostatic phenomena and those of the
uniform conduction of heat in solid bodies was first pointed out by
Sir W. Thomson in a paper `On the Uniform Motion of Heat in
Homogeneous Solid Bodies, and its connection with the Mathematical
Theory of Electricity,' published in the \textit{Cambridge Mathematical
Journal}, Feb.\ 1842; reprinted in the \textit{Phil.\ Mag}.\ 1854, and in the
reprint of Thomson's papers on \textit{Electrostatics and Magnetism}. The
analogy is of the following nature:---

\begin{table}[H]
\centering
\footnotesize
\begin{tabular}{>{\hangindent=1em}p{0.45\textwidth}|>{\hangindent=1em}p{0.45\textwidth}}
\multicolumn{1}{c|}{\textit{Electrostatics.}} &
\multicolumn{1}{c}{\textit{Heat.}}\\
The electric field. & An unequally heated body.\\
A dielectric medium. & A body which conducts heat.\\
The electric potential at different points of the field. &
The temperature at different points in the body.\\
The electromotive force which tends to
move positively electrified bodies from
places of higher to places of lower potential. &
The flow of heat by conduction from
places of higher to places of lower temperature.\\
A conducting body. & A perfect conductor of heat.\\
The positively electrified surface of a conductor. &
A surface through which heat flows into the body.\\
The negatively electrified surface of a conductor.&
A surface through which heat escapes from the body.\\
A positively electrified body. & A source of heat.\\
A negatively electrified body.&
A sink of heat, that is, a place at which
heat disappears from the body.\\
An equipotential surface.&An isothermal surface.\\
A line or tube of induction.&A line or tube of flow of heat.
\end{tabular}
\end{table}

By a judicious use of this analogy and other analogies of the
same nature the progress of physical science has been greatly assisted.
%%-----File: 068.png-----%%
In order to avoid the dangers of crude hypotheses we must
study the true nature of analogies of this kind. We must not conclude
from the partial similarity of some of the relations of the
phenomena of heat and electricity that there is any real physical
similarity between the causes of these phenomena. The similarity
is a similarity between relations, not a similarity between the things
related.

This similarity is so complete as far as it goes that any result we
may have obtained either about electricity or about the conduction
of heat may be at once translated out of the language of the one
science into that of the other without fear of error; and in pursuing
our investigations in either subject we are at liberty to make use
of the ideas belonging to the other, if by so doing we are enabled
to see more clearly the connection between one step and another of
the reasoning.

We must bear in mind that at the time when Sir W. Thomson
pointed out the analogy between electrostatic and thermal phenomena
men of science were as firmly convinced that electric attraction
was a direct action between distant bodies as that the
conduction of heat was the continuous flow of a material fluid
through a solid body. The dissimilarity, therefore, between the
things themselves appeared far greater to the men of that time than
to the readers of this book, who, unless they have been previously
instructed, have not yet learned either that heat is a fluid or that
electricity acts at a distance.

\article{65} But we must now consider the limits of the analogy---the
points beyond which we must not push it.
\Runhead{LIMITATION TO THE USE OF ANALOGIES.}

In the first place, it is only a particular class of cases of the
conduction of heat that have analogous cases in electrostatics. In
general, when heat is flowing through a body it causes the temperature
of some parts of the body to rise and that of others to fall,
and the flow of heat, which depends on the relation of these temperatures,
is therefore variable. If the supply of heat is maintained
uniform, the temperatures of the different parts of the body tend to
adjust themselves to a state in which they remain constant. The
quantity of heat which enters any given portion of the body is then
exactly equal to that which leaves it during the same time. Under
these circumstances the flow of heat is said to be steady.

Now the analogy with electric phenomena applies to the steady
flow of heat only. The more general case, that of variable flow of
heat, has nothing in electrostatics analogous to it. Even the restricted
%%-----File: 069.png-----%%
case of steady flow of heat differs in a most important
element from the electrostatic analogue. The steady flow of heat
must be kept up by the continual supply of heat at a constant rate
and the continual withdrawal of heat at an equal rate. This involves
a continual expenditure of energy to maintain the flow of
heat in a constant state, so that though the state of the body
remains constant and independent of time, the element of time
enters into the calculation of the amount of heat required.

The element of time does not enter into the corresponding case
in electrostatics. So far as we know, a set of electrified bodies
placed in a perfectly insulating medium might remain electrified
for ever without a supply of anything from external sources.
There is nothing in this case to which we can apply the term
`flow,' which we apply to the case of the transmission of heat
with the same propriety that we apply it to the case of a current
of electricity, of water, or of time itself.

\article{66} Another limitation to the analogy is that the temperature
of a body cannot be altered without altering its physical state.
The density, conductivity, electric properties, \&c.\ all vary when the
temperature rises.

The electrical potential, however, which is the analogue of temperature
is a mere scientific concept. We have no reason to
regard it as denoting a physical state. If a number of bodies
are placed within a hollow metallic vessel which completely surrounds
them, we may charge the outer surface of the vessel and
discharge it as we please without producing any physical effect
whatever on the bodies within. But we know that the electric
potential of the enclosed bodies rises and falls with that of the
vessel. This may be proved by passing a conductor connected
to the earth through a hole in the vessel. The relation of the
enclosed bodies to this conductor will be altered by charging and
discharging the vessel. But if the conductor be removed, the
simultaneous rise and fall of the potentials of the bodies in the
vessel is not attended with any physical effect whatever.

\article{67} Faraday\footnote{
\textit{Exp.\ Res}.\ 1173.}
proved this by constructing a hollow cube, twelve
feet in the side, covered with good conducting materials, insulated
from the ground and highly electrified by a powerful machine.
`I went into this cube,' he says, `and lived in it, but though I
used lighted candles, electrometers, and all other tests of electrical
states, I could not find the least influence upon them, or indication
%%-----File: 070.png-----%%
of anything particular given by them, though all the time the
outside of the cube was powerfully charged and large sparks and
brushes were starting off from every part of its outer surface.'

It appears, therefore, that the most sudden changes of potential
produce no physical effects on matter, live or dead, provided these
changes take place simultaneously on all the bodies in the field.

If Faraday, instead of raising his cube to a high electric potential,
had raised it to a high temperature, the result, as we know, would
have been very different.
\Runhead{FARADAY'S CUBE.}

\article{68} It appears, therefore, that the analogy between the conduction
of heat and electrostatic phenomena has its limits, beyond
which we must not attempt to push it. At the time when it was
pointed out by Thomson, men of science were already acquainted
with the great work of Fourier on the conduction of heat in solid
bodies, and their minds were more familiar with the ideas there
developed than with those belonging to current electricity, or to
the theory of the displacements of a medium.

It is true that Ohm had, in 1827, applied the results obtained
by Fourier for heat to the theory of the distribution of electric
currents in conductors, but it was long before the practical value
of Ohm's work was understood, and till men became familiar with
the idea of electric currents in solid conductors, any illustration of
electrostatic phenomena drawn from such currents would have served
rather to obscure than to enlighten their minds.

\article{69} When an electric current flows through a solid conductor,
the direction of the current at any point is from places of higher
to places of lower potential, and its intensity is proportional to the
rate at which the potential decreases from point to point of a line
drawn in the direction of the current.
\Runhead{CURRENT.}

We may suppose equipotential surfaces drawn in the conducting
medium. The lines of flow of the current are everywhere at right
angles to the equipotential surfaces, and the rate of flow is proportional
to the number of equipotential surfaces which would be
cut by a line of unit length drawn in the direction of the current.

It appears, therefore, that this case of a conducting medium
through which an electric current is passing has certain points
of analogy with that of a dielectric medium bounded by electrified
bodies.

In both the medium is divided into layers by a series of equipotential
surfaces. In both there is a system of lines which are
everywhere perpendicular to these surfaces. In the one case these
%%-----File: 071.png-----%%
lines are called current lines or lines of flow; in the other they
are called lines of electric force or electric induction.

An assemblage of such lines drawn from every point of a given
line is called a surface of flow. Since the lines of which this
surface is formed are everywhere in the direction of the electric
current, no part of the current passes through the surface of flow.
Such a surface therefore may be regarded as impervious to the
current without in any way altering the state of things.

If the line from which the assemblage of lines of flow is drawn
is one which returns into itself, which we shall call a \textit{closed} curve,
or, more briefly, a \textit{ring}, the surface of flow will have the form of a
tube and is called a tube of flow. Any two sections of the same
tube of flow correspond to each other in the sense defined in \hyperref[art:54]{Art.\ 54},
and the quantities of electricity which in the same time flow across
these two sections are equal.

Here then we have the analogue of Faraday's law, that the
quantities of electricity upon corresponding areas of opposed conducting
surfaces are equal and opposite.

Faraday made great use of this analogy between electrostatic
phenomena and those of the electric current, or, as he expressed
it, between induction in dielectrics and conduction in conductors,
and he proved that, in many cases, induction and conduction are
associated phenomena. \textit{Exp.\ Res}.\ 1320, 1326.

We must remember, however, that the electric current cannot
be maintained constant through a conductor which resists its
passage except by a continual expenditure of energy, whereas
induction in a perfectly insulating dielectric between oppositely
electrified conductors may be maintained in it for an indefinitely
long time without any expenditure of energy, except that which
is required to produce the original electrification. The element of
time enters into the question of conduction in a way in which it
does not appear in that of induction.

\article{70} But we may arrive at a more perfect mental representation
of induction by comparing it, not with the instantaneous state of
a current, but with the small displacements of a medium of invariable
density.

Returning to the case of an electric current through a solid
conductor, let us suppose that the current, after flowing for a
very short time, ceases. If we consider a surface drawn within
the solid, then if this surface intersects the tubes of flow, a certain
quantity of electricity will have passed from one side of the surface
%%-----File: 072.png-----%%
to the other during the time when the current was flowing. This
passage of electricity through the surface is called \textit{electric displacement},
and the displacement through a given surface is the
quantity of electricity which passes through it. In the case of
a continuous current the displacement increases continuously as
long as the current is kept up, but if the current lasts for a finite
time, the displacement reaches its final value and then remains
constant. The lines, surfaces, and tubes of flow of the transient
current are also lines, surfaces, and tubes of displacement. The
displacements across any two sections of the same tube of displacement
are equal. At the beginning of each unit tube of
displacement there is a unit of positive electricity, and at the end
of the tube there is a unit of negative electricity.

At every point of the medium there is a state of stress consisting
of a tension in the direction of the line of displacement
through the point and a pressure in all directions at right angles
to this line. The numerical value of the tension is equal to that
of the pressure, namely, the square of the intensity of the electric
force divided by \(4 \pi\).

\article{71} By the consideration of the properties of the tubes of
induction and the equipotential surfaces we may easily prove
several important general theorems in the theory of electricity,
the demonstration of which by the older methods is long and
difficult. The properties of a tube of induction have already
been stated, but for the sake of what follows we may state them
again:---
\Runhead{TUBES OF INDUCTION AND LINES OF FORCE.}

(1) If a tube of induction is cut by an imaginary surface, the
quantity of electricity displaced across a section of the tube is the
same at whatever part of the tube the section be made.

(2) In every part of the course of a line of electrostatic force
it cuts the equipotential surfaces at right angles, and is proceeding
from a place of higher to a place of lower potential.

\textit{Note.} This statement is true only when the distribution of
electric force can be completely represented by means of a set of
equipotential surfaces. This is always the case when the electricity
is in equilibrium, but when there are electric currents, though in
some parts of the field it may be possible to draw a set of equipotential
surfaces, there are other parts of the field where the
distribution of electric force cannot be represented by means of
such surfaces. For an electric current is always of the nature of
a circuit which returns into itself, and such a circuit cannot in
%%-----File: 073.png-----%%
every part of its course be proceeding from places of higher to
places of lower potential.

\article{72} It may be observed that in (1) we have used the words
`tube of induction,' and in (2) the words `line of electrostatic
force.' In a fluid dielectric, such as air, the line of electrostatic
force is always in the same direction as the tube of induction, and
it may seem pedantic to maintain a distinction between them.
There are other cases, however, in which it is very important to
remember that a tube of induction is defined with respect to the
phenomenon which we have called electric displacement, while a
line of force is defined with respect to the electric force. In fluids
the electric displacement is always in the direction of the electric
force, but there are solid bodies in which this is not the case\footnote{
See the experiments of Boltzmann on crystals of sulphur. Vienna Sitzungsb.
9 Jan.\ 1873.}, and
in which, therefore, the tubes of induction do not coincide in direction
with the lines of force.

\article{73} It follows from (1) that every tube of induction begins
at a place where there is a certain quantity of positive electricity
and ends at a place where there is an equal quantity of negative
electricity, and that, conversely, from any place where there is positive
electricity a tube may be drawn, and that wherever there is
negative electricity a tube must terminate.
\Runhead{ELECTRIFICATION AT ENDS OF INDUCTION TUBE.}

\article{74} It follows from (2) that the potential at the beginning of a
tube is higher than at the end of it. Hence, no tube can return
into itself, for in that case the same point would have two different
potentials, which is impossible.

\article{75} From this we may prove that if the potential at every
point of a closed surface is the same, and if there is no electrified
body within that surface, the potential at any point within the
region enclosed within the closed surface is the same as that at
the surface.

For if there were any difference of potential between one point
and another within this region, there would be lines of force from
the places of higher towards the places of lower potential. These
lines, as we have seen, cannot return into themselves. Hence they
must have their extremities either within the region or without it.
Neither extremity of a line of force can be within the region, for
there must be positive electrification at the beginning and negative
electrification at the end of a line of force, but by our hypothesis
there is no electrification within the region. On the other hand, a
%%-----File: 074.png-----%%
line of force within the region cannot have its extremities without
the region, for in that case it must enter the region at one point
of the surface and leave it at another, and therefore by (2) the
potential must be higher at the point of entry than at the point
of issue, which is contrary to our hypothesis that the potential is
the same at every point of the surface.

Hence no line of force can exist within the region, and therefore
the potential at any point within the region is the same as that at
the surface itself.

\article{76} It follows from this theorem, that if the closed surface is
the internal surface of a hollow conducting vessel, and if no electrified
body is within the surface, there is no electrification on the
internal surface. For if there were, lines of force would proceed
from the electrified parts of the surface into the region within,
and we have already proved that there are no lines of force in
that region.
\Runhead{NO ELECTRIFICATION WITHIN A HOLLOW CONDUCTOR.}

We have already proved this by experiment (\hyperref[art:20]{Art.\ 20}), but we
now see that it is a necessary consequence of the properties of
lines of force.

\Subsection{Superposition of electric systems.}

\article{77}  We have already (\hyperref[art:29]{Art.\ 29}) given some examples of the
superposition of electric effects, but we must now state the principle
of superposition more definitely.

\textit{If the same system is electrified in three different ways, then if the
potential at any point in the third case is the sum of the potentials in
the first and second cases, the electrification of any part of the system
in the third case will be the sum of the electrifications of the same part
in the first and second cases.}

By reversing the sign of the electrifications and potentials in
one of these cases, we may enunciate the principle with respect
to the case in which the potential and the electrification are at
every point the excess of what they are in the first case over what
they are in the second.

\article{78} We may now establish a theorem which is of the greatest
importance in the theory of electricity.

If the electric field under consideration consist of a finite portion
of a dielectric medium, and if at every point of the boundary of
this region the potential is given, and if the distribution of electrification
within the region be also given, then the potential at any
%%-----File: 075.png-----%%
point within the region can have one and only one value consistent
with these conditions.

One value at least of the potential must be possible, because the
conditions of the theorem are physically possible. Again, if at
any point two values of the potential were possible, then by
taking the excess of the first value over the second for every
point of the system, a third case might be formed in which the
potential is everywhere the excess of the first case above the second.
At the boundary of the region the potential in the third case is
everywhere zero. Within the region the electrification is everywhere
zero. Hence, by (\hyperref[art:75]{Art.\ 75}), at every point within the region
the potential in the third case is zero.

There is, therefore, no difference between the distribution of
potential in the first case and in the second, or, in other words,
the potential at any point within the region can have only one
value.

If in any case we can find a distribution of potential which
satisfies the given conditions, then by this theorem we are assured
that this distribution is the only possible solution of the problem.
Hence the importance of this theorem in the theory of electricity.
\Runhead{THOMSON'S THEOREM.}

\article{79} For instance, let \(A\) be an electrified body and let \(B\) be
one of the
\wrapfig{0.24}{075.png}{Fig. 18.}
equipotential surfaces surrounding the body. Let the
potential of the surface \(B\) be equal to \(P\). Now
let a conducting body be constructed and placed
so that its external surface coincides with the
closed surface \(B\), and let it be so electrified that
its potential is \(P\). Then the conditions of the
region outside \(B\) are the same as when it was
acted on by the body \(A\) only. For the potential
over the whole bounding surface of the region is \(P\), the same as
before, and whatever electrified bodies exist outside of \(B\) remain
unchanged. Hence the potential at every point outside of \(B\) \textit{may},
consistently with the conditions, be the same as before. By our
theorem, therefore, the potential at every point outside \(B\) \textit{must} be
the same, when, instead of the body \(A\), we have a conducting
surface \(B\), raised to the potential \(P\).

\article{80} The charge of every part of the surface of a conductor is
of the same sign as its potential, unless there is another body
in the field whose potential is of the same sign but numerically
greater.

Let us suppose the potential of the body to be positive; then,
%%-----File: 076.png-----%%
if on any part of its surface there is negative electricity, lines of
force must terminate on this part of the surface, and these lines
of force must begin at some electrified surface whose potential is
higher than that of the body. Hence, if there is no other body
whose potential is higher than that of the given body, no part
of the surface of the given body can be charged with negative
electricity.

If an uninsulated conductor is placed in the same field with a
charged conductor, the charge on every part of the surface of
the uninsulated conductor is of the opposite sign to the charge of
the charged conductor.

For since the potential of the uninsulated body is zero, there
can be no line of force between it and the walls of the room, or
infinite space where the potential is always zero. The line of force
which has one end at any point of the surface of this body must
therefore have its other end at some point of the charged body,
and since the two extremities of a line of force are oppositely
electrified, the electrification of the surface of the uninsulated body
must be everywhere opposite to the charge of the charged body.
\Runhead{INDUCED ELECTRIFICATION.}

The charged body in this experiment is called the Inductor, and
the other body the induced body.

When the induced body is uninsulated, the electricity spread
over every part of its surface is, as we have just proved, of the
opposite sign to that of the inductor.

The total charge, \(E_A\), of the induced body, which we may call \(A\),
may be found by multiplying \(P_B\), the potential of the inductor \(B\),
by \(Q_{BA}\), the mutual coefficient of induction between the bodies,
which is always a negative quantity.

This electrification induced on an uninsulated body is called
by some writers on electricity the Induced Electrification of the
First Species. Since the potential of \(A\) is already zero, it is
manifest that if any part of its surface is touched by a fine wire
communicating with the ground there will be no discharge.

Next, let us suppose that the body, \(A\), instead of being uninsulated
is insulated, but originally without charge. Under the
action of the inductor \(B\) part of its surface, on the side next to \(B\),
will become electrified oppositely to \(B\); but since the algebraic
sum of its electrification is zero, some other part of its surface must
be electrified similarly to \(B\).

This electrification, of the same name as that of \(B\), is called by
writers on electricity the Induced Electrification of the Second
%%-----File: 077.png-----%%
Species. If a wire connected with the ground be now made to
touch any part of the surface of \(A\), electricity of the same name
as that of \(B\) will be discharged, its amount being equal and opposite
to the negative charge (of the first species) which remains
on the body \(A\), which is now reduced to potential zero.

In order to obtain a clearer idea of the distribution of electricity
on the surface of \(A\) under various conditions, let us begin by
supposing the potential of \(A\) to be zero and that of \(B\) to be unity.
Let the surface-density at a given point \(P\) on the surface of \(A\)
be \(-\sigma_1\), and let the whole charge of \(A\) be \(-q_{AB}\). The negative
sign is prefixed to the symbols of these quantities because the
quantities themselves are always negative.
\Runhead{COEFFICIENTS OF CAPACITY.}

The charge of \(B\) in this case is \(q_B\).

Let us next suppose the potential of \(A\) to be unity and that of \(B\)
to be zero, and let the surface-density at the point \(P\) be now \(\sigma_2\), and
the whole charge on \(A\), \(q_A\).

These quantities are both essentially positive, and \(q_A\) is called
the \textit{capacity} of \(A\). The value of both is increased on account of
the presence of \(B\) in the field.

Let us now suppose that the potentials of \(A\) and \(B\) are \(P_A\) and
\(P_B\) respectively; then the surface-density at the point \(P\) is
\begin{align*}
\sigma &= P_A  \sigma _2 - P_B \sigma _1\text{,}\\
\shortintertext{and the charge of \(A\) is}
E_A &= P_A q_A - P_B q_{AB}\text{,}\\
\shortintertext{and that of \(B\) is}
E_B &= P_Bq_B - P_A q_{AB}\text{.\quad [See \hyperref[art:39]{Art.\ 39}.]}
\end{align*}

If \(A\) is insulated and without charge \(E_A = 0\), which gives
\begin{align*}
P_A &= P_B \frac{q_{AB}}{q_A}\text{,}\\
\shortintertext{and the surface-density at \(P\) is}
\sigma &= \frac{P_B}{q_A}(q_{AB}  \sigma _2 - q_A \sigma _1)\text{.}
\end{align*}

On a region of the surface of \(A\) next to \(B\), \(\sigma\) will be of the
opposite sign from \(P_B\); and on a region on the other side from \(B\),
\(\sigma\) will be of the same sign with \(P_B\). The boundary between these
two regions forms what is called the neutral line, the form and
position of which depend on the form and position of \(A\) and \(B\).
%%-----File: 078.png-----%%

\newchapter
\Chapter{CHAPTER VI.}

\Subheading{PARTICULAR CASES OF ELECTRIFICATION.}

\article{81} \textsc{A spherical} conductor is electrified and insulated within
the concentric spherical internal surface of a conducting vessel.

On account of the perfect symmetry of this system in all directions,
it is manifest that the distribution of electricity will be
uniform over each of the opposed spherical surfaces, that the lines
of force will be in the directions passing through the common
centre of the spheres, and that the equipotential surfaces will be
spheres having this point for their centre.

If \(e\) is the quantity of electricity on the inner sphere and \(E\) that on
the internal surface of the outer sphere, then by Experiment VIII
\[
E = -e\text{.}\tag{1}
\]

If \(r\) and \(R\) are the radii of the spheres, \(s\) and \(S\) their surfaces,
and  \(\sigma\) and  \(\Sigma\) the surface-densities of the electricity on these
surfaces, then by geometry,
\[
s = 4 \pi r^2\text{,}\qquad S = 4 \pi R^2\text{,}\tag{2}
\]
where \(\pi\) denotes the ratio of the circumference of a circle to its
diameter.

The whole charge on either sphere is found by multiplying the
surface into the surface-density, or
\begin{align}
e = s \sigma\text{,} \quad E &= S \Sigma\text{.}\tag{3}\\
\shortintertext{Hence,}
\sigma = \frac{e}{4 \pi r^2}\text{,} \quad \Sigma &= \frac{E}{4 \pi R^2}\text{,}\tag{4}\\
\shortintertext{and by (1),}
\Sigma &= \frac{-e}{4 \pi R^2}\text{.}\tag{5}
\end{align}

It appears, therefore, that when the charge, \(e\), of the inner
sphere is given, the surface-density, \(\Sigma\), on the internal surface of
the vessel is inversely as the square of the distance of that surface
from the centre of the electrified sphere.

Hence by Coulomb's law (Experiment XIII, \hyperref[art:47]{Art.\ 47}) the electromotive
force at the outer spherical surface is inversely as the
square of the distance from the centre of the sphere.
%%-----File: 079.png-----%%

This is the law according to which the electric force varies
at different distances from a sphere uniformly electrified. The
amount of the force is independent of the radius of the inner
electrified sphere, and depends only on the whole charge upon it.
If we suppose the radius of the inner sphere to become very small
till at last the sphere cannot be distinguished from a point, we
may imagine the whole charge concentrated at this point, and
we may then express our result by saying that the electric
action of a uniformly electrified sphere at any point outside the
sphere is the same as that of the whole charge of the sphere would
be if concentrated at the centre of the sphere.

We must bear in mind, however, that it is physically impossible
to charge the small sphere with more than a certain quantity of
electricity on each unit of area of its surface. If the surface-density
exceed this limit, electricity will fly off in the form of the
brush discharge. Hence the idea of an electrified point is a mere
mathematical fiction which can never be realised in nature. The
imaginary charge concentrated at the centre of the sphere, which
produces an effect outside the sphere equivalent to that of the
actual distribution of electricity on the surface, is called the
\textit{Electrical Image} of that distribution. See \hyperref[art:100]{Art.\ 100}.

\Subsection{Measurement of Electricity.}
\Runhead{ELECTROSTATIC UNIT OF ELECTRICITY}

\article{82} We have already described methods of comparing the
quantity of electrification on different bodies, but in each case we
have only compared one quantity of electricity with another,
without determining the absolute value of either. To determine
the absolute value of an electric charge we must compare it with
some definite quantity of electricity, which we assume as a unit.

The unit of electricity adopted in electrostatics is that quantity
of positive or vitreous electricity which, if concentrated in a point,
and placed at the unit of distance from an equal charge, also
concentrated in a point, would repel it with the unit of mechanical
force. The dielectric medium between the two charged points is
supposed to be air.

\article{83} Let us now suppose two bodies, whose dimensions are small
compared with the distance between them, to be charged with
electricity. Let the charge of the first body be \(e\) units of electricity
and that of the second \(e'\) units, and let the distance between
the bodies be \(r\).
%%-----File: 080.png-----%%

Then, since the force varies inversely as the square of the
distance, the force with which each unit of electricity in the
first body repels each unit of electricity in the second body will
be \(\xp\dfrac{1}{r^2}\), and since the number of pairs of units, one in each body,
is \(ee'\), the whole repulsion between the bodies will be
\[f = \frac{ee'}{r^2}\]

If the charge of the first or the second body is negative we
must consider \(e\) or \(e'\) negative. If the one charge is positive and
the other negative, \(f\) will be negative, or the force between the
bodies will be an attraction instead of a repulsion. If the charges
are both positive or both negative, the force between the bodies
will be a repulsion.

\article{84} \textit{Definition}.---The electric or electromotive force at a point
is the force which would be experienced by a small body charged
with the unit of positive electricity and placed at that point, the
electrification of the system being supposed to remain undisturbed
by the presence of this unit of electricity.
\Runhead{ELECTROMOTIVE FORCE AT A POINT.}

We shall use the German letter \(\mathfrak{E}\) as the symbol of electric
force.

\article{85} Let us now return to the case of a sphere whose radius
is \(r\), the external surface of which is uniformly electrified, the
surface-density of the electrification being \(\sigma\). As we have already
proved, the whole charge of the sphere is
\[
e = 4 \pi r^2 \sigma\text{.}
\]
At any point outside the sphere such that the distance from
the centre of the sphere is \(r'\) the electromotive force, \(\mathfrak{E}\), is directed
\textit{from} the centre, and its value is
\[
\mathfrak{E} = \frac{e}{r'^2}\text{.}
\]
If the point is close to the surface of the sphere, \(r' = r\), and
\[
\mathfrak{E} = \frac{e}{r^2} = 4 \pi \sigma\text{,}
\]
or the electric force close to the surface of an electrified sphere is
at right angles to the surface and is equal to the surface-density
multiplied by \(4 \pi\).

We have already seen that in all cases the electric force close
to the surface of a conductor is at right angles to that surface, and
is proportional to the surface-density. We now, by means of this
%%-----File: 081.png-----%%
particular case, find that the constant ratio of the electric force
to the surface-density is \(4 \pi\) for a uniformly electrified sphere, and
therefore this is the ratio for a conductor of any form.

The equation
\[
\mathfrak{E} = 4 \pi \sigma
\]
is the complete expression of the law discovered by Coulomb and
referred to in Arts.\ 47 and 81.

\Subsection{Value of the Potential.}

\article{86} We must next consider the values of the potential at
different distances from a small electrified body.
\Runhead{VALUE OF THE POTENTIAL.}

\textit{Definition}. The electric potential at any point is the work which
must be expended in order to bring a body charged with unit
of electricity from an infinite distance to that point.

If \(\psi\) is the potential at \(A\) and \(\psi'\) that at \(B\), then the work
which must be spent by the external agency in overcoming
electrical force while carrying a unit of electricity from \(A\) to \(B\)
is \(\psi' - \psi\).

The quantity \(\psi' - \psi\) would also represent the work which would
be done \textit{by the electrical forces} in assisting the transfer of the unit
of electricity from \(B\) to \(A\) if the motion were reversed.

If the force from \(B\) to \(A\) were constant and equal to \(\mathfrak{E}\), then
\[
\psi' - \psi = \overline{BA} \ldot \mathfrak{E}\text{.}
\]
In general, the electric force varies as the body moves from \(B\) to \(A\),
so that we cannot at once apply this method of finding the difference
of potentials. But, by breaking up the path \(BA\) into a
sufficient number of parts, we may make these parts so small that
the electric force may be regarded as uniform during the passage
of the body along any one of these parts. We may then ascertain
the parts of the work done in each part of the path, and by adding
them together, obtain the whole work done during the passage
from \(B\) to \(A\).

\widefig{0.7}{081.png}{Fig. 19.}
Let us suppose a unit of electricity placed at \(O\), and let the
distances of the points \(A\), \(B\), \(C\), \ldots\ \(Z\) from \(O\) be \(a\), \(b\), \(c\), \ldots\ \(z\). The
electric force at \(A\) is \(\xp\dfrac{1}{a^2}\), at \(B\) \(\xp\dfrac{1}{b^2}\), and so on, all in the direction
from \(O\) to \(A\).
\Runhead{POTENTIAL AT A POINT.}
%%-----File: 082.png-----%%

To find the work which must be done in order to bring a
unit of electricity from \(A\) to \(B\) we must multiply the distance \(AB\)
by the average of the electromotive force at the various points
between \(A\) and \(B\). The least value of the force is \(\xp\dfrac{1}{a^2}\) and the
greatest value is \(\xp\dfrac{1}{b^2}\)· Hence the work required is greater than \(\xp\dfrac{AB}{a^2}\)
and less than \(\xp\dfrac{AB}{b^2}\). Now \(AB\) is \(a - b\), and the true value of the
work is the excess of the potential at \(B\) over that at \(A\). Hence
if we now write \(A\), \(B\), \(C\), \ldots\ \(Z\) for the potentials at the corresponding
points, we may express the work required to bring the unit
of electricity from \(A\) to \(B\) by \(B - A\). This quantity therefore is
greater than
\[
\frac{a - b}{a^2} \text{ or } \left(\frac{1}{b} - \frac{1}{a}\right) \frac{b}{a}\text{,}
\]
but less than
\[
\frac{a - b}{b^2} \text{ or } \left(\frac{1}{b} - \frac{1}{a}\right) \frac{a}{b}\text{.}
\]
We may express this by the double inequality
\[
\left(\frac{1}{b} - \frac{1}{a}\right)\frac{a}{b} < B - A < \left(\frac{1}{b} - \frac{1}{a}\right) \frac{a}{b}\text{.}
\]
Similarly
\[
\left(\frac{1}{c} - \frac{1}{b}\right) \frac{c}{b} < C - B < \left(\frac{1}{c} - \frac{1}{b}\right) \frac{b}{c}\text{,}
\]
and so on. The ratios \(\xp\dfrac{a}{b}\), \(\xp\dfrac{b}{c}\), \&c., are all greater than unity. Let
us suppose that the greatest of these ratios is equal to \(p\). The
ratios \(\xp\dfrac{b}{a}\), \&c., are the reciprocals of these; they are therefore all
less than unity, but none less than \(\xp\dfrac{1}{p}\). Hence
\begin{align*}
\left(\frac{1}{b} - \frac{1}{a}\right) \frac{1}{p} < &B - A < \left(\frac{1}{b} - \frac{1}{a}\right) p\\
\left(\frac{1}{c} - \frac{1}{b}\right) \frac{1}{p} < &C - B < \left(\frac{1}{c} - \frac{1}{b}\right) p\\
&\ldots \ldots \ldots\\
\left(\frac{1}{z} - \frac{1}{y}\right) \frac{1}{p} < &Z - Y < \left(\frac{1}{z} - \frac{1}{y}\right) p\text{.}
\end{align*}
Adding these inequalities we find
\[
\left(\frac{1}{z} - \frac{1}{a}\right) \frac{1}{p} < Z - A < \left(\frac{1}{z} - \frac{1}{y}\right) p\text{.}
\]
%%-----File: 083.png-----%%
By increasing the number of points between \(A\) and \(Z\) and making
the intervals between them smaller we may make the greatest
ratio, \(p\), as near to unity as we please, and we may therefore
assert that, as the line \(AZ\) is more and more minutely divided,
the quantity \(p\) and its reciprocal \(\xp\dfrac{1}{p}\) approach unity as their common
limit. In the limit, therefore,
\[
Z - A = \frac{1}{z} - \frac{1}{a}\text{.}
\]

We have thus found the difference between the potentials at
\(A\) and \(Z\). To determine the actual value of the potential, say at \(Z\),
we must refer to the definition of the potential, that it is the
work expended in bringing unit of electricity from an infinite
distance to the given point. We have therefore in the above
expression to suppose the point \(A\) removed to an infinite distance
from \(O\), in which case the potential \(A\) is zero, and the reciprocal of
the distance, or \(\xp\dfrac{1}{a}\), is also zero. The equation is therefore reduced to
the form
\[
Z = \frac{1}{z}\text{,}
\]
or in words, the numerical value of the potential at a given point
due to unit of electricity at a given distance is the reciprocal of the
number expressing that distance.

If the charge is \(e\), then the potential at a distance \(z\) is \(\xp\dfrac{e}{z}\).

The potential due to a number of charges placed at different
distances from the given point is found by adding the potentials
due to each separate charge, regard being had to the sign of each
potential.

\article{87} Since, as we have seen, the electric force at any point
outside a uniformly electrified spherical surface is the same as if the
electric charge of the surface had been concentrated at its centre,
the potential due to the electrified surface must be, for points
outside it,
\[
\psi = \frac{e}{r}\text{,}
\]
where \(e\) is the whole charge of the surface, and \(r\) is the distance of
the given point from the centre.

Let \(a\) be the radius of the spherical surface, then this expression
for the potential is true as long as \(r\) is greater than \(a\). At the
%%-----File: 084.png-----%%
surface, \(r\) is equal to \(a\). The potential at the surface due to its
own electrification is therefore
\[
\psi_a = \frac{e}{a}
\]
[since there can be no discontinuity in the value of the potential
between the surface and a point just outside it].

Within the surface there is no electromotive force, and the
potential is therefore the same as at the surface for all points
within the sphere.

If the potential of the spherical surface is unity, then
\[
e = a\text{,}
\]
or the charge is numerically equal to the radius.

Now the electric capacity of a body in a given field is measured
by the charge which raises its potential to unity. Hence the
electric capacity of a conducting sphere placed in air at a considerable
distance from any other conductor is numerically equal
to the radius of the sphere.

If by means of an electrometer we can measure the potential of
the sphere, we can ascertain its charge by multiplying this potential
by the radius of the sphere. This method of measuring a quantity
of electricity was employed by Weber and Kohlrausch in their
determination of the ratio of the unit employed in electromagnetic
to that employed in electrostatic researches. Since there is no
electric force within a uniformly electrified sphere the potential
within the sphere is constant and equal to \(\xp\dfrac{e}{a}\).

\article{88} We are now able to complete the theory of the electrification
of two concentric spherical surfaces.
\Runhead{CAPACITY OF TWO CONCENTRIC SPHERES.}

Let a spherical conductor of radius \(a\) be insulated within a
hollow conducting vessel, the internal surface of which is a sphere
of radius \(b\) concentric with the inner sphere. Let the charge
on the inner sphere be \(e\), then, as we have already seen, the
charge on the interior surface of the vessel will be \(-e\). At any
point outside both spherical surfaces and distant \(r\) from the
centre the electric potential due to the inner sphere will be
\(\xp\dfrac{e}{r}\), and that due to the outer sphere will be \(\xp\dfrac{-\,e}{r}\). Since these
two quantities are numerically equal, but of opposite sign, they
destroy each other, and the potential at every point for which \(r\)
is greater than \(b\) is zero.
%%-----File: 085.png-----%%

Between the two spherical surfaces, at a point distant \(r\) from the
centre, the potential due to the inner sphere is \(\xp\dfrac{e}{r}\), and that due to
the outer sphere is \(\xp\dfrac{-\,e}{b}\). Hence the whole potential in this intermediate
space is \(e\xp\left(\dfrac{1}{r} - \dfrac{1}{b} \right)\).

At the surface of the inner sphere \(r = a\), so that the potential of
the inner sphere is \(e\xp\left(\dfrac{1}{a} - \dfrac{1}{b} \right)\).

The potential at all points within the inner sphere is uniform and
equal to \(e\xp\left(\dfrac{1}{a} - \dfrac{1}{b} \right)\).

The capacity of the inner sphere is numerically equal to the value
of \(e\) when the potential is made equal to unity. In this case
\[
e = \frac{1}{\dfrac{1}{a} - \dfrac{1}{b}} = \frac{ab}{b - a}\text{,}
\]
or, the capacity of a sphere insulated within a concentric spherical
surface is a fourth proportional to the distances \((b - a)\) between the
surfaces and radii (\(a\), \(b\)) of the surfaces.

By diminishing the interval, \(b - a\), between the surfaces, the
capacity of the system may be made very great without making
use of very large spheres.

This example may serve to illustrate the principle of the Leyden
jar, which consists of two metallic surfaces separated by insulating
material. The smaller the distance between the surfaces and the
greater the area of the surfaces, the greater the capacity of the jar.
\Runhead{LEYDEN JAR.}

Hence, if an electrical machine which can charge a body up to a
given potential is employed to charge a Leyden jar, one surface of
which is connected with the earth, it will, if worked long enough,
communicate a much greater charge to the jar than it would to a
very large insulated body placed at a great distance from any other
conductor.

The capacity of the jar, however, depends on the nature of the
dielectric which is between the two metallic surfaces as well as on
its thickness and area. See \hyperref[art:131]{Art.\ 131} et sqq.
%%-----File: 086.png-----%%

\Section{Two Parallel Planes.}

\article{89} Another simple case of electrification is that in which the
electrodes are two parallel plane surfaces at a distance \(c\). We shall
suppose the dimensions of these surfaces to be very great compared
with the distance between them, and we shall consider the electrical
action only in that part of the space between the planes
whose distance from the edges of the plates is many times greater
than \(c\).

\wrapfig{0.57}{086.png}{Fig. 20.}
Let \(A\) be the potential of the upper plane in the figure, and \(B\)
that of the lower plane. Then
the electric force at any point
\(P\) between the planes, and not
near the edge of either plane,
is \(\xp\dfrac{A-B}{c}\), acting from \(A\) to \(B\). The electric density on the upper
plane is found by Coulomb's Law by dividing this quantity by \(4 \pi\) .
If \(\sigma\) be the surface density
\[
\sigma = \frac{A - B}{4 \pi c}\text{.}\tag{1}
\]
The surface density on the plane \(B\) is equal to this in magnitude
but opposite in sign.
\Runhead{FORCE BETWEEN TWO PARALLEL PLANES.}

Let us now consider the quantity of electricity on an area \(S\),
which we may suppose cut out from the upper plane by an
imaginary closed curve. Multiplying \(S\) into \(\sigma\), we find
\[
e = \frac{A - B}{4 \pi c} S\text{.}\tag{2}
\]
The quantity of electricity on an equal area of the plane \(B\) taken
exactly opposite to \(S\) will be \(-e\). The energy of the electrification
of these two portions of electricity is, by \hyperref[art:31]{Art.\ 31},
\[
Q = \tfrac{1}{2}\{Ae + B(-e)\} = \tfrac{1}{2} (A - B)e\text{.}\tag{3}
\]

Expressing this in terms of \(e\) it becomes
\[
Q = \frac{2 \pi}{S} e^2c\text{.}\tag{4}
\]

If \(c\), the distance between the surfaces, be made to increase to \(c'\)
the charges of the surfaces remaining the same, the energy will
become
\[
Q' = \frac{2 \pi}{S} e^2c'\text{.}\tag{5}
\]
%%-----File: 087.png-----%%
The augmentation of the potential energy is
\[
Q' - Q = \frac{2 \pi}{S} e^2(c' - c)\text{,}\tag{6}
\]
and this is the work done by external agency in pulling the planes
asunder against the electric attraction.

If \(F\) is the electric attraction between the two areas \(S\),
\[
F(c' - c) = \frac{2 \pi}{S} e^2(c' - c)\text{,}\tag{7}
\]
or
\[
F = \frac{2 \pi}{S} e^2\text{.}\tag{8}
\]

\article{90} This result gives us the best experimental method of measuring
the quantity of electricity on the area \(S\), for by this equation
\[
e = \sqrt\frac{FS}{2 \pi}\text{.}\tag{9}
\]

In this expression \(F\) is the force of attraction on the area \(S\) determined
in dynamical measure from observation of its effects. \(S\) is
the area of the surface and \(\pi\) is the ratio of the circumference of a
circle to its diameter.

The difference between the potentials, \(A\) and \(B\), of the two planes
is easily found in terms of \(e\) by means of equation (2), thus,
\[
A - B = 4 \pi c \frac{e}{S} = c \sqrt{\frac{8 \pi F}{S}}\text{.}\tag{10}
\]

\article{91} In Sir William Thomson's attracted disk electrometers a
disk is so arranged that when in its proper position the surface of
the disk forms part of a much larger plane surface extending for a
considerable distance on all sides of the disk. The part of the surface
outside the moveable disk is called the Guard Ring and the
surface of the disk and guard ring together may be considered as
the surface of a large disk, part of which, near its centre, is
moveable. Opposite this disk is placed another disk having its
surface parallel to the first disk and much larger than the moveable
disk. The electrification of the moveable disk is then the
same as that of a small portion of one of the large opposed planes
taken at a considerable distance from the edge of the plane, and
only very small corrections are needed to make the formulæ already
given apply to the case of the moveable disk.
\Runhead{ATTRACTED DISK ELECTROMETERS.}

The distribution of electrification and of electric force near the
edges of the large disks is by no means so simple. It is calculated
%%-----File: 088.png-----%%
in Art.\ 202 of my larger Treatise, and the lines of force and
equipotential surfaces are shown in Plate V at the end of this
book.

\article{92} The direct problem of electrostatics---the problem which
the circumstances of every electrostatic experiment present to us---may
be stated as follows.

A system of insulated conductors is given in form and position,
and the electric charge of each conductor is given, required the
distribution of electricity on each conductor and the electric potential
at any point of the field.

The mathematical difficulties of the solution of this problem have
been overcome hitherto only in a small number of cases, and it is
only by a study of what we may call the inverse problem that the
results we possess have been obtained.

In the inverse problem, a possible distribution of potential
being given, it is required to find the forms, positions, and charges
of a system of conductors which shall be consistent with this distribution
of potential.
\Runhead{INVERSE PROBLEM OF ELECTROSTATICS.}

Any number of solutions of this latter problem may be obtained
by taking, instead of the electrified bodies of the original distribution,
any set of equipotential surfaces surrounding them, and supposing
these surfaces to be the surfaces of conductors, the charge of each
conductor being equal to the sum of the charges of all the bodies
of the original distribution which it encloses.

Every electrical problem of which we know the solution has been
constructed by an inverse process of this kind. It is therefore of
great importance to the electrician that he should know what results
have been obtained in this way, since the only method by which he
can expect to solve a new problem is by reducing it to one of the
cases in which a similar problem has been constructed by the
inverse process.

This historical knowledge of results can be turned to account in
two ways. If we are required to devise an instrument for making
electrical measurements with the greatest accuracy, we may select
those forms for the electrified surfaces which correspond to cases of
which we know the accurate solution. If, on the other hand, we
are required to estimate what will be the electrification of bodies
whose forms are given, we may begin with some case in which
one of the equipotential surfaces takes a form somewhat resembling
the given form, and then by a tentative method we may
modify the problem till it more nearly corresponds to the given
%%-----File: 089.png-----%%
case. This method is evidently very imperfect, considered from a
mathematical point of view, but it is the only one we have, and if
we are not allowed to choose our conditions, we can make only an
approximate calculation of the electrification. It appears, therefore,
that what we want is a knowledge of the forms of equipotential
surfaces and lines of induction in as many different cases as we can
collect together and remember. In certain classes of cases, such
as those relating to spheres, we may proceed by mathematical
methods. In other cases we cannot afford to despise the humbler
method of actually drawing tentative figures on paper, and selecting
that which appears least unlike the figure we require.

This latter method, I think, may be of some use, even in cases
in which the exact solution has been obtained, for I find that an
eye knowledge of the forms of the equipotential surfaces often leads
to a right selection of a mathematical method of solution.

I have therefore drawn several diagrams of systems of equipotential
surfaces and lines of force, so that the student may make
himself familiar with the forms of the lines.
\Runhead{DIAGRAMS OF EQUIPOTENTIAL SURFACES.}

\article{93} In the \hyperref[plate:1]{first plate} at the end of this volume we have the
equipotential surfaces surrounding two points electrified with quantities
of electricity of the same kind and in the ratio of 20 to 5.

Here each point is surrounded by a system of equipotential
surfaces which become more nearly spheres as they become smaller,
but none of them are accurately spheres. If two of these surfaces,
one surrounding each sphere, be taken to represent the surfaces
of two conducting bodies, nearly but not quite spherical, and if
these bodies be charged with the same kind of electricity, the
charges being as 4 to 1, then the diagram will represent the
equipotential surfaces, provided we expunge all those which are
drawn inside the two bodies. It appears from the diagram that
the action between the bodies will be the same as that between
two points having the same charges, these points being not exactly
in the middle of the axis of each body, but somewhat more remote
than the middle point from the other body.

The same diagram enables us to see what will be the distribution
of electricity on one of the oval figures, larger at one end
than the other, which surround both centres. Such a body, if electrified
with a charge 25 and free from external influence, will
have the surface-density greatest at the small end, less at the large
end, and least in a circle somewhat nearer the smaller than the
larger end.
%%-----File: 090.png-----%%

There is one equipotential surface, indicated by a dotted line,
which consists of two lobes meeting at the conical point \(P\). That
point is a point of equilibrium, and the surface-density on a body
of the form of this surface would be zero at this point.

The lines of force in this case form two distinct systems, divided
from one another by a surface of the sixth degree, indicated by a
dotted line, passing through the point of equilibrium, and somewhat
resembling one sheet of the hyperboloid of two sheets.

This diagram may also be taken to represent the lines of force
and equipotential surfaces belonging to two spheres of gravitating
matter whose masses are as 4 to 1.

\article{94} In the \hyperref[plate:2]{second Plate} we have again two points whose charges
are as 4 to 1, but the one positive and the other negative. In this
case one of the equipotential surfaces, that, namely, corresponding
to potential zero, is a sphere. It is marked in the diagram by the
dotted circle \(Q\). The importance of this spherical surface will be
seen when we come to the theory of Electrical Images.

We may see from this diagram that if two round bodies are
charged with opposite kinds of electricity they will attract each
other as much as two points having the same charges but placed
somewhat nearer together than the middle points of the round
bodies.

Here, again, one of the equipotential surfaces, indicated by a
dotted line, has two lobes, an inner one surrounding the point
whose charge is 5 and an outer one surrounding both bodies, the
two lobes meeting in a conical point \(P\) which is a point of equilibrium.

If the surface of a conductor is of the form of the outer lobe, a
roundish body having, like an apple, a conical dimple at one end of
its axis, then, if this conductor be electrified, we shall be able to
determine the superficial density at any point. That at the bottom
of the dimple will be zero.

Surrounding this surface we have others having a rounded
dimple which flattens and finally disappears in the equipotential
surface passing through the point marked \(M\).

The lines of force in this diagram form two systems divided by a
surface which passes through the point of equilibrium.

If we consider points on the axis on the further side of the point
\(B\), we find that the resultant force diminishes to the double point \(P\),
where it vanishes. It then changes sign, and reaches a maximum
at \(M\), after which it continually diminishes.
%%-----File: 091.png-----%%

This maximum, however, is only a maximum relatively to other
points on the axis, for if we draw a surface perpendicular to the
axis, \(M\) is a point of minimum force relatively to neighbouring
points on that surface.
\Runhead{EQUIPOTENTIAL SURFACES AND LINES OF INDUCTION.}

\article{95} \hyperref[plate:3]{Plate III} represents the equipotential surfaces and lines
of force due to an electrified point whose charge is 10 placed at
\(A\), and surrounded by a field of force, which, before the introduction
of the electrified point, was uniform in direction and
magnitude at every part. In this case, those lines of force which
belong to \(A\) are contained within a surface of revolution which
has an asymptotic cylinder, having its axis parallel to the undisturbed
lines of force.

The equipotential surfaces have each of them an asymptotic
plane. One of them, indicated by a dotted line, has a conical
point and a lobe surrounding the point \(A\). Those below this surface
have one sheet with a depression near the axis. Those above have
a closed portion surrounding \(A\) and a separate sheet with a slight
depression near the axis.

If we take one of the surfaces below \(A\) as the surface of a conductor,
and another a long way below \(A\) as the surface of another
conductor at a different potential, the system of lines and surfaces
between the two conductors will indicate the distribution of electric
force. If the lower conductor is very far from \(A\) its surface will
be very nearly plane, so that we have here the solution of the
distribution of electricity on two surfaces, both of them nearly
plane and parallel to each other, except that the upper one has
a protuberance near its middle point, which is more or less prominent
according to the particular equipotential line we choose for
the surface.

\article{96} \hyperref[plate:4]{Plate IV} represents the equipotential surfaces and lines
of force due to three electrified points \(A\), \(B\) and \(C\), the charge of \(A\)
being 15 units of positive electricity, that of \(B\) 12 units of negative
electricity, and that of \(C\) 20 units of positive electricity. These
points are placed in one straight line, so that
\[
AB=9\text{,} \quad BC=16\text{,} \quad AC=25\text{.}
\]

In this case, the surface for which the potential is unity consists
of two spheres whose centres are \(A\) and \(C\) and their radii 15 and 20.
These spheres intersect in the circle which cuts the plane of the
paper in \(D\) and \(D'\), so that \(B\) is the centre of this circle and its
radius is 12. This circle is an example of a line of equilibrium, for
the resultant force vanishes at every point of this line.
%%-----File: 092.png-----%%

If we suppose the sphere whose centre is \(A\) to be a conductor
with a charge of 3 units of positive electricity, and placed under
the influence of 20 units of positive electricity at \(C\), the state of
the case will be represented by the diagram if we leave out all the
lines within the sphere \(A\). The part of this spherical surface within
the small circle \(DD'\) will be negatively electrified by the influence
of \(C\). All the rest of the sphere will be positively electrified, and
the small circle \(DD'\) itself will be a line of no electrification.

We may also consider the diagram to represent the electrification
of the sphere whose centre is \(C\), charged with 8 units of positive
electricity, and influenced by 15 units of positive electricity placed
at \(A\).

The diagram may also be taken to represent the case of a
conductor whose surface consists of the larger segments of the
two spheres meeting in \(DD'\), charged with 23 units of positive
electricity.

\article{97} I am anxious that these diagrams should be studied as
illustrations of the language of Faraday in speaking of `lines of
force,' the `forces of an electrified body,' \&c.

In strict mathematical language the word Force is used to signify
the supposed cause of the tendency which a material body is found
to have towards alteration in its state of rest or motion. It is
indifferent whether we speak of this observed tendency or of its
immediate cause, since the cause is simply inferred from the effect,
and has no other evidence to support it.

Since, however, we are ourselves in the practice of directing the
motion of our own bodies, and of moving other things in this way,
we have acquired a copious store of remembered sensations relating
to these actions, and therefore our ideas of force are connected in
our minds with ideas of conscious power, of exertion, and of fatigue,
and of overcoming or yielding to pressure. These ideas, which give
a colouring and vividness to the purely abstract idea of force, do
not in mathematically trained minds lead to any practical error.

But in the vulgar language of the time when dynamical science
was unknown, all the words relating to exertion, such as force,
energy, power, \&c., were confounded with each other, though some
of the schoolmen endeavoured to introduce a greater precision into
their language.

The cultivation and popularization of correct dynamical ideas
since the time of Galileo and Newton have effected an immense
change in the language and ideas of common life, but it is only
%%-----File: 093.png-----%%
within recent times, and in consequence of the increasing importance
of machinery, that the ideas of force, energy and power
have become accurately distinguished from each other. Very few,
however, even of scientific men, are careful to observe these distinctions;
hence we often hear of the force of a cannon-ball when
either its energy or its momentum is meant, and of the force of an
electrified body when the quantity of its electrification is meant.

Now the quantity of electricity in a body is measured, according
to Faraday's ideas, by the \textit{number} of lines of force, or rather of
induction, which proceed from it. These lines of force must all
terminate somewhere, either on bodies in the neighbourhood, or on
the walls and roof of the room, or on the earth, or on the heavenly
bodies, and wherever they terminate there is a quantity of electricity
exactly equal and opposite to that on the part of the body
from which they proceeded. By examining the diagrams this will
be seen to be the case. There is therefore no contradiction between
Faraday's views and the mathematical result of the old theory,
but, on the contrary, the idea of lines of force throws great light
on these results, and seems to afford the means of rising by a continuous
process from the somewhat rigid conceptions of the old
theory to notions which may be capable of greater expansion, so
as to provide room for the increase of our knowledge by further
researches.

\article{98} These diagrams are constructed in the following manner:---
\Runhead{CONSTRUCTION OF DIAGRAMS.}

First, take the case of a single centre of force, a small electrified
body with a charge \(E\). The potential at a distance \(r\) is \(V = \xp\dfrac{E}{r}\);
hence, if we make \(r = \xp\dfrac{E}{V}\), we shall find \(r\), the radius of the sphere
for which the potential is \(V\). If we now give to \(V\) the values
1, 2, 3, \&c., and draw the corresponding spheres, we shall obtain
a series of equipotential surfaces, the potentials corresponding to
which are measured by the natural numbers. The sections of these
spheres by a plane passing through their common centre will be
circles, which we may mark with the number denoting the potential
of each. These are indicated by the dotted circles on the right
hand of Fig. 21.

If there be another centre of force, we may in the same way draw
the equipotential surfaces belonging to it, and if we now wish to
find the form of the equipotential surfaces due to both centres
together, we must remember that if \(V_1\) be the potential due to one
%%-----File: 094.png-----%%
centre, and \(V_2\) that due to the other, the potential due to both will be
\(V_1 + V_2 = V\), Hence, since at every intersection of the equipotential
surfaces belonging to the two series we know both \(V_1\) and \(V_2\), we
also know the value of \(V\). If therefore we draw a surface which
passes through all those intersections for which the value of \(V\) is
the same, this surface will coincide with a true equipotential surface
at all these intersections, and if the original systems of surfaces
be drawn sufficiently close, the new surface may be drawn with
any required degree of accuracy. The equipotential surfaces due to
two points whose charges are equal and opposite are represented by
the continuous lines on the right hand side of Fig. 21.

This method may be applied to the drawing of any system of
equipotential surfaces when the potential is the sum of two potentials,
for which we have already drawn the equipotential surfaces.

The lines of force due to a single centre of force are straight
lines radiating from that centre. If we wish to indicate by these
lines the intensity as well as the direction of the force at any point,
we must draw them so that they mark out on the equipotential
surfaces portions over which the surface-integral of induction has
definite values. The best way of doing this is to suppose our
plane figure to be the section of a figure in space formed by the
revolution of the plane figure about an axis passing through the
centre of force. Any straight line radiating from the centre and
making an angle \(\theta\) with the axis will then trace out a cone,
and the surface-integral of the induction through that part of any
surface which is cut off by this cone on the side next the positive
direction of the axis, is \(2 \pi E(1 - cos \theta)\).

If we further suppose this surface to be bounded by its intersection
with two planes passing through the axis, and inclined at
the angle whose arc is equal to half the radius, then the induction
through the surface so bounded is
\begin{gather*}
E(1 - cos \theta) = 2 \Psi \text{, say;}
\shortintertext{and}
\theta = cos^{-1}\left(1 - 2 \frac{\Psi}{E}\right)\text{.}
\end{gather*}
\Runhead{EQUIPOTENTIAL SURFACES AND LINES OF INDUCTION.}

If we now give to \(\Psi\) a series of values 1, 2, 3 \ldots\ \(E\), we shall find
a corresponding series of values of \(\theta\), and if \(E\) be an integer, the
number of corresponding lines of force, including the axis, will be
equal to \(E\).

We have therefore a method of drawing lines of force so that
the charge of any centre is indicated by the number of lines which
converge to it, and the induction through any surface cut off in the
%%-----File: 095.png-----%%
%%-----File: 096.png-----%%
%%-----File: 097.png-----%%
way described is measured by the number of lines of force which
pass through it. The dotted straight lines on the left hand side
of Fig. 21 represent the lines of force due to each of two electrified
points whose charges are \(10\) and \(-10\) respectively.

% this figure should float so can't use thispagestyle{empty}
% use \fancyhead[...]{\iffloatpage{}} to empty it
\begin{figure}[htp!]
\centering
\caption*{Fig. 21.}
\includegraphics[width=.95\textwidth]{095.png}
\caption*{\textit{Method of drawing
Lines of Force and Equipotential Surfaces.}}
\end{figure}

If there are two centres of force on the axis of the figure we
may draw the lines of force for each axis corresponding to values
of \(\Psi_1\) and \(\Psi_2\), and then, by drawing lines through the consecutive
intersections of these lines, for which the value of \(\Psi_1 + \Psi_2\) is the
same, we may find the lines of force due to both centres, and in
the same way we may combine any two systems of lines of force
which are symmetrically situated about the same axis. The continuous
curves on the left hand side of Fig. 21 represent the lines
of force due to the electrified points acting at once.

After the equipotential surfaces and lines of force have been
constructed by this method, the accuracy of the drawing may be
tested by observing whether the two systems of lines are everywhere
orthogonal, and whether the distance between consecutive
equipotential surfaces is to the distance between consecutive lines
of force as half the distance from the axis is to the assumed unit of
length.

In the case of any such system of finite dimensions the line of
force whose index number is \(\Psi\) has an asymptote which passes
through the centre of gravity of the system, and is inclined to the
axis at an angle whose cosine is \(1 - 2 \xp\dfrac{\Psi}{E}\), where \(E\) is the total
electrification of the system, provided \(\Psi\) is less than \(E\). Lines of
force whose index is greater than \(E\) are finite lines.

The lines of force corresponding to a field of uniform force parallel
to the axis are lines parallel to the axis, the distances from the
axis being the square roots of an arithmetical series.
%%-----File: 098.png-----%%

\newchapter
\Chapter{CHAPTER VII.}
\Subheading{THEORY OF ELECTRICAL IMAGES.}
\Runhead{ELECTRICAL IMAGES.}

\article{99} \textsc{The} calculation of the distribution of electrification on the
surface of a conductor when electrified bodies are placed near it is in
general an operation beyond the powers of existing mathematical
methods.

When the conductor is a sphere, and when the distribution of
electricity on external bodies is given, a solution, depending on
an infinite series was obtained by Poisson. This solution agrees
with that which was afterwards given in a far simpler form by
Sir W. Thomson, and which is the foundation of his method of
Electric Images.

By this method he has solved problems in electricity which
have never been attempted by any other method, and which, even
after the solution has been pointed out, no other method seems
capable of attacking. This method has the great advantage of
being intelligible by the aid of the most elementary mathematical
reasoning, especially when it is considered in connection with the
diagrams of equipotential surfaces described in Arts.\ 93-96.

\article{100} The idea of an image is most easily acquired by considering
the optical phenomena on account of which the term image was
first introduced into science.

We are accustomed to make use of the visual impressions we
receive through our eyes in order to ascertain the positions of
distant objects. We are doing this all day long in a manner
sufficiently accurate for ordinary purposes. Surveyors and astronomers
by means of artificial instruments and mathematical deductions
do the same thing with greater exactness. In whatever
way, however, we make our deductions, we find that they are
consistent with the hypothesis that an object exists in a certain
position in space, from which it emits light which travels to our
eyes or to our instruments in straight lines.
%%-----File: 099.png-----%%

But if we stand in front of a plane mirror and make observations
on the apparent direction of the objects reflected therein, we find
that these observations are consistent with the hypothesis that
there is no mirror, but that certain objects exist in the region
beyond the plane of the mirror. These hypothetical objects are
geometrically related to certain real objects in front of the plane of
the mirror, and they are called the \textit{images} of these objects.

We are not provided with a special sense for enabling us to
ascertain the presence and the position of distant bodies by means
of their electrical effects, but we have instrumental methods by
which the distribution of potential and of electric force in any part
of the field may be ascertained, and from these data we obtain a
certain amount of evidence as to the position and electrification of
the distant body.

If an astronomer, for instance, could ascertain the direction and
magnitude of the force of gravitation at any desired point in the
heavenly spaces, he could deduce the positions and masses of the
bodies to which the force is due. When Adams and Leverrier
discovered the hitherto unknown planet Neptune, they did so by
ascertaining the direction and magnitude of the gravitating force
due to the unseen planet at certain points of space. In the electrical
problem we employed an electrified pith ball, which we
moved about in the field at pleasure. The astronomers employed
for a similar purpose the planet Uranus, over which, indeed, they
had no control, but which moved of itself into such positions that
the alterations of the elements of its orbit served to indicate the
position of the unknown disturbing planet.

\artlabel{101}
\wrapfig{0.45}{100.png}{Fig. 22.}
101.] In one of the electrified systems which we have already
investigated, that of a spherical conductor \(A\) within a concentric
spherical conducting vessel \(B\), we have one of the simplest cases of
the principle of electric images.

The electric field is in this case the region which lies between
the two concentric spherical surfaces. The electric force at any
point \(P\) within this region is in the direction of the radius \(OP\)
and numerically equal to the charge of the inner sphere, \(A\), divided
by the square of the distance, \(OP\), of the point from the common
centre. It is evident, therefore, that the force within this region
will be the same if we substitute for the electrified spherical surfaces,
\(A\) and \(B\), any other two concentric spherical surfaces, \(C\) and
\(D\), one of them, \(C\), lying within the smaller sphere, \(A\), and the
other, \(D\), lying outside of \(B\), the charge of \(C\) being equal to that
%%-----File: 100.png-----%%
of \(A\) in the former case. The electric phenomena in the region
between \(A\) and \(B\) are therefore the same as before, the only difference
between the cases is that in the region between \(A\) and \(C\) and
also in the region between \(B\) and \(D\) we now find electric forces
acting according to the same law
as in the region between \(A\) and
\(B\), whereas when the region was
bounded by the conducting surfaces
\(A\) and \(B\) there was no electrical
force whatever in the regions
beyond these surfaces. We may
even, for mathematical purposes,
suppose the inner sphere \(C\) to be
reduced to a physical point at \(O\),
and the outer sphere \(D\) to expand
to an infinite size, and thus we
assimilate the electric action in
the region between \(A\) and \(B\) to that due to an electrified point at
\(O\) placed in an infinite region.

It appears, therefore, that when a spherical surface is uniformly
electrified, the electric phenomena in the region outside the sphere
are exactly the same as if the spherical surface had been removed,
and a very small body placed at the centre of the sphere, having
the same electric charge as the sphere.
\Runhead{CONCENTRIC SPHERES.}

This is a simple instance in which the phenomena in a certain
region are consistent with a false hypothesis as to what exists
beyond that region. The action of a uniformly electrified spherical
surface in the region outside that surface is such that the phenomena
may be attributed to an imaginary electrified point at the centre of
the sphere.

The potential, \(\psi\), of a sphere of radius \(a\), placed in infinite space
and charged with a quantity \(e\) of electricity, is \(\xp\dfrac{e}{a}\). Hence if \(\psi\) is
the potential of the sphere, the imaginary charge at its centre
is \(\psi a\).

\article{102} Now let us calculate the potential at a point \(P\) (Fig. 23.)
in a spherical surface whose centre is \(C\) and radius \(\overline{CP}\), due to two
electrified points \(A\) and \(B\) in the same radius produced, and such
that the product of their distances from the centre is equal to the
square of the radius. Points thus related to one another are called
\textit{inverse} points with respect to the sphere.
%%-----File: 101.png-----%%

Let \(a = \overline{CP}\) be the radius of the sphere. Let \(\overline{CA} = ma\), then \(\overline{CB}\)
will be \(\xp\dfrac{a}{m}\).

Also the triangle \(APC\) is similar to \(PCB\), and
\begin{gather*}
\overline{AP} : \overline{PB} : : \overline{AC} : \overline{PC}\text{,}\\
\shortintertext{or}
\overline{AP} = m \overline{BP}\text{.\quad See Euclid vi.\ prop.\ E.}
\end{gather*}
Now let a charge of electricity equal to \(e\) be placed at \(A\) and a
charge \(e'= -\xp\dfrac{e}{m}\) of the opposite kind be placed at \(B\). The potential
due to these charges at \(P\) will be
\begin{align*}
V &= \frac{e}{\overline{AP}} + \frac{e'}{\overline{BP}},\\
&= \frac{e}{m \overline{BP}} - \frac{e}{m \overline{BP}},\\
&= 0;
\end{align*}
or the potential due to the charges at \(A\) and \(B\) at any point \(P\) of the
spherical surface is zero.

\wrapfig{0.45}{101.png}{Fig. 23.}
We may now suppose the spherical
surface to be a thin shell of metal.
Its potential is already zero at every
point, so that if we connect it by a
fine wire with the earth there will
be no alteration of its potential,
and therefore the potential at every
point, whether within or without
the surface, will remain unaltered, and will be that due to the two
electrified points \(A\) and \(B\).

If we now keep the metallic shell in connection with the earth
and remove the electrified point \(B\), the potential at every point
within the sphere will become zero, but outside it will remain as
before. For the surface of the sphere still remains of the same
potential, and no change has been made in the distribution of
electrified bodies in the region outside the sphere.

Hence, if an electrified point \(A\) be placed outside a spherical conductor
which is at potential zero, the electrical action at all points
outside the sphere will be equivalent to that due to the point \(A\)
together with another point, \(B\), within the sphere, which is the
inverse point to \(A\), and whose charge is to that of \(A\) as \(-1\) is to \(m\).
The point \(B\) with its imaginary charge is called the \textit{electric image} of \(A\).
\Runhead{IMAGE OF A POINT.}

In the same way by removing \(A\) and retaining \(B\), we may shew
%%-----File: 102.png-----%%
that if an electrified point \(B\) be placed inside a hollow conductor
having its inner surface spherical, the electrical action within the
hollow is equivalent to that of the point \(B\), together with an
imaginary point, \(A\), outside the sphere, whose charge is to that
of \(B\) as \(m\) is to \(-1\).

If the sphere, instead of being in connection with the earth, and
therefore at potential zero, is at potential \(\psi\) the electrical effects
outside the sphere will be the same as if, in addition to the image
of the electrified point, another imaginary charge equal to \(\psi a\) were
placed at the centre of the sphere.

Within the sphere the potential will simply be increased by \(\psi\).
\Runhead{ELECTRICAL IMAGES.}

\article{103} As an example of the method of electric images let us
calculate the electric state of two spheres whose radii are \(a\) and \(b\)
respectively, and whose potentials are \(P_a\) and \(P_b\), the distance between
their centres being \(c\). We shall suppose \(b\) to be small compared
with \(c\).

\widefig{.95}{102.png}{Fig. 24.}
We may consider the actual electrical effects at any point outside
the two spheres as due to a series of electric images.

In the first place, since the potential of the sphere \(A\) is \(P_a\) we
must place an image at the centre \(A\) with a charge \(aP_a\).
\Runhead{TWO SPHERES.}

Similarly at \(B\), the centre of the other sphere, we must place a
charge \(bP_b\).

Each of these images will have an image of the second order in
the other sphere. The image of \(B\) in the sphere \(a\) will be at \(D\),
where
\[
AD = \frac{a^2}{c}, \text{ and the charge } D = -\frac{a}{c} \cdot bP_b\text{.}
\]

The image of \(A\) in the sphere \(b\) will be at \(E\), where
\[
BE = \frac{b^2}{c}, \text{ and the charge } E = -\frac{b}{c} \cdot aP_a\text{.}
\]
%%-----File: 103.png-----%%

Each of these will have an image of the third order. That of \(E\)
in \(a\) will be at \(F\), where
\[AF = \frac{a^2}{AE} = \frac{a^2 c}{c^2 - b^2}, \text{ and } F = \frac{a^2 b}{c^2 - b^2} P_a\text{.}\]

That of \(D\) in \(b\) will be at \(G\), where
\[
BG = \frac{b^2}{DB} = \frac{b^2 c}{c^2 - a^2}, \text{ and } G = \frac{ab^2}{c^2 - a^2} P_b\text{.}\]

The images of the fourth order will be,
\begin{gather*}
\begin{multlined}
\text{of \(G\) in \(a\) at \(H\), where}\\
AH = \frac{a^2}{AG} = \frac{a^2 (c^2 - a^2)}{c (c^2 - a^2 - b^2)} \text{ and } H = - \frac{a^2 b^2}{c (c^2 - a^2 - b^2)} P_b\text{,}
\end{multlined}\\
\begin{multlined}
\text{of \(F\) in \(B\) at \(I\), where}\\
BI = \frac{b^2}{FB} = \frac{b^2 (c^2 - b^2)}{c (c^2 - a^2 - b^2)} \text{ and } I = - \frac{a^2 b^2}{c (c^2 - a^2 - b^2)} P_a\text{.}
\end{multlined}
\end{gather*}

We might go on with a series of images for ever, but if \(b\) is small
compared with \(c\), the images will rapidly become smaller and may
be neglected after the fourth order.

If we now write
\begin{align*}
q_{aa} &= a + \frac{a^2 b}{c^2 - b^2} + \text{\&c.,}\\
q_{ab} &= - \frac{ab}{c} - \frac{a^2 b^2}{c (c^2 - a^2 - b^2)} - \text{\&c.,}\\
q_{bb} &= b + \frac{ab^2}{c^2 -a^2} + \text{\&c.,}
\end{align*}
the whole charge of the sphere \(a\) will be
\[E_a = q_{aa} P_a + q_{ab} P_b\text{,}\]
and that of the sphere \(b\) will be
\[E_b = q_{ab} P_a + q_{bb} P_b\text{.}\]

\article{104} From these results we may calculate the potentials of the
two spheres when their charges are given, and if we neglect
terms involving \(b^3\) we find
\[
   \begin{aligned}
     &P_a = \frac{1}{a} E_a + \frac{1}{c} E_b\text{,}\\
     &P_b = \frac{1}{c} E_a + \left\{\frac{1}{b} - \frac{a^3}{c^2 (c^2 - a^2)}\right\} E_b\text{.}
   \end{aligned}
\]

The electric energy of the system is
\[\frac{1}{2} (E_a P_a + E_b P_b) = \frac{1}{2} \frac{1}{a}\, E_a^2 + \frac{1}{c}\, E_a E_b + \frac{1}{2} \left\{\frac{1}{b} - \frac{a^3}{c^2 (c^2 - a^2 )}\right\} E_b^2\text{.}\]
%%-----File: 104.png-----%%

The repulsion, \(R\), between the two spheres is measured by the
rate at which the energy diminishes as \(c\) increases; therefore,
\[R =\frac{ E_b}{c^2}\left\{E_a - E_b\frac{a^3 (2c^2 - a^2)}{c(c^2 - a^2)^2}\right\}\text{.}\]

In order that the force may be repulsive it is necessary that the
charges of the spheres should be of the same sign, and
\[E_a \text{  must be greater than  } E_b\frac{ a^3(2c^2 - a^2)}{c(c^2 - a^2)^2}.\]

Hence the force is always attractive,
\begin{enumerate}[leftmargin=4em, itemindent=0em, nosep]
  \item When either sphere is uninsulated;
  \item When either sphere has no charge;
  \item When the spheres are very nearly in contact, if their potentials
are different.
\end{enumerate}

When the potentials of the two spheres are equal the force is
always repulsive.

\article{105} To determine the electric force at any point just outside of
the surface of a conducting sphere connected with the earth arising
from the presence of an electrified point \(A\) outside the sphere.

The electrical conditions at all points outside the sphere are equivalent,
as we have seen, to those due to the point \(A\) together with its
image at \(B\). If \(e\) is the charge of the point \(A\) (Fig. 23), the force
due to it at \(P\) is \(\xp\dfrac{e}{AP^2}\) in the direction \(AP\). Resolving this force in
a direction parallel to \(AC\) and along the radius, its components are
\(\xp\dfrac{e}{AP^3} AC\) in the direction parallel to \(AC\) and \(\xp\dfrac{e}{AP^3} CP\) in the direction
\(CP\). The charge of the image of \(A\) at \(B\) is \(-e\xp\dfrac{CP}{CA}\), and the
force due to the image at \(P\) is \(e\xp\dfrac{CP}{CA} \cdot \dfrac{1}{BP^2}\) in the direction \(PB\). Resolving
this force in the same direction as the other, its components
are
\[
\begin {aligned}
&e \frac{CP}{CA} \cdot \frac{CB}{BP^3} \text{ in a direction parallel to } CA \text{, and}\\
&e \frac{CP^2}{CA \cdot BP^3} \text{ in the direction } PC\text{.}
\end{aligned}
\]

If \(a\) is the radius of the sphere and if \(CA = f = ma\) and \(AP = r\),
then \(CB = \xp\dfrac{1}{m}\,a\) and \(BP = \xp\dfrac{1}{m}\,r\); and if \(e\) is the charge of the point
\(A\), the charge of its image at \(B\) is \(-\xp\dfrac{1}{m}\,e\).

The force at \(P\) due to the charge \(e\) at \(A\) is \(\xp\dfrac{e}{r^2}\) in the direction \(AP\).
%%-----File: 105.png-----%%
Resolving this force in the direction of the radius and a direction
parallel to \(AC\), its components are
\begin{align*}
  &\frac{e}{r^2} \cdot \frac{ma}{r} \text{ in the direction } AC, \text{ and}\\
  &\frac{e}{r^2} \cdot \frac{a}{r} \text{ in the direction } CP\text{.}
\end{align*}

The force at \(P\) due to the image \(-\xp\dfrac{1}{m}\,e\) at \(B\) is \(\xp\dfrac{1}{m}\,e\,\dfrac{1}{BP^2}\) or \(e\xp\dfrac{m}{r^2}\)
in the direction \(PB\). Resolving this in the same directions as the
other force, its components are
\begin{align*}
  &e\,\frac{m}{r^2}\,\frac{BC}{BP} = \frac{ema}{r^3}\text{ in the direction } CA \text{, and}\\
  &e\,\frac{m \cdot CP}{r^2BP}\text{ or }\frac{em^2a}{r^3}\text{ in the direction } PC\text{.}
\end{align*}

The components in the direction parallel to \(AC\) are equal but in
opposite directions. The resultant force is therefore in the direction
of the radius, which confirms what we have already proved,
that the sphere is an equipotential surface to which the resultant
force is everywhere normal. The resultant force is therefore in the
direction \(PC\), and is equal to \(\xp\dfrac{ea}{r^3} (m^2 - 1)\) in the direction \(PC\), that is
to say, towards the centre of the sphere.
\Runhead{DENSITY OF INDUCED CHARGE.}

From this we may ascertain the surface density of the electrification
at any point of the sphere, for, by Coulomb's law, if \(\sigma\) is the
surface density,
\[
4 \pi \sigma = R, \text{ where } R \text{ is the resultant force} \textit{ acting outwards.}
\]

Hence, as the resultant force in this case acts inwards, the surface
density is everywhere negative, and is
\[
  \sigma = - \frac{1}{4\pi} \frac{ea}{r^3} (m^2 - 1)\text{.}
\]

Hence the surface density is inversely as the cube of the distance
from the inducing point \(A\).

\article{106} In the case of the two spheres \(A\) and \(B\) (Fig. 24), whose
radii are \(a\) and \(b\) and potentials \(P_a\) and \(P_b\), the distance between
their centres being \(c\), we may determine the surface density at any
point of the sphere \(A\) by considering it as due to the action of a
charge \(aP_a\) at \(A\), together with that due to the pairs of points \(B\),
\(D\) and \(E\), \(F\) \&c., the successive pairs of images.
%%-----File: 106.png-----%%

Putting
\[
r = PB, \quad r_1 = PE, \quad r_2 = PG\text{, \&c.,}
\]
we find
\begin{multline*}
  \sigma = \frac{1}{4\pi} P_a \left[ \frac{1}{a} + \frac{b}{{r_1}^3} \frac{\{(c^2-b^2)^2-a^2c^2\}}{a^2c} + \text{ \&c.}\right]\\
  - \frac{1}{4\pi} P_b \left[ \frac{b}{ar^3} (c^2 - a^2) + \frac{b^2c^2}{{r_2}^3(c^2-a^2)} \left\{\left(\frac{c^2-a^2-b^2}{c^2-a^2} \right)^2 - \frac{a^2}{c^2}\right\} + \text{ \&c.}\right]
\end{multline*}

If we call \(B\) the inducing body and \(A\) the induced body, then we
may consider the electrification induced on \(A\) as consisting of two
parts, one depending on the potential of \(B\) and the other on its
own potential.

The part depending on \(P_b\) is called by some writers on electricity
the \textit{induced electrification of the first species}. When \(A\) is not insulated
it constitutes the whole electrification, and if \(P_b\) is positive
it is negative over every part of the surface, but greatest in
numerical nature at the point nearest to \(B\).

The part depending on \(P_a\) is called the \textit{induced electrification of
the second species}. It can only exist when \(A\) is insulated, and it
is everywhere of the same sign as \(P_a\). If \(A\) is insulated and without
charge, then the induced electrifications of the first and second
species must be equal and opposite. The surface-density is negative
on the side next to \(B\) and positive on the side furthest from \(B\), but
though the total quantities of positive and negative electrification
are equal, the negative electrification is more concentrated than the
positive, so that the neutral line which separates the positive from
the negative electrification is not the equator of the sphere, but lies
nearer to \(B\).

The condition that there shall be both positive and negative electrification
on the sphere is that the value of \(\sigma\) at the points nearest
to \(B\) and farthest from \(B\) shall have opposite signs. If \(a\) and \(b\) are
small compared with \(c\), we may neglect all the terms of the coefficients
of \(P_a\) and \(P_b\) after the first. The values of \(r\) lie between
\(c+a\) and \(c-a\). Hence, if \(P_a\) is between \(P_b \xbp\dfrac{b\,(c-a)}{(c+a)^2}\) and \(P_b\xbp\dfrac{b\,(c+a)}{(c-a)^2}\),
there will be both positive and negative electrification on \(A\), divided
by a neutral line, but if \(P_a\) is beyond these limits, the electrification
of every part of the surface will be of one kind; negative if \(P_a\) is
below the lower limit, and positive if it is above the higher limit.
%%-----File: 107.png-----%%

\newchapter
\Chapter{CHAPTER VIII.}
\Subheading{ON ELECTROSTATIC CAPACITY.}

\article{107} \textsc{The} capacity of a conductor is measured by the charge of
electricity which will raise its potential to the value unity, the
potential of all other conductors in the field being kept at zero.
The capacity of a conductor depends not only on its own form and
size, but on the form and position of the other conductors in the
field. The nearer the uninsulated conductors are placed the greater
is the capacity of the charged conductor.

An apparatus consisting of two insulated conductors, each presenting
a large surface to the other with a small distance between
them, is called a \textit{condenser}, because a small electromotive force is
able to charge such an apparatus with a large quantity of electricity.

The simplest form of condenser, that to which the name is most
commonly applied, consists of two disks placed parallel to each
other, the medium between them being air. When one of these
disks is connected to the zinc and the other to the copper electrode
of a voltaic battery, the disks become charged with negative and
positive electricity respectively, and the amount of the charge is
the greater the nearer the disks are placed to each other, being
approximately inversely as the distance between them. Hence by
bringing the disks very close to each other, connecting them with
the electrodes of the battery and then disconnecting them from the
battery, we have two large charges of opposite kinds insulated on
the disks. If we now remove one of the disks from the other we
do work against the electric attraction which draws them together,
and we may thus increase the energy of the system so much that,
though the original electromotive force was only that of a single
voltaic cell, either of the disks when separated may be raised to so
%%-----File: 108.png-----%%
high a potential that the gold leaves of an electrometer connected
with it are deflected.

It was in this way that Volta demonstrated that the electrification
due to a voltaic cell is of the same kind as that due to friction,
the copper electrode being positive with respect to the zinc electrode.
In this condenser the capacity of each disk depends principally
on the distance between it and the other disk, but it also
depends in a smaller degree on the nature of the electric field at
the back of the disk.

There are other forms of condensers, however, in which one of
the conductors is almost or altogether surrounded by the other.
In this case the capacity of the inner conductor is almost or altogether
independent of everything but the outer conductor. This is
the case in the Leyden jar, and in a cable with a copper core surrounded
by an insulator the outside of which is protected by a
sheathing of iron wires.

\article{108} But in most cases the charge of each conductor depends
not only on the difference between its potential and that of the
other conductor, but also in part on the difference between its
potential and that of some other body, such as the earth, or the
walls of the room where the experiment is made. The charges of
the two conductors may, therefore, in the simpler cases be written
\begin{align*}
Q &= K(P-p) + HP\text{,}\tag{1}\\
q &= K(p-P) + hp\text{,}\tag{2}
\end{align*}
where \(P\) and \(p\) are the potentials, that of the walls of the room
being zero, \(Q\) and \(q\) the charges of the two conductors respectively,
\(K\) is the capacity of the condenser in so far as it depends on the
mutual relation of the two conductors, and \(H\) and \(h\) represent those
parts of the capacity of each conductor which depend on their relation
to external objects, such as the walls of the room.
\Runhead{DISCHARGE BY ALTERNATE CONTACTS.}

If we connect the second conductor with the earth we make \(p\)
zero while \(Q\) remains the same, and we get for the new values of
\(P\), \(Q\), and \(q\),
\[
P_1=P-\frac{K}{K+H}p\text{,}\quad Q_1=(K+H)P_1\text{,}\quad q_1=-KP_1\text{,}\tag{3}
\]

If we now insulate the second conductor and connect the first
with the earth we make \(P\) zero, and
\[
p_2=-\frac{K}{K+h}P_1\text{,}\quad Q_2=-Kp_2\text{,}\quad q_2=(K+h)p_2\text{,}\tag{4}
\]
%%-----File: 109.png-----%%

If we again insulate the first conductor and put the second to
earth,
\[
P_3=-\frac{K}{K+H}p_2\text{,}\quad Q_3=(K+H)P_3\text{,}\quad q_3=-KP_3\text{.}\tag{5}
\]

From this it appears that if we connect first the one and then
the other conductor with the earth the values of the potentials and
charges will be diminished in the ratio of \(\dfrac{K^2}{(K+H)(K+h)}\)  to
unity.

\Subsection{Comparison of two condensers.}

\article{109} Let us suppose the condensers to be Leyden jars having
an inner and an outer coating.
\Runhead{COMPARISON OF TWO CONDENSERS.}

Let the inner coating of the first jar and the outer coating of
the second be connected with a source of electricity and brought to
the potential \(P\), while the outer coating of the first and the inner
coating of the second are connected with the earth.

Then if \(Q_1\) and \(Q_2\) are the charges of the inner coatings of the
two jars,
\[
Q_1=(K_1+H_1)P\text{,}\quad Q_2=-K_2 P\text{.}\tag{7}
\]

Now let the outer coatings of both jars be connected with the
earth, and let the inner coatings be connected with each other.
Required the common potential of the inner coatings.

Hence we have
\begin{gather*}
{p_1}'={p_2}'=0\text{,}\\
Q_1 + Q_2 = {Q_1}' + {Q_2}'\text{,}\tag{8}\\
{P_1}' = {P_2}' = P'\text{,}\tag{9}
\end{gather*}
and we have to find \(P'\).

Equation (8) becomes, in virtue of (9),
\[
(K_1+H_1-K_2)P=(K_1+H_1+K_2+H_2)P'\text{.}
\]
If \(K_1 + H_1 = K_2\) the discharge is complete.

\article{110} The following method, by which the existence of a determinate
relation between the capacities of four condensers may be
verified, has been employed by Sir W. Thomson.\footnote{Gibson and Barclay.} It corresponds
in electrostatics to Wheatstone's Bridge in current electricity.

\widefig{0.75}{110.png}{Fig. 25.}
In Fig. 25 the condensers are represented as Leyden jars. Two
of these, \(P\) and \(Q\), are placed with their external coatings in contact
with an insulating stand \(\beta\); the other two, \(R\) and \(S\), have their
%%-----File: 110.png-----%%
external coatings connected to the earth. The inner coatings of \(P\)
and \(R\) are permanently connected; so are those of \(Q\) and \(S\). In
performing the experiment the internal coatings of \(P\) and \(R\) are
first charged to a potential, \(A\), while those of \(Q\) and \(S\) are charged
to a different potential, \(C\). During this process the stand \(\beta\) is
connected to the earth. The stand \(\beta\) is then disconnected from the
earth and connected to one electrode of an electrometer, the other
electrode being connected to earth. Since \(\beta\) is already reduced to
potential zero by connection with the earth, there will be no disturbance
of the electrometer unless there is leakage in some of the
jars. We shall assume, however, that there is no leakage, and
that the electrometer remains at zero.
%%-----File: 111.png-----%%

The inner coatings of the four jars are now made to communicate
with each other by dropping the small insulated wire \(w\) so as to
fall on the two hooks connected with \(\alpha\) and \(\gamma\). Since the potentials
of \(\alpha\) and \(\gamma\) are different a discharge will occur, and the potential of
\(\beta\) will in general be affected, and this will be indicated by the
electrometer. If, however, there is a certain relation among the
capacities of the jars the potential of \(\beta\) will remain zero.

\widefig{0.57}{110_1.png}{Fig. 26.}
\article{111} Let us ascertain what this relation must be. In Fig. 26
the same electrical arrangement is represented under a simpler
form, in which the condensers consist each of a pair of disks.
Under this form the analogy with Wheatstone's Bridge becomes
apparent to the eye. We have to consider the potentials and
charges of four conductors. The first consists of the inner coatings
of \(P\) and \(R\), together with the connecting wire. We shall call this
conductor \(\alpha\), its charge \(a\), and its potential \(A\). The second consists
of the outer coatings of \(P\) and \(Q\), together with the insulating stand
\(\beta\). We shall call this conductor \(\beta\), its charge \(b\), and its potential
\(B\). The third consists of the inner coatings of \(Q\) and \(S\) and the
connecting wire \(\gamma\). We shall call this \(\gamma\), its charge \(c\), and its
potential \(C\). The fourth consists of the outer coatings of \(R\) and \(S\)
and of the earth with which they are kept connected. We might
use the letters \(\delta\), \(d\), and \(D\) with reference to this conductor, but as
its potential is always zero and its charge equal and opposite to
that of the other conductors we shall not require to consider it.

The charge of any one of the conductors depends on its own
potential together with the potentials of the two adjacent conductors,
and also, but in a very slight degree, on that of the opposite
conductor.

Let the coefficients of induction between the different pairs of
the four conductors be as in the following scheme,---
\Runhead{COMPARISON OF CONDENSERS.}

\widefig{0.45}{111.png}{Fig. 27.}
\noindent in which \(\xi\) and \(\eta\) are very small compared with \(P\), \(Q\), \(R\), and \(S\).
The coefficient of capacity of any one of the conductors will exceed
%%-----File: 112.png-----%%
the sum of its three coefficients of induction by a quantity which
will be small if the capacity of the knobs of the jars and their
connecting wires are small compared with the whole capacities of
the jars. Let us denote this excess by the symbols \(\alpha\), \(\beta\), \(\gamma\), \(\delta\), which
belong to the conductors. The capacities therefore will be,
\begin{align*}
&P + R + \alpha + \eta\text{,}\\
&P + Q + \beta + \xi\text{,}\\
&Q + S + \gamma + \eta\text{,}\\
&R + S + \delta + \xi\text{,}
\end{align*}
and the charges will therefore be,
\begin{align*}
\shortintertext{for \(\alpha\),}
a &= (P+R+\alpha+\eta)A-PB-RD-\eta C\text{,}\\
\shortintertext{for \(\beta\),}
b &= (P+Q+\beta+\xi)B-PA-QC-\xi D\text{,}\\
\shortintertext{for \(\gamma\),}
c &= (Q+S+\gamma+\eta)C-QB-SD-\eta A\text{,}\\
\shortintertext{for \(\delta\),}
d &= (R+S+\delta+\xi)D-RA-SC-\xi B\text{.}
\end{align*}

In the first part of the experiment the potentials of \(\alpha\) and \(\gamma\) are
\(A\) and \(C\) respectively, while those of \(\beta\) and \(\delta\) are zero. Hence, at
first,
\begin{align*}
  a &= (P+R+\alpha+\eta)A-\eta C\text{,}\\
  b &= \hphantom{(P+R+\alpha}-PA-QC\text{,}\\
  c &= (Q+S+\gamma+\eta)C-\eta A\text{.}
\end{align*}
We need not determine the charge of \(\delta\).

Now let a communication be made between \(\alpha\) and \(\gamma\), and let us
denote the charges and potentials of the conductors after the discharge
by accented letters. The potentials of \(\alpha\) and \(\gamma\) will become
equal; let us call their common potential \(y\), then
\[
  A' = C' = y.
\]

The sum of their charges remains the same, or
\[
  a' + c' = a + c.
\]

The charge of \(\beta\) remains the same as before, or
\[
  b' = b,
\]
but its potential is no longer zero, but \(B'\), and we have to determine
the value of \(B'\) in terms of \(A\) and \(C\) by eliminating the other
quantities entering into the equations.

After discharge,
\begin{align*}
  a' &= (P+R+\alpha)y-PB'\text{,}\\
  b' &= (P+Q+\beta+\xi)B' - (P+Q)y\text{,}\\
  c' &= (Q+S+\gamma)y-QB'\text{.}
\end{align*}
%%-----File: 113.png-----%%

Hence, the equation \(a' + c' = a + c\) becomes
\[
(P+R+Q+S+\alpha+\gamma)y-(P+Q)B'=(P+R+\alpha)A+(Q+S+\gamma)C\text{,}
\]
and \(b' = b\) becomes
\[
(P+Q+\beta+\xi)B'-(P+Q)y = -PA-QC\text{.}
\]

Eliminating \(y\) from these equations, we find
\begin{multline*}
B'\{(P+Q)(R+S)+(P+Q)(\alpha+\beta+\gamma+\xi)+(R+S+\alpha+\gamma)(\beta+\xi)\}\\
=\{Q(R+\alpha)-P(S+\gamma)\}(A-C)\text{.}
\end{multline*}

If, therefore, the electrometer is not disturbed by the discharge,
\({B' = 0}\), and
\[
P : Q :: R + \alpha : S + \gamma\text{.}
\]
%%-----File: 114.png-----%%

\newchapter
\Chapter{CHAPTER IX.}

\Subheading{THE ELECTRIC CURRENT.}

\article{112} \textsc{Let} \(A\) and \(B\) be two metal bodies connected respectively
with the inner and outer coatings of a Leyden jar, the inner
coating of the jar being positive, so that the
potential of \(A\) is higher than that of \(B\).

\wrapfig{0.28}{114.png}{Fig. 28.}
Let \(C\) be a gilt pith ball suspended by a silk
thread. If \(C\) is brought into contact with \(A\) and
\(B\) alternately, it will receive a small charge of
positive electricity from \(A\) every time it touches
it, and will communicate positive electricity to \(B\)
when it touches \(B\).

There will thus be a transference of positive
electricity from \(A\) to \(B\) along the path travelled
over by the pith ball, and this is what occurs
in every electric current, namely, the passage of electricity along a
definite direction. During the motion of the pith ball from \(A\) to
\(B\) it is charged positively, and the electric force between \(A\) and
\(B\) tends to move it in the direction from \(A\) to \(B\). After touching
\(B\), it becomes charged negatively, so that the electric force, during
its return journey, acts from \(B\) to \(A\). Hence the ball is acted
on by the electric force always in the direction in which it is
moving at the time, so that if it is properly suspended the electric
force will not only keep up the backward and forward motion, but
will communicate to the moving ball an amount of energy which it
will expend in a series of rattling blows against the balls \(A\) and \(B\).
The current of positive electricity from \(A\) to \(B\) is thus kept up by
means of the electromotive force from \(A\) to \(B\).

\article{113} The phenomenon we have been describing may be called
a current of Convection. The motion of the electrification takes
%%-----File: 115.png-----%%
place in virtue of the motion of the electrified body which \textit{conveys}
or \textit{carries} the electricity as it moves from one place to another.
But if, instead of the pith ball, we take a metal wire carried
by an insulating handle, and cause the two ends of the wire
to touch \(A\) and \(B\) respectively, there will be a transference of
electricity from \(A\) to \(B\) along the wire, though the wire itself does
not move.
\Runhead{CONVECTION AND CONDUCTION CURRENTS.}

What takes place in the wire is called a current of Conduction.
The effects of the current of conduction on the electrical state of
\(A\) and \(B\) are of precisely the same kind as those of the current of
convection. In both cases there is a transference of electrification
from one place to another along a continuous path.

In the case of the convection of the charge on the pith ball we
may observe the actual motion of the ball, and therefore in this
case we may distinguish between the act of carrying a positive
charge from \(A\) to \(B\) and that of carrying a negative charge from
\(B\) to \(A\), though the Electrical effects of these two operations are
identical. We may also distinguish between the act of carrying a
number of small charges from \(A\) to \(B\) in rapid succession and
with great velocity, and the act of carrying a single great
charge, equivalent to the sum of these charges, slowly from \(A\) to
\(B\) in the time occupied by the whole series of journeys in the other
case.

But in the case of the current conduction through a wire we
have no reason to suppose that the mode of transference of the
charge resembles one of these methods rather than another. All
that we know is that a charge of so much electricity is conveyed
from \(A\) to \(B\) in a certain time, but whether this is effected by
carrying positive electricity from \(A\) to \(B\), or by carrying negative
electricity from \(B\) to \(A\), or by a combination of both processes, is a
question which we have no means of determining. We are equally
unable to determine whether the `velocity of electricity' in the
wire is great or small. If there be a substance pervading bodies,
the motion of which constitutes an electric current, then the excess
of this substance in connexion with a body above a certain normal
value constitutes the observed charge of that body, and is a
quantity capable of measurement. But we have no means of
estimating the normal charge itself. The only evidence we possess
is deduced from experiments on the quantity of electricity evolved
during the decomposition of one grain of an electrolyte, and this
quantity is enormous when compared with any positive or negative
%%-----File: 116.png-----%%
charge which we can accumulate within the space occupied by the
electrolyte. If, then, the normal charge of a portion of the wire
the millionth of an inch in length is equal to the total charge
transferred from \(A\) to \(B\), the transference may be effected by the
displacement of the electricity in the wire whose linear extent is
only the millionth of an inch.

It is therefore quite possible that the velocity of electricity in a
telegraph wire may be exceedingly small, less, say, than the
hundredth of an inch in an hour, though signals, that is to say,
changes in the state of the current, may be propagated along the
wire many thousands of miles in a second.
\Runhead{MEASURE OF CURRENT.}

Since, therefore, we are ignorant of the true linear velocity of an
electric current, we must measure the \textit{strength} of the current by the
quantity of electricity discharged through any section of the conductor
in the unit of time, just as engineers measure the discharge
of water and gas through pipes, not by the velocity of the water or
gas, but by the quantity which passes in a minute.

\article{114} In many cases we have to consider the whole quantity of
electricity which passes rather than the rate at which it passes.
This is especially the case when the current lasts only a very short
time, or when the current is considered merely as a transition
from one permanent state of the system to another. In these cases
it is convenient to speak of the total current as the Electric
Displacement, the word displacement indicating the final result
of a motion without reference to the rate at which it takes place.
The passage of a given quantity of electricity along a given path is
called an Electric Discharge.

\Subsection{Classification of bodies according to their relation to the
transference of electricity.}

\article{115} For the sake of distinction we shall consider a portion of
matter whose ends are formed by two equipotential surfaces having
different potentials, and whose sides are formed by lines of electric
current or displacement.

\phantomsection\label{art:115a}
The ends of the body are called its Electrodes, that at which
electricity enters is called the Anode, and that at which it leaves
the body is called the Cathode.

\phantomsection\label{art:115b}
The excess of the potential of the anode over that of the cathode
is called the External Electromotive Force.

The Form of the body may vary from that of a long wire surrounded
%%-----File: 117.png-----%%
by air or other insulating matter to that of a thin sheet of
the substance, the electricity passing through the thickness of the
sheet.

\phantomsection\label{art:115c}
Bodies may be divided into three great classes according to the
mode in which they are acted on by electromotive force,---Metals,
Electrolytes, and Dielectrics.

\Section{First Class.---Metals, \&c.}

\article{116} The first class includes all the metals, whether in the solid
or liquid state, together with some other substances not regarded
by chemists as metals. In these the smallest external electromotive
force is capable of producing an electric current, and this current
continues to flow as long as the electromotive force continues to
act, without producing any change in the chemical properties of the
substance. The strength of the permanent current is proportional
to the electromotive force. The ratio of the numerical value of the
electromotive force to that of the current is called the Resistance
of the conductor. The same thing may be otherwise stated by
saying that the flow of the current is opposed by an internal
electromotive force, proportional to the strength of the current,
and to a quantity called the Resistance of the conductor, depending
on its form and nature. When the strength of the current is such
that this internal electromotive force balances the external electromotive
force the current neither increases nor diminishes in strength.
It is then said to be a \textit{steady} current.

These relations were first established by Dr.\ G. S. Ohm, in a
work published in 1827. They are expressed by the formula,
\[
\text{Electromotive force} = \text{Current} \times \text{Resistance,}
\]
which is called Ohm's Law.
\Runhead{OHM'S LAW.}

\Subsection{Generation of Heat by the current.}

\article{117} During the flow of a steady current through a conductor
of uniform material of the first class heat is generated in the
conductor, but the substance of the conductor will not be affected
in any way, for if the heat is allowed to escape as fast as it is
generated, the temperature and every other physical condition of
the conductor remains the same.

The whole work done by the external electromotive force in
urging electricity through the body is therefore spent in generating
%%-----File: 118.png-----%%
heat. The dynamical equivalent of the heat generated is therefore
equal to the electrical work spent, that is, to the product of the
electromotive force into the quantity of electricity transmitted by
the current.

Now, the electromotive force is, by Ohm's law the product of
the strength of the current into the resistance, and the quantity
of electricity is, by the definition of a current, the product of the
current into the time during which it flows, so that we find,
\begin{multline*}
\text{Heat generated measured in dynamical units}\\
= \text{Square of Current} \times \text{Resistance} \times \text{Time.}
\end{multline*}
This relation was first established by Dr.\ Joule, and is therefore
called Joule's law. It was also established independently by Lenz.
\Runhead{JOULE'S LAW.}

\Section{Second Class.---Electrolytes.}

\article{118} The second class of substances consists of compound bodies,
generally in the liquid form, called Electrolytes.

When an electric current passes through fused chloride of silver,
which is an electrolyte, chlorine appears at the anode where the
current enters, and silver at the cathode where the current leaves
the electrolyte. The quantities of these two substances are such
that if combined they would form chloride of silver. The composition
of those portions of the electrolyte which lie between the
electrodes remains unaltered. Hence, if we fix our attention upon
a portion of the electrolyte between two fixed planes perpendicular
to the direction of the current, the quantity of silver or of chlorine
which enters the portion through one plane must be equal to the
quantity which leaves it through the other plane. It follows from
this that in every part of the electrolyte the silver is moving in the
direction of the current, and the chlorine in the opposite direction.

This operation, in which a compound body is decomposed by an
electric current, is called Electrolysis, and the mode in which the
current is transmitted is called Electrolytic Conduction. The
compound body is called an Electrolyte, and the components into
which it is separated are called Ions. That which appears at the
anode is called the Anion, and that which appears at the cathode is
called the Cation.

\phantomsection\label{art:118a}
The quantity of the substance which is decomposed is proportional
to the total quantity of electricity which passes through it,
and is independent of the time during which the electricity is
%%-----File: 119.png-----%%
passing. The quantity corresponding to the passage of one unit
of electricity is called the Electrochemical Equivalent of the substance.
Thus, when one unit of electricity is passed through fused
chloride of silver, one electrochemical equivalent of silver appears
at the cathode and one electrochemical equivalent of chlorine at
the anode, and one electrochemical equivalent of chloride of silver
disappears.

\article{119} The electrochemical equivalents of the same substance, as
deduced from experiments on different electrolytes which contain
it, are consistent with each other. Thus the electrochemical
equivalent of chlorine is the same, whether we deduce it from
experiments on chloride of silver, or from experiments on hydrochloric
acid, and that of silver is the same, whether we deduce
it from experiments on chloride of silver, or from experiments on
nitrate of silver. These laws of electrolysis were established by
Faraday\footnote{\textit{Exp.\ Res.}, series vii and viii.}.
If they are strictly true the conduction of electricity
through an electrolyte is always electrolytic conduction, that is to
say, the electric current is always associated with a flow of the
components of the electrolyte in opposite directions.
\Runhead{FARADAY'S LAWS OF ELECTROLYSIS.}

\phantomsection\label{art:119a}
Such a flow of the components necessarily involves their appearance
in a separate form at the anode and the cathode. To effect
this separation a certain electromotive force is required depending
on the energy of combination of the electrolyte. Thus the electromotive
force of one of Daniell's cells is not sufficient to decompose
dilute sulphuric acid.

If, therefore, an electrolytic cell, consisting of a vessel of
acidulated water, in which two platinum plates are placed as
electrodes, is inserted in the circuit of a single Daniell's cell, along
with a galvanometer to measure the current, it will be found that
though there is a transient current at the instant the circuit is
closed, this current rapidly diminishes in intensity, so as to become
in a very short time too weak to be measured except by a very
sensitive galvanometer.

Neither oxygen nor hydrogen, the chemical components of water,
appear in a gaseous form at the electrodes, but the electrodes themselves
acquire new properties, shewing that a chemical action has
taken place at the surface of the platinum plates.

\article{120} If the Daniell's cell is taken out of the circuit, and the
circuit again closed, the galvanometer indicates a current passing
through the electrolytic cell in the opposite direction to the original
%%-----File: 120.png-----%%
current. This current rapidly diminishes in strength and soon
vanishes, so that the whole quantity of electricity which is transmitted
by it is never greater than that of the primitive current.
This reverse current indicates that the platinum plates have acquired
a difference of properties by being used as electrodes. They
are said to be polarized. The cathode is polarized positively and
the anode negatively, so that an electromotive force is exerted in
the circuit opposite to that of the Daniell's cell. This electromotive
force, which is called the electromotive force of polarization, is the
cause of the rapid diminution in the strength of the original current,
and of its final cessation.
\Runhead{POLARIZATION.}

A chemical examination of the platinum plates shews that a
certain quantity of hydrogen has been deposited on the cathode.
This hydrogen is not in the ordinary gaseous form, but adheres to
the surface of the platinum so firmly that it is not easy to remove
the last traces of it.

\article{121} Faraday's law, that conduction takes place in electrolytes
only by electrolysis, was long supposed not to be strictly true.
In the experiment in which a single Daniell's cell furnishes the
electromotive force in a circuit containing an electrolyte and a
galvanometer, it is found that the current soon becomes very feeble
but never entirely vanishes, so that if the electromotive force is
maintained long enough, a very considerable quantity of electricity
may be passed through the electrolyte without any visible decomposition.

Hence it was argued that electrolytes conduct electricity in two
different ways, by electrolysis in a very conspicuous manner and
also, but in a very slight degree, in the manner of metals, without
decomposition. Helmholtz has recently\footnote
{\textit{Ueber galvanische Polarisation in gusfreien Flüssigkeiten. Monatsbericht d.\ K.
Akad.\ d.\ Berlin}, July 1873, p.\ 587.} shewn that the feeble
permanent current can be explained in a different manner, and that
we have no evidence that an electrolyte can conduct electricity
without electrolysis.

\article{122} In the case of platinum plates immersed in dilute sulphuric
acid, if the liquid is carefully freed from all trace of oxygen or
of hydrogen in solution, and if the surfaces of the platinum plates
are also freed from adhering oxygen or hydrogen, the current continues
only till the platinum plates have become polarized and no
permanent current can be detected, even by means of a sensitive
galvanometer. When the experiment is made without these precautions,
%%-----File: 121.png-----%%
there is generally a certain amount of oxygen or of hydrogen
in solution in the liquid, and this, when it comes in contact with
the hydrogen or the oxygen adhering to the platinum surface,
combines slowly with it, as even the free gases do in presence
of platinum. The polarization is thus diminished, and the electromotive
force is consequently enabled to keep up a permanent
current, by what Helmholtz has called electrolytic convection.
Besides this, it is probable that the molecular motion of the liquid
may be able occasionally to dislodge molecules of oxygen or of
hydrogen adhering to the platinum plates. These molecules when
thus absorbed into the liquid will travel according to the ordinary
laws of diffusion, for it is only when in chemical combination that
their motions are governed by the electromotive force. They will
therefore tend to diffuse themselves uniformly through the liquid,
and will thus in time reach the opposite electrode, where, in contact
with a platinum surface, they combine with and neutralize part of
the other constituent adhering to that surface. In this way a
constant circulation is kept up, each of the constituents travelling
in one direction by electrolysis, and back again by diffusion, so that
a permanent current may exist without any visible accumulation
of the products of decomposition. We may therefore conclude that
the supposed inaccuracy of Faraday's law has not yet been confirmed
by experiment.
\Runhead{ELECTROLYTIC CONVECTION.}

\article{123} The verification of Ohm's law as applied to electrolytic
conduction is attended with considerable difficulty, because the
varying polarization of the electrodes introduces a variable electromotive
force, and renders it difficult to ascertain the true electromotive
force at any instant. By using electrodes in the form of
plates, having an area large compared with the section of the
electrolyte, and employing currents alternately in opposite directions,
the effect of polarization may be diminished relatively to
that of true resistance. \phantomsection\label{art:123a}It appears from experiments conducted
in this way that Ohm's law is true for electrolytes as well as
for metals, that is to say, that the current is always proportional
to the electromotive force, whatever be the amount of that force.
The reason that the external resistance of an electrolyte appears
greater for small than for large electromotive forces is that the
external electromotive force between the metallic electrodes is not
the true electromotive force acting on the electrolyte. There is,
in general, a force of polarization acting in the opposite direction
to the external electromotive force, and it is only the excess of
%%-----File: 122.png-----%%
the external force above the force of polarization that really acts on
the electrolyte.

It appears, therefore, that the very smallest electromotive force,
if it really acts on the electrolyte, is able to produce conduction by
electrolysis. How, then, is this to be reconciled with the fact that
in order to produce complete decomposition a very considerable
electromotive force is required?

\article{124} Clausius\footnote{\textit{Pogg.\ Ann.}\ CI. 338 (1857).}
has pointed out that on the old theory of
electrolysis, according to which the electromotive force was supposed
to be the sole agent in tearing asunder the components
of the molecules of the electrolyte, there ought to be no decomposition
and no current as long as the electromotive force is
below a certain value, but that as soon as it has reached this
value a vigorous decomposition ought to commence, accompanied
by a strong current. This, however, is by no means the case,
for the current is strictly proportional to the electromotive force for
all values of that force.
\Runhead{ELECTROLYSIS.}

Clausius explains this in the following way:---

According to the theory of molecular motion of which he has
himself been the chief founder, every molecule of the fluid is
moving in an exceedingly irregular manner, being driven first
one way and then another by the impacts of other molecules which
are also in a state of agitation.

This molecular agitation goes on at all times independently of
the action of electromotive force. The diffusion of one fluid through
another is brought about by this molecular agitation, which increases
in velocity as the temperature rises. The agitation being
exceedingly irregular, the encounters of the molecules take place
with various degrees of violence, and it is probable that even at
low temperatures some of the encounters are so violent that one
or both of the compound molecules are split up into their constituents.
Each of these constituent molecules then knocks about
among the rest till it meets with another molecule of the opposite
kind and unites with it to form a new molecule of the compound.
In every compound, therefore, a certain proportion of the molecules
at any instant are broken up into their constituent atoms.
At high temperatures the proportion becomes so large as to
produce the phenomenon of dissociation studied by M. St. Claire
Deville\footnote{
[\textit{Leçons sur la Dissociation, professées devant la Société Chimique.} L. Hachette
et C\textsuperscript{ie}, 1866.]}.
%%-----File: 123.png-----%%

\article{125} Now Clausius supposes that it is on the constituent molecules
in their intervals of freedom that the electromotive force acts,
deflecting them slightly from the paths they would otherwise have
followed, and causing the positive constituents to travel, on the
whole, more in the positive than in the negative direction, and
the negative constituents more in the negative direction than in
the positive. The electromotive force, therefore, does not produce
the disruptions and reunions of the molecules, but finding these
disruptions and reunions already going on, it influences the motion
of the constituents during their intervals of freedom. The amount
of this influence is proportional to the electromotive force when the
temperature is given. The higher the temperature, however, the
greater the molecular agitation, and the more numerous are the free
constituents. Hence the conductivity of electrolytes increases as
the temperature rises.

This effect of temperature is the opposite of that which takes
place in metals, the resistance of which increases as the temperature
rises. This difference of the effect of temperature is sometimes
used as a test whether a conductor is of the metallic or the
electrolytic kind. The best test, however, is the existence of
polarization, for even when the quantity of the free ions is too
small to be observed or measured, their presence may be indicated
by the electromotive force which they excite.

\article{126} Kohlrausch\footnote{\textit{Göttingen Nachrichten}, 5 Aug., 1874, 17 May, 1876, and 4 April, 1877.}
finds that if the electromotive force is one
volt per centimetre in length of the electrolyte, then if the electrolyte
differs but slightly from pure water at 18°C the velocity of
hydrogen is about 0·0029 centimetres per second, and that the
actual force on a gramme of hydrogen in the solution required
to make it move at the rate of one centimetre per second
through the solution is equal to the weight of 330,000,000 kilogrammes.

The velocities of the components of unibasic acids and their salts
were found by Kohlrausch to be in the following proportion:---
\begin{table}[H]
\centering
\caption*{\textsc{Table I.}}
\footnotesize
\begin{tabular}{c c c c c c c c c}
H & K & \ce{NH4} & Na &  Li & \ce{1/2Ba} & \ce{1/2Sr} & \ce{1/2Ca} & \ce{1/2Mg}\\[0.7ex]
273 & 48 & 46 & 30 & 19 & 31 & 28 & 24 & 21\\[1.3ex]
& I & Br & Cl & F & \ce{NO3} & \ce{ClO3} & \ce{C2H3O2} &\\[0.7ex]
& 55 & 53 & 50 & 29 & 47 & 36 & 22 &\\
\end{tabular}
\end{table}
%%-----File: 124.png-----%%

\article{127} The specific molecular conductivity \(\left( l\right)\) of an electrolyte
is the sum of the velocities of its components\footnote
{[Compare \textit{Cavendish Papers}, pp.\ 446, 447.]}, and the actual
conductivity of any weak solution is found by multiplying the
number \(l\) by the number of grammes of the substance in a litre
and dividing by the molecular weight of the substance, that of
hydrogen being 1.

\article{128} We have reason to believe that water is not an electrolyte,
and that it is not a conductor of the electric current. It is
exceedingly difficult to obtain water free from foreign matter.
Kohlrausch\footnote{
[\textit{Poggendorff}, \textit{Ergänzungsband}, VIII (1876), pp.\ 7, 9, 11.]}, however, has obtained water so pure that its resistance
was enormous compared with ordinary distilled water. When
exposed to the air for [4·3 hours its conductivity rose 70 per
cent.], and [in 1060 hours it was increased nearly fortyfold. After
long exposure to the air the conductivity was more than doubled in
4·5 hours by the action of tobacco smoke.] Water kept in glass
vessels very soon dissolves enough of foreign matter to enable it
to conduct freely.
\Runhead{WATER NOT AN ELECTROLYTE.}

Kohlrausch\footnote{
[\textit{Pogg.\ Ann.}\ Vol.\ CLIV (1875), p.\ 215; Vol.\ CLIX (1876), p.\ 242; \textit{Phil.\ Mag.}\
June 1875.]} has determined the resistance of water containing a
very small percentage of different electrolytes, and he finds that
the results agree very well with the hypothesis that the velocity
with which each ion travels through the liquid is proportional
to the electromotive force, the velocity corresponding to unit of
electromotive force being different for different ions, but the same
for the same ion, whatever the other ion may be with which it is
combined. The velocities of different ions in centimetres per
second, corresponding to an electromotive force of one volt, are
given in Table II.

\begin{table}[H]
\centering
\caption*{\textsc{Table II.}}
\footnotesize
\begin{tabular}{c c c c c c c c c}
H & K & \ce{NH4} & Na & Li & Ba & Sr & Ca & Mg\\
·0029 & ·00051 & ·00049 & ·00032 & ·00020 & ·00033 & ·00030 & ·00025 & ·00022\\[1.3ex]
& I & Br & Cl & F & \ce{NO3} & \ce{ClO3} & \ce{C2H3O2} &\\
& ·00058 & ·00056 & ·00053 & ·00031 & ·00050 & ·00038 & ·00023 &
\end{tabular}
\end{table}

When the water contains a large percentage of foreign matter
the velocities of the ions are no longer the same, as it is no longer
through water, but through a liquid of quite different physical
properties that they have to make their way. It appears from
%%-----File: 125.png-----%%
Table III\footnote{[See also p.\ \pageref{232:1}.]} that though for small percentages of sulphuric acid in
water the conducting power is proportional to the percentage of
acid, yet as the proportion of acid increases the conducting power
increases more slowly till a maximum conducting power is reached,
after which the addition of acid diminishes the conducting power\footnote{
[A similar result was found with nitric acid and some viscous saline solution.]}.

\begin{table}[H]
\centering
\captionsetup{justification=centering}
\caption*{\textsc{Table III.}\\
\textit{Conductivity of Sulphuric Acid at 18°C referred to that\\
of Mercury at 0°C as unity.}}
\footnotesize
\begin{tabular}{|S[table-format = 2.1]|S[table-format = 4.0]||
  S[table-format = 2.0]|S[table-format = 4.0]||
  S[table-format = 2.1]|S[table-format = 4.0]|}
\hline
\multicolumn{1}{|m{2cm}|}{\centering \bigstrut[t] Percentage of \ce{H2SO4}}\bigstrut[b] &
\multicolumn{1}{m{1.4cm}||}{\centering \(10^8\)K} &
\multicolumn{1}{m{2cm}|}{\centering Percentage of \ce{H2SO4}} &
\multicolumn{1}{m{1.4cm}||}{\centering \(10^8\)K} &
\multicolumn{1}{m{2cm}|}{\centering Percentage of \ce{H2SO4}} &
\multicolumn{1}{m{1.4cm}|}{\centering \(10^8\)K} \\
\hline
 1  &  429 & 60 & 3487 & 87  &  944 \bigstrut[t]\\
 2.5& 1020 & 65 & 2722 & 88  &  965\\
 5  & 1952 & 70 & 2016 & 89  &  986\\
10  & 3665 & 75 & 1421 & 90  & 1005\\
15  & 5084 & 78 & 1158 & 91  & 1022\\
20  & 6108 & 80 & 1032 & 92  & 1030\\
25  & 6710 & 81 &  985 & 93  & 1024\\
30  & 6912 & 82 &  947 & 94  & 1001\\
35  & 6776 & 83 &  924 & 95  &  958\\
40  & 6361 & 84 &  915 & 96  &  885\\
45  & 5766 & 85 &  916 & 97  &  750\\
50  & 5055 & 86 &  926 & 99.4&   80\\
55\bigstrut[b]  & 4280 & & & &\\
\hline
\end{tabular}
\end{table}

\article{129} The oxygen and hydrogen which are given off at the
electrodes in so many experiments on water containing foreign
ingredients are, therefore, not the ions of water separated by strict
electrolysis, but secondary products of the electrolysis of the matter
in solution. Thus, if the cation is a metal which decomposes water,
it unites with an equivalent of oxygen and allows the two equivalents
of hydrogen to escape in the form of gas. The anion may
be a [compound radicle] which cannot exist in a separate state,
[but which exists in the nascent condition, and] contains one equivalent
[or more] of [some electronegative element which reacts
upon water and liberates oxygen.]

\Section{Third Class.---Dielectrics.}
\Runhead{PROPERTIES OF DIELECTRICS.}

\article{130} The third class of bodies has an electric resistance so much
greater than that of metals, or even of electrolytes, that they are
often called insulators of electricity. All the gases, many liquids
which are not electrolytes, such as spirit of turpentine, naphtha, \&c.,
and many solid bodies, such as gutta-percha, caoutchouc in its
various forms, amber and resins, crystallized electrolytes, glass
when cold, \&c., are insulators.
%%-----File: 126.png-----%%

They are called insulators because they do not allow a current
of electricity to pass through them. They are called dielectrics
because certain electrical actions can be transmitted through them.
According to the theory adopted in this book, when an electromotive
force acts on a dielectric it causes the electricity to be
displaced within it in the direction of the electromotive force, the
amount of the displacement being proportional to the electromotive
force, but depending also on the nature of the dielectric, the displacement
due to equal electromotive forces being greater in solid
and liquid dielectrics than in air or other gases.

When the electromotive force is increasing, the increase of
electric displacement is equivalent to an electric current in the
same direction as the electromotive force. When the electromotive
force is constant there is still displacement, but no current. When
the electromotive force is diminishing, the diminution of the electric
displacement is equivalent to a current in the opposite direction.

\article{131} In a dielectric, electric displacement calls into action an
internal electromotive force in a direction opposite to that of the
displacement, and tending to reduce the displacement to zero.
The seat of this internal force is in every part of the dielectric
where displacement exists.

To produce electric displacement in a dielectric requires an
expenditure of work measured by half the product of the electromotive
force into the electric displacement. This work is stored
up as energy within the dielectric, and is the source of the
energy of an electrified system which renders it capable of doing
mechanical work.
\Runhead{SPECIFIC INDUCTIVE CAPACITY.}

The amount of displacement produced by a given electromotive
force is different in different dielectrics. The ratio of the displacement
in any dielectric to the displacement in a vacuum due to the
same electromotive force is called the Specific Inductive Capacity
of the dielectric, or more briefly, the Dielectric Constant. This
quantity is greater in dense bodies than in a so-called vacuum, and
is approximately equal to the square of the index of refraction.
Thus Dr.\ L. Boltzmann\footnote{
[\textit{Pogg.\ Ann.}\ CLI. (1874), p.\ 482.]} finds for various substances,
\begin{center}
\footnotesize
\begin{tabular}{lccc}
&
\multicolumn{1}{p{2cm}}{\centering \(\rD\).} &
\multicolumn{1}{p{2cm}}{\centering \(\sqrt{\rD}\).} &
\multicolumn{1}{b{2cm}}{\centering Index of refraction.}\\
Sulphur (cast)      & 3·84 & 1·960 & 2·040\\
Colophonium         & 2·55 & 1·597 & 1·543\\
Paraffin            & 2·32 & 1·523 & 1·536\\
Ebonite (Hartgummi) & 3·15 & 1·775 &
\end{tabular}
\end{center}
%%-----File: 127.png-----%%

For a sphere cut from a crystal of sulphur Boltzmann finds \(D\)
by electrical experiments for the three principal axes, and compares
them with the results as calculated from the three indices of
refraction.

\begin{center}
\footnotesize
\begin{tabular}{lccc}
By electrical experiments & \(\rD_1 = 4·773\) & \(\rD_2 = 3·970\) & \(\rD_3 = 3·811\)\\
By optical measurements   & \(\rD_1 = 4·596\) & \(\rD_2 = 3·886\) & \(\rD_3 = 3·591\)
\end{tabular}
\end{center}
\{Sitzungsb. (Vienna), 9 Jan., 1873.\}

\article{132} Schiller (\textit{Pogg.\ Ann.}\ CLII. 535) ascertained the time of
the electrical vibrations when a condenser is discharged through an
electromagnet. He finds in this way the following values of the
dielectric coefficients of various substances, and compares them with
those found by Siemens by the method of a rapid commutator.
\Runhead{PROPERTIES OF A DIELECTRIC.}

\newcommand{\dittoparaf}{\lowditto{Paraffin, }}
\begin{center}
\footnotesize
\begin{tabular}{lcccc}
& Schiller. & Siemens. & \(\mu^2\). & \(\mu\).\\
Ebonite (Hartgummi)            & 2·21 & 2·76 & &\\
Pure rubber                    & 2·12 & 2·34 & 2·25 & 1·50\\
Vulcanized grey, do.           & 2·69 & 2·94 & &\\
Paraffin, quick cooled, clear  & 1·68 & & &\\
\dittoparaf slow cooled, milk white & 1·81 & 1·92 & 2·19 & 1·48\\
\dittoparaf another specimen & 1·89 & 2·47 & 2·34 & 1·53\\
Straw coloured glass           & 2·96 & 4·12 & &\\
\lowditto{Straw }\lowditto{coloured glass} & 3·66 & & &\\
White mirror glass             & 5·83 & 6·34 & &
\end{tabular}
\end{center}

P. Silow \{\textit{Pogg.\ Ann.}\ CLVI (1875), [p.\ 395]\}\footnote{
[See also CLVIII. (1876), pp.\ 306 \textit{et sqq.}]} finds for oil of
turpentine
{\footnotesize\[
\mathrm D = 2·21 \qquad \sqrt {\mathrm D} = 1·490  \qquad\ \mu _\infty = 1·456\text{.}
\]}

Faraday did not succeed in detecting any difference in the
dielectric constants of different gases. Dr.\ Boltzmann\footnote{
[\textit{Pogg.\ Ann.}\ CLI. (1875), p.\ 403.]} however
has succeeded by a very ingenious method in determining it for
various gases at 0°C, and at one atmosphere pressure.

\begin{center}
\footnotesize
\begin{tabular}{lccc}
& \(\mathrm D\). & \(\sqrt{\mathrm D}\). & \(\mu\).\\
Air             & 1·000590 & 1·000295 & 1·000294\\
Carbonic Acid   & 1·000946 & 1·000473 & 1·000449\\
Hydrogen        & 1·000264 & 1·000132 & 1·000138\\
Carbonic Oxide  & 1·000690 & 1·000345 & 1·000340\\
Nitrous Oxide   & 1·000994 & 1·000497 & 1·000503\\
Olefiant Gas    & 1·001312 & 1·000656 & 1·000678\\
Marsh Gas       & 1·000944 & 1·000472 & 1·000443
\end{tabular}
\end{center}

\Section{Disruptive Discharge.}
\Runhead{DISRUPTIVE DISCHARGE.}

\article{133} If the electromotive force acting at any point of a dielectric
is gradually increased, a limit is at length reached at which there
%%-----File: 128.png-----%%
is a sudden electrical discharge through the dielectric, generally
accompanied with light and sound. The dielectric, if solid, is often
pierced, cracked, or broken, and portions of it are often dispersed in
the form of vapour. This phenomenon appears to be analogous
to the rupture of a solid body when exposed to a continually
increasing stress. This analogy is so complete that we may make
use of the same terms in describing the behaviour of media under
the action of electromotive force as we apply to bodies under the
action of stress. Thus electromotive force and electric displacement
correspond to ordinary force and ordinary displacement; the electromotive
force which produces disruptive discharge corresponds to
the breaking stress. Conduction, or the transmission of electricity,
corresponds to permanent bending.

Thus if we consider the twisting of a wire on the one hand, and
the transmission of electricity through a body on the other, the
moment of the couple which twists the wire will correspond to
the electromotive force acting on the body, and the angle through
which the wire is twisted will correspond to the electric displacement.
If the wire, when the force is removed, returns to its
former shape and becomes completely untwisted it is said to be
elastic. Such a wire corresponds to a dielectric which acts as a
perfect insulator with respect to the electromotive force employed.
If the twisting couple is increased up to a certain limit the wire
gives way and is broken. This corresponds to disruptive discharge,
and the ultimate strength of the wire corresponds to the greatest
electromotive force which the dielectric can support, which we may
call its electric strength.

If before rupture takes place the wire yields so that it will no
longer completely untwist itself when the force is removed it is
said to be plastic. It corresponds to a dielectric which conducts
electricity to a certain extent.

If no such permanent twist can be given to the wire by a force
which is not sufficient to break it, the wire is called brittle. In
like manner we may speak of those dielectrics such as air, which
will not transmit electricity except by the disruptive discharge, as
electrically brittle.

\article{134} Many wires after being kept twisted for some time and
then set free immediately untwist themselves, but through a smaller
angle than they were twisted. In the course of time, however, they
go on untwisting themselves, but very slowly, the process sometimes
going on for days or weeks. In like manner many dielectrics
%%-----File: 129.png-----%%
such as the glass of a Leyden jar or the gutta percha of a submarine
cable, after being subjected for some time to electromotive force
and then placed in a closed circuit give an instantaneous discharge
which is less than the original charge. After this discharge, however,
they are capable of giving residual discharges which become
more and more feeble, and if the circuit is kept closed a quantity of
electricity will slowly ooze out through the circuit, the current
becoming feebler and feebler as the charge is more nearly
exhausted.

\Subsection{Mechanical Illustration of the Properties of a Dielectric.}
\Runhead{MECHANICAL ANALOGIES.}

\article{135*} Five tubes of equal sectional area \(A\), \(B\), \(C\), \(D\) and \(P\) are
arranged in circuit as in the figure. \(A\), \(B\), \(C\) and \(D\) are vertical
and equal, and \(P\) is horizontal.

The lower halves of \(A\), \(B\), \(C\), \(D\)
are filled with mercury, their upper
halves and the horizontal tube \(P\) are
filled with water.

A tube with a stopcock \(Q\) connects
the lower part of \(A\) and \(B\)
with that of \(C\) and \(D\), and a piston
\(P\) is made to slide in the horizontal
tube.

Let us begin by supposing that
the level of the mercury in the four
tubes is the same, and that it is indicated
by \(A_0\), \(B_0\), \(C_0\), \(D_0\), that the
piston is at \(P_0\), and that the stopcock
\(Q\) is shut.

\wrapfig{0.45}{129.png}{Fig. 29.}
Now let the piston be moved from
\(P_0\) to \(P_1\), a distance \(a\). Then, since
the sections of all the tubes are equal, the level of the mercury
in \(A\) and \(C\) will rise a distance \(a\), or to \(A_1\) and \(C_1\), and the mercury
in \(B\) and \(D\) will sink an equal distance \(a\), or to \(B_1\) and \(D_1\).

The difference of pressure on the two sides of the piston will
be represented by \(4a\).

This arrangement may serve to represent the state of a dielectric
acted on by an electromotive force \(4a\).

The excess of water in the tube \(D\) may be taken to represent a
positive charge of electricity on one side of the dielectric, and the
%%-----File: 130.png-----%%
excess of mercury in the tube \(A\) may represent the negative charge
on the other side. The excess of pressure in the tube \(P\) on the
side of the piston next \(D\) will then represent the excess of potential
on the positive side of the dielectric.

If the piston is free to move it will move back to \(P_0\) and be in
equilibrium there. This represents the complete discharge of the
dielectric.

During the discharge there is reversed motion of the liquids
throughout the whole tube, and this represents that change of
electric displacement which we have supposed to take place in a
dielectric.

I have supposed every part of the system of tubes filled with
incompressible liquids, in order to represent the property of all
electric displacement that there is no real accumulation of electricity
at any place.

Let us now consider the effect of opening the stopcock \(Q\) while
the piston \(P\) is at \(P_1\).

The level of \(A_1\) and \(D_1\) will remain unchanged, but that of \(B\) and
\(C\) will become the same, and will coincide with \(B_0\) and \(C_0\).

The opening of the stopcock \(Q\) corresponds to the existence of
a part of the dielectric which has a slight conducting power, but
which does not extend through the whole dielectric so as to form
an open channel.

The charges on the opposite sides of the dielectric remain insulated,
but their difference of potential diminishes.

In fact, the difference of pressure on the two sides of the piston
sinks from \(4a\) to \(2a\) during the passage of the fluid through \(Q\).

If we now shut the stopcock \(Q\) and allow the piston \(P\) to move
freely, it will come to equilibrium at a point \(P_2\), and the discharge
will be apparently only half of the charge.

The level of the mercury in \(A\) and \(B\) will be \(\tstrut\frac{1}{2}a\) above its
original level, and the level in the tubes \(C\) and \(D\) will be \(\tstrut\frac{1}{2}a\)
below its original level. This is indicated by the levels \(A_2\), \(B_2\),
\(C_2\), \(D_2\).

If the piston is now fixed and the stopcock opened, mercury will
flow from \(B\) to \(C\) till the level in the two tubes is again at \(B_0\) and
\(C_0\). There will then be a difference of pressure \(= a\) on the two
sides of the piston \(P\). If the stopcock is then closed and the piston
\(P\) left free to move, it will again come to equilibrium at a point \(P_3\),
half way between \(P_2\) and \(P_0\). This corresponds to the residual
charge which is observed when a charged dielectric is first discharged
%%-----File: 131.png-----%%
and then left to itself. It gradually recovers part of its
charge, and if this is again discharged a third charge is formed, the
successive charges diminishing in quantity. In the case of the
illustrative experiment each charge is half of the preceding, and the
discharges, which are \(\tstrut\frac{1}{2}\), \(\tstrut\frac{1}{4}\), \&c.\ of the original charge, form a series
whose sum is equal to the original charge.
\Runhead{RESIDUAL DISCHARGE.}

If, instead of opening and closing the stopcock, we had allowed it
to remain nearly, but not quite, closed during the whole experiment,
we should have had a case resembling that of the electrification of a
dielectric which is a perfect insulator and yet exhibits the phenomenon
called `electric absorption.'

To represent the case in which there is true conduction through
the dielectric we must either make the piston leaky, or we must
establish a communication between the top of the tube \(A\) and the
top of the tube \(D\).

In this way we may construct a mechanical illustration of the
properties of a dielectric of any kind, in which the two electricities
are represented by two real fluids, and the electric potential is
represented by fluid pressure. Charge and discharge are represented
by the motion of the piston \(P\), and electromotive force by
the resultant force on the piston.

\article{136} The electric strength of a dielectric medium depends on the
nature of the medium and its density and temperature. Thus the
electromotive force required to produce a disruptive discharge is
greater in glass or ebonite than in air.
\Runhead{ELECTRIC STRENGTH OF AIR.}

The electric strength of air or any other gas may be tested by
causing sparks to pass through a portion of the gas between two
balls of metal. If the experiment is conducted in a glass vessel
from which the air may be exhausted by an air pump, it is found
that the electromotive force necessary to produce the discharge
diminishes, while the pressure is reduced from that of the atmosphere
to that of about 3 millimetres of mercury. If the supply of
electricity is kept up at a constant rate, the sparks become smaller
and more frequent, till at last there appears to be a continuous flow.
If, however, the exhaustion be carried further, the electric strength
again increases, till in the most perfect vacuum hitherto made the
electromotive force required to produce a spark between electrodes
·6 centimetres apart is so great that the discharge does not take
place between the electrodes, but passes round the outside of the
vessel through a distance of 20 centimetres of air at the ordinary
pressure. It would therefore seem as if a perfect vacuum would
%%-----File: 132.png-----%%
present an almost insuperable resistance to the passage of electricity.
A small quantity of gas, however, introduced into the empty space
renders it incapable of withstanding even a small electromotive
force. This diminution of the electric strength, however, does not
go on when the density of the gas is still further increased, but for
pressures of a centimetre and upwards the electric strength increases
as the density increases.

\article{137} The electric strength of air diminishes rapidly as the temperature
rises. The heated air which rises from a flame conducts
electricity freely. The best way of discharging the electrification
of the surface of a solid dielectric is to pass the electrified body
over a flame. In most experiments with heated air the air is in
motion. It is therefore desirable that experiments should be made
on the conductivity of air at various temperatures, contained in a
closed vessel and free from currents.

\article{138} In order to test the insulating properties of air and other
gases I made the following experiment:---

\widefig{0.89}{132.png}{Fig. 30.}
A tube half an inch in diameter, \(CD\), is supported on an insulated
stand \(c\). A rod \(AB\), a quarter of an inch in diameter, is supported
by the insulating stand \(a\) so that about 6 inches of the rod is within
the tube with a cylindrical shell of air about an eighth of an inch
thick between it and the inside of the tube. The tube is connected
with one electrode of a battery of 50 Leclanché cells, the other
electrode being connected to earth. The rod is connected to one
electrode of Thomson's quadrant electrometer, the other electrode
being connected to earth. A tube, \(F\), which is fixed so as not to
touch the tube \(CD\), is used for sending a current of hot air or steam
through the tube \(CD\). The part of the tube \(CD\) which contains the
rod \(AB\) is surrounded by a wider tube \(E\) of thick brass which may
be heated by a gas furnace so as to keep the inner tube and rod hot
%%-----File: 133.png-----%%
without exposing them to the current of the products of combustion
of the burner.

The sensitiveness of this apparatus was shewn by the effect of
communicating a small charge to the tube \(E\). The electrometer
was immediately deflected on account of induction between the
tube and the rod \(AB\). The rod \(AB\) was then discharged to earth
so that the electrometer indicated zero, the tube remaining at a
higher potential. If any conduction were now to take place
through the air between the tube and the rod it would be indicated
by the electrometer. No conduction however could be observed
even after the lapse of a quarter of an hour, and when hot air and
steam were blown through the tube. At the end of the experiment
the tube was discharged to earth, when a negative deflection
of the electrometer was observed, shewing that the tube had remained
charged during the whole experiment.

\article{139} Other experiments were afterwards made in which mercury
and sodium were made to boil in a bent glass tube while raised to
a high potential by a battery of 50 Leclanché cells. A thick
copper wire (Fig. 31) was placed on an insulating stand so that the
end of the wire was within the glass tube and surrounded by the
vapour of the metal. It was necessary that the wire should not
be allowed to touch the tube, because glass at a high temperature
is a good conductor. It was also necessary to see that the
products of combustion from the Bunsen burner did not come
in contact with the wire after becoming electrified by the hot
tube.
\Runhead{CONDUCTIVITY OF GASES.}

\widefig{0.69}{133.png}{Fig. 31.}
The wire was connected with the electrometer, but no evidence of
conduction of electricity could be observed, even when the mercury
was boiling briskly, and its vapour was being condensed on the
%%-----File: 134.png-----%%
wire. But whenever so much mercury had collected on the wire
that a drop fell off at the end of the wire, there was a deflection
of the electrometer because the drop had become charged by induction
from the tube and the removal of this charge affected
the electrometer. This however was no evidence of conduction
through the metallic vapour, but only indicated that the apparatus
was in such a state of electrification that any conduction, if it took
place, would produce a sensible indication at the electrometer.

It is difficult to reconcile these experiments on the insulating
power of hot gases and vapours with the well-known phenomena
of the communication of electricity along the stream of heated
matter rising from a flame or even from red-hot metal. This
stream acts as a powerful conductor of electricity between the flame
and bodies placed at a foot or a yard above it where the temperature
of the ascending current is much lower than it was in the experiment
of the tube and rod.

\article{140} The whole theory of the electric properties of gases is in a
very imperfect state. According to the kinetic theory of gases,
their molecules are in a state of agitation so that they are continually
striking against each other. The velocity of this agitation
is greater the higher the temperature. It would appear, therefore,
that the electric conduction of gases is of the nature of convection.
At every collision the whole charge of two of the molecules would
be equally divided between them, and thus the tendency of the
agitation would be to equalize the charges of all the molecules.

But we can hardly admit a theory of this kind when we consider
that we have hitherto obtained no evidence of the conduction of
electricity through air at the ordinary pressure and temperature
under a feeble electromotive force.

Whenever a body free from projecting points and sharp edges
and charged to a low potential is found to lose its charge, the
result can always be traced to conduction through the substance
or along the surface of the apparatus which is required to support
it. The more perfectly insulating we make this apparatus the
more slowly does the electrified body lose its charge, so that it
is probable that if we could support an electrified body on a perfectly
insulating stand so that it could lose its charge only by
conduction through the air, it would never lose its charge.
%%-----File: 135.png-----%%

\Subsection{Electric Phenomena of Tourmaline.}

\article{141} Certain crystals of tourmaline and of other minerals
possess what may be called Electric Polarity. Suppose a crystal
of tourmaline to be at a uniform temperature, and apparently
free from electrification on its surface. Let its temperature be
now raised, the crystal remaining insulated. One end will be
found positively and the other end negatively electrified. Let the
surface be deprived of this apparent electrification by means of a
flame or otherwise; then if the crystal be made still hotter, electrification
of the same kind as before will appear, but if the crystal be
cooled the end which was positive when the crystal was heated will
become negative.

These electrifications are observed at the extremities of the crystallographic
axis. Some crystals are terminated by a six-sided
pyramid at one end and by a three-sided pyramid at the other.
In these the end having the six-sided pyramid becomes positive
when the crystal is heated.

Sir W. Thomson supposes every portion of these and other hemihedral
crystals to have a definite electric polarity, the intensity
of which depends on the temperature. When the surface is passed
through a flame, every part of the surface becomes electrified to
such an extent as to exactly neutralize, for all external points,
the effect of the internal polarity. The crystal then has no external
electrical action, nor any tendency to change its mode of
electrification. But if it be heated or cooled the interior polarization
of each particle of the crystal is altered, and can no longer
be balanced by the superficial electrification, so that there is a
resultant external action.
\Runhead{PYRO-ELECTRIC PHENOMENA.}

In tourmaline and other pyroelectric crystals it is probable
that a state of electric polarization exists, which depends upon
temperature, and does not require an external electromotive force
to produce it. If the interior of a body were in a state of
permanent electric polarisation, the outside would gradually become
charged in such a manner as to neutralize the action of the internal
electrification for all points outside the body. This external superficial
charge could not be detected by any of the ordinary tests,
and could not be removed by any of the ordinary methods for
discharging superficial electrification. The internal polarization
of the substance would therefore never be discovered unless by
some means, such as change of temperature, the amount of the
%%-----File: 136.png-----%%
internal polarization could be increased or diminished. The external
electrification would then be no longer capable of neutralizing
the external effect of the internal polarization, and an apparent
electrification would be observed, as in the case of tourmaline.

\Subsection{The Electric Glow.}
\Runhead{THE ELECTRIC GLOW.}

\article{142} It can be proved by the mathematical theory of electricity
that if a conductor having on its surface a sharp conical point
is placed in a perfectly insulating medium and electrified, the
surface-density of the electricity will increase without limit for
points nearer and nearer to the conical point, so that at the conical
point itself the surface-density, and therefore the electromotive
force, would be infinite. But this result depends on the hypothesis
that the air or other surrounding dielectric has an invincible
insulating power, which is not the case, and therefore as soon
as the electromotive force at the conical point reaches a certain
limiting value the insulating power of the air gives way, and
there is a disruptive discharge of electricity into the air. A small
portion of air close to the conical point thus becomes electrified.
The electrified system now consists of the metal conductor with
its conical point, together with a rounded mass of electrified air,
which covers the point and acts as a sort of sheath to it, so that
the boundary of the electrified system is no longer pointed.

This electrified system, if it were solid, would retain its charge,
for the electromotive force is not great enough at any place to
produce disruptive discharge, but since the air is fluid, and since
the electromotive force is greatest in the line of prolongation of
the conical point, the electrified particles of air move off in that
direction. When they are gone other unelectrified particles take
their place round the point, and the point being no longer
protected by electrified air, there is another discharge, as at first.

Thus there is continually kept up an influx of uncharged air
to the point, a luminous discharge of electricity from the point,
called the Electric Glow, and a stream of charged air in the
direction of the prolongation of the axis of the cone called the
Electric Wind. By checking the influx of air behind the point
we may weaken the glow and by increasing the current of air by
blowing we may make the glow stronger.

\article{143} The electric wind which blows from the conical point
may be made to drive a little windmill, or if the conductor is
%%-----File: 137.png-----%%
made of two wires crossed and having their sharpened ends bent
backwards, as in Fig. 32, and supported so as to be capable of
rotating, the reaction of the electric
wind will make the cross rotate in
the direction of the arrows.
\Runhead{THE ELECTRIC WIND.}

\phantomsection\label{art:143a}
\wrapfig{0.4}{137.png}{Fig. 32.}
It is only close to the electrified
point that the motion of the electrified
air is in any degree influenced
by its electrification. At a short distance
from the point the electrified
air becomes mixed with other air, and
is carried about by the ordinary currents of the atmosphere as an
invisible electric cloud.

\phantomsection\label{art:143b}
If we calculate the force due to the electrification of a large
body of air at a considerable distance from other electrified bodies,
we shall find that it is not capable of producing effects on the
motion of so large a mass which are at all comparable to the
effects of the slight variations of density and other causes which
produce the movements of the atmosphere. Hence the motion of
thunder clouds is due almost entirely to atmospheric currents and
is not sensibly affected by their electrification.

\article{144} When an electrified portion of air comes near the surface
of a conductor, it induces on that surface an electrification opposite
to its own and is attracted towards the surface, but since the
electromotive force is small the electrified particles may remain for
a long time in the neighbourhood of the conductor without being
drawn into contact with it and discharged.

To detect the presence of this electrified atmosphere clinging
to a conductor we have only to insulate the conductor and connect
it with an electrometer. If we now blow away the electrified air
from its surface, the electrometer will indicate the electrification
of the conductor itself, which is of course of the opposite kind
to that of the electrified air. Thus, if we hold in the hand a
hollow metal cylinder over an electrified point, we may electrify
the air within it. If we then place it on an insulated stand in
connexion with the electrometer, the electrometer will remain at
zero till the electrified air is removed, which may be done by
blowing air through the cylinder. The electrometer will then
indicate the electrification of the cylinder, which is of the opposite
kind from that of the electrified air.

\article{145} The glow is more easily formed in rare air than in dense
%%-----File: 138.png-----%%
air, and more easily when the point is positive than when it is
negative. This and many other differences between positive and
negative electrification seem to depend upon a condition analogous
to electrolytic polarization in the stratum of air in contact with
the electrode. It appears that the electromotive force required
to cause an electric discharge to take place is somewhat smaller
where the electrode at which the discharge begins is negative, but
that the quantity of electricity in each discharge is greatest when
the electrode at which the discharge begins is positive.

\article{146} A fine point may be used instead of a proof plane to determine
the nature of the electrification of any part of the surface of
a conductor when electricity is induced upon it in presence of
another electrified body. For this purpose the point is fixed to the
conductor so as to project a few millimetres from its surface. If
the part of the surface to which it is fixed is electrified positively
the point gives off positive electricity to the air, and the conductor
loses positive electricity or gains negative electricity. This may
be ascertained either by removing or discharging the inductor and
ascertaining the character of the charge of the induced body, or by
connecting the induced body with the electrometer and observing
the change of potential as the point throws off its electricity.
\Runhead{ACTION OF POINTS.}

It has been found that some vegetable thorns, prickles, or spines
act more perfectly in throwing off electricity than the finest pointed
needles which can be procured.

The action of the point may be assisted by blowing air from
a blowpipe over the point, and in this way we may prevent the
electrified air from discharging itself on the surface of the inductor.

\Subsection{The Electric Brush.}

\article{147} The electric brush is a phenomenon which may be produced
by electrifying a blunt point or a small ball in air so as to
produce an electric field in which the tension diminishes as the
distance from the ball increases, but not so rapidly as in the case of
a sharp point. The brush consists of a succession of discharges,
ramifying as they diverge from the ball into the air, and terminating
either by charging portions of air or by reaching some other
conductor. The brush produces a sound, the pitch of which depends
on the interval between the successive discharges, and there is no
current of air as in the case of the glow.
%%-----File: 139.png-----%%

\Subsection{The Electric Spark.}

\article{148} When the tension in the space between the two electrodes
is considerable all the way between them, as in the case of two balls
whose distance is not very great compared with their radii, the
discharge, when it occurs, usually takes the form of a spark, by
which nearly the whole electrification is discharged at once.
\Runhead{DISRUPTIVE DISCHARGE.}

In this case, when any part of the dielectric has given way, the
part next to it in the direction of the electric force is put into
a state of greater tension, so that it also gives way, and so the
discharge proceeds right through the dielectric. We may compare
this breaking down of the dielectric to what occurs when we make
a little rent perpendicular to the edge of a piece of paper and then
apply tension to the paper in the direction of the edge. The paper
is torn through, the disruption beginning at the little rent, but
diverging occasionally so as to take in weak places in the paper.
The electric spark in the same way begins at the point where the
electric tension first overcomes the `electric strength' of the dielectric,
and proceeds from that point, in an apparently irregular path,
so as to take in other weak points, such as particles of dust floating
in the air.

\article{149} The investigation of the phenomena of the luminous
electric discharge has been greatly assisted by the use of the spectroscope.
The light of the spark or other discharge is made to fall
on the slit of the collimator of the spectroscope, and after being
analysed by the prisms is observed through the telescope. The
light as thus analysed is found to consist of a great number of
bright lines and bands forming what is called the spectrum of the
light. By comparing light from different sources it is found that
these bright lines may be divided into groups, each group being
due to the presence of a particular substance in the medium through
which the discharge takes place.

By using the method introduced by Mr.\ Lockyer of forming
an image of the spark upon the slit by means of a lens, we may
obtain at one view a comparison of the constituents of the medium
which are rendered luminous by the dielectric discharge at the
different points of its path. Close to either electrode the lines are
principally due to the metal or metals of which that electrode
consists. At greater distances these lines become fainter, thinner,
and less numerous, but the spectrum belonging to the gas through
which the discharge takes place remains visible.
\Runhead{SPECTRUM OF THE ELECTRIC SPARK.}
%%-----File: 140.png-----%%

Some of the lines due to the metals appear longer than others,
shewing that they can be formed in regions of the spark where
others are no longer visible, owing either to a deficiency in the
amount of the metallic vapour or to a want of vigour in the electric
disturbance. It thus appears that the electric discharge separates
an appreciable amount of matter even from the hardest metals and
carries the particles through the air to a distance of several millimetres
from the surface of the metal. It also appears by a comparison
of sparks from different electrodes and through different
gases that no part of the light is emitted by any substance common
to all the different cases, but that every line is due to one or other
of the chemical substances present.

It follows from this that neither the electric fluid, if there be
such a substance, nor any ethereal medium such as is supposed
to pervade all ordinary matter, is rendered luminous during the
discharge, for if it were so its spectrum would be visible in all
discharges.

\Subsection{On Steady Currents.}

\article{150*} In the case of the current between two insulated conductors
at different potentials the operation is soon brought to an
end by the equalization of the potentials of the two bodies, and the
current is therefore essentially a Transient current.

But there are methods by which the difference of potential of
the conductors may be maintained constant, in which case the
current will continue to flow with uniform strength as a Steady
Current.

\Subsection{The Voltaic Battery.}

The most convenient method of producing a steady current is by
means of the Voltaic Battery.

For the sake of distinctness we shall describe Daniell's Constant
Battery:---

A solution of sulphate of zinc is placed in a cell of porous earthenware,
and this cell is placed in a vessel containing a saturated
solution of sulphate of copper. A piece of zinc is dipped into the
sulphate of zinc, and a piece of copper is dipped into the sulphate
of copper. Wires are soldered to the zinc and to the copper above
the surface of the liquid. This combination is called a cell or
element of Daniell's battery. See \hyperref[art:193*]{Art.\ 193.*}
\Runhead{DANIELL'S BATTERY.}
%%-----File: 141.png-----%%

\article{151*} If the cell is insulated by being placed on a non-conducting
stand, and if the wire connected with the copper is put
in contact with an insulated conductor \(A\), and the wire connected
with the zinc is put in contact with \(B\), another insulated conductor
of the same metal as \(A\), then it may be shewn by means of a delicate
electrometer that the potential of \(A\) exceeds that of \(B\) by a
certain quantity. This difference of potentials is called the Electromotive
Force of Daniell's Cell.

If \(A\) and \(B\) are now disconnected from the cell and put in
communication by means of a wire, a transient current passes
through the wire from \(A\) to \(B\), and the potentials of \(A\) and \(B\)
become equal. \(A\) and \(B\) may then be charged again by the cell,
and the process repeated as long as the cell will work. But if
\(A\) and \(B\) be connected by means of the wire \(C\), and at the same
time connected with the battery as before, then the cell will maintain
a constant current through \(C\), and also a constant difference
of potentials between \(A\) and \(B\). This difference will not, as we
shall see, be equal to the whole electromotive force of the cell, for
part of this force is spent in maintaining the current through the
cell itself.

A number of cells placed in series so that the zinc of the first
cell is connected by metal with the copper of the second, and so
on, is called a Voltaic Battery. The electromotive force of such a
battery is the sum of the electromotive forces of the cells of which
it is composed. If the battery is insulated it may be charged with
electricity as a whole, but the potential of the copper end will
always exceed that of the zinc end by the electromotive force of
the battery, whatever the absolute value of either of these potentials
may be. The cells of the battery may be of very various construction,
containing different chemical substances and different metals,
provided they are such that chemical action does not go on when
no current passes.

\article{152*} Let us now consider a voltaic battery with its ends insulated
from each other. The copper end will be positively or
vitreously electrified, and the zinc end will be negatively or resinously
electrified.

Let the two ends of the battery be now connected by means
of a wire. An electric current will commence, and will in a very
short time attain a constant value. It is then said to be a Steady
Current.
%%-----File: 142.png-----%%

\Subsection{Magnetic Action of the Current.}

\article{153*} Oersted discovered that a magnet placed near a straight
electric current tends to place itself at right angles to the plane
passing through the magnet and the current.
\Runhead{OERSTED'S DISCOVERY.}

If a man were to place his body in the line of the current so
that the current from copper through the wire to zinc should flow
from his head to his feet, and if he were to direct his face towards
the centre of the magnet, then that end of the magnet which tends
to point to the north would, when the current flows, tend to point
towards the man's right hand. Thus we see that the electric
current has a magnetic action which is exerted outside the current,
and by which its existence can be ascertained and its intensity
measured without breaking the circuit or introducing anything
into the current itself.

The amount of the magnetic action has been ascertained to be
strictly proportional to the strength of the current as measured
by the products of electrolysis in the voltameter, and to be quite
independent of the nature of the conductor in which the current
is flowing, whether it be a metal or an electrolyte.

\article{154*} An instrument which indicates the strength of an electric
current by its magnetic effects is called a Galvanometer.

Galvanometers in general consist of one or more coils of silk-covered
wire within which a magnet is suspended with its axis
horizontal. When a current is passed through the wire the magnet
tends to set itself with its axis perpendicular to the plane of the
coils. If we suppose the plane of the coils to be placed parallel
to the plane of the earth's equator, and the current to flow round
the coil from east to west in the direction of the apparent motion
of the sun, then the magnet within will tend to set itself with
its magnetization in the same direction as that of the earth considered
as a great magnet, the north pole of the earth being similar
to that end of the compass needle which points south.

The galvanometer is the most convenient instrument for measuring
the strength of electric currents. We shall therefore assume
the possibility of constructing such an instrument in studying the
laws of these currents, and when we say that an electric current is
of a certain strength we suppose that the measurement is effected
by the galvanometer.
%%-----File: 143.png-----%%

\Subsection{On Systems of Linear Conductors.}

\article{155*} Any conductor may be treated as a linear conductor if it
is arranged so that the current must always pass in the same manner
between two portions of its surface which are called its electrodes.
For instance, a mass of metal of any form, the surface of which is
entirely covered with insulating material except at two places, at
which the exposed surface of the conductor is in metallic contact
with electrodes formed of a perfectly conducting material, may be
treated as a linear conductor. For if the current be made to enter
at one of these electrodes and escape at the other the lines of flow
will be determinate, and the relation between electromotive force,
current, and resistance will be expressed by Ohm's Law, for the
current in every part of the mass will be a linear function of \(E\).
But if there be more possible electrodes than two, the conductor
may have more than one independent current through it.

\Subsection{Ohm's Law.}
\Runhead{OHM'S LAW.}

\article{156*} Let \(E\) be the electromotive force in a linear conductor
from the electrode \(A_1\) to the electrode \(A_2\). (See \hyperref[art:5]{Art.\ 5}.) Let
\(C\) be the strength of the electric current along the conductor, that
is to say, let \(C\) units of electricity pass across every section in
the direction \(A_1\) \(A_2\) in unit of time, and let \(R\) be the resistance of
the conductor, then the expression of Ohm's Law is
\[
E = CR \text{.}\tag{1}
\]

The Resistance of a conductor is defined to be the ratio of
the electromotive force to the strength of the current which it
produces. The introduction of this term would have been of no
scientific value unless Ohm had shewn, as he did experimentally,
that it corresponds to a real physical quantity, that is, that it has
a definite value which is altered only when the nature of the conductor
is altered.

In the first place, then, the resistance of a conductor is independent
of the strength of the current flowing through it.

In the second place the resistance is independent of the electric
potential at which the conductor is maintained, and of the density
of the distribution of electricity on the surface of the conductor.

It depends entirely on the nature of the material of which the
conductor is composed, the state of aggregation of its parts and its
temperature.
%%-----File: 144.png-----%%

The resistance of a conductor may be measured to within one
ten thousandth or even one hundred thousandth part of its value,
and so many conductors have been tested that our assurance of the
truth of Ohm's Law is now very high\footnote{
[See Report of British Association, 1876.]}.

\Subsection{Linear Conductors arranged in Series.}

\article{157*} Let \(A_1\), \(A_2\) be the electrodes of the first conductor and let
the second conductor be placed with one of its electrodes in contact
with \(A_2\), so that the second conductor has for its electrodes \(A_2\), \(A_3\).
The electrodes of the third conductor may be denoted by \(A_3\)
and \(A_4\).
\Runhead{RESISTANCE OF CONDUCTORS IN SERIES.}

Let the electromotive force along each of these conductors be
denoted by \(E_{12}\), \(E_{23}\), \(E_{34}\), and so on for the other conductors.

Let the resistance of the conductors be
\[
R_{12}\text{,} \quad R_{23}\text{,} \quad R_{34}\text{,} \quad \text{\&c.}
\]
Then, since the conductors are arranged in series so that the same
current \(C\) flows through each, we have by Ohm's Law,
\[
E_{12} = CR_{12}\text{,} \quad E_{23} = CR_{23}\text{,} \quad E_{34} = CR_{34}\text{.} \tag{2} \]

If \(E\) is the resultant electromotive force, and \(R\) the resultant
resistance of the system, we must have by Ohm's Law,
\[
E = CR \text{.}\tag{3}
\]

Now
\begin{align*}
E &= E_{12} + E_{23} + E_{34}\text{,} \tag{4}
\shortintertext{\hfill the sum of the separate electromotive forces,}
&= C(R_{12} + R_{23} + R_{34}) \text {  by equations (2).}
\end{align*}

Comparing this result with (3), we find
\[
R = R_{12} + R_{23} + R_{34} \tag{5}
\]
Or, \textit{the resistance of a series of conductors is the sum of the resistances
of the conductors taken separately}.

\Subsection{Potential at any Point of the Series.}

Let \(A\) and \(C\) be the electrodes of the series, \(B\) a point between
them, \(a\), \(c\), and \(b\) the potentials of these points respectively. Let
\(R_1\) be the resistance of the part from \(A\) to \(B\), \(R_2\) that of the part
from \(B\) to \(C\), and \(R\) that of the whole from \(A\) to \(C\), then, since
\[
a - b = R_1C\text{,} \quad b - c = R_2C\text{,}\quad \text{and} \quad a - c = RC\text{,}
\]
the potential at \(B\) is
\[
b = \frac {R_2a + R_1c}{R}\text{,} \tag{6}
\]
%%-----File: 145.png-----%%
which determines the potential at \(B\) when those at \(A\) and \(C\) are
given.

\Subsection{Resistance of a Multiple Conductor.}

\article{158*} Let a number of conductors \(ABZ\), \(ACZ\), \(ADZ\) be arranged
side by side with their extremities in contact with the same two
points \(A\) and \(Z\). They are then said to be arranged in multiple
arc.

Let the resistances of these conductors be \(R_1\), \(R_2\), \(R_3\) respectively,
and the currents \(C_1\), \(C_2\), \(C_3\), and let the resistance of the
multiple conductor be \(R\), and the total current \(C\). Then, since the
potentials at \(A\) and \(Z\) are the same for all the conductors, they have
the same difference, which we may call \(E\). We then have
\[
E = C_1R_1 = C_2R_2 = C_3R_3 = CR\text{,}
\]
but
\[
C = C_1 + C_2 + C_3\text{,}
\]
whence
\[
\frac{1}{R} = \frac{1}{R_1} + \frac{1}{R_2} + \frac{1}{R_3}\text{.} \tag{7}
\]
Or, \textit{the reciprocal of the resistance of a multiple conductor is the sum
of the reciprocals of the component conductors}.

If we call the reciprocal of the resistance of a conductor the
conductivity of the conductor, then we may say that \textit{the conductivity
of a multiple conductor is the sum of the conductivities of
the component conductors}.
\Runhead{RESISTANCE OF CONDUCTORS IN MULTIPLE ARC.}

\Subsection{Current in any Branch of a Multiple Conductor.}

From the equations of the preceding article, it appears that if
\(C_1\) is the current in any branch of the multiple conductor, and
\(R_1\) the resistance of that branch,
\[
C_1 = C\frac{R}{R_1}\text{,} \tag{8}
\]
where \(C\) is the total current, and \(R\) is the resistance of the multiple
conductor as previously determined.

\phantomsection\label{art:158a}
Kirchhoff has stated the conditions of a linear system in the
following manner, in which the consideration of the potential is
avoided.

(1) (Condition of `continuity.') At any point of the system the
sum of all the currents which flow towards that point is zero.

(2) In any complete circuit formed by the conductors the sum
of the electromotive forces taken round the circuit is equal to the
%%-----File: 146.png-----%%
sum of the products of the currents in each conductor multiplied
by the resistance of that conductor.

\Subsection{Longitudinal Resistance of Conductors of Uniform Section.}

\article{159*} Let the resistance of a cube of a given material to a
current parallel to one of its edges be \(\rho\), the side of the cube being
unit of length, \(\rho\) is called the `specific resistance of that material
for unit of volume.'
\Runhead{RESISTANCE OF A WIRE.}

Consider next a prismatic conductor of the same material whose
length is \(l\), and whose section is unity. This is equivalent to \(l\)
cubes arranged in series. The resistance of the conductor is therefore
\(l \rho\).

Finally, consider a conductor of length \(l\) and uniform section \(s\).
This is equivalent to \(s\) conductors similar to the last arranged in
multiple arc. The resistance of this conductor is therefore
\[
R = \frac{l\rho}{s}\text{.}
\]

When we know the resistance of a uniform wire we can determine
the specific resistance of the material of which it is made if we can
measure its length and its section.

The sectional area of small wires is most accurately determined
by calculation from the length, weight, and specific gravity of the
specimen. The determination of the specific gravity is sometimes
inconvenient, and in such cases the resistance of a wire of unit
length and unit mass is used as the `specific resistance per unit of
weight.'

If \(r\) is this resistance, \(l\) the length, and \(m\) the mass of a wire, then
\[
R = \frac{l^2r}{m}\text{.}
\]
%%-----File: 147.png-----%%

\newchapter
\Chapter{CHAPTER X.}

\Subheading{PHENOMENA OF AN ELECTRIC CURRENT WHICH FLOWS
THROUGH HETEROGENEOUS MEDIA.}

\Subsection{\textup{1}. Thermo-electric phenomena.}

\article{160} Seebeck, in 1822, discovered that if a circuit is formed of
two different metals, and if the two junctions of the metals are
kept at different temperatures, an electric current tends to flow
round the circuit. If the metals are iron and copper at temperatures
below 280°C, the current flows from copper to iron through
the hotter junction. There is therefore, in general, an electromotive
force acting in a definite direction round the circuit, whenever
the two junctions are at different temperatures.

In a circuit formed of any number of metals all at the same
temperature, there can be no current, for if there were a current
it might be constantly employed to work a machine or to generate
heat in a conductor, and this without any energy being supplied
to the system from without, for in order to keep the circuit at
a constant temperature nothing is required except to prevent heat
from entering or leaving it. Hence at any given temperature
the electromotive force in a circuit of three metals, \(A\), \(B\), \(C\) must
be zero for the whole circuit. Hence if the electromotive force
from \(C\) to \(A\) is \(a\), and that from \(C\) to \(B\) is \(b\), and that from \(B\) to \(A\)
\(x\), then in the circuit \(A\), \(B\), \(C\), the total electromotive force is
\(a - b - x = 0\), so that \(x\), the electromotive force from \(B\) to \(A\) is
represented by \(a - b\), where \(a\) and \(b\) are quantities determined by
observation of the electromotive force from any third metal \(C\)
to the metals \(A\) and \(B\). We may express this by saying that
the quantities \(a\) and \(b\) are the \textit{potentials} of the metals \(A\) and \(B\)
with respect to a third metal \(C\) at the given temperature. The
potential of \(A\) with respect to \(B\) is \(a - b\). The actual determination
of the relative potentials of the metals will be explained in
\hyperref[art:182]{Art.\ 182}.
%%-----File: 148.png-----%%

\widefig{0.76}{148.png}{Fig. 33.}
\article{161} It has been shewn by Magnus\footnote{
[\textit{Pogg.\ Ann.}\ 1851.]} that if a circuit be formed
of a single metal, no current will be formed in it, however the
temperature and the section of the conducting circuit may vary
in different parts. Since in this case there is necessarily conduction
of heat, and consequently dissipation of energy, we cannot,
as in the former case, consider the result as self-evident. The
electromotive force, for instance, between two portions of the
circuit at given temperatures might depend on the length or
the mode of variation of the section of the intermediate portion
of the circuit. In fact the experiments of Le Roux and others
have shewn that the law of Magnus is no longer applicable in
a circuit in which there is a very abrupt variation of temperature,
as at the instant when the circuit is closed by a hot wire coming
in contact with a cold wire of the same metal. Even without
any physical discontinuity of the circuit such as is implied in
the contact of two separate pieces of wire, a sufficiently abrupt
variation of temperature may be produced by taking a thick
wire and filing down a certain length of it till it is very thin.
If the junction of the thick and the thin portions is placed in
a flame, the thin portion will be heated so much more rapidly
than the thick portion, that the variation of temperature will be
so abrupt that the law of Magnus fails, and we obtain a current
in a circuit of one metal; we must therefore modify the statement
of the law of Magnus as follows:---
\Runhead{LAW OF MAGNUS.}

\textit{The electromotive force from one point of a conductor of homogeneous
metal to another depends only on the temperature of these points unless
at any part of the conductor a sensible variation of temperature occurs
between points whose distance is within the limits of molecular action.}

\Subsection{Thermo-electric power of a metal at a given temperature.}

\article{162} Let us now consider a linear circuit made up of alternate
pieces of two metals, say lead and iron. We shall assume lead
to be the standard metal, and study the properties of iron in
relation to lead.

\widefig{0.95}{149.png}{Fig. 34.}
In the figure the pieces of iron are distinguished by shading.
Let the temperatures of the junctions be those indicated in the
%%-----File: 149.png-----%%
figure, in which the temperatures of the extremities of each piece
of iron differ by one degree, but the temperatures of the extremities
of each of the intermediate pieces of lead are equal. The total
electromotive force round the circuit is the sum of the electromotive
forces due to the thermo-electric action of the different pairs of
junctions. Now if we consider the pairs \(A\) and \(B\), \(C\) and \(D\), \(E\)
and \(F\) belonging to the pieces of iron we find that the temperature
rises one degree in each piece, but if we take the pairs \(B\) and \(C\),
\(D\) and \(E\) belonging to the pieces of lead, the temperature in
each piece is uniform and therefore there is no electromotive force
in these pieces. We may therefore leave the intermediate pieces
of lead out of account, and consider the electromotive force due
to the junctions \(A\) and \(F\) as equivalent to the sum of the electromotive
forces of the three pairs of junctions \(A\) and \(B\), \(C\) and \(D\),
\(E\) and \(F\).

Hence if a diagram is constructed in which the axis \(OZ\) is
marked with the degrees of the thermometric scale and in which
the area \(0° PQ 1°\) represents the electromotive force when the
junctions are at 0° and 1° and so on, then the electromotive force
when the junctions are at any given temperatures will be represented
by the area included between the axis, the ordinates at
the given temperatures and the line \(PQRST\).
\Runhead{THERMOELECTRIC POWER.}

\article{163} Any ordinate such as \( 0°P\), \(1°Q\), \&c., is called the Thermoelectric
Power of iron with respect to lead at \(0°\), \(1°\), \&c., and is
%%-----File: 150.png-----%%
reckoned positive when, for a small difference of temperature, the
current is from lead to iron through the hot junction.

\begin{figure}[H]
\centering
\includegraphics[width=0.74\textwidth]{149_1.png}
\caption*{\small Fig. 35.}
\end{figure}
We may also on the same diagram construct other lines, the
ordinates of which represent the thermo-electric powers of any
other metals with respect to lead, being reckoned positive and
measured upwards when for a small difference of temperatures
the current sets from lead to that metal through the hot junction.
Such a diagram is called a thermo-electric diagram, and from it
we can deduce the electromotive force due to any pair of metals
with their junctions at any given temperatures.

\wrapfig{0.42}{150.png}{Fig. 36.}
Thus if \(a\,A\) is the line representing the metal \(A\), and \(b\,B\) another
representing the metal \(B\), and \(T\), \(t\) the
temperatures of the junctions, the
electromotive force of the circuit is
represented by the area \(ABbaA\) and
it acts in the direction indicated,
namely, from the metal \(A\) to the
metal \(B\) through the hot junction.

If, instead of lead, we had assumed
any other metal as the standard
metal, the diagram would have been
altered in form, but all areas measured
on the diagram would have remained
the same, the change of form being
due to a shearing strain in which the slipping is along vertical
lines.

\Subsection{Thermo-electric Inversion.}

\article{164} Cumming in 1823 discovered several cases in which the
thermo-electric order of two metals as observed at ordinary temperatures
becomes inverted at high temperatures. The lines corresponding
to these metals on the thermo-electric diagram must
therefore cross one another at some intermediate temperature, called
the Neutral Temperature for these metals.

Tait has recently investigated the lines which represent a considerable
number of metals in the thermo-electric diagram, and he
finds that for most metals they are nearly if not exactly straight
lines. The lines for iron and nickel however have considerable
sinuosities, so that they may intersect the straight lines belonging
to another metal in several different points corresponding to several
different neutral temperatures.
\Runhead{THERMO-ELECTRIC INVERSION.}
%%-----File: 151.png-----%%

\Subsection{Thermal effects of the Current.}

\article{165} By applying the principle of the conservation of energy to
the case of a thermo-electric current, it is easy to see that certain
thermal effects must accompany the electric current.

Let us consider what takes place while one unit of electricity
is transmitted across any section of the circuit. The work done
on the electric current is the product of the electromotive force
into the quantity of electricity transmitted, and since this latter
quantity is unity, the work is numerically equal to the electromotive
force, and is represented by the area \(ABba\) in the thermo-electric
diagram. If the current is allowed to flow without
anything to impede it except the resistance of the circuit, the
whole of the work will be converted into heat, but if the resistance
of any part of the circuit such as a long and fine wire greatly
exceeds that of the thermo-electric couple, the heat generated in
that part of the circuit will greatly exceed that generated in the
thermo-electric couple itself. Instead of allowing the current to
generate heat, we may make it drive a magneto-electric engine,
and so convert any given proportion of the work into mechanical
work.

Thus for every unit of electricity which is transmitted, a certain
amount of work is done by the thermo-electric forces on the current.
The only source of this work is the heat of the thermo-electric
couple, and therefore, by the principle of the conservation of energy,
we conclude that an amount of heat, dynamically equivalent to this
work, must have disappeared in some part of the circuit.

\article{166} Now Peltier\footnote{
Annales de Chimie et de Physique, lvi.\ p.\ 371 (1834).} in 1834 found that when an electric current
is made to pass from one metal to another which has a higher
thermo-electric power, the junction is cooled, or, since there is no
permanent change in the metals, there is a disappearance of heat.
When the current is made to flow in the opposite direction the
junction is heated, indicating a generation of heat.
\Runhead{PELTIER EFFECT.}

This thermal effect of the current at the junction is of quite
a different kind from the ordinary generation of heat by the current
while it overcomes the resistance of a conductor. The latter,
which we may call with Thomson the \textit{frictional} generation of heat,
is the same when the direction of the current is reversed, and
varies as the square of the strength of the current. The former
%%-----File: 152.png-----%%
which we shall call the Peltier effect, is reversed when the current
is reversed, and depends simply on the strength of the current.

\article{167} But Thomson has shewn that besides the Peltier effect,
there must in
\wrapfig{0.55}{152.png}{Fig. 37.}
certain metals be another reversible thermal effect
of the current. The current must generate or absorb heat when
it passes from hotter to colder or from colder to hotter parts of
the same metal. Thus, let a thermo-electric couple of copper and
iron be kept with one junction \(AB\) at the neutral temperature
which is about 280°C, and the other, \(ab\), at some lower temperature.
The thermo-electric current is from copper to iron at the hot
junction \(AB\) and from iron to copper at the cold junction \(ab\).

Now the Peltier effect at the hot junction, \(AB\), is zero, for that
junction is at the neutral temperature, and the Peltier effect at the
cold junction, \(ab\), is a generation of heat, for the current is there
passing from the metal of
higher to the metal of lower
thermo-electric power. Hence
the absorption of heat which
must exist in order to account
for the work done by the current
must take place in some
other part of the circuit, either
in the copper where the current
is flowing from cold to hot, or in the iron where it is flowing
from hot to cold, or in both metals. This thermal effect of the
current was predicted by Thomson as the result of reasoning
similar to that here given. He afterwards verified this prediction
experimentally, and found that in iron unequally heated
a current from hot to cold cools the metal, while a current from
cold to hot heats it, and that the reverse thermal effect takes
place in copper. We shall refer to this thermal effect as the
Thomson effect.
\Runhead{THOMSON EFFECT.}

\article{168} Thomson has shewn that a very close analogy subsists between
these thermo-electric phenomena and those of a fluid circulating
in a tube consisting of two vertical branches connected by
two horizontal branches. A fluid, heated in one part of the circuit,
and passing on into cooler parts of the system, will give out heat,
and when it passes from colder to warmer parts will absorb heat,
the amount of heat emitted or absorbed depending on the specific
heat of the fluid. According to this analogy, positive or vitreous
electricity carries heat with it in copper as if it were a real fluid,
%%-----File: 153.png-----%%
but in iron it behaves as if its specific heat were a negative quantity,
which cannot be the case in a real fluid. Hence Thomson
expresses the fact by saying that negative or resinous electricity
carries heat with it in iron. Neither kind of electricity, therefore,
can be regarded in this respect as a real fluid. We may therefore
adhere to the usual convention, and speaking of the positive electricity
only, we may say that in copper it behaves as if its specific
heat were positive, and in iron as if it were negative.

\article{169} M. Le Roux\footnote{
Annales de Chimie et de Physique (4), x.\ p.\ 243 (1867).}, who has made some very careful experiments
on the Thomson effect, finds that in lead the specific heat of
electricity is either zero or very small indeed. Professor Tait has
therefore adopted lead as the standard metal in his thermo-electric
measurements.

\article{170} We may express both the Peltier and the Thomson effects
by stating that when an electric current is flowing from places of
smaller to places of greater thermo-electric power, heat is absorbed,
and when it is flowing in the reverse direction heat is generated,
and this, whether the difference of thermo-electric power in the two
places arises from a difference in the nature of the metal or from a
difference of temperature in the same metal.
\Runhead{SPECIFIC HEAT OF ELECTRICITY.}

\widefig{0.7}{153.png}{Fig. 38.}
\article{171} The amount of heat absorbed corresponding to a given
increase of thermo-electric power, must depend on the temperature
as well as on the amount of that increase. For consider a circuit
consisting of two metals, neither of which exhibits the Thomson
effect. Such a circuit would be represented in the thermo-electric
diagram by the parallelogram \(AabB\) with horizontal and vertical
sides. If the current flows in the direction \(AabB\) heat is absorbed
in \(BA\) and generated on \(ab\), and no reversible thermal effect occurs
elsewhere. Also the heat absorbed in \(BA\) exceeds that generated
%%-----File: 154.png-----%%
in \(ab\) by a quantity represented by the parallelogram \(BAab\). Hence
if we produce \(Aa\) and \(Bb\) and draw the vertical line \(\alpha \beta\) at such a
distance that the heat absorbed at the junction \(AB\) is represented
by the parallelogram \(BA \alpha \beta\), the heat generated at the junction \(ab\),
which, as we have seen, is less than this by the parallelogram \(BAab\),
will be represented by the parallelogram \(ab \beta \alpha\). The Peltier effect
therefore is measured by the product of the increase of thermo-electric
power in passing from the first metal to the second into
the temperature reckoned from some point lower than any observed
temperature, and is of the form \((\phi_2 - \phi_1)(t - t_1)\), when the current
flows from a metal in which the thermo-electric power is \(\phi_1\) to
a metal in which it is \(\phi_2\), and \(t\) is the thermometer reading, and
\(t_1\) is a constant, the value of which can be ascertained only by
experiment.

\article{172} Thus far we are led by the principle of the Conservation
of Energy. It is a consequence, however, of the Second Law of
Thermodynamics, that in all strictly reversible operations in which
heat is transformed into work or work into heat, the amount of heat
absorbed or emitted at the higher temperature is to that emitted or
absorbed at the lower temperature as the higher temperature is to
the lower temperature, both being reckoned from absolute zero of
the thermodynamic scale. It follows that the line \(\alpha \beta\) must be
drawn in the position corresponding to the absolute zero of the
thermodynamic scale, and that the expression for the heat absorbed
may be written \((\phi_2 - \phi_1) \theta\), where \(\theta\) is the temperature reckoned
from absolute zero. It is true that the thermo-electric operations
cannot be made completely reversible, as the conduction of heat,
which is an irreversible operation, is always going on, and cannot
be prevented. We must therefore consider the application of the
Second Law of Thermodynamics to the reversible part of the
phenomena as a very probable conjecture consistent with other
parts of the theory of heat, and verified approximately by the
measurements of the Peltier and Thomson effects by Le Roux.
\Runhead{SECOND LAW OF THERMODYNAMICS.}

\article{173} We are now able to express all the thermal and electromotive
effects in terms of the areas in the thermo-electric diagram.
Let \(Ii\) be the line for one metal, say iron, \(Cc\) that for another, say
copper. Let \(T\) be the higher temperature and \(t\) the lower, and let
\(O\) represent the position of absolute zero. Let the current flow in
the direction \(CIic\) till one unit of electricity has passed. Then the
heat absorbed at the hot junction will be represented by the area
\(CIQR\). This is the Peltier effect.
%%-----File: 155.png-----%%

\begin{center}
\begin{tabular}{@{}l@{ }l@{\ldots}l}
The heat absorbed in the iron is represented & by \(IiPQ\) & {Thomson effect.}\\[.5ex]
The heat generated in the cold junction, & by \(icSP\) & Peltier effect.\\[.5ex]
The heat absorbed in the copper, & by \(cCRS\) & Thomson effect.
\end{tabular}
\end{center}

\widefig{0.72}{155.png}{Fig. 39.}
The whole heat absorbed is therefore represented by \(CIiPSc\), and
the heat generated by \(icSP\), leaving \(CIic\) for the heat absorbed as
the result of the whole operation. This heat is converted into the
work done on the electric current.

\article{174} Entropy\footnote{
[Arts.\ 174-181 consist principally of a repetition of Arts.\ 167-173, but expressed
in the language of the doctrine of Entropy. It was probably the intention of Professor
Clerk Maxwell to insert them or some modification of them in place of the
foregoing Articles, but it has been thought best not to alter the continuous MS., but
simply to insert the separate Articles here as representing a slightly different method
of applying the Second Law of Thermodynamics to thermo-electric phenomena.]}, in Thermodynamics, is a quantity relating to
a body such that its increase or diminution implies that heat has
entered or left the body. The amount of heat which enters or
leaves the body is measured by the product of the increase or
diminution of entropy into the temperature at which it takes
place.

In this treatise we have avoided making any assumption that
electricity is a body or that it is not a body, and we must also avoid
any statement which might suggest that, like a body, electricity
may receive or emit heat.

We may, however, without any such assumption, make use of
the idea of entropy, introduced by Clausius and Rankine into the
theory of heat, and extend it to certain thermo-electric phenomena,
always remembering that entropy is not a thing but a mere instrument
of scientific thought, by which we are enabled to express in a
%%-----File: 156.png-----%%
compact and convenient manner the conditions under which heat
is emitted or absorbed.
\Runhead{ENTROPY.}

\article{175} When an electric current passes from one metal to another
heat is emitted or absorbed at the junction of the metals. We shall
therefore suppose that the electric entropy has diminished or increased
when the electricity passes from the one metal to the other,
the electric entropy being different according to the nature of the
medium in which the electricity is, and being affected by its
temperature, stress, strain, \&c. It is only, however, during the
motion of electricity that any thermo-electric phenomena are produced.

\article{176} It is proved in treatises on thermodynamics that in all reversible
thermal operations, what is called the entropy of the system
remains the same. (Maxwell's Theory of Heat, 5th ed.\ p.\ 190.)

The entropy of a body is a quantity which when the body receives
(or emits) a quantity of heat, \(H\), increases (or diminishes) by
a quantity \(\xp\dfrac{H}{\theta}\), where \(\theta\) is the temperature reckoned on the thermodynamic
scale. The entropy of a material system is the sum of
the entropies of its parts.
\Runhead{ELECTRIC ENTROPY.}

\article{177} The thermal effects of electric currents are in part reversible
and in part irreversible, but the reversible effects, such as
those discovered by Peltier and Thomson, are always small compared
with the irreversible effects---the frictional generation of heat
and the diffusion of heat by conduction. Hence we cannot extend
the demonstration of the theorem, which applies to completely reversible
thermal operations, to thermo-electric phenomena.

But, as Sir Wm. Thomson has pointed out, we have great
reason to conjecture that the reversible portion of the thermo-electric
effects are subject to the same condition as other reversible thermal
operations. This conjecture has not hitherto been disproved by any
experiments, and it may hereafter be verified by careful electric and
calorimetric measurements. In the meantime the consequences which
flow from this conjecture may be conveniently described by an extension
of the term entropy to electric phenomena.

The term Electric Entropy, as we shall use it, corresponds to the
term Thermo-electric Power, as defined by Sir W. Thomson in his
fifth paper on the Dynamical Theory of Heat (Trans.\ R. S. E.
1st May, 1854; Art.\ 140, p.\ 151).
%%-----File: 157.png-----%%

\Subsection{Thermo-electric Diagram.}

\article{178} The most convenient method of studying the theory of
thermo-electric phenomena is by means of a diagram in which the
temperature and electric entropy of a metal at any instant are
represented by the horizontal and vertical coordinates of a point
on the diagram. Thus, if \(OC\) represents the temperature, reckoned
from absolute zero on the thermodynamic scale, of a piece of a
certain metal, and if \(CA\) represents the electric entropy corresponding
to the same piece of metal, then the point \(A\) will indicate
by its position in the diagram the thermo-electric state of the piece
of metal. In the same way we may find other points in the
diagram corresponding to the same metal under other conditions or
to other metals.

\widefig{0.73}{157.png}{Fig. 40.}
\Runhead{THERMO-ELECTRIC DIAGRAM.}
If in the path of an electric current electricity passes from one
metal to another or from one portion of a metal to another at
a different temperature, the different points of the electric circuit
will be represented by corresponding points on the thermo-electric
diagram. The path of the current will thus be represented by
a line or path on the thermo-electric diagram. When the current
flows in a single metal, \(A\), from a point at a temperature \(OC\) to
another at a temperature \(Oc\), the path is represented by the line \(Aa\),
the points of which represent the state of the metal at intermediate
temperatures. The form of the path depends on the nature of
the metal and on the other influences which act on it besides
temperature, such as stress and strain. Professor Tait, however,
%%-----File: 158.png-----%%
finds that for most of the metals except iron and nickel, the path
on the thermo-electric diagram is a straight line.

When the current flows from the metal \(A\) to another metal \(B\)
at the same temperature, the path is represented by \(AB\), a vertical
straight line. The circuit traversed by the electric current will
thus be represented by a circuit on the thermo-electric diagram.

The heat generated while a unit of electricity moves along the
path \(Aa\) is represented by the area of the figure \(AaQPA\), bounded
by the path \(Aa\), the horizontal ordinate at \(a\), the line of zero temperature
and the horizontal ordinate at \(A\). If this area lies on the
right of the path, it represents heat generated; if it lies to the left
of the path it represents heat absorbed.

\article{179} If electricity were a fluid, running through the conductor
as water does through a tube, and always giving out or absorbing
heat till its temperature is that of the conductor, then in passing
from hot to cold it would give out heat and in passing from cold to
hot it would absorb heat, and the amount of this heat would depend
on the specific heat of the fluid.

In the diagram the specific heat of the fluid at \(A\) would be
represented by the line \(\alpha P\), where \(\alpha\) is the point where the tangent
to the path at \(A\) cuts the line of zero temperature, and \(P\)
is the intersection with the same line of the horizontal ordinate
through \(A\).

The line \(Aa \alpha\) in the diagram is such that the electric entropy
increases as the temperature rises. This is the case with copper,
and therefore we may assert that the specific heat of electricity in
copper is positive.
\Runhead{SPECIFIC HEAT OF ELECTRICITY.}

In other metals, as for instance iron, the electric entropy
diminishes as the temperature rises, as is represented by the line
\(\beta bB\). The specific heat of electricity in such metals is negative,
and at \(B\) is represented by the line \(\beta T\).

\article{180} Thomson, who discovered first from theory and then by
experimental verification the thermal effect of an electric current in
an unequally heated metal, expresses the fact by saying that
vitreous electricity carries heat with it in copper, while resinous
electricity carries heat with it in iron.

We must remember, however, that these phrases are not intended
by Thomson, and must not be understood by us, to imply
that electricity either positive or negative is a fluid which can
be heated or cooled and which has a definite specific heat. Since,
therefore, the whole set of phrases are merely analogical we shall
%%-----File: 159.png-----%%
adhere to the ordinary convention according to which vitreous
electricity is reckoned positive, and we shall say that the specific
heat of electricity is positive in copper but negative in iron.

The obvious fact that no real fluid can have a negative specific
heat need not disturb us, for we do not assert that electricity is a
real fluid.

\article{181} Let us next consider a circuit consisting of two linear
conductors of the metals \(A\) and \(B\) respectively, the two junctions
being kept at different temperatures, represented in the diagram
by \(OC\) and \(Oc\). This electric circuit will be represented in the
diagram by the circuit \(AabBA\). If the current flows in the
direction \(AabB\) till one unit of electricity has been transmitted,
the following thermal effects will take place.

(1) In the metal \(A\) heat will be generated as the electricity flows
from the hot junction to the cold junction. The amount of this
heat is represented by the area \(AaQPA\).

(2) At the cold junction, where the electricity passes from the
metal \(A\) to the metal \(B\), heat will be generated. The amount of
this heat is represented by the area \(abSQa\).

(3) In the metal \(B\) heat will be generated as the electricity flows
from the cold junction to the hot junction. The amount of this
heat is represented by the area \(bBTSb\).

(4) At the hot junction, where the electricity passes from the
metal \(B\) to the metal \(A\), heat will be absorbed. The amount of
this heat is represented by the area \(BAPTB\). The reverse order of
the letters shews that this area is to be taken negatively.

The whole heat generated is therefore represented by the area
\(AabBTPA\), and the whole heat absorbed by \(BAPTB\). The total
effect is therefore an absorption of heat represented by the area
\(AabBA\).
\Runhead{ELECTROMOTIVE FORCE.}

The energy corresponding to this heat cannot be lost. It is
transformed into electrical work spent upon the current by an
electromotive force acting in the direction of the current. Since
the quantity of electricity transmitted by the current is supposed
to be unity, the energy, which is the product of the electromotive
force into the quantity of electricity transmitted, must be equal to
the electromotive force itself.

Hence the electromotive force is represented by the area \(AabBA\),
and it acts in the direction represented by the order of the letters---that
is,
\begin{center}
Hot, metal \(A\), cold, metal \(B\), hot.
\end{center}
%%-----File: 160.png-----%%

This electromotive force will, if the resistance of the circuit is
finite, produce an actual current\footnote{
[The energy expended in driving the current will, if not otherwise employed, be
ultimately converted into heat through the frictional resistance of the metals. The
heat produced by this irreversible action must be distinguished from the Thomson
and Peltier effects, and is represented on the Thermo-electric diagram by the area
\(ABbaA\).]}. It was by means of such
currents that the thermo-electric properties of metallic circuits
were first discovered by Seebeck in 1822.

\article{182} The electrical effects due to heat were discovered before
the thermal effects due to the electric current, but the application
of the thermal effects of the current to determine the electromotive
forces acting along different portions of the circuit is due to Sir
W. Thomson\footnote{
\textit{Trans.\ R. S. Edin.}\ 1854.}. It is manifest that in a heterogeneous circuit
we cannot determine the electromotive force acting from the point
\(A\) to the point \(B\) by simply connecting these points by wires to the
electrodes of a galvanometer or electrometer, for we are ignorant of
the electromotive forces acting at the junctions of these wires with
the matter of the circuit at \(A\) and \(B\).
\Runhead{MEASUREMENT OF ELECTROMOTIVE FORCE.}

But if we cause a current of known strength to flow from \(A\) to \(B\),
and if this current causes the generation of a quantity of heat equal
to \(H\) in that portion of the circuit, and if no chemical, magnetic or
other permanent effect takes place in the matter of the conductor
between \(A\) and \(B\), then we know that if \(Q\) is the total quantity of
electricity which has been transmitted from \(A\) to \(B\), and \(E\) the
electromotive force in the direction from \(B\) to \(A\) which the current
has to overcome, then the work done by the current is \(QE\). This
work is done within a definite region, namely the portion \(AB\) of
the conductor, and it is entirely expended in generating heat within
that region. Hence, if the quantity of heat generated in the
portion \(AB\) is \(H\), as expressed in dynamical measure, we have the
equation
\[
QE = H\text{,}
\]
and since \(Q\) and \(H\) are capable of being measured we can determine
the electromotive force \(E\) acting against the current. When the
electromotive force acts in the same direction as the current is
flowing, the quantity of heat generated is negative; or, in other
words, there is an absorption of heat.

In this investigation we must remember that \(E\) represents the
\textit{whole} electromotive force acting against the current. Now part of
this electromotive force arises from the electric resistance of the
%%-----File: 161.png-----%%
conductor. This part always acts against the current, and is proportional
to the current according to Ohm's law.

The other part of the electromotive force acts in a definite direction,
either from \(A\) to \(B\) or from \(B\) to \(A\), and is independent of the
direction of the current. It is generally this latter part of the
electromotive force which is referred to as the electromotive force
from \(A\) to \(B\).

It is easy to eliminate the part due to resistance by making two
experiments in which currents of equal strength are made to flow
in one case from \(A\) to \(B\) and in the other from \(B\) to \(A\). The excess
of the heat generated in the second case over that generated in the
first case, per unit of electricity transmitted, is numerically equal
to twice the electromotive force from \(A\) to \(B\).

\article{183} The total electromotive force round any circuit is easily
measured by breaking the circuit in a place where it is homogeneous,
and determining the difference of potentials of the two ends.
This may be done by any of the ordinary methods for determining
electromotive force or difference of potentials, because in this case
the two ends are of the same substance and at the same temperature.
But we cannot by this method determine how much of this
electromotive force has its seat in a particular part of the circuit,
as for instance, between \(A\) and \(B\), where \(A\) and \(B\) are of different
substances or at different temperatures. The only method by which
we can determine where the electromotive force acts is that of
measuring the heats generated or absorbed during the transmission
of a unit of electricity from \(A\) to \(B\).

\article{184} In the cases we have hitherto considered the only permanent
effect of the current has been the generation or absorption
of heat, for metals are not altered in any respect by the continuous
flow of a current through them. But when the current flows from
a metal to an electrolyte or from an electrolyte to a metal, there
are chemical changes, and in applying the principle of the conservation
of energy we must take account of these as well as of the
thermal effects.

We shall consider the current as flowing through an electrolyte
from the anode to the cathode. The fundamental phenomenon of
electrolysis is the liberation of the components or ions of the electrolyte,
the anion at the anode and the cation at the cathode. This is
the only purely electrolytic effect; the subsequent phenomena
depend on the nature of the ions, the electrodes and the electrolyte,
and take place according to chemical and physical laws in a manner
%%-----File: 162.png-----%%
apparently independent of the electric current. Thus the ion, when
liberated at the electrode, may behave in several different ways,
according to the conditions in which it finds itself. It may be in
such a condition that it acts neither on the electrode nor on the
electrolyte, as when it is a gas which escapes in bubbles, or substance
insoluble in the electrolyte, which is precipitated. It may
be deposited on the surface of the electrode, as hydrogen is on
platinum, and may adhere to it with various degrees of tenacity,
from mere juxtaposition up to chemical combination. If it is
soluble in the electrolyte, it will diffuse through the electrolyte
according to the ordinary law of diffusion, and the rate of this
diffusion is not, so far as we know, affected by the existence of the
electric current through the electrolyte, for it is only when in combination,
and not when in mere solution, that the current produces
the electrolytic transfer of the ions. Thus when hydrogen is an
ion, part of it may escape in bubbles, part of it may be condensed
on the electrode, and part of it may be absorbed into the electrolyte
without combination, and travel through it by ordinary
diffusion.
\Runhead{E. M. F. BETWEEN METAL AND ELECTROLYTE.}

\article{185} The liberated ion may also act chemically on the electrode
or on the electrolyte. The results of such action are called
secondary products of electrolysis, and these secondary products may
remain at the surface of the electrodes, or may become diffused
through the electrolyte. Thus, when the same current is passed,
first through a solution of sulphate of soda between platinum electrodes,
and then through sulphuric acid, equal volumes of oxygen
are given off at the anodes of the two electrolytes, and equal
volumes of hydrogen, each equal to double the volume of oxygen,
are given off at the cathodes.

But if the electrolysis is conducted in suitable vessels, such as
U-shaped tubes or vessels with a porous diaphragm, so that the
substance surrounding each electrode may be examined, it is found
that at the anode of the sulphate of soda there is an equivalent
of sulphuric acid as well as an equivalent of oxygen, and at
the cathode there is an equivalent of soda as well as two equivalents
of hydrogen. It would at first sight appear as if (according to the
old theory of the constitution of salts) the sulphate of soda were
electrolysed into its constituents, sulphuric acid and soda, while the
water of the solution is electrolysed at the same time into oxygen
and hydrogen. But this explanation would involve the assumption
that the same current which passing through dilute sulphuric acid
%%-----File: 163.png-----%%
electrolyses one equivalent of water, when it passes through solution
of sulphate of soda electrolyses two equivalents, one of the
salt and one of water, and this would be contrary to the law of
electrochemical equivalents. But if we suppose that the components
of sulphate of soda are not \ce{SO3} and \ce{Na2O}, but \ce{SO4} and \ce{Na2}---not
sulphuric acid and soda but sulphion and sodium---then an
equivalent of sulphion travels to the anode and is set free, but being
unable to exist in a free state, it breaks up into sulphuric anhydride
and oxygen, one equivalent of each. At the same time [two] equivalents
of sodium are set free at the cathode, and then decompose the
water of the solution, forming two equivalents of soda [\ce{NaHO}]
and two of hydrogen.

In the dilute sulphuric acid, the gases collected at the electrodes
are the constituents of water, namely one volume of oxygen
and two volumes of hydrogen. There is also an increase
of sulphuric acid at the anode, but its amount is less than one
equivalent.

\article{186} It follows from these considerations that in order to ascertain
the electromotive force acting from a metal to an electrolyte,
we must take account of the whole permanent effects of the passage
of one unit of electricity from the metal to the electrolyte. Thus,
if the electrolyte is sulphate of zinc, with zinc electrodes, a certain
amount of heat is generated at the anode for every unit of electricity
and at the same time one equivalent of zinc combines with
one equivalent of sulphion and forms sulphate of zinc. Now the
quantity of heat generated when one equivalent of zinc combines
with oxygen is known from the experiments of Andrews and others,
and also the heat generated when an equivalent of oxide of zinc
combines with sulphuric acid, and is dissolved in water so as to
form a solution of sulphate of zinc of the same strength as that
which surrounds the electrode. The sum of these quantities of
heat, which we may call \(H\), is equivalent to the total work done by
the chemical action at the anode, which is therefore \(JH\) [where
\(J\) represents Joule's equivalent, or the mechanical equivalent of
heat]. Let \(h\) be the quantity of heat generated at the anode during
the passage of one unit of electricity, and let \(E\) be the electromotive
force acting from the zinc to the electrolyte, that is, in the direction
of the current. Then the work done in generating heat is \(Jh\), and
the work done in driving the current is \(E\) so that the equation of
work is
\begin{align*}
JH &= Jh + E \\
\shortintertext{or}
E &= J(H - h)\text{.}
\end{align*}
%%-----File: 164.png-----%%

Of these quantities \(H\) is known very accurately but it is somewhat
difficult to measure \(h\), the quantity of heat generated at the
electrode, because the electrode must be in contact with the electrolyte,
and therefore a large and unknown fraction of the heat
generated will be carried away by conduction and convection
through the electrolyte. The only method which seems likely
to succeed is to compare the stationary temperature at a certain
distance from the electrode with the temperature at the same
point when in the place of the electrode we put a fine wire of
known resistance through which we pass a known current so as
to generate heat at a known rate. If the temperatures are equal
in the two cases we may conclude that the heat is generated at the
same rate in the zinc electrode and in the wire. But if the current
is a strong one a very sensible portion of the whole heat generated
will be due to the work done by the current in overcoming the
ordinary resistance of the electrode and the electrolyte. As the electrode
is generally made of a metal whose resistance is very small
compared with that of the electrolyte, this frictional generation of
heat will take place principally in the electrolyte. This frictional
generation of heat may be made very small compared with the
reversible part by diminishing the strength of the current, but then
the rate of generation of heat becomes so small that it is difficult
to measure it in the presence of unavoidable thermal disturbances,
such as arise from changes in the temperature of the air, \&c. The
experimental investigation is therefore one of considerable difficulty,
and I am not aware that the electromotive force from a metal to an
electrolyte has as yet been measured even approximately\footnote{
[See \hyperref[art:192]{Art.\ 192} and last two paragraphs of note, starting on p.\ \pageref{note:192}.]}. If, however,
we assume that the electromotive forces from the metals \(A\)
and \(B\) to the electrolyte \(C\) are \(A\) and \(B\) respectively, and that the
thermo-electric powers of these metals at the temperature \(\theta\) are \(a\)
and \(b\) respectively, then the electromotive force from \(A\) to \(B\) at
their junction is \((b-a)\theta\).

The total electromotive force round the circuit in the cyclical
direction \(ABC\) is
\[
(b-a)\theta+B-A\text{.}
\]
\Runhead{MEASUREMENT OF ELECTROMOTIVE FORCE.}

\Subsection{On the Conservation of Energy in Electrolysis.}

\article{187} Consider an electric current flowing in a circuit consisting
partly of metals and partly of electrolytes placed in series.

During the passage of one unit of electricity through any section
%%-----File: 165.png-----%%
of the circuit one electrochemical equivalent of each of the electrolytes
is electrolysed. There is therefore a definite amount of
chemical action corresponding to a definite quantity of electricity
passed through the circuit. The energy equivalent to any chemical
process can be ascertained either directly or indirectly. When the
process is such that it will go on of itself, and if the only effect
external to the system is the giving off of heat generated during
the process, then the intrinsic energy of the system must be
diminished during the process by a quantity of energy equivalent
to the heat given out. If a material system consisting of definite
quantities of so many chemical substances is capable of existing in
several different states, and if the system will not of itself pass
from one of these states \((A)\) to another \((B)\) we can still find the
relative energy of the state \((A)\) with respect to the state \((B)\)
provided we can cause both the state \((A)\) and the state \((B)\) to
pass into the state \((C)\) which we may suppose to be the state
in which all the energies of combination of the system have been
exhausted.

Thus if the substances in the system are oxygen, hydrogen and
carbon, and if the states \((A)\) and \((B)\) consist of two different
hydrocarbons with free hydrogen and oxygen, we cannot in general
cause the state \((A)\) to pass into the state \((B)\), but we can cause
either \((A)\) or \((B)\) to pass into the state \((C)\) in which all the
hydrogen is combined with oxygen as water and all the carbon
is combined with oxygen as carbonic acid. In this way the
energy of the state \((A)\) relatively to the state \((B)\) can be determined
by measurements of heat.

\article{188} It has been proved experimentally by Joule that the heat
developed throughout the whole electric circuit is the same for the
same amount of chemical action whatever be the resistance of the
circuit provided no other form of energy than heat is given off by
the system.

Thus in a battery the electrodes of which are connected by a
short thick wire the current is very strong and the heat is generated
principally in the cells of the battery and to a much smaller
extent in the wire; but if the wire is long and thin, the heat
generated in the wire is far greater than that generated in the
cells, but if we take into account the heat generated in the wire
as well as that generated in the cells, we find that the whole
heat generated for each grain of zinc dissolved is the same in
both cases.
\Runhead{JOULE'S EXPERIMENTS.}
%%-----File: 166.png-----%%

\article{189} If, however, the circuit includes a cell in which dilute
acid is electrolysed into oxygen and hydrogen the heat generated
in the circuit, per grain of zinc dissolved, is less than before, by the
quantity of heat which would be generated if the oxygen and
hydrogen evolved in the electrolytic cell were made to combine.

Or if the circuit includes an electromagnetic engine which is
employed to do work, the heat generated in the circuit is less than
that corresponding to the zinc consumed by an amount equal to
the heat which would be generated if the work done by the engine
were entirely expended in friction.

\article{190} If the arrangement is such that the amount of chemical
action depends entirely on the quantity of electricity transmitted
we can determine the electromotive force of the circuit by the
following method, first given by Thomson (\textit{Phil.\ Mag.}, Dec.\ 1851).
Let the resistance of the circuit be made so great that the heat
generated by the current in the electrolytes may be neglected.
Let \(E\) be the electromotive force of the circuit; then the work
done in driving one unit of electricity through the circuit is
numerically equal to \(E\). But during this process one electrochemical
equivalent of the electrolyte undergoes the chemical
process which goes on in the cell. Hence, if the energy given
out during this process is entirely expended in maintaining the
current, the dynamical value of the process must be numerically
equal to \(E\), the electromotive force of the circuit, or, as Thomson
stated it,

`The electromotive force of an electrochemical apparatus is in
absolute measure equal to the mechanical equivalent of the chemical
action on one electrochemical equivalent of the substance.'

\Section{Examples.}

\article{191} If the action in the cell consists in part of irreversible
processes, such as
\begin{enumerate}[leftmargin=4em, itemindent=-1em, nosep]
  \item The frictional generation of heat by resistance in the electrolyte,
  \item Diffusion of the primary or secondary products of electrolysis
through the electrolyte, or,
  \item Any other action which is not reversed when the direction of
the current is reversed,
\end{enumerate}
there will be a certain amount of dissipation of energy and the
electromotive force of the circuit will be less than the loss of
%%-----File: 167.png-----%%
intrinsic energy corresponding to the electrolysis of one electrochemical
equivalent.

It is only the strictly reversible processes that must be taken
into account in calculating the electromotive force of the circuit.

\article{192} The determination of the total electromotive force in an
electrochemical circuit is therefore always possible. If, however,
we wish to determine the precise points in the circuit where the
different portions of this electromotive force are exerted, we find
the investigation much more difficult than in the case of a purely
metallic circuit.
\Runhead{E. M. F. IN A VOLTAIC CIRCUIT.}

For the chemical action at the junction of a metal with an
electrolyte is generally of such a kind that it cannot take place
by itself, that is to say, without an action equivalent to that
which takes place at the other electrode. Thus, when a current
passes between silver electrodes through fused chloride of silver,
chlorine is liberated at the anode which immediately acts on the
electrode so as to form chloride of silver and silver is deposited on
the cathode.

Now we know the amount of heat given out when an equivalent
of free chlorine combines with an equivalent of silver, and this is
equivalent to the energy which must be spent in electrolysing
chloride of silver into free chlorine and free silver, but the process
that takes place at the anode is the combination of silver, not with
free chlorine, but with chlorine in the act of being electrolysed out
of chloride of silver\footnote{\phantomsection\label{note:192}
[The following note is an extract from Professor Maxwell's letter on Potential
published in the \textit{Electrician}, April 26th, 1879.]

In a voltaic circuit the sum of the electromotive forces from zinc to the electrolyte,
from the electrolyte to copper, and from copper to zinc, is not zero but is what is called
the electromotive force of the circuit---a measurable quantity. Of these three electromotive
forces only one can be separately measured by a legitimate process, that,
namely, from copper to zinc.

Now it is found by thermoelectric experiments that this electromotive force is exceedingly
small at ordinary temperatures, being less than a microvolt, and that it is
from zinc to copper.

Hence the statement deduced from experiments in which air is the third medium,
that the electromotive force from copper to zinc is .75 volts, cannot be correct. In
fact, what is really measured is the difference between the potential in air near the
surface of copper, and the potential in air near the surface of zinc, the zinc and copper
being in contact. The number .75 is therefore the electromotive force, in volts of
the circuit copper, zinc, air, copper, and is the sum of three electromotive forces, only
one of which has as yet been measured.

Mr.\ J. Brown has shown (\textit{Phil.\ Mag.}, Aug.\ 1878, p.\ 142), by the divided ring method
of Sir W. Thomson, that whereas copper is negative with respect to iron in air it is
positive with respect to iron in hydrogen sulphide.

It would appear, therefore, that the reason why the results of the comparison of
metals by the ordinary `contact force' experiments harmonise so well with the comparison
by dipping both metals in water or an oxidizing electrolyte is not because the
electromotive force between a metal and a gas or an electrolyte is small, but because
the properties of air agree, to a certain extent, with those of oxidising electrolytes.
For, if the active component of the electrolyte is sulphur, the results are quite different,
and the same kind of difference occurs when hydrogen sulphide is substituted for air.

We know so little about the nature of the ions as they exist in an electrolyte that,
even if we could measure the quantity of heat generated or absorbed when unit of
electricity passes from a metal to an electrolyte, or from an electrolyte to a metal, we
could not determine from this the value of the electromotive force from the metal to
the electrolyte.

If this is the case with liquid electrolytes, we have still less hope of determining the
electromotive force from a metal to a gas, for we cannot produce a current from the
one to the other without tumultuary and non-reversible effects, such as disintegration
of the metal and violent disturbance of the gas by the discontinuous discharge.
}.
%%-----File: 168.png-----%%

\Subsection{On Constant Voltaic Elements.}

\article{193*} When a series of experiments is made with a voltaic
battery in which polarization occurs, the polarization diminishes
during the time that the current is not flowing, so that when
it begins to flow again the current is stronger than after it has
flowed for some time. If, on the other hand, the resistance of the
circuit is diminished by allowing the current to flow through a
short shunt, then, when the current is again made to flow through
the ordinary circuit, it is at first weaker than its normal strength
on account of the great polarization produced by the use of the
short circuit.

To get rid of these irregularities in the current, which are
exceedingly troublesome in experiments involving exact measurements,
it is necessary to get rid of the polarization, or at least
to reduce it as much as possible.

It does not appear that there is much polarization at the surface
of the zinc plate when immersed in a solution of sulphate of zinc
or in dilute sulphuric acid. The principal seat of polarization is
at the surface of the negative metal. When the fluid in which
the negative metal is immersed is dilute sulphuric acid, it is seen
to become covered with bubbles of hydrogen gas, arising from the
electrolytic decomposition of the fluid. Of course these bubbles,
by preventing the fluid from touching the metal, diminish the
surface of contact and increase the resistance of the circuit. But
besides the visible bubbles it is certain that there is a thin coating
of hydrogen, probably not in a free state, adhering to the metal,
and as we have seen that this coating is able to produce an electromotive
force in the reverse direction, it must necessarily diminish
the electromotive force of the battery.
\Runhead{CONSTANT BATTERIES.}

Various plans have been adopted to get rid of this coating of
hydrogen. It may be diminished to some extent by mechanical
%%-----File: 169.png-----%%
means, such as stirring the liquid, or rubbing the surface of the
negative plate. In Smee's battery the negative plates are vertical,
and covered with finely divided platinum from which the
bubbles of hydrogen easily escape, and in their ascent produce a
current of liquid which helps to brush off other bubbles as they
are formed.

A far more efficacious method, however, is to employ chemical
means. These are of two kinds. In the batteries of Grove and
Bunsen the negative plate is immersed in a fluid rich in oxygen,
and the hydrogen, instead of forming a coating on the plate,
combines with this substance. In Grove's battery the plate is
of platinum immersed in strong nitric acid. In Bunsen's first
battery it is of carbon in the same acid. Chromic acid is also used
for the same purpose, and has the advantage of being free from the
acid fumes produced by the reduction of nitric acid.

A different mode of getting rid of the hydrogen is by using
copper as the negative metal, and covering the surface with a coat
of oxide. This, however, rapidly disappears when it is used as
the negative electrode. To renew it Joule has proposed to make
the copper plates in the form of disks, half immersed in the liquid,
and to rotate them slowly, so that the air may act on the parts
exposed to it in turn.

The other method is by using as the liquid an electrolyte, the
cation of which is a metal highly negative to zinc.

In Daniell's battery a copper plate is immersed in a saturated
solution of sulphate of copper. When the current flows through
the solution from the zinc to the copper no hydrogen appears on
the copper plate, but copper is deposited on it. When the solution
is saturated, and the current is not too strong, the copper appears
to act as a true cation, the anion \ce{SO4} travelling towards the zinc.

When these conditions are not fulfilled hydrogen is evolved at
the cathode, but immediately acts on the solution, throwing down
copper, and uniting with \ce{SO4} to form oil of vitriol. When this
is the case, the sulphate of copper next the copper plate is replaced
by oil of vitriol, the liquid becomes colourless, and polarization by
hydrogen gas again takes place. The copper deposited in this way
is of a looser and more friable structure than that deposited by true
electrolysis.

To ensure that the liquid in contact with the copper shall be
saturated with sulphate of copper, crystals of this substance must
be placed in the liquid close to the copper, so that when the solution
%%-----File: 170.png-----%%
is made weak by the deposition of the copper, more of the crystals
may be dissolved.

We have seen that it is necessary that the liquid next the copper
should be saturated with sulphate of copper. It is still more
necessary that the liquid in which the zinc is immersed should be
free from sulphate of copper. If any of this salt makes its way
to the surface of the zinc it is reduced, and copper is deposited
on the zinc. The zinc, copper, and fluid then form a little circuit
in which rapid electrolytic action goes on, and the zinc is eaten
away by an action which contributes nothing to the useful effect
of the battery.

To prevent this, the zinc is immersed either in dilute sulphuric
acid or in a solution of sulphate of zinc, and to prevent the solution
of sulphate of copper from mixing with this liquid, the two liquids
are separated by a division consisting of bladder or porous earthenware,
which allows electrolysis to take place through it, but
effectually prevents mixture of the fluids by visible currents.

In some batteries sawdust is used to prevent currents. The
experiments of Graham, however, shew that the process of diffusion
goes on nearly as rapidly when two liquids are separated by a
division of this kind as when they are in direct contact, provided
there are no visible currents, and it is probable that if a septum
is employed which diminishes the diffusion, it will increase in
exactly the same ratio the resistance of the element, because electrolytic
conduction is a process the mathematical laws of which
have the same form as those of diffusion, and whatever interferes
with one must interfere equally with the other. The only difference
is that diffusion is always going on, while the current flows
only when the battery is in action.

In all forms of Daniell's battery the final result is that the
sulphate of copper finds its way to the zinc and spoils the battery.
To retard this result indefinitely, Sir W. Thomson\footnote{
\textit{Proc.\ R. S.}, Jan.\ 19, 1871.} has constructed
Daniell's battery in the form shewn in Fig. 41.
\Runhead{DANIELL'S BATTERY.}

\widefig{0.72}{171.png}{Fig. 41.}
In each cell the copper plate is placed horizontally at the bottom,
and a saturated solution of sulphate of zinc is poured over it. The
zinc is in the form of a grating and is placed horizontally near the
surface of the solution. A glass tube is placed vertically in the
solution with its lower end just above the surface of the copper
plate. Crystals of sulphate of copper are dropped down this tube,
and, dissolving in the liquid, form a solution of greater density
%%-----File: 171.png-----%%
than that of sulphate of zinc alone, so that it cannot get to the
zinc except by diffusion. To retard this process of diffusion, a
siphon, consisting of a glass tube stuffed with cotton wick, is
placed with one extremity midway between the zinc and copper,
and the other in a vessel outside the cell, so that the liquid is
very slowly drawn off near the middle of its depth. To supply
its place, water, or a weak solution of sulphate of zinc, is added
above when required. In this way the greater part of the sulphate
of copper rising through the liquid by diffusion is drawn off by the
siphon before it reaches the zinc, and the zinc is surrounded by
liquid nearly free from sulphate of copper, and having a very slow
downward motion in the cell, which still further retards the upward
motion of the sulphate of copper. During the action of the battery
copper is deposited on the copper plate, and \ce{SO4} travels slowly
through the liquid to the zinc with which it combines, forming
sulphate of zinc. Thus the liquid at the bottom becomes less dense
by the deposition of the copper, and the liquid at the top becomes
more dense by the addition of the zinc. To prevent this action
from changing the order of density of the strata, and so producing
instability and visible currents in the vessel, care must be taken to
keep the tube well supplied with crystals of sulphate of copper,
and to feed the cell above with a solution of sulphate of zinc sufficiently
dilute to be lighter than any other stratum of the liquid
in the cell.

Daniell's battery is by no means the most powerful in common
use. The electromotive force of Grove's cell is 192,000,000, of
Daniell's 107,900,000, and that of Bunsen's 188,000,000.
%%-----File: 172.png-----%%

The resistance of Daniell's cell is in general greater than that of
Grove's or Bunsen's of the same size.
\Runhead{ELECTROMOTIVE FORCE OF BATTERIES.}
These defects, however, are more than counterbalanced in all
cases where exact measurements are required, by the fact that
Daniell's cell exceeds every other known arrangement in constancy
of electromotive force. It has also the advantage of continuing
in working order for a long time, and of emitting no gas.
%%-----File: 173.png-----%%

\newchapter
\Chapter{CHAPTER XI.}
\Subheading{METHODS OF MAINTAINING AN ELECTRIC CURRENT.}

\article{194} \textsc{The} principal methods of maintaining a steady electric
current are---

\vspace{1ex}
\hspace{3em}
\begin{tabular}{l l}
(1)& The Frictional Machine.\\
(2)& The Voltaic Battery.\\
(3)& The Thermo-electric Battery.\\
(4)& The Magneto-electric Machine.
\end{tabular}

\Subsection{\textup{(1)} The Frictional Electric Machine.}

\article{195} The electrification is here produced between the surfaces of
two different substances, such as glass and amalgam or ebonite and
fur. By the motion of the machine one of these electrified surfaces
is carried away from the other, and both are made to discharge
their electrification into the electrodes of the machine, from which
the current is conveyed along any required circuit.
\Runhead{FRICTIONAL ELECTRIC MACHINE.}

In the ordinary form of the machine a circular plate or a cylinder
of glass is made to revolve about its axis. Let us suppose that the
revolving part is a plate of glass. The rubber is fixed so that it
presses against the surface of the plate as it rotates. The surface
of the rubber is of leather, on which is spread an amalgam of zinc
and mercury. By the friction between the glass and the amalgam
the surface of the glass becomes electrified positively, and that of
the rubber negatively. As the plate revolves the electrified surface
of the glass is carried away from under the rubber, and another
part of the surface of the glass, previously unelectrified, is brought
under the rubber to be electrified. As long as the oppositely
electrified surfaces of the glass and the rubber remain in contact,
the electrical effects in the neighbourhood are very small, but when
%%-----File: 174.png-----%%
the glass is removed from the rubber, strong electrical forces are
developed. The potential of the rubber becomes negative, and as,
on account of the amalgam, it conducts freely its electrification is
at once carried off to the negative electrode. At the same time the
potential of the electrified glass becomes highly positive, but as the
glass is an insulating substance it does not so readily part with its
electrification. The positive electrode of the machine is therefore
furnished with a comb, consisting of a number of sharp pointed
wires terminating near the electrified surface of the glass. As the
potential at the surface of the glass is much higher than that of
the comb there is a great accumulation of negative electrification
at the point of the comb, and this breaks into a negative electric
glow accompanied by an electric wind blowing from the comb to
the glass. The negatively electrified particles of air spread themselves
over the positively electrified surface of the glass, and cause
the electrification of the glass to be discharged. It is possible,
however, that part of them may be carried round with the glass till
they are wiped off by the rubber, though I have not been able to
obtain experimental evidence of this.

Thus the rotation of the machine carries the positive electrification
of the surface of the glass from the rubber to the comb,
and the negative electric wind of the comb either neutralizes the
positively electrified surface, or is carried round with it to the
rubber, so that there is a continual current of positive electricity
kept up from the rubber to the comb, or, what is the same thing,
of negative electricity from the comb to the rubber, or, since the
mode of expressing the fact is indifferent, we may, if we please,
describe it as consisting of a positive current in the one direction
combined with a negative current in the other the arithmetical
sum of these two imaginary currents being the actual current
observed. The action of the machine thus depends on the electrification
of the surface of the glass by the rubber, the convection of
this electrification, by the motion of the machine, to the comb and
the discharge of the electrification by the comb.

\article{196} The strength of the current produced depends on the
surface-density of the electrification, the area of the electrified
surface and the number of turns in a minute.

The electromotive force of the machine is the excess of the
potential of the comb above that of the rubber. The most convenient
test of the electromotive force of an electrical machine is
the length of the sparks which it will give.
%%-----File: 175.png-----%%

During the passage of the electrified surface from the rubber to
the comb it is passing from places of low to places of high potential,
and is therefore acted on by a force in the direction opposite to
that of its motion. The work done in turning the machine therefore
exceeds that necessary to overcome the friction of the rubber,
the axle, and other mechanical resistances by the electrical work
done in carrying the electricity from the rubber to the comb.

At every point of its course the electricity on the surface of the
glass plate is acted on by a force the value of which is measured
by the rate at which the potential varies from one point to another
of the surface. If this force exceeds a certain value it will cause
the electrification to slide along the surface of the plate, and this
will take place under the action of a much smaller force than that
which is required to remove the electricity from the surface. This
discharge along the surface of the plate may be seen when the
electric machine is worked in a dark room, and it is evident that
the electricity which thus flashes back is so much lost from the
principal current of the machine.

In order that the machine may work to the best advantage
this slipping back of the electricity must be prevented. The
slipping takes place whenever the rate of variation of the potential
from point to point of the surface exceeds a certain value. If by
any distribution of the electrification the rate of variation of the
potential can be kept just below this value all the way from the
rubber to the comb the electromotive force of the machine will have
its highest possible value.

In most electrical machines flaps of oiled silk are attached to
the rubber so that as the plate revolves the electrified surface as
it leaves the rubber is covered with the silk flap which extends
from the rubber nearly up to the comb. These silk flaps become
negatively electrified and therefore adhere of themselves to the
surface of the glass. If in any part of the revolution of the plate,
the rate of increase of the potential is so great that a slipping
back of the electrification occurs, the positive electricity which so
slips back neutralizes part of the negative electrification of the
silk flap and so raises the electric potential just behind the place
where the slipping occurred. In this way the slope of the electric
potential is equalized and the electromotive force of the machine is
raised to its highest possible value, so as to give the longest sparks
which a machine of given dimensions can furnish.
\Runhead{ACTION OF SILK FLAPS.}

When the silk flaps are removed the slope of the potential
%%-----File: 176.png-----%%
becomes much greater close to the rubber than at any other place,
the electricity slips back on the glass just as it leaves the rubber
and very little electricity, and that at a comparatively low potential,
reaches the comb.

In the best machines, in which the slope of the potential is
uniform from the rubber to the comb, the length of the spark
must depend principally on the distance between the rubber and
the comb. Hence a machine which, like Winter's, has the rubber
and the comb at opposite extremities of a diameter of the plate will
give a longer spark than one from a machine whose plate has the
same diameter but which like Cuthbertson's has two rubbers and
two combs, the distance between each rubber and its comb being a
quadrant.

\Subsection{On Machines producing Electrification by Mechanical Work.}

\article{197*} In the ordinary frictional electrical machine the work done
in overcoming friction is far greater than that done in increasing
the electrification. Hence any arrangement by which the electrification
may be produced entirely by mechanical work against
the electrical forces is of scientific importance if not of practical
value. The first machine of this kind seems to have been Nicholson's
Revolving Doubler, described in the \textit{Philosophical Transactions} for
1788 as `an instrument which by the turning of a Winch produces
the two states of Electricity without friction or communication
with the Earth.'
\Runhead{THE REVOLVING DOUBLER.}

\article{198*} It was by means of the revolving doubler that Volta
succeeded in developing from the electrification of the pile an
electrification capable of affecting his electrometer. Instruments
on the same principle have been invented independently by Mr.\
C. F. Varley\footnote{
Specification of Patent, Jan.\ 27, 1860, No.\ 206.} and Sir W. Thomson.

These instruments consist essentially of insulated conductors of
various forms, some fixed and others moveable. The moveable
conductors are called Carriers, and the fixed ones may be called
Inductors, Receivers, and Regenerators. The inductors and receivers
are so formed that when the carriers arrive at certain points in
their revolution they are almost completely surrounded by a conducting
body. As the inductors and receivers cannot completely
surround the carrier and at the same time allow it to move freely
in and out without a complicated arrangement of moveable pieces,
the instrument is not theoretically perfect without a pair of regenerators,
%%-----File: 177.png-----%%
which store up the small amount of electricity which
the carriers retain when they emerge from the receivers.

For the present, however, we may suppose the inductors and
receivers to surround the carrier completely when it is within them,
in which case the theory is much simplified.

We shall suppose the machine to consist of two inductors \(A\) and
\(C\), and of two receivers \(B\) and \(D\), with two carriers \(F\) and \(G\).

Suppose the inductor \(A\) to be positively electrified so that its
potential is \(A\), and that the carrier \(F\) is within it and is at
potential \(F\). Then, if \(Q\) is the coefficient of induction (taken
positive) between \(A\) and \(F\), the quantity of electricity on the carrier
will be \(Q(F - A)\).

If the carrier, while within the inductor, is put in connexion with
the earth, then \(F = 0\), and the charge on the carrier will be \(-QA\),
a negative quantity. Let the carrier be carried round till it is
within the receiver \(B\), and let it then come in contact with a spring
so as to be in electrical connexion with \(B\). It will then, as was
shewn in \hyperref[art:20]{Art.\ 20}, become completely discharged, and will communicate
its whole negative charge to the receiver \(B\).

The carrier will next enter the inductor \(C\), which we shall suppose
charged negatively. While within \(C\) it is put in connexion with
the earth and thus acquires a positive charge, which it carries off
and communicates to the receiver \(D\), and so on.

In this way, if the potentials of the inductors remain always
constant, the receivers \(B\) and \(D\) receive successive charges, which
are the same for every revolution of the carrier, and thus every
revolution produces an equal increment of electricity in the receivers.

But by putting the inductor \(A\) in communication with the receiver
\(D\), and the inductor \(C\) with the receiver \(B\), the potentials
of the inductors will be continually increased, and the quantity
of electricity communicated to the receivers in each revolution will
continually increase.

For instance, let the potential of \(A\) and \(D\) be \(U\), and that of \(B\)
and \(C, V\), and when the carrier is within \(A\) let the charge on \(A\)
and \(C\) be \(x\), and that on the carrier \(z\), then, since the potential
of the carrier is zero, being in contact with earth, its charge is
\(z = -QU\). The carrier enters \(B\) with this charge and communicates
it to \(B\). If the capacity of \(B\) and \(C\) is \(B\), their potential will be
changed from \(V\) to \(V - \xp\dfrac{Q}{B} U\).
%%-----File: 178.png-----%%

\hypertarget{198:1}{}
If the other carrier has at the same time carried a charge \(-QV\)
from \(C\) to \(D\), it will change the potential of \(A\) and \(O\) from \(U\) to
\(U - \xp\dfrac{Q'}{A} V\), if \(Q'\) is the coefficient of induction between the carrier
and \(C\), and \(A\) the capacity of \(A\) and \(D\). If, therefore, \(U_n\) and \(V_n\)
be the potentials of the two inductors after \(n\) half revolutions, and
\(U_{n+1}\) and \(V_{n+1}\) be the potentials after \(n + 1\) half revolutions,
\[
  \begin{aligned}
    U_{n + 1} &= U_n - \frac {Q'}{A} V_n,\\
    V_{n + 1} &= V_n - \frac {Q}{B} U_n.
  \end{aligned}
\]

If we write \(p^2 = \xp\dfrac {Q}{B}\) and \(q^2 = \xp\dfrac {Q'}{A}\), we find
\[
  \begin{aligned}
    pU_{n + 1} + qV_{n + 1} &= \left(pU_n + qV_n \right) \left( 1 - pq \right) = \left(pU_0 + qV_0 \right) \left( 1 - pq \right)^{n + 1}, \\
    pU_{n + 1} - qV_{n + 1} &= \left( pU_n - qV_n \right) \left( 1 + pq \right) = \left( pU_0 - qV_0 \right) \left( 1 + pq \right)^{n + 1}.
  \end{aligned}
\]

Hence
\[
  \begin{aligned}
    U_n &= U_0 \left\{ ( 1 - pq )^n + ( 1 + pq )^n \right\} + \frac{q}{p} V_0 \left\{ ( 1 - pq )^n - ( 1 + pq )^n \right\}, \\
    V_n &= \frac{p}{q} U_0 \left\{ ( 1 - pq )^n - ( 1 + pq )^n \right\} + V_0 \left\{ ( 1 - pq )^n + ( 1 + pq )^n \right\}.
  \end{aligned}
\]

It appears from these equations that the quantity \(pU + qV\) continually
diminishes, so that whatever be the initial state of electrification
the receivers are ultimately oppositely electrified, so that
the potentials of \(A\) and \(B\) are in the ratio of \(q\) to \(-p\).

On the other hand, the quantity \(pU - qV\) continually increases,
so that, however little \(pU\) may exceed or fall short of \(qV\) at first,
the difference will be increased in a geometrical ratio in each
revolution till the electromotive forces become so great that the
insulation of the apparatus is overcome.

Instruments of this kind may be used for various purposes.

For producing a copious supply of electricity at a high potential,
as is done by means of Mr.\ Varley's large machine.

For adjusting the charge of a condenser, as in the case of
Thomson's electrometer, the charge of which can be increased or
diminished by a few turns of a very small machine of this kind,
which is then called a Replenisher.

For multiplying small differences of potential. The inductors
may be charged at first to an exceedingly small potential, as, for
instance, that due to a thermo-electric pair, then, by turning the
machine, the difference of potentials may be continually multiplied
%%-----File: 179.png-----%%
till it becomes capable of measurement by an ordinary electrometer.
By determining by experiment the ratio of increase of this difference
due to each turn of the machine, the original electromotive force
with which the inductors were charged may be deduced from the
number of turns and the final electrification.

In most of these instruments the carriers are made to revolve
about an axis and to come into the proper positions with respect
to the inductors by turning an axle. The connexions are made by
means of springs so placed that the carriers come in contact with
them at the proper instants.

\article{199*} Sir W. Thomson\footnote{
\textit{Proc.\ R. S.}, June 20, 1867.}, however, has constructed a machine
for multiplying electrical charges in which the carriers are drops of
water falling out of the inside of an inductor into an insulated
receiver. The receiver is thus continually supplied with electricity
of opposite sign to that of the inductor. If the inductor is electrified
positively, the receiver will receive a continually increasing charge
of negative electricity.
\Runhead{WATER DROPPING ACCUMULATOR.}

The water is made to escape from the receiver by means of a
funnel, the nozzle of which is almost surrounded by the metal of
the receiver. The drops falling from this nozzle are therefore
nearly free from electrification. Another inductor and receiver of
the same construction are arranged so that the inductor of the
one system is in connexion with the receiver of the other. The
rate of increase of charge of the receivers is thus no longer constant,
but increases in a geometrical progression with the time, the
charges of the two receivers being of opposite signs. This increase
goes on till the falling drops are so diverted from their course by
the electrical action that they fall outside of the receiver or even
strike the inductor.

In this instrument the energy of the electrification is drawn
from that of the falling drops.

\article{200} In Holtz's `Influence-Machine' a plate of varnished glass
is made to rotate in front of a fixed plate of varnished glass. The
inductors consist of two pointed pieces of card sometimes covered
with tin foil and placed on the further side of the fixed plate so
that their points are at opposite extremities of a diameter. Holes
are cut in the fixed plate opposite the points of the inductors. The
electrodes are first put in connexion with each other and the
machine is set in rotation. One of the inductors is then electrified,
either by an ordinary machine or by an excited piece of ebonite.
%%-----File: 180.png-----%%
Let us suppose it electrified positively. The comb in front of the
charged inductor immediately begins to glow and discharges negative
electricity against the rotating disk. This negative electrification
is carried round by the disk to the other side where it is free
from the influence of the positive inductor. The other inductor
now discharges positive electricity from its point and becomes
itself negatively charged, and the comb of the negative electrode
discharges positive electricity, which is carried round the disk on
the other side back to the positive electrode. In this way there
is kept up an electric current from the positive to the negative
electrode. A rushing noise is heard and in the dark a glow is
seen extending itself from the positive comb over the surface of the
rotating disk in the direction opposite to its motion. If the electrodes
are now gradually separated a succession of sparks will pass
between them.
\Runhead{HOLTZ'S MACHINE.}

\Subsection{Influence Machine.}
\begin{description}[leftmargin=5em, itemindent=-4em, parsep=0.3ex, itemsep=0ex]
\item[]1865. Holtz exhibited his machine to the Berlin Academy, April
1865. 8 to 10 cm.\ diam.
\item[]1866. Töpler, metal inductors, two metal carriers on a glass
disk.
\item[]1867. Töpler's multiple machine, 8 rotating disks, 32 cm.\ diam.\
sparks 6 to 9 cm.
\item[]1867. Holtz with two disks rotating oppositely.
\item[]1868. Kundt.
\item[]\hphantom{1868. }Carré, inductor disk 38 cm.\ induced 49, spark 15 to 18.
\end{description}

\vspace{.5ex}
\article{201*} In the electrical machines already described sparks occur
whenever the carrier comes in contact with a conductor at a
different potential from its own.

Now we have shewn that whenever this occurs there is a loss
of energy, and therefore the whole work employed in turning the
machine is not converted into electrification in an available form,
but part is spent in producing the heat and noise of electric
sparks.
\Runhead{MACHINE WITHOUT SPARKS.}

I have therefore thought it desirable to shew how an electrical
machine may be constructed which is not subject to this loss of
efficiency. I do not propose it as a useful form of machine, but
as an example of the method by which the contrivance called in
heat-engines a regenerator may be applied to an electrical machine
to prevent loss of work.
%%-----File: 181.png-----%%

\wrapfig{0.51}{181.png}{Fig. 42.}
In the figure let \(A\), \(B\), \(C\), \(A'\), \(B'\), \(C'\) represent hollow fixed
conductors, so arranged that the carrier \(P\) passes in succession
within each of them. Of these \(A\), \(A'\) and \(B\), \(B'\) nearly surround the
carrier when it is at the middle point of its passage, but \(C\), \(C'\) do not
cover it so much.

We shall suppose \(A\), \(B\), \(C\) to be connected with a Leyden jar
of great capacity at potential \(V\), and \(A'\), \(B'\), \(C'\) to be connected with
another jar at potential \(-V'\).

\(P\) is one of the carriers moving in a circle from \(A\) to \(C'\), \&c.\
and touching in its course certain
springs, of which \(a\) and
\(a'\) are connected with \(A\) and \(A'\)
respectively, and \(e\), \(e'\) are connected
with the earth.

Let us suppose that when
the carrier \(P\) is in the middle
of \(A\) the coefficient of induction
between \(P\) and \(A\) is \(-A\). The
capacity of \(P\) in this position
is greater than \(A\), since it is not
completely surrounded by the
receiver \(A\). Let it be \(A + a\).

Then if the potential of \(P\) is \(U\), and that of \(A\), \(V\), the charge
on \(P\) will be \(( A + a )U - AV\).

Now let \(P\) be in contact with the spring \(a\) when in the middle
of the receiver \(A\), then the potential of \(P\) is \(V\), the same as that
of \(A\), and its charge is therefore \(aV\).

If \(P\) now leaves the spring \(a\) it carries with it the charge \(aV\).
As \(P\) leaves \(A\) its potential diminishes, and it diminishes still more
when it comes within the influence of \(C'\), which is negatively
electrified.

If when \(P\) comes within \(C\) its coefficient of induction on \(C\) is
\(-C'\), and its capacity is \(C' + c'\), then, if \(U\) is the potential of \(P\)
the charge on \(P\) is
\[
(C'+c')U+C'V'=aV\text{.}\]

If
\[
C'V'=aV\text{,}
\]
then at this point \(U\) the potential of \(P\) will be reduced to zero.

Let \(P\) at this point come in contact with the spring \(e'\) which is
connected with the earth. Since the potential of \(P\) is equal to that
of the spring there will be no spark at contact.

This conductor \(C'\), by which the carrier is enabled to be connected
%%-----File: 182.png-----%%
to earth without a spark, answers to the contrivance called a
regenerator in heat-engines. We shall therefore call it a Regenerator.

Now let \(P\) move on, still in contact with the earth-spring \(e'\), till
it comes into the middle of the inductor \(B\), the potential of which
is \(V\). If \(-B\) is the coefficient of induction between \(P\) and \(B\) at
this point, then, since \(U = 0\) the charge on \(P\) will be \(-BV\).

When \(P\) moves away from the earth-spring it carries this charge
with it. As it moves out of the positive inductor \(B\) towards the
negative receiver \(A'\) its potential will be increasingly negative. At
the middle of \(A'\), if it retained its charge, its potential would be
\[
-\frac{A'V'+BV}{A'+a'}\text{,}
\]
and if \(BV\) is greater than \(a'V'\) its numerical value will be greater
than that of \(V'\). Hence there is some point before \(P\) reaches the
middle of \(A'\) where its potential is \(-V'\). At this point let it come
in contact with the negative receiver-spring \(a'\). There will be no
spark since the two bodies are at the same potential. Let \(P\) move
on to the middle of \(A'\), still in contact with the spring, and therefore
at the same potential with \(A'\). During this motion it communicates
a negative charge to \(A'\). At the middle of \(A'\) it leaves the spring
and carries away a charge \(-a'V'\) towards the positive regenerator
\(C\), where its potential is reduced to zero and it touches the earth-spring
\(e\). It then slides along the earth-spring into the negative
inductor \(B'\), during which motion it acquires a positive charge \(B'V'\)
which it finally communicates to the positive receiver \(A\), and the
cycle of operations is repeated.

During this cycle the positive receiver has lost a charge \(aV\) and
gained a charge \(B'V'\). Hence the total gain of positive electricity
is
\[
B'V' - aV\text{.}
\]

Similarly the total gain of negative electricity is \(BV - a'V'\).

By making the inductors so as to be as close to the surface of
the carrier as is consistent with insulation, \(B\) and \(B'\) may be made
large, and by making the receivers so as nearly to surround the
carrier when it is within them, \(a\) and \(a'\) may be made very small,
and then the charges of both the Leyden jars will be increased in
every revolution.

The conditions to be fulfilled by the regenerators are
\[
C'V'=aV\text{,\quad and\quad}CV = a'V'\text{.}
\]

Since \(a\) and \(a'\) are small the regenerators do not require to be
either large or very close to the carriers.
%%-----File: 183.png-----%%

\Subsection{Coulomb's Torsion Balance.}
\Runhead{TORSION BALANCE.}

\article{202*} A great number of the experiments by which Coulomb
established the fundamental laws of electricity were made by measuring
the force between two small spheres charged with electricity,
one of which was fixed while the other was held in equilibrium by
two forces, the electrical action between the spheres, and the
torsional elasticity of a glass fibre or metal wire.

The balance of torsion consists of a horizontal arm of gum-lac,
suspended by a fine wire or glass fibre, and carrying at one end a
little sphere of elder pith, smoothly gilt. The suspension wire is
fastened above to the vertical axis of an arm which can be moved
round a horizontal graduated circle, so as to twist the upper end
of the wire about its own axis any number of degrees.

The whole of this apparatus is enclosed in a case. Another little
sphere is so mounted on an insulating stem that it can be charged
and introduced into the case through a hole, and brought so that
its centre coincides with a definite point in the horizontal circle
described by the suspended sphere. The position of the suspended
sphere is ascertained by means of a graduated circle engraved on
the cylindrical glass case of the instrument.

Now suppose both spheres charged, and the suspended sphere
in equilibrium in a known position such that the torsion-arm makes
an angle \(\theta\) with the radius through the centre of the fixed sphere.
The distance of the centres is then \(2 a \sin \tstrut\frac{1}{2} \theta\), where \(a\) is the radius
of the torsion-arm, and if \(F\) is the force between the spheres the
moment of this force about the axis of torsion is \(Fa \cos \tstrut\frac{1}{2} \theta\).

Let both spheres be completely discharged, and let the torsion-arm
now be in equilibrium at an angle \(\phi\) with the radius through
the fixed sphere.

Then the angle through which the electrical force twisted the
torsion-arm must have been \(\theta - \phi\), and if \(M\) is the moment of
the torsional elasticity of the fibre, we shall have the equation
\[
Fa \cos \tfrac{1}{2} \theta = M ( \theta - \phi )\text{.}
\]

Hence, if we can ascertain \(M\), we can determine \(F\), the actual
force between the spheres at the distance \(2 a \sin \tstrut\frac{1}{2} \theta\).

To find \(M\), the moment of torsion, let \(I\) be the moment of inertia
of the torsion-arm, and \(T\) the time of a double vibration of the arm
under the action of the torsional elasticity, then
\[
M = \frac{4\pi^2I}{T^2}\text{.}
\]
%%-----File: 184.png-----%%

In all electrometers it is of the greatest importance to know
what force we are measuring. The force acting on the suspended
sphere is due partly to the direct action of the fixed sphere, but
partly also to the electrification, if any, of the sides of the case.
\Runhead{INFLUENCE OF THE CASE.}

If the case is made of glass it is impossible to determine the
electrification of its surface otherwise than by very difficult measurements
at every point. If, however, either the case is made
of metal, or if a metallic case which almost completely encloses the
apparatus is placed as a screen between the spheres and the glass
case, the electrification of the inside of the metal screen will depend
entirely on that of the spheres, and the electrification of the glass
case will have no influence on the spheres. In this way we may
avoid any indefiniteness due to the action of the case.

To illustrate this by an example in which we can calculate all
the effects, let us suppose that the case is a sphere of radius \(b\),
that the centre of motion of the torsion-arm coincides with the
centre of the sphere and that its radius is \(a\); that the charges on
the two spheres are \(E_1\) and \(E\), and that the angle between their
positions is \(\theta\); that the fixed sphere is at a distance \(a_1\) from the
centre, and that \(r\) is the distance between the two small spheres.

Neglecting for the present the effect of induction on the distribution
of electricity on the small spheres, the force between
them will be a repulsion
\[
=\frac{EE_1}{r^2}\text{,}
\]
and the moment of this force round a vertical axis through the
centre will be
\[
\frac{EE_1aa_1 \sin \theta}{ r^3}\text{.}
\]

The image of \(E_1\) due to the spherical surface of the case is a point
in the same radius at a distance \(\xp\dfrac{b^2}{a_1}\) with a charge \(-E_1 \xp\dfrac{b}{a_1}\), and the
moment of the attraction between \(E\) and this image about the axis
of suspension is
\begin{gather*}
EE_1 \frac{b}{a_1} \frac{a \dfrac{b^2}{a_1} \sin \theta} {\left\{ a^2 - 2 \dfrac{ab^2}{a_1} \cos \theta + \dfrac{b^4}{{a_1}^2} \right\}^\frac{3}{2}}\\
= EE_1 \frac{aa_1\sin\theta}{b^3 \left\{ 1 - 2 \dfrac{aa_1}{b^2} \cos \theta + \dfrac{a^2{a_1}^2}{b^4} \right\}^\frac{3}{2}}.
\end{gather*}
%%-----File: 185.png-----%%

If \(b\), the radius of the spherical case, is large compared with \(a\)
and \(a_1\), the distances of the spheres from the centre, we may neglect
the second and third terms of the factor in the denominator. The
whole moment tending to turn the torsion-arm may then be written
\[
EE_1 aa_1\sin\theta\left\{\frac{1}{r^3}-\frac{1}{b^3}\right\}=M(\theta-\phi)\text{.}
\]

\Subsection{Electrometers for the Measurement of Potentials.}

\article{203*} In all electrometers the moveable part is a body charged
with electricity, and its potential is different from that of certain
of the fixed parts round it. When, as in Coulomb's method, an
insulated body having a certain charge is used, it is the charge
which is the direct object of measurement. We may, however,
connect the balls of Coulomb's electrometer, by means of fine wires,
with different conductors. The charges of the balls will then
depend on the values of the potentials of these conductors and on
the potential of the case of the instrument. The charge on each
ball will be approximately equal to its radius multiplied by the
excess of its potential over that of the case of the instrument,
provided the radii of the balls are small compared with their
distances from each other and from the sides or opening of the
case.

Coulomb's form of apparatus, however, is not well adapted for
measurements of this kind, owing to the smallness of the force
between spheres at the proper distances when the difference of
potentials is small. A more convenient form is that of the
Attracted Disk Electrometer. The first electrometers on this
principle were constructed by Sir W. Snow Harris\footnote{
\textit{Phil.\ Trans.}\ 1834.}. They have
since been brought to great perfection, both in theory and construction,
by Sir W. Thomson\footnote{
See an excellent report on Electrometers by Sir W. Thomson. \textit{Report of the
British Association}, Dundee, 1867.}.

When two disks at different potentials are brought face to face
with a small interval between them there will be a nearly uniform
electrification on the opposite faces and very little electrification
on the backs of the disks, provided there are no other conductors
or electrified bodies in the neighbourhood. The charge on the
positive disk will be approximately proportional to its area, and to
the difference of potentials of the disks, and inversely as the distance
%%-----File: 186.png-----%%
between them. Hence, by making the areas of the disks large
and the distance between them small, a small difference of potential
may give rise to a measurable force of attraction.
\Runhead{ATTRACTED DISK ELECTROMETERS.}

\article{204*} The addition of the guard-ring to the attracted disk is
one of the chief improvements which Sir W. Thomson has made
on the apparatus.

\widefig{0.63}{186.png}{Fig. 43.}
Instead of suspending the whole of one of the disks and determining
the force acting upon it, a central portion of the disk is
separated from the rest to form the attracted disk, and the outer
ring forming the remainder of the disk is fixed. In this way the
force is measured only on that part of the disk where it is most
regular, and the want of uniformity of the electrification near the
edge is of no importance, as it occurs on the guard-ring and not
on the suspended part of the disk.

Besides this, by connecting the guard-ring with a metal case
surrounding the back of the attracted disk and all its suspending
apparatus, the electrification of the back of the disk is rendered
impossible, for it is part of the inner surface of a closed hollow
conductor all at the same potential.

Thomson's Absolute Electrometer therefore consists essentially
%%-----File: 187.png-----%%
of two parallel plates at different potentials, one of which is made
so that a certain area, no part of which is near the edge of the
plate, is moveable under the action of electric force. To fix our
ideas we may suppose the attracted disk and guard-ring uppermost.
The fixed disk is horizontal, and is mounted on an insulating stem
which has a measurable vertical motion given to it by means of
a micrometer screw. The guard-ring is at least as large as the
fixed disk; its lower surface is truly plane and parallel to the fixed
disk. A delicate balance is erected on the guard-ring to which
is suspended a light moveable disk which almost fills the circular
aperture in the guard-ring without rubbing against its sides. The
lower surface of the suspended disk must be truly plane, and we
must have the means of knowing when its plane coincides with that
of the lower surface of the guard-ring, so as to form a single plane
interrupted only by the narrow interval between the disk and its
guard-ring.

For this purpose the lower disk is screwed up till it is in contact
with the guard-ring, and the suspended disk is allowed to rest
upon the lower disk, so that its lower surface is in the same plane
as that of the guard-ring. Its position with respect to the guard-ring
is then ascertained by means of a system of fiducial marks.
Sir W. Thomson generally uses for this purpose a black hair
attached to the moveable part. This hair moves up or down just
in front of two black dots on a white enamelled ground and is
viewed along with these dots by means of a plano convex lens with
the plane side next the eye. If the hair as seen through the lens
appears straight and bisects the interval between the black dots
it is said to be in its \textit{sighted position}, and indicates that the suspended
disk with which it moves is in its proper position as regards
height. The horizontality of the suspended disk may be tested by
comparing the reflexion of part of any object from its upper surface
with that of the remainder of the same object from the upper
surface of the guard-ring.
\Runhead{ABSOLUTE ELECTROMETER.}

The balance is then arranged so that when a known weight is
placed on the centre of the suspended disk it is in equilibrium
in its sighted position, the whole apparatus being freed from
electrification by putting every part in metallic communication.
A metal case is placed over the guard-ring so as to enclose the
balance and suspended disk, sufficient apertures being left to see
the fiducial marks.

The guard-ring, case, and suspended disk are all in metallic
%%-----File: 188.png-----%%
communication with each other, but are insulated from the other
parts of the apparatus.

Now let it be required to measure the difference of potentials
of two conductors. The conductors are put in communication with
the upper and lower disks respectively by means of wires, the
weight is taken off the suspended disk, and the lower disk is
moved up by means of the micrometer screw till the electrical
attraction brings the suspended disk down to its sighted position.
We then know that the attraction between the disks is equal to
the weight which brought the disk to its sighted position.

If \(W\) be the numerical value of the weight, and \(g\) the force of
gravity, the force is \(Wg\), and if \(A\) is the area of the suspended
disk, \(D\) the distance between the disks, and \(V\) the difference of the
potentials of the disks,
\begin{align*}
Wg &= \frac{V^2A}{8 \pi D^2}\text{,}\\
\shortintertext{or}
V &= D\, \sqrt{\frac{8 \pi gW}{A}}\text{.}
\end{align*}

If the suspended disk is circular, of radius \(R\), and if the radius of
the aperture of the guard-ring is \(R'\) then
\[
A = \tfrac{1}{2} \pi (R^2 + R'^2)\text{\footnotemark, and }V = 4D\, \sqrt{ \frac{gW}{R^2 + R'^2}}.
\]
\footnotetext{Let us denote the radius of the suspended disk by \(R\), and that of the aperture of
the guard-ring by \(R'\), then the breadth of the annular interval between the disk and
the ring will be \(B = R'- R\).

If the distance between the suspended disk and the large fixed disk is \(D\), and the
difference of potentials between these disks is \(V\), then (see \textit{Electricity and Magnetism},
Art.\ 201) the quantity of electricity on the suspended disk will be
\begin{align*}
Q &= V \left\{ \frac{R^2 + R'^2}{8D} - \frac{R'^2 - R^2}{8D} \frac{\alpha}{D + \alpha} \right\}\text{,}\\
\shortintertext{where}
\alpha &= B \frac{log_e2}{\pi}, \quad \text{or} \quad \alpha = 0.220635 (R' - R)\text{.}
\end{align*}

If the surface of the guard-ring is not exactly in the plane of the surface of
the suspended disk, let us suppose that the distance between the fixed disk and
the guard-ring is not \(D\) but \(D+z=D'\), then (see \textit{Electricity and Magnetism}, Art.\ 225)
there will be an additional charge of electricity near the edge of the disk on
account of its height \(z\) above the general surface of the guard-ring. The whole
charge in this case is therefore
\hypertarget{204:1}{}
\[
Q = V \left\{ \frac{R^2 + R'^2}{8D} - \frac{R'^2 - R^2}{8D} \frac{\alpha}{D + \alpha} + \frac{R+R'}{D}(D'-D) \log_e \frac{4 \pi(R+R')}{D'-D} \right\}\text{,}
\]
and in the expression for the attraction we must substitute for \(A\), the area of the disk,
the corrected quantity
\[
A = \tfrac{1}{2} \pi \left\{R^2 + R'^2 -(R'^2 - R^2) \frac{\alpha}{D + \alpha} + 8 (R + R')(D' - D) \log_e \frac{4 \pi (R + R')}{D' - D} \right\},
\]
\begin{align*}
\shortintertext{where}
R &= \text{radius of suspended disk,} \\
R' &= \text{radius of aperture in the guard-ring,} \\
D &= \text{distance between fixed and suspended disks,} \\
D' &= \text{distance between fixed disk and guard-ring,} \\
\alpha &= 0.220635 (R' - R).
\end{align*}

When \(\alpha\) is small compared with \(D\) we may neglect the second term, and when
\(D' - D\) is small we may neglect the last term.
}
%%-----File: 189.png-----%%

\article{205*} Since there is always some uncertainty in determining the
micrometer reading corresponding to \(D = 0\), and since any error
in the position of the suspended disk is most important when \(D\)
is small, Sir W. Thomson prefers to make all his measurements
depend on differences of the electromotive force \(V\). Thus, if \(V\) and
\(V'\) are two potentials, and \(D\) and \(D'\) the corresponding distances,
\[
  V - V' = (D - D') \sqrt{ \frac{8 \pi g W}{A}}\text{.}
\]

For instance, in order to measure the electromotive force of a
galvanic battery, two electrometers are used.

By means of a condenser, kept charged if necessary by a replenisher,
the lower disk of the principal electrometer is maintained
at a constant potential. This is tested by connecting the lower
disk of the principal electrometer with the lower disk of a secondary
electrometer, the suspended disk of which is connected with the
earth. The distance between the disks of the secondary electrometer
and the force required to bring the suspended disk to
its sighted position being constant, if we raise the potential of the
condenser till the secondary electrometer is in its sighted position,
we know that the potential of the lower disk of the principal
electrometer exceeds that of the earth by a constant quantity which
we may call \(V\).

If we now connect the positive electrode of the battery to earth,
and connect the suspended disk of the principal electrometer to the
negative electrode, the difference of potentials between the disks
will be \(V + v\), if \(v\) is the electromotive force of the battery. Let
\(D\) be the reading of the micrometer in this case, and let \(D'\) be the
reading when the suspended disk is connected with earth, then
\[
 v = (D - D') \sqrt{ \frac{8 \pi gW}{A}}\text{.}
\]

In this way a small electromotive force \(v\) may be measured
by the electrometer with the disks at conveniently measurable
distances. When the distance is too small a small change of
absolute distance makes a great change in the force, since the
%%-----File: 190.png-----%%
force varies inversely as the square of the distance, so that any
error in the absolute distance introduces a large error in the result
unless the distance is large compared with the limits of error of
the micrometer screw.
\Runhead{SMALL ELECTROMOTIVE FORCES MEASURED.}

The effects of small irregularities of form in the surface of the
disks and of the interval between them diminish according to the
inverse cube and higher inverse powers of the distance, and whatever
be the form of a corrugated surface, the eminences of which
just reach a plane surface, the electrical effect at any distance
which is considerable compared to the breadth of the corrugations,
is the same as that of a plane at a certain small distance behind
the plane of the tops of the eminences.

By means of the auxiliary electrification, tested by the auxiliary
electrometer, a proper interval between the disks is secured.

The auxiliary electrometer may be of a simpler construction, in
which there is no provision for the determination of the force
of attraction in absolute measure, since all that is wanted is to
secure a constant electrification. Such an electrometer may be
called a \textit{gauge} electrometer.

This method of using an auxiliary electrification besides the electrification
to be measured is called the Heterostatic method of
electrometry, in opposition to the Idiostatic method in which the
whole effect is produced by the electrification to be measured.

In several forms of the attracted disk electrometer, the attracted
disk is placed at one end of an arm which is supported by being
attached to a platinum wire passing through its centre of gravity
and kept stretched by means of a spring. The other end of the
arm carries the hair which is brought to a sighted position by
altering the distance between the disks, and so adjusting the force
of the electric attraction to a constant value. In these electrometers
this force is not in general determined in absolute measure,
but is known to be constant, provided the torsional elasticity of
the platinum wire does not change.

The whole apparatus is placed in a Leyden jar, of which the inner
surface is charged and connected with the attracted disk and
guard-ring. The other disk is worked by a micrometer screw and
is connected first with the earth and then with the conductor whose
potential is to be measured. The difference of readings multiplied
by a constant to be determined for each electrometer gives the
potential required.
%%-----File: 191.png-----%%

\Subsection{On the Measurement of Electric Potential.}

\article{206*} In order to determine large differences of potential in absolute
measure we may employ the attracted disk electrometer, and
compare the attraction with the effect of a weight. If at the same
time we measure the difference of potential of the same conductors
by means of the quadrant electrometer, we shall ascertain the
absolute value of certain readings of the scale of the quadrant
electrometer, and in this way we may deduce the value of the scale
readings of the quadrant electrometer in terms of the potential
of the suspended part, and the moment of torsion of the suspension
apparatus.
\Runhead{MEASUREMENT OF POTENTIAL.}

To ascertain the potential of a charged conductor of finite size
we may connect the conductor with one electrode of the electrometer,
while the other is connected to earth or to a body of
constant potential. The electrometer reading will give the potential
of the conductor after the division of its electricity between it
and the part of the electrometer with which it is put in contact.
If \(K\) denote the capacity of the conductor, and \(K'\) that of this part
of the electrometer, and if \(V, V'\) denote the potentials of these
bodies before making contact, then their common potential after
making contact will be
\[
  \overline{V} = \frac{KV + K'V'}{K + K'}\text{.}
\]

Hence the original potential of the conductor was
\[
  V = \overline{V} + \frac{K'}{K} (\overline{V} - V')\text{.}
\]

If the conductor is not large compared with the electrometer,
\(K'\) will be comparable with \(K\), and unless we can ascertain the
values of \(K\) and \(K'\) the second term of the expression will have
a doubtful value. But if we can make the potential of the electrode
of the electrometer very nearly equal to that of the body before
making contact, then the uncertainty of the values of \(K\) and \(K'\)
will be of little consequence.

If we know the value of the potential of the body approximately,
we may charge the electrode by means of a `replenisher' or otherwise
to this approximate potential, and the next experiment will
give a closer approximation. In this way we may measure the
potential of a conductor whose capacity is small compared with that
of the electrometer.
%%-----File: 192.png-----%%

\Subsection{To Measure the Potential at any Point in the Air.}

\article{207*} \textit{First Method.} Place a sphere, whose radius is small compared
with the distance of electrified conductors, with its centre
at the given point. Connect it by means of a fine wire with the
earth, then insulate it, and carry it to an electrometer and ascertain
the total charge on the sphere.
\Runhead{POTENTIAL AT ANY POINT IN THE AIR.}

Then, if \(V\) be the potential at the given point, and \(a\) the
radius of the sphere, the charge of the sphere will be \(-Va = Q\),
and if \(V'\) be the potential of the sphere as measured by an
electrometer when placed in a room whose walls are connected
with the earth, then
\begin{gather*}
Q = V'a\text{,}\\
\shortintertext{whence}
V + V' = 0\text{,}
\end{gather*}
or the potential of the air at the point where the centre of the
sphere was placed is equal but of opposite sign to the potential of
the sphere after being connected to earth, then insulated, and
brought into a room.

This method has been employed by M. Delmann of Creuznach in
measuring the potential at a certain height above the earth's
surface\footnote{[Compare \hyperref[art:50]{Art.\ 50}.]}.

\textit{Second Method.} We have supposed the sphere placed at the
given point and first connected to earth, and then insulated, and
carried into a space surrounded with conducting matter at potential
zero.

Now let us suppose a fine insulated wire carried from the electrode
of the electrometer to the place where the potential is to
be measured. Let the sphere be first discharged completely. This
may be done by putting it into the inside of a vessel of the same
metal which nearly surrounds it and making it touch the vessel.
Now let the sphere thus discharged be carried to the end of the
wire and made to touch it. Since the sphere is not electrified it
will be at the potential of the air at the place. If the electrode
wire is at the same potential it will not be affected by the contact,
but if the electrode is at a different potential it will by contact
with the sphere be made nearer to that of the air than it was
before. By a succession of such operations, the sphere being
alternately discharged and made to touch the electrode, the potential
of the electrode of the electrometer will continually approach
that of the air at the given point.
%%-----File: 193.png-----%%

\article{208*} To measure the potential of a conductor without touching
it, we may measure the potential of the air at any point in the
neighbourhood of the conductor, and calculate that of the conductor
from the result. If there be a hollow nearly surrounded by the
conductor, then the potential at any point of the air in this hollow
will be very nearly that of the conductor.
\Runhead{POTENTIAL OF A CONDUCTOR.}

In this way it has been ascertained by Sir W. Thomson that if
two hollow conductors, one of copper and the other of zinc, are
in metallic contact, then the potential of the air in the hollow
surrounded by zinc is positive with reference to that of the air in
the hollow surrounded by copper.

\textit{Third Method.} If by any means we can cause a succession of
small bodies to detach themselves from the end of the electrode,
the potential of the electrode will approximate to that of the surrounding
air. This may be done by causing shot, filings, sand, or
water to drop out of a funnel or pipe connected with the electrode.
The point at which the potential is measured is that at which
the stream ceases to be continuous and breaks into separate parts
or drops.
%%-----File: 194.png-----%%

\newchapter
\Chapter{CHAPTER XII.}
\Subheading{THE MEASUREMENT OF ELECTRIC RESISTANCE.}

\article{209*} \textsc{In} the present state of electrical science, the determination
of the electric resistance of a conductor may be considered
as the cardinal operation in electricity, in the same sense that the
determination of weight is the cardinal operation in chemistry.

The reason of this is that the determination in absolute measure
of other electrical magnitudes, such as quantities of electricity,
electromotive forces, currents, \&c., requires in each case a complicated
series of operations, involving generally observations of
time, measurements of distances, and determinations of moments
of inertia, and these operations, or at least some of them, must be
repeated for every new determination, because it is impossible to preserve
a unit of electricity, or of electromotive force, or of current, in
an unchangeable state, so as to be available for direct comparison.

But when the electric resistance of a properly shaped conductor
of a properly chosen material has been once determined, it is found
that it always remains the same for the same temperature\footnote{
[Recent observations have shewn that it is far from easy to find a material
satisfying this condition.]}, so that
the conductor may be used as a standard of resistance, with which
that of other conductors can be compared, and the comparison of
two resistances is an operation which admits of extreme accuracy.

When the unit of electrical resistance has been fixed on, material
copies of this unit, in the form of `Resistance Coils,' are prepared
for the use of electricians, so that in every part of the world
electrical resistances may be expressed in terms of the same unit.
These unit resistance coils are at present the only examples of
material electric standards which can be preserved, copied, and used
for the purpose of measurement. Measures of electrical capacity,
which are also of great importance, are still defective, on account
of the disturbing influence of electric absorption.

\article{210*} The unit of resistance may be an entirely arbitrary one,
as in the case of Jacobi's Etalon, which was a certain copper
%%-----File: 195.png-----%%
wire of 22·4932 grammes weight, 7·61975 metres length, and 0·667
millimetres diameter. Copies of this have been made by Leyser of
Leipsig, and are to be found in different places.

According to another method the unit may be defined as the
resistance of a portion of a definite substance of definite dimensions.
Thus, Siemens' unit is defined as the resistance of a column of
mercury of one metre long, and one square millimetre section, at
the temperature 0°C.
\Runhead{UNIT OF RESISTANCE.}

\article{211*} Finally, the unit may be defined with reference to the electrostatic
or the electromagnetic system of units. In practice the electromagnetic
system is used in all telegraphic operations, and therefore
the only systematic units actually in use are those of this system.

In the electromagnetic system a resistance is a quantity homogeneous
with a velocity, and may therefore be expressed as a velocity.

\article{212*} The first actual measurements on this system were made
by Weber, who employed as his unit one millimetre per second.
Sir W. Thomson afterwards used one foot per second as a unit,
but a large number of electricians have now agreed to use the
unit of the British Association, which professes to represent a
resistance which, expressed as a velocity, is ten millions of metres
per second. The magnitude of this unit is more convenient than
that of Weber's unit, which is too small. It is sometimes referred
to as the B.A. unit, but in order to connect it with the name of
the discoverer of the laws of resistance, it is called the Ohm.

\article{213*} To recollect its value in absolute measure it is useful
to know that ten millions of metres is professedly the distance
from the pole to the equator, measured along the meridian of Paris.
A body, therefore, which in one second travels along a meridian
from the pole to the equator would have a velocity which, on the
electromagnetic system, is professedly represented by an Ohm.

I say professedly, because, if more accurate researches should
prove that the Ohm, as constructed from the British Association's
material standards, is not really represented by this velocity, electricians
would not alter their standards\footnote{[Electricians have scarcely acted up to this principle in the reform of the Ohm.]}, but would apply a correction.
In the same way the metre is professedly one ten-millionth
of a certain quadrantal arc, but though this is found not to be
exactly true, the length of the metre has not been altered, but the
dimensions of the earth are expressed by a less simple number.

According to the system of the British Association, the absolute
value of the unit is \textit{originally chosen} so as to represent as nearly
%%-----File: 196.png-----%%
as possible a quantity derived from the electromagnetic absolute
system.

\article{214*} When a material unit representing this abstract quantity
has been made, other standards are constructed by copying this unit,
a process capable of extreme accuracy---of much greater accuracy
than, for instance, the copying of foot-rules from a standard foot.
\Runhead{STANDARD RESISTANCE COILS.}

\wrapfig{0.35}{196.png}{Fig. 44.}
These copies, made of the most permanent materials, are distributed
over all parts of the world, so that it is not likely that
any difficulty will be found in obtaining copies of them if the
original standards should be lost.

But such units as that of Siemens can without very great
labour be reconstructed with considerable accuracy, so that as the
relation of the Ohm to Siemens unit is known, the Ohm can be
reproduced even without having a standard to copy, though the
labour is much greater and the accuracy
much less than by the method of copying.

Finally, the Ohm may be reproduced by
the electromagnetic method by which it
was originally determined. This method,
which is considerably more laborious than
the determination of a foot from the seconds
pendulum, is probably inferior in accuracy
to that last mentioned. On the other hand,
the determination of the electromagnetic
unit in terms of the Ohm with an amount
of accuracy corresponding to the progress
of electrical science, is a most important
physical research and well worthy of being
repeated.

The actual resistance coils constructed
to represent the Ohm were made of an
alloy of two parts of silver and one of platinum
in the form of wires from ·5 millimetres
to ·8 millimetres diameter, and from
one to two metres in length. These wires
were soldered to stout copper electrodes. The wire itself was
covered with two layers of silk, imbedded in solid paraffin, and
enclosed in a thin brass case, so that it can be easily brought to
a temperature at which its resistance is accurately one Ohm.
This temperature is marked on the insulating support of the coil.
(See Fig. 44.)
%%-----File: 197.png-----%%

\Subsection{On the Forms of Resistance Coils.}

\article{215*} A Resistance Coil is a conductor capable of being easily
placed in the voltaic circuit, so as to introduce into the circuit
a known resistance.

The electrodes or ends of the coil must be such that no appreciable
error may arise from the mode of making the connexions.
For resistances of considerable magnitude it is sufficient that the
electrodes should be made of stout copper wire or rod well amalgamated
with mercury at the ends, and that the ends should be
made to press on flat amalgamated copper surfaces placed in
mercury cups.
\Runhead{FORMS OF RESISTANCE COILS.}

For very great resistances it is sufficient that the electrodes
should be thick pieces of brass, and that the connexions should
be made by inserting a wedge of brass or copper into the interval
between them. This method is found very convenient.

The resistance coil itself consists of a wire well covered with
silk, the ends of which are soldered permanently to the electrodes.

The coil must be so arranged that its temperature may be easily
observed. For this purpose the wire is coiled on a tube and
covered with another tube, so that it may be placed in a vessel
of water, and that the water may have access to the inside and the
outside of the coil.

To avoid the electromagnetic effects of the current in the coil
the wire is first doubled back on itself and then coiled on the tube,
so that at every part of the coil there are equal and opposite
currents in the adjacent parts of the wire.

When it is desired to keep two coils at the same temperature the
wires are sometimes placed side by side and coiled up together.
This method is especially useful when it is more important to
secure equality of resistance than to know the absolute value of
the resistance, as in the case of the equal arms of Wheatstone's
Bridge (Art.\ 221).

When measurements of resistance were first attempted, a resistance
coil, consisting of an uncovered wire coiled in a spiral groove
round a cylinder of insulating material, was much used. It was
called a Rheostat. The accuracy with which it was found possible
to compare resistances was soon found to be inconsistent with the
use of any instrument in which the contacts are not more perfect
than can be obtained in the rheostat. The rheostat, however, is
%%-----File: 198.png-----%%
still used for adjusting the resistance where accurate measurement
is not required.

Resistance coils are generally made of those metals whose resistance
is greatest and which vary least with temperature. German
silver fulfils these conditions very well, but some specimens are
found to change their properties during the lapse of years. Hence
for standard coils, several pure metals, and also an alloy of platinum
and silver, have been employed, and the relative resistance of these
during several years has been found constant up to the limits of
modern accuracy\footnote{
[More recent experiments indicate a small change in resistance in course of time.]}.

\article{216*} For very great resistances, such as several millions of
Ohms, the wire must be either very long or very thin, and the
construction of the coil is expensive and difficult. Hence tellurium
and selenium have been proposed as materials for constructing
standards of great resistance. A very ingenious and easy method
of construction has been lately proposed by Phillips\footnote{
\textit{Phil.\ Mag.}, July, 1870.}. On a piece
of ebonite or ground glass a fine pencil-line is drawn. The ends
of this filament of plumbago are connected to metallic electrodes,
and the whole is then covered with insulating varnish. If it
should be found that the resistance of such a pencil-line remains
constant, this will be the best method of obtaining a resistance of
several millions of Ohms.

\article{217*} There are various arrangements by which resistance coils
may be easily introduced into a circuit.

For instance, a series of coils of which the resistances are 1, 2,
4, 8, 16, \&c., arranged according to the powers of 2, may be placed
in a box in series.

\widefig{0.6}{198.png}{Fig. 45.}
The electrodes consist of stout brass plates, so arranged on the
%%-----File: 199.png-----%%
outside of the box that by inserting a brass plug or wedge between
two of them as a shunt, the resistance of the corresponding coil
may be put out of the circuit. This arrangement was introduced
by Siemens.
\Runhead{RESISTANCE BOXES.}

Each interval between the electrodes is marked with the resistance
of the corresponding coil, so that if we wish to make the
resistance box equal to 107 we express 107 in the binary scale as
\(64 + 32 + 8 + 2 + 1\) or \(1101011\). We then take the plugs out of
the holes corresponding to 64, 32, 8, 2 and 1, and leave the plugs
in 16 and 4.

This method, founded on the binary scale, is that in which the
smallest number of separate coils is needed, and it is also that
which can be most readily tested. For if we have another coil
equal to \(1\) we can test the equality of \(1\) and \(1'\), then that of \(1 + 1'\)
and \(2\), then that of \(1 + 1' + 2\) and \(4\), and so on.

The only disadvantage of the arrangement is that it requires
a familiarity with the binary scale of notation, which is not
generally possessed by those accustomed to express every number
in the decimal scale.

\article{218*} A box of resistance coils may be arranged in a different
way for the purpose of measuring
conductivities instead of
resistances.

\wrapfig{0.52}{199.png}{Fig. 46.}
The coils are placed so that
one end of each is connected
with a long thick piece of
metal which forms one electrode
of the box, and the other
end is connected with a stout piece of brass plate as in the former
case.

The other electrode of the box is a long brass plate, such that
by inserting brass plugs between it and the electrodes of the coils
it may be connected to the first electrode through any given set of
coils. The conductivity of the box is then the sum of the conductivities
of the coils.

In the figure, in which the resistances of the coils are 1, 2, 4, \&c.,
and the plugs are inserted at 2 and 8, the conductivity of the box
is \(\tstrut\frac{1}{2} + \tstrut\frac{1}{8} = \tstrut\frac{5}{8}\), and the resistance of the box is therefore \(\tstrut\frac{8}{5}\) or 1·6.

This method of combining resistance coils for the measurement
of fractional resistances was introduced by Sir W. Thomson under
the name of the method of multiple arcs. (See \hyperref[art:158*]{Art.\ 158}.)
%%-----File: 200.png-----%%

\Subsection{On the Comparison of Resistances.}

\article{219*} If \(E\) is the electromotive force of a battery, and \(R\) the
resistance of the battery and its connexions, including the galvanometer
used in measuring the current, and if the strength of the
current is \(I\) when the battery connexions are closed, and \(I_1\), \(I_2\)
when additional resistances \(r_1\), \(r_2\) are introduced into the circuit,
then, by Ohm's Law,
\Runhead{THE COMPARISON OF RESISTANCES.}
\[E = IR = I_1( R + r_1 ) = I_2( R + r_2 )\text{.}\]

Eliminating \(E\), the electromotive force of the battery, and \(R\)
the resistance of the battery and its connexions, we get Ohm's
formula
\[
\frac{r_1}{r_2} = \frac{(I-I_1)I_2}{(I-I_2)I_1}\text{.}
\]
This method requires a measurement of the ratios of \(I\), \(I_1\) and \(I_2\),
and this implies a galvanometer graduated for absolute measurements.

If the resistances \(r_1\) and \(r_2\) are equal, then \(I_1\) and \(I_2\) are equal,
and we can test the equality of currents by a galvanometer which
is not capable of determining their ratios.

But this is rather to be taken as an example of a faulty method
than as a practical method of determining resistance. The electromotive
force \(E\) cannot be maintained rigorously constant, and the
internal resistance of the battery is also exceedingly variable, so
that any methods in which these are assumed to be even for a short
time constant are not to be depended on.

\article{220*} The comparison of resistances can be made with extreme
accuracy by either of two methods, in which the result is independent
of variations of \(R\) and \(E\).
\widefig{0.54}{200.png}{Fig. 47.}
%%-----File: 201.png-----%%

The first of these methods depends on the use of the differential
galvanometer, an instrument in which there are two coils, the
currents in which are independent of each other, so that when
the currents are made to flow in opposite directions they act in
opposite directions on the needle, and when the ratio of these
currents is that of \(m\) to \(n\) they have no resultant effect on the
galvanometer needle.

Let \(I_1\), \(I_2\) be the currents through the two coils of the galvanometer,
then the deflexion of the needle may be written
\[
\delta = mI_1 - nI_2\text{.}
\]

Now let the battery current \(I\) be divided between the coils of
the galvanometer, and let resistances \(A\) and \(B\) be introduced into
the first and second coils respectively. Let the remainder of the
resistance of the coils and their connexions be \(\alpha\) and \(\beta\) respectively,
and let the resistance of the battery and its connexions
between \(C\) and \(D\) be \(r\), and its electromotive force \(E\).

Then we find, by Ohm's Law, for the difference of potentials
between \(C\) and \(D\),
\begin{gather*}
C-D = I_1(A+\alpha) = I_2(B+\beta) = E-Ir\text{,}\\
\shortintertext{and since}
I_1 + I_2 = I\text{,}\\
I_1 = E\,\frac {B + \beta}{D}\text{,}\quad I_2=E\,\frac{A+\alpha}{D}\text{,}\quad I = E\,\frac{A+\alpha+B+\beta}{D}\text{,}\\
\shortintertext{where}
D = (A + \alpha)(B + \beta) + r(A + \alpha + B + \beta)\text{.}
\end{gather*}

The deflexion of the galvanometer needle is therefore
\[
\delta =\frac{E}{D} \{m(B + \beta) - n(A + \alpha)\}\text{,}
\]
and if there is no observable deflexion, then we know that the
quantity enclosed in brackets cannot differ from zero by more than
a certain small quantity, depending on the power of the battery,
the suitableness of the arrangement, the delicacy of the galvanometer,
and the accuracy of the observer.

Suppose that \(B\) has been adjusted so that there is no apparent
deflexion.

Now let another conductor \(A'\) be substituted for \(A\), and let \(A'\) be
adjusted till there is no apparent deflexion. Then evidently to a
first approximation \(A' = A\).

To ascertain the degree of accuracy of this estimate, let the
altered quantities in the second observation be accented, then
%%-----File: 202.png-----%%
\begin{align*}
&m(B + \beta) - n( A + \alpha ) = \frac{D}{E} \delta\text{,}\\
&m(B + \beta) - n( A' + \alpha ) = \frac{D'}{E'} \delta'\text{.}\\
\shortintertext{Hence}
&n(A' - A) = \frac{D}{E} \delta - \frac{D'}{E'} \delta'\text{.}
\end{align*}

If \(\delta\) and \(\delta'\), instead of being both apparently zero, had been only
observed to be equal, then, unless we also could assert that \(E = E'\),
the right-hand side of the equation might not be zero. In fact, the
method would be a mere modification of that already described.

\hypertarget{220:1}{}
The merit of the method consists in the fact that the thing
observed is the absence of any deflexion, or in other words, the
method is a Null method, one in which the non-existence of a force
is asserted from an observation in which the force, if it had been
different from zero by more than a certain small amount, would
have produced an observable effect.

Null methods are of great value where they can be employed,
but they can only be employed where we can cause two equal and
opposite quantities of the same kind to enter into the experiment
together.

In the case before us both \(\delta\) and \(\delta'\) are quantities too small to be
observed, and therefore any change in the value of \(E\) will not affect
the accuracy of the result.
\Runhead{MEASUREMENT OF RESISTANCE.}

The actual degree of accuracy of this method might be ascertained
by taking a number of observations in each of which \(A'\)
is separately adjusted, and comparing the result of each observation
with the mean of the whole series.

But by putting \(A'\) out of adjustment by a known quantity, as,
for instance, by inserting at \(A\) or at \(B\) an additional resistance
equal to a hundredth part of \(A\) or of \(B\), and then observing
the resulting deviation of the galvanometer needle, we can estimate
the number of degrees corresponding to an error of one per cent.
To find the actual degree of precision we must estimate the smallest
deflexion which could not escape observation, and compare it with
the deflexion due to an error of one per cent.

\footnote{This investigation is taken from Weber's treatise on Galvanometry. \textit{Göttingen
Transactions}, x.\ p.\ 65.}If the comparison is to be made between \(A\) and \(B\), and if the
positions of \(A\) and \(B\) are exchanged, then the second equation
becomes
%%-----File: 203.png-----%%
\begin{gather*}
m(A+\beta) - n(B+\alpha) = \frac{D'}{E} \delta'\text{,}\\
\shortintertext{whence}
(m + n)(B - A) = \frac{D}{E} \delta - \frac{D'}{ E} \delta'\text{.}
\end{gather*}

If \(m\) and \(n\), \(A\) and \(B\), \(\alpha\) and \(\beta\) are approximately equal, then
\[
B - A = \frac{1}{2nE} (A + \alpha)(A + \alpha + 2r)(\delta - \delta')\text{.}
\]
Here \(\delta - \delta'\) may be taken to be the smallest observable deflexion
of the galvanometer.

If the galvanometer wire be made longer and thinner, retaining
the same total mass, then \(n\) will vary as the length of the wire
and \(\alpha\) as the square of the length. Hence there will be a minimum
value of \(\xp\dfrac{(A + \alpha)(A + \alpha + 2r)}{n}\) when
\[
\alpha = \tfrac{1}{3}(A + r) \left\{2 \sqrt{1 - \frac{3}{4}\frac{r^2}{(A + r)^2}}-1\right\}\text{.}
\]

If we suppose \(r\), the battery resistance, small compared with \(A\),
this gives
\[
\alpha = \tfrac{1}{3} A\text{;}
\]
or, \textit{the resistance of each coil of the galvanometer should be one-third
of the resistance to be measured.}

We then find
\[
B - A = \frac{8}{9}\frac{A^2}{nE}(\delta - \delta')\text{.}
\]

If we allow the current to flow through one only of the coils
of the galvanometer, and if the deflexion thereby produced is \(\Delta\)
(supposing the deflexion strictly proportional to the deflecting
force), then
\[
\Delta = \frac{mE}{A + \alpha + r} = \frac{3}{4}\frac{nE}{A}\text{ if }r = 0\text{ and }\alpha = \frac{1}{3} A\text{.}
\]

Hence
\[
\frac{B - A}{A} = \frac{2}{3}\frac{\delta - \delta'}{\Delta}\text{.}
\]

In the differential galvanometer two currents are made to
produce equal and opposite effects on the suspended needle. The
force with which either current acts on the needle depends not
only on the strength of the current, but on the position of the
windings of the wire with respect to the needle. Hence, unless
the coil is very carefully wound, the ratio of \(m\) to \(n\) may change
when the position of the needle is changed, and therefore it is
necessary to determine this ratio by proper methods during each
%%-----File: 204.png-----%%
course of experiments if any alteration of the position of the needle
is suspected.
\Runhead{DIFFERENTIAL GALVANOMETER.}

The other null method, in which Wheatstone's Bridge is used,
requires only an ordinary galvanometer, and the observed zero
deflexion of the needle is due, not to the opposing action of two
currents, but to the non-existence of a current in the wire. Hence
we have not merely a null deflexion, but a null current as the
phenomenon observed, and no errors can arise from want of
regularity or change of any kind in the coils of the galvanometer.
The galvanometer is only required to be sensitive enough to detect
the existence and direction of a current, without in any way
determining its value or comparing its value with that of another
current.

\article{221*} Wheatstone's Bridge consists essentially of six conductors
connecting four points. An electromotive
force \(E\) is made to act between two of the
points by means of a voltaic battery introduced
between \(B\) and \(C\). The current
between the other two points \(O\) and \(A\) is
measured by a galvanometer.

Under certain circumstances this current
becomes zero. The conductors \(BC\) and \(OA\)
are then said to be \textit{conjugate} to each other,
which implies a certain relation between the resistances of the
other four conductors, and this relation is made use of in measuring
resistances.

\wrapfig{0.3}{204.png}{Fig. 48.}
If the current in \(OA\) is zero, the potential at \(O\) must be equal
to that at \(A\). Now when we know the potentials at \(B\) and \(C\) we
can determine those at \(O\) and \(A\) by the rule given at \hyperref[art:157*]{Art.\ 157},
provided there is no current in \(OA\),
\[
  O = \frac{B \gamma + C \beta}{ \beta + \gamma}\text{,} \qquad A = \frac{Bb + Cc}{b + c}\text{,}
\]
whence the condition is
\[
  b \beta = c \gamma\text{,}
\]
where \(b\), \(c\), \(\beta\), \(\gamma\) are the resistances in \(CA\), \(AB\), \(BO\) and \(OC\) respectively.

To determine the degree of accuracy attainable by this method
we must ascertain the strength of the current in \(OA\) when this
condition is not fulfilled exactly.
\Runhead{WHEATSTONE'S BRIDGE.}

Let \(A\), \(B\), \(C\) and \(O\) be the four points. Let the currents along
\(BC\), \(CA\) and \(AB\) be \(x\), \(y\) and \(z\), and the resistances of these
%%-----File: 205.png-----%%
conductors \(a\), \(b\) and \(c\). Let the currents along \(OA\), \(OB\) and \(OC\) be
\(\xi\), \(\eta\), \(\zeta\), and the resistances \(\alpha\), \(\beta\) and \(\gamma\). Let an electromotive force
\(E\) act along \(BC\). Required the current \(\xi\) along \(OA\).

Let the potentials at the points \(A\), \(B\), \(C\) and \(O\) be denoted by
the symbols \(A\), \(B\), \(C\) and \(O\). The equations of conduction are
\begin{align*}
ax &= B-C+ E\text{,} & \alpha \xi &= O-A\text{,}\\
by &= C-A & \beta \eta &= O-B\text{,}\\
cz &= A-B & \gamma \zeta &= O-C\text{;}
\end{align*}
with the equations of continuity
\begin{align*}
\xi + y - z &= 0\text{,}\\
\eta + z - x &= 0\text{,}\\
\zeta + x - y &= 0\text{.}
\end{align*}

By considering the system as made up of three circuits \(OBC\),
\(OCA\) and \(OAB\) in which the currents are \(x\), \(y\), \(z\) respectively, and
applying Kirchhoff's rule [\hyperref[art:158*]{Art.\ 158}] to each cycle, we eliminate the
values of the potentials \(O\), \(A\), \(B\), \(C\), and the currents \(\xi\), \(\eta\), \(\zeta\), and
obtain the following equations for \(x\), \(y\) and \(z\),
\begin{align*}
(a + \beta + \gamma )x - \gamma y - \beta z &= E\text{,}\\
- \gamma x + (b + \delta + \alpha )y - \alpha z &= 0\text{,}\\
- \beta x - \alpha y + (c + \alpha + \beta )z &= 0\text{.}
\end{align*}

Hence, if we put
\[
D =
\begin{vmatrix}
\alpha + \beta + \gamma & -\gamma & - \beta  \\
-\gamma & b + \gamma + \alpha & - \alpha \\
- \beta & - \alpha & c + \alpha + \beta \\
\end{vmatrix}
\text{,}
\]
\begin{align*}
\shortintertext{we find}
\xi &= \frac{E}{D}(b\beta - c\gamma)\text{,}\\
\shortintertext{and}
x &= \frac{E}{D} \{(b + \gamma)(c + \beta) + \alpha (b + c + \beta + \gamma )\}\text{.}
\end{align*}

\article{222*} The value of \(D\) may be expressed in the symmetrical form,
\[
D = abc + bc(\beta+\gamma)+ca(\gamma+\alpha)+ab(\alpha+\beta)+(a+b+c)(\beta\gamma+\gamma\alpha+\alpha\beta)
\]
or, since we suppose the battery in the conductor \(a\) and the
galvanometer in \(\alpha\), we may put \(B\) the battery resistance for \(a\) and
\(G\) the galvanometer resistance for \(\alpha\). We then find
\begin{multline*}
D = BG(b+c+\beta+\gamma)+B(b+\gamma)(c+\beta)\\
+ G(b+c)(\beta+\gamma)+bc(\beta+\gamma)+\beta\gamma(b+c)\text{.}
\end{multline*}

If the electromotive force \(E\) were made to act along \(OA\), the
resistance of \(OA\) being still \(\alpha\), and if the galvanometer were placed
%%-----File: 206.png-----%%
in \(BC\), the resistance of \(BC\) being still \(a\), then the value of \(D\) would
remain the same, and the current in \(BC\) due to the electromotive
force \(E\) acting along \(OA\) would be equal to the current in \(OA\) due
to the electromotive force \(E\) acting in \(BC\).

But if we simply disconnect the battery and the galvanometer,
and without altering their respective resistances connect the battery
to \(O\) and \(A\) and the galvanometer to \(B\) and \(C\), then in the value of
\(D\) we must exchange the values of \(B\) and \(G\). If \(D'\) be the value of
\(D\) after this exchange, we find
\[
  \begin{aligned}
    D - D' &= (G - B) \left\{ (b + c)(\beta + \gamma) - (b + \gamma)(\beta + c) \right\}\text{,}\\
    &= (B - G) \left\{ (b - \beta)(c - \gamma) \right\}\text{.}
  \end{aligned}
\]

Let us suppose that the resistance of the galvanometer is greater
than that of the battery.

Let us also suppose that in its original position the galvanometer
connects the junction of the two conductors of least resistance \(\beta\), \(\gamma\)
with the junction of the two conductors of greatest resistance \(b\), \(c\),
or, in other words, we shall suppose that if the quantities \(b\), \(c\), \(\gamma\), \(\beta\)
are arranged in order of magnitude, \(b\) and \(c\) stand together, and
\(\gamma\) and \(\beta\) stand together. Hence the quantities \(b - \beta\) and \(c - \gamma\) are
of the same sign, so that their product is positive, and therefore
\(D - D'\) is of the same sign as \(B - G\).

If therefore the galvanometer is made to connect the junction of
the two greatest resistances with that of the two least, and if
the galvanometer resistance is greater than that of the battery,
then the value of \(D\) will be less, and the value of the deflexion of
the galvanometer greater, than if the connexions are exchanged.

The rule therefore for obtaining the greatest galvanometer deflexion
in a given system is as follows:

Of the two resistances, that of the battery and that of the
galvanometer, connect the greater resistance so as to join the two
greatest to the two least of the four other resistances.

\article{223*} We shall suppose that we have to determine the ratio of
the resistances of the conductors \(AB\) and \(AC\), and that this is to be
done by finding a point \(O\) on the conductor \(BOC\), such that when
the points \(A\) and \(O\) are connected by a wire, in the course of which
a galvanometer is inserted, no sensible deflexion of the galvanometer
needle occurs when the battery is made to act between \(B\)
and \(C\).

The conductor \(BOC\) may be supposed to be a wire of uniform
resistance divided into equal parts, so that the ratio of the resistances
of \(BO\) and \(OC\) may be read off at once.
%%-----File: 207.png-----%%

Instead of the whole conductor being a uniform wire, we may
make the part near \(O\) of such a wire, and the parts on each side
may be coils of any form, the resistance of which is accurately
known.

We shall now use a different notation instead of the symmetrical
notation with which we commenced.

Let the whole resistance of \(BAC\) be \(R\).

Let \(c = mR\) and \(b = (1 - m)R\).

Let the whole resistance of \(BOC\) be \(S\).

Let \(\beta = nS\) and \(\gamma = (1 - n)S\).

The value of \(n\) is read off directly, and that of \(m\) is deduced from
it when there is no sensible deviation of the galvanometer.

Let the resistance of the battery and its connexions be \(B\), and
that of the galvanometer and its connexions \(G\).

We find as before
\begin{multline*}
D = G\{BR + BS + RS\} + m(1 - m)R^2(B + S) + n(1 - n)S^2(B + R)\\
+ (m + n - 2mn)BRS\text{,}
\end{multline*}
and if \(\xi\) is the current in the galvanometer wire
\[
\xi = \frac{ERS}{D}(n - m)\text{.}
\]

In order to obtain the most accurate results we must make the
deviation of the needle as great as possible compared with the
value of \((n - m)\). This may be done by properly choosing the
dimensions of the galvanometer and the standard resistance wire.

It may be shewn that when the form of a galvanometer wire
is changed while its mass remains constant, the deviation of the
needle for unit current is proportional to the length, but the
resistance increases as the square of the length. Hence the
maximum deflexion is shewn to occur when the resistance of the
galvanometer wire is equal to the constant resistance of the rest
of the circuit.

In the present case, if \(\delta\) is the deviation,
\[
\delta = C \sqrt{G} \xi\text{,}
\]
where \(C\) is some constant, and \(G\) is the galvanometer resistance
which varies as the square of the length of the wire. Hence we
find that in the value of \(D\), when \(\delta\) is a maximum, the part involving
\(G\) must be made equal to the rest of the expression.

If we also put \(m = n\), as is the case if we have made a correct
observation, we find the best value of \(G\) to be
\[
G = n(1 - n)(R + S)\text{.}
\]
%%-----File: 208.png-----%%

This result is easily obtained by considering the resistance from
\(A\) to \(O\) through the system, remembering that \(BC\), being conjugate
to \(AO\), has no effect on this resistance.

In the same way we should find that if the total area of the
acting surfaces of the battery is given, the most advantageous arrangement
of the battery is when
\[
B =\frac{RS}{R + S}\text{.}
\]

Finally, we shall determine the value of \(S\) such that a given
change in the value of \(n\) may produce the greatest galvanometer
deflexion. By differentiating the expression for \(\xi\) we find
\[
S^2 = \frac{BR}{B + R}\left(R +\frac{G}{n(1 - n)}\right)\text{.}
\]

If we have a great many determinations of resistance to make
in which the actual resistance has nearly the same value, then it
may be worth while to prepare a galvanometer and a battery for
this purpose. In this case we find that the best arrangement is
\[
S = R\text{,}\qquad B = \tfrac{1}{2} R\text{,}\qquad G = 2n(1-n)R\text{,}
\]
and if \(n = \tstrut\frac{1}{2}\),  \(G = \tstrut\frac{1}{2} R\).

\Subsection{On the Use of Wheatstone's Bridge.}

\article{224*} We have already explained the general theory of Wheatstone's
Bridge, we shall now consider some of its applications.

\widefig{0.57}{208.png}{Fig. 49.}
The comparison which can be effected with the greatest exactness
is that of two equal resistances.
\Runhead{USE OF WHEATSTONE'S BRIDGE.}
%%-----File: 209.png-----%%

Let us suppose that \(\beta\) is a standard resistance coil, and that we
wish to adjust \(\gamma\) to be equal in resistance to \(\beta\).

Two other coils, \(b\) and \(c\), are prepared which are equal or nearly
equal to each other, and the four coils are placed with their electrodes
in mercury cups so that the current of the battery is divided
between two branches, one consisting of \(\beta\) and \(\gamma\) and the other
of \(b\) and \(c\). The coils \(b\) and \(c\) are connected by a wire \(PR\), as
uniform in its resistance as possible, and furnished with a scale of
equal parts.

The galvanometer wire connects the junction of \(\beta\) and \(\gamma\) with
a point \(Q\) of the wire \(PR\), and the point of contact at \(Q\) is made
to vary till on closing first the battery circuit and then the
galvanometer circuit, no deflexion of the galvanometer needle is
observed.

The coils \(\beta\) and \(\gamma\) are then made to change places, and a new
position is found for \(Q\). If this new position is the same as the
old one, then we know that the exchange of \(\beta\) and \(\gamma\) has produced
no change in the proportions of the resistances, and therefore \(\gamma\)
is rightly adjusted. If \(Q\) has to be moved, the direction and
amount of the change will indicate the nature and amount of the
alteration of the length of the wire of \(\gamma\), which will make its resistance
equal to that of \(\beta\).

If the resistances of the coils \(b\) and \(c\), each including part of the
wire \(PR\) up to its zero reading, are equal to that of \(b\) and \(c\) divisions
of the wire respectively, then, if \(x\) is the scale reading of \(Q\) in the
first case, and \(y\) that in the second,
\begin{gather*}
\frac{c + x}{b - x} = \frac{\beta}{\gamma}\text{,} \qquad \frac{c + y}{b - y} = \frac{\gamma}{\beta}\text{,}\\
\shortintertext{whence}
\frac{\gamma ^2}{\beta ^2} = 1 + \frac{(b+c)(y-x)}{(c+x)(b-y)}\text{.}
\end{gather*}

Since \(b - y\) is nearly equal to \(c + x\), and both are great with
respect to \(x\) or \(y\), we may write this
\begin{align*}
\frac{\gamma^2}{\beta^2} &= 1 + 4\frac{y - x}{b + c}\text{,}\\
\shortintertext{and}
\gamma &= \beta\left(1 + 2 \frac{y - x}{b + c}\right)\text{.}
\end{align*}

When \(\gamma\) is adjusted as well as we can, we substitute for \(b\) and \(c\)
other coils of (say) ten times greater resistance.

The remaining difference between \(\beta\) and \(\gamma\) will now produce
a ten times greater difference in the position of \(Q\) than with the
%%-----File: 210.png-----%%
original coils \(b\) and \(c\), and in this way we can continually increase
the accuracy of the comparison.

The adjustment by means of the wire with sliding contact piece
is more quickly made than by means of a resistance box, and it is
capable of continuous variation.

The battery must never be introduced instead of the galvanometer
into the wire with a sliding contact, for the passage of a
powerful current at the point of contact would injure the surface
of the wire. Hence this arrangement is adapted for the case in
which the resistance of the galvanometer is greater than that of the
battery.

When \(\gamma\), the resistance to be measured, \(a\), the resistance of the
battery, and \(\alpha\), the resistance of the galvanometer, are given, the
best values of the other resistances have been shewn by Mr.\ Oliver
Heaviside (\textit{Phil.\ Mag.}, Feb.\ 1873) to be
\[
c=\sqrt{a\alpha}\text{,} \qquad b=\sqrt{a\gamma\frac{\alpha + \gamma}{a + \gamma}}\text{,} \qquad \beta = \sqrt{\alpha \gamma \frac{a + \gamma }{\alpha + \gamma}}\text{.}
\]

\Subsection{Thomson's\footnote{
\textit{Proc.\ R. S.}, Jan.\ 19, 1871.} Method for the Determination of the Resistance of
the Galvanometer.}

\article{225*} An arrangement similar to Wheatstone's Bridge has been
employed with advantage by
Sir W. Thomson in determining
the resistance of the galvanometer
when in actual use.
It was suggested to Sir W.
Thomson by Mance's Method.
(See \hyperref[art:226*]{Art.\ 226}.)

Let the battery be placed,
as before, between \(B\) and \(C\)
in the figure of Article 221,
but let the galvanometer be
placed in \(CA\) instead of in
\(OA\). If \(b \beta - c \gamma\) is zero, then
the conductor \(OA\) is conjugate
to \(BC\), and, as there is no current
produced in \(OA\) by the battery in \(BC\), the strength of the
current in any other conductor is independent of the resistance
%%-----File: 211.png-----%%
in \(OA\). Hence, if the galvanometer is placed in \(CA\) its deflexion
will remain the same whether the resistance of \(OA\) is small or
great. We therefore observe whether the deflexion of the galvanometer
remains the same when \(O\) and \(A\) are joined by a conductor
of small resistance, as when this connexion is broken, and if, by
properly adjusting the resistances of the conductors, we obtain this
result, we know that the resistance of the galvanometer is
\[
b = \frac{c\gamma}{\beta}\text{,}
\]
where \(c\), \(\gamma\), and \(\beta\) are resistance coils of known resistance.

\wrapfig{0.52}{210.png}{Fig. 50.}
It will be observed that though this is not a null method, in the
sense of there being no current in the galvanometer, it is so in
the sense of the fact observed being the negative one, that the
deflexion of the galvanometer is not changed when a certain contact
is made. An observation of this kind is of greater value
than an observation of the equality of two different deflexions of
the same galvanometer, for in the latter case there is time for
alteration in the strength of the battery or the sensitiveness of
the galvanometer, whereas when the deflexion remains constant,
in spite of certain changes which we can repeat at pleasure, we are
sure that the current is quite independent of these changes.

The determination of the resistance of the coil of a galvanometer
can easily be effected in the ordinary way of using Wheatstone's
Bridge by placing another galvanometer in \(OA\). By the method
now described the galvanometer itself is employed to measure its
own resistance.

\Subsection{Mance's\footnote{
\textit{Proc.\ R. S.}, Jan.\ 19, 1871.} Method of determining the Resistance of the Battery.}

\article{226*} The measurement of the resistance of a battery when in
action is of a much higher order of difficulty, since the resistance
of the battery is found to change considerably for some time after
the strength of the current through it is changed. In many of the
methods commonly used to measure the resistance of a battery such
alterations of the strength of the current through it occur in the
course of the operations, and therefore the results are rendered
doubtful.
\Runhead{MANCE'S METHOD.}

In Mance's method, which is free from this objection, the battery
is placed in \(BC\) and the galvanometer in \(CA\). The connexion
between \(O\) and \(B\) is then alternately made and broken.
%%-----File: 212.png-----%%

If the deflexion of the galvanometer remains unaltered, we know
that \(OB\) is conjugate to \(CA\), whence \(c\gamma = a\alpha\), and \(a\), the resistance
of the battery, is obtained in terms of known resistances \(c\), \(\gamma\), \(\alpha\).

\widefig{0.54}{212.png}{Fig. 51.}
When the condition \(c\gamma=a\alpha\) is fulfilled, then the current through
the galvanometer is
\[
y = \frac{E\alpha}{b\alpha + c(b+\alpha+\gamma)}\text{,}
\]
and this is independent of the resistance \(\beta\) between \(O\) and \(B\). To
test the sensibility of the method let us suppose that the condition
\(c\gamma=a\alpha\) is nearly, but not accurately, fulfilled, and that \(y_0\) is the
current through the galvanometer when \(O\) and \(B\) are connected
by a conductor of no sensible resistance, and \(y_1\) the current when
\(O\) and \(B\) are completely disconnected.

To find these values we must make \(\beta\) equal to 0 and to \(\infty\) in the
general formula for \(y\), and compare the results.

In this way we find
\[
\frac{y_0-y_1}{y} = \frac{\alpha}{\gamma}\frac{c\gamma-a\alpha}{(c+\alpha )(\alpha+\gamma)}\text{,}
\]
where \(y_0\) and \(y_1\) are supposed to be so nearly equal that we may,
when their difference is not in question, put either of them equal
to \(y\), the value of the current when the adjustment is perfect.

The resistance, \(c\), of the conductor \(AB\) should be equal to \(a\),
that of the battery, \(\alpha\) and \(\gamma\), should be equal and as small as
possible, and \(b\) should be equal to \(\alpha + \gamma\).

Since a galvanometer is most sensitive when its deflexion is
small, we should bring the needle nearly to zero by means of fixed
magnets before making contact between \(O\) and \(B\).

In this method of measuring the resistance of the battery, the
current in the battery is not in any way interfered with during the
operation, so that we may ascertain its resistance for any given
%%-----File: 213.png-----%%
strength of current, so as to determine how the strength of current
affects the resistance.

If \(y\) is the current in the galvanometer, the actual current through
the battery is \(x_0\) with the key down and \(x_1\) with the key up, where
\[
  x_1 = y \left(1 + \frac{b}{\alpha + \gamma} \right)\text{,} \qquad
  x_0 = y \left(1 + \frac{b}{\gamma} + \frac{\alpha c}{\gamma (\alpha + c)} \right)\text{,}
\]
the resistance of the battery is
\[
  a = \frac{c \gamma}{\alpha},
\]
and the electromotive force of the battery is
\[
  E = y \left( b + c + \frac{c}{\alpha} ( b + \gamma ) \right)\text{.\footnotemark}
\]
\footnotetext{
[This method, as has been pointed out by Professor Oliver Lodge, is not free from
error on account of the variation of the E.M.F. of the battery, as the current through
it is diminished or increased by raising or depressing the key.]}

The method of \hyperref[art:225*]{Art.\ 225} for finding the resistance of the galvanometer
differs from this only in making and breaking contact
between \(O\) and \(A\) instead of between \(O\) and \(B\), and by exchanging
\(\alpha\) and \(\beta\) we obtain for this case
\[
\frac{y_0 - y_1}{y} = \frac{\beta}{\gamma} \frac{c \gamma - b \beta}{(c + \beta)(\beta + \gamma)}\text{.}
\]

\Subsection{On the Comparison of Electromotive Forces.}

\article{227*} The following method of comparing the electromotive
forces of voltaic and thermoelectric arrangements, when no current
passes through them, requires only a set of resistance coils and a
constant battery.
\Runhead{COMPARISON OF ELECTROMOTIVE FORCES.}

\widefig{0.51}{213.png}{Fig. 52.}
Let the electromotive force \(E\) of the battery be greater than that
of either of the electromotors to be compared, then, if a sufficient
%%-----File: 214.png-----%%
resistance, \(R_1\), be interposed between the points \(A_1\), \(B_1\) of the
primary circuit \(EB_1A_1E\), the electromotive force from \(B_1\) to \(A_1\)
may be made equal to that of the electromotor \(E_1\). If the electrodes
of this electromotor are now connected with the points
\(A_1\), \(B_1\) no current will flow through the electromotor. By placing
a galvanometer \(G_1\) in the circuit of the electromotor \(E_1\), and
adjusting the resistance between \(A_1\) and \(B_1\), till the galvanometer
\(G_1\) indicates no current, we obtain the equation
\[
  E_1 = R_1C\text{,}
\]
where \(R_1\) is the resistance between \(A_1\) and \(B_1\), and \(C\) is the strength
of the current in the primary circuit.

In the same way, by taking a second electromotor \(E_2\) and placing
its electrodes at \(A_2\) and \(B_2\), so that no current is indicated by the
galvanometer \(G_2\),
\[
  E_2 = R_2C\text{,}
\]
where \(R_2\) is the resistance between \(A_2\) and \(B_2\). If the observations
of the galvanometers \(G_1\) and \(G_2\) are simultaneous, the value of \(C\),
the current in the primary circuit, is the same in both equations,
and we find
\[
 E_1 : E_2 :: R_1 : R_2\text{.}
\]

In this way the electromotive force of two electromotors may be
compared\footnote{
[Any number of batteries may be compared by the help of only one galvanometer
if one pole of each battery is connected with the same electrode of the galvanometer,
the other poles being connected through separate keys to points \(A_1\), \(A_2\), \&c.\ upon
the wire and the keys being depressed one at a time but in rapid succession.]
}. The absolute electromotive force of an electromotor
may be measured either electrostatically by means of the electrometer,
or electromagnetically by means of an absolute galvanometer.

This method, in which, at the time of the comparison, there
is no current through either of the electromotors, is a modification
of Poggendorff's method, and is due to Mr.\ Latimer Clark, who
has deduced the following values of electromotive forces:
\Runhead{POGGENDORFF'S COMPENSATION METHOD.}

\newpage
\begin{center}
\footnotesize
\begin{tabular}{l@{ }l@{ }c@{ }c@{ }r>{\centering}p{1.7cm}l@{ }l}
& & & & & \multicolumn{1}{m{1.7cm}}{\scriptsize \centering Concentrated solution of} & & {\scriptsize Volts.}\\
\textit{Daniell} & I. & Amalgamated Zinc & \ce{H2SO4} + & 4 aq. & \ce{CuSO4} & Copper & = 1.079\\
& II. & „ & \ce{H2SO4} + & 12 aq. & \ce{CuSO4} & Copper & = 0.978\\
& III. & „ & \ce{H2SO4} + & 12 aq. & \ce{Cu2(NO3)} & Copper & = 1.00\\
\textit{Bunsen} & I. & „ & „ & \multicolumn{1}{c}{„} & \ce{HNO3} & Carbon & = 1.964\\
& II. & „ & „ & \multicolumn{1}{c}{„} & sp.\ g.\ l. 38 & Carbon & = 1.888\\
\textit{Grove} & & „ & \ce{H2SO4} + & 4 aq. & \ce{HNO3} & Platinum & = 1.956\\
\end{tabular}
\end{center}

\textit{\footnotesize A Volt is an electromotive force equal to 100,000,000 units of the centimetre-gramme-second
system.}
%%-----File: 215.png-----%%

\newchapter
\Chapter{CHAPTER XIII.}
\Subheading{ON THE ELECTRIC RESISTANCE OF SUBSTANCES.}

\article{228*} \textsc{There} are three classes in which we may place different
substances in relation to the passage of electricity through them.

The first class contains all the metals and their alloys, some
sulphurets, and other compounds containing metals, to which we
must add carbon in the form of gas-coke, and selenium in the
crystalline form.

In all these substances conduction takes place without any
decomposition, or alteration of the chemical nature of the substance,
either in its interior or where the current enters and leaves the
body. In all of them the resistance increases as the temperature
rises.

The second class consists of substances which are called electrolytes,
because the current is associated with a decomposition of
the substance into two components which appear at the electrodes.
As a rule a substance is an electrolyte only when in the liquid
form, though certain colloid substances, such as glass at 100°C,
which are apparently solid, are electrolytes. It would appear from
the experiments of Sir B. C. Brodie that certain gases are capable
of electrolysis by a powerful electromotive force.

In all substances which conduct by electrolysis the resistance
diminishes as the temperature rises.

The third class consists of substances the resistance of which is
so great that it is only by the most refined methods that the
passage of electricity through them can be detected. These are
called Dielectrics. To this class belong a considerable number
of solid bodies, many of which are electrolytes when melted, some
liquids, such as turpentine, naphtha, melted paraffin, \&c., and all
gases and vapours. Carbon in the form of diamond, and selenium
in the amorphous form, belong to this class.

The resistance of this class of bodies is enormous compared with
that of the metals. It diminishes as the temperature rises. It
%%-----File: 216.png-----%%
is difficult, on account of the great resistance of these substances,
to determine whether the feeble current which we can force through
them is or is not associated with electrolysis.

\Subsection{On the Electric Resistance of Metals.}

\article{229*} There is no part of electrical research in which more
numerous or more accurate experiments have been made than in
the determination of the resistance of metals. It is of the utmost
importance in the electric telegraph that the metal of which the
wires are made should have the smallest attainable resistance.
Measurements of resistance must therefore be made before selecting
the materials. When any fault occurs in the line, its position is
at once ascertained by measurements of resistance, and these measurements,
in which so many persons are now employed, require
the use of resistance coils, made of metal the electrical properties
of which have been carefully tested.
\Runhead{RESISTANCE OF METALS.}

The electrical properties of metals and their alloys have been
studied with great care by MM. Matthiessen, Vogt, and Hockin,
and by MM. Siemens, who have done so much to introduce exact
electrical measurements into practical work.

It appears from the researches of Dr.\ Matthiessen, that the effect
of temperature on the resistance is nearly the same for a considerable
number of the \textit{pure} metals, the resistance at 100°C being to that
at 0°C in the ratio of 1.414 to 1, or of 1 to .707. For pure iron
the ratio is 1.645, and for pure thallium 1.458.

The resistance of metals has been observed by Dr.\ C. W. Siemens\footnote{
\textit{Proc.\ R. S.}, April 27, 1871.}
through a much wider range of temperature, extending from the
freezing point to 350°C, and in certain cases to 1000°C. He finds
that the resistance increases as the temperature rises, but that the
rate of increase diminishes as the temperature rises. The formula,
which he finds to agree very closely both with the resistances
observed at low temperatures by Dr.\ Matthiessen and with his
own observations through a range of 1000°C, is
\[
  r = \alpha T^{\frac{1}{2}} + \beta T + \gamma\text{,}
\]
where \(T\) is the absolute temperature reckoned from \(-273\)°C, and
\(\alpha\), \(\beta\), \(\gamma\) are constants. Thus, for
\begin{align*}
&\text{Platinum} & r &= 0.039369T^{\frac{1}{2}} + 0.00216407T - 0.2413\text{,}\\
&\text{Copper} & r &= 0.026577T^{\frac{1}{2}} + 0.0031443T - 0.22751\text{,}\\
&\text{Iron} & r &= 0.072545T^{\frac{1}{2}} + 0.0038133T - 1.23971\text{.}
\end{align*}
%%-----File: 217.png-----%%

From data of this kind the temperature of a furnace may be
determined by means of an observation of the resistance of a
platinum wire placed in the furnace.

Dr.\ Matthiessen found that when two metals are combined to
form an alloy, the resistance of the alloy is in most cases greater
than that calculated from the resistance of the component metals
and their proportions. In the case of alloys of gold and silver, the
resistance of the alloy is greater than that of either pure gold or
pure silver, and, within certain limiting proportions of the constituents,
it varies very little with a slight alteration of the proportions.
For this reason Dr.\ Matthiessen recommended an alloy
of two parts by weight of gold and one of silver as a material
for reproducing the unit of resistance.

The effect of change of temperature on electric resistance is
generally less in alloys than in pure metals.

Hence ordinary resistance coils are made of German silver, on
account of its great resistance, and its small variation with temperature.

An alloy of silver and platinum is also used for standard coils.

\article{230*} In the following table \(R\) is the resistance in Ohms of a
column one metre long and one gramme weight at 0°C, and \(r\) is
the resistance in centimetres per second of a cube of one centimetre,
according to the experiments of Matthiessen\footnote{\textit{Phil.\ Mag.}, May, 1865.}.

\begin{center}
\footnotesize
\begin{tabular}{l S[table-format = 2.3] c S[table-format = 2.4]
  S[table-format = 5.0] S[table-format = 1.3]}
& \multicolumn{1}{b{1.2cm}}{\scriptsize \centering Specific gravity} & &
\multicolumn{1}{b{1.3cm}}{\scriptsize \centering \(R\)} &
\multicolumn{1}{b{1.3cm}}{\scriptsize \centering \(r\)} &
\multicolumn{1}{b{1.7cm}}{\scriptsize \centering Percentage increment of resistance for 1°C at 20°C.}\\
Silver           & 10.50 & hard drawn     &  0.1689 & 1609 &   0.377\\
Copper           &  8.95 & hard drawn     &  0.1469 & 1642 &   0.388\\
Gold             & 19.27 & hard drawn     &  0.4150 & 2154 &   0.365\\
Lead             & 11.391 &  pressed      &   2.257 & 19847 &  0.387\\
Mercury          & 13.595 & liquid        &  13.071 & 96146 &  0.072\\
Gold 2, Silver 1 & 15.218 & hard or annealed & 1.668 & 10988 & 0.065\\
Selenium at 100°C & & Crystalline form & &
\multicolumn{1}{S[table-format = 1e2]}{6e13} & 1.00
\end{tabular}
\end{center}

It appears from the researches of Matthiessen and Hockin that
the resistance of a uniform column of mercury of one metre in
length, and weighing one gramme at 0°C, is 13.071 Ohms, whence
it follows that if the specific gravity of mercury is 13·595, the
resistance of a column of one metre in length and one square
millimetre in section is 0.96146 Ohms.
%%-----File: 218.png-----%%

\Subsection{On the Electric Resistance of Electrolytes.}

\article{231*} The measurement of the electric resistance of electrolytes
is rendered difficult on account of the polarization of the electrodes,
which causes the observed difference of potentials of the metallic
electrodes to be greater than the electromotive force which actually
produces the current.
\Runhead{RESISTANCE OF ELECTROLYTES.}

This difficulty can be overcome in various ways. In certain
cases we can get rid of polarization by using electrodes of proper
material, as, for instance, zinc electrodes in a solution of sulphate
of zinc. By making the surface of the electrodes very large compared
with the section of the part of the electrolyte whose resistance
is to be measured, and by using only currents of short duration
in opposite directions alternately, we can make the measurements
before any considerable intensity of polarization has been excited
by the passage of the current.

Finally, by making two different experiments, in one of which
the path of the current through the electrolyte is much longer than
in the other, and so adjusting the electromotive force that the
actual current, and the time during which it flows, are nearly the
same in each case, we can eliminate the effect of polarization
altogether.

\article{232*} In the experiments of Dr.\ Paalzow\footnote{
Berlin \textit{Monatsbericht}, July, 1868.} the electrodes were
in the form of large disks placed in separate flat vessels filled with
the electrolyte, and the connexion was made by means of a long
siphon filled with the electrolyte and dipping into both vessels.
Two such siphons of different lengths were used.

The observed resistances of the electrolyte in these siphons
being \(R_1\) and \(R_2\), the siphons were next filled with mercury, and
their resistances when filled with mercury were found to be \({R_1}'\)
and \({R_2}'\).

The ratio of the resistance of the electrolyte to that of a mass
of mercury at 0°C of the same form was then found from the
formula
\[
  \rho = \frac{R_1 - R_2}{{R_1}' - {R_2}'}.
\]

To deduce from the values of \(\rho\) the resistance of a centimetre in
%%-----File: 219.png-----%%
length having a section of a square centimetre, we must multiply
them by the value of \(r\) for mercury at 0°C. (See \hyperref[art:230*]{Art.\ 230}.)

The results given by Paalzow are as follow:
\begin{center}
\small
\phantomsection\label{232:1}
\begin{tabular}{l@{ } c @{ }r c S[table-format = 6.0]}
\multicolumn{5}{c}{\textit{Mixtures of Sulphuric Acid and Water.}}\\
& & & \multicolumn{1}{m{1.7cm}}{\footnotesize \centering Temp.}
& \multicolumn{1}{m{3.2cm}}{\footnotesize \centering Resistance compared with mercury.}\\
\ce{H2SO4} & & &               15°C & 96950\\
\ce{H2SO4} & + & 14 \ce{H2O} &  19°C & 14157\\
\ce{H2SO4} & + & 13 \ce{H2O} &  22°C & 13310\\
\ce{H2SO4} & + & 499 \ce{H2O} & 22°C & 184773\\[1.2ex]
\multicolumn{5}{c}{\textit{Sulphate of Zinc and Water.}}\\[0.5ex]
\ce{ZnSO4} & + & 23 \ce{H2O} & 23°C & 194400\\
\ce{ZnSO4} & + & 24 \ce{H2O} & 23°C & 191000\\
\ce{ZnSO4} & + & 105 \ce{H2O} & 23°C & 354000\\[1.2ex]
\multicolumn{5}{c}{\textit{Sulphate of Copper and Water.}}\\[0.5ex]
\ce{CuSO4} & + & 45 \ce{H2O} & 22°C & 202410\\
\ce{CuSO4} & + & 105 \ce{H2O} & 22°C & 339341\\[1.2ex]
\multicolumn{5}{c}{\textit{Sulphate of Magnesium and Water.}}\\[0.5ex]
\ce{MgSO4} & + & 34 \ce{H2O} & 22°C & 199180\\
\ce{MgSO4} & + & 107 \ce{H2O} & 22°C & 324600\\[1.2ex]
\multicolumn{5}{c}{\textit{Hydrochloric Acid and Water.}}\\[0.5ex]
\ce{HCl} & + & 15 \ce{H2O} & 23°C & 13626\\
\ce{HCl} & + & 500 \ce{H2O} & 23°C & 86679
\end{tabular}
\end{center}

\article{233*} MM. F. Kohlrausch and W. A. Nippoldt\footnote{
\textit{Pogg.\ Ann.}\ cxxxviii, p.\ 286, Oct.\ 1869.} have determined
the resistance of mixtures of sulphuric acid and water.
They used alternating magneto-electric currents, the electromotive
force of which varied from \(\tstrut\frac{1}{2}\) to \(\tstrut\frac{1}{74}\) of that of a Grove's cell, and
by means of a thermoelectric copper-iron pair they reduced the
electromotive force to \(\tstrut\frac{1}{429000}\) that of a Grove's cell. They found
that Ohm's law was applicable to this electrolyte throughout the
range of these electromotive forces.

The resistance is a minimum in a mixture containing about one-third
of sulphuric acid.

The resistance of electrolytes diminishes as the temperature
increases. The percentage increment of conductivity for a rise of
1°C is given in the following table:
%%-----File: 220.png-----%%
\begin{table}[ht]
\centering
\footnotesize
\captionsetup{margin=2em,font=small,justification=centering}
\caption*{\textit{Resistance of Mixtures of Sulphuric Acid and Water at 22°C in terms
of Mercury at 0°C.} MM. Kohlrausch and Nippoldt.}
\begin{tabular}{S[table-format = 2.4] S[table-format = 3.2]
  S[table-format = 7.0] S[table-format = 2.3]}
\multicolumn{1}{m{2.4cm}}{\footnotesize \centering Specific gravity at 18°5} &
\multicolumn{1}{m{2.2cm}}{\footnotesize \centering Percentage of \ce{H2SO4}} &
\multicolumn{1}{m{2.2cm}}{\footnotesize \centering Resistance at 22°C (Hg = 1)} &
\multicolumn{1}{m{2.2cm}}{\footnotesize \centering Percentage increment of conductivity for 1°C}\\
0.9985 & 0.0  & 746300 & 0.47\\
1.00   & 0.2  & 465100 & 0.47\\
1.0504 &  8.3 &  34530 & 0.653\\
1.0989 & 14.2 &  18946 & 0.646\\
1.1431 & 20.2 &  14990 & 0.799\\
1.2045 & 28.0 &  13133 & 1.317\\
1.2631 & 35.2 &  13132 & 1.259\\
1.3163 & 41.5 &  14286 & 1.410\\
1.3547 & 46.0 &  15762 & 1.674\\
1.3994 & 50.4 &  17726 & 1.582\\
1.4482 & 55.2 &  20796 & 1.417\\
1.5026 & 60.3 &  25574 & 1.794
\end{tabular}
\end{table}

\Subsection{On the Electrical Resistance of Dielectrics.}

\article{234*} A great number of determinations of the resistance of
gutta-percha, and other materials used as insulating media, in the
manufacture of telegraphic cables, have been made in order to
ascertain the value of these materials as insulators.
\Runhead{RESISTANCE OF DIELECTRICS.}

The tests are generally applied to the material after it has been
used to cover the conducting wire, the wire being used as one
electrode, and the water of a tank, in which the cable is plunged,
as the other. Thus the current is made to pass through a cylindrical
coating of the insulator of greater area and small thickness.

It is found that when the electromotive force begins to act, the
current, as indicated by the galvanometer, is by no means constant.
\hypertarget{234:1}{}
The first effect is of course a transient current of considerable
intensity, the total quantity of electricity being that required to
charge the surfaces of the insulator with the superficial distribution
of electricity corresponding to the electromotive force. This first
current therefore is a measure not of the conductivity, but of the
capacity of the insulating layer.

But even after this current has been allowed to subside the
residual current is not constant, and does not indicate the true
conductivity of the substance. It is found that the current continues
to decrease for at least half an hour, so that a determination
of the resistance deduced from the current will give a greater value
if a certain time is allowed to elapse than if taken immediately
after applying the battery.
%%-----File: 221.png-----%%

Thus, with Hooper's insulating material the apparent resistance
at the end of ten minutes was four times, and at the end of
nineteen hours twenty-three times that observed at the end of
one minute. When the direction of the electromotive force is
reversed, the resistance falls as low or lower than at first and
then gradually rises.

These phenomena seem to be due to a condition of the gutta-percha,
which, for want of a better name, we may call polarization,
and which we may compare on the one hand with that of a series
of Leyden jars charged by cascade, and, on the other, with Ritter's
secondary pile.

If a number of Leyden jars of great capacity are connected in
series by means of conductors of great resistance (such as wet
cotton threads in the experiments of M. Gaugain), then an electromotive
force acting on the series will produce a current, as
indicated by a galvanometer, which will gradually diminish till
the jars are fully charged.

The apparent resistance of such a series will increase, and if the
dielectric of the jars is a perfect insulator it will increase without
limit. If the electromotive force be removed and connexion made
between the ends of the series, a reverse current will be observed,
the total quantity of which, in the case of perfect insulation, will be
the same as that of the direct current. Similar effects are observed
in the case of the secondary pile, with the difference that the final
insulation is not so good, and that the capacity per unit of surface
is immensely greater.

In the case of the cable covered with gutta-percha, \&c., it is
found that after applying the battery for half an hour, and then
connecting the wire with the external electrode, a reverse current
takes place, which goes on for some time, and gradually reduces
the system to its original state.

These phenomena are of the same kind with those indicated
by the `residual discharge' of the Leyden jar, except that the
amount of the polarization is much greater in gutta-percha, \&c.\
than in glass.

This state of polarization seems to be a directed property of the
material, which requires for its production not only electromotive
force, but the passage, by displacement or otherwise, of a considerable
quantity of electricity, and this passage requires a considerable
time. When the polarized state has been set up, there
is an internal electromotive force acting in the substance in the
%%-----File: 222.png-----%%
reverse direction, which will continue till it has either produced
a reversed current equal in total quantity to the first, or till the
state of polarization has quietly subsided by means of true conduction
through the substance.

The whole theory of what has been called residual discharge,
absorption of electricity, electrification, or polarization, deserves
a careful investigation, and will probably lead to important discoveries
relating to the internal structure of bodies.

\article{235*} The resistance of the greater number of dielectrics diminishes
as the temperature rises.

Thus the resistance of gutta-percha is about twenty times as great
at 0°C as at 24°C. Messrs. Bright and Clark have found that the
following formula gives results agreeing with their experiments.
If \(r\) is the resistance of gutta-percha at temperature \(T\) centigrade,
then the resistance at temperature \(T + t\) will be
\[
R = r \times 0.8878^t\text{,}
\]
the number varies between 0.8878 and 0.9.

Mr.\ Hockin has verified the curious fact that it is not until some
hours after the gutta-percha has taken its temperature that the
resistance reaches its corresponding value.

The effect of temperature on the resistance of india-rubber is not
so great as on that of gutta-percha.

The resistance of gutta-percha increases considerably on the application
of pressure.

The resistance, in Ohms, of a cubic metre of various specimens of
gutta-percha used in different cables is as follows\footnote{Jenkin's \textit{Cantor Lectures}.}.

\begin{center}
\sisetup{output-decimal-marker = {.}}
\begin{tabular}{p{5cm} S[table-format = 2.3e2]@{ }c@{ }S[table-format = 1.3e2]}
\multicolumn{1}{p{5cm}}{\small \centering Name of Cable.} &&&\\
Red Sea \dotfill & .267e12 & to & .362e12\\
Malta-Alexandria \dotfill & 1.23e12&&\\
Persian Gulf \dotfill & 1.80e12&&\\
Second Atlantic \dotfill & 3.42e12&&\\
Hooper's Persian Gulf Core \dotfill & 74.7e12&&\\
Gutta-percha at 24°C \dotfill & 3.53e12
\end{tabular}
\end{center}

\article{236*} The following table, calculated from the experiments of
M. Buff\footnote{
[\textit{Annalen der Chemie und Pharmacie}, bd.\ xc.\ 257 (1854).]}, shews the resistance of a cubic metre of glass in Ohms
at different temperatures:
%%-----File: 223.png-----%%

\begin{center}
\begin{tabular}{>{\raggedleft}p{2cm}@{}p{1.5cm}r}
\multicolumn{2}{c}{\small \centering Temperature.} &
Resistance.\\
200° & C & 227000\\
250° &&     13900\\
300° &&      1480\\
350° &&      1035\\
400° &&       735
\end{tabular}
\end{center}

\article{237*} Mr.\ C. F. Varley\footnote{
\textit{Proc.\ R. S.}, Jan.\ 12, 1871.} has recently investigated the conditions
of the current through rarefied gases, and finds that the electromotive
force \(E\) is equal to a constant \(E_0\) together with a part
depending on the current according to Ohm's Law, thus
\[
E = E_0 + RC\text{.}
\]

For instance, the electromotive force required to cause the
current to begin in a certain tube was that of 323 Daniell's cells,
but an electromotive force of 304 cells was just sufficient to
maintain the current. The intensity of the current, as measured
by the galvanometer, was proportional to the number of cells above
304. Thus for 305 cells the deflexion was 2, for 306 it was 4,
for 307 it was 6, and so on up to 380, or \(304 + 76\) for which the
deflexion was 150, or \(76 \times 1.97\).

From these experiments it appears that there is a kind of
polarization of the electrodes, the electromotive force of which
is equal to that of 304 Daniell's cells, and that up to this electromotive
force the battery is occupied in establishing this state of
polarization. When the maximum polarization is established, the
excess of electromotive force above that of 304 cells is devoted to
maintaining the current according to Ohm's Law.

The Law of the current in a rarefied gas is therefore very similar
to the law of the current through an electrolyte in which we have
to take account of the polarization of the electrodes.

In connexion with this subject we should study Thomson's results\footnote
{[\textit{Proc.\ R. S.}, 1860, or Reprint, chap.\ xix.]},
in which the electromotive force required to produce a spark
in air was found to be proportional not to the distance, but to the
distance together with a constant quantity. The electromotive
force corresponding to this constant quantity may be regarded as
the intensity of polarization of the electrodes.

\article{238*} MM. Wiedemann and Rühlmann have recently\footnote{
\textit{Berichte der Königl. Sächs. Gesellschaft}, Oct.\ 20, 1871.} investigated
the passage of electricity through gases. The electric current
was produced by Holtz's machine, and the discharge took place
%%-----File: 224.png-----%%
between spherical electrodes within a metallic vessel containing
rarefied gas. The discharge was in general discontinuous, and the
interval of time between successive discharges was measured by
means of a mirror revolving along with the axis of Holtz's machine.
The images of the series of discharges were observed by means of
a heliometer with a divided object-glass, which was adjusted till
one image of each discharge coincided with the other image of
the next discharge. By this method very consistent results were
obtained. It was found that the quantity of electricity in each
discharge is independent of the strength of the current and of
the material of the electrodes, and that it depends on the nature
and density of the gas, and on the distance and form of the
electrodes.

These researches confirm the statement of Faraday\footnote{
\textit{Exp.\ Res.}, 1501.} that the
electric tension (see \hyperref[art:46]{Art.\ 46}) required to cause a disruptive discharge
to begin at the electrified surface of a conductor is a little less
when the electrification is negative than when it is positive, but
that when a discharge does take place, much more electricity passes
at each discharge when it begins at a positive surface. They also
tend to support the hypothesis, that the stratum of gas condensed
on the surface of the electrode plays an important part in the
phenomenon, and they indicate that this condensation is greatest
at the positive electrode.

\phantomsection\label{art:239}
\Subsection{Note on Wheatstone's Bridge.}

[The following method of determining the current in the Galvanometer
of Wheatstone's Bridge was given by Professor Maxwell
in his last course of lectures, and is a good illustration of the method
of treating a system of linear conductors. It has been communicated
to the present editor by Professor J. A. Fleming of University
College, Nottingham. The method simply assumes Ohm's Law for
each conductor, and that the whole electromotive force around a
linear circuit is the sum of the electromotive forces in the several
conductors forming the circuits, and therefore equal to the sum of
the products of the resistance of each conductor and the current
flowing in it, the currents being taken in cyclic order.
\Runhead{WHEATSTONE'S BRIDGE.}

\widefig{0.6}{225.png}{Fig. 53.}
Let \(P\), \(Q\), \(S\), \(R\), \(G\) and \(B\) (Fig. 53) denote the resistances in the
several conductors forming the bridge, and let them be arranged as
indicated in the figure. Now the six conductors may be considered
%%-----File: 225.png-----%%
as forming three independent circuits viz.:---\(PGQ\), \(RSG\), and \(QSB\).
Let \(x + y\), \(y\) and \(z\) denote the currents in these circuits respectively,
each current being considered as flowing in the directions indicated
by the arrows. Then the actual current in \(Q\) is \(z - x - y\), that in
\(S\) is \(z - y\) and that in \(G\), is \(x\), and the electromotive force between
the ends of \(Q\) is \(Q(z - y - x)\) and so on for the other conductors.
Of the three circuits specified above the E. M. F. in the first two is
zero while that in the third is \(E\), the electromotive force of the
battery. Hence, applying Ohm's Law to each circuit in order we
have
\begin{gather*}
\left.
\begin{aligned}
 ( P + G + Q )\overline{ x + y} - Gy - Qz &= 0\\
 ( R + S + G )y - Sz - G \overline{x + y} &= 0\\
 ( Q + S + B )z - Sy - Q \overline{x + y} &= E
\end{aligned}
\,\right\}\tag{I}\\
\shortintertext{or}
\left.
\begin{aligned}
( P + G + Q )x + ( P + Q )y - Qz &= 0 \\
-Gx + ( R + S )y - Sz &= 0\\
-Qx - ( S + Q )y + ( Q + S + B )z &= E
\end{aligned}
\,\right\}\tag{II}
\end{gather*}

Solving for \(x\) we obtain
\[
\begin{aligned}
  x & = \frac {E
    \begin{vmatrix}
      \overline{P+Q},&-Q\\
      \overline{R+S},&-S
    \end{vmatrix}
    }
    {\Delta}\\
    & = \frac{E(QR-PS)}{\Delta}\text{,}
\end{aligned}
\]
where \(\Delta\) is the determinant of the system of equations (II).

The condition for no current in the galvanometer is \(x = 0\), or
\[
QR - PS = 0 \text{, or } \frac{P}{Q} =\frac{R}{S}\text{.}
\]
%%-----File: 226.png-----%%

To obtain the current equations, (I), the rule is---

`Multiply each cycle sign (i.e.\ current) by the sum of all the
resistances which bound that cycle, and subtract from it the sign
of each neighbouring cycle multiplied by the resistance separating
the cycles, and equate the result to the E. M. F. in the cycle.'

It will be seen that the method is a simple application of
Kirchhoff's second law, but the above rule is very convenient
in its application.]

\begin{center}
\vfill
\footnotesize
THE END.
\vfill
\scriptsize
\end{center}
%%-----File: 227.png-----%%
\plategeometry
\newchapter
\begin{center}
\hspace{0pt}
\vfill
\LARGE
\textls[200]{PLATES}.
\vfill
\end{center}
%%-----File: 228.png-----%%
%%-----File: 229.png-----%%

\phantomsection\label{plate:1}
\plate{.99}{229.png}
{\textsc{Plate} I.\\ \small{\hyperref[art:93]{Art.\ 93}.}}
{\textit{Lines of Force and Equipotential Surfaces.}}
{\small \(A = 20\). \qquad \(B = 5\). \qquad \textit{P. Point of Equilibrium.} \qquad \(AP = \tstrut\frac{2}{3}AB\)}.
%%-----File: 230.png-----%%
%%-----File: 231.png-----%%
\phantomsection\label{plate:2}
\plate{.86}{231.png}
{\textsc{Plate} II.\\ \small{\hyperref[art:94]{Art.\ 94}.}}
{\textit{Lines of Force and Equipotential Surfaces.}}
{\small \(A =20\). \qquad \(B = -5\). \qquad \textit{P. Point of Equilibrium.} \qquad \(AP = 2AB\).

\textit{Q. Spherical surface of Zero potential.}

\textit{M. Point of Maximum Force along the axis.}

\textit{The dotted line is the Line of Force \(\Psi = 0.1\) thus \(-\cdot-\cdot-\cdot-\cdot-\)}}

%%-----File: 232.png-----%%
%%-----File: 233.png-----%%
\phantomsection\label{plate:3}
\plate{.99}{233.png}
{\textsc{Plate} III.\\ \small{\hyperref[art:95]{Art.\ 95}.}}
{\textit{Lines of Force and Equipotential Surfaces.}}
\begin{center}
{\small \(A = 10\).}
\end{center}
%%-----File: 234.png-----%%
%%-----File: 235.png-----%%

\phantomsection\label{plate:4}
\plate{.99}{235.png}
{\textsc{Plate} IV.\\ \small{\hyperref[art:96]{Art.\ 96}.}}
{\textit{Lines of Force and Equipotential Surfaces.}}
\begin{center}
{\small \(A = 15\). \qquad \(B = -12\). \qquad \( C = 20\).}
\end{center}
%%-----File: 236.png-----%%
%%-----File: 237.png-----%%

\plate{.99}{237.png}
{\textsc{Plate} V.\\ \small{\hyperref[art:193*]{Art.\ 193}.}}
{\textit{Lines of Force near the edge of a Plate.}}
%%-----File: 238.png-----%%
%%-----File: 239.png-----%%

\plate{.7}{239.png}
{\textsc{Plate} VI.}
{\textit{Lines of Force near a Grating}}
%%-----File: 240.png-----%%
%%-----File: 241.png-----%%
\restoregeometry

\pagenumbering{arabic} % resets page counter
% add prefix to page number to avoid duplicate page numbers
\renewcommand*{\thepage}{A\arabic{page}}
\newchapter
\normalsize

\begin{center}
\LARGE
\textls{CLARENDON PRESS, OXFORD.}\\[2mm]
{\sansfont \large \textls{SELECT LIST OF STANDARD WORKS.}\par}
\end{center}
\begin{center}
\footnotesize
\begin{tabular}{l r@{ } r}
STANDARD LATIN WORKS & Page & \pageref{adv:1}\\
STANDARD GREEK WORKS & \dittopage &\pageref{adv:2}\\
OXFORD CLASSICAL TEXTS & \dittopage &\pageref{adv:3}\\
MISCELLANEOUS STANDARD WORKS & \dittopage &\pageref{adv:4}
\end{tabular}
\end{center}

\phantomsection\label{adv:1}
\AdvSection{1. STANDARD LATIN WORKS.}
\Runhead{STANDARD LATIN WORKS.}
\begin{advlist}
\item[Aetna.] A critical recension
of the Text, based on a new
examination of MSS., with Prolegomena,
Translation, Textual and
Exegetical Commentary, Excursus,
and complete Index of the words.
By Robinson Ellis, M.A., LL.D.
Crown 8vo. 7\textit{s.}\ 6\textit{d.}\ \textit{net.}

\item[Avianus.] \textit{The Fables.} Edited,
with Prolegomena, Critical Apparatus,
Commentary, \&c., by Robinson
Ellis, M.A., LL.D. 8vo. 8\textit{s.}\ 6\textit{d.}

\item[Catulli Veronensis] \textit{Liber.}
Iterum recognovit, Apparatum Criticum
Prolegomena Appendices addidit,
R. Ellis, A.M. 8vo. 16\textit{s.}

\item[Catullus,] \textit{a Commentary on.}
By Robinson Ellis, M.A. \textit{Second
Edition.} 8vo. 18\textit{s.}

\item[Cicero.] \textit{De Oratore Libri
Tres.} With Introduction and Notes.
By A. S. Wilkins, Litt.D. 8vo. 18\textit{s.}\\
\hspace*{5cm} \textit{Also, separately}---
\begin{vollist}
\item Book I.   7\textit{s.}\ 6\textit{d.}
\item Book II.  5\textit{s.}
\item Book III. 6\textit{s.}
\end{vollist}

\item[\longdash] \textit{Pro Milone.} Edited by
A. C. Clark, M.A. 8vo. 8\textit{s.}\ 6\textit{d.}

\item[\longdash] \textit{Select Letters.} With
English Introductions, Notes, and
Appendices. By Albert Watson,
M.A. \textit{Fourth Edition.} 8vo. 18\textit{s.}

\pagebreak
\item[Horace.] With a Commentary.
By E. C. Wickham, D. D. \textit{Two Vols.}
\nopagebreak
\begin{vollist}
\item Vol.\ I. The Odes, Carmen Seculare,
and Epodes. \textit{Third Edition.}
8vo. 12\textit{s.}
\item Vol.\ II. The Satires, Epistles, and
De Arte Poetica. 8vo. 12\textit{s.}
\end{vollist}

\item[Juvenal.] \textit{Thirteen Satires.}
Edited, with Introduction and
Notes, by C. H. Pearson, M.A., and
Herbert A. Strong, M.A., LL.D.
\textit{Second Edition.} Crown 8vo. 9\textit{s.}

\item[\longdash] \textit{Ad Satiram Sextam
in Codice Bodl. Canon. XLI Additi
versus XXXVI.} Exscripsit E. O.
Winstedt, Accedit Simulacrum
Photographicum. In Wrapper.
1\textit{s.}\ \textit{net.}

\item[Livy.] \textit{Book I.} With Introduction,
Historical Examination,
and Notes. By Sir J. R. Seeley,
M.A. \textit{Third Edition.} 8vo. 6\textit{s.}

\item[Manilius.] \textit{Noctes Manilianae;
sive Dissertationes in Astronomica Manilii.
Accedunt Coniecturae in Germanici
Aratea.} Scripsit R. Ellis.
Crown 8vo. 6\textit{s.}

\item[Merry.] \textit{Selected Fragments
of Roman Poetry.} Edited, with Introduction
and Notes, by W. W.
Merry, D.D. \textit{Second Edition.} Crown
8vo. 6\textit{s.}\ 6\textit{d.}

\item[Ovid.] \textit{P. Ovidii Nasonis Ibis.}
Ex Novis Codicibus edidit, Scholia
Vetera Commentarium cum Prolegomenis
Appendice Indice addidit.
R. Ellis, A.M. 8vo. 10\textit{s.}\ 6\textit{d.}

\item[\longdash] \textit{P. Ovidi Nasonis Tristium
Libri V.} Recensuit S. G. Owen,
A.M. 8vo. 16\textit{s.}
%%-----File: 242.png-----%%

\item[Persius.] \textit{The Satires.} With
a Translation and Commentary.
By John Conington, M.A. Edited
by Henry Nettleship, M.A. \textit{Third
Edition.} 8vo. 8\textit{s.}\ 6\textit{d.}

\item[Plautus.] \textit{Rudens.} Edited,
with Critical and Explanatory
Notes, by E. A. Sonnenschein,
M.A. 8vo. 8\textit{s.}\ 6\textit{d.}

\item[\longdash] \textit{The Codex Turnebi of
Plautus.} By W. M. Lindsay, M.A.
8vo, 21\textit{s.}\ \textit{net.}

\item[Quintilian.] \textit{Institutions
Oratoriae Liber Decimus.} A Revised
Text, with Introductory Essays,
Critical Notes, \&c. By W. Peterson,
M.A., LL.D. 8vo. 12\textit{s.}\ 6\textit{d.}

\item[Rushforth.] \textit{Latin Historical
Inscriptions, illustrating the History of
the Early Empire.} By G. McN.
Rushforth, M.A. 8vo. 10\textit{s.}\ \textit{net.}

\item[Tacitus.] \textit{The Annals.} Edited,
with Introduction and Notes, by
H. Furneaux, M.A. 2 vols. 8vo.
\begin{vollist}
\item Vol.\ I, Books I-VI. \textit{Second Edition.}
18\textit{s.}
\item Vol.\ II, Books XI-XVI. 20\textit{s.}
\end{vollist}

\item[\longdash]\textit{De Germania.} By the
same Editor. 8vo. 6\textit{s.}\ 6\textit{d.}

\item[\longdash]\textit{Vita Agricolae.} By the
same Editor. 8vo. 6\textit{s.}\ 6\textit{d.}

\item[\longdash]\textit{Dialogus de Oratoribus.}
A Revised Text, with Introductory
Essays, and Critical and Explanatory
Notes. By W. Peterson, M.A.,
LL.D. 8vo. 10\textit{s.}\ 6\textit{d.}

\item[Velleius Paterculus] \textit{ad M.
Vinicium Libri Duo.} Ex Amerbachii
praecipue Apographo edidit et
emendavit R. Ellis, Litterarum
Latinarum Professor publicus apud
Oxonienses. Crown 8vo, paper
boards. 6\textit{s.}

\item \singleline

\item[Anthologia Oxoniensis, Nova.]
\textit{Translations into Greek and Latin Verse.}
Edited by Robinson Ellis, M.A.,
and A. D. Godley, M.A. Crown
8vo, buckram extra, 6\textit{s.}\ \textit{net}; India
Paper, 7\textit{s.}\ 6\textit{d.}\ \textit{net}.

\item[King and Cookson.] \textit{The Principles
of Sound and Inflexion, as illustrated
in the Greek and Latin Languages.}
By J. E. King, M.A., and Christopher
Cookson, M.A. 8vo. 18\textit{s.}

\item[\longdash]\textit{An Introduction to the
Comparative Grammar of Greek and
Latin.} Crown 8vo. 5\textit{s.}\ 6\textit{d.}

\item[Lewis and Short.] \textit{A Latin
Dictionary}, founded on Andrews'
edition of Freund's Latin Dictionary,
revised, enlarged, and in great
part re-written by Charlton T.
Lewis, Ph.D., and Charles Short,
LL.D. 4to, 25\textit{s.}

\item[Lindsay.] \textit{The Latin Language.}
An Historical Account of
Latin Sounds, Stems and Flexions.
By W. M. Lindsay, M.A. Demy
8vo. 21\textit{s.}

\item[Nettleship.] \textit{Lectures and
Essays on Subjects connected with Latin
Scholarship and Literature.}

\item[\longdash]Second Series. Edited by
F. J. Haverfield, with Memoir by
Mrs.\ Nettleship. Crown 8vo. 7\textit{s.}\ 6\textit{d.}

\item[\longdash]\textit{Contributions to Latin
Lexicography.} 8vo. 21\textit{s.}

\item[Sellar.] \textit{Roman Poets of the
Augustan Age.} By W. Y. Sellar,
M.A.; viz.
\begin{vollist}
\item I. \textsc{Virgil.} \textit{New Edition.} Crown
8vo. 9\textit{s.}
\item II. \textsc{Horace} and the \textsc{Elegiac
Poets}. With a Memoir of the
Author by Andrew Lang, M.A.
\textit{Second Edition.} Crown 8vo, 7\textit{s.}\ 6\textit{d.}
\end{vollist}

\item[\longdash]\textit{Roman Poets of the Republic.}
\textit{Third Edition.} Crown 8vo. 10\textit{s.}

\item[Wordsworth.] \textit{Fragments and
Specimens of Early Latin.} With Introductions
and Notes. By J. Wordsworth, D.D.
8vo. 18\textit{s.}
\end{advlist}
%%-----File: 243.png-----%%
\phantomsection\label{adv:2}
\AdvSection{2. STANDARD GREEK WORKS.}
\Runhead{STANDARD GREEK WORKS.}

\begin{advlist}
\item[Chandler.] \textit{A Practical Introduction
to Greek Accentuation}, by H. W.
Chandler, M. A. \textit{Second Edition}. 10\textit{s.}\ 6\textit{d.}

\item[Farnell.] \textit{The Cults of the Greek
States.} \textit{With Plates.} By L. R. Farnell,
M.A.
\begin{vollist}
\item Vols.\ I and II. 8vo. 32\textit{s.}\ \textit{net}.
\item Vols.\ III and IV \textit{in the Press}.
\end{vollist}

\item[Grenfell.] \textit{An Alexandrian
Erotic Fragment and other Greek Papyri,
chiefly Ptolemaic.} Edited by B. P.
Grenfell, M.A. Sm. 4to. 8\textit{s.}\ 6\textit{d.}\ \textit{net}.

\item[Grenfell and Hunt.] \textit{New
Classical Fragments and other Greek
and Latin Papyri.} Edited by B. P.
Grenfell, M.A., and A. S. Hunt,
M.A. With Plates, 12\textit{s.}\ 6\textit{d.}\ \textit{net}.

\item[\longdash] \textit{Menander's} {\greekfont Γεωργός}.
A Revised Text of the Geneva
Fragment. With a Translation
and Notes by the same Editors.
8vo, stiff covers, 1\textit{s.}\ 6\textit{d.}

\item[Grenfell and Mahaffy.] Revenue
Laws of Ptolemy Philadelphus.
2 vols. Text and Plates. 31\textit{s.}\ 6\textit{d.}\ \textit{net}.

\item[Haigh.] \textit{The Attic Theatre.}
A Description of the Stage and
Theatre of the Athenians, and of the
Dramatic Performances at Athens.
By A. E. Haigh, M.A. \textit{Second Edition,
Revised and Enlarged.} 8vo. 12\textit{s.}\ 6\textit{d.}

\item[\longdash] \textit{The Tragic Drama of
the Greeks.} With Illustrations.
8vo. 12\textit{s.}\ 6\textit{d.}

\item[Head.] \textit{Historia Numorum}:
A Manual of Greek Numismatics.
By Barclay V. Head. Royal 8vo,
half-bound, 42\textit{s.}

\item[Hicks.] \textit{A Manual of Greek
Historical Inscriptions.} New and
Revised Edition by E. L. Hicks,
M.A., and G. F. Hill, M.A. 8vo.
12\textit{s.}\ 6\textit{d.}

\item[Hill.] \textit{Sources for Greek History
between the Persian and Peloponnesian
Wars.} Collected and arranged
by G. F. Hill, M.A. 8vo. 10\textit{s.}\ 6\textit{d.}

\item[Kenyon.] \textit{The Palaeography
of Greek Papyri.} By Frederic G.
Kenyon, M.A. 8vo, with Twenty
Facsimiles, and a Table of Alphabets.
10\textit{s.}\ 6\textit{d.}

\item[Liddell and Scott.] \textit{A Greek-English
lexicon}, by H. G. Liddell,
D.D., and Robert Scott, D.D. \textit{Eighth
Edition, Revised.} 4to. 36\textit{s.}

\item[Monro.] \textit{Modes of Ancient
Greek Music.} By D. B. Monro, M.A.
8vo. 8\textit{s.}\ 6\textit{d.}\ \textit{net}.

\item[Paton and Hicks.] \textit{The Inscriptions
of Cos.} By W. R. Paton
and E. L. Hicks. Royal 8vo, linen,
with Map, 28\textit{s.}

\item[Smyth.] \textit{The Sounds and
Inflections of the Greek Dialects} (Ionic).
By H. Weir Smyth, Ph.D. 8vo. 24\textit{s.}

\item[Thompson.] \textit{A Glossary of
Greek Birds.} By D'Arcy W. Thompson.
8vo, buckram, 10\textit{s.}\ \textit{net}.

\item \singleline

\item[Aeschylus.] \textit{In Single Plays.}
With Introduction and Notes, by
Arthur Sidgwick, M.A. \textit{New
Edition.} Extra fcap.\ 8vo. 3\textit{s.}\ each.
\begin{vollist}
\item I. Agamemnon. II. Choephoroi.
III. Eumenides. IV. Persae.
\item V. Septem contra Thebas.
\end{vollist}

\item[\longdash] \textit{Prometheus Bound}. By
A. O. Prickard, M.A. \textit{Third Edition.} 2\textit{s.}

\item[Aristophanes.] \textit{In Single Plays.}
Edited, with English Notes, Introductions,
\&c., by W. W. Merry, D.D.
Extra fcap.\ 8vo.
\begin{vollist}
\item The Acharnians. \textit{Fourth Edition}, 3\textit{s.}
\item The Birds. \textit{Third Edition}, 3\textit{s.}\ 6\textit{d.}
\item The Clouds. \textit{Third Edition}, 3\textit{s.}
\item The Frogs. \textit{Third Edition}, 3\textit{s.}
\item The Knights. \textit{Second Edition}, 3\textit{s.}
\item The Peace. 3\textit{s.}\ 6\textit{d.}
\item The Wasps. \textit{Second Edition}, 3\textit{s.}\ 6\textit{d.}
\end{vollist}

\item[Aristotle.] Ex recensione
Im. Bekkeri. Accedunt Indices
Sylburgiani. Tomi XI. 8vo. The
volumes (except I and IX which are
out of print) may be had separately,
price 5\textit{s.}\ 6\textit{d.}\ each.

\item[Aristotle.] \textit{Ethica Nicomachea},
recognovit brevique Adnotatione
critica instruxit I. Bywater. 8vo. 6\textit{s.}
\textit{Also in crown 8vo, paper cover}, 3\textit{s.}\ 6\textit{d.}

\item[\longdash] Contributions to the
Textual Criticism of the Nicomachean
Ethics. By I. Bywater. 2\textit{s.}\ 6\textit{d.}

\item[\longdash] Notes on the Nicomachean
Ethics. By J. A. Stewart, M.A.
2 vols. 8vo. 32\textit{s.}

\item[\longdash] \textit{Selecta ex Organo Aristoteleo
Capitula.} In usum Scholarum
Academicarum. Crown 8vo,
stiff covers. 3\textit{s.}\ 6\textit{d.}

\item[\longdash] \textit{De Arte Poetica Liber.}
Recognovit Brevique Adnotatione
Critica Instruxit I. Bywater. Post
8vo, stiff covers, 1\textit{s.}\ 6\textit{d.}

\item[\longdash] \textit{The Politics}, with Introductions,
Notes, \&c., by W. L. Newman,
M.A. Medium 8vo. Vols.\ I
and II. 28\textit{s.}\ \textit{net.} Vols.\ III and
IV. 28\textit{s.}\ \textit{net.}

\item[\longdash] \textit{The Politics}, translated
into English, with Introduction,
Marginal Analysis, Notes,
and Indices, by B. Jowett, M.A.
Medium 8vo. 2 vols. 21\textit{s.}

\item[Aristoxenus.] {\greekfont Ἀριστοξένου
Ἁρμονικὰ Στοιχεῖα.} The Harmonics
of Aristoxenus. Edited, with
Translation, Notes, \&c., by H. S.
Macran, M.A. Crown 8vo, 10\textit{s.}\ 6\textit{d.}

\item[Demosthenes and Aeschines.]
The Orations of Demosthenes and
Aeschines on the Crown. By
G. A. Simcox, M.A., and W. H.
Simcox, M.A. 8vo. 12\textit{s.}

\item[Demosthenes.] \textit{Orations
against Philip.} With Introduction
and Notes, by Evelyn Abbott, M.A.,
and P. E. Matheson, M.A.
\begin{vollist}
\item Vol.\ I. Philippic I. Olynthiacs
I-III. Extra fcap.\ 8vo. 3\textit{s.}
\item Vol.\ II. De Pace, Philippic II.
De Chersoneso, Philippic III.
Extra fcap.\ 8vo. 4\textit{s.}\ 6\textit{d.}
\end{vollist}

\item[\longdash] \textit{On the Crown.} 3\textit{s.}\ 6\textit{d.}

\item[\longdash] \textit{Against Meidias.} By
J. R. King. 3\textit{s.}\ 6\textit{d.}

\item[Euripides.] \textit{Tragoediae et
Fragmenta}, ex recensione Guil. Dindorfii.
Tomi II. 8vo. 10\textit{s.}

\item[Heracliti.] \textit{Ephesii Reliquiae.}
Recensuit I. Bywater, M.A. Appendicis
loco additae sunt Diogenis
Laertii Vita Heracliti, Particulae
Hippocratei De Diaeta Lib.\ I., Epistolae
Heracliteae. 8vo. 6\textit{s.}

\item[Herodas.] {\greekfont ΗΡΩΙΔΟΥ ΜΙΜΙΑΜΒΟΙ.}
\textit{The Mimes of Herodas.}
Edited, with Introduction, Critical
Notes, Commentary, and Illustrations,
by J. Arbuthnot Nairn, M.A.
8vo, 12\textit{s.}\ 6\textit{d.}\ net.

\item[Herodotus.] \textit{Books V and VI},
Terpsichore and Erato. Edited,
with Notes and Appendices, by
Evelyn Abbott, M.A., LL.D. 8vo,
with two Maps, 6\textit{s.}

\item[Homer.] \textit{A Complete Concordance
to the Odyssey and Hymns of
Homer}; to which is added a Concordance
to the Parallel Passages in
the Iliad, Odyssey, and Hymns.
By Henry Dunbar, M.D. 4to. 21\textit{s.}

\item[\longdash] \textit{A Grammar of the Homeric
Dialect.} By D. B. Monro, M.A.
8vo. Second Edition. 14\textit{s.}

\item[\longdash] \textit{Ilias}, ex rec. Guil. Dindorfii.
8vo. 5\textit{s.}\ 6\textit{d.}

\item[\longdash] \textit{Scholia Graeca in
Iliadem.} Edited by W. Dindorf,
after a new collation of the Venetian
mss.\ by D. B. Monro, M.A. 4 vols.
8vo. 50\textit{s.}

\item[\longdash] \textit{Scholia Graeca in
Iliadem Townleyana}. Recensuit
Ernestus Maass. 2 vols. 8vo. 36\textit{s.}

\item[\longdash] \textit{Odyssea}, ex rec. G.
Dindorfii. 8vo. 5\textit{s.}\ 6\textit{d.}

\item[\longdash] \textit{Scholia Graeca in
Odysseam.} Edidit Guil. Dindorfius.
Tomi II. 8vo. 15\textit{s.}\ 6\textit{d.}

\item[\longdash] \textit{Odyssey.} Books I-XII.
Edited with English Notes, Appendices,
\&c. By W. W. Merry,
D.D., and James Riddell, M.A.
\textit{Second Edition.} 8vo. 16\textit{s.}
%%-----File: 245.png-----%%

\item[Homer.] \textit{Odyssey.} Books XIII-XXIV.
Edited with English Notes
and Appendices, by D. B. Monro,
M.A. 8vo. 16\textit{s.}

\item[\longdash] \textit{Hymni Homerici.} Codicibus
denuo collatis recensuit
Alfredus Goodwin. Small folio.
With four Plates. 21\textit{s.}\ \textit{net}.

\item[Homeri Opera et Reliquiae.]
Monro. Crown 8vo. India Paper.
\textit{Cloth}, 10\textit{s.}\ 6\textit{d.}\ \textit{net}.

\textit{Also in various leather bindings.}

\item[Plato.] \textit{Apology}, with a revised
Text and English Notes, and
a Digest of Platonic Idioms, by
James Riddell, M.A. 8vo. 8\textit{s.}\ 6\textit{d.}

\item[\longdash] \textit{Philebus}, with a revised
Text and English Notes, by Edward
Poste, M.A. 8vo. 7\textit{s.}\ 6\textit{d.}

\item[\longdash] \textit{Republic.} The Greek
Text. Edited, with Notes and
Essays, by B. Jowett, M.A., and
Lewis Campbell, M.A. In three
vols. Medium 8vo. 42\textit{s.}

\item[\longdash] \textit{Sophistes} and \textit{Politicus},
with a revised Text and English
Notes, by L. Campbell, M.A. 8vo.
10\textit{s.}\ 6\textit{d.}

\item[\longdash] \textit{Theaetetus}, with a revised
Text and English Notes, by L. Campbell,
M.A. \textit{Second Edition.} 8vo. 10\textit{s.}\ 6\textit{d.}

\item[\longdash] \textit{The Dialogues}, translated
into English, with Analyses
and Introductions, by B. Jowett,
M.A. \textit{Third Edition.} 5 vols. Medium
8vo. Cloth, 84\textit{s.}; half-morocco, 100\textit{s.}

\item[\longdash] \textit{The Republic}, translated
into English, with Analysis and
Introduction, by B. Jowett, M.A.
\textit{Third Edition.} Medium 8vo. 12\textit{s.}\ 6\textit{d.};
half-roan, 14\textit{s.}

\item[\longdash] \textit{Res Publica}: recognovit
brevique adnotatione critica
instruxit Ioannes Burnet. On 4to
paper for marginal notes. 10\textit{s.}\ 6\textit{d.}

\item[\longdash] \textit{With Introduction and
Notes.} By St. George Stock, M.A.
Extra fcap.\ 8vo.
\begin{vollist}
\item I. The Apology, 2\textit{s.}\ 6\textit{d.}
II. Crito, 2\textit{s.}
III. Meno, 2\textit{s.}\ 6\textit{d.}
\end{vollist}

\item[Plato.] \textit{Selections. With Introductions
and Notes.} By John Purves,
M.A., and Preface by B. Jowett,
M.A. \textit{Second Edition.} Extra fcap.\
8vo. 5\textit{s.}

\item[\longdash] \textit{A Selection of Passages
from Plato for English Readers}; from
the Translation by B. Jowett, M.A.
Edited, with Introductions, by
M. J. Knight. 2 vols. Crown 8vo,
gilt top. 12\textit{s.}

\item[Polybius.] \textit{Selections.} Edited
by J. L. Strachan-Davidson, M.A.
With Maps. Medium 8vo. 21\textit{s.}

\item[Sophocles.] \textit{The Plays and
Fragments.} With English Notes and
Introductions, by Lewis Campbell,
M.A. 2 vols. 8vo, 16\textit{s.}\ \textit{each}.

\item[\longdash] \textit{Tragoediae et Fragmenta},
ex recensione et cum commentariis
Guil. Dindorfii. \textit{Third Edition.}
2 vols. Fcap.\ 8vo. 21\textit{s.}\ Each Play
separately, limp, 2\textit{s.}\ 6\textit{d.}

\item[Sophocles.] \textit{Tragoediae et
Fragmenta} cum Annotationibus Guil.
Dindorfii. Tomi II. 8vo. 10\textit{s.}
\begin{vollist}
\item The Text, Vol.\ I. 5\textit{s.}\ 6\textit{d.}
\item The Notes, Vol.\ II. 4\textit{s.}\ 6\textit{d.}
\end{vollist}

\item[Strabo.] \textit{Selections}, with an
Introduction on Strabo's Life and
Works. By H. F. Tozer, M.A.,
F.R.G.S.    8vo. With Maps and
Plans. 12\textit{s.}

\item[Thucydides.] Translated into
English, to which is prefixed an
Essay on Inscriptions and a Note on
the Geography of Thucydides. By
B. Jowett, M.A. \textit{Second Edition, Revised.}
2 vols., 8vo, cloth, 15\textit{s.}
\begin{vollist}
\item Vol.\ I. Essay on Inscriptions
and Books I-III.
\item Vol.\ II. Books IV-VIII and
Historical Index.
\end{vollist}

\item[Xenophon.] A Commentary,
with Introduction and Appendices,
on the Hellenica of Xenophon. By
G. E. Underhill, M.A. Crown 8vo,
7\textit{s.}\ 6\textit{d.}
%%-----File: 246.png-----%%

\end{advlist}
%%-----File: 243.png-----%%
\phantomsection\label{adv:3}
\AdvSection{3. OXFORD CLASSICAL TEXTS.}
\Runhead{OXFORD CLASSICAL TEXTS.}
{\centering
\small
(\textit{a}) \textsc{Paper Covers}; (\textit{b}) \textsc{Limp Cloth}; (\textit{c}) \textsc{India Paper.}

\textsc{Crown 8vo.}
\par}

\begin{center}
GREEK.
\end{center}

\begin{advlist}
\item[\mdseries\textsc{Aeschyli Tragoediae.}] Cum
Fragmentis. \textsc{A. Sidgwick.} (\textit{a})
3\textit{s.}; (\textit{b}) 3\textit{s.}\ 6\textit{d.}; (\textit{c}) 4\textit{s.}\ 6\textit{d.}

\item[\mdseries\textsc{Apollonii Rhodii}] Argonautica.
\textsc{R. C. Seaton.} (\textit{a}) 2\textit{s.}\ 6\textit{d.}; (b) 3\textit{s.}

\item[\mdseries\textsc{Aristophanis Comoediae}] Cum
Fragmentis. \textsc{F. W. Hall} and
\textsc{W. M. Geldart}. Tom. I. (\textit{a})
3\textit{s.}; (\textit{b}) 3\textit{s.}\ 6\textit{d.} Tom. II. (\textit{a}) 3\textit{s.};
(\textit{b}) 3\textit{s.}\ 6\textit{d.} Complete, (\textit{c}) 8\textit{s.}\ 6\textit{d.}

\item[\mdseries\textsc{Demosthenis Orationes.}] \textsc{S. H.
Butcher.} Tom. I. (\textit{a}) 4\textit{s.}; (\textit{b})
4\textit{s.}\ 6\textit{d.}

\item[\mdseries\textsc{Euripidis Tragoediae.}] \textsc{G. G. A.
Murray.} Tom. I. (\textit{a}) 3\textit{s.}; (\textit{b})
3\textit{s.}\ 6\textit{d.} Tom. II. (\textit{a}) 3\textit{s.}; (\textit{b}) 3\textit{s.}\ 6\textit{d.}
Complete, (\textit{c}) 9\textit{s.}

\item[\mdseries\textsc{Homeri Opera.}] Ilias. \textsc{D. B.
Monro} and \textsc{T. W. Allen}. Tom.
I. (\textit{a}) 2\textit{s.}\ 6\textit{d.}; (\textit{b}) 3\textit{s.}\ Tom. II.
(\textit{a}) 2\textit{s.}\ 6\textit{d.}; (\textit{b}) 3\textit{s.}\ Complete, (\textit{c}) 7\textit{s.}

\item[\mdseries\textsc{Platonis Opera.}] \textsc{J. Burnet.}
Tom. I (Tetralogiae I, II). (\textit{a})
5\textit{s.}; (\textit{b}) 6\textit{s.}; (\textit{c}) 7\textit{s.}\ Tom. II
(Tetralogiae III, IV). (\textit{a}) 5\textit{s.};
(\textit{b}) 6\textit{s.}; (\textit{c}) 7\textit{s.}\ Tom. III (Tetralogiae
V-VII). (\textit{a}) 5\textit{s.}; (\textit{b}) 6\textit{s.}; (\textit{c}) 7\textit{s.}
Tom. IV (Tetralogia VIII). (\textit{a}) 6\textit{s.};
(\textit{b}) 7\textit{s.}; (\textit{c}) 8\textit{s.}\ 6\textit{d.} Res Publica.
(\textit{a}) 5\textit{s.}; (\textit{b}) 6\textit{s.}; (\textit{c}) 7\textit{s.} Also on 4to
paper for marginal notes, 10\textit{s.}\ 6\textit{d.}
Clitopho, Timaeus, Critias. (\textit{a}) 2\textit{s.}

\item[\mdseries\textsc{Thucydidis Historiae.}] \textsc{H.
Stuart Jones.} Tom. I (Libri
I-IV). (\textit{a}) 3\textit{s.}; (\textit{b}) 3\textit{s.}\ 6\textit{d.} Tom.
II (Libri V-VIII). (\textit{a}) 3\textit{s.}; (\textit{b})
3\textit{s.}\ 6\textit{d.} Complete, (\textit{c}) 8\textit{s.}\ 6\textit{d.}

\item[\mdseries\textsc{Xenophontis Opera.}] \textsc{E. C.
Marchant.} Tom. I (Historia
Graeca). (\textit{a}) 2\textit{s.}\ 6\textit{d.}; (\textit{b}) 3\textit{s.} Tom.
II (Libri Socratici). (\textit{a}) 3\textit{s.}; (\textit{b})
3\textit{s.}\ 6\textit{d.} Tom. III. (Expeditio
Cyri). (\textit{a}) 2\textit{s.}\ 6\textit{d.}; (\textit{b}) 3\textit{s.}
\end{advlist}
\begin{center}
LATIN.
\end{center}
\begin{advlist}
\item[\mdseries\textsc{Caesaris Commentarii.}] \textsc{R. L.
A. Du Pontet.} De Bello Gallico.
(\textit{a}) 2\textit{s.}; (\textit{b}) 2\textit{s.}\ 6\textit{d.} De Bello Civili,
(\textit{a}) 2\textit{s.}\ 6\textit{d.}; (\textit{b}) 3\textit{s.} Complete, (\textit{c}) 7\textit{s.}

\item[\mdseries\textsc{Catulli Carmina.}] \textsc{R. Ellis.}
(\textit{a}) 2\textit{s.}; (\textit{b}) 2\textit{s.}\ 6\textit{d.}

\item[\mdseries\textsc{Ciceronis Orationes}] Pro Milone,
Caesarianae, Philippicae I-XIV.
\textsc{A. C. Clark.} (\textit{a}) 2\textit{s.}\ 6\textit{d.}; (\textit{b}) 3\textit{s.}

\item[\mdseries\textsc{Ciceronis Epistulae.}] \textsc{L. C.
Purser.} (Complete) (\textit{c}) 21\textit{s.} Ad
Familiares. (\textit{a}) 5\textit{s.}; (\textit{b}) 6\textit{s.} Ad
Atticum. In two Parts, each (\textit{a})
4\textit{s.}; (\textit{b}) 4\textit{s.}\ 6\textit{d.} Ad Q. Fratrem.
(\textit{a}) 2\textit{s.}\ 6\textit{d.}; (\textit{b}) 3\textit{s.}

\item[\mdseries\textsc{Ciceronis Opera Rhetorica.}]
\textsc{A. S. Wilkins.} Tom. I. (\textit{a})
2\textit{s.}\ 6\textit{d.}; (\textit{b}) 3\textit{s.} Tom. II. (\textit{a}) 3\textit{s.};
(\textit{b}) 3\textit{s.}\ 6\textit{d.} Complete (\textit{c}) 7\textit{s.}\ 6\textit{d.}

\item[\mdseries\textsc{Corneli Nepotis Vitae.}] \textsc{E. O.
Winstedt.} (\textit{a}) 1\textit{s.}\ 6\textit{d.}; (\textit{b}) 2\textit{s.}

\item[\mdseries\textsc{Horati Opera.}] \textsc{E. C. Wickham.}
(\textit{a}) 2\textit{s.}\ 6\textit{d.}; (\textit{b}) 3\textit{s.}; (\textit{c}) 4\textit{s.}\ 6\textit{d.}

\item[\mdseries\textsc{Lucreti Cari}] De Rerum Natura.
\textsc{C. Bailey.} (\textit{a}) 2\textit{s.}\ 6\textit{d.}; (\textit{b}) 3\textit{s.}; (\textit{c}) 4\textit{s.}

\item[\mdseries\textsc{Martialis Epigrammata.}] \textsc{W.
M. Lindsay.} (\textit{a}) 5\textit{s.}; (\textit{b}) 6\textit{s.};
(\textit{c}) 7\textit{s.}\ 6\textit{d.}\\
School Edition, expurgated, 3\textit{s.}\ 6\textit{d.}

\item[\mdseries\textsc{Persi et Juvenalis Satirae.}]
\textsc{S. G. Owen.} (\textit{a}) 2\textit{s.}\ 6\textit{d.}; (\textit{b}) 3\textit{s.}; (\textit{c}) 4\textit{s.}

\item[\mdseries\textsc{Plauti Comoediae.}] \textsc{W. M. Lindsay.}
Tom. I (\textit{a}) 5\textit{s.}; (\textit{b}) 6\textit{s.} Tom.
II (\textit{a}) 5\textit{s.}; (\textit{b}) 6\textit{s.} Complete, (\textit{c}) 16\textit{s.}

\item[\mdseries\textsc{Sexti Properti Carmina.}] \textsc{J. S.
Phillimore.} (\textit{a}) 2\textit{s.}\ 6\textit{d.}; (\textit{b}) 3\textit{s.}

\item[\mdseries\textsc{Stati Silvae.}] \textsc{J. S. Phillimore.}
(\textit{a}) 3\textit{s.}; (\textit{b}) 3\textit{s.}\ 6\textit{d.}

\item[\mdseries\textsc{Cornelii Taciti}] Opera Minora.
\textsc{H. Furneaux.} (\textit{a}) 1\textit{s.}\ 6\textit{d.}; (\textit{b}) 2\textit{s.}

\item[\mdseries\textsc{Terenti Comoediae.}] \textsc{R. Y.
Tyrrell.} (\textit{a}) 3\textit{s.}; (\textit{b}) 3\textit{s.}\ 6\textit{d.}; (\textit{c}) 5\textit{s.}

\item[\mdseries\textsc{Tibulli sive Albi Tibulli.}] \textsc{J. P.
Postgate.} (\textit{a}) 1\textit{s.}\ 6\textit{d.}; (\textit{b}) 2\textit{s.}\ 6\textit{d.}

\item[\mdseries\textsc{Vergili Opera.}] \textsc{F. A. Hirtzel.}
(\textit{a}) 3\textit{s.}; (\textit{b}) 3\textit{s.}\ 6\textit{d.}; (\textit{c}) 4\textit{s.}\ 6\textit{d.}
%%-----File: 247.png-----%%

\end{advlist}
\phantomsection\label{adv:4}
\AdvSection{4. MISCELLANEOUS STANDARD WORKS.}
\Runhead{MISCELLANEOUS STANDARD WORKS.}

\begin{advlist}
\item[Arbuthnot.] \textit{The Life and
Works of John Arbuthnot.} By G. A.
Aitken. 8vo, with Portrait, 16\textit{s.}

\item[Bryce.] \textit{Studies in History
and Jurisprudence.} By the Right
Hon.\ James Bryce, D.C.L. 2 vols.
8vo, 25\textit{s.}\ \textit{net}.

\item[Casaubon (Isaac),] 1559-1614.
By Mark Pattison, late Rector of
Lincoln College. Ed. II. 8vo. 16\textit{s.}

\item[Chambers.] \textit{The Mediaeval
Stage.} 2 vols., 8vo. 25\textit{s.}\ \textit{net}.

\item[Chaucer.] \textit{The Complete Works
of Geoffrey Chaucer.} Edited, from
numerous Manuscripts, by W. W.
Skeat, Litt.D. In Six Volumes,
Demy 8vo, with Portrait and Facsimiles.
96\textit{s.}, or 16\textit{s.}\ each volume.

\item[\longdash] \textit{Chaucerian and other
Pieces.} Edited, from numerous
Manuscripts, by W. W. Skeat,
Litt.D. 8vo. 18\textit{s.}

\item[Cromwell.] \textit{The Life and
Letters of Thomas Cromwell.} By R. B.
Merriman, A.M. Harv., B.Litt.
Oxon. With a Portrait and Facsimile.
2 vols. 8vo. 18\textit{s.}\ \textit{net}.

\item[De Necessariis Observantiis
Scaccarii Dialogus:] commonly
called `Dialogus de Scaccario,' by
Richard, Son of Nigel, Treasurer of
England and Bishop of London.
Edited by A. Hughes, C. G. Crump,
and C. Johnson.

Revised Text and full Critical
Notes, an Introduction, Notes, and
an Index. 8vo. 12\textit{s.}\ 6\textit{d.}\ \textit{net}.

\item[Finlay.] \textit{A History of Greece
from its Conquest by the Romans to the
present time}, \textsc{b.c.}\ 146 to \textsc{a.d.}\ 1864.
By George Finlay, LL.D. New and
revised Edition, by H. F. Tozer,
M.A. 7 vols. 8vo. 70\textit{s.}

\item[Fisher.] \textit{Studies in Napoleonic
Statesmanship. Germany.} By H.
A. L. Fisher, M.A. 8vo. 12\textit{s.}\ 6\textit{d.}\ \textit{net}.

\item[Gower.] \textit{The Complete Works
of John Gower.} Edited from the
MSS., with Introductions, Notes,
and Glossaries, by G. C. Macaulay,
M.A. With facsimiles. 4 vols. 8vo,
buckram. 16\textit{s.}\ each.
\begin{vollist}
\item Vol.\ I, The French Works. Vols.\
II and III, The English Works.
\item Vol.\ IV, The Latin Works.
\end{vollist}

\item[Greene.] \textit{The Plays and Poems
of Robert Greene.} Edited, with Introduction
and Notes, by J. Churton
Collins, Litt.D. 2 vols. 8vo. 18\textit{s.}\ \textit{net}.

\item[Hodgkin.] \textit{Italy and her Invaders.}
8 vols. With Plates and
Maps. By Thomas Hodgkin, D.C.L.
\textsc{a.d.}\ 376-744. 8vo. Vols.\ I and II,
\textit{Second Edition}, 42\textit{s.} Vols.\ III and
IV, \textit{Second Edition}, 36\textit{s.} Vols.\ V and
VI, 36\textit{s.} Vols.\ VII and VIII, 24\textit{s.}

\item[Hollis.] \textit{The Masai; their
Language and Folklore.} By A. C.
Hollis. 8vo, with 27 full-page
Illustrations. 14\textit{s.}\ \textit{net}.

\item[Ilbert.] \textit{The Government of
India}; being a Digest of the Statute
Law relating thereto. By Sir
Courtenay Ilbert, K.C.S.I. 8vo,
half-roan, 21\textit{s.}

\item[\longdash] \textit{Legislative Methods and
Forms.} 8vo, half-roan. 16\textit{s.}

\item[Justinian.] \textit{Imperatoris Iustiniani
Institutionum Libri Quattuor};
with Introductions, Commentary,
Excursus and Translation. By J. B.
Moyle, D.C.L. \textit{Fourth Edition.} 2 vols.
8vo. 22\textit{s.}

\item[Kyd.] \textit{The Works of Thomas
Kyd.} Edited from the original
Texts, with Introduction, Notes,
and Facsimiles, by Frederick S.
Boas, M.A. 8vo. 15\textit{s.}\ \textit{net}.

\item[Lyly.] \textit{The Works of John
Lyly}, now for the first time collected
and edited from the earliest Quartos,
with Life, Bibliography, Essays,
Notes, and Index, by R. Warwick
Bond, M.A. 3 vols. 8vo. 42\textit{s.}\ \textit{net}.
%%-----File: 248.png-----%%

\item[Morris.] \textit{The Welsh Wars of
Edward the First.} By John E. Morris,
M.A. With a Map and Pedigrees.
8vo. 9\textit{s.}\ 6\textit{d.}\ \textit{net}.

\item[Oman.] \textit{History of the Peninsular
War.} By Charles Oman, M.A.
8vo. With Maps, Plans, and Portraits.
Vol.\ I, 1807-1809 (from the
Treaty of Fontaineblau to the Battle
of Corunna). 14\textit{s.}\ \textit{net}. Vol.\ II,
Jan.-Sept., 1809 (from the Battle of
Corunna to the end of the Talavera
Campaign), 14\textit{s.}\ \textit{net}. (The work
will be completed in six volumes.)

\item[Oxford History of Music.]
8vo. 15\textit{s.}\ \textit{net} each; but upon issue
Vols.\ II and VI will be sold together
for 15\textit{s.}\ \textit{net}, and the temporary
price of the whole set of
six volumes will be \textit{£}3 15\textit{s.}\ \textit{net}.
\begin{vollist}
\item Vol.\ I. \textit{The Polyphonic Period.} Part I.
\textit{Method of Musical Art}, 330-1330.
By H. E. Wooldridge, M.A.
\item Vol.\ II. \textit{The Ecclesiastical Period.} By
H. E. Wooldridge, M.A.
\item Vol.\ III. \textit{The Music of the Seventeenth
Century.} By Sir C. Hubert H.
Parry, Bart.
\item Vol.\ IV. \textit{The Age of Bach and Handel.}
By J. A. Fuller Maitland.
\item Vol.\ V. \textit{The Viennese Period.} By
W. H. Hadow.
\item Vol.\ VI. \textit{The Romantic Period.} By
E. Dannreuther, M.A.
\end{vollist}
\item[Pattison.] \textit{Essays by the late
Mark Pattison}, sometime Rector of
Lincoln College. Collected and
Arranged by Henry Nettleship,
M.A. 2 vols. 8vo. 24\textit{s.}

\item[Payne.] \textit{History of the New
World called America.} By E. J.
Payne, M.A. 8vo. Vol.\ I, 18\textit{s.};
Vol.\ II, 14\textit{s.}

\item[Poole.] \textit{Historical Atlas of
Modern Europe}, from the decline of
the Roman Empire. Comprising
also Maps of parts of Asia and of
the New World connected with
European history. Edited by
R. L. Poole, M.A., Ph.D. Imperial
4to. Half-Persian, 115\textit{s.}\ 6\textit{d.}; each
map sold separately at 1\textit{s.}\ 6\textit{d.}

\item[Ramsay.] \textit{The Cities and
Bishoprics of Phrygia.} By W. M.
Ramsay, D.C.L., LL.D. Vol.\ I. Part
I. \textit{The Lycos Valley and South-Western
Phrygia.} Royal 8vo, linen, 18\textit{s.}\ \textit{net}.
Vol.\ I. Part II. \textit{West and West-Central
Phrygia.} Royal 8vo, linen, 21\textit{s.}\ \textit{net}.

\item[Ranke.] \textit{A History of England},
principally in the Seventeenth Century.
By L. von Ranke. Translated
under the superintendence of
G. W. Kitchen, D.D., and C. W.
Boase, M.A. 6 vols. 8vo. 63\textit{s.}

\textit{Revised Index separately}, paper
covers, 1\textit{s.}

\item[Rhŷs.] \textit{Celtic Folklore, Welsh
and Manx.} By John Rhŷs, M.A.,
D.Litt. 2 vols. 8vo. 21\textit{s.}

\item[\longdash] \textit{Studies in the Arthurian
Legend.} 8vo. 12\textit{s.}\ 6\textit{d.}

\item[Sanday.] \textit{Sacred Sites of the
Gospels.} By W. Sanday, D.D. With
63 full-page Illustrations, Maps and
Plans. 8vo. 13\textit{s.}\ 6\textit{d.}

\item[\longdash] \textit{The Criticism of the
Fourth Gospel.} 8vo. 7\textit{s.}\ 6\textit{d.}\ \textit{net}.

\item[Thomson.] \textit{A Handbook of
Anatomy for Art Students.} With many
Illustrations. By Prof. Arthur
Thomson, M.A. \textit{Second Edition.}
8vo, buckram. 16\textit{s.}\ \textit{net}.

\item[Walpole.] \textit{The Letters of
Horace Walpole.} Edited by Mrs.\
Paget Toynbee, with over one
hundred Letters hitherto unpublished
and fifty Photogravure Portraits
of Walpole and his circle.
16 vols. Vol.\ XVI comprises the
Indexes, which have been prepared,
with great care and labour, under
the superintendence of Andrew
Clark, M.A. It contains Indexes
of Plates, Persons, and Subjects.
In three styles, viz.\ limited edition
of 260 copies, in demy 8vo, on handmade
paper, cloth, \textit{£}16 \textit{net}; Crown
8vo, on Oxford India Paper, in 8
double volumes, \textit{£}6 16\textit{s.}\ \textit{net}; Crown
8vo, in 16 volumes, on ordinary
paper, \textit{£}4 16\textit{s.}\ \textit{net}.
\end{advlist}
\singleline
\vfill
\begin{center}
\large
\textls{OXFORD}\\
\normalsize
\textls{AT THE CLARENDON PRESS}\\
\small
LONDON: HENRY FROWDE\\
\footnotesize
OXFORD UNIVERSITY PRESS WAREHOUSE, AMEN CORNER, E.C.
\end{center}

%%%% transcriber's note %%%%
\normalsize
\newchapter
\begin{center}
Transcriber's notes
%\vspace*{0.5cm}
\end{center}

\begin{itemize}
\small
\item
Obvious typographical errors have been silently corrected. The following changes have been made:
\item
The reference to article 135 in the contents list has been changed to 135* to be consistent with the article itself.
\item
A \hyperlink{198:1}{paragraph} in article 198 which seemed to be corrupted has been changed. The paragraph originally read:

\vspace*{1ex}
If the other carrier has at the same time carried a charge \(-QV\)
from \(C\) to \(D\), it will change the potential of \(A\) and \(O\) from \(U\) to
\(U - \xp\dfrac{Q'}{A} V\), if \(Q'\) is the coefficient of induction between the carrier
and \(C\), and \(A\) the capacity of \(A\) and \(D\). If, therefore, \(U_n\) and \(V_n\)
be the potentials of the two inductors after \(n\) half revolutions, and
\[
  U_{n + 1 +}\, n + 1 \text{ half revolutions,}
\]
\[
  \begin{aligned}
    U_{n + 1} &= U_n - \frac {Q'}{A} V_n,\\
    V_{n + 1} &= V_n - \frac {Q}{B} U_n.
  \end{aligned}
\]
\item
A missing prime has been added to \hyperlink{204:1}{an equation} in a footnote in article 204. The original equation was:
{\footnotesize
\[
Q = V \left\{ \frac{R^2 + R'^2}{8D} - \frac{R'^2 - R^2}{8D} \frac{\alpha}{D + \alpha} + \frac{R+R'}{D}(D-D) \log_e \frac{4 \pi(R+R')}{D'-D} \right\}\text{,}
\]
}
\item
in \hyperlink{220:1}{a sentence} in article 220 `in' has been changed to `is'. Originally:

`The merit of the method consists in the fact that the thing
observed in the absence of any deflexion, \dots'

\item In the \hyperref[232:1]{table} in article 232 `H\textsuperscript{2}O' has been changed to `\ce{H2O}'.

\item In a \hyperlink{234:1}{sentence} in article 234 the word `change' has been changed to `charge'. Originally:

`\dots\ the total quantity of electricity being that required to
change the surfaces of the insulator \dots'

\end{itemize}

\PGLicense
\begin{PGtext}
\begin{center}
*** END OF THE PROJECT GUTENBERG EBOOK AN ELEMENTARY TREATISE ON ELECTRICITY ***
\end{center}
\InputIfFileExists{pgfooter.tex}{}{}
\end{PGtext}
\end{document}
