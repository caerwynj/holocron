% %%%%%%%%%%%%%%%%%%%%%%%%%%%%%%%%%%%%%%%%%%%%%%%%%%%%%%%%%%%%%%%%%%%%%%% %
%                                                                         %
% The Project Gutenberg EBook of Treatise on Thermodynamics, by Max Planck%
%                                                                         %
% This eBook is for the use of anyone anywhere in the United States and most
% other parts of the world at no cost and with almost no restrictions     %
% whatsoever.  You may copy it, give it away or re-use it under the terms of
% the Project Gutenberg License included with this eBook or online at     %
% www.gutenberg.org.  If you are not located in the United States, you'll have
% to check the laws of the country where you are located before using this ebook.
%                                                                         %
% Title: Treatise on Thermodynamics                                       %
%                                                                         %
% Author: Max Planck                                                      %
%                                                                         %
% Translator: Alexander Ogg                                               %
%                                                                         %
% Release Date: January 9, 2016 [EBook #50880]                            %
% Most recently updated: June 11, 2021                         %
%                                                                         %
% Language: English                                                       %
%                                                                         %
% Character set encoding: UTF-8                                           %
%                                                                         %
% *** START OF THIS PROJECT GUTENBERG EBOOK TREATISE ON THERMODYNAMICS ***%
%                                                                         %
% %%%%%%%%%%%%%%%%%%%%%%%%%%%%%%%%%%%%%%%%%%%%%%%%%%%%%%%%%%%%%%%%%%%%%%% %

\def\ebook{50880}
%%%%%%%%%%%%%%%%%%%%%%%%%%%%%%%%%%%%%%%%%%%%%%%%%%%%%%%%%%%%%%%%%%%%%%
%%                                                                  %%
%% Packages and substitutions:                                      %%
%%                                                                  %%
%% book:     Required.                                              %%
%% inputenc: Latin-1 text encoding. Required.                       %%
%%                                                                  %%
%% babel:    German hyphenation. Required.                          %%
%%                                                                  %%
%% ifthen:   Logical conditionals. Required.                        %%
%%                                                                  %%
%% amsmath:  AMS mathematics enhancements. Required.                %%
%% amssymb:  Additional mathematical symbols. Required.             %%
%%                                                                  %%
%% array:    Enhancements to array/tabular environments. Required.  %%
%%                                                                  %%
%% alltt:    Fixed-width font environment. Required.                %%
%%                                                                  %%
%% footmisc: Start footnote numbering on each page. Required.       %%
%%                                                                  %%
%% mhchem:   Chemical formulas. Required.                           %%
%%                                                                  %%
%% multicol: Multicolumn environment for index. Required.           %%
%% makeidx:  Indexing. Required.                                    %%
%%                                                                  %%
%% graphicx: Standard interface for graphics inclusion. Required.   %%
%%                                                                  %%
%% calc:     Length calculations. Required.                         %%
%%                                                                  %%
%% fancyhdr: Enhanced running headers and footers. Required.        %%
%%                                                                  %%
%% geometry: Enhanced page layout package. Required.                %%
%% hyperref: Hypertext embellishments for pdf output. Required.     %%
%%                                                                  %%
%%                                                                  %%
%% Producer's Comments:                                             %%
%%                                                                  %%
%%   OCR text for this ebook was obtained December 28, 2015, from   %%
%%   http://www.archive.org/details/treatiseonthermo00planrich.     %%
%%                                                                  %%
%%   Minor changes to the original are noted in this file in three  %%
%%   ways:                                                          %%
%%     1. \Typo{}{} for typographical corrections, showing original %%
%%        and replacement text side-by-side.                        %%
%%     2. \Chg{}{} and \Add{}, for inconsistent/missing punctuation,%%
%%        italicization, and capitalization.                        %%
%%     3. [** TN: Note]s for lengthier or stylistic comments.       %%
%%                                                                  %%
%%                                                                  %%
%% Compilation Flags:                                               %%
%%                                                                  %%
%%   The following behavior may be controlled by boolean flags.     %%
%%                                                                  %%
%%   ForPrinting (false by default):                                %%
%%   If false, compile a screen optimized file (one-sided layout,   %%
%%   blue hyperlinks). If true, print-optimized PDF file: Larger    %%
%%   text block, two-sided layout, black hyperlinks.                %%
%%                                                                  %%
%%                                                                  %%
%% PDF pages: 352 (if ForPrinting set to false)                     %%
%% PDF page size: 4.75 x 6.75" (non-standard)                       %%
%%                                                                  %%
%% Summary of log file:                                             %%
%% * Two overfull hboxes (neither visible).                         %%
%%                                                                  %%
%% Compile History:                                                 %%
%%                                                                  %%
%% January, 2016: (Andrew D. Hwang)                                 %%
%%             texlive2012, GNU/Linux                               %%
%%                                                                  %%
%% Command block:                                                   %%
%%                                                                  %%
%%     pdflatex x3                                                  %%
%%     makeindex                                                    %%
%%     pdflatex x3                                                  %%
%%                                                                  %%
%%                                                                  %%
%% January 2016: pglatex.                                           %%
%%   Compile this project with:                                     %%
%%   pdflatex 50880-t.tex ..... THREE times                         %%
%%   makeindex 50880-t.idx                                          %%
%%   pdflatex 50880-t.tex ..... THREE times                         %%
%%                                                                  %%
%%   pdfTeX, Version 3.1415926-2.5-1.40.14 (TeX Live 2013/Debian)   %%
%%                                                                  %%
%%%%%%%%%%%%%%%%%%%%%%%%%%%%%%%%%%%%%%%%%%%%%%%%%%%%%%%%%%%%%%%%%%%%%%
\documentclass[12pt]{book}[2005/09/16]
\listfiles

%%%%%%%%%%%%%%%%%%%%%%%%%%%%% PACKAGES %%%%%%%%%%%%%%%%%%%%%%%%%%%%%%%
\usepackage[utf8]{inputenc}[2006/05/05]

\usepackage[german,UKenglish]{babel}

\usepackage{ifthen}[2001/05/26]  %% Logical conditionals

\usepackage{amsmath}[2000/07/18] %% Displayed equations
\usepackage{amssymb}[2002/01/22] %% and additional symbols

\usepackage{array}[2008/09/09]

\usepackage{alltt}[1997/06/16]   %% boilerplate, credits, license

\usepackage[perpage,symbol]{footmisc}[2005/03/17]

\usepackage[version=3]{mhchem}[2006/12/17]

\usepackage{multicol}[2006/05/18]
\usepackage{makeidx}[2000/03/29]

\usepackage{graphicx}[1999/02/16]%% For diagrams

\usepackage{calc}[2005/08/06]

\usepackage{fancyhdr} %% For running heads

%%%%%%%%%%%%%%%%%%%%%%%%%%%%%%%%%%%%%%%%%%%%%%%%%%%%%%%%%%%%%%%%%
%%%% Interlude:  Set up PRINTING (default) or SCREEN VIEWING %%%%
%%%%%%%%%%%%%%%%%%%%%%%%%%%%%%%%%%%%%%%%%%%%%%%%%%%%%%%%%%%%%%%%%

% ForPrinting=true                     false (default)
% Asymmetric margins                   Symmetric margins
% 1 : 1.6 text block aspect ratio      3 : 4 text block aspect ratio
% Black hyperlinks                     Blue hyperlinks
% Start major marker pages recto       No blank verso pages
%
\newboolean{ForPrinting}

%% UNCOMMENT the next line for a PRINT-OPTIMIZED VERSION of the text %%
%\setboolean{ForPrinting}{true}

%% Initialize values to ForPrinting=false
\newcommand{\Margins}{hmarginratio=1:1}     % Symmetric margins
\newcommand{\HLinkColor}{blue}              % Hyperlink color
\newcommand{\PDFPageLayout}{SinglePage}
\newcommand{\TransNote}{Transcriber's Note}
\newcommand{\TransNoteCommon}{%
  The camera-quality files for this public-domain ebook may be
  downloaded \textit{gratis} at
  \begin{center}
    \texttt{www.gutenberg.org/ebooks/\ebook}.
  \end{center}

  This ebook was produced using scanned images and OCR text generously
  provided by the University of California, Berkeley, through the
  Internet Archive.
  \bigskip

  Minor typographical corrections and presentational changes have been
  made without comment.
  \bigskip
}

\newcommand{\TransNoteText}{%
  \TransNoteCommon

  This PDF file is optimized for screen viewing, but may be recompiled
  for printing. Please consult the preamble of the \LaTeX\ source file
  for instructions and other particulars.
}
%% Re-set if ForPrinting=true
\ifthenelse{\boolean{ForPrinting}}{%
  \renewcommand{\Margins}{hmarginratio=2:3} % Asymmetric margins
  \renewcommand{\HLinkColor}{black}         % Hyperlink color
  \renewcommand{\PDFPageLayout}{TwoPageRight}
  \renewcommand{\TransNote}{Transcriber's Note}
  \renewcommand{\TransNoteText}{%
    \TransNoteCommon

    This PDF file is optimized for printing, but may be recompiled for
    screen viewing. Please consult the preamble of the \LaTeX\ source
    file for instructions and other particulars.
  }
}{% If ForPrinting=false, don't skip to recto
  \renewcommand{\cleardoublepage}{\clearpage}
}
%%%%%%%%%%%%%%%%%%%%%%%%%%%%%%%%%%%%%%%%%%%%%%%%%%%%%%%%%%%%%%%%%
%%%%  End of PRINTING/SCREEN VIEWING code; back to packages  %%%%
%%%%%%%%%%%%%%%%%%%%%%%%%%%%%%%%%%%%%%%%%%%%%%%%%%%%%%%%%%%%%%%%%

\ifthenelse{\boolean{ForPrinting}}{%
  \setlength{\paperwidth}{8.5in}%
  \setlength{\paperheight}{11in}%
% 1:1.6
  \usepackage[body={4.5in,8in},\Margins]{geometry}[2002/07/08]
}{%
  \setlength{\paperwidth}{4.75in}%
  \setlength{\paperheight}{6.75in}%
  \raggedbottom
% 3:4
  \usepackage[body={4.5in,5.875in},\Margins,includeheadfoot]{geometry}[2002/07/08]
}

\providecommand{\ebook}{00000}    % Overridden during white-washing
\usepackage[pdftex,
  hyperfootnotes=false,
  pdfauthor={Max Planck},
  pdftitle={The Project Gutenberg eBook \#\ebook: Treatise on Thermodynamics.},
  pdfkeywords={University of California, The Internet Archive, Andrew D. Hwang},
  pdfstartview=Fit,    % default value
  pdfstartpage=1,      % default value
  pdfpagemode=UseNone, % default value
  bookmarks=true,      % default value
  linktocpage=false,   % default value
  pdfpagelayout=\PDFPageLayout,
  pdfdisplaydoctitle,
  pdfpagelabels=true,
  bookmarksopen=true,
  bookmarksopenlevel=0,
  colorlinks=true,
  linkcolor=\HLinkColor]{hyperref}[2007/02/07]

%% Fixed-width environment to format PG boilerplate %%
\newenvironment{PGtext}{%
\begin{alltt}
%****
\fontsize{8.1}{10}\ttfamily\selectfont}%
{\end{alltt}}

% Errors found during digitization
\newcommand{\Typo}[2]{#2}

% Stylistic changes made for consistency
\newcommand{\Chg}[2]{#2}
%\newcommand{\Chg}[2]{#1} % Use this to revert inconsistencies in the original
\newcommand{\Add}[1]{\Chg{}{#1}}

% Book's list of errata
\newcommand{\Erratum}[2]{#2}

%% Miscellaneous global parameters %%
% No hrule in page header
\renewcommand{\headrulewidth}{0pt}

% Loosen spacing
\setlength{\emergencystretch}{1em}
\newcommand{\Loosen}{\spaceskip 0.375em plus 0.75em minus 0.25em}

% Scratch pad for length calculations
\newlength{\TmpLen}

%% Tables
\newcommand{\TableFont}{\footnotesize}%{\scriptsize}%
\newcommand{\ColHead}[1]{\multicolumn{1}{c}{#1}}
\newcommand{\ColumnHeading}[2]{%
  \settowidth{\TmpLen}{#1}%
  \parbox[c]{\TmpLen}{\centering\medskip #2\medskip}%
}
\newcommand{\Strut}[1][12pt]{\rule{0pt}{#1}}

%% Running heads %%
\newcommand{\FlushRunningHeads}{\cleardoublepage\fancyhf{}}
\newcommand{\InitRunningHeads}{%
  \setlength{\headheight}{15pt}
  \pagestyle{fancy}
  \thispagestyle{empty}
  \ifthenelse{\boolean{ForPrinting}}
             {\fancyhead[RO,LE]{\thepage}}
             {\fancyhead[R]{\thepage}}
}

\newcommand{\RHead}[1]{\textsc{\MakeLowercase{#1}}}

% Asymmetric running heads in print-formatted version
\ifthenelse{\boolean{ForPrinting}}{%
  \newcommand{\SetRunningHeads}[1]{%
    \fancyhead[CE]{\RHead{Thermodynamics.}}%
    \fancyhead[CO]{\RHead{#1}}}%
  }{% Uniform running head in screen mode
  \newcommand{\SetRunningHeads}[1]{%
    \fancyhead[C]{\RHead{#1}}}%
}

\newcommand{\BookMark}[2]{\phantomsection\pdfbookmark[#1]{#2}{#2}}

%% Major document divisions %%
\newcommand{\PGBoilerPlate}{%
  \pagenumbering{Alph}
  \pagestyle{empty}
  \BookMark{0}{PG Boilerplate}
}
\newcommand{\FrontMatter}{%
  \cleardoublepage
  \frontmatter
  \BookMark{-1}{Front Matter}
}
\newcommand{\MainMatter}{%
  \FlushRunningHeads
  \InitRunningHeads
  \mainmatter
  \BookMark{-1}{Main Matter}
}
\newcommand{\BackMatter}{%
  \FlushRunningHeads
  \InitRunningHeads
  \backmatter
  \BookMark{-1}{Back Matter}
}
\newcommand{\PGLicense}{%
  \FlushRunningHeads
  \pagenumbering{Roman}
  \InitRunningHeads
  \BookMark{-1}{PG License}
  \SetRunningHeads{License}
}

%% ToC formatting %%
\AtBeginDocument{\renewcommand{\contentsname}%
  {\protect\thispagestyle{empty}%
    \protect\SectTitle{CONTENTS}\protect\vspace{-1.5\baselineskip}}
}

\newcommand{\TableofContents}{%
  \FlushRunningHeads
  \InitRunningHeads
  \SetRunningHeads{Contents}
  \BookMark{0}{Contents}
  \tableofcontents
}

% Set the section number in a fixed-width box
\newcommand{\ToCBox}[1]{\settowidth{\TmpLen}{99.}%
  \makebox[\TmpLen][r]{#1}\hspace*{1em}%
}
% For internal use, to determine if we need the Sect./Page line
\newcommand{\ToCAnchor}{}

% \ToCLine{Counter}{SecNo}{Title}
\newcommand{\ToCLine}[3]{%
  \label{toc:#1}%
  \ifthenelse{\not\equal{\pageref{toc:#1}}{\ToCAnchor}}{%
    \renewcommand{\ToCAnchor}{\pageref{toc:#1}}%
    \noindent\makebox[\textwidth][r]{\scriptsize CHAPTER\hfill PAGE}\\%
  }{}%
  \settowidth{\TmpLen}{9999}%
  \noindent\strut\parbox[b]{\textwidth-\TmpLen}{\small%
     \ToCBox{#2}\hangindent4em\raggedright\textsc{#3}\dotfill}%
  \makebox[\TmpLen][r]{\pageref{chap:#1}}%
  \smallskip
}

%% Index formatting
\makeindex
\makeatletter
\renewcommand{\@idxitem}{\par\hangindent 30\p@\global\let\idxbrk\nobreak}
\renewcommand\subitem{\idxbrk\@idxitem \hspace*{12\p@}\let\idxbrk\relax}
\renewcommand{\indexspace}{\par\penalty-3000 \vskip 10pt plus5pt minus3pt\relax}

\renewenvironment{theindex}{%
  \setlength\columnseprule{0.5pt}\setlength\columnsep{18pt}%
  \begin{multicols}{2}[{\Appendix{Index}\small}]%
    \setlength\parindent{0pt}\setlength\parskip{0pt plus 0.3pt}%
    \thispagestyle{empty}\let\item\@idxitem\raggedright%
  }{%
  \end{multicols}\FlushRunningHeads
}
\makeatother

\newcommand{\indexnote}[1]{\hyperpage{#1}~\textit{n}.}
\renewcommand{\see}[1]{See~#1.}

%% Sectional units %%
% Typographical abstraction
\newcommand{\ChapHead}[2]{%
  \section*{\centering\normalfont\large\MakeUppercase{Chapter #1}}
  \subsection*{\centering\normalfont\normalsize\itshape\MakeUppercase{#2}}
}

\newcommand{\SectTitle}[2][\large]{%
  \section*{\centering#1\normalfont #2}
}
\newcommand{\SubsectTitle}[2][\normalsize]{%
  \subsection*{\centering#1\normalfont\textsc{#2}}
}

% Internal bookkeeping
\newcounter{ChapNo}%
\newboolean{ChapterClear}

% Title for title page and start of Part I.
\newcommand{\BookHead}{%
  \LARGE TREATISE \\[8pt]
  \normalsize ON \\[12pt]
  \TPage{0.9}{THERMODYNAMICS}
}

\newcommand{\Part}[2]{%
  \FlushRunningHeads
  \InitRunningHeads
  % Cross-referencing
  \BookMark{0}{Part #1}
  \addtocontents{toc}{\protect\SectTitle{PART #1}}
  \addtocontents{toc}{\protect\SubsectTitle[\protect\footnotesize]{\MakeUppercase{#2}}}
  % Page heading
  \setboolean{ChapterClear}{false}% Tell \Chapter not to start of a new page
  \ifthenelse{\equal{#1}{I.}}{% Special heading for Part I.
    \begin{center}
      \BookHead
    \end{center}
  }{}% Part II., etc.
  \SectTitle{PART #1}
  \SubsectTitle{#2}
}

% \Chapter{Title}
\newcommand{\Chapter}[3][]{%
  % Start on a new page unless we've just started a \Part
  \ifthenelse{\boolean{ChapterClear}}{%
    \FlushRunningHeads
    \InitRunningHeads
  }{% Else we've just started a part; set flag
    \setboolean{ChapterClear}{true}%
  }
  % ToC entry
  \refstepcounter{ChapNo}%
  \phantomsection\label{chap:\theChapNo}%
  \addtocontents{toc}{\protect\ToCLine{\theChapNo}{#2}{#3}}
  % PDF bookmark and running heads
  \ifthenelse{\equal{#1}{}}{% No manual running head
    \BookMark{1}{#3}
    \SetRunningHeads{#3}
    \ChapHead{#2}{#3}
  }{% Manual running head
    \BookMark{1}{#1}
    \SetRunningHeads{#1}
    \ChapHead{#2}{#3}
  }
}

\newcommand{\Section}[1]{
  \medskip\par\textbf{§\;#1}
  \label{section:#1}
}

% Unnumbered \Chapter-like units with no ToC entry
\newcommand{\Preface}[1]{%
  \FlushRunningHeads
  \InitRunningHeads
  \BookMark{0}{#1}
  \SetRunningHeads{#1}
  \SectTitle{\MakeUppercase{#1}}
}

% Unnumbered \Chapter-like units with ToC entry
\newcommand{\Appendix}[1]{%
  \FlushRunningHeads
  \InitRunningHeads
  % ToC entry
  \refstepcounter{ChapNo}%
  \phantomsection\label{chap:\theChapNo}%
  \ifthenelse{\equal{#1}{Catalogue}}{%
    \addtocontents{toc}{\protect\ToCLine{\theChapNo}{}{Catalogue of the Author's Papers on Thermodynamics}}
  }{%
    \addtocontents{toc}{\protect\ToCLine{\theChapNo}{}{#1}}
  }
  % PDF bookmark
  \BookMark{0}{#1}
  \SetRunningHeads{#1}
  \SectTitle{\MakeUppercase{#1}}
}

%% Cross-references %%
\newcommand{\SecNum}[1]{\hyperref[section:#1.]{{\upshape #1}}}
\newcommand{\SecRef}[2][§\;]{\hyperref[section:#2.]{{\upshape #1#2}}}
\newcommand{\SSecRef}[1]{\SecRef[§§\;]{#1}}
\newcommand{\SecRefs}[3][§§\;]{#1\SecNum{#2}--\SecNum{#3}}

% Equation tags
\newcommand{\Tag}[1]{%
  \phantomsection\label{eqn:#1}\tag*{\ensuremath{#1}}%
}
\newcommand{\Eq}[1]{%
  \hyperref[eqn:#1]{\ensuremath{#1}}%
}

% Figure labels
\newcommand{\Fig}[1]{\hyperref[fig:#1]{Fig.~{\upshape #1}}}

%% Diagrams %%
\renewcommand{\floatpagefraction}{0.8}
%\renewcommand{\topfraction}{0.95}
%\renewcommand{\bottomfraction}{0.95}

\newcommand{\Graphic}[2]{%
  \phantomsection\label{fig:#2}%
    \ifthenelse{\equal{#1}{}}{%
      \includegraphics{./images/#2.pdf}%
    }{% else width specified
      \includegraphics[width=#1]{./images/#2.pdf}%
    }
}
% \Figure[width]{figure number}
\newcommand{\Figure}[2][]{%
  \begin{figure}[hbtp!]
    \centering
    \phantomsection\label{fig:#2}
    \Graphic{#1}{fig#2}
  \end{figure}\ignorespaces%
}

% Page separators
\newcommand{\PageSep}[1]{\ignorespaces}

% Small semantic structures and abbreviations
% \TPage{fraction}{content}: scale content for title page
\newcommand{\TPage}[2]{\resizebox{#1\textwidth}{!}{#2}}
\newcommand{\Signature}[4]{%
  \medskip
  \null\hfill \large #1\hspace{\parindent}
  \bigskip

  \small%
  \textsc{#2} \\
  \hspace*{2.5\parindent}\textit{#3} #4\normalsize\par
}

\newcommand{\Topic}[1]{\textbf{#1}}
\newcommand{\Bibitem}{\medskip\par\noindent\hangindent2em}

\newcommand{\const}{\text{const.}}
\newcommand{\eg}{\emph{e.g.}}
\newcommand{\ie}{\emph{i.e.}}
\renewcommand{\cf}{\emph{cf.}}
\newcommand{\Cf}{\emph{Cf.}}
\DeclareMathOperator{\aq}{\, aq.}

\newcommand{\Tr}{\textsc{Tr}}

\newcommand{\First}[1]{\textsc{#1}}
\newcommand{\Emph}[1]{\emph{\textbf{#1}}}

%% Miscellaneous mathematical formatting %%
\newcommand{\efrac}[2]{\tfrac{#1}{#2}}
\newcommand{\dd}{\partial}
\newcommand{\tsum}{\mathop{\textstyle\sum}\limits}
\newcommand{\Z}{\phantom{0}}

\DeclareInputMath{176}{{}^{\circ}}
\DeclareInputMath{183}{\cdot}

\newcommand{\Unit}[1]{\text{\upshape #1}}
\newcommand{\eps}{\varepsilon}

% Large integral sign
\newcommand{\Bigint}[1][2]{%
  \scalebox{#1}{$\displaystyle\int$}{\kern-4pt}%
}

\newcommand{\EmRule}{\rule{12pt}{0pt}}

\newcommand{\Bigintlimits}[3][2]{%
  \mathop{\scalebox{#1}{$\displaystyle\int$}}\limits_{\mbox{$#2$\EmRule}}^{\mbox{\EmRule$#3$}}\kern-2pt%
}

% \PadTo[alignment]{width text}{visible text}
\newcommand{\PadTo}[3][c]{%
  \settowidth{\TmpLen}{$#2$}%
  \makebox[\TmpLen][#1]{$#3$}%
}

%%%%%%%%%%%%%%%%%%%%%%%% START OF DOCUMENT %%%%%%%%%%%%%%%%%%%%%%%%%%
\begin{document}
%% PG BOILERPLATE %%
\PGBoilerPlate
\begin{center}
\begin{minipage}{\textwidth}
\small
\begin{PGtext}
The Project Gutenberg EBook of Treatise on Thermodynamics, by Max Planck

This eBook is for the use of anyone anywhere in the United States and most
other parts of the world at no cost and with almost no restrictions
whatsoever.  You may copy it, give it away or re-use it under the terms of
the Project Gutenberg License included with this eBook or online at
www.gutenberg.org.  If you are not located in the United States, you'll have
to check the laws of the country where you are located before using this ebook.

Title: Treatise on Thermodynamics

Author: Max Planck

Translator: Alexander Ogg

Release Date: January 9, 2016 [EBook #50880]
Most recently updated: June 11, 2021

Language: English

Character set encoding: UTF-8     

*** START OF THIS PROJECT GUTENBERG EBOOK TREATISE ON THERMODYNAMICS ***
\end{PGtext}
\end{minipage}
\end{center}
\newpage
%% Credits and transcriber's note %%
\begin{center}
\begin{minipage}{\textwidth}
\begin{PGtext}
Produced by Andrew D. Hwang
\end{PGtext}
\end{minipage}
\vfill
\end{center}

\begin{minipage}{0.85\textwidth}
\small
\BookMark{0}{Transcriber's Note.}
\subsection*{\centering\normalfont\scshape%
\normalsize\MakeLowercase{\TransNote}}%

\raggedright
\TransNoteText
\end{minipage}
%%%%%%%%%%%%%%%%%%%%%%%%%%% FRONT MATTER %%%%%%%%%%%%%%%%%%%%%%%%%%
\PageSep{i}
\FrontMatter
\ifthenelse{\boolean{ForPrinting}}{%
\null\vfill
\TPage{0.9}{TREATISE ON THERMODYNAMICS}
\vfill
\cleardoublepage
}{}% Omit half-title in screen version
\PageSep{ii}
%[Blank page]
\PageSep{iii}
{\centering
% [** TN: Commend prints ``Treatise on Thermodynamics'']
\BookHead
\vfill

\footnotesize
BY \\
\Large \textsc{Dr.\ MAX PLANCK} \\[-4pt]
\TPage{0.9}{\scriptsize
PROFESSOR OF THEORETICAL PHYSICS IN THE UNIVERSITY OF BERLIN}
\bigskip
\vfill

\normalsize\textit{TRANSLATED WITH THE AUTHOR'S SANCTION} \\[8pt]
\footnotesize BY \\[8pt]
\large \textsc{ALEXANDER OGG, M.A., B.Sc., Ph.D.}

\TPage{0.9}{\scriptsize LATE 1851 EXHIBITION SCHOLAR AND UNIVERSITY ASSISTANT, ABERDEEN UNIVERSITY} \\[-4pt]
\TPage{0.8}{\scriptsize ASSISTANT MASTER, ROYAL NAVAL ENGINEERING COLLEGE, DEVONPORT}
\vfill

%% Publisher's device
\Graphic{1.5in}{device}
\vfill

\TPage{0.8}{LONGMANS, GREEN, AND CO.} \\
\TPage{0.6}{39 PATERNOSTER ROW, LONDON} \\
\TPage{0.4}{NEW YORK AND BOMBAY} \\
1903 \\
\footnotesize
\textit{All rights reserved}

}% End \centering
\newpage
\PageSep{iv}
% [** Blank page]
\PageSep{v}

\Preface{Translator's Notice}

\First{The} modern developments of Thermodynamics, and the
applications to physical and chemical problems, have
become so important, that I have ventured to translate
Professor Planck's book, which presents the whole subject
from a uniform point of view.

A few notes have been added to the present English
edition by Professor Planck. He has not found it necessary
to change the original text in any way.

To bring the notation into conformity with the usual
English notation, several symbols have been changed.
This has been done with the author's sanction. Here I
have followed J.~J. van~Laar and taken $\Psi$ to signify what
he calls the \emph{Planck'sches Potential}, \Chg{i.e.}{\ie}\ the thermodynamic
potential of Gibbs and Duhem divided by~$-\theta$.

Professor Planck's recent paper, ``Über die Grundlage
der Lösungstheorie'' (Ann.\ d.\ Phys.\ \textbf{10}, p.~436, 1903), ought
to be read in connection with his thermodynamical theory
of solution.

I am indebted to Herren Veit \&~Co., Leipzig, for
kindly supplying the blocks of the five figures in the text.

\Signature{A. O.}{Devonport,}{June,}{1903.}
\PageSep{vi}
%[** Blank page]
\PageSep{vii}


\Preface{Preface}

\First{The} oft-repeated requests either to publish my collected
papers on Thermodynamics, or to work them up into a
comprehensive treatise, first suggested the writing of this
book. Although the first plan would have been the
simpler, especially as I found no occasion to make any
important changes in the line of thought of my original
papers, yet I decided to rewrite the whole subject-matter,
with the intention of giving at greater length, and with
more detail, certain general considerations and demonstrations
too concisely expressed in these papers. My chief
reason, however, was that an opportunity was thus offered
of presenting the entire field of Thermodynamics from a
uniform point of view. This, to be sure, deprives the work
of the character of an original contribution to science, and
stamps it rather as an introductory text-book on Thermodynamics
for students who have taken elementary courses in
Physics and Chemistry, and are familiar with the elements
of the Differential and Integral Calculus.

Still, I do not think that this book will entirely supersede
my former publications on the same subject. Apart
from the fact that these contain, in a sense, a more original
presentation, there may be found in them a number of
details expanded at greater length than seemed advisable
in the more comprehensive treatment here required. To
\PageSep{viii}
enable the reader to revert in particular cases to the
original form for comparison, a list of my publications on
Thermodynamics has been appended, with a reference in
each case to the section of the book which deals with
the same point.

The numerical values in the examples, which have been
worked, as applications of the theory, have, almost all of
them, been taken from the original papers; only a few,
that have been determined by frequent measurement, have
been taken from the tables in Kohlrausch's ``Leitfaden der
praktischen Physik.'' It should be emphasized, however,
that the numbers used, notwithstanding the care taken,
have not undergone the same amount of critical sifting
as the more general propositions and deductions.
\bigskip

Three distinct methods of investigation may be clearly
recognized in the previous development of Thermodynamics.
The first penetrates deepest into the nature of the processes
considered, and, were it possible to carry it out exactly,
would be designated as the most perfect. Heat, according
to it, is due to the definite motions of the chemical
molecules and atoms considered as distinct masses, which
in the case of gases possess comparatively simple properties,
but in the case of solids and liquids can be only very
roughly sketched. This kinetic theory, founded by Joule,
Waterston, Krönig and Clausius, has been greatly extended
mainly by Maxwell and Boltzmann. Obstacles, at present
unsurmountable, however, seem to stand in the way of its
further progress. These are due not only to the highly
complicated mathematical treatment, but principally to
essential difficulties, not to be discussed here, in the
mechanical interpretation of the fundamental principles of
Thermodynamics.
\PageSep{ix}

Such difficulties are avoided by the second method,
developed by Helmholtz. It confines itself to the most
important hypothesis of the mechanical theory of heat,
that heat is due to motion, but refuses on principle to
specialize as to the character of this motion. This is a
safer point of view than the first, and philosophically quite
as satisfactory as the mechanical interpretation of nature
in general, but it does not as yet offer a foundation of
sufficient breadth upon which to build a detailed theory.
Starting from this point of view, all that can be obtained
is the verification of some general laws which have already
been deduced in other ways direct from experience.

A third treatment of Thermodynamics has hitherto
proved the most fruitful. This method is distinct from
the other two, in that it does not advance the mechanical
theory of heat, but, keeping aloof from definite assumptions
as to its nature, starts direct from a few very general
empirical facts, mainly the two fundamental principles of
Thermodynamics. From these, by pure logical reasoning,
a large number of new physical and chemical laws are
deduced, which are capable of extensive application, and
have hitherto stood the test without exception.

This last, more inductive, treatment, which is used exclusively
in this book, corresponds best to the present state
of the science. It cannot be considered as final, however,
but may have in time to yield to a mechanical, or perhaps
an electro-magnetic theory. Although it may be of advantage
for a time to consider the activities of nature---Heat,
Motion, Electricity, etc.---as different in quality, and to
suppress the question as to their common nature, still our
aspiration after a uniform theory of nature, on a mechanical
basis or otherwise, which has derived such powerful encouragement
from the discovery of the principle of the
\PageSep{x}
conservation of energy, can never be permanently repressed.
Even at the present day, a recession from the assumption
that all physical phenomena are of a common nature
would be tantamount to renouncing the comprehension of
a number of recognized laws of interaction between different
spheres of natural phenomena. Of course, even then, the
results we have deduced from the two laws of Thermodynamics
would not be invalidated, but these two laws
would not be introduced as independent, but would be
deduced from other more general propositions. At present,
however, no probable limit can be set to the time which it
will take to reach this goal.

\Signature{THE AUTHOR.}{Berlin,}{April,}{1897.}
\PageSep{xi}
%[** Table of contents]
\TableofContents

\iffalse

CONTENTS

PART I.

Fundamental Facts and Definitions

CHAPTER PAGE

I. Temperature 1

II. Molecular Weight 22

III. Quantity of Heat 32

PART II.

The First Fundamental Principle of Thermodynamics

I. General Exposition 38

II. Applications to Homogeneous Systems 46

III. Applications to Non-Homogeneous Systems 67

PART III.

The Second Fundamental Principle of Thermodynamics

I. Introduction 77

II. Proof 86

III. General Deductions . 105
\PageSep{xii}

PART IV.

Applications to Special States of Equilibrium

CHAPTER PAGE

I. Homogeneous Systems . 119

II. System in Different States of Aggregation . . . . 132

III. System of any Number of Independent Constituents . 173

IV. Gaseous System 207

V. Dilute Solutions 223

Catalogue of the Author's Papers on Thermodynamics 264

Index . 267
\PageSep{xiii}

ERRATA

The reader is requested kindly to make the following corrections in the
pages where they occur:---

Page 26, line 8, for " occurs " read " occur."

30, 13 from bottom, for " Hydrobromamylene " read " Amy-
lene hydrobromide."

40, 4 in 58, for "different "read "definite."

78, 5, for " restablishment " read " re-establishment."

78, 14, for "stakes" read "states."

79, 9 in 108, for "Occasionally," etc., read "That attempts
are still made to represent this law as contained in the
principle of energy may be seen from the fact that the
too restricted term ' Energetics ' is sometimes applied to
all investigations on these questions."

87, 2 in 118, for " heat " read " work."

104, 10 from end, for " metaphysicists " read " metaphysicians."

149, 3, /or "v" read "V'



152, lines 3 and 1 from end, read and throughout.

\O v /2 \O V

176, line 8, for AM a " read " AM,'."

177, equation (149), for " M 2 " " read " 5M 2 "."
180, line 22, for " quintiple " read " quintuple."
185, 2, for " dM 2 " read " dM 2 '."



186, equation 153, for dM 2 5 read dM 2 'S-,

G/iVl 2 O^'-z

191, line 3, left-hand side of equation, for " d log p " read " d log p "
202, lines 6 and 8, for " &lt;/&gt; " read " &lt;p."
214. line 15, for " n^c^ " read " n^."

229, 14, for " are " read " is."

230, 14, for second minus read plus.

232, 2 from end, for " equations " read * equation."
241, equation (225), for " 2 " read " 2 ."

241, line 5 from end, for " carbonic " read " carbon."

242, 8,/or"0 2 "read0 2 ."

243, 11 from end, for "molecule " read "gram molecule."
% End of table of contents and errata text
\fi
\PageSep{xiv}
% [** Blank page]
\PageSep{1}


\MainMatter

% [** TN: Text written by the \Part macro]
%TREATISE
%ON
%THERMODYNAMICS

\Part{I.}{Fundamental Facts and Definitions.}
\index{Laws of thermodynamics@Laws of thermodynamics|see{\textit{First} and \text{Second}}}%

\Chapter{I.}{Temperature.}

\Section{1.} \First{The} conception of ``heat'' arises from that particular
\index{Free energy}%
\index{Heat!conception of}%
sensation of warmth or coldness which is immediately
experienced on touching a body. This direct sensation,
however, furnishes no quantitative scientific measure of a
body's state with regard to heat; it yields only qualitative
results, which vary according to external circumstances.
For quantitative purposes we utilize the change of volume
which takes place in all bodies when heated under constant
pressure, for this admits of exact measurement. Heating
produces in most substances an increase of volume, and thus
we can tell whether a body gets hotter or colder, not merely
by the sense of touch, but also by a purely mechanical
observation affording a much greater degree of accuracy.
We can also tell accurately when a body assumes a former
state of heat.

\Section{2.} If two bodies, one of which feels warmer than the
other, be brought together (for example, a piece of heated
metal and cold water), it is invariably found that the hotter
body is cooled, and the colder one is heated up to a certain
\PageSep{2}
point, and then all change ceases. The two bodies are then
said to be in \emph{thermal equilibrium}. Experience shows that
\index{Equilibrium!thermal}%
\index{Thermal equilibrium}%
such a state of equilibrium finally sets in, not only when
two, but also when any number of differently heated bodies
are brought into mutual contact. From this follows the
important proposition: \emph{If a body,~$A$, be in thermal equilibrium
with two other bodies, $B$~and~$C$, then $B$~and $C$ are in
thermal equilibrium with one another.}\footnote
  {\Loosen As is well known, there exists no corresponding proposition for electrical
  equilibrium. For if we join together the substances \ce{Cu $\mid$ CuSO4 $\aq$ $\mid$
  ZnSO4 $\aq$ $\mid$ Zn} to form a conducting ring, no electrical equilibrium is
  possible.}
For, if we bring $A$,~$B$,
and~$C$ together so that each touches the other two, then,
according to our supposition, there will be equilibrium at
the points of contact $AB$ and~$AC$, and, therefore, also at the
contact~$BC$. If it were not so, no general thermal equilibrium
would be possible, which is contrary to experience.

\Section{3.} These facts enable us to compare the degree of heat
of two bodies, $B$~and~$C$, without bringing them into contact
with one another; namely, by bringing each body into
contact with an arbitrarily selected standard body,~$A$ (for
example, a mass of mercury enclosed in a vessel terminating
in a fine capillary tube). By observing the volume of~$A$
in each case, it is possible to tell whether $B$~and $C$ are in
thermal equilibrium or not. If they are not in thermal
equilibrium, we can tell which of the two is the hotter. The
degree of heat of~$A$, or of any body in thermal equilibrium
with~$A$, can thus be very simply defined by the volume of~$A$,
or, as is usual, by the difference between the volume of~$A$
and its volume when in thermal equilibrium with melting
ice under atmospheric pressure. This volumetric difference,
which, by an appropriate choice of unit, is made to read $100$
when $A$~is in contact with steam under atmospheric pressure,
is called the \emph{temperature} in degrees Centigrade with regard
\index{Temperature!definition of}%
to~$A$ as thermometric substance. Two bodies of equal
temperature are, therefore, in thermal equilibrium, and \textit{vice
versâ}.
\PageSep{3}

\Section{4.} The temperature readings of no two thermometric
\index{Temperature!definition of}%
substances agree, in general, except at~$0°$ and~$100°$. The
definition of temperature is therefore somewhat arbitrary.
This we may remedy to a certain extent by taking gases, in
particular those hard to condense, such as hydrogen, oxygen,
nitrogen, and carbon monoxide, as thermometric substances.
They agree almost completely within a considerable range
of temperature, and their readings are sufficiently in accordance
for most purposes. Besides, the coefficient of expansion
of these different gases is the same, inasmuch as equal
volumes of them expand under constant pressure by the
same amount---about $\frac{1}{273}$~of their volume---when heated
from $0°$~C. to $1°$~C\@. Since, also, the influence of the external
pressure on the volume of these gases can be represented by
a very simple law, we are led to the conclusion that these
regularities are based on a remarkable simplicity in their
constitution, and that, therefore, it is reasonable to define
the common temperature given by them simply as temperature.
We must consequently reduce the readings of other
thermometers to those of the gas thermometer, and preferably
\index{Gas!thermometer}%
\index{Thermometer, gas}%
to those of the hydrogen thermometer.

\Section{5.} The definition of temperature remains arbitrary in
cases where the requirements of accuracy cannot be satisfied
by the agreement between the readings of the different
gas thermometers, for there is no sufficient reason for the
preference of any one of these gases. A definition of temperature
completely independent of the properties of any
individual substance, and applicable to all stages of heat
and cold, becomes first possible on the basis of the \emph{second
law of thermodynamics} (\SecRef{160}, etc.). In the mean time, only
such temperatures will be considered as are defined with
sufficient accuracy by the gas thermometer.

\Section{6.} In the following we shall deal chiefly with homogeneous,
isotropic bodies of any form, possessing throughout
\index{Isotropic bodies}%
their substance the same temperature and density, and
subject to a uniform pressure acting everywhere perpendicular
to the surface. They, therefore, also exert the same
\PageSep{4}
pressure outwards. Surface phenomena are thereby disregarded.
The condition of such a body is determined by
its chemical nature; its mass,~$M$; its volume,~$V$; and its
temperature,~$t$. On these must depend, in a definite manner,
all other properties of the particular state of the body,
especially the pressure, which is uniform throughout, internally
and externally. The pressure,~$p$, is measured by the
force acting on the unit of area---in the C.G.S. system, in
dynes per square centimeter, a \emph{dyne} being the force which
\index{Dyne}%
imparts to a mass of one gramme in one second a velocity
of one centimeter per second.

\Section{7.} As the pressure is generally given in atmospheres,
\index{Atmospheric pressure}%
the value of an atmosphere in absolute C.G.S. units is here
calculated. The pressure of an atmosphere is the weight
of a column of mercury at~$0°$~C., $76~\Unit{cm.}$ high, and $1~\Unit{sq.}\ \Unit{cm.}$
in cross-section, when placed in mean geographical latitude.
This latter condition must be added, because the weight,
\ie\ the force of the earth's attraction, varies with the locality.
The volume of the column of mercury is $76~\Chg{\Unit{c.c.}}{\Unit{cm.}^{3}}$; and since
the density of mercury at~$0°$~C. is $13.596$, the mass is $76 × 13.596$.
Multiplying the mass by the acceleration of gravity
in mean latitude, we find the pressure of one atmosphere in
absolute units to be
\[
76 × 13.596 × 981 = 1,013,650\ \frac{\Unit{dynes}}{\Unit{cm.}^{2}}
\quad\text{or}\quad \frac{\Unit{gr.}}{\Unit{cm.-sec.}^{2}}.
\]
This, then, is the factor for converting atmospheres into
absolute units. If, as was formerly the custom in mechanics,
we use as the unit of force the weight of a gramme in mean
geographical latitude instead of the dyne, the pressure of
an atmosphere would be $76 × 13.596 = 1033.3~\Unit{grms.}$ per
square centimeter.

\Section{8.} Since the pressure in a given substance is evidently
controlled by its internal physical condition only, and not
by its form or mass, it follows that $p$~depends only on the
temperature and the ratio of the mass~$M$ to the volume~$V$
\PageSep{5}
(\ie\ the density), or on the reciprocal of the density, the
volume of unit mass---
\[
\frac{V}{M} = v,
\]
which is called the specific volume of the substance. For
every substance, then, there exists a characteristic relation---
\[
p = f(v, t),
\]
which is called the \emph{characteristic equation} of the substance.
\index{Characteristic equation}%
\index{Equation!characteristic}%
For gases, the function~$f$ is invariably positive; for liquids
and solids, however, it may have also negative values under
certain circumstances.

\Section{9.} \Topic{Perfect Gases.}---The characteristic equation assumes
\index{Boyle's law}%
\index{Gases, perfect}%
\index{Laws:!Boyle's}%
\index{Laws:!Mariotte's}%
\index{Mariotte's law}%
\index{Perfect gases}%
its simplest form for the substances which we used
in \SecRef{4} for the definition of temperature. If the temperature
be kept constant, then, according to the Boyle-Mariotte
law, the product of the pressure and the specific volume
remains constant for gases---
\[
pv = T,
\Tag{(1)}
\]
where $T$, for a given gas, depends only on the temperature.

But if the pressure be kept constant, then, according to~\SecRef{3},
the temperature is proportional to the difference between
the present volume~$v$ and the volume~$v_{0}$ at~$0°$; \ie---
\[
t = (v - v_{0})P,
\Tag{(2)}
\]
where $P$~depends only on the pressure~$p$. Equation~\Eq{(1)} for~$v_{0}$
becomes
\[
pv_{0} = T_{0},
\Tag{(3)}
\]
where $T_{0}$~is the value of the function~$T$, when $t = 0°$~C\@.

Finally, as has already been mentioned in~\SecRef{4}, the
expansion of all permanent gases on heating from~$0°$~C. to
$1°$~C. is the same fraction~$\alpha$ (about~$\frac{1}{273}$) of their volume at~$0°$
\PageSep{6}
(Gay-Lussac's law). Putting $t = 1$, we have $v - v_{0} = \alpha v_{0}$,
\index{Laws:!Gay-Lussac's}%
and equation~\Eq{(2)} becomes
\index{Equation!characteristic}%
\[
1 = \alpha v_{0} P\Add{.}
\Tag{(4)}
\]
By eliminating $P$,~$v_{0}$, and~$v$ from \Eq{(1)}, \Eq{(2)}, \Eq{(3)},~\Eq{(4)}, we obtain
the temperature function of the gas---
\index{Temperature!absolute}%
\[
T = T_{0} (1 + \alpha t),
\]
which is seen to be a linear function of~$t$. The characteristic
equation~\Eq{(1)} becomes
\[
p = \frac{T_{0}}{v} (1 + \alpha t).
\]

\Section{10.} The form of this equation is considerably simplified
by shifting the zero of temperature, arbitrarily fixed in~\SecRef{3},
by $\dfrac{1}{\alpha}$ degrees, and calling the melting point of ice, not $0°$~C.,
but $\dfrac{1°}{\alpha}$~C. (\ie\ about $273°$~C.). For, putting $t + \dfrac{1}{\alpha} = \theta$
(absolute temperature), and the constant $\alpha T_{0} = C$, the
\index{Absolute temperature}%
characteristic equation becomes
\index{Characteristic equation}%
\[
p = \frac{C}{v} \theta = \frac{CM}{V} \theta\Add{.}
\Tag{(5)}
\]
This introduction of \emph{absolute} temperature is evidently tantamount
to measuring temperature no longer, as in~\SecRef{3}, by a
change of volume, but by the volume itself.

\Section{11.} The constant~$C$, which is characteristic for the
perfect gas under consideration, can be calculated, if the
specific volume~$v$ be known for any pair of values of~$\theta$ and~$p$
(\eg\ $0°$~and $1$~atmosphere). For different gases, taken at
the same temperature and pressure, the constants~$C$ evidently
vary directly as the specific volumes, or inversely as the
densities~$\dfrac{1}{v}$. It may be affirmed, then, that, taken at the
same temperature and pressure, the densities of all perfect
gases bear a constant ratio to one another. A gas is,
therefore, often characterized by the constant ratio which its
\PageSep{7}
density bears to that of a normal gas at the same temperature
\index{Density!specific}%
and pressure (\emph{specific density} relative to air or hydrogen).
\index{Specific density}%
At $0°$~C. ($\theta = 273°$) and $1$~atmosphere pressure, the densities
of the following gases are:
% [** TN: Removing dot leaders]
\begin{center}
\TableFont
\begin{tabular}{ll}
Hydrogen & $0.00008988~\dfrac{\Unit{gr.}}{\Unit{cm.}^{3}}$ \\
Oxygen & $0.0014291$ \\
Nitrogen & $0.0012507$ \\
Atmospheric nitrogen & $0.0012571$ \\
Air & $0.0012930$ \\
\end{tabular}
\end{center}
whence the corresponding values of~$C$ in absolute units can
be readily calculated.

All questions with regard to the behaviour of a substance
when subjected to changes of temperature, volume, and
pressure are completely answered by the characteristic
equation of the substance.

\Section{12.} \Topic{Behaviour under Constant Pressure} (Isopiestic
or Isobaric Changes).---\emph{Coefficient of expansion} is the name
\index{Coefficient!of elasticity}%
\index{Coefficient!of expansion}%
\index{Coefficient!of pressure}%
\index{Elasticity, coefficient of}%
\index{Expansion, coefficient of}%
\index{Isobaric change}%
\index{Isochoric change}%
\index{Isopiestic change}%
\index{Isopycnic change}%
given to the ratio of the increase of volume for a rise of
temperature of $1°$~C. to the volume at $0°$~C\@. This increase
for a perfect gas is, according to~\Eq{(5)}, $\dfrac{CM}{p}$. The same equation~\Eq{(5)}
gives the volume of the gas at $0°$~C. as $\dfrac{CM}{p} × 273$,
hence the ratio of the two quantities, or the coefficient of
expansion, is $\frac{1}{273} = \alpha$.

\Section{13.} \Topic{Behaviour at Constant Volume} (Isochoric or
Isopycnic Changes).---The \emph{pressure coefficient} is the ratio
\index{Pressure coefficient}%
of the increase of pressure for a rise of temperature of~$1°$ to
the pressure at $0°$~C\@. For a perfect gas, this increase, according
to equation~\Eq{(5)}, is~$\dfrac{CM}{V}$. The pressure at $0°$~C. is $\dfrac{CM}{V} × 273$,
whence the required ratio, \ie\ the pressure coefficient, is~$\frac{1}{273}$,
therefore equal to the coefficient of expansion~$\alpha$.

\Section{14.} \Topic{Behaviour at Constant Temperature} (Isothermal
Changes).---\emph{Coefficient of elasticity} is the ratio of an
\PageSep{8}
infinitely small increase of pressure to the resulting contraction
of unit volume of the substance. In a perfect gas,
according to equation~\Eq{(5)}, the contraction of volume~$V$, in
consequence of an increase of pressure~$dp$, is
\[
dV = \frac{CM\theta}{p^{2}}\, dp = \frac{V}{p}\, dp.
\]
The contraction of unit volume is therefore
\[
-\frac{dV}{V} = \frac{dp}{p},
\]
and the coefficient of elasticity of the gas is
\index{Coefficient!of compressibility}%
\[
\frac{\;\;dp\;\;}{\dfrac{dp}{p}} = p,
\]
that is, equal to the pressure.

The reciprocal of the coefficient of elasticity, \ie\ the ratio
of an infinitely small contraction of unit volume to the
corresponding increase of pressure, is called the \emph{coefficient
of compressibility}.

\Section{15.} The three coefficients which characterize the behaviour
of a substance subject to isopiestic, isopycnic, and
isothermal changes are not independent of one another,
but are in every case connected by a definite relation.
The general characteristic equation, on being differentiated,
gives
\[
dp = \left(\frac{\dd p}{\dd \theta}\right)_{v} d\theta + \left(\frac{\dd p}{\dd v}\right)_{\theta} dv,
\]
where the suffixes indicate the variables to be kept constant
while performing the differentiation. By putting $dp = 0$
we impose the condition of an isopiestic change, and obtain
the relation between $dv$ and~$d\theta$ in isopiestic processes:---
\[
\left(\frac{\dd v}{\dd \theta}\right)_{p}
  = -\frac{\left(\dfrac{\dd p}{\dd \theta}\right)_{v}}
          {\left(\dfrac{\dd p}{\dd v}\right)_{\theta}}\Add{.}
\Tag{(6)}
\]
\PageSep{9}

For every state of a substance, one of the three
coefficients, viz.\ of expansion, of pressure, or of compressibility,
may therefore be calculated from the other two.

Take, for example, mercury at $0°$~C. and under atmospheric
pressure. Its coefficient of expansion is (\SecRef{12})
\[
\left(\frac{\dd v}{\dd \theta}\right)_{p} · \frac{1}{v_{0}} = 0.00018,
\]
its coefficient of compressibility in atmospheres (\SecRef{14}) is
\[
-\left(\frac{\dd v}{\dd p}\right)_{\theta} · \frac{1}{v_{0}} = 0.000003,
\]
therefore its pressure coefficient in atmospheres (\SecRef{13}) is
\index{Pressure coefficient!of mercury}%
\[
\left(\frac{\dd p}{\dd \theta}\right)_{v}
  = -\left(\frac{\dd p}{\dd v}\right)_{\theta} · \left(\frac{\dd v}{\dd \theta}\right)_{p}
  = -\frac{\left(\dfrac{\dd v}{\dd \theta}\right)_{p}}
          {\left(\dfrac{\dd v}{\dd p}\right)_{\theta}}
  = \frac{0.00018}{0.000003}
  = 60.
\]
This means that an increase of pressure of $60$~atmospheres
is required to keep the volume of mercury constant when
heated from $0°$~C. to $1°$~C\@.

\Section{16.} \Topic{Mixture of Perfect Gases.}---If any quantities of
\index{Gas mixture}%
\index{Mixture of gases}%
the \emph{same} gas at the same temperatures and pressures be
at first separated by partitions, and then allowed to come
suddenly in contact with another by the removal of these
partitions, it is evident that the volume of the entire system
will remain the same and be equal to the sum-total of the
partial volumes. Starting with quantities of \emph{different} gases,
experience still shows that, when pressure and temperature
are maintained uniform and constant, the total volume
continues equal to the sum of the volumes of the constituents,
notwithstanding the slow process of intermingling---diffusion---which
\index{Diffusion}%
takes place in this case. Diffusion goes
on until the mixture has become at every point of precisely
the same composition, \ie\ physically homogeneous.

\Section{17.} Two views regarding the constitution of mixtures
thus formed present themselves. Either we might assume
\PageSep{10}
that the individual gases, while mixing, split into a large
number of small portions, all retaining their original volumes
and pressures, and that these small portions of the different
gases, without penetrating each other, distribute themselves
evenly throughout the entire space. In the end each gas
would still retain its original volume (partial volume), and
all the gases would have the same common pressure. Or,
we might suppose---and this view will be shown below (\SecRef{32})
to be the correct one---that the individual gases change
and interpenetrate in every infinitesimal portion of the
volume, and that after diffusion each individual gas, in so
far as one may speak of such, fills the total volume, and is
consequently under a lower pressure than before, diffusion.
This so-called partial pressure of a constituent of a gas
mixture can easily be calculated.

\Section{18.} Denoting the quantities referring to the individual
gases by suffixes---$\theta$~and $p$ requiring no special designation,
as they are supposed to be the same for all the gases,---the
characteristic equation~\Eq{(5)} gives for each gas before
diffusion
\[
p = \frac{C_{1}M_{1} \theta}{V_{1}};\quad
p = \frac{C_{2}M_{2} \theta}{V_{2}};\quad \dots\Add{.}
\]
The total volume,
\[
V = V_{1} + V_{2} + \dots,
\]
remains constant during diffusion. After diffusion we ascribe
to each gas the total volume, and hence the partial
pressures become
\[
p_{1} = \frac{C_{1}M_{1} \theta}{V} = \frac{V_{1}}{V} p;\quad
p_{2} = \frac{C_{2}M_{2} \theta}{V} = \frac{V_{2}}{V} p;\quad \dots\Add{,}
\Tag{(7)}
\]
and by addition
\[
p_{1} + p_{2} + \dots = \frac{V_{1} + V_{2} + \dots}{V} p = p\Add{.}
\Tag{(8)}
\]
This is Dalton's law, that in a homogeneous mixture of
\index{Dalton's law}%
\index{Laws:!Dalton's}%
\index{Partial pressures}%
\PageSep{11}
gases the pressure is equal to the sum of the partial pressures
of the gases. It is also evident that
\[
p_{1} : p_{2} : \dots = V_{1} : V_{2} : \dots = C_{1}M_{1} : C_{2}M_{2} : \dots\Add{,}
\Tag{(9)}
\]
\ie\ the partial pressures are proportional to the volumes of
the gases before diffusion, or to the partial volumes which
the gases would have according to the first view of diffusion
given above.

\Section{19.} The characteristic equation of the mixture, according
\index{Characteristic equation}%
\index{Equation!characteristic}%
to \Eq{(7)}~and \Eq{(8)}, is
\begin{align*}
p &= (C_{1}M_{1} + C_{2}M_{2} + \dots) \frac{\theta}{V} \\
  &= \left(\frac{C_{1}M_{1} + C_{2}M_{2} + \dots}{M}\right) \frac{M}{V} \theta
\Tag{(10)}
\end{align*}
which corresponds to the characteristic equation of a perfect
gas with the following characteristic constant:---
\index{Characteristic constant}%
\[
C = \frac{C_{1}M_{1} + C_{2}M_{2} + \dots}{M_{1} + M_{2} + \dots}\Add{.}
\Tag{(11)}
\]
Hence the question as to whether a perfect gas is a
chemically simple one, or a mixture of chemically different
gases, cannot in any case be settled by the investigation of
the characteristic equation.

\Section{20.} The composition of a gas mixture is defined, either
\index{Air, composition of}%
by the ratios of the masses, $M_{1}$,~$M_{2}$,~\dots or by the ratios
of the partial pressures $p_{1}$,~$p_{2}$,~\dots or the partial volumes
$V_{1}$,~$V_{2}$,~\dots of the individual gases. Accordingly we
speak of per cent.\ by weight or by volume. Let us take
for example atmospheric air, which is a mixture of oxygen~(1)
and ``atmospheric'' nitrogen~(2).

The ratio of the densities of oxygen, ``atmospheric''
nitrogen and air is, according to~\SecRef{11},
\[
0.0014291 : 0.0012571 : 0.0012930 = \frac{1}{C_{1}} : \frac{1}{C_{2}} : \frac{1}{C_{3}}\Add{.}
\]
\PageSep{12}
Taking into consideration the relation~\Eq{(11)}---
\[
C = \frac{C_{1}M_{1} + C_{2}M_{2}}{M_{1} + M_{2}},
\]
we find the ratio $M_{1} : M_{2} = 0.2998$, \ie\ $23.1$ per~cent.\
by weight of oxygen and $76.9$ per~cent.\ of nitrogen.
Furthermore,
\[
C_{1}M_{1} : C_{2}M_{2} = p_{1} : p_{2} = V_{1} : V_{2} = 0.2637
\]
\ie\ $20.9$ per~cent.\ by volume of oxygen and $79.1$ per~cent.\ of
nitrogen.

\Section{21.} \Topic{Characteristic Equation of Other Substances.}---The
characteristic equation of perfect gases, even in the
case of the substances hitherto discussed, is only an approximation,
though a close one, to the actual facts. A still
further deviation from the behaviour of perfect gases is
shown by the other gaseous bodies, especially by those easily
condensed, which for this reason were formerly classed as
\emph{vapours}. For these a modification in the form of the
characteristic equation is necessary. It is worthy of notice,
however, that the more rarefied the state in which we observe
these gases, the less does their behaviour deviate from that
of perfect gases, so that all gaseous substances, when sufficiently
rarefied, may be said in general to act like perfect
gases. The general characteristic equation of gases and
vapours, for very large values of~$v$, will pass over, therefore,
into the special form for perfect gases.

\Section{22.} We may obtain by various graphical methods an
idea of the character and magnitude of the deviations from
the ideal gaseous state. An isothermal curve may, \eg, be
drawn, taking $v$~and $p$ for some given temperature as the
abscissa and ordinate, respectively, of a point in a plane.
The entire system of isotherms gives us a complete representation
of the characteristic equation. The more the
behaviour of the vapour in question approaches that of a
perfect gas, the closer do the isotherms approach those of
equilateral hyperbolæ having the rectangular co-ordinate
\PageSep{13}
axes for asymptotes, for $pv = \const$ is the equation of an
isotherm of a perfect gas. The deviation from the hyperbolic
form yields at the same time a measure of the
departure from the ideal state.

\Section{23.} The deviations become still more apparent when
\index{Deviation from perfect gases}%
the isotherms are drawn taking the product~$pv$ (instead of~$p$)
as the ordinate and say $p$~as the abscissa. Here a perfect
gas has evidently for its isotherms straight lines parallel to
the axis of abscissæ. In the case of actual gases, however, the
isotherms slope gently towards a minimum value of~$pv$, the
position of which depends on the temperature and the nature
of the gas. For lower pressures (\ie\ to the left of the
minimum), the volume decreases at a more rapid rate, with
increasing pressure, than in the case of perfect gases; for
higher pressures (to the right of the minimum), at a slower
rate. At the minimum point the compressibility coincides
with that of a perfect gas. In the case of hydrogen the
minimum lies far to the left, and it has hitherto been
possible to observe it only at very low temperatures.

\Section{24.} To van~der Waals is due the first analytical formula
\index{Equation!Van der Waals'}%
\index{Van der Waals'!equation}%
for the general characteristic equation, applicable also to
the liquid state. He also explained physically, on the basis
of the kinetic theory of gases, the deviations from the
behaviour of perfect gases. As we do not wish to introduce
here the hypothesis of the kinetic theory, we consider van~der
Waals' equation merely as an approximate expression of
the facts. His equation is
\[
p = \frac{R\theta}{v - b} - \frac{a}{v^{2}},
\]
where $R$,~$a$, and~$b$ are constants which depend on the nature
of the substance. For large values of~$v$, the equation, as
required, passes into that of a perfect gas; for small values
of~$v$ and corresponding values of~$\theta$, it represents the characteristic
equation of a liquid.

Expressing $p$~in atmospheres and calling the specific
\PageSep{14}
volume~$v$ unity for $\theta = 273$ and $p = 1$, van~der Waals'
constants for carbon dioxide are
\index{Carbon dioxide!Van der Waals' constants for}%
\[
R = 0.00369;\quad
a = 0.00874;\quad
b = 0.0023.
\]

As the volume of $1~\Unit{gr.}$ of carbon dioxide at $0°$~C. and
atmospheric pressure is $505~\Chg{\Unit{c.c.}}{\Unit{cm.}^{3}}$, the values of~$v$ calculated
from the formula must be multiplied by~$505$ to
obtain the specific volumes in absolute units.

\Section{25.} Van~der Waals' equation not being sufficiently
\index{Equation!Clausius'}%
\index{Van der Waals'!constants for \ce{CO2}}%
accurate, Clausius supplemented it by the introduction of
an additional constant. Clausius' equation is
\index{Clausius' equation}%
\[
p = \frac{R\theta}{v - a} - \frac{c}{\theta(v + b)^{2}}.
\Tag{(12)}
\]
For large values of~$v$, this too approaches the ideal
characteristic equation. In the same units as above, Clausius'
constants for carbon dioxide are:
\[
R = 0.003688;\quad
a = 0.000843;\quad
b = 0.000977;\quad
c = 2.0935.
\]

Andrews' observations on the compressibility of gaseous
\index{Andrews}%
and liquid carbon dioxide are satisfactorily represented by
Clausius' equation.

\Section{26.} If we draw the system of isotherms with the aid of
Clausius' equation, employing the graphical method described
in~\SecRef{22}, the characteristic graphs for carbon dioxide---\Fig{1}---are
\index{Apt, Dr.\ Richard}%
obtained.\footnote
  {For the calculation and construction of the curves, I am indebted to
  Dr.\ Richard Apt.}
For high temperatures the isotherms
approach equilateral hyperbolæ, as may be seen from equation~\Eq{(12)}.
In general, however, the isotherm is a curve of the
third degree, three values of~$v$ corresponding to one of~$p$.
Hence, in general, a straight line parallel to the axis of
abscissæ intersects an isotherm in three points, of which
two, as actually happens for large values of~$\theta$, may be
imaginary. At high temperatures there is, consequently,
\PageSep{15}
% [** TN: ~10pt too tall in screen format]
\ifthenelse{\boolean{ForPrinting}}{%
  \Figure[\textwidth]{1}
}{%
  \Figure[3.5in]{1}
}
\index{Carbon dioxide!isotherms of}%
\index{Isotherms of \ce{CO2}}%
\PageSep{16}
only one real volume corresponding to a given pressure,
while at lower temperatures, there are three real values of
the volume for a given pressure. Of these three values
(indicated on the figure by $\alpha$,~$\beta$,~$\gamma$, for instance) only the
smallest~($\alpha$) and the largest~($\gamma$) represent practically realizable
states, for at the middle point~($\beta$) the pressure along
the isotherm would increase with increasing volume, and the
compressibility would accordingly be negative. Such a
state has, therefore, only a theoretical signification.

\Section{27.} The point~$\alpha$ corresponds to liquid carbon dioxide,
and $\gamma$~to the gaseous condition at the temperature of the
isotherm passing through the points and under the pressure
measured by the ordinates of the line~$\alpha\beta\gamma$. In general
only one of these states is stable (in the figure, the liquid
state at~$\alpha$). For, if we compress gaseous carbon dioxide,
enclosed in a cylinder with a movable piston, at constant
temperature, \eg\ at $20°$~C., the gas assumes at first states
corresponding to consecutive points on the $20°$~isotherm to
the extreme right. The point representative of the physical
state of the gas, then moves farther and farther to the left
until it reaches a certain place~$C$. After this, further compression
does not move the point beyond~$C$, but there now
takes place a partial condensation of the substance---a splitting
into a liquid and a gaseous portion. Both parts, of course,
possess common pressure and temperature. The state of the
gaseous portion continues to be characterized by the point~$C$,
that of the liquid portion by the point~$A$ of the same
isotherm. $C$~is called the saturation point of carbon dioxide
\index{Saturation point}%
gas for the particular temperature considered. Isothermal
compression beyond~$C$ merely results in precipitating more
of the vapour in liquid form. During this part of the
isothermal compression no change takes place but the condensation
of more and more vapour; the internal conditions
(pressure, temperature, specific volume) of both parts of the
substance are always represented by the two points $A$~and~$C$.
At last, when all the vapour has been condensed, the whole
substance is in the liquid condition~$A$, and again behaves
\PageSep{17}
as a homogeneous substance, so that further compression
gives an increase of density and pressure along the isotherm.
The substance will now pass through the point~$\alpha$ of the
\index{Point!of inflection}%
figure. On this side, as may be seen from the figure, the
isotherm is much steeper than on the other, \ie\ the compressibility
is much smaller. At times, it is possible to follow the
isotherm beyond the point~$C$ towards the point~$\gamma$, and to
prepare a so-called supersaturated vapour. Then only a
more or less unstable condition of equilibrium is obtained,
as may be seen from the fact that the smallest disturbance
of the equilibrium is sufficient to cause an immediate condensation.
The substance passes by a jump into the stable
condition. Nevertheless, by the study of supersaturated
vapours, the theoretical part of the curve also receives a
direct meaning.

\Section{28.} On any isotherm, which for certain values of~$p$
admits of three real values of~$v$, there are, therefore, two
definite points, $A$~and~$C$, corresponding to the state of
saturation. The position of these points is not immediately
deducible from the graph of the isotherm. The propositions
of thermodynamics, however, lead to a simple way of finding
these points, as will be seen in~\SecRef{172}. The higher the temperature,
\index{Temperature!critical}%
the smaller becomes the region in which lines
drawn parallel to the axis of abscissæ intersect the isotherm
in three real points, and the closer will these three points
approach one another. The transition to the hyperbola-like
isotherms, which any parallel to the axis of abscissæ cuts
in one point only, is formed by that particular isotherm
on which the three points of intersection coalesce into one,
giving a point of inflection. The tangent to the curve at
\index{Inflection!point of}%
this point is parallel to the axis of abscissæ. It is called
the \emph{critical point} ($K$~of \Fig{1}) of the substance, and its
\index{Critical point}%
\index{Critical point!pressure}%
\index{Critical point!specific volume}%
\index{Critical point!temperature}%
position indicates the critical temperature, the critical
specific volume, and the critical pressure of the substance.
Here there is no longer any difference between the saturated
vapour and its liquid precipitate. Above the critical temperature
and critical pressure, condensation does not exist,
\PageSep{18}
as the diagram plainly shows. Hence all attempts to
condense hydrogen, oxygen, and nitrogen necessarily failed
as long as the temperature had not been reduced below
the critical temperature, which is very low for these
gases.

\Section{29.} It further appears from our figure that there is
no definite boundary between the gaseous and liquid states,
since from the region of purely gaseous states, as at~$C$,
that of purely liquid ones, as at~$A$, may be reached on
a circuitous path that nowhere passes through a state of
saturation---on a curve, for instance, drawn around the critical
point. Thus a vapour may be heated at constant volume
above the critical temperature, then compressed at constant
temperature below the critical volume, and finally cooled
under constant pressure below the critical temperature.
Condensation nowhere occurs in this process, which leads,
nevertheless, to a region of purely liquid states. The
earlier fundamental distinction between liquids, vapours,
and gases should therefore be dropped as no longer tenable.
A more modern proposal to denote as gaseous all states
above the critical temperature, and as vaporous or liquid
all others according as they lie to the right or left of the
theoretical regions (\Fig{1}), has also this disadvantage, that
thereby a boundary is drawn between liquid and gas on
the one hand, and vapour and gas on the other hand, which
has no physical meaning. The crossing of the critical
temperature at a pressure other than the critical pressure
differs in no way from the crossing of any other temperature.

\Section{30.} The position of the critical point may be readily
calculated from the general characteristic equation. According
to~\SecRef{28} we have
\[
\left(\frac{\dd p}{\dd v}\right)_{\theta} = 0,\quad\text{and}\quad
\left(\frac{\dd^{2} p}{\dd v^{2}}\right)_{\theta} = 0.
\]
The first of these means that the tangent to the isotherm
at~$K$ is parallel to the axis of abscissæ; and the second, that
\PageSep{19}
the isotherm has a point of inflection at~$K$. On the basis
of Clausius' form of the characteristic equation~\Eq{(12)}, we
obtain for the critical point
\index{Critical point!of \ce{CO2}}%
\[
\theta^{2} = \frac{8c}{27(a + b)R},\quad
p^{2} = \frac{cR}{216(a + b)^{3}},\quad
v = 3a + 2b.\footnotemark
\]
\footnotetext{Obtained as follows:---
\begin{align*}
p &= \frac{R\theta}{v - a} - \frac{c}{\theta(v + b)^{2}}\Add{,}
\tag*{(1)} \\
\left(\frac{\dd p}{\dd v}\right)_{\theta}
  &= -\frac{R\theta}{(v - a)^{2}} + \frac{2c}{\theta(v + b)^{3}} = 0\Add{,}
\tag*{(2)} \\
\left(\frac{\dd^{2} p}{\dd v^{2}}\right)_{\theta}
  &= \frac{2R\theta}{(v - a)^{3}} - \frac{6c}{\theta(v + b)^{4}} = 0\Add{.}
\tag*{(3)}
\end{align*}
From (2) and~(3),
\begin{align*}
v &= 3a + 2b\Add{.}
\tag*{(4)}
\intertext{Substituting (4) in (2) and reducing, we get}
\theta^{2} &= \frac{8c}{27(a + b)R}\Add{.}
\tag*{(5)}
\intertext{And substituting (4) and (5) in (1) and reducing, we have}
p^{2} &= \frac{cR}{216(a + b)^{3}}\Add{.}
\tag*{(6)\ \Tr.}
\end{align*}}
These equations give for carbon dioxide from the above
data
\[
\theta = 304 = 273° + 31°,\quad
p = 77~\Unit{atm.},\quad
v = 2.27\, \frac{\Chg{\Unit{c.c.}}{\Unit{cm.}^{3}}}{\Unit{gr.}}.
\]
Qualitatively, all substances conform to these regularities,
but the values of the constants differ widely.

\Section{31.} Regarding the transition from the liquid to the
\index{Theoretical regions}%
solid state, the same considerations hold as for that from
the gaseous to the liquid state. The system of isotherms
might be drawn for this process, and it is probable that
\emph{theoretical} regions and a critical point would be verified
here also, if the means of experimental investigation were
adequate. A continuous passage from the liquid to the
solid state would then become possible along a path intersecting
the critical isotherm on either side of the critical
\PageSep{20}
point. In fact, there are certain substances which under
ordinary pressures pass without appreciable discontinuity
from the solid to the liquid state (pitch, glass, etc.), while
others possess for a definite temperature a definite pressure
\index{Liquefaction pressure}%
\index{Pressure!of liquefaction}%
of liquefaction or pressure of solidification, at which the
\index{Solidification!pressure}%
substance splits into two portions of different densities. The
pressure of liquefaction, however, varies with temperature
at a much greater rate than the pressure of the saturated
vapour. This view is physically justified, in particular by
the experiments of Barus and Spring, in which the pressures
\index{Barus}%
\index{Spring}%
were varied within wide limits.

In its most complete form the characteristic equation
would comprise the gaseous, liquid, and solid states simultaneously.
No formula of such generality, however, has as
yet been established for any substance.

\Section{32.} \Topic{Mixtures.}---While, as shown in~\SecRef{19}, the characteristic
\index{Dalton's law}%
\index{Laws:!Dalton's}%
\index{Mixtures}%
equation of a mixture of perfect gases reduces in
a simple manner to that of its components, no such simplification
takes place, in general, when substances of any
kind are mixed. Only for gases and vapours does Dalton's
law hold, at least with great approximation, that the total
pressure of a mixture is the sum of the partial pressures
which each gas would exert if it alone filled the total
volume at the given temperature. This law enables us
to establish the characteristic equation of any gas mixture,
provided that of the constituent gases be known. It also
decides the question, unanswered in~\SecRef{17}, whether to the
individual gases of a mixture common pressure and different
volumes, or common volume and different pressures, should
be ascribed. From the consideration of a vapour differing
widely from an ideal gas, it follows that the latter of these
views is the only one admissible. Take, for instance, atmospheric
air and water vapour at $0°$~C. under atmospheric
pressure. Here the water vapour cannot be supposed to be
subject to a pressure of $1~\Unit{atm.}$, since at $0°$~C. no water
vapour exists at this pressure. The only choice remaining
is to assign to the air and water vapour a common volume
\PageSep{21}
(that of the mixture) and different pressures (partial
pressures).

For mixtures of solid and liquid substances no law
of general validity has been found, that reduces the
characteristic equation of the mixture to those of its
constituents.
\PageSep{22}


\Chapter{II.}{Molecular Weight.}
\index{Molecular weight}%
\index{Weights molecular and equivalent}%

\Section{33.} \First{In} the preceding chapter only such physical changes
have been discussed as concern temperature, pressure, and
density. The chemical constitution of the substance or
mixture in question has been left untouched. Cases are
frequent, however (much more so, in fact, than was formerly
supposed) in which the chemical nature of a substance is
altered by a change of temperature or pressure. The more
recent development of thermodynamics has clearly brought
out the necessity of establishing a fundamental difference
between physical and chemical changes such as will exclude
continuous transition from the one kind to the other (\cf\
\SecRef{42}, \textit{et seq.}, and~\SecRef{238}). It has, however, as yet not been
possible to establish a practical criterion for distinguishing
them, applicable to all cases. However strikingly most
chemical processes differ from physical ones in their violence,
suddenness, development of heat, changes of colour and
other properties, yet there are, on the other hand, numerous
changes of a chemical nature that take place with continuity
and comparative slowness; for example, dissociation. One
of the main tasks of physical chemistry in the near future
will be the further elucidation of this essential difference.\footnote
  {In a word, we may, in a certain sense, say, that physical changes take
  place continuously, chemical ones, on the other hand, discontinuously. In
  consequence, the science of physics deals, primarily, with continuously varying
  numbers, the science of chemistry, on the contrary, with whole, or with
  simple rational numbers.}

\Section{34.} Experience shows that all chemical reactions take
place according to constant proportions by weight. A
\PageSep{23}
\index{Weights molecular and equivalent}%
certain weight (strictly speaking, a mass) may therefore
be used as a characteristic expression for the nature of
a given chemically homogeneous substance, whether an
element or a compound. Such a weight is called an \emph{equivalent
weight}. It is arbitrarily fixed for one element---generally
for hydrogen at $1~\Unit{gr.}$---and then the equivalent
weight of any other element (\eg~oxygen) is that weight
which will combine with $1~\Unit{gr.}$ of hydrogen. The weight of
the compound thus formed is, at the same time, its equivalent
weight. By proceeding in this way, the equivalent
weights of all chemically homogeneous substances may be
found. The equivalent weights of elements that do not
combine directly with hydrogen can easily be determined,
since in every case a number of elements can be found that
combine directly with the element in question and also
with hydrogen.

The total weight of a body divided by its equivalent
\index{Equivalents, number of}%
weight is called the \emph{number of equivalents} contained in the
body. Hence we may say that, in every chemical reaction,
an equal number of equivalents of the different substances
react with one another.

\Section{35.} There is, however, some ambiguity in the above
definition, since two elements frequently combine in more
ways than one. For such cases there would exist several
values of the equivalent weight. Experience shows, however,
\index{Equivalent weight}%
that the various possible values are always simple
multiples or submultiples of any one of them. The
ambiguity in the equivalent weight, therefore, reduces itself
to multiplying or dividing that quantity by a simple integer.
We must accordingly generalize the foregoing statement,
that an equal number of equivalents react with one another,
and say, that the number of equivalents that react with one
another are in simple numerical proportions. Thus $16$~parts
by weight of oxygen combine with $28$~parts by weight of
nitrogen to form nitrous oxide, or with $14$~parts to form
\index{Nitrogen!oxides}%
\index{Oxides of nitrogen}%
nitric oxide, or with $9\frac{1}{3}$~parts to form nitrous anhydride, or
with $7$~parts to form nitrogen tetroxide, or with $5\frac{3}{5}$~parts to
\PageSep{24}
form nitric anhydride. Any one of these numbers may be
assigned to nitrogen as its equivalent weight, if $16$~be taken
as that of oxygen. They are in simple rational proportions,
since
\[
28 : 14 : 9\tfrac{1}{3} : 7 : 5\tfrac{3}{5} = 60 : 30 : 20 : 15 : 12.
\]

\Section{36.} The ambiguity in the definition of the equivalent
weight of nitrogen, exemplified by the above series of
numbers, is removed by selecting a particular one of them
to denote the \emph{molecular weight} of nitrogen. In the definition
of the molecular weight as a quite definite quantity depending
only on the particular state of a substance, and
independent of possible chemical reactions with other substances,
lies one of the most important and most fruitful
achievements of theoretical chemistry. Its exact statement
can at present be given only for special cases, viz.\ for
perfect gases and dilute solutions. We need consider only
the former of these, as we shall see from thermodynamics
that the latter is also thereby determined.

The definition of the molecular weight for a chemically
homogeneous perfect gas is rendered possible by the further
empirical law, that gases combine, not only in simple
multiples of their equivalents, but also, at the same temperature
and pressure, in simple volume proportions (Gay-Lussac).
\index{Gay-Lussac}%
\index{Laws:!Gay-Lussac's}%
It immediately follows that the number of equivalents, contained
in equal volumes of different gases, must bear simple
ratios to one another. The values of these ratios, however,
are subject to the above-mentioned ambiguity in the
selection of the equivalent weight. The ambiguity is,
however, removed by putting all these ratios $= 1$, \ie\ by
establishing the condition that equal volumes of different
gases shall contain an equal number of equivalents. Thus
a definite choice is made from the different possible values,
and a definite equivalent weight obtained for the gas, which
is henceforth denoted as the \emph{molecular weight} of the gas. At
the same time the number of equivalents in a quantity of
the gas, which may be found by dividing the total weight
by the molecular weight, is defined as the \emph{number of
\PageSep{25}
molecules} contained in that quantity. Hence, \emph{equal volumes
of perfect gases at the same temperature and pressure contain
an equal number of molecules} (Avogadro's law). The molecular
\index{Avogadro's law}%
\index{Laws:!Avogadro's}%
\index{Molecules, number of}%
weights of chemically homogeneous gases are, therefore,
directly proportional to the masses contained in equal
volumes, \ie\ to the densities. The ratio of the densities is
equal to the ratio of the molecular weights.

\Section{37.} Putting the molecular weight of hydrogen $= m_{0}$,
that of any other chemically homogeneous gas must be
equal to $m_{0}$~multiplied by its specific density relative to
hydrogen (\SecRef{11}). The following table gives the specific
densities relative to hydrogen, and the molecular weights of
several gases:---
\begin{center}
\TableFont
\begin{tabular}{lr<{\qquad} r<{\qquad}}
                 & \ColHead{Specific Density.} & \ColHead{Molecular Weight.} \\
Hydrogen & $1.0$  & $m_{0}$ \\
Oxygen & $16.0$ & $16.0\, m_{0}$ \\
Nitrogen & $14.0$ & $14.0\, m_{0}$ \\
Water vapour & $9.0$ & $9.0\, m_{0}$ \\
Ammonia & $8.5$ & $8.5\, m_{0}$ \\
\end{tabular}
\end{center}

Now, since water vapour consists of $1$~part by weight of
hydrogen and $8$~parts by weight of oxygen, the molecule
of water vapour, $9\, m_{0}$, must consist of $m_{0}$~parts by weight of
hydrogen and $8\, m_{0}$~parts by weight of oxygen---\ie, according
to the above table, of one molecule of hydrogen and half a
molecule of oxygen. In the same manner ammonia, according
to analysis, consisting of $1$~part by weight of hydrogen
and $4\frac{2}{3}$~parts by weight of nitrogen, its molecule $8.5\, m_{0}$ must
necessarily contain $1.5\, m_{0}$~parts by weight of hydrogen and
$7\, m_{0}$~parts by weight of nitrogen---\ie, according to the
table, $1\frac{1}{2}$~molecules of hydrogen and $\frac{1}{2}$~molecule of nitrogen.
Thus Avogadro's law enables us to give in quite definite
numbers the molecular quantities of each constituent present
in the molecule of any chemically homogeneous gas, provided
we know its density and its chemical composition.

\Section{38.} The smallest weight of a chemical element entering
\index{Atom, definition of}%
into the molecules of its compounds is called an \emph{atom}.
\PageSep{26}
Hence half a molecule of hydrogen is called an atom of
hydrogen,~\ce{H}; similarly, half a molecule of oxygen an atom
of oxygen,~\ce{O}; and half a molecule of nitrogen an atom
of nitrogen,~\ce{N}. The diatomic molecules of these substances
are represented by \ce{H2}, \ce{O2},~\ce{N2}. An atom of mercury, on the
contrary, is equal to a whole molecule, because in the molecules
of its compounds no fractions of the molecular weight
of mercury vapour \Erratum{occurs}{occur}. It is usual to put the atomic
weight of hydrogen $\ce{H} = 1$. Then its molecular weight
becomes $\ce{H2} = m_{0} = 2$, and the molecular weights of our
table become:\Add{---}
\begin{center}
\TableFont
\settowidth{\TmpLen}{Molecular}
\begin{tabular}{l>{\qquad}l}
        & \ColHead{Molecular Weight.} \\
Hydrogen & $\Z2 = \ce{H2}$ \\
Oxygen & $32 = \ce{O2}$ \\
Nitrogen & $28 = \ce{N2}$ \\
Water vapour & $18 = \ce{H2O}$ \\
Ammonia & $17 = \ce{H3N}$ \\
\end{tabular}
\end{center}

\Section{39.} In general, then, the molecular weight of a
chemically homogeneous gas is twice its density relative to
hydrogen. Conversely, the molecular weight,~$m$, of a gas
being known, its specific density, and consequently the
constant~$C$ in the characteristic equation~\Eq{(5)}, can be calculated.
Denoting all quantities referring to hydrogen by
the suffix~$0$, we have, at any temperature and pressure, for
hydrogen,
\[
p = \frac{C_{0} \theta}{v_{0}},
\]
for any other gas at the same temperature and pressure,
\begin{align*}
p &= \frac{C\theta}{v}, \\
\therefore
C : C_{0} &= \frac{1}{v_{0}} : \frac{1}{v} = m_{0} : m,
\intertext{or}
C &= \frac{m_{0}C_{0}}{m}\Add{.}
\Tag{(13)}
\end{align*}
\PageSep{27}
Now $m_{0} = 2$, and $C_{0}$~is to be calculated from the density of
hydrogen at $0°$~C. and atmospheric pressure (\SecRef{11}).

Since
\begin{gather*}
\frac{1}{v_{0}} = 0.00008988,\quad
p = 1013650,\quad
\theta = 273, \\
\therefore
C = \frac{m_{0}C_{0}}{m}
  = \frac{m_{0}}{m} · \frac{pv_{0}}{\theta}
  = \frac{2 · 1013650}{m · 273 · 0.00008988}
  = \frac{82600000}{m}\Add{.}
\end{gather*}

Putting, for shortness, $82600000 = R$, the characteristic
equation of a chemically homogeneous perfect gas of molecular
\index{Gas!constant}%
\index{Gas!volume}%
weight~$m$ becomes
\[
p = \frac{R}{m} · \frac{\theta}{v},
\Tag{(14)}
\]
where $R$, being independent of the nature of the individual
gas, is generally called the absolute gas constant. The
molecular weight may be deduced directly from the characteristic
equation by the aid of the constant~$R$, since
\[
m = \frac{R}{C}\Add{.}
\Tag{(15)}
\]

Since $v = \dfrac{V}{M}$, we have
\[
V = \frac{R\theta}{p} · \frac{M}{m}.
\]
But $\dfrac{M}{m}$~is the quantity defined above as the number of
molecules in the gas, and, therefore, if $\dfrac{M}{m} = n$,
\[
V = \frac{R\theta}{p} · n,
\]
which means that at a given temperature and pressure the
volume of a quantity of gas depends only on the number of
the molecules present, and not at all on the nature of the
gas.

\Section{40.} In a mixture of chemically homogeneous gases of
\index{Avogadro's law}%
\PageSep{28}
\index{Gas mixture!volume}%
molecular weights $m_{1}$,~$m_{2}$,~\dots the relation between the
partial pressures is, according to~\Eq{(9)},
\[
p_{1} : p_{2} : \dots = C_{1}M_{1} : C_{2}M_{2} \Add{:} \dots\Add{.}
\]
But in~\Eq{(15)} we have
\begin{gather*}
C_{1} = \frac{R}{m_{1}};\quad
C_{2} = \frac{R}{m_{2}};\quad \dots\Add{,} \\
\therefore
p_{1} : p_{2} : \dots = \frac{M_{1}}{m_{1}} : \frac{M_{2}}{m_{2}} : \dots
  = n_{1} : n_{2} : \dots\Add{,}
\end{gather*}
\ie\ the ratio of the partial pressures is also the ratio of the
number of molecules of each gas present. Equation~\Eq{(10)}
gives for the total volume
\begin{align*}
V &= \frac{(C_{1}M_{1} + C_{2}M_{2} + \dots)\theta}{p} \\
  &= \frac{R\theta}{p} \left(\frac{M_{1}}{m_{1}} + \frac{M_{2}}{m_{2}} + \dots\right) \\
  &= \frac{R\theta}{p} (n_{1} + n_{2} + \dots) \\
  &= \frac{R\theta}{p}\, n\Add{.}
\Tag{(16)}
\end{align*}
The volume of the mixture is therefore determined by
the total number of the molecules present, just as in the
case of a chemically homogeneous gas.

\Section{41.} It is evident that we cannot speak of the molecular
weight of a mixture. Its \emph{apparent} molecular weight, however,
\index{Molecular weight!apparent}%
may be defined as the molecular weight which a
chemically homogeneous gas would have if it contained in
the same mass the same number of molecules as the
mixture. If we denote the apparent molecular weight by~$m$,
we have
\[
\frac{M_{1} + M_{2} + \dots}{m}
  = \frac{M_{1}}{m_{1}} + \frac{M_{2}}{m_{2}} + \dots
\]
and
\[
m = \frac{M_{1} + M_{2} + \dots}{\dfrac{M_{1}}{m_{1}} + \dfrac{M_{2}}{m_{2}} + \dots}\Add{.}
\]
\PageSep{29}

The apparent molecular weight of air may thus be calculated.
Since
\[
m_{1} = \ce{O2} = 32;\quad
m_{2} = \ce{N2} = 28;\quad
M_{1} : M_{2} = 0.3
\]
we have
\[
m = \frac{0.3 + 1}{\dfrac{0.3}{32} + \dfrac{1}{28}} = 28.8,
\]
which is somewhat larger than the molecular weight of
nitrogen.

\Section{42.} The characteristic equation of a perfect gas, whether
\index{Membranes, semipermeable}%
\index{Semipermeable membranes}%
chemically homogeneous or not, gives, according to~\Eq{(16)}, the
total number of molecules, but yields no means of deciding
whether or not these molecules are all of the same kind. In
order to answer this question, other methods must be resorted
to, none of which, however, is practically applicable to all
cases. A decision is often reached by an observation of the
process of diffusion through a porous or, better, a semipermeable
membrane. The individual gases of a mixture
will separate from each other by virtue of the differences in
their velocities of diffusion, which may even sink to zero in
the case of \Chg{semi-permeable}{semipermeable} membranes, and thus disclose
the inhomogeneity of the substance. The chemical constitution
of a gas may often be inferred from the manner in
which it originated. It is by means of the expression for
the entropy (\SecRef{237}) that we first arrive at a fundamental
definition for a chemically homogeneous gas.

\Section{43.} Should a gas or vapour not obey the laws of perfect
gases, or, in other words, should its specific density depend
on the temperature or the pressure, Avogadro's definition of
molecular weight is nevertheless applicable. The number
of molecules in this case, instead of being a constant, will
be dependent upon the momentary physical condition of the
substance. We may, in such cases, either assume the number
of molecules to be variable, or refrain from applying Avogadro's
definition of the number of molecules. In other words,
the cause for the deviation from the ideal state may be
\PageSep{30}
sought for either in the chemical or physical conditions. The
latter view preserves the chemical nature of the gas. The
molecules remain intact under changes of temperature and
pressure, but the characteristic equation is more complicated
than that of Boyle and Gay-Lussac like that, for example,
of van~der Waals or of Clausius. The other view differs
essentially from this, in that it represents any gas, not obeying
the laws of perfect gases, as a mixture of various kinds
of molecules (in nitrogen peroxide \ce{N2O4} and~\ce{NO2}, in phosphorus
\index{Nitrogen!peroxide}%
pentachloride \ce{PCl5}, \ce{PCl3}, and~\ce{Cl2}). The volume of
\index{Phosphorus pentachloride}%
these is supposed to have at every moment the exact value
theoretically required for the total number of molecules of
the mixture of these gases. The volume, however, does not
vary with temperature and pressure in the same way as that
of a perfect gas, because chemical reactions take place
between the different kinds of molecules, continuously altering
the number of each kind present, and thereby also the
total number of molecules in the mixture. This hypothesis
has proved fruitful in cases of great differences of density---so-called
\index{Density!abnormal vapour}%
abnormal vapour densities---especially where, beyond
\index{Abnormal vapour densities}%
\index{Vapour densities, abnormal}%
a certain range of temperature or pressure, the specific
density once more becomes constant. When this is the
case, the chemical reaction has been completed, and for this
reason the molecules henceforth remain unchanged. \Erratum{Hydrobromamylene}{Amylene hydrobromide},
\Erratum{\index{Hydrobromamylene}}{\index{Amylene hydrobromide}}%
for instance, acts like a perfect gas below
$160°$ and above $360°$, but shows only half its former density
at the latter temperature. The doubling of the number of
molecules corresponds to the equation
\[
\ce{C5H11Br} = \ce{C5H10 + HBr}.
\]
Mere insignificant deviations from the laws of perfect
gases are generally attributed to physical causes---as, \eg, in
water vapour and carbon dioxide---and are regarded as the
forerunners of condensation. The separation of chemical
from physical actions by a principle which would lead to a
more perfect definition of molecular weight for variable
vapour densities, cannot be accomplished at the present
time. The increase in the specific density which many
\PageSep{31}
vapours exhibit near their point of condensation might
just as well be attributed to such chemical phenomena as
the formation of double or multiple molecules. In fact,
differences of opinion exist in a number of such cases. The
molecular weight of sulphur vapour below $800°$, for instance,
\index{Sulphur}%
is generally assumed to be $\ce{S6} = 192$; but some assume a
mixture of molecules $\ce{S8} = 256$ and $\ce{S2} = 64$, and others still
different mixtures. In doubtful cases it is safest, in general,
to leave this question open, and to admit both chemical
and physical changes as causes for the deviations from the
laws of perfect gases. This much, however, may be affirmed,
that for small densities the physical influences will be of far
less moment than the chemical ones, for, according to
experience, all gases approach the ideal condition as their
densities decrease (\SecRef{21}). This is an important point, which
we will make use of later.
\PageSep{32}


\Chapter{III.}{Quantity of Heat.}
\index{Heat!quantity}%
\index{Heat!unit}%
\index{Quantity of heat}%
\index{Unit of heat}%

\Section{44.} If we plunge a piece of iron and a piece of lead, both
of equal weight and at the same temperature ($100°$~C.), into
two precisely similar vessels containing equal quantities of
water at $0°$~C., we find that, after thermal equilibrium has
been established in each case, the vessel containing the iron
has increased in temperature much more than that containing
the lead. Conversely, a quantity of water at~$100°$ is
cooled to a much lower temperature by a piece of iron at~$0°$,
than by an equal weight of lead at the same temperature.
This phenomenon leads to a distinction between \emph{temperature}
and \emph{quantity of heat}. As a measure of the heat given out
or received by a body, we take the increase or decrease of
temperature which some \emph{normal} substance (\eg\ water) undergoes
when it alone is in contact with the body, provided all
other causes of change of temperature (as compression, etc.)\
are excluded. The quantity of heat given out by the body
is assumed to be equal to that received by the normal substance,
and \textit{vice versâ}. The experiment described above
proves, then, that a piece of iron in cooling through a given
interval of temperature gives out more heat than an equal
weight of lead (about four times as much), and conversely,
that, in order to bring about a certain increase of temperature,
iron requires a correspondingly larger supply of heat
than lead.

\Section{45.} It was, in general, customary to take as the unit of
heat that quantity which must be added to $1~\Unit{gr.}$ of
water to raise its temperature from $0°$~C. to $1°$~C. (zero
\PageSep{33}
calorie). This is almost equal to the quantity of heat which
\index{Calorie!laboratory}%
\index{Calorie!large}%
\index{Calorie!mean}%
\index{Calorie!small}%
\index{Calorie!zero}%
\index{Heat!capacity}%
\index{Heat!specific (definition)}%
will raise $1\Unit{gr.}$ of water $1°$~C. at any temperature. The
refinement of calorimetric measurements has since made it
necessary to take account of the initial temperature of the
water, and it is often found convenient to define the calorie
as that quantity of heat which will raise $1~\Unit{gr.}$ of water
of mean laboratory temperature ($15°$~to~$20°$) $1$~degree of
the Centigrade scale. This laboratory calorie is about
$\dfrac{1}{1.006}$~of a zero calorie. Finally, a \emph{mean calorie} has been
\index{Zero!calorie}%
introduced, namely, the hundredth part of the heat required
to raise $1~\Unit{gr.}$ of water from $0°$~C. to $100°$~C\@. The mean
calorie is about equal to the zero calorie. Besides these
so-called \emph{small} calories, there are a corresponding number
of \emph{large} or kilogram calories, which contain $1000$~small
calories.

\Section{46.} The ratio of~$Q$, the quantity of heat each gram
of a substance receives, to~$\Delta\theta$, the corresponding increase of
temperature, is called the \emph{mean specific heat}, or \emph{mean heat
\index{Specific heat}%
capacity} of $1~\Unit{gr.}$ of the substance between the initial and
final temperatures of the process---
\[
\frac{Q}{\Delta\theta} = c_{m}.
\]
Hence, the mean heat capacity of water between $0°$~and
$1°$ is equal to one zero calorie.

Passing to infinitely small differences of temperature,
the specific heat of a substance, at the temperature~$\theta$,
becomes
\[
\frac{Q}{d\theta} = c.
\]
This, in general, varies with temperature, but very slowly
for most substances. It is usually permissible to put the
specific heat at a certain temperature equal to the mean
specific heat of an adjoining interval of moderate size.

\Section{47.} The heat capacity of solids and liquids is very
\PageSep{34}
nearly independent of any variations of external pressure
that may take place during the process of heating. Hence
the definition of the heat capacity is not, usually, encumbered
\index{Atomic heat}%
\index{Heat!atomic}%
\index{Heat!molecular}%
\index{Molecular heat}%
with a condition regarding pressure. The specific heat of
gases, however, is influenced considerably by the conditions
of the heating process. In this case the definition of
specific heat would, therefore, be incomplete without some
statement as to the accompanying conditions. Nevertheless,
we speak of the specific heat of a gas, without further
specification, when we mean its specific heat at constant
(atmospheric) pressure, as this is the value most readily
determined.

\Section{48.} That the heat capacities of different substances
should be referred to unit mass is quite arbitrary. It arises
from the fact that quantities of matter can be most easily
compared by weighing them. Heat capacity might, quite
as well, be referred to unit volume. It is more rational to
compare masses which are proportional to the molecular
and atomic weights of substances, for then certain regularities
at once become manifest. The corresponding heat
capacities are obtained by multiplying the specific heats
(per unit mass) by the molecular or atomic weights. The
values thus obtained are known as the \emph{molecular} or \emph{atomic
heats}.

\Section{49.} The chemical elements, especially those of high
\index{Laws:!Dulong and Petit's}%
atomic weight, are found to have nearly the constant atomic
heat of~$6.4$ (Dulong and Petit). It cannot be claimed that
this law is rigorously true, since the heat capacity depends
on the molecular constitution, as in the case of carbon, and
on the state of aggregation, as in the case of mercury, as
well as on the temperature. The effect of temperature is
especially marked in the elements, carbon, boron, and
silicon, which show the largest deviations from Dulong
and Petit's law. The conclusion is, however, justified, that
Dulong and Petit's law is founded on some more general
\index{Dulong and Petit's law}%
law of nature, which has not yet been formulated.
\PageSep{35}

\Section{50.} Similar regularities, as appear in the atomic heats of
\index{Laws:!Neumann's (Regnault)}%
\index{Neumann, F.}%
elements, are also found in the molecular heats of compounds,
especially with compounds of similar chemical constitution.
According to F.~Neumann's law, subsequently confirmed by
Regnault, compounds of similar constitution, when solid,
have equal molecular heats. Joule and Woestyn further
extended this law by showing that the molecular heat is
merely the sum of the atomic heats, or that in any combination
every element preserves its atomic heat, whether
or not the latter be~$6.4$, according to Dulong and Petit's
law. This relation also is only approximately true.

\Section{51.} Since all calorimetric measurements, according to
\SecRef{44}, extend only to quantities of heat imparted to bodies or
given out by them, they do not lead to any conclusion as
to the total amount of heat \emph{contained in} a body of given
temperature. It would be absurd to define the heat contained
in a body of given temperature, density, etc., as the
number of calories absorbed by the body in its passage from
some normal state into its present state, for the quantity
thus defined would assume different values according to the
way in which the change was effected. A gas at~$0°$ and
atmospheric pressure can be brought to a state where its
temperature is~$100°$ and its pressure $10$~atmospheres, either
by heating to~$100°$ under constant pressure, and then compressing
at constant temperature; or by compressing
isothermally to $10$~atmospheres, and then heating isopiestically
to~$100°$; or, finally, by compressing and heating
simultaneously or alternately in a variety of ways. The
total number of calories absorbed would in each case be
different (\SecRef{77}). It is seen, then, that it is useless to speak
of a certain quantity of heat which must be applied to a
body in a given state to bring it to some other state. If
the ``total heat contained in a body'' is to be expressed
numerically, as is done in the kinetic theory of heat, where
the heat of a body is defined as the total energy of its
internal motions, it must not be interpreted as the sum-total
of the quantities of heat applied to the body. As we
\PageSep{36}
shall make no use of this quantity in our present work, no
definition of it need be attempted.

\Section{52.} In contrast to the above representation of the facts,
the older (Carnot's) theory of heat, which started from the
\index{Carnot's!theory}%
\index{Heat!total}%
hypothesis that heat is an indestructible substance, necessarily
reached the conclusion that the ``heat contained in a
body'' depends solely on the number of calories absorbed
or given out by it. The heating of a body by other means
than direct application of heat, by compression or by friction
for instance, according to that theory produces no change in
the ``total heat.'' To explain the rise of temperature which
takes place notwithstanding, it was necessary to make the
assumption that compression and friction so diminish the
body's heat capacity, that the same amount of heat now
produces a higher temperature, just as, for example, a
moist sponge appears more moist if compressed, although
the quantity of liquid in the sponge remains the same. In
the meantime, Rumford and Davy proved by direct experiment
\index{Davy}%
\index{Rumford}%
that bodies, in which any amount of heat can be
generated by an adequate expenditure of work, do not in
the least alter their heat capacities with friction. Regnault,
likewise, showed, by accurate measurements, that the heat
capacity of gases is independent of or only very slightly
dependent on volume; that it cannot, therefore, diminish,
in consequence of compression, as much as Carnot's theory
would require. Finally, W.~Thomson and Joule have
\index{Joule}%
\index{Thomson}%
demonstrated by careful experiments that a gas, when expanding
without overcoming external pressure, undergoes
no change of temperature, or an exceedingly small one
(\cf~\SecRef{70}), so that the cooling of gases generally observed
when they expand is not due to the increase of volume \textit{per~se},
but to the work done in the expansion. Each one of
these experimental results would by itself be sufficient to
disprove the hypothesis of the indestructibility of heat, and
to overthrow the older theory.

\Section{53.} While, in general, the heat capacity varies continuously
with temperature, every substance possesses,
\PageSep{37}
\index{Heat effect}%
\index{Singular values}%
under certain external pressures, so-called \emph{singular} values
of temperature, for which the heat capacity, together
\index{Heat!of fusion}%
\index{Heat!of sublimation}%
\index{Heat!of vaporization}%
with other properties, is discontinuous. At such temperatures
the heat absorbed no longer affects the entire body,
but only one of the parts into which it has split; and it no
longer serves to increase the temperature, but simply to
alter the state of aggregation, \ie\ to melt, evaporate, or
sublime. Only when the entire substance has again become
homogeneous will the heat imparted produce a rise in
temperature, and then the heat capacity becomes once more
capable of definition. The quantity of heat necessary to
change $1$~gram of a substance from one state of aggregation
to another is called the \emph{latent heat}, in particular, the \emph{heat of
\index{Latent heat}%
fusion}, \emph{of vaporization}, or \emph{of sublimation}. The same amount
of heat is set free when the substance returns to its former
state of aggregation. Latent heat, as in the case of specific
heat, is best referred, not to unit mass, but to molecular or
atomic weight. Its amount largely depends on the external
conditions under which the process is carried out (\SecRef{47}),
\index{Endothermal process}%
\index{Exothermal process}%
\index{Process!endothermal}%
\index{Process!exothermal}%
constant pressure being the most important condition.

\Section{54.} Like the changes of the state of aggregation, all
processes involving mixture, or solution, and all chemical
reactions are accompanied by an evolution of heat of greater
or less amount, which varies according to the external conditions.
This we shall henceforth designate as the \emph{heat
effect} (Wärmetonung) of the process under consideration, in
particular as the heat of mixture, of solution, of combination,
of dissociation, etc. It is reckoned \emph{positive} when heat
is set free or developed, \ie\ given out by the body (exothermal
processes); \emph{negative}, when heat is absorbed, or
rendered latent, \ie\ taken up by the body (endothermal
processes).
\PageSep{38}


\Part{II.}{The First Fundamental Principle of Thermodynamics.}

\Chapter{I.}{General Exposition.}
\index{First law of thermodynamics}%

\Section{55.} \First{The} \emph{first law} of thermodynamics is nothing more
than the principle of the conservation of energy applied
\index{Conservation of energy}%
\index{Energy!conservation of}%
to phenomena involving the production or absorption of
heat. Two ways lead to a deductive proof of this principle.
We may take for granted the correctness of the mechanical
view of nature, and assume that all changes in nature can
be reduced to motions of material points between which
there act forces which have a potential. Then the principle
of energy is simply the well-known mechanical theorem of
kinetic energy, generalized to include all natural processes.
Or we may, as is done in this work, leave open the question
concerning the possibility of reducing all natural processes
to those of motion, and start from the fact which has been
tested by centuries of human experience, and repeatedly
verified, viz.\ that \emph{it is in no way possible, either by mechanical,
thermal, chemical, or other devices, to obtain perpetual motion},
\ie\ it is impossible to construct an engine which will work
in a cycle and produce continuous work, or kinetic energy,
from nothing. We shall not attempt to show how this single
fact of experience, quite independent of the mechanical
view of nature, serves to prove the principle of energy in
its generality, mainly for the reason that the validity of
the energy principle is nowadays no longer disputed. It
\PageSep{39}
will be different, however, in the case of the \emph{second law} of
thermodynamics, the proof of which, at the present stage
of the development of our subject, cannot be too carefully
presented. The general validity of this law is still contested
from time to time, and its significance variously
interpreted, even by the adherents of the principle.

\Section{56.} \Topic{The energy} of a body, or system of bodies, is
\index{Energy!definition of}%
\index{External conditions of equilibrium!effect}%
a magnitude depending on the momentary condition of
the system. In order to arrive at a definite numerical
expression for the energy of the system in a given state,
it is necessary to fix upon a certain \emph{normal} arbitrarily
selected state (\eg\ $0°$~C. and atmospheric pressure). The
energy of the system in a given state, referred to the
arbitrarily selected normal state, is then equal to \emph{the algebraic
sum of the mechanical equivalents of all the effects
produced outside the system when it passes in any way from
the given to the normal state}. The energy of a system is,
therefore, sometimes briefly denoted as the faculty to
produce external effects. Whether or not the energy of a
system assumes different values according as the transition
from the given to the normal state is accomplished in
different ways is not implied in the above definition. It
will be necessary, however, for the sake of completeness,
to explain the term ``mechanical equivalent of an external
effect.''

\Section{57.} Should the external effect be mechanical in nature---should
it consist, \eg, in lifting a weight, overcoming
atmospheric pressure, or producing kinetic energy---then
its mechanical equivalent is simply equal to the mechanical
work done by the system on the external body (weight,
atmosphere, projectile). It is positive if the displacement
take place in the direction of the force exercised by the
system---when the weight is lifted, the atmosphere pushed
back, the projectile discharged,---negative in the opposite
sense.

But if the external effect be thermal in nature---if it
consist, \eg, in heating surrounding bodies (the atmosphere,
\PageSep{40}
a calorimetric liquid, etc.)---then its mechanical equivalent is
\index{Mechanical equivalent!of heat}%
equal to the number of calories which will produce the same
rise of temperature in the surrounding bodies multiplied
by an absolute constant, which depends only on the units
of heat and mechanical work, the so-called \emph{mechanical
equivalent of heat}. This proposition, which appears here
only as a definition, receives through the principle of the
conservation of energy a physical meaning, which may be
\index{Conservation of energy}%
\index{Energy!conservation of}%
put to experimental test.

\Section{58.} \Topic{The Principle of the Conservation of Energy}
asserts, generally and exclusively, that the energy of a
system in a given state, referred to a fixed normal state,
has a quite \Erratum{different}{definite} value; in other words---substituting
the definition given in~\SecRef{56}---that the algebraic sum of the
mechanical equivalents of the external effects produced
outside the system, when it passes from the given to the
normal state, is independent of the manner of the transformation.
On passing into the normal state the system
thus produces a definite total of effects, as measured in
mechanical units, and it is this sum---the ``work-value''
of the external effects---that represents the energy of the
system in the given state.

\Section{59.} The validity of the principle of the conservation
of energy may be experimentally verified by transferring a
system in various ways from a given state to a certain other
state, which may here be designated as the normal state,
and measuring the mechanical equivalents of all external
effects in each case. Special care must be taken, however,
that the initial state of the system is the same each time,
and that none of the external effects is overlooked or taken
into account more than once.

\Section{60.} As a first application we shall discuss Joule's famous
\index{Joule's experiments|(}%
experiments, in which the external effects produced by
weights falling from a certain height were compared, first,
when performing only mechanical work (\eg\ lifting a load),
and second, when by suitable contrivances generating heat
\PageSep{41}
by friction. The initial and final position of the weights
may be taken as the two states of the system, the work or
heat produced, as the external effects. The first case,
where the weights produce only mechanical work, is simple,
and requires no experiment. Its mechanical equivalent is
the product of the sum of the weights, and the height
through which they fall. The second case requires accurate
measurement of the increase of temperature, which the
surrounding substances (water, mercury) undergo in consequence
of the friction, as well as of their heat capacities, for
the determination of the number of calories which will
produce in them the same rise of temperature. It is, of
course, entirely immaterial what our views may be with
regard to the details of the frictional generation of heat, or
with regard to the ultimate form of the heat thus generated.
The only point of importance is that the state produced
in the liquid by friction is identical with a state produced
by the absorption of a definite number of calories.

Joule, by equating the mechanical work, corresponding
\index{Gram-calorie, mechanical equivalent of}%
\index{Mechanical equivalent!of a gram-calorie}%
to the fall of the weights, to the mechanical equivalent of
the heat produced by friction, showed that the mechanical
equivalent of a gram-calorie is, under all circumstances,
equal to the work done in lifting a weight of a gram through
a height of $423.55$~meters. That all his experiments with
different weights, different calorimetric substances, and
different temperatures, led to the same value, goes to prove
the correctness of the principle of the conservation of
energy.

\Section{61.} In order to determine the mechanical equivalent
of heat in absolute units, we must bear in mind that Joule's
result refers to laboratory calories (\SecRef{45}), and the readings
of a mercury thermometer. At the temperature of the
laboratory, $1°$~of the mercury thermometer represents about
$\dfrac{1}{1.007}$~of $1°$~of the gas thermometer. A calorie referred to
the gas thermometer has, therefore, a mechanical equivalent
of $423.55 × 1.007 = 427$.
\PageSep{42}

The acceleration of gravity must also be considered,
since raising a gram to a certain height represents, in
general, different amounts of work in different latitudes.
The absolute value of the work done is obtained by multiplying
the weight, \ie\ the product of the mass and the
acceleration of gravity, by the height of fall. The following
table gives the mechanical equivalent of heat in the
\index{Mechanical equivalent!of heat!in absolute units}%
different calories:---
\begin{center}
\TableFont
\begin{tabular}{l|c|c}
\hline
%[** TN: Re-breaking column headings]
\ColumnHeading{Unit of heat referred}{Unit of heat referred \\
to gas thermometer.}
&
\ColumnHeading{Corresponding height in}{Corresponding height in \\
meters to which $1~\Unit{gr.}$ \\
must be raised in places
of mean latitude.}
&
\ColumnHeading{equivalent (C.G.S.)}{Absolute value of \\
the mechanical \\
equivalent (C.G.S. \\
system, erg).} \\
\hline
\Strut
Laboratory caloric & $427$ & $419 × 10^{5}$ \\
Zero calorie & $430$ & $422 × 10^{5}$ \\
\hline
\end{tabular}
\end{center}

The numbers of the last column are derived from those
of the preceding one by multiplying by $98,100$, to reduce
grams to dynes, and meters to centimeters. Joule's results
have been substantially confirmed by recent careful measurements
\index{Joule's experiments|)}%
by Rowland and others.

\Section{62.} The determination of the mechanical equivalent
of heat enables us to express quantities of heat in ergs
directly, instead of calories. The advantage of this is, that
a quantity of heat is not only proportional to, but directly
equal to its mechanical equivalent, whereby the mathematical
expression for the energy is greatly simplified.
This unit of heat will be used in all subsequent equations.
The return to calories is, at any time, readily accomplished
by dividing by $419 × 10^{5}$.

\Section{63.} Some further propositions immediately follow from
\index{First law of thermodynamics}%
the above exposition of the principle of energy. The
energy, as stated, depends on the momentary condition of
the system. To find the change of energy, $U_{1} - U_{2}$,
accompanying the transition of the system from a state~$1$
\PageSep{43}
to a state~$2$, we should, according to the definition of the
energy in~\SecRef{58}, have to measure~$U_{1}$ as well as~$U_{2}$ by the
\index{Energy!change of}%
mechanical equivalent of the external effects produced in
passing from the given states to the normal state. But,
supposing we so arrange matters that the system passes
from state~$1$, through state~$2$, into the normal state, it is
evident then that $U_{1} - U_{2}$ is simply the mechanical equivalent
of the external effects produced in passing from~$1$ to~$2$.
The decrease of the energy of a system subjected to any
change is, then, the mechanical equivalent of the external
effects resulting from that change; or, in other words, the
\emph{increase of the energy} of a system which undergoes any
change, is equal to the mechanical equivalent of the heat
absorbed and the work expended in producing the change:\Add{---}
\[
U_{2} - U_{1} = Q + W\Add{.}
\Tag{(17)}
\]
$Q$~is the mechanical equivalent of the heat absorbed by the
system, \eg\ by conduction, and $W$~is the amount of work
expended on the system. $W$~is positive if the change takes
place in the direction of the external forces. The sum
$Q + W$ represents the mechanical equivalent of all the
thermal and mechanical operations of the surrounding
bodies on the system. We shall use $Q$~and $W$ always in
this sense.

The value of $Q + W$ is independent of the manner of
the transition from~$1$ to~$2$, and evidently also of the selection
of the normal state. When differences of energy of
one and the same system are considered, it is, therefore, not
even necessary to fix upon a normal state. In the expression
for the energy of the system there remains then an
arbitrary additive constant undetermined.

\Section{64.} The difference $U_{2} - U_{1}$ may also be regarded as
the energy of the system in state~$2$, referred to state~$1$ as
the normal state. For, if the latter be thus selected, then
$U_{1} = 0$, since it takes no energy to change the system from~$1$
to the normal state, and $U_{2} - U_{1} = U_{2}$. The normal
\PageSep{44}
state is, therefore, sometimes called the state of zero
\index{Zero!energy}%
energy.
\index{Energy!zero}%

\Section{65.} States $1$~and $2$ may be identical, in which case the
system changing from~$1$ to~$2$ passes through a so-called
\index{Perfect gases!system}%
\index{System!perfect}%
\emph{cycle of operations}. In this case,
\index{Cycle of operations}%
\[
U_{2} = U_{1}\quad\text{and}\quad Q + W = 0\Add{.}
\Tag{(18)}
\]
The mechanical equivalent of the external effects is zero, or
the external heat effect is equal in magnitude and opposite
in sign to the external work. This proposition shows the
impracticability of perpetual motion, which necessarily
presupposes engines working in complete cycles.

\Section{66.} If no external effects ($Q = 0$, $W = 0$) be produced
by a change of state of the system, its energy remains
constant (conservation of the energy). The quantities, on
which the state of the system depends, may undergo considerable
changes in this case, but they must obey the
condition $U = \const$.

A system which changes without being acted on by
external agents is called a \emph{perfect system}. Strictly speaking,
no perfect system can be found in nature, since there
is constant interaction between all material bodies of the
universe. It is, however, of importance to observe that by
an adequate choice of the system which is to undergo the
contemplated change, we have it in our power to make
the external effect as small as we please, in comparison
with the changes of energy of portions of the system itself.
Any particular external effect may be eliminated by making
the body which produces this effect, as well as the recipient,
a part of the system under consideration. In the case of a
gas which is being compressed by a weight sinking to a
lower level, if the gas by itself be the system considered,
the external effect on it is equal to the work done by the
weight. The energy of the system accordingly increases.
If, however, the weight and the earth be considered parts of
the system, all external effects are eliminated, and the
\PageSep{45}
energy of this system remains constant. The expression
\index{Energy!potential}%
\index{Potential, energy}%
for the energy now contains a new term representing the
potential energy of the weight. The loss of the potential
energy of the weight is exactly compensated by the gain
of the internal energy of the gas. All other cases admit
of similar treatment.
\PageSep{46}


\Chapter{II.}{Applications to Homogeneous Systems.}

\Section{67.} \First{We} shall now apply the first law of thermodynamics
\index{First law of thermodynamics}%
as expressed in equation~\Eq{(17)},
\[
U_{2} - U_{1} = Q + W,
\]
to a homogeneous substance, whose state is determined,
besides by its chemical nature and mass~$M$, by two variables,
the temperature~$\theta$ and the volume~$V$, for instance.
The term \emph{homogeneous} is used here in the sense of \emph{physically
homogeneous}, and is applied to any system which appears
of completely uniform structure throughout. The substance
may be chemically homogeneous, \ie\ it may consist
entirely of the same kind of molecules, or chemical transformations
may take place at some stage of the process,
as, for example, in the case of a vapour, which partially
dissociates on being heated. The homogeneous state must,
however, be a single valued function of the temperature
and the volume. As long as the system is at rest, the
total energy consists of the so-called \emph{internal} energy~$U$,
which depends only on the internal state of the substance
as determined by its density and temperature, and on its
mass, to which it is evidently proportional. In other cases
the total energy contains, besides the internal energy~$U$,
another term, namely, the kinetic energy, which is known
from the principles of mechanics.

In order to determine the functional relation between $U$,~$\theta$,
and~$V$, the state of the system must be changed, and the
external effects of this change calculated. Equation~\Eq{(17)}
then gives the corresponding change of energy.
\PageSep{47}

\Section{68.} If a gas, initially at rest and at uniform temperature,
be allowed to suddenly expand by the opening of a \Chg{stopcock}{stop-cock},
which makes communication with a previously exhausted
vessel, a number of intricate mechanical and thermal
changes will at first take place. The portion of the gas
flowing into the vacuum is thrown into violent motion, then
heated by impact against the sides of the vessel and by
compression of the particles crowding behind, while the
portion remaining in the first vessel is cooled down by
expansion, etc. Assuming the walls of the vessels to be
absolutely rigid and non-conducting, and denoting by~$2$ any
particular state after communication between the vessels
has been established, then, according to equation~\Eq{(17)}, the
total energy of the gas in state~$2$ is precisely equal to that
\index{Energy!internal}%
in state~$1$, for neither thermal nor mechanical forces have
acted on the gas from without. The reaction of the walls
does not perform any work. The energy in state~$2$ is,
in general, composed of many parts, viz.\ the kinetic and
internal energies of the gas particles, each one of which, if
taken sufficiently small, may be considered as homogeneous
and uniform in temperature and density. If we wait until
complete rest and thermal equilibrium have been re-established,
and denote this state by~$2$, then in~$2$, as in~$1$, the total
energy consists only of the internal energy~$U$, and we have
\index{Internal energy}%
$U_{2} = U_{1}$. But the variables $\theta$~and~$V$, on which $U$~depends,
have passed from $\theta_{1}$,~$V_{1}$ to $\theta_{2}$,~$V_{2}$, where $V_{2} > V_{1}$. By
measuring the temperatures and the volumes, the relation
between the temperature and the volume in processes where
the internal energy remains constant may be established.

\Section{69.} Joule performed such an experiment as described,
\index{Joule's experiments}%
and found that for perfect gases $\theta_{2} = \theta_{1}$. He put the two
communicating vessels, one filled with air at high pressure,
the other exhausted, into a common water-bath at the
same temperature, and found that, after the air had expanded
and equilibrium had been established, the change
of temperature of the water-bath was inappreciable. It
immediately follows that, if the walls of the vessels were
\PageSep{48}
non-conducting, the final temperature of the total mass of
the gas would be equal to the initial temperature; for otherwise
the change in temperature would have communicated
itself to the water-bath in the above experiment.

Hence, if the internal energy of a nearly perfect gas
\index{Internal energy!of perfect gas}%
remains unchanged after a considerable change of volume,
then its temperature also remains almost constant. In other
words, \emph{the internal energy of a perfect gas depends only on the
temperature, and not on the volume}.

\Section{70.} For a conclusive proof of this important deduction,
\index{Joule and Thomson's absolute temperature!experiments}%
\index{Porous plug experiments}%
much more accurate measurements are required. In
Joule's experiment described above, the heat capacity of
the gas is so small compared with that of the vessel and the
water-bath, that a considerable change of temperature in
the gas would have been necessary to produce an appreciable
change of temperature in the water. More reliable results
are obtained by a modification of the above method devised
by Sir~William Thomson (Lord Kelvin), and used by him,
along with Joule, for accurate measurements. Here the
outflow of the gas is artificially retarded, so that the gas
passes immediately into its second state of equilibrium.
The temperature~$\theta_{2}$ is then directly measured in the stream
of outflowing gas. No limited quantity of gas rushes
tumultuously into a vacuum, but a gas is slowly transferred
in a steady flow from a place of high pressure,~$p_{1}$, to one of
low pressure,~$p_{2}$ (the atmosphere), by forcing it through a
boxwood tube stopped at one part of its length by a porous
plug of cotton wool or filaments of silk. The results of the
experiment show that when the flow has become steady there
is, for air, a very small change of temperature, and, for hydrogen,
a still smaller, hardly appreciable change. Hence the
conclusion appears justified, that, for a perfect gas, the
change of temperature vanishes entirely.

This leads to an inference with regard to the internal
energy of a perfect gas. When, after the steady state of
the process has been established, a certain mass of the gas
has been completely pushed through the plug, it has been
\PageSep{49}
operated upon by external agents during its change from
the volume,~$V_{1}$, at high pressure, to the larger volume,~$V_{2}$,
at atmospheric pressure. The mechanical equivalent of
these operations, $Q + W$, is to be calculated from the
external changes. The state of the porous plug remains
the same throughout; hence the processes that take place
in it may be neglected. No change of temperature occurs
outside the tube, as the material of which it is made is
practically non-conducting; hence $Q = 0$. The mechanical
work done by a piston in pressing the gas through the plug
at the constant pressure~$p_{1}$ is evidently~$p_{1}V_{1}$, and this for a
perfect gas at constant temperature is, according to Boyle's
law, equal to the work~$p_{2}V_{2}$, which is gained by the
escaping gas pushing a second piston at pressure~$p_{2}$ through
a volume~$V_{2}$. Hence the sum of the external work~$W$ is
also zero, and therefore, according to equation~\Eq{(17)}, $U_{2} = U_{1}$.
As the experimental results showed the temperature to be
practically unchanged while the volume increased very considerably,
the internal energy of a perfect gas can depend
only on the temperature and not on the volume, \ie,
\[
\left(\frac{\dd U}{\dd V}\right)_{\theta} = 0\Add{.}
\Tag{(19)}
\]

For nearly perfect gases, as hydrogen, air, etc., the
actual small change of temperature observed shows how far
the internal energy depends on the volume. It must,
however, be borne in mind that for such gases the external
work,
\[
W = p_{1}V_{1} - p_{2}V_{2},
\]
does not vanish; hence the internal energy does not remain
constant. For further discussion, see~\SecRef{158}.

\Section{71.} Special theoretical importance must be attached to
\index{Infinitely slow!process|(}%
those thermodynamical processes which progress infinitely
slowly, and which, therefore, consist of a succession of
states of equilibrium. Strictly speaking, this expression is
vague, since a process presupposes changes, and, therefore,
disturbances of equilibrium. But where the time taken is
\PageSep{50}
immaterial, and the result of the process alone of consequence,
these disturbances may be made as small as we
please, certainly very small in comparison with the other
quantities which characterize the state of the system under
observation. Thus, a gas may be compressed very slowly to
any fraction of its original volume, by making the external
pressure, at each moment, just a trifle greater than the
internal pressure of the gas. Wherever external pressure
enters---as, for instance, in the calculation of the work of
compression---a very small error will then be committed, if
the pressure of the gas be substituted for the external
pressure. On passing to the limit, even that error vanishes.
In other words, the result obtained becomes rigorously
exact for \emph{infinitely slow} compression.
\index{Infinitely slow!compression}%

This holds for compression at constant as well as at
variable pressure. The latter may be given the required
value at each moment by the addition or removal of small
weights. This may be done either by hand (by pushing
weights to one side), or by means of some automatic device
which acts merely as a release, and therefore does not contribute
towards the work done.

\Section{72.} The conduction of heat to and from the system may
be treated in the same way. When it is not a question of
time, but only of the amount of heat received or given out
by the system, it is sufficient, according as heat is to be
added to or taken from the system, to connect it with a heat-reservoir
of slightly higher or lower temperature than that
of the system. This small difference serves, merely, to
determine the direction of the flow of the heat, while its
magnitude is negligible compared with the changes of the
system, which result from the process. We, therefore, speak
of the conduction of heat between bodies of equal temperature,
just as we speak of the compression of a gas by an
external pressure equal to that of the gas. This is merely
anticipating the result of passing to the limit from a small
finite difference to an infinitesimal difference of temperature
between the two bodies.
\PageSep{51}
\index{External conditions of equilibrium!work in reversible process}%
\index{Work!external, in reversible process}%

This applies not only to strictly isothermal processes,
but also to those of varying temperature. One heat-reservoir
of constant temperature will not suffice for carrying
out the latter processes. These will require either an
auxiliary body, the temperature of which may be arbitrarily
changed, \eg\ a gas that can be heated or cooled at pleasure
by compression or expansion; or a set of constant-temperature
reservoirs, each of different temperature. In the latter
case, at each stage of the process we apply that particular
heat-reservoir whose temperature lies nearest to that of the
system at that moment.

\Section{73.} The value of this method of viewing the process
lies in the fact that we may imagine each \emph{infinitely slow}
process to be carried out also in the opposite direction. If
a process consist of a succession of states of equilibrium
with the exception of very small changes, then evidently a
suitable change, quite as small, is sufficient to reverse the
process. This small change will vanish when we pass over
to the limiting case of the infinitely slow process, for a
\index{Infinitely slow!process|)}%
definite result always contains a quite definite error, and if
this error be smaller than any quantity, however small, it
must be zero.

\Section{74.} We pass now to the application of the first law to
a process of the kind indicated, and, therefore, reversible in
its various parts. Taking the volume~$V$ (abscissa) and
the pressure~$p$ (ordinate) as the independent variables, we
may graphically illustrate our process by plotting its successive
states of equilibrium in the form of a curve in the
plane of the co-ordinates. Each point in this plane corresponds
to a certain state of our system, the chemical
nature and mass of which are supposed to be given, and
each curve corresponds to a series of continuous changes
of state. Let the curve a from~$1$ to~$2$ represent a reversible
process which takes the substance from a state~$1$ to a state~$2$
(\Fig{2}). Along~$\alpha$, according to equation~\Eq{(17)}, the increase
of the energy is
\[
U_{2} - U_{1} = W + Q,
\]
\PageSep{52}
where $W$~is the mechanical work expended on the substance,
and $Q$~the total heat absorbed by it.

\Section{75.} The value of~$W$ can be readily determined. $W$~is
\index{External conditions of equilibrium!work in reversible process}%
made up of the elementary quantities of work done on the
system during the infinitesimal changes corresponding to
the elements of arc of the curve~$\alpha$. The external pressure
is at any moment equal to that of the substance, since the
process is supposed to be reversible. Consequently, by the
\Figure{2}
laws of hydrodynamics, the work done by the external
forces in the infinitely small change is equal to the product
of the pressure~$p$, and the decrease of the volume,~$-dV$, no
matter what the geometrical form of the surface of the
body may be. Hence the external work done during the
whole process is
\[
W = -\int_{1}^{2} p\, dV,
\Tag{(20)}
\]
in which the integration extends from~$1$ to~$2$ along the
curve~$\alpha$. If $p$~be positive, as in the case of gases, and
$V_{2} > V_{1}$ as in \Fig{2}, $W$~is negative.
\PageSep{53}

In order to perform the integration, the curve~$\alpha$, \ie\ the
relation between $p$ and~$V$, must be known. As long as only
the points $1$~and $2$ are given, the integral has no definite
value. In fact, it assumes an entirely different value along
a different curve,~$\beta$, joining $1$~and~$2$. Therefore $p\, dV$~is
not a perfect differential. Mathematically this depends on
the fact that $p$~is in general not only a function of~$V$, but
also of another variable, the temperature~$\theta$, which also changes
along the path of integration. As long as $\alpha$~is not given,
no statement can be made with regard to the relation between
$\theta$~and~$V$, and the integration cannot be performed.

The external work,~$W$, is evidently represented by the
area (taken negative) of the plane figure bounded by the
curve~$\alpha$, the ordinates at $1$~and~$2$, and the axis of abscissæ.
This, too, shows that $W$~depends on the path of the curve~$\alpha$.
Only for infinitesimal changes, \ie\ when $1$~and $2$ are
infinitely near one another and $\alpha$~shrinks to a curve element,
is $W$~determined by the initial and final points of the
curve alone.

\Section{76.} The second measurable quantity is~$Q$, the heat
\index{Heat!absorbed}%
\index{Heat and work, analogy between}%
\index{Work!and heat, analogy between}%
absorbed. It may be determined by calorimetric methods
in calories, and then expressed in mechanical units by multiplying
by the mechanical equivalent of heat. We shall
now consider the theoretical determination of~$Q$. It is, like~$W$,
the algebraical sum of the infinitely small quantities of
heat added to the body during the elementary processes
corresponding to the elements of the curve~$\alpha$. Such an
increment of heat cannot, however, be immediately calculated,
from the position of the curve element in the co-ordinate
plane, in a manner similar to that of the increment of work.
To establish an analogy between the two, one might, in
imitation of the expression~$-p\, dV$, put the increment of
heat $= C\, d\theta$, where $d\theta$~is the increment of temperature, and
$C$~the heat capacity, which is usually a finite quantity.
But $C$~has not, in general, a definite value. It does not
depend, as the factor~$p$ in the expression for the increment
of work, alone on the momentary state of the substance, \ie\
\PageSep{54}
\index{External conditions of equilibrium!work in complete cycle}%
on the position of the point of the curve considered, but
also on the direction of the curve element. In isothermal
changes $C$~is evidently $= \pm\infty$, because $d\theta = 0$, and the
heat added or withdrawn is a finite quantity. In adiabatic
changes $C = 0$, for here the temperature may change in
any way, while no heat is added or withdrawn. For a given
point, $C$~may, therefore, in contradistinction to~$p$, assume
all values between~$+\infty$ and~$-\infty$. (\Cf~\SecRef{47}.) Hence the
analogy is incomplete in one essential, and does not, in the
general case, simplify the problem in hand. We shall also
find that the breaking up of the heat absorbed into the two
factors $\theta$~and~$d\Phi$ (\SecRef{120}), is permissible only in some very
special cases.

\Section{77.} Although the value of~$Q$ cannot, in general, be
directly determined, equation~\Eq{(17)} enables us to draw some
important inferences regarding it. Substituting the value
of~$W$ from equation~\Eq{(20)} in equation~\Eq{(17)}, we obtain
\[
Q = U_{2} - U_{1} + \int_{1}^{2} p\, dV,
\Tag{(21)}
\]
which shows that the value of~$Q$ depends not only on the
position of the points $1$~and~$2$, but also on the connecting
path ($\alpha$~or~$\beta$). Carnot's theory of heat cannot be reconciled
with this proposition, as we have shown at length in
% [** Two section signs in the original]
\SecRef{51}~and~\SecNum{52}.

\Section{78.} The complete evaluation of~$Q$ is possible in the
case where the substance returns to its initial state, having
gone through a cycle of operations. This might be done by
first bringing the system from~$1$ to~$2$ along~$\alpha$, then back
from~$2$ to~$1$ along~$\beta$. Then, as in all cycles (\SecRef{65}),
\[
Q = -W.
\]
The external work is
\[
W = -\int_{1}^{1} p\, dV,
\]
the integral to be taken along the closed curve $1\alpha 2\beta 1$.
$W$~evidently represents the area bounded by the curve,
\PageSep{55}
and is positive if the process follows the direction of the
arrow in \Fig{2}.

\Section{79.} We shall now consider the special case where the
\index{Clausius' equation!notation}%
curve~$\alpha$, which characterizes the change of state, shrinks
into an element, so that the points $1$~and~$2$ lie infinitely
near one another. $W$~here becomes the increment of work,~$-p\, dV$,
and the change of the internal energy is~$dU$.
Hence, according to~\Eq{(21)}, the heat absorbed assumes the
value:\footnote
  {It is usual to follow the example of Clausius, and denote this quantity
  by~$dQ$, to indicate that it is infinitely small. This notation, however, has
  frequently given rise to misunderstanding, for $dQ$~has been repeatedly
  regarded as the differential of a known finite quantity~$Q$. We therefore
  adhere to the notation given above. Other authors use~$d'Q$, in order to
  obviate the aforesaid misunderstanding.}
\[
Q = dU + p\, dV.
\]
Per unit mass, this equation becomes
\[
q = du + p\, dv,
\Tag{(22)}
\]
where the small letters denote the corresponding capitals
divided by~$M$. In subsequent calculations it will often be
advisable to use $\theta$~as an independent variable, either in
conjunction with~$p$, or~$v$. We shall, in each case, select as
independent variables those which are most conducive to a
simplification of the problem in hand. The meaning of the
differentiation will be indicated whenever a misunderstanding
is possible.

We shall now apply our last equation~\Eq{(22)} to the most
important reversible processes.

\Section{80.} It has been repeatedly mentioned that the specific
heat of a substance may be defined in very different ways
according to the manner in which the heating is carried out.
But, according to \SecRef{46} and equation~\Eq{(22)}, we have, for any
heating process,
\[
c = \frac{q}{d\theta} = \frac{du}{d\theta} + p\, \frac{dv}{d\theta}.
\Tag{(23)}
\]
\PageSep{56}

In order to give a definite meaning to the differential
coefficients, some arbitrary condition is required, which will
prescribe the direction of the change. A single condition
is sufficient, since the state of the substance depends on two
variables only.

\Section{81.} \Topic{Heating at Constant Volume.}---Here $dv = 0$,
\index{Heating!at constant pressure}%
\index{Heating!at constant volume}%
$c = c_{v}$, the specific heat at constant volume. Hence, according
\index{Specific heat!at constant pressure}%
\index{Specific heat!at constant volume}%
to equation~\Eq{(23)},
\begin{align*}
c_{v} &= \left(\frac{\dd u}{\dd \theta}\right)_{v}\Add{,}
\Tag{(24)} \\
\intertext{or}
c_{v} &= \left(\frac{\dd u}{\dd p}\right)_{v} \left(\frac{\dd p}{\dd \theta}\right)_{v}\Add{.}
\Tag{(25)}
\end{align*}

\Section{82.} \Topic{Heating under Constant Pressure.}---Here $dp = 0$,
$c = c_{p}$, the specific heat at constant pressure. According
to equation~\Eq{(23)},
\begin{align*}
c_{p} &= \left(\frac{\dd u}{\dd \theta}\right)_{p} + p\left(\frac{\dd v}{\dd \theta}\right)_{p}\Add{,}
\Tag{(26)} \\
\intertext{or}
c_{p} &= \left[\left(\frac{\dd u}{\dd v}\right)_{p} + p\right]\left(\frac{\dd v}{\dd \theta}\right)_{p}\Add{.}
\Tag{(27)}
\end{align*}
By the substitution of
\[
\left(\frac{\dd u}{\dd \theta}\right)_{p}
  = \left(\frac{\dd u}{\dd \theta}\right)_{v}
  + \left(\frac{\dd u}{\dd v}\right)_{\theta} \left(\frac{\dd v}{\dd \theta}\right)_{p}
\]
in~\Eq{(26)}, $c_{p}$~may be written in the form
\[
c_{p} = \left(\frac{\dd u}{\dd \theta}\right)_{v}
  + \left[\left(\frac{\dd u}{\dd v}\right)_{\theta} + p\right]\left(\frac{\dd v}{\dd \theta}\right)_{p},
\]
or, by~\Eq{(24)},
\[
c_{p} = c_{v} + \left[\left(\frac{\dd u}{\dd v}\right)_{\theta} + p\right]\left(\frac{\dd v}{\dd \theta}\right)_{p}.
\Tag{(28)}
\]

\Section{83.} By comparing \Eq{(25)} and~\Eq{(27)} and eliminating~$u$, we
are led to a direct experimental test of the theory.

By~\Eq{(25)},
\[
\left(\frac{\dd u}{\dd p}\right)_{v} = c_{v} \left(\frac{\dd \theta}{\dd p}\right)_{v},
\]
and by~\Eq{(27)},
\[
\left(\frac{\dd u}{\dd v}\right)_{p} = c_{p} \left(\frac{\dd \theta}{\dd v}\right)_{p} - p;
\]
\PageSep{57}
whence, differentiating the former equation with respect to~$v$,
keeping $p$~constant, and the latter with respect to~$p$,
keeping $v$~constant, and equating, we have
\[
\frac{\dd}{\dd v}\left(c_{v}\, \frac{\dd \theta}{\dd p}\right)
  = \frac{\dd}{\dd p}\left(c_{p}\, \frac{\dd \theta}{\dd v} - p\right)\Add{,}
\]
or
\[
(c_{p} - c_{v})\, \frac{\dd^{2} \theta}{\dd p\, \dd v}
  + \frac{\dd c_{p}}{\dd p} · \frac{\dd \theta}{\dd v}
  - \frac{\dd c_{v}}{\dd v} · \frac{\dd \theta}{\dd p} = 1\Add{.}
\Tag{(29)}
\]
This equation contains only quantities which may be
experimentally determined, and therefore furnishes a means
for testing the first law of thermodynamics by observations
on any homogeneous substance.

\Section{84.} \Topic{Perfect Gases.}---The above equations undergo
\index{Boyle's law}%
\index{Gases, perfect}%
\index{Joule and Thomson's absolute temperature!experiments}%
\index{Laws:!Avogadro's}%
\index{Laws:!Boyle's}%
\index{Laws:!Gay-Lussac's}%
\index{Perfect gases}%
considerable simplifications for perfect gases. We have,
from~\Eq{(14)},
\[
p = \frac{R}{m} · \frac{\theta}{v},
\Tag{(30)}
\]
where $R = 826 × 10^{5}$ and $m$~is the (real or apparent) molecular
weight. Hence
\[
\theta = \frac{m}{R}\, pv,
\]
and equation~\Eq{(29)} becomes
\[
c_{p} - c_{v} + p\, \frac{\dd c_{p}}{\dd p} - v\, \frac{\dd c_{v}}{\dd v} = \frac{R}{m}.
\]
Assuming that only the laws of Boyle, Gay-Lussac, and
\index{Gay-Lussac}%
Avogadro hold, no further conclusions can be drawn from
the first law of thermodynamics with regard to perfect gases.

\Section{85.} We shall now make use of the additional property
of perfect gases, established by Thomson and Joule (\SecRef{70}),
that the internal energy of a perfect gas depends only on
\index{Energy!internal!of perfect gas}%
the temperature, and not on the volume, and that hence
per unit mass, according to~\Eq{(19)},
\[
\left(\frac{\dd u}{\dd v}\right)_{\theta} = 0.
\Tag{(31)}
\]
\PageSep{58}
The general equation,
\[
du = \left(\frac{\dd u}{\dd \theta}\right)_{v} d\theta + \left(\frac{\dd u}{\dd v}\right)_{\theta} dv,
\]
then becomes, for perfect gases,
\[
du = \left(\frac{\dd u}{\dd \theta}\right)_{v} d\theta,
\]
and, according to~\Eq{(24)},
\[
du = c_{v} · d\theta.
\Tag{(32)}
\]
It follows from~\Eq{(28)} that
\[
c_{p} = c_{v} + p \left(\frac{\dd v}{\dd \theta}\right)_{p},
\]
or, considering the relation~\Eq{(30)},
\[
c_{p} = c_{v} + \frac{R}{m};
\]
\ie\ there is a constant difference between the specific heat
\index{Heat!molecular!of perfect gases}%
\index{Molecular heat!of perfect gases}%
at constant pressure and the specific heat at constant
volume. Referring the heat capacity to the molecular
weight~$m$, instead of to unit mass, we have
\[
mc_{p} - mc_{v} = R.
\Tag{(33)}
\]
The difference is, therefore, independent even of the nature
of the gas.

\Section{86.} Only the specific heat at constant pressure,~$c_{p}$, is
capable of direct experimental determination, because a
quantity of gas enclosed in a vessel of constant volume has
far too small a heat capacity to produce sufficient thermal
effects on the surrounding bodies. Since $c_{v}$, according to~\Eq{(24)},
like~$u$, depends on the temperature only, and not on
the volume, the same follows for~$c_{p}$, according to~\Eq{(33)}.
This conclusion was first confirmed by Regnault's experiments.
\index{Regnault}%
He found $c_{p}$~constant within a considerable range
of temperature. By~\Eq{(33)}, $c_{v}$~is constant within the same
range.
\PageSep{59}

If the molecular heats be expressed in calories, $R$~must
be divided by Joule's equivalent~$J$. The difference between
the molecular heats at constant pressure and at
constant volume is then
\[
mc_{p} - mc_{v} = \frac{R}{J} = \frac{826 · 10^{5}}{419 · 10^{5}} = 1.971\Add{.}
\Tag{(34)}
\]

\Section{87.} The following table contains the specific heats
\index{Ratio of specific heats}%
\index{Specific heats!ratio of}%
\index{Specific heat!at constant pressure}%
\index{Specific heat!at constant volume}%
and molecular heats of several gases at constant pressure,
measured by direct experiment; also the molecular heats
at constant volume found by subtracting~$1.97$, and also
the ratio $\dfrac{c_{p}}{c_{v}} = \gamma$:---
\begin{center}
\TableFont
\begin{tabular}{l*{5}{|@{\;}c@{\;}}}
\hline
\Strut[16pt]
& $c_{p}$ & $m$ & $mc_{p}$ & $mc_{v}$ & $\dfrac{c_{p}}{c_{v}} = \gamma$ \\
%[** TN: Re-breaking column headings]
& \ColumnHeading{Specific heat}{Specific heat \\
at const. \\
pressure.}
&
\ColumnHeading{Molecular}{Molecular \\
weight.}
&
\ColumnHeading{Molecular}{Molecular \\
heat at \\
const. \\
pressure.}
&
\ColumnHeading{Molecular}{Molecular \\
heat at \\
const. \\
volume.}
& \\
\cline{2-6}
\Strut
Hydrogen & $3.410\Z$ & $\Z2.0$ & $6.82$ & $4.85$ & $1.41$ \\
Oxygen & $0.2175$ & $31.9$ & $6.94$ & $4.97$ & $1.40$ \\
Nitrogen & $0.2438$ & $28.0$ & $6.83$ & $4.86$ & $1.41$ \\
Air & $0.2375$ & $28.8$ & $6.84$ & $4.87$ & $1.41$ \\
\hline
\end{tabular}
\end{center}

The specific heat generally increases slowly on considerable
increase of temperature. Within the range of
temperature in which the specific heat is constant, equation~\Eq{(32)}
can be integrated, giving
\[
u = c_{v} \theta + \const
\Tag{(35)}
\]
The constant of integration depends on the selection of the
zero point of energy. For perfect gases, we consider $c_{p}$~and
$c_{v}$ as constants throughout, hence the last equation holds
good in general.

\Section{88.} \Topic{Adiabatic Process.}---The characteristic feature
\index{Adiabatic process}%
of the adiabatic process is that $q = 0$, and, according to
equation~\Eq{(22)},
\[
0 = du + p\, dv.
\]
\PageSep{60}

Assuming, again, a perfect gas, and substituting the
values of~$du$ from~\Eq{(32)} and of~$p$ from~\Eq{(30)}, we have
\[
0 = c_{v}\, d\theta + \frac{R}{m} · \frac{\theta}{v}\, dv,
\Tag{(36)}
\]
or, on integrating,
\[
c_{v} \log\theta + \frac{R}{m} \log v = \const
\]
Replacing $\dfrac{R}{m}$ according to~\Eq{(33)} by $c_{p} - c_{v}$, and dividing
by~$c_{v}$, we get
\[
\log \theta + (\gamma - 1) \log v = \const
\Tag{(37)}
\]
(\ie\ during adiabatic expansion the temperature decreases)\Add{.}
Remembering that according to the characteristic equation\Eq{(30)}
\[
\log p + \log v - \log \theta = \const,
\]
we have, on eliminating~$v$,
\[
-\gamma \log \theta + (\gamma - 1) \log p = \const
\]
(\ie\ during adiabatic compression the temperature rises);
or, on eliminating~$\theta$,
\[
\log p + \gamma \log v = \const
\]
The values of the constants of integration are given by
the initial state of the process.

If we compare our last equation in the form
\[
pv^{\gamma} = \const
\Tag{(38)}
\]
with Boyle's law $pv = \const$, it is seen that during adiabatic
compression the volume decreases more slowly for an increase
of pressure than during isothermal compression,
because during adiabatic compression the temperature rises.
The adiabatic curves in the $\Chg{pv}{(p, v)}$-plane (\SecRef{22}) are, therefore,
steeper than the hyperbolic isotherms.

\Section{89.} Adiabatic processes may be used in various ways
\PageSep{61}
for the determination of~$\gamma$, the ratio of the specific heats.
The agreement of the results with the value calculated from
the mechanical equivalent of heat forms an important
confirmation of the theory.

Thus, the measurement of the velocity of sound in a gas
\index{Sound, velocity of}%
may be used for determining the value of~$\gamma$. It is proved
in hydrodynamics that the velocity of sound in a fluid is
$\sqrt{\dfrac{dp}{d\rho}}$, where $\rho = \dfrac{1}{v}$, the density of the fluid. Since gases
are bad conductors of heat, the compressions and expansions
which accompany sound-vibrations must be considered as
adiabatic, and not isothermal, processes. The relation
between the pressure and the density is, therefore, in
the case of perfect gases, not that expressed by Boyle's
law $\dfrac{p}{\rho} = pv = \const$, but that given by equation~\Eq{(38)}, viz.---
\[
\frac{p}{\rho^{\gamma}} = \const
\]
Hence, by differentiation
\[
\frac{dp}{d\rho} = \frac{\gamma p}{\rho} = \gamma pv,
\]
or, according to~\Eq{(30)},
\begin{align*}
\frac{dp}{d\rho} &= \gamma \frac{R}{m} \theta, \\
\gamma &= \frac{m}{R\theta} · \frac{dp}{d\rho}.
\end{align*}

In air at~$0°$, the velocity of sound is $\sqrt{\dfrac{dp}{d\rho}} = 33280\, \dfrac{\Unit{cm.}}{\Unit{sec.}}$;
hence, according to our last equation, taking the values of~$m$
from \SecRef{41}, and of~$R$ from \SecRef{84}, and $\theta = 273$,
\[
\gamma = \frac{28.8}{826 · 10^{5}} · \frac{33280^{2}}{273} = 1.41.
\]
This agrees with the value calculated in~\SecRef{87}.

Conversely, the value of~$\gamma$, calculated from the velocity
\PageSep{62}
\index{Meyer, Robert}%
of sound, may be used in the calculation of~$c_{v}$ in calories,
for the determination of the mechanical equivalent of heat
from~\Eq{(33)}. This method of evaluating the mechanical
equivalent of heat was first proposed by Robert Meyer in
1842. It is true that the assumption expressed in equation~\Eq{(31)},
that the internal energy of air depends only on the
temperature, is essential to this method. In other words,
this means that the difference of the specific heats at constant
pressure and constant volume depends only on the
external work. The direct proof of this fact, however,
must be considered as first given by the experiments of
Thomson and Joule, described in~\SecRef{70}.

\Section{90.} We shall now consider a more complex process,
a reversible cycle of a special kind, which has played an
important part in the development of thermodynamics,
known as Carnot's cycle, and shall apply the first law to it
\index{Carnot's!cycle}%
in detail.

Let a substance of unit mass, starting from an initial
state characterized by the values $\theta_{1}$,~$v_{1}$, first be compressed
\emph{adiabatically} until its temperature rises to~$\theta_{2}$ ($\theta_{2} > \theta_{1}$) and its
volume reduced to~$v_{2}$ ($v_{2} < v_{1}$) (\Fig{3}). Second, suppose it
be now allowed to expand \emph{isothermally} to volume~$v_{2}'$ ($v_{2}' > v_{2}$),
in constant connection with a heat-reservoir of constant
temperature,~$\theta_{2}$, which gives out the heat of expansion~$Q_{2}$.
Third, let it be further expanded \emph{adiabatically} until its
temperature falls to~$\theta_{1}$, and the volume thereby increased
to~$v_{1}'$. Fourth, let it be compressed \emph{isothermally} to the
original volume~$v_{1}$, while a heat-reservoir maintains the
temperature at~$\theta_{1}$, by absorbing the heat of compression.
All these operations are to be carried out in the reversible
manner described in~\SecRef{71}. The sum of the heat absorbed
by the system, and the work done on the system during
this cycle is, by the first law,
\[
Q + W = 0.
\Tag{(39)}
\]
The heat~$Q$, that has been absorbed by the substance, is
\[
Q = Q_{1} + Q_{2}
\Tag{(40)}
\]
\PageSep{63}
($Q_{1}$~is here negative). The external work~$W$ may be calculated
from the adiabatic and the isothermal compressibility
of the substance. According to~\Eq{(20)},
\[
W = -\int_{v_{1}, \theta_{1}}^{v_{2}, \theta_{2}} p\, dv
    -\int_{v_{2}, \theta_{2}}^{v_{2}', \theta_{2}} p\, dv
    -\int_{v_{2}', \theta_{2}}^{v_{1}', \theta_{1}} p\, dv
    -\int_{v_{1}', \theta_{1}}^{v_{1}, \theta_{1}} p\, dv.
\]
These integrals are to be taken along the curves $1$,~$2$, $3$,~$4$
respectively; $1$~and $3$ being adiabatic, $2$~and $4$ isothermal.
\Figure{3}

Assuming the substance to be a perfect gas, the above
integrals can readily be found. If we bear in mind the
relations \Eq{(30)} and~\Eq{(36)}, we have
\[
W = \int_{\theta_{1}}^{\theta_{2}} c_{v}\, d\theta
  - \frac{R}{m} \int_{v_{2}}^{v_{2}'} \frac{\theta_{2}}{v}\, dv
  + \int_{\theta_{2}}^{\theta_{1}} c_{v}\, d\theta
  - \frac{R}{m} \int_{v_{1}'}^{v_{1}} \frac{\theta_{1}}{v}\, dv\Add{.}
\Tag{(41)}
\]
The work of the adiabatic compression in the first part of
the process is equal in value and opposite in sign to that
of the adiabatic expansion in the third part of the process.
\PageSep{64}
There remains, therefore, the sum of the work in the
isothermal portions\Add{:}---
\[
W = -\frac{R}{m} \left(\theta_{2} \log \frac{v_{2}'}{v_{2}} + \theta_{1} \log \frac{v_{1}'}{v_{1}}\right).
\]
Now, the state $(v_{2}, \theta_{2})$ was developed from $(v_{1}, \theta_{1})$ by an
adiabatic process; therefore, by~\Eq{(37)},
\[
\log \theta_{2} + (\gamma - 1) \log v_{2}
  = \log \theta_{1} + (\gamma - 1) \log v_{1}.
\]
Similarly, for the adiabatic process, which leads from $(v_{2}', \theta_{2})$
to~$(v_{1}', \theta_{1})$,
\[
\log \theta_{2} + (\gamma - 1) \log v_{2}'
  = \log \theta_{1} + (\gamma - 1) \log v_{1}'.
\]
From these equations, it follows that
\[
\frac{v_{2}'}{v_{2}} = \frac{v_{1}'}{v_{1}},
\]
and
\[
\therefore
W = -\frac{R}{m} (\theta_{2} - \theta_{1}) \log \frac{v_{1}'}{v_{1}}.
\]
Since, in the case considered, $\theta_{2} > \theta_{1}$, and $\dfrac{v_{1}'}{v_{1}} = \dfrac{v_{2}'}{v_{2}} > 1$,
the total external work~$W$ is negative, \ie\ mechanical work
has been gained by the process. But, from \Eq{(39)}~and~\Eq{(40)},
\[
Q = Q_{1} + Q_{2}= -W;
\Tag{(42)}
\]
therefore $Q$~is positive, \ie\ the heat-reservoir at temperature~$\theta_{2}$
has lost more heat than the heat-reservoir at temperature~$\theta_{1}$
has gained.

The value of~$W$, substituted in the last equation,
gives
\[
Q = Q_{1} + Q_{2} = \frac{R}{m} (\theta_{2} - \theta_{1}) \log \frac{v_{1}'}{v_{1}}.
\Tag{(43)}
\]
The correctness of this equation is evident from the direct
calculation of the values of $Q_{1}$ and~$Q_{2}$. The gas expands
isothermally while the heat-reservoir at temperature~$\theta_{2}$ is
in action. The internal energy of the gas therefore remains
constant, and the heat absorbed is equal in magnitude and
\PageSep{65}
opposite in sign to the external work. Hence, by equating
$Q_{2}$ to the second integral in~\Eq{(41)},
\[
Q_{2} = \frac{R}{m} \theta_{2} \log \frac{v_{2}'}{v_{2}}
     = \frac{R}{m} \theta_{2} \log \frac{v_{1}'}{v_{1}},
\]
and, similarly, by equating $Q_{1}$ to the fourth integral in~\Eq{(41)},
\[
Q_{1} = \frac{R}{m} \theta_{1} \log \frac{v_{1}}{v_{1}'}
     = -\frac{R}{m} \theta_{1} \log \frac{v_{1}'}{v_{1}},
\]
which agrees with equation~\Eq{(43)}.

There exists, then, between the quantities $Q_{1}$,~$Q_{2}$,~$W$,
besides the relation given in~\Eq{(42)}, this new relation---
\[
Q_{1} : Q_{2}: W = (-\theta_{1}): \theta_{2} : (\theta_{1} - \theta_{2})\Add{.}
\Tag{(44)}
\]

\Section{91.} In order, now, to survey all the effects of the
above Carnot cycle, we shall compare the initial and final
states of all the bodies concerned. The gas operated upon
has not been changed in any way by the process, and may
be left out of account. It has done service only as a transmitting
agent, in order to bring about changes in the
surroundings. The two reservoirs, however, have undergone
a change, and, besides, a positive amount of external work,
$W' = -W$, has been gained; \ie\ at the close of the process
certain weights, which were in action during the compression
and the expansion, are found to be at a higher level than at
the beginning, or a spring, serving similar purposes, is at a
greater tension, etc. On the other hand, the heat-reservoir
at~$\theta_{2}$ has given out heat to the amount~$Q_{2}$, and the cooler
reservoir at~$\theta_{1}$ has received the smaller amount $Q_{1}' = -Q_{1}$.
The heat that has vanished is equivalent to the work gained.
This result may be briefly expressed as follows: The
quantity of heat~$Q_{2}$, at temperature~$\theta_{2}$, has passed in part~($Q_{1}'$)
to a lower temperature~($\theta_{1}$), and has in part ($Q_{2} - Q_{1}' = Q_{1} + Q_{2}$)
been transformed into mechanical work. Carnot's
cycle, performed with a perfect gas, thus affords a means of
drawing heat from a body and of gaining work in its stead,
without introducing any changes in nature except the
\PageSep{66}
transference of a certain quantity of heat from a body of
higher temperature to one of lower temperature.

But, since the process described is reversible in all its
parts, it may be put into effect in such a way that all the
quantities, $Q_{1}$,~$Q_{2}$,~$W$, change sign, $Q_{1}$~and $W$ becoming
positive, $Q_{2} = -Q_{2}'$ negative. In this case the hotter
reservoir at~$\theta_{2}$ receives heat to the amount~$Q_{2}'$, partly from
the colder reservoir (at~$\theta_{1}$), and partly from the mechanical
work expended~($W$). By reversing Carnot's cycle, we have,
then, a means of transferring heat from a colder to a hotter
body without introducing any other changes in nature than
the transformation of a certain amount of mechanical work
into heat. We shall see, later, that, for the success of
Carnot's reversible cycle, the nature of the transmitting
agent or working substance is immaterial, and that perfect
gases are, in this respect, neither superior nor inferior to
other substances (\cf~\SecRef{137}).
\PageSep{67}


\Chapter{III.}{Applications to Non-Homogeneous Systems.}

\Section{92.} \First{The} propositions discussed in the preceding chapter
are, in a large part, also applicable to substances which are
not perfectly homogeneous in structure. We shall, therefore,
in this chapter consider mainly such phenomena as
characterize the inhomogeneity of a system.

Let us consider a system composed of a number of
homogeneous bodies in juxtaposition, separated by given
bounding surfaces. Such a system may, or may not, be
chemically homogeneous. A liquid in contact with its
vapour is an example of the first case, if the molecules
of the latter be identical with those of the former. The
beginning of a chemical reaction, inasmuch as a substance
is in contact with another of different chemical constitution,
is an example of the second. Whether a system is physically
homogeneous or not, can, in most cases, be ascertained
beyond doubt, by finding surfaces of contact within the
system, or, by other means---in the case of emulsions, for
example, by determining the vapour pressure or the freezing
point. The question as to the chemical homogeneity, \ie\
the presence of one kind of molecule only, is much more
difficult, and has hitherto been answered only in special
cases. For this reason we classify substances according
to their physical and not according to their chemical
homogeneity.

\Section{93.} One characteristic of processes in non-homogeneous
systems consists in their being generally accompanied by
considerable changes of temperature, \eg\ in evaporation or
\PageSep{68}
in oxidation. To maintain the initial temperature and
pressure consequently requires considerable exchange of
heat with the surroundings and corresponding external
work. The latter, however, is generally small compared
with the external heat, and may be neglected in most
chemical processes. In thermochemistry, therefore, the
external effects,
\[
Q + W = U_{2} - U_{1},
\Tag{(45)}
\]
are generally measured in calories (the heat equivalent of
the external effects). The external work,~$W$, is small compared
with~$Q$. Furthermore, most chemical processes are
accompanied by a rise in temperature, or, if the initial
temperature be re-established, by an external yield of heat
(exothermal processes). Therefore, in thermochemistry, the
heat given out to the surroundings in order to restore the
initial temperature is denoted as the ``positive heat effect''
\index{Heat effect!in thermochemistry}%
of the process. In our equations we shall therefore use~$Q$
(the heat absorbed) with the negative sign, in processes with
positive heat effect (\eg\ combustion); with the positive
sign, in those with negative heat effect (\eg\ evaporation,
fusion, dissociation).

\Section{94.} To make equation~\Eq{(45)} suitable for thermochemistry
\index{Thermochemical symbols}%
\index{Thomsen, J.}%
it is expedient to denote the internal energy~$U$ of a system
in a given state, by a symbol denoting its chemical constitution.
J.~Thomsen introduced a symbol of this kind.
He denoted by the formulæ for the atomic or molecular
weight of the substances enclosed in brackets, the internal
energy of a corresponding weight referred to an arbitrary
zero of energy. Thus \ce{[Pb]}, \ce{[S]}, \ce{[PbS]} denote the energies
of an atom of lead, an atom of sulphur, and a molecule of
lead sulphide respectively. In order to express the fact that
\index{Lead sulphide}%
the formation of a molecule of lead sulphide from its atoms
is accompanied by a heat effect of $18,400~\Unit{cal.}$, the external
work of the process being negligible, we put
\begin{gather*}
U_{1} = \ce{[Pb] + [S]};\quad U_{2} = \ce{[PbS]}; \\
W = 0;\quad Q = -18,400~\Unit{cal.},
\end{gather*}
\PageSep{69}
and equation~\Eq{(45)} becomes
\[
-18,400~\Unit{cal.} = \ce{[PbS] - [Pb] - [S]},
\]
or, as usually written,
\[
\ce{[Pb] + [S] - [PbS]} = 18,400~\Unit{cal.}
\]
This means that the internal energy of lead and sulphur,
when separate, is $18,400$ calories greater than that of their
combination at the same temperature. That the internal
energies compared actually refer to the same material
system, can be checked by the use of the molecular formulæ.
The equation could be simplified by selecting the uncombined
state of the elements \ce{Pb} and \ce{S} as the zero of energy. Then
(\SecRef{64}), $\ce{[Pb] + [S]} = 0$, and
\[
\ce{[PbS]} = -18,400~\Unit{cal.}
\]

\Section{95.} To define accurately the state of a substance, and
\index{Aggregation, states of}%
\index{States of aggregation}%
thereby its energy, besides its chemical nature and mass, its
temperature and pressure must be given. If no special
statement is made, as in the above example, mean laboratory
temperature, \ie\ about $18°$~C., is generally assumed, and the
pressure is supposed to be atmospheric pressure. The
pressure has, however, very little influence on the internal
energy; in fact, none at all in the case of perfect gases
[equation~\Eq{(35)}].

The state of aggregation should also be indicated. This
may be done, where necessary, by using brackets for the
solids, parentheses for liquids, and braces for gases. Thus
\ce{[H2O]}, \ce{(H2O)}, \ce{\{H2O\}} denote the energies of a molecule of
ice, water, and water vapour respectively. Hence, for the
fusion of ice at~$0°$~C.,
\[
\ce{(H2O) - [H2O]} = 80 × 18 = 1440~\Unit{cal.}
\]
It is often desirable, as in the case of solid carbon, sulphur,
arsenic, or isomeric compounds, to denote by some means
the special modification of the substance.

These symbols may be treated like algebraic quantities,
whereby considerations, which would otherwise present
\PageSep{70}
considerable complications, may be materially shortened.
Examples of this are given below.

\Section{96.} To denote the energy of a solution or mixture of
\index{Energy!of a solution}%
several compounds, we may write the formulæ for the molecular
weights with the requisite number of molecules.
Thus,
\[
\ce{(H2SO4) + 5(H2O) - (H2SO4 . 5H2O)} = 13,100~\Unit{cal.}
\]
means that the solution of $1$~molecule of sulphuric acid in
$5$~molecules of water gives out $13,100$ calories of heat.
Similarly, the equation
\[
\ce{(H2SO4) + 10(H2O) - (H2SO4 . 10H2O)} = 15,100~\Unit{cal.}
\]
gives the heat effect on dissolving the same in ten molecules
\index{Heat effect!of dilution of \ce{H2SO4}}%
of water. By subtracting the first equation from the second,
we get
\[
\ce{(H2SO4 . 5H2O) + 5(H2O) - (H2SO4 . 10H2O)} = 2000~\Unit{cal.},
\]
\ie\ on diluting a solution of $1$~molecule of sulphuric acid
dissolved in $5$~molecules of water, by the addition of another
$5$~molecules of water, $2000$ calories are given out.

\Section{97.} As a matter of experience, in very dilute solutions
further dilution no longer yields any appreciable amount of
\index{Dilution!infinite}%
heat. Thus, in indicating the internal energy of a dilute
solution it is often unnecessary to give the number of molecules
of the solvent. We write briefly
\[
\ce{(H2SO4) + ($\aq$) - (H2SO4 $\aq$)} = 17,900~\Unit{cal.}
\]
to express the heat effect of infinite dilution of a molecule
\index{Infinite dilution}%
of sulphuric acid. Here ($\aq$)\ denotes any amount of water
sufficient for the practical production of an infinitely dilute
solution.

\Section{98.} Volumetric changes being very slight in chemical
processes which involve only solids and liquids, the heat
equivalent of the external work~$W$ (\SecRef{93}) is a negligible
\PageSep{71}
quantity compared with the heat effect. The latter alone,
\index{Heat effect!at constant pressure}%
then, represents the change of energy of the system\Add{:}---
\[
U_{2} - U_{1} = Q.
\]
It, therefore, depends on the initial and final states only,
and not on the intermediate steps of the process. These
considerations do not apply, in general, when gaseous substances
enter into the reaction. It is only in the combustions
in the ``calorimetric bomb,'' extensively used by Berthelot
\index{Berthelot|(}%
\index{Calorimetric bomb}%
and Stohmann in their investigations, that the volume
\index{Stohmann}%
remains constant and the external work is zero. In these
reactions the heat effect observed represents the total change
of energy. In other cases, however, the amount of external
work~$W$ may assume a considerable value, and it is
materially influenced by the process itself. Thus, a gas may
be allowed to expand, at the same time performing work,
which may have any value within certain limits, from zero
upwards. But since its change of energy $U_{2} - U_{1}$ depends
on the initial and final states only, a greater amount of work
done against the external forces necessitates a smaller heat
effect for the process, and \textit{vice versâ}. To find the latter, not
only the change of the internal energy, but also the amount
of the external work must be known. This renders necessary
an account of the external conditions under which the process
takes place.

\Section{99.} Of all the external conditions that may accompany
a chemical process, constant (atmospheric) pressure is the
one which is of the most practical importance: $p = p_{0}$. The
external work is then, according to equation~\Eq{(20)},
\[
W = -\int_{1}^{2} p_{0}\, dV = p_{0} (V_{1} - V_{2});
\Tag{(46)}
\]
that is, equal to the product of the pressure and the decrease
of volume. This, according to~\Eq{(45)}, gives
\[
U_{2} - U_{1} = Q + p_{0} (V_{1} - V_{2}).
\Tag{(47)}
\]
Now, the total decrease of volume, $V_{1} - V_{2}$, may generally
\PageSep{72}
be put equal to the decrease of volume of the gaseous
portions of the system, neglecting that of the solids and
liquids. Since, by~\Eq{(16)},
\[
V_{1} - V_{2} = R \frac{\theta}{p_{0}} (n_{1} - n_{2}),
\]
where $n_{1}$,~$n_{2}$ are the number of gas molecules present before
and after the reaction, the heat equivalent of the external
work at constant pressure is, by \Eq{(46)}~and~\Eq{(34)},
\[
\frac{W}{J}
  = \frac{p_{0} (V_{1} - V_{2})}{J}
  = \frac{R}{J} \theta (n_{1} - n_{2})
  = 1.97 \theta (n_{1} - n_{2})~\Unit{cal.}
\]
The heat effect of a process at constant pressure is
therefore
\[
-Q = U_{1} - U_{2} + 1.97 \theta (n_{1} - n_{2})~\Unit{cal.}
\Tag{(48)}
\]

If, for instance, one gram molecule of hydrogen and
half a gram molecule of oxygen, both at $18°$~C., combine
at constant pressure to form water at $18°$~C., we put
\[
U_{1} = \ce{\{H2\} + $\tfrac{1}{2}$ \{O2\}};\quad
U_{2} = \ce{(H2O)};\quad
n_{1} = \tfrac{3}{2};\
n_{2} = 0;\
\theta = 291.
\]
The heat of combustion is, therefore, by~\Eq{(48)},
\[
-Q = \ce{\{H2\} + $\tfrac{1}{2}$ \{O2\}} - \ce{(H2O)} + 860~\Unit{cal.},
\]
\ie\ $860~\Unit{cal.}$ more than would correspond to the decrease of
the internal energy, or to the combustion without the
simultaneous performance of external work.

\Section{100.} If we write equation~\Eq{(47)} in the form
\[
(U + p_{0} V)_{2} - (U + p_{0} V)_{1} = Q,
\Tag{(49)}
\]
it will be seen that, in processes under constant pressure~$p_{0}$,
the heat effect depends only on the initial and final states,
just as in the case when there is no external work. The
heat effect, however, is not equal to the difference of the
internal energies~$U$, but to the difference of the values of
the quantity $(U + p_{0} V)$ at the beginning and end of the
process. This quantity is Gibbs's ``\emph{heat function at constant
\PageSep{73}
\index{Gibbs}%
\index{Heat function at constant pressure}%
pressure}.'' If, then, only processes at constant pressure be
considered, it will be expedient to regard the symbols \ce{\{H2\}},
\ce{\{H2O\}}, etc., as representing the above function $(U + p_{0} V)$,
instead of simply the energy~$U$. Thus the difference in
the two values of the function will, in all cases, directly
represent the heat effect. This notation is therefore adopted
\index{Heat!of neutralization}%
\index{Neutralization, heat of}%
in the following.

\Section{101.} To determine the heat effect of a chemical reaction
\index{Thomsen, J.}%
at constant pressure, the initial and final values of the heat
function, $U + p_{0} V$, of the system suffice. The general
solution of this problem, therefore, amounts to finding the
heat functions of all imaginable material systems in all
possible states. Frequently, different ways of transition
from one state of a system to another may be devised,
which may serve either as a test of the theory, or as a check
upon the accuracy of the observations. Thus J.~Thomsen
found the heat of neutralization of a solution of sodium
\index{Sodium!carbonate}%
\index{Sodium!hydrate}%
bicarbonate with caustic soda to be:
\[
\ce{(NaHCO3 $\aq$) + (NaHO $\aq$) - (Na2CO3 $\aq$)} = 9200~\Unit{cal.}
\]
He also found the heat of neutralization of carbon dioxide
to be:
\[
\ce{(CO2 $\aq$) + 2(NaHO $\aq$) - (Na2CO3 $\aq$)} = 20,200~\Unit{cal.}
\]
By subtraction
\[
\ce{(CO2 $\aq$) + (NaHO $\aq$) - (NaHCO3 $\aq$)} = 11,000~\Unit{cal.}
\]
This is the heat effect corresponding to the direct combination
of carbon dioxide and caustic soda to form
sodium bicarbonate. Berthelot verified this by direct
\index{Berthelot|)}%
measurement.

\Section{102.} Frequently, of two ways of transition, one is
better adapted for calorimetric measurements than the
other. Thus, the heat effect of the decomposition of
hydrogen peroxide into water and oxygen cannot readily
be measured directly. Thomsen therefore oxidized a solution
\PageSep{74}
of stannous chloride in hydrochloric acid first by means of
hydrogen peroxide:
\index{Hydrogen peroxide}%
\[
\ce{(SnCl2 . 2HCl $\aq$) + (H2O2 $\aq$) - (SnCl4 $\aq$)} = 88,800~\Unit{cal.},
\]
then by means of oxygen gas:
\[
\ce{(SnCl2 . 2HCl $\aq$) + $\tfrac{1}{2}$ \{O2\} - (SnCl4 $\aq$)} = 65,700~\Unit{cal.}
\]
Subtraction gives
\[
\ce{(H2O2 $\aq$) - $\tfrac{1}{2}$ \{O2\} - ($\aq$)} = 23,100~\Unit{cal.}
\]
for the heat effect of the decomposition of dissolved hydrogen
peroxide into oxygen and water.

\Section{103.} The \emph{heat of formation} of carbon monoxide from
\index{Carbon, combustion of}%
\index{Combustion, of carbon}%
solid carbon and oxygen cannot be directly determined,
because carbon never burns completely to carbon monoxide,
but always, in part, to carbon dioxide as well. Therefore
Favre and Silbermann determined the heat effect of the
\index{Favre}%
\index{Silbermann}%
complete combustion of carbon to carbon dioxide:
\[
\ce{[C] + \{O2\} - \{CO2\}} = 97,000~\Unit{cal.},
\]
and then determined the heat effect of the combustion of
carbon monoxide to carbon dioxide:
\[
\ce{\{CO\} + $\tfrac{1}{2}$ \{O2\} - \{CO2\}} = 68,000~\Unit{cal.}
\]
By subtraction we get
\[
\ce{[C] + $\tfrac{1}{2}$ \{O2\} - \{CO\}} = 29,000~\Unit{cal.},
\]
the required heat of formation of carbon monoxide.

\Section{104.} According to the above, theory enables us to
calculate the heat effect of processes which cannot be
directly realized, for as soon as the heat function of a
system has been found in any way, it may be compared
with other heat functions.

Let the problem be, \eg, to find the heat of formation
of liquid carbon bisulphide from solid carbon and solid
sulphur, which do not combine directly. The following
represent the reactions:---
\PageSep{75}

The combustion of solid sulphur to sulphur dioxide gas:
\[
\ce{[S] + \{O2\} - \{SO2\}} = 71,100~\Unit{cal.}
\]
The combustion of solid carbon to carbon dioxide:
\[
\ce{[C] + \{O2\} - \{CO2\}} = 97,000~\Unit{cal.}
\]
The combustion of carbon bisulphide vapour to carbon
dioxide and sulphur dioxide:
\[
\ce{\{CS2\} + 3\{O2\} - \{CO2\} - 2\{SO2\}} = 265,100~\Unit{cal.}
\]
The condensation of carbon bisulphide vapour:
\[
\ce{\{CS2\} - (CS2)} = 6400~\Unit{cal.}
\]
Elimination by purely mathematical processes furnishes the
required heat of formation:
\index{Heat!of combustion}%
\index{Heat!of formation of \ce{CO2}, of \ce{CS2}, of \ce{CH4}}%
\[
\ce{[C] + 2[S] - (CS2)} = - 19,500~\Unit{cal.},
\]
hence negative.

In organic thermochemistry the most important method
of determining the heat of formation of a compound consists
in determining the heat of combustion, first of the compound,
and then of its constituents.

Methane (marsh gas) gives by the complete combustion
to carbon dioxide and water (liquid):
\begin{alignat*}{2}
&&\ce{\{CH4\} + 2\{O2\} - \{CO2\} - 2(H2O)} &= 211,900~\Unit{cal.}, \\
\text{but}&&
\ce{\{H2\} + $\tfrac{1}{2}$ \{O2\} - (H2O)} &= \Z68,400~\Unit{cal.},
\Tag{(50)} \\
\text{and}&&
\ce{[C] + \{O2\} - \{CO2\}} &= \Z97,000~\Unit{cal.};
\end{alignat*}
therefore, by elimination, we obtain the heat of formation
of methane from solid carbon and hydrogen gas:
\[
\ce{[C] + 2\{H2\} - \{CH4\}} = 21,900~\Unit{cal.}
\]

\Section{105.} The external heat,~$Q$, of a given change at constant
pressure will depend on the temperature at which the process
is carried out. In this respect the first law of thermodynamics
leads to the following relation:---
\PageSep{76}

From equation~\Eq{(49)} it follows that, for any two given
temperatures, $\theta$~and~$\theta'$,
\[
(U_{2} + p_{0} V_{2})_{\theta} - (U_{1} + p_{0} V_{1})_{\theta} = Q_{\theta}
\]
and
\[
(U_{2} + p_{0} V_{2})_{\theta'} - (U_{1} + p_{0} V_{1})_{\theta'} = Q_{\theta'}.
\]
Hence, by subtraction,
\begin{multline*}
Q_{\theta'} - Q_{\theta}
  = \bigl[(U_{2} + p_{0} V_{2})_{\theta'} - (U_{2} + p_{0} V_{2})_{\theta}\bigr] \\
  - \bigl[(U_{1} + p_{0} V_{1})_{\theta'} - (U_{1} + p_{0} V_{1})_{\theta}\bigr];
\end{multline*}
\ie\ the difference in the heat effects ($Q_{\theta} - Q_{\theta'}$) resulting
\index{Heat!of combustion}%
from performing the process at different temperatures, is
equal to the difference in the quantities of heat which,
before and after the reaction, would be required to raise the
temperature of the system from~$\theta$ to~$\theta'$.

Thus the influence of the temperature on the combustion
\index{Combustion, of carbon!influence of temperature on}%
\index{Influence!of temperature on combustion}%
of hydrogen to water (liquid) may be found by comparing
the heat capacity of the mixture (\ce{H2 + O2}) with that of
the water (\ce{H2O}). The former is equal to the molecular
heat of hydrogen plus half the molecular heat of oxygen.
According to the table in~\SecRef{87}, this is
\[
6.82 + 3.47 = 10.29.
\]
The latter is
\[
1 × 18 = 18.
\]
The difference between these values is~$-7.71$, and, therefore,
the heat of combustion of a gram molecule of hydrogen
decreases with rising temperature by $7.7~\Unit{cal.}$ per degree
Centigrade.
\PageSep{77}


\Part{III.}{The Second Fundamental Principle of~Thermodynamics.}

\Chapter{I.}{Introduction.}

\Section{106.} \First{The} second law of thermodynamics is essentially
\index{Second law of thermodynamics!introduction}%
different from the first law, since it deals with a question in
no way touched upon by the first law, viz.\ the direction in
which a process takes place in nature. Not every change
which is consistent with the principle of the conservation of
energy satisfies also the additional conditions which the
second law imposes upon the processes, which actually take
place in nature. In other words, the principle of the conservation
of energy does not suffice for a unique determination
of natural processes.

If, for instance, an exchange of heat by conduction takes
place between two bodies of different temperature, the first
law, or the principle of the conservation of energy, merely
demands that the quantity of heat given out by the one body
shall be equal to that taken up by the other. Whether the
flow of heat, however, takes place from the colder to the
hotter body, or \textit{vice versâ}, cannot be answered by the energy
principle alone. The very notion of temperature is alien to
that principle, as can be seen from the fact that it yields no
exact definition of temperature. Neither does the general
equation~\Eq{(17)} of the first law contain any statement with
\PageSep{78}
regard to the direction of the particular process. The
special equation~\Eq{(50)}, for instance,
\[
\ce{\{H2\} + $\tfrac{1}{2}$ \{O2\} - (H2O)} = 68,400~\Unit{cal.},
\]
means only that, if hydrogen and oxygen combine under
constant pressure to form water, the \Erratum{restablishment}{re-establishment} of the
initial temperature requires a certain amount of heat to be
given up to surrounding bodies; and \textit{vice versâ}, that this
amount of heat is absorbed when water is decomposed into
hydrogen and oxygen. It offers no information, however,
as to whether hydrogen and oxygen actually combine to
form water, or water decomposes into hydrogen and oxygen,
or whether such a process can take place at all in either
direction. From the point of view of the first law, the
initial and final \Erratum{stakes}{states} of any process are completely
equivalent.

\Section{107.} In one particular case, however, does the principle
of the conservation of energy prescribe a certain direction
to a process. This occurs when, in a system, one of the
various forms of energy is at an absolute maximum (or
minimum). It is evident that, in this case, the direction of
the change must be such that the particular form of energy
will decrease (or increase). This particular case is realized
in mechanics by a system of particles at rest. Here the
kinetic energy is at an absolute minimum, and, therefore,
any change of the system is accompanied by an increase of
the kinetic energy, and, if it be an isolated system, by a
decrease of the potential energy. This gives rise to an
important proposition in mechanics, which characterizes the
direction of possible motion, and lays down, in consequence,
the general condition of mechanical equilibrium. It is
evident that, if both the kinetic and potential energies be
at a minimum, no change can possibly take place, since
none of these can increase at the expense of the other.
The system must, therefore, remain at rest.

If a heavy liquid be initially at rest at different levels in
two communicating tubes, then motion will set in, so as to
\PageSep{79}
equalize the levels, for the centre of gravity of the system
is thereby lowered, and the potential energy diminished.
Equilibrium exists when the centre of gravity is at its
lowest, and therefore the potential energy at a minimum,
\ie\ when the liquid stands at the same level in both tubes.
If no special assumption be made with regard to the initial
velocity of the liquid, the above proposition no longer holds.
The potential energy need not decrease, and the higher level
might rise or sink according to circumstances.

If our knowledge of thermal phenomena led us to
recognize a state of minimum energy, a similar proposition
would hold for this, but only for this, particular state. In
reality no such minimum has been detected. It is, therefore,
hopeless to seek to reduce the general laws regarding
the direction of thermodynamical changes, as well as those
of thermodynamical equilibrium, to the corresponding propositions
in mechanics which hold good only for systems at
rest.

\Section{108.} Although these considerations make it evident
that the principle of the conservation of energy cannot
serve to determine the direction of a thermodynamical
process, and therewith the conditions of thermodynamical
equilibrium, unceasing attempts have been made to make
the principle of the conservation of energy in some way or
other serve this purpose. These attempts have, in many
encases, stood in the way of a clear presentation of the second
law. \Erratum{Occasionally we still find the endeavour made to
represent this law as contained in the energy principle, in
that the doubtless too restricted term of ``energetics'' is
\index{Energetics}%
applied to all investigations on these questions.}{That attempts
are still made to represent this law as contained in the principle
of energy may be seen from the fact that the too restricted term
\index{Energetics}%
``energetics'' is sometimes applied to all investigations on these
questions.}
The conception
of energy is not sufficient for the second law. It
cannot be exhaustively treated by breaking up a natural
process into a series of changes of energy, and then investigating
the direction of each change. We can always
tell, it is true, what are the different kinds of energy
exchanged for one another; for there is no doubt that the
principle of energy must be fulfilled, but the expression of
\PageSep{80}
the conditions of these changes remains arbitrary, and
this ambiguity cannot be completely removed by any
general assumption.

We often find the second law stated as follows: The
change of mechanical work into heat may be complete, but,
on the contrary, that of heat into work must needs be
incomplete, since, whenever a certain quantity of heat is
transformed into work, another quantity of heat must
undergo a corresponding and compensating change; \eg\
transference from higher to lower temperature. This is
quite correct in certain very special cases, but it by no
means expresses the essential feature of the process, as a
simple example will show. An achievement which is
closely associated with the discovery of the principle of
energy, and which is one of the most important for the
theory of heat, is the proposition expressed in equation~\Eq{(19)},
\SecRef{70}, that the total internal energy of a gas depends only
on the temperature, and not on the volume. If a perfect
gas be allowed to expand, doing external work, and be prevented
from cooling by connecting it with a heat-reservoir
of higher temperature, the temperature of the gas, and at
the same time its internal energy, remains unchanged, and
it may be said that the amount of heat given out by the
reservoir is completely changed into work without an exchange
of energy taking place anywhere. Not the least
objection can be made to this. The proposition of the
``incomplete transformability of heat into work'' cannot
\index{Transformability of heat into work}%
be applied to this case, except by a different way of viewing
the process, which, however, changes nothing in the physical
facts, and cannot, therefore, be confirmed or refuted by
them, namely, by the introduction of new kinds of energy,
only invented \textit{ad~hoc}. This consists in dividing the
energy of the gas into several parts, which may then
individually depend also on the volume. This division has,
however, to be carried out differently for different cases (\eg,
in one way for isothermal, in another for adiabatic processes),
and necessitates complicated considerations even in
cases of physical simplicity. But when we pass from the
\PageSep{81}
consideration of the first law of thermodynamics to that of
the second, we have to deal with a new fact, and it is evident
that no definition, however ingenious, although it contain
no contradiction in itself, will ever permit of the deduction
of a new fact.

\Section{109.} There is but one way of clearly showing the significance
of the second law, and that is to base it on facts
by formulating propositions which may be proved or disproved
by experiment. The following proposition is of this
character: It is in no way possible to completely reverse
any process in which heat has been produced by friction.
For the sake of example we shall refer to Joule's experiments
on friction, described in~\SecRef{60}, for the determination
of the mechanical equivalent of heat. Applied to these,
our proposition says that, when the falling weights have
generated heat in water or mercury by the friction of the
paddles, no process can be invented which will completely
restore everywhere the initial state of that experiment, \ie\
which will raise the weights to their original height, cool
the liquid, and otherwise leave no change. The appliances
used may be of any kind whatsoever, mechanical, thermal,
chemical, electrical, etc., but the condition of \emph{complete}
restoration of the initial state renders it necessary that all
materials and machines used must ultimately be left exactly
in the condition in which they were before their application.
Such a proposition cannot be proved \textit{a~priori}, neither does
it amount to a definition, but it contains a definite assertion,
to be stated precisely in each case, which may be
verified by actual experiment. The proposition is therefore
correct or incorrect.

\Section{110.} Another proposition of this kind, and closely
connected with the former, is the following: It is in no
way possible to completely reverse any process in which a
gas expands without performing work or absorbing heat,
\ie\ with constant total energy (as described in~\SecRef{68}). The
word ``completely'' again refers to the accurate reproduction
of the initial conditions. To test this, the gas, after
\PageSep{82}
it had assumed its new state of equilibrium, might first be
compressed to its former volume by a weight falling to a
lower level. External work is done on the gas, and it is
thereby heated. The problem is now to bring the gas to
its initial condition, and to raise the weight. The gas might
be reduced to its original temperature by conducting the
heat of compression into a colder heat-reservoir. In order
that the process may be completely reversed, the reservoir
must be deprived of the heat gained thereby, and the weight
raised to its original position. This is, however, exactly
what was asserted in the preceding paragraph to be impracticable.

\Section{111.} A third proposition in point refers to the conduction
of heat. Supposing that a body receives a certain
quantity of heat from another of higher temperature, the
problem is to completely reverse this process, \ie\ to convey
back the heat without leaving any change whatsoever. In
the description of Carnot's reversible cycle it has been
pointed out, that heat can at any time be drawn from a
heat-reservoir and transferred to a hotter reservoir without
leaving any change except the expenditure of a certain
amount of work, and the transference of an equivalent
amount of heat from one reservoir to the other. If this
heat could be removed, and the corresponding work recovered
without other changes, the process of heat-conduction
would be completely reversed. Here, again, we have
the problem which was declared in~\SecRef{109} to be impracticable.

Further examples of processes to which the same considerations
\index{Processes!reversible and irreversible}%
apply are, diffusion, the freezing of an overcooled
liquid, the condensation of a supersaturated vapour, all
explosive reactions, and, in fact, every transformation of a
system into a state of greater stability.

\Section{112.} A process which can in no way be completely
reversed is termed \emph{irreversible}, all other processes \emph{reversible}.
That a process may be irreversible, it is not
sufficient that it cannot be directly reversed. This is the
\PageSep{83}
case with many mechanical processes which are not irreversible
\index{Processes!periodic}%
(\cf~\SecRef{113}). The full requirement is, that it be
impossible, even with the assistance of all agents in nature,
to restore everywhere the exact initial state when the
process has once taken place. The propositions of the three
preceding paragraphs, therefore, declare, that the generation
of heat by friction, the expansion of a gas without the performance
of external work and the absorption of external
heat, the conduction of heat, etc., are irreversible processes.

\Section{113.} We now turn to the question of the actual
existence of reversible and irreversible processes. Numerous
reversible processes can at least be imagined, as, for instance,
those consisting of a succession of states of equilibrium, as
fully explained in~\SecRef{71}, and, therefore, directly reversible in
all their parts. Further, all perfectly periodic processes,
\eg\ an ideal pendulum or planetary motion, are reversible,
for, at the end of every period, the initial state is completely
restored. Also, all mechanical processes with absolutely
rigid bodies and absolutely incompressible liquids, as far as
friction can be avoided, are reversible. By the introduction
of suitable machines with absolutely unyielding connecting
rods, frictionless joints and bearings, inextensible belts, etc.,
it is always possible to work the machines in such a way as
to bring the system completely into its initial state without
leaving any change in the machines, for the machines of
themselves do not perform work.

If, for instance, a heavy liquid, originally at rest at
different levels in two communicating tubes (\SecRef{107}), be set
in motion by gravity, it will, in consequence of its kinetic
energy, go beyond its position of equilibrium, and, since
the tubes are supposed frictionless, again swing back to its
exact original position. The process at this point has been
completely reversed, and therefore belongs to the class of
reversible processes. As soon as friction is admitted, however,
its reversibility is at least questionable. Whether
reversible processes exist in nature or not, is not \textit{a~priori}
evident or demonstrable. There is, however, no purely
\PageSep{84}
logical objection to imagining that a means may some day
be found of completely reversing some process hitherto considered
irreversible: one, for example, in which friction or
heat-conduction plays a part. But it can be demonstrated---and
this will be done in the following chapter---that if, in
a single instance, one of the processes declared to be irreversible
in \SSecRef{109},~etc., should be found to be reversible,
then all of these processes must be reversible in all cases.
Consequently, either all or none of these processes are
irreversible. There is no third possibility. If those processes
are not irreversible, the entire edifice of the second
law will crumble. None of the numerous relations deduced
from it, however many may have been verified by experience,
could then be considered as universally proved, and theoretical
work would have to start from the beginning. (The
so-called proofs of ``energetics'' are not a substitute, for a
\index{Energetics}%
closer test shows all of them to be more or less imperfect
paraphrases of the propositions to be proved. This is not
the place, however, to demonstrate this point.) It is this
foundation on the physical fact of irreversibility which forms
the strength of the second law. If, therefore, it must be
admitted that a single experience contradicting that fact
would render the law untenable, on the other hand, any
confirmation of part supports the whole structure, and gives
to deductions, even in seemingly remote regions, the full
significance possessed by the law itself.

\Section{114.} Since the decision as to whether a particular
process is irreversible or reversible depends only on whether
the process can in any manner whatsoever be completely
reversed or not, the nature of the initial and final states,
and not the intermediate steps of the process, entirely settle
it. The question is, whether or not it is possible, starting
from the final state, to reach the initial one in any way without
any other change. The second law, therefore, furnishes
a relation between the quantities connected with the initial
and final states of any natural process. The final state of
an irreversible process is evidently in some way discriminate
\PageSep{85}
from the initial state, while in reversible processes the two
states are in certain respects equivalent. The second law
points out this characteristic property of both states, and
also shows, when the two states are given, whether a transformation
is possible in nature from the first to the second,
or from the second to the first, without leaving changes in
other bodies. For this purpose, of course, the two states
must be fully characterized. Besides the chemical constitution
of the systems in question, the physical conditions---viz.\
the state of aggregation, temperature, and pressure in
both states---must be known, as is necessary for the application
of the first law.

The relation furnished by the second law will evidently
be simpler the nearer the two states are to one another.
On this depends the great fertility of the second law in its
treatment of cyclic processes, which, however complicated
they may be, give rise to a final state only slightly different
from the initial state~(\SecRef{91}).

\Section{115.} Since there exists in nature no process entirely
free from friction or heat-conduction, all processes which
actually take place in nature, if the second law be correct,
are in reality irreversible; reversible processes form only an
ideal limiting case. They are, however, of considerable
importance for theoretical demonstration and for application
to states of equilibrium.
\PageSep{86}


\Chapter{II.}{Proof.}
\index{Second law of thermodynamics!proof}%

\Section{116.} \First{The} second fundamental principle of thermodynamics
being, like the first, an empirical law, we can
speak of its proof only in so far as its total purport may be
deduced from a single self-evident proposition. We, therefore,
put forward the following proposition as being given
directly by experience: \emph{It is impossible to construct an engine
which will work in a complete cycle, and produce no effect except
the raising of a weight and the cooling of a heat-reservoir.}
Such an engine could be used simultaneously as a motor
and a refrigerator without any waste of energy or material,
and would in any case be the most profitable engine ever
made. It would, it is true, not be equivalent to perpetual
motion, for it does not produce work from nothing, but from
the heat, which it draws from the reservoir. It would not,
therefore, like perpetual motion, contradict the principle of
energy, but would, nevertheless, possess for man the essential
advantage of perpetual motion, the supply of work without
cost; for the inexhaustible supply of heat in the earth, in
the atmosphere, and in the sea, would, like the oxygen
of the atmosphere, be at everybody's immediate disposal.
For this reason we take the above proposition as our starting
point. Since we are to deduce the second law from it, we
expect, at the same time, to make a most serviceable application
of any natural phenomenon which may be discovered
to deviate from the second law. As soon as a phenomenon
is found to contradict any legitimate conclusions from the
second law, this contradiction must arise from an inaccuracy
in our first assumption, and the phenomenon could be used
\PageSep{87}
for the construction of the above-described engine. We
shall in the following, according to the proposal of Ostwald,
speak of perpetual motion of the second kind, since it stands
in the same relation to the second law as perpetual motion
of the first kind does to the first law. In connection with all
objections to the second law, it must be borne in mind that,
if no errors are to be found in the line of proof, they are
ultimately directed against the impossibility of perpetual
motion of the second kind (\SecRef{136}).\footnote
  {I desire to emphasize here, that the starting point selected by me for
  the proof of the second law coincides fundamentally with that which R.~Clausius,
\index{Clausius}%
  or which Sir W.~Thomson, or which J.~Clerk Maxwell used for the
\index{Maxwell}%
\index{Thomson}%
  same purpose. The fundamental proposition which each of these investigators
  placed at the beginning of his deductions asserts each time, only in
  different form, the impossibility of the realization of perpetual motion of the
  second kind. I have selected the above form of expression, because of its
  apparent technical significance. Not a single really rational proof of the
  second law has thus far been advanced which does not require this fundamental
  principle, however numerous the attempts in this direction may
  have been in recent times, nor do I believe that such an attempt will ever
  meet with success.}

\Section{117.} From the impossibility of perpetual motion of
the second kind, it follows, in the first place, that the
generation of heat by friction is \emph{irreversible} (\cf\ def.~\SecRef{112}).
For supposing it were not so, \ie\ supposing a method could
be found by which a process involving generation of heat
by friction could be completely reversed, this very method
would produce what is identically perpetual motion of the
second kind: viz.\ a change which consists of nothing but
the production of work, and the absorption of an equivalent
amount of heat.

\Section{118.} It follows, further, that the expansion of a gas
without the performance of external \Erratum{heat}{work}, or the absorption
of heat, is irreversible. For, suppose a method were known
of completely reversing this process, \ie\ of reducing the
volume of a gas, without leaving any other change whatsoever,
this method could be utilized for the production of
perpetual motion of the second kind in the following manner.
Allow the gas to do work by expansion, supplying the energy
\PageSep{88}
lost thereby by the conduction of heat from a reservoir at
the same or higher temperature, and then, by the assumed
method, reduce the volume of the gas to its initial value
without leaving any other change. This process might be
repeated as often as we please, and would therefore represent
an engine working in a complete cycle, and producing no
effect except the performance of work, and the withdrawal
of heat from a reservoir, \ie\ perpetual motion of the second
kind.

On the basis of the proposition we have just proved, that
the expansion of a gas without the performance of work
and the absorption of heat is irreversible, we shall now carry
through the proof of the second law for those bodies whose
thermodynamical properties are most completely known,
viz.\ for perfect gases.

\Section{119.} If a perfect gas be subjected to infinitely slow
compression or expansion, and if, at the same time, heat be
applied or withdrawn, we have, by equation~\Eq{(22)}, in each
infinitely small portion of the process, per unit mass,
\[
q = du + p\, dv
\]
or, since for a perfect gas,
\[
du = c_{v}\, d\theta,
\]
and
\begin{align*}
p &= \frac{R}{m} · \frac{\theta}{v}, \\
q &= c_{v}\, d\theta + \frac{R}{m} · \frac{\theta}{v}\, dv.
\end{align*}

If the process be adiabatic, then $q = 0$, and the integration
of the above equation gives (as in~\SecRef{88}) the function
\[
c_{v} \log \theta + \frac{R}{m} \log v
\]
equal to a constant. We shall now put
\[
\phi = c_{v} \log \theta + \frac{R}{m} \log v + \const,
\Tag{(51)}
\]
\PageSep{89}
and call this function, after Clausius, the \emph{entropy} of unit mass
\index{Entropy!of a gas}%
of the gas. The constant, which has to be added, can be
determined by arbitrarily fixing the zero state. Accordingly
\[
\Phi = M\phi = M \left(c_{v} \log \theta + \frac{R}{m} \log v + \const\right)
\Tag{(52)}
\]
is the entropy of mass~$M$ of the gas. The entropy of the
gas, therefore, remains constant during the described
adiabatic change of state.

\Section{120.} On the application of heat, the entropy of the
gas changes, in the case considered, by
\[
d\Phi
  = M \left(c_{v}\, \frac{d\theta}{\theta} + \frac{R}{m}\, \frac{dv}{v}\right)
  = \frac{M · q}{\theta} = \frac{Q}{\theta}.
\Tag{(53)}
\]
It increases or decreases according as heat is absorbed or
evolved.

The absorbed heat~$Q$ has here been broken up into two
factors, $\theta$~and~$d\Phi$. According to a view which has recently
been brought forward, this breaking up of heat into factors is
regarded as a general property of heat. It should, however,
be emphasized that equation~\Eq{(53)} is by no means generally
true. It holds only in the particular case where the external
work performed by the gas is expressed by~$p\, dV$. The
relation
\[
d\Phi
  = M \left(c_{v}\, \frac{d\theta}{\theta} + \frac{R}{m}\, \frac{dv}{v}\right)
  = \frac{dU + p\, dV}{\theta}
\]
holds, quite generally, for any process in which the temperature
of the gas is increased by~$d\theta$, and the volume by~$dV$.
It is, in fact, only a different mathematical form for the
definition of the entropy given in~\Eq{(52)}. On the other
hand, the equation
\[
Q = dU + p\, dV
\]
holds by no means in all cases, but should, in general, be
replaced by
\[
Q + W = dU,
\]
\PageSep{90}
where $W$, the work done on the substance, may have any
value within certain limits. For instance, $W = 0$, if the
gas be conveyed into its new state of equilibrium without
performing external work (as described in~\SecRef{68}). In this
case, $Q = dU$, and the equation $Q = \theta\, d\Phi$ no longer holds.

\Section{121.} We shall now consider two gases which can communicate
heat to one another by conduction, but may, in
general, be under different pressures. If the volume of
one, or both, of the gases be changed by some reversible
process, care being taken that the temperatures of the gases
equalize at each moment, and that no exchange of heat
takes place with surrounding bodies, we have, according
to equation~\Eq{(53)}, during any element of time, for the first
gas,
\[
d\Phi_{1} = \frac{Q_{1}}{\theta_{1}},
\]
and, for the second gas,
\[
d\Phi_{2} = \frac{Q_{2}}{\theta_{2}}.
\]
According to the conditions of the process,
\[
\theta_{1} = \theta_{2}\quad\text{and}\quad Q_{1} + Q_{2} = 0,
\]
whence,
\[
d\Phi_{1} + d\Phi_{2} = 0
\]
or, for a finite change,
\[
\Phi_{1} + \Phi_{2} = \const
\Tag{(54)}
\]
The sum of the entropies of the two gases remains constant
during the described process.

\Section{122.} Any such process with two gases is evidently
reversible in all its parts, for it may be directly reversed
without leaving changes in the surroundings. From this
follows the proposition that it is always possible to bring
two gases, by a reversible process, without leaving changes
\PageSep{91}
\index{Condition of complete reversibility!of a process}%
in\footnote
  {The emphasis is to be put on the word ``in.'' Changes of position of
  ponderable bodies (for example, the raising or lowering of weights) are not
  internal changes; but, of course, temperature and density changes are.}
other bodies, from any given state to any other given
state, if the sum of the entropies in the two states be equal.

Let an initial state of the gases be given by the temperatures
$\theta_{1}$,~$\theta_{2}$, and the specific volumes $v_{1}$,~$v_{2}$; a second state
by the corresponding values $\theta_{1}'$,~$\theta_{2}'$; $v_{1}'$,~$v_{2}'$. We now
suppose that
\[
\Phi_{1} + \Phi_{2} = \Phi_{1}' + \Phi_{2}'\Add{.}
\Tag{(55)}
\]
Bring the first gas to the temperature~$\theta_{2}$ by a reversible
adiabatic compression or expansion; then place the two gases
in thermal contact with one another, and continue to compress
or expand the first infinitely slowly. Heat will now pass
between the two gases, and the entropy of the first one will
change, and it will be possible to make this entropy assume
the value~$\Phi_{1}'$. But, according to~\Eq{(54)}, during the above
process the sum of the two entropies remains constant, and
$= \Phi_{1} + \Phi_{2}$; therefore the entropy of the second gas is
$(\Phi_{1} + \Phi_{2}) - \Phi_{1}'$, which is, according to~\Eq{(55)}, equal to~$\Phi_{2}'$. If
we now separate the two gases, and compress or expand
each one adiabatically and reversibly until they have the
required temperatures $\theta_{1}'$ and~$\theta_{2}'$, the specific volumes must
then be $v_{1}'$ and~$v_{2}'$, and the required final state has been
reached.

This process is reversible in all its parts, and no changes
remain in other bodies; in particular, the surroundings
have neither gained nor lost heat. The conditions of the
problem have therefore been fulfilled, and the proposition
proved.

\Section{123.} A similar proposition can readily be proved for
any number of gases. It is always possible to bring a
system of $n$~gases from any one state to any other by
a reversible process without leaving changes in other bodies,
if the sum of the entropies of all the gases is the same in
both states, \ie\ if
\[
\Phi_{1} + \Phi_{2} + \dots + \Phi_{n} = \Phi_{1}' + \Phi_{2}' + \dots + \Phi_{n}'.
\Tag{(56)}
\]
\PageSep{92}
\index{Zero!state}%
By the process described in the preceding paragraph we
may, by the successive combination of pairs of gases of the
system, bring the first, then the second, then the third, and
so on to the $(n - 1)$th~gas, to the required entropy. Now,
\index{Entropy!of a system of gases}%
in each of the successive processes the sum of the entropies
of all the gases remains constant, and, since the entropies of
the first $(n - 1)$ gases are $\Phi_{1}'$, $\Phi_{2}'$\Add{,}~\dots\Add{,} $\Phi_{n-1}'$, the entropy of
the $n$th~gas is necessarily
\[
(\Phi_{1} + \Phi_{2} + \dots + \Phi_{n}) - (\Phi_{1}' + \Phi_{2}' + \dots + \Phi_{n-1}').
\]
This is, according to~\Eq{(56)}, the required value~$\Phi_{n}'$. Each gas
can now be brought by an adiabatic reversible process into
the required state, and the problem is solved.

If we call the sum of the entropies of all the gases the
entropy of the whole system, we may then say: \emph{If a system
of gases has the same entropy in two different states it may
be transformed from the one to the other by a reversible process,
without leaving changes in other bodies.}

\Section{124.} We now introduce the proposition proved in~\SecRef{118},
that the expansion of a perfect gas, without performing
external work or absorbing heat, is irreversible; or,
what is the same thing, that the transition of a perfect gas
to a state of greater volume and equal temperature, without
external effects, as described in~\SecRef{68}, is irreversible. Such
a process corresponds to an increase of the entropy, according
to the definition~\Eq{(52)}. It immediately follows that it
is altogether impossible to decrease the entropy of a gas
without producing a change in surrounding objects. If this
were possible, the irreversible expansion of a gas could be
completely reversed. After the gas had expanded without
external effects, and had assumed its new state of equilibrium,
the entropy of the gas could be reduced to its initial
value, without leaving changes in other bodies, by the
supposed method, and then, by an adiabatic reversible
process, brought to its initial temperature, and thereby also
to its original volume. This would completely reverse the
\PageSep{93}
first expansion, and furnish, according to~\SecRef{118}, perpetual
motion of the second kind.

\Section{125.} A system of two or more gases behaves in the
same way. There exists, in nature, no means of diminishing
the entropy of a system of perfect gases, without leaving
\index{Entropy!diminution of}%
changes in bodies outside the system. A contrivance
which would accomplish this, be it mechanical, thermal,
chemical, or electrical in nature, might be used to reduce
the entropy of a single gas without leaving changes in
other bodies.

Suppose a system of gases to have passed in any manner
from one state in which their entropies are $\Phi_{1}$, $\Phi_{2}$\Add{,}~\dots\Add{,} $\Phi_{n}$,
to a state where they are $\Phi_{1}'$, $\Phi_{2}'$\Add{,}~\dots\Add{,} $\Phi_{n}'$, and that no
change has been produced in any body outside the system,
and let
\[
\Phi_{1}' + \Phi_{2}' + \dots + \Phi_{n}' < \Phi_{1} + \Phi_{2} + \dots + \Phi_{n},
\Tag{(57)}
\]
then it is possible, according to the proposition proved in~\SecRef{123},
to bring the system by a reversible process, without
leaving changes in other bodies, into any other state in
which the sum of the entropies is
\[
\Phi_{1}' + \Phi_{2}' + \dots + \Phi_{n}',
\]
and accordingly into a state in which the first gas has
the entropy~$\Phi_{1}$, the second the entropy~$\Phi_{2}$\Add{,}~\dots, the
$(n - 1)$th the entropy~$\Phi_{n-1}$, and the $n$th in consequence the
entropy
\[
(\Phi_{1}' + \Phi_{2}' + \dots + \Phi_{n}') - \Phi_{1} - \Phi_{2} - \dots - \Phi_{n-1}\Add{.}
\Tag{(58)}
\]

The first $(n - 1)$ gases may now be reduced to their
original state by reversible adiabatic processes. The $n$th~gas
possesses the entropy~\Eq{(58)}, which is, according to the supposition~\Eq{(57)},
smaller than the original entropy~$\Phi_{n}$. The entropy
of the $n$th~gas has, therefore, been diminished without
leaving changes in other bodies. This we have already
proved in the preceding paragraph to be impossible.
\PageSep{94}

The general proposition has, therefore, been proved, and
we may immediately add the following.

\Section{126.} \emph{If a system of perfect gases pass in any way from
\index{Reversibility of a process!complete!condition of}%
one state to another, and no changes remain in surrounding
bodies, the entropy of the system is certainly not smaller,
but either greater than, or, in the limit, equal to that of the
initial state; in other words, the total change of the entropy
$\geq 0$.} The sign of inequality corresponds to an irreversible
process, the sign of equality to a reversible one. The
equality of the entropies in both states is, therefore, not
only a sufficient, as described in~\SecRef{123}, but also a necessary
condition of the complete reversibility of the transformation
from the one state to the other, provided no changes are to
remain in other bodies.

\Section{127.} The scope of this proposition is considerable,
since there have designedly been imposed no restrictions
regarding the way in which the system passes from its
initial to its final state. The proposition, therefore, holds
not only for slow and simple processes, but also for physical
and chemical ones of any degree of complication, provided
that at the end of the process no changes remain in any
body outside the system. It must not be supposed that
the entropy of a gas has a meaning only for states of
equilibrium. We may assume each sufficiently small
particle, even of a gas in turmoil, to be homogeneous and
at a definite temperature, and must, therefore, according to~\Eq{(52)},
assign to it a definite value of the entropy. $M$,~$v$,
and~$\theta$ are then the mass, specific volume, and temperature
of the particle under consideration. A summation extending
over all the particles of the mass---within which the values
of $v$~and~$\theta$ may vary from particle to particle---gives the
entropy of the whole mass of the gas in the particular state.
The proposition still holds, that the entropy of the whole
gas must continually increase during any process which
does not give rise to changes in other bodies, \eg\ when a
gas flows from a vessel into a vacuum (\SecRef{68}). It will be
seen that the velocity of the gas particles does not influence
\PageSep{95}
the value of the entropy; neither does their height above
a certain horizontal plane, although they are considered to
have weight.

\Section{128.} The laws which we have deduced for perfect gases
may be transferred to any substance in exactly the same
way. The main difference is, that the expression for the
entropy of any body cannot, in general, be written down
in finite quantities, since the characteristic equation is not
generally known. But it can be demonstrated---and this is
the deciding point---that, for any other body, there exists a
function with the characteristic properties of the entropy.

Imagine any homogeneous body to pass through a certain
reversible or irreversible cycle and to be brought back
to its exact original state, and let the external effects of
this process consist in the performance of work and in the
addition or withdrawal of heat. The latter may be brought
about by means of any required number of suitable heat-reservoirs.
After the process, no changes remain in the
substance itself; the heat-reservoirs alone have suffered
change. We shall now assume all the heat-reservoirs to be
perfect gases, kept either at constant volume or at constant
pressure, but, at any rate, subject only to reversible changes
of volume. According to our last proposition, the sum of
the entropies of all these gases cannot have decreased, since
after the process no change remains in any other body.

If $Q$~denote the amount of heat given to the substance
during an infinitely small element of time by one of the
reservoirs; $\theta$,~the temperature of the reservoir at that
moment; then, according to equation~\Eq{(53)}, the reservoir's
change of entropy during that element of time is
\[
-\frac{Q}{\theta}.
\]
The change of the entropy of all the reservoirs, during all
the elements of time considered, is
\[
-\tsum \frac{Q}{\theta}.
\]
\PageSep{96}
\index{Clausius' equation!form of second law}%
Now, according to~\SecRef{126}, we have the following condition:---
\[
-\tsum \frac{Q}{\theta} \geq 0
\]
or
\[
\tsum \frac{Q}{\theta} \leq 0.
\]
This is the form in which the second law was first enunciated
by Clausius.

A further condition is given by the first law; for, according
to~\Eq{(17)} in~\SecRef{63}, we have, during every element of
time of the process,
\[
Q + W = dU,
\]
$U$~being the initial energy of the body, and $W$~the work
done on the body during the element of time.

\Section{129.} If we now make the special assumption that the
external pressure is, at any moment, equal to the pressure~$p$
of the substance, the work of compression becomes,
according to~\Eq{(20)},
\[
W = -p\, dV,
\]
whence
\[
Q = dU + p\, dV.
\]

If, further, each heat-reservoir be exactly at the temperature
of the substance at the moment when brought into
operation, the cyclic process is reversible, and the inequality
of the second law becomes an equality\Add{:}---
\[
\tsum \frac{Q}{\theta} = 0,
\]
or, on substituting the value of~$Q$,
\[
\tsum \frac{dU + p\, dV}{\theta} = 0.
\]

All the quantities in this equation refer to the state of
the substance itself. It admits of interpretation without
reference to the heat-reservoirs, and amounts to the following
proposition.
\PageSep{97}

\Section{130.} \emph{If a homogeneous body be taken through a series of
states of equilibrium {\upshape(\SecRef{71})}, that follow continuously from one
another, back to its initial state, then the summation of the
differential
\[
\frac{dU + p\, dV}{\theta}
\]
extending over all the states of that process gives the value
zero.} It follows that, if the process be not continued until
the initial state,~$1$, is again reached, but be stopped at a
certain state,~$2$, the value of the summation
\[
\int_{1}^{2} \frac{dU + p\, dV}{\theta}
\Tag{(59)}
\]
depends only on the states $1$~and~$2$, not on the manner
of the transformation from state~$1$ to state~$2$. If two
series of changes leading from~$1$ to~$2$ be considered
(\eg\ curves $\alpha$~and~$\beta$ in \Fig{2}, \SecRef{75}), these can be combined
into an infinitely slow cyclic process. We may,
for example, go from~$1$ to~$2$ along~$\alpha$, and return to~$1$
along~$\beta$.

It has been demonstrated that over the entire cycle\Add{:}---
\[
\int_{1\; (\alpha)}^{2} \frac{dU + p\, dV}{\theta}
  + \int_{2\; (\beta)}^{1} \frac{dU + p\, dV}{\theta} = 0,
\]
whence
\[
\int_{1\; (\alpha)}^{2} \frac{dU + p\, dV}{\theta}
  = \int_{1\; (\beta)}^{2} \frac{dU + p\, dV}{\theta}.
\]

The integral~\Eq{(59)} with the above-proved properties has
been called by Clausius the \emph{entropy} of the body in state~$2$,
\index{Entropy!definition}%
referred to state~$1$ as the zero state. The entropy of a
body in a given state, like the internal energy, is completely
determined up to an additive constant, whose value depends
on the zero state.
\PageSep{98}

Denoting the entropy, as formerly, by~$\Phi$, we have:
\[
\Phi = \int \frac{dU + p\, dV}{\theta}
\]
and
\[
d\Phi = \frac{dU + p\, dV}{\theta}
\Tag{(60)}
\]
or per unit mass:
\[
d\phi = \frac{du + p\, dv}{\theta}\Add{.}
\Tag{(61)}
\]

This, again, leads to the value~\Eq{(51)} for a perfect gas.
The expression for the entropy of any body may be found
by immediate integration (\SecRef{254}), provided its energy,
$U = Mu$, and its volume, $V = Mv$, are known as functions,
say, of $\theta$~and~$p$. Since, however, these are not completely
known except for perfect gases, we have to content ourselves
in general with the differential equation. For the proof,
and for many applications of the second law, it is, however,
sufficient to know that this differential equation contains
in reality a unique definition of the entropy.

\Section{131.} We may, therefore, just as in the case of perfect
gases, speak of the entropy of any substance as of a finite
quantity determined by the momentary values of temperature
and volume, even when the substance undergoes
reversible or irreversible changes. The differential equation~\Eq{(61)}
holds, as was stated in~\SecRef{120} in the case of perfect
gases, for any change of state, including irreversible changes.
This more general application of the conception of the
entropy in no wise contradicts the manner of its deduction.
The entropy of a body in any given state is measured by
means of a reversible process which brings the body from
that state to the zero state. This ideal process, however,
has nothing to do with any actual reversible or irreversible
changes which the body may have undergone or be about
to undergo.

On the other hand, it should be stated that the differential
equation~\Eq{(60)}, while it holds for changes of volume
and temperature, does not apply to changes of mass, for
\PageSep{99}
this kind of change was in no way referred to in the
definition of the entropy.

Finally, we shall call the sum of the entropies of a
number of bodies briefly the entropy of the system composed
of those bodies. Thus the entropy of a body whose particles
are not at uniform temperature, and have different velocities,
may be found, as in the case of gases (\SecRef{127}), by a summation
extending over all its elements of mass, provided
the temperature and density within each infinitely small
element of mass may be considered uniform. Neither the
velocity nor the weight of the particles enter into the
expression for the entropy.

\Section{132.} The existence and the value of the entropy
having been established for all states of a body, there is
no difficulty in transferring the proof, which was given for
perfect gases (beginning in~\SecRef{119}), to any system of bodies.
Just as in~\SecRef{119} we find that, during reversible adiabatic
expansion or compression of a body, its entropy remains
constant, while by the absorption of heat the change of the
entropy is
\[
d\Phi = \frac{Q}{\theta}\Add{.}
\Tag{(62)}
\]
This relation holds only for reversible changes of volume,
as was shown for perfect gases in~\SecRef{120}. Besides, it is found,
as in~\SecRef{121}, that during reversible expansion or compression
of two bodies at a common temperature, if they be allowed
to exchange heat by conduction with one another, but not
with surrounding bodies, the sum of their entropies remains
constant. A line of argument corresponding fully to that
advanced for perfect gases then leads to the following
general result:\footnote
  {With regard to the generalization of the theorem which was proved for
  a perfect gas in~\SecRef{124}, it may be stated that a certain difficulty arises in the
  special case of an incompressible body. In this case the body cannot be
  expanded. Professor Krigar-Menzel, who drew my attention to this, sent
\index{Krigar@Krigar-Manzel|indexnote}%
  me at the same time the following proof. \emph{Proposition}: It is impossible to
  diminish the entropy of an incompressible body without leaving changes in
  other bodies. \emph{Proof}: Bring the body into thermal contact with a perfect
  gas, isolate the system adiabatically, and diminish the volume of the gas by
  reversible compression. Heat thereby passes from the gas into the body, and
  the entropy of the gas diminishes in consequence, while that of the body
  increases by an equal amount. Now separate the body from the gas. If the
  proposition were false, and there existed an uncompensated entropy diminishing
  process, we could by means of it bring the body back to its original
  smaller entropy, and therewith to its initial state. The only outstanding
  change of the whole process would be the diminution of the entropy of the
  perfect gas. But this contradicts \SecRef{118}. The proposition is therefore not
  false, but true.}
\emph{It is impossible in any way to diminish the
\PageSep{100}
entropy of a system of bodies without thereby leaving behind
\index{Entropy!principle of increase of}%
changes in other bodies.} If, therefore, a system of bodies has
changed its state in a physical or chemical way, without
leaving any change in bodies not belonging to the system,
then the entropy in the final state is greater than, or, in the
limit, equal to the entropy in the initial state. The limiting
case corresponds to reversible, all others to irreversible,
processes.

\Section{133.} The restriction, hitherto indispensable, that no
\index{Carnot's!cycle}%
changes must remain in bodies outside the system is easily
dispensed with by including in the system all bodies that
may be affected in any way by the process considered. The
proposition then becomes: \emph{Every physical or chemical process
in nature takes place in such a way as to increase the sum of
the entropies of all the bodies taking any part in the process.
In the limit, \ie\ for reversible processes, the sum of the
entropies remains unchanged.} This is the most general
statement of the second law of Thermodynamics.

\Section{134.} As the impossibility of perpetual motion of the
first kind leads to the first law of Thermodynamics, or the
principle of the conservation of energy; so the impossibility
of perpetual motion of the second kind has led to the second
law, properly designated as the \emph{principle of the increase of
the entropy}. This principle may be presented under other
forms, which possess certain practical advantages, especially
for isothermal or isopiestic processes. They will be mentioned
in our next chapter. It should be emphasized, however,
that the form here given is the only one of unrestricted
\PageSep{101}
applicability to any finite process, and that no other universal
measure of the irreversibility of processes exists than the
amount of the increase of the entropy to which they lead.
All other forms of the second law are either applicable to
infinitesimal changes only, or presuppose, when extended to
finite changes, the existence of some special condition imposed
upon the process (\SecRef{140}, etc.). The real meaning of
the second law has frequently been looked for in a ``dissipation
of energy.'' This view, proceeding, as it does, from
\index{Dissipation of energy.}%
\index{Energy!dissipation of}%
the irreversible phenomena of conduction and radiation of
heat, presents only one side of the question. There are
irreversible processes in which the final and initial states
show exactly the same form of energy, \eg\ the diffusion of
two perfect gases (\SecRef{238}), or further dilution of a dilute
solution. Such processes are accompanied by no perceptible
transference of heat, nor by external work, nor by any noticeable
transformation of energy.\footnote
  {In reply to a criticism of this statement, I have simply to refer to
  \SecRef{108}, wherein the remark is made, that, to be sure, by the introduction of
  new kinds of energy, conceived \textit{ad~hoc}, it is possible to speak of an energy
  transformation even for the cases now under discussion. There is, however,
  nothing arbitrary in the statement made in the text, where the energy
  appears as completely defined by~\SecRef{56}, but rather in the introduction of the
  new kinds of energy.}
They occur only for the
reason that they lead to an appreciable increase of the
entropy. The amount of ``lost work'' yields a no more
definite general measure of irreversibility than does that of
``dissipated energy.'' This is possible only in the case of
isothermal processes (\SecRef{143}). An exhaustive general statement
of the second law can be made only by means of the
conception of the entropy.

\Section{135.} Clausius summed up the first law by saying that
\index{Clausius' equation!statement of first and second laws}%
the energy of the world remains constant; the second by
saying that the entropy of the world tends towards a
maximum. Objection has justly been raised to this form
of expression. The energy and the entropy of the world
have no meaning, because such quantities admit of no
accurate definition. Nevertheless, it is not difficult to
\PageSep{102}
express the characteristic feature of those propositions of
Clausius in such a way as to give them a meaning, and to
bring out more clearly what Clausius evidently wished to
express by them.

The energy of any system of bodies changes according
to the measure of the effects produced by external agents.
It remains constant, only, if the system be isolated. Since,
strictly speaking, every system is acted on by external
agents---for complete isolation cannot be realized in nature---the
energy of a finite system may be approximately, but
never absolutely, constant. Nevertheless, the more extended
the system, the more negligible, in general, will the
external effects become, in comparison with the magnitude
of the energy of the system, and the changes of energy of its
parts (\SecRef{55}); for, while the external effects are of the order of
magnitude of the surface of the system, the internal energy
is of the order of magnitude of the volume. In very small
systems (elements of volume) the opposite is the case for
the same reason, since here the energy of the system may
be neglected in comparison with any one of the external
effects. Frequent use is made of this proposition, \eg\ in
establishing the limiting conditions in the theory of the conduction
of heat. In the case here considered, it may, therefore,
be said that the more widely extended a system we
assume, the more approximately, in general, will its energy
remain constant. A comparatively small error will be committed
in assuming the energy of our solar system to be
constant, a proportionately smaller one if the system of all
known fixed stars be included. In this sense an actual
significance belongs to the proposition, that the energy
of an infinite system, or the energy of the world, remains
constant.

The proposition regarding the increase of the entropy
should be similarly understood. If we say that the entropy
of a system increases quite regardless of all outside
changes, an error will, in general, be committed, but the
more comprehensive the system, the smaller does the proportional
error become.
\PageSep{103}

\Section{136.} In conclusion, we shall briefly discuss the question
\index{Second law of thermodynamics!possible limitations}%
of the possible limitations to the second law. If there exist
any such limitations---a view still held by many scientists
and philosophers---this much may be asserted, that their
existence presupposes an error in our starting-point, viz.\
the impossibility of perpetual motion of the second kind, or
a fault in our method of proof. From the beginning we
have recognized the legitimacy of the first of these objections,
and it cannot be removed by any line of argument. The
second objection generally amounts to the following. The
impracticability of perpetual motion of the second kind is
granted, yet its absolute impossibility is contested, since our
limited experimental appliances, supposing it were possible,
would be insufficient for the realization of the ideal processes
which the line of proof presupposes. This position, however,
proves untenable. It would be absurd to assume that the
validity of the second law depends in any way on the skill
of the physicist or chemist in observing or experimenting.
The gist of the second law has nothing to do with experiment;
the law asserts briefly that \emph{there exists in nature a
quantity which changes always in the same sense in all natural
processes}. The proposition stated in this general form may be
correct or incorrect; but whichever it may be, it will remain
so, irrespective of whether thinking and measuring beings
exist on the earth or not, and whether or not, assuming
they do exist, they are able to measure the details of physical
or chemical processes more accurately by one, two, or a
hundred decimal places than we can.\footnote
  {We do not say that the second law is applicable to every single
  detail of a process. Upon closer examination the matter appears to be thus.
  The entropy, like temperature, pressure, and density, cannot be defined as an
  absolute, continuous quantity, but as a certain average value of a large
  number of single values. As long, therefore, as we regard simply one or
  more single values, the entropy cannot be defined any more than the
  temperature or pressure, and the second law neither applied nor proved.}
The limitations
to the law, if any, must lie in the same province as its
essential idea, in the observed Nature, and not in the
Observer. That man's experience is called upon in the deduction
of the law is of no consequence; for that is, in fact,
\PageSep{104}
our only way of arriving at a knowledge of natural law.
But the law once discovered must receive recognition of its
independence, at least in so far as Natural Law can be said
to exist independent of Mind. Should any one deny this,
he would have to deny the possibility of natural science.

The case of the first law is quite similar. To most
unprejudiced scientists the impossibility of perpetual motion
of the first kind is certainly the most direct of the general
proofs of the principle of energy. Nevertheless, hardly
any one would now think of making the validity of that
principle depend on the degree of accuracy of the experimental
proof of that general empirical proposition. Presumably
the time will come when the principle of the
increase of the entropy will be presented without any
connection with experiment. Some \Erratum{metaphysicists}{metaphysicians} may
even put it forward as being \textit{a~priori} valid. In the
mean time, no more effective weapon can be used by both
champions and opponents of the second law than indefatigable
endeavour to follow the real purport of this law
to the utmost consequences, taking the latter one by one
to the highest court of appeal---experience. Whatever
the decision may be, lasting gain will accrue to us from
such a proceeding, since thereby we serve the chief end of
natural science---the enlargement of our stock of knowledge.
\PageSep{105}


\Chapter{III.}{General Deductions.}
\index{Deductions from second law of thermodynamics}%
\index{Second law of thermodynamics!deductions}%

\Section{137.} \First{Our} first application of the principle of the entropy
which was expressed in its most general form in the preceding
chapter, will be to Carnot's cycle, described in detail
for perfect gases in~\SecRef{90}. This time, the system operated
upon may be of any character whatsoever, and chemical reactions,
too, may take place, provided they are reversible.
Resuming the notation used in~\SecRef{90}, we may at once state
the result.

In a cyclic process, according to the first law, the heat,~$Q_{2}$,
given out by the hotter reservoir is equivalent to the
sum of the work done by the system, $W' = -W$, and the
heat received by the colder reservoir, $Q_{1}' = -Q_{1}$:
\[
Q_{2} = W' + Q_{1}'
\]
or
\[
Q_{1} + Q_{2} + W = 0\Add{.}
\Tag{(63)}
\]

According to the second law, since the process is reversible,
all bodies which show any change of state after
the process, \ie\ the two heat-reservoirs only, possess the
same total entropy as before the process. The change of
the entropy of the two reservoirs is, according to~\Eq{(62)}:
\[
\frac{Q_{1}'}{\theta_{1}} = -\frac{Q_{1}}{\theta_{1}}
\quad\text{for the first, and}\quad
-\frac{Q_{2}}{\theta_{2}}\quad\text{for the second,}
\Tag{(64)}
\]
their sum:
\[
\frac{Q_{1}}{\theta_{1}} + \frac{Q_{2}}{\theta_{2}} = 0
\Tag{(65)}
\]
whence, by~\Eq{(63)},
\[
Q_{1} : Q_{2} : W = (-\theta_{1}) : \theta_{2} : (\theta_{1} - \theta_{2})
\]
\PageSep{106}
as in~\Eq{(44)}, but without any assumption as to the nature of
the substance passing through the cycle of operations.

In order, therefore, to gain the mechanical work,~$W'$,
by means of a reversible Carnot cycle of operations with
any substance between two heat-reservoirs at the temperatures
$\theta_{1}$~and~$\theta_{2}$ ($\theta_{2} > \theta_{1}$), the quantity of heat
\[
Q_{1}' = \frac{\theta_{1}}{\theta_{2} - \theta_{1}} W'
\]
must pass from the hotter to the colder reservoir. In
other words, the passage of the quantity of heat~$Q_{1}'$ from~$\theta_{2}$
to~$\theta_{1}$ may be taken advantage of to gain the mechanical
work
\[
W' = \frac{\theta_{2} - \theta_{1}}{\theta_{1}} Q_{1}'\Add{.}
\Tag{(66)}
\]

\Section{138.} For an irreversible cycle, \ie\ one involving any
irreversible physical or chemical changes of the substance
operated upon, the equation of energy~\Eq{(63)} still holds, but
the equation for the change of the entropy~\Eq{(65)} is replaced
by the inequality:
\[
- \frac{Q_{1}}{\theta_{1}} - \frac{Q_{2}}{\theta_{2}} > 0.
\]

Observe, however, that the expressions~\Eq{(64)} for the
change of the entropy of the reservoirs are still correct,
provided we assume that any changes of volume of the
substances used as reservoirs are reversible. Thus,
\[
\frac{Q_{1}}{\theta_{1}} + \frac{Q_{2}}{\theta_{2}} < 0
\Tag{(67)}
\]
hence, from \Eq{(67)}~and \Eq{(63)},
\[
W' = -W = Q_{1} + Q_{2} < Q_{1} + \frac{\theta_{2}}{\theta_{1}} Q_{1}'\Add{,}
\]
or
\[
W' < \frac{\theta_{2} - \theta_{1}}{\theta_{1}} Q_{1}'.
\]
\PageSep{107}
This means that the amount of work,~$W'$, to be gained by
means of a cyclic process from the transference of the heat,~$Q_{1}'$,
from a hotter to a colder reservoir, is always smaller
for an irreversible process than for a reversible one. Consequently
the equation~\Eq{(66)} represents the maximum amount
of work to be gained from any cyclic process between heat-reservoirs
at the temperatures $\theta_{2}$ and~$\theta_{1}$.

In particular, if $W' = 0$, it follows from the equation of
energy~\Eq{(63)} that
\[
Q_{2} = -Q_{1} = Q_{1}'
\]
and the inequality~\Eq{(67)} becomes
\[
Q_{2} \left(\frac{1}{\theta_{2}} - \frac{1}{\theta_{1}}\right) < 0.
\]
In this case the cyclic process results in the transference
of heat~($Q_{2}$) from the reservoir of temperature~$\theta_{2}$ to that of
temperature~$\theta_{1}$, and the inequality means that this flow
of heat is always directed from the hotter to the colder
reservoir.

Again, a special case of this type of process is the direct
passage of heat by conduction between heat-reservoirs, without
any actual participation of the system supposed to pass
through the cycle of operations. It is seen to be an irreversible
change, since it brings about an increase of the
sum of the entropies of the two heat-reservoirs.

\Section{139.} We shall now apply the principle of the entropy
to any reversible or irreversible cycle with any system of
bodies, in the course of which only one heat-reservoir of
constant temperature~$\theta$ is used. Whatever may be the
nature of the process in detail, there remains at its close no
change of the entropy except that undergone by the heat-reservoir.
According to the first law, we have
\[
W + Q = 0.
\]
$W$~is the work done on the system, and $Q$~the heat absorbed
by the system from the reservoir.

According to the second law, the change of the entropy
\PageSep{108}
of the reservoir, within which only reversible changes of
volume are supposed to take place, is
\[
-\frac{Q}{\theta} \geq 0
\]
or
\[
Q \leq 0,
\]
whence
\[
W \geq 0.
\]
Work has been expended on the system, and heat added
to the reservoir. If, in the limit, the process be reversible,
the signs of inequality disappear, and both the work~$W$
and the heat~$Q$ are zero. On this proposition rests the great
fertility of the second law in its application to isothermal
reversible cycles.

\Section{140.} We shall no longer deal with cycles, but shall
\index{Direction of natural process}%
\index{Natural process, direction of}%
consider the general question of the direction in which a
change will set in, when any system in nature is given.
For chemical reactions in particular is this question of
importance. It is completely answered by the second
law in conjunction with the first, for the second law contains
a condition necessary for all natural processes. Let
us imagine any homogeneous or heterogeneous system of
bodies at the common temperature~$\theta$, and investigate the
conditions for the starting of any physical or chemical
change. According to the first law, we have for any
infinitesimal change:
\[
dU = Q + W,
\Tag{(68)}
\]
where $U$~is the total internal energy of the system, $Q$~the
heat absorbed by the system during the process, and $W$~the
work done on the system.

According to the second law, the change of the total
entropy of all the bodies taking part in the process is
\[
d\Phi + d\Phi_{0} \geq 0\Add{,}
\]
where $\Phi$~is the entropy of the system, $\Phi_{0}$~the entropy of
the surrounding medium (air, calorimetric liquid, walls of
\PageSep{109}
vessels, etc.). Here the sign of equality holds for reversible
cases, which, it is true, should be considered as an ideal
limiting case of actual processes (\SecRef{115}).

If we assume that all changes of volume in the surrounding
medium are reversible, we have, according to~\Eq{(62)},
\[
d\Phi_{0} = -\frac{Q}{\theta},
\]
or, by~\Eq{(68)},
\[
d\Phi_{0} = -\frac{dU - W}{\theta}.
\]
On substituting the value of~$d\Phi_{0}$, we have
\[
d\Phi - \frac{dU - W}{\theta} \geq 0\Add{,}
\Tag{(69)}
\]
or
\[
dU - \theta\, d\Phi \leq W\Add{.}
\Tag{(70)}
\]
All conclusions with regard to thermodynamic chemical
changes, hitherto drawn by different authors in different
ways, culminate in this relation~\Eq{(70)}. It cannot in general
be integrated, since the left-hand side is not, in general,
a perfect differential. The second law, then, does not lead
to a general statement with regard to finite changes of a
system taken by itself unless something be known of the
external conditions to which it is subject. This was to be
expected, and holds for the first law as well. To arrive at a
law governing finite changes of the system, the knowledge
of such external conditions as will permit the integration of
the differential is indispensable. Among these the following
are singled out as worthy of note.

\Section{141.} \Topic{Case I\@. Adiabatic Process.}---No exchange of
\index{Adiabatic process}%
\index{Process!adiabatic}%
heat with the surroundings being permitted, we have $Q = 0$,
and, by~\Eq{(68)},
\[
dU = W.
\]
Consequently, by~\Eq{(70)},
\[
d\Phi \geq 0.
\]
The entropy of the system increases or remains constant, a
case which has already been sufficiently discussed.
\PageSep{110}

\Section{142.} \Topic{Case II\@. Isothermal Process.}---The temperature
\index{Process!isothermal}%
$\theta$ being kept constant, \Eq{(70)}~passes into
\[
d(U - \theta\Phi) \leq W\Add{,}
\]
\ie\ the increment of the quantity $(U - \theta\Phi)$ is smaller than,
or, in the limit, equal to, the work done on the system.
This theorem is well adapted for application to chemical
processes, since isothermal changes play an important part
in nature.

Putting
\[
U - \theta\Phi = F,
\Tag{(71)}
\]
we have, for reversible isothermal changes:
\[
dF = W
\]
and, on integrating,
\[
F_{2} - F_{1} = \tsum W\Add{.}
\Tag{(72)}
\]
For finite reversible isothermal changes the total work
done on the system is equal to the increase of~$F$; or, the
entire work performed by the system is equal to the decrease
of~$F$, and, therefore, depends only on the initial and final
states of the system. Where $F_{1} = F_{2}$, as in cyclic processes,
the external work is zero.

The function~$F$, thus bearing the same relation to the
external work that the energy~$U$ does to the sum of the
\index{Energy!free}%
\index{Energy!latent}%
\index{Energy!total}%
\index{Heat, latent!theory}%
external heat and work, has been called by H.~v.~Helmholtz
the \emph{free energy} (freie Energie) of the system. (It should
\index{Free energy}%
rather be called ``free energy for isothermal processes.'')
\index{Isothermal processes}%
Corresponding to this, he calls~$U$ the \emph{total energy} (Gesammtenergie),
and the difference $U - F = \theta\Phi$, the \emph{latent energy}
\index{Latent energy}%
(gebundene Energie) of the system. The change of the
latter in reversible isothermal processes gives the amount
of the external heat absorbed. This splitting up of total
energy into free and latent energy is applicable to isothermal
processes only.

In irreversible processes, on the other hand, $dF < W$,
and on integrating we have
\[
F_{2} - F_{1} < \tsum W\Add{.}
\Tag{(73)}
\]
\PageSep{111}
The free energy increases by a less amount than that which
corresponds to the work done on the system. The results
\index{Work!maximum}%
for reversible and irreversible processes may be stated thus.
In irreversible isothermal processes the work done on the
system is more, or the work done by the system is less, than
it would be if the same change were brought about by a
reversible process, for in that case it would be the difference
of the free energies at the beginning and end of the
process~\Eq{(72)}.

Hence, any reversible transformation of the system from
one state to another yields the maximum amount of work
that can be gained by any isothermal process between those
two states. In all irreversible processes a certain amount of
work is lost, viz.\ the difference between the maximum work
\index{Maximum work}%
to be gained (the decrease of the free energy) and the work
actually gained.

The fact that, in the above, irreversible as well as
reversible processes between the same initial and final states
were considered, does not contradict the proposition that
between two states of a system either only reversible or
only irreversible processes are possible, if no external
changes are to remain in other bodies. In fact, the process
here discussed involves such changes in the surrounding
medium; for, in order to keep the system at constant
temperature, an exchange of heat between it and the surrounding
medium must take place in one direction or the
other.

\Section{143.} If the work done during an isothermal process
vanish, as is practically the case in most chemical reactions,
we have
\[
\tsum W = 0,
\]
and, by~\Eq{(73)},
\[
F_{2} - F_{1} < 0,
\]
\ie\ the free energy decreases. The amount of this decrease
may be used as a measure of the work done by the forces
(chemical affinity) causing the process, for the same is not
available for external work.

For instance, let an aqueous solution of some non-volatile
\PageSep{112}
\index{Affinity of hydrogen for oxygen}%
\index{Decrease of free energy by dilution}%
\index{Hydrogen, affinity of, for oxygen}%
salt be diluted isothermally, the heat of dilution being
\index{Dilution!decrease of free energy by}%
furnished or received by a heat-reservoir according as the
energy,~$U_{2}$, of the diluted solution (final state) is greater or
less than the sum,~$U_{1}$, of the energies of the undiluted
solution and the water added (initial state). The free
energy,~$F_{2}$, of the diluted solution, on the other hand, is
necessarily smaller than the sum,~$F_{1}$, of the free energies of
the undiluted solution and the water added. The amount
of the decrease of the free energy, or the work done by the
\index{Free energy!decrease of, by dilution}%
``affinity of the solution for water'' during the process of
dilution may be measured. For this purpose, the dilution
should be performed in some reversible isothermal manner,
when, according to~\Eq{(72)}, the quantity to be measured is
actually gained in the form of external work. For instance,
evaporate the water, which is to be added, infinitely slowly
under the pressure of its saturated vapour. When it has all
been changed to water vapour, allow the latter to expand
isothermally and reversibly until its density equals that
which saturated water vapour would possess at that temperature
when in contact with the solution. Now establish
lasting contact between the water vapour and the solution,
whereby the equilibrium will not be disturbed. Finally, by
isothermal compression, condense the water vapour infinitely
slowly when in direct contact with the solution. It will
then be uniformly distributed throughout the latter. Such
a process, as here described, is composed only of states of
equilibrium. Hence it is reversible, and the external work
thereby gained represents at the same time the decrease
of the free energy, $F_{2} - F_{1}$, which takes place on directly
mixing the solution and the water.

As a further example, we shall take a mixture of
hydrogen and oxygen which has been exploded by means
of an electric spark. The spark acts only the secondary
part of a release, its energy being negligible in comparison
with the energies obtained by the reaction. The work
of the chemical affinities in this process is equal to the
mechanical work that might be gained by chemically combining
the oxygen and hydrogen in some reversible and
\PageSep{113}
isothermal way. Dividing this quantity by the number of
oxidized molecules of hydrogen, we obtain a measure of the
force with which a molecule of hydrogen tends to become
oxidized. This definition of chemical force, however, has
only a meaning in so far as it is connected with that work.

\Section{144.} In chemical processes the changes of the first
\index{Berthelot's principle}%
\index{Principle of Berthelot}%
term,~$U$, of the expression for the free energy~\Eq{(71)}, frequently
\index{Energy!free!of perfect gas}%
\index{Free energy!change of, with temperature.}%
\index{Free energy!of a perfect gas}%
far surpass those of the second,~$\theta\Phi$. Under such
circumstances, instead of the decrease of~$F$, that of~$U$, \ie\
the heat effect, may be considered as a measure of the
chemical work. This leads to the proposition that chemical
reactions, in which there is no external work, take place in
such a manner as to give the greatest heat effects (Berthelot's
principle). For high temperatures, where~$\theta$, and for gases
and dilute solutions, where $\Phi$~is large, the term~$\theta\Phi$ can no
longer be neglected without considerable error. In these
cases, therefore, chemical changes often do take place in
such a way as to increase the total energy, \ie\ with the
absorption of heat.

\Section{145.} It should be borne in mind that all these propositions
refer only to isothermal processes. To answer the
question as to how the free energy acts in other processes,
it is only necessary to form the differential of~\Eq{(71)} viz.:
\[
dF = dU - \theta\, d\Phi - \Phi\, d\theta\Add{,}
\]
and to substitute in the general relation~\Eq{(70)}. We have
then
\[
dF \leq W - \Phi\, d\theta
\]
for any physical or chemical process. This shows that,
with change of temperature, the relation between the
external work and the free energy is far more complicated.
This relation cannot, in general, be used with advantage.

\Section{146.} We shall now compute the value of the free
energy of a perfect gas. Here, according to~\Eq{(35)},
\[
U = Mu = M(c_{v} \theta + \const),
\]
\PageSep{114}
and, by~\Eq{(52)},
\[
\Phi = M\phi = M(c_{v} \log \theta + \frac{R}{m} \log v + \const).
\]
Substituting in~\Eq{(71)}, we obtain
\[
F = M\{c_{v} \theta (\const - \log \theta) - \frac{R\theta}{m} \log v + \const\}
\Tag{(74)}
\]
which contains an arbitrary linear function of~$\theta$.

For isothermal changes of the gas, we have, by~\SecRef{142},
\[
dF \leq W,
\]
or, by~\Eq{(74)}, since $\theta$~is $\const$,
\[
dF = -\frac{M\theta R}{m} · \frac{dv}{v} = -p\, dV \leq W.
\]
If the change be reversible, the external work on the
gas is $W = -p\, dV$, but if it be irreversible, then the
sign of inequality shows that the work of compression is
greater, or that of expansion smaller, than in a reversible
process.
\index{Process!isothermal-isopiestic}%

\Section{147.} \Topic{Case III\@. Isothermal-isopiestic Process.}---If,
\index{Isothermal-isopiestic process}%
besides the temperature~$\theta$, the external pressure~$p$ be
also kept constant, then the external work is given by the
formula,
\[
W = -p\, dV,
\]
and the left-hand side of~\Eq{(69)} becomes a complete differential:
\[
d\left(\Phi - \frac{U + pV}{\theta}\right) \geq 0\Add{.}
\]

In this case, it may be stated that for finite changes the
\index{Function $\Psi$}%
function,
\[
\Phi - \frac{U + pV}{\theta} = \Psi,
\Tag{(75)}
\]
\PageSep{115}
must increase, and will remain constant only in the limit
\index{Duhem@Duhem|indexnote}%
\index{Potential, energy!thermodynamic@thermodynamic|indexnote}%
\index{Thermodynamic potential}%
when the change is reversible.\footnote
  {Multiplying \Eq{(75)} by~$-\theta$, we get P.~Duhem's \emph{thermodynamic potential at
  constant pressure},
  \[
  U + pV - \theta\Phi,
  \]
  for which, so long as $\theta$~remains constant, the same propositions hold as for
  the function~$\Psi$. However, the equation~\Eq{(153)} in \SecRef{211}, which is important
  for the dependence of the equilibrium on temperature and pressure, can be
\index{Equilibrium!conditions of}%
  more conveniently deduced from the function~$\Psi$ than from the thermodynamic
  potential.}

\Section{148.} \Topic{Conditions of Equilibrium.}---The most general
\index{Condition of complete reversibility!of equilibrium}%
condition of equilibrium for any system of bodies is derived
from the proposition that no change can take place in the
system if it be impossible to satisfy the condition necessary
for a change.

Now, by~\Eq{(69)}, for any actual change of the system,
\[
d\Phi - \frac{dU - W}{\theta} > 0.
\]
The sign of equality is omitted, because it refers to ideal
changes which do not actually occur in nature. Equilibrium
is, therefore, maintained if the fixed conditions imposed on
the system be such that they will permit only changes in
which
\[
\delta\Phi - \frac{\delta U - W}{\theta} \leq 0.
\]
Here $\delta$~is used to signify a virtual infinitely small change,
in contrast to~$d$, which corresponds to an actual change.

\Section{149.} In most of the cases subsequently discussed, if any
given virtual change be compatible with the fixed conditions
of the system, its exact opposite is also, and is
represented by changing the sign of all variations involved.
This is true if the fixed conditions be expressed by equations,
not by inequalities. Assuming this to be the case,
if we should have, for any particular virtual change,
\[
\delta\Phi - \frac{\delta U - W}{\theta} < 0,
\]
\PageSep{116}
which, by~\Eq{(69)}, would make its occurrence in nature impossible,
its opposite would conform to the condition for
actual changes~\Eq{(69)}, and could therefore take place in
nature. To ensure equilibrium in such cases, it is necessary,
therefore, that, for any virtual change compatible with the
fixed conditions,
\[
\delta\Phi - \frac{\delta U - W}{\theta} = 0\Add{.}
\Tag{(76)}
\]
This equation contains a condition always sufficient, but, as
we have seen, not always necessary to its full extent, for the
maintenance of equilibrium. As a matter of experience,
equilibrium will occasionally subsist when equation~\Eq{(76)} is
not fulfilled, even though the fixed conditions permit of a
change of sign of all variations. This is to say, that
occasionally a certain change will not take place in nature,
though it satisfy the fixed conditions as well as the demands
of the second law. Such cases lead to the conclusion that
in some way the setting in of a change meets with a certain
resistance, which, on account of the direction in which it
acts, has been termed inertia resistance, or passive resistance.
\index{Inertia resistance}%
\index{Resistance inertia}%
States of equilibrium of this description are always
unstable. Often a very small disturbance, not comparable
in size with the quantities within the system, suffices to
produce the change, which under these conditions often
occurs with great violence. We have examples of this in
overcooled liquids, supersaturated vapour, supersaturated
solutions, explosive substances, etc. We shall henceforth
discuss mainly the conditions of stable equilibrium deducible
from~\Eq{(76)}.

This equation may, under certain circumstances, be
expressed in the form of a condition for a maximum or
minimum. This can be done when, and only when, the
conditions imposed upon the system are such that the left-hand
side of~\Eq{(76)} represents the variation of some one
function. The most important of these cases are dealt with
separately in the following paragraphs. They correspond
exactly to the propositions which we have already deduced
\PageSep{117}
for special cases. From these propositions it may at once
be seen whether it is a case of a maximum or a minimum.

\Section{150.} \Topic{First Case} (\SecRef{141}).---If no exchange of heat take
place with the surrounding medium, the first law gives
\[
\delta U = W,
\]
hence, by~\Eq{(76)},
\[
\delta \Phi = 0\Add{.}
\Tag{(77)}
\]
Among all the states of the system which can proceed from
one another by adiabatic processes, the state of equilibrium
is distinguished by a maximum of the entropy. Should
\index{Entropy!maximum value}%
\index{Maximum value!of entropy}%
\index{Maximum value!of free energy}%
there be several states in which the entropy has a maximum
value, each one of them is a state of equilibrium; but if
the entropy be greater in one than in all the others, then
that state represents absolutely stable equilibrium, for it
could no longer be the starting-point of any change whatsoever.

\Section{151.} \Topic{Second Case} (\SecRef{142}).---If the temperature be
kept constant, equation~\Eq{(76)} passes into
\[
\delta\left(\Phi - \frac{U}{\theta}\right) + \frac{W}{\theta} = 0,
\]
and, by~\Eq{(71)},
\[
- \delta F = -W.
\]
Among all the states which the system may assume at a
given temperature, a state of equilibrium is characterized
by the fact that the free energy of the system cannot
\index{Free energy!minimum value of}%
decrease without performing an equivalent amount of work.

If the external work be a negligible quantity, as it is
when the volume is kept constant or in numerous chemical
processes, then $W = 0$, and the condition of equilibrium
becomes
\[
\delta F = 0,
\]
\ie\ among the states which can proceed from one another
by isothermal processes, without the performance of external
work, the state of most stable equilibrium is distinguished
by an absolute minimum of the free energy.
\PageSep{118}

\Section{152.} \Topic{Third Case} (\SecRef{147}).---Keeping the temperature~$\theta$
\index{Maximum value!of $\Psi$}%
and the pressure~$p$ constant and uniform, we have
\[
W = -p\, \delta V,
\Tag{(78)}
\]
and the condition of equilibrium~\Eq{(76)} becomes
\[
\delta\left(\Phi - \frac{U + pV}{\theta}\right) = 0,
\]
or, by~\Eq{(75)},
\[
\delta\Psi = 0\Add{,}
\Tag{(79)}
\]
\ie\ at constant temperature and constant pressure, the state
of most stable equilibrium is characterized by an absolute
maximum of the function~$\Psi$.

We shall now proceed to consider, in succession, states
of equilibrium of various systems by means of the theorems
we have just deduced, going from simpler to more complicated
cases.
\PageSep{119}


\Part{IV.}{Applications to Special States of Equilibrium.}

\Chapter{I.}{Homogeneous Systems.}
\index{Homogeneous!system|(}%

\Section{153.} \First{Let} the state of a homogeneous system be determined,
\index{System!homogeneous|(}%
as hitherto, by its mass,~$M$; its temperature,~$\theta$; and
either its pressure,~$p$, or its specific volume, $v = \dfrac{V}{M}$. For
the present, besides~$M$, let $\theta$~and $v$ be the independent
variables. Then the pressure~$p$, the specific energy $u = \dfrac{U}{M}$,
and the specific entropy $\phi = \dfrac{\Phi}{M}$ are functions of $\theta$~and~$v$,
\index{Entropy!specific}%
\index{Specific entropy}%
the definition of the specific entropy~\Eq{(61)} being
\[
d\phi = \frac{du + p\, dv}{\theta} = \frac{1}{\theta} \left(\frac{\dd u}{\dd \theta}\right)_{v} d\theta + \frac{\left(\dfrac{\dd u}{\dd v}\right)_{\theta} + p}{\theta}\, dv.
\]
On the other hand,
\[
d\phi = \left(\frac{\dd \phi}{\dd \theta}\right)_{v} d\theta
      + \left(\frac{\dd \phi}{\dd v}\right)_{\theta} dv.
\]
Therefore, since $d\theta$~and $dv$ are independent of each other,
\begin{align*}
\left(\frac{\dd \phi}{\dd \theta}\right)_{v} &= \frac{1}{\theta} \left(\frac{\dd u}{\dd \theta}\right)_{v}
\Tag{(79a)} \\
\intertext{and}
\left(\frac{\dd \phi}{\dd v}\right)_{\theta} &= \frac{\left(\dfrac{\dd u}{\dd v}\right)_{\theta} + p}{\theta}\Add{.}
\end{align*}
\PageSep{120}
These two equations lead to an experimental test of the
second law; for, differentiating the first with respect to~$v$,
the second with respect to~$\theta$, we have
\[
\frac{\dd^{2} \phi}{\dd \theta\, \dd v}
  = \frac{1}{\theta} · \frac{\dd^{2} u}{\dd \theta\, \dd v}
  = \frac{\dfrac{\dd^{2} u}{\dd \theta\, \dd v} + \left(\dfrac{\dd p}{\dd \theta}\right)_{v}}{\theta}
  - \frac{\left(\dfrac{\dd u}{\dd v}\right)_{\theta} + p}{\theta^{2}}\Add{,}
\]
or
\[
\left(\frac{\dd u}{\dd v}\right)_{\theta}
  = \theta \left(\frac{\dd p}{\dd \theta}\right)_{v} - p\Add{.}
\Tag{(80)}
\]
By this and equation~\Eq{(24)}, the above expressions for the
differential coefficients of~$\phi$ become:
\[
\left(\frac{\dd \phi}{\dd \theta}\right)_{v} = \frac{c_{v}}{\theta};
\quad\text{and}\quad
\left(\frac{\dd \phi}{\dd v}\right)_{\theta} = \left(\frac{\dd p}{\dd \theta}\right)_{v}\Add{.}
\Tag{(81)}
\]

\Section{154.} Equation~\Eq{(80)}, together with~\Eq{(28)} of the first
law, gives the relation:
\[
c_{p} - c_{v} = \theta \left(\frac{\dd p}{\dd \theta}\right)_{v} · \left(\frac{\dd v}{\dd \theta}\right)_{p}\Add{,}
\Tag{(82)}
\]
which is useful either as a test of the second law or for the
calculation of~$c_{v}$ when $c_{p}$~is given. But since in many cases
$\left(\dfrac{\dd p}{\dd \theta}\right)_{v}$ cannot be directly measured, it is better to introduce
the relation~\Eq{(6)}, and then
\[
c_{p} - c_{v} = -\theta \left(\frac{\dd p}{\dd v}\right)_{\theta} · \left(\frac{\dd v}{\dd \theta}\right)_{p}^{2}\Add{.}
\Tag{(83)}
\]
As $\left(\dfrac{\dd p}{\dd v}\right)_{\theta}$ is necessarily negative, $c_{p}$~is always greater than~$c_{v}$,
except in the limiting case, when the coefficient of expansion
is $= 0$, as in the case of water at $4°$~C.\Chg{,}{}; then $c_{p} - c_{v} = 0$.

As an example, we may calculate the specific heat at
\index{Specific heat!at constant pressure}%
\index{Specific heat!at constant volume}%
constant volume,~$c_{v}$, of mercury at $0°$~C. from the following
data:
\begin{gather*}
c_{p} = 0.0333;\quad \theta = 273°; \\
\left(\frac{\dd p}{\dd v}\right)_{\theta} = -\frac{1014000}{0.00000295 · v},
\end{gather*}
\PageSep{121}
where the denominator is the coefficient of compressibility
in atmospheres (\SecRef{15}); the numerator, the pressure of an
atmosphere in absolute units (\SecRef{17}); $v = \dfrac{1}{13.6}$, the volume
of $1~\Unit{gr.}$ of mercury at $0°$~C.; $\left(\dfrac{\dd v}{\dd p}\right)_{p} = 0.0001812 · v$, the
coefficient of thermal expansion~(\SecRef{15}).

To obtain $c_{v}$~in calories, it is necessary to divide by the
mechanical equivalent of heat, $419 × 10^{5}$ (\SecRef{61}). Thus we
obtain from~\Eq{(83)}
\[
c_{p} - c_{v} = \frac{273 × 1014000 × 0.0001812^{2}}
                    {0.00000295 × 13.6 × 419 × 10^{5}}
  = 0.0054,
\]
whence, from the above value for~$c_{p}$,
\[
c_{v} = 0.0279.
\]

\Section{155.} This method of calculating the difference of the
\index{Difference of specific heats}%
\index{Specific heats!difference of}%
specific heats $c_{p} - c_{v}$ applicable to any substance, discloses
at the same time the order of magnitude of the different
influences to which this quantity is subject. According to
equation~\Eq{(28)} of the first law, the difference of the specific
heats is
\[
c_{p} - c_{v} = \left\{\left(\frac{\dd u}{\dd v}\right)_{\theta} + p\right\} \left(\frac{\dd v}{\dd \theta}\right)_{p}\Add{.}
\]
The two terms of this expression, $\left(\dfrac{\dd u}{\dd v}\right)_{\theta} \left(\dfrac{\dd v}{\dd \theta}\right)_{p}$ and $p\left(\dfrac{\dd v}{\dd \theta}\right)_{p}$,
depend on the rate of change of the energy with the volume,
and on the external work performed by the expansion
respectively. In order to find which of these two terms has
the greater influence on the quantity $c_{p} - c_{v}$, we shall find
the ratio of the first to the second:
\[
\frac{1}{p} · \left(\frac{\dd u}{\dd v}\right)_{\theta},
\]
or, by~\Eq{(80)},
\[
\frac{\theta}{p} · \left(\frac{\dd p}{\dd \theta}\right)_{v} - 1\Add{,}
\Tag{(84)}
\]
or by~\Eq{(6)},
\[
-\frac{\theta}{p} · \left(\frac{\dd v}{\dd \theta}\right)_{p} · \left(\frac{\dd p}{\dd v}\right)_{\theta} - 1\Add{.}
\]
\PageSep{122}
A glance at the tables of the coefficients of thermal
expansion and of the compressibility of solids and liquids
shows that, in general, the first term of this expression is
a large number, making the second,~$-1$, a negligible
quantity. For mercury at~$0°$, \eg, the above data give the
first term to be
\[
273 × \frac{0.0001812}{0.00000295} = 16800.
\]
Water at $4°$~C. is an exception.

It follows that, for solids and liquids, the difference
$c_{p} - c_{v}$ depends rather on the relation between the energy
and the volume than on the external work of expansion.
For perfect gases the reverse is the case, since the internal
energy is independent of the volume, \ie---
\[
\left(\frac{\dd u}{\dd v}\right)_{\theta} = 0.
\]
During expansion, therefore, the influence of the internal
energy vanishes in comparison with that of the external
work; in fact, the expression~\Eq{(84)} vanishes for the characteristic
equation of a perfect gas. With ordinary gases,
however, both the internal energy and the external work
must be considered.

\Section{156.} The sum of both these influences, \ie\ the whole
\index{Ratio of specific heats}%
\index{Specific heats!ratio of}%
expression $c_{p} - c_{v}$, may be said to have a small value for
most solids and liquids; thus the ratio $\dfrac{c_{p}}{c_{v}} = \gamma$ is but slightly
greater than unity. This means that in solids and liquids
the energy depends far more on the temperature than on the
volume. For gases, $\gamma$~is large; and, in fact, the fewer the
number of atoms in a molecule of the gas, the larger does
it become. Hydrogen, oxygen, and most gases with diatomic
molecules have $\gamma = 1.41$ (\SecRef{87}). The largest value of~$\gamma$ ever
observed is that found by Kundt and Warburg for the
\index{Kundt}%
\index{Warburg}%
monatomic vapour of mercury, viz.~$1.666$.

\Section{157.} For many applications of the second law it is
\PageSep{123}
\index{Deviation from perfect gases}%
\index{Gay-Lussac's law, deviations from}%
\index{Influence!of pressure on specific heat}%
\index{Specific heat!influence of temperature on, at constant pressure}%
convenient to introduce~$p$ instead of~$v$ as an independent
variable. We have, by~\Eq{(61)},
%[** TN: Re-breaking]
\begin{align*}
d\phi &= \frac{du + p\, dv}{\theta} \\
  &= \biggl\{\left(\frac{\dd u}{\dd \theta}\right)_{p} + p\left(\frac{\dd v}{\dd \theta}\right)_{p}\biggr\} \frac{d\theta}{\theta}
  + \biggl\{\left(\frac{\dd u}{\dd p}\right)_{\theta} + p\left(\frac{\dd v}{\dd p}\right)_{\theta}\biggr\} \frac{dp}{\theta}\Add{.}
\end{align*}
On the other hand,
\[
d\phi = \left(\frac{\dd \phi}{\dd \theta}\right)_{p} d\theta + \left(\frac{\dd \phi}{\dd p}\right)_{\theta} dp\Add{,}
\]
% [** TN: Next two displayed equations brace-grouped in the original]
whence,
\[
\left(\frac{\dd \phi}{\dd \theta}\right)_{p}
  = \frac{\left(\dfrac{\dd u}{\dd \theta}\right)_{p} + p\left(\dfrac{\dd v}{\dd \theta}\right)_{p}}{\theta}
\]
and
\[
\left(\frac{\dd \phi}{\dd p}\right)_{\theta}
  = \frac{\left(\dfrac{\dd u}{\dd p}\right)_{\theta} + p\left(\dfrac{\dd v}{\dd p}\right)_{\theta}}{\theta}\Add{.}
\]
Differentiating the first of these with respect to~$p$, the
second with respect to~$\theta$, we get
%[** TN: Re-breaking]
\begin{align*}
\frac{\dd^{2} \phi}{\dd \theta\, \dd p}
  &= \frac{\dfrac{\dd^{2} u}{\dd \theta\, \dd p} + p \dfrac{\dd^{2} v}{\dd \theta\, \dd p} + \left(\dfrac{\dd v}{\dd \theta}\right)_{p}}{\theta} \\
  &= \frac{\dfrac{\dd^{2} u}{\dd \theta\, \dd p} + p \dfrac{\dd^{2} v}{\dd \theta\, \dd p}}{\theta} - \frac{\left(\dfrac{\dd u}{\dd p}\right)_{\theta} + p\left(\dfrac{\dd v}{\dd p}\right)_{\theta}}{\theta^{2}}\Add{,}
\end{align*}
whence
\[
\left(\frac{\dd u}{\dd p}\right)_{\theta}
  = -\theta \left(\frac{\dd v}{\dd \theta}\right)_{p} - p\left(\frac{\dd v}{\dd p}\right)_{\theta}.
\]
The differential coefficients of~$\phi$ become, then, by~\Eq{(26)},
\[
\left(\frac{\dd \phi}{\dd \theta}\right)_{p} = \frac{c_{p}}{\theta};
\quad\text{and}\quad
\left(\frac{\dd \phi}{\dd p}\right)_{\theta} = -\left(\frac{\dd v}{\dd \theta}\right)_{p}.
\]
Finally, differentiating the first of these with respect to~$p$,
the second with respect to~$\theta$, and equating, we have
\[
\left(\frac{\dd c_{p}}{\dd p}\right)_{\theta} = -\theta \left(\frac{\dd^{2} v}{\dd \theta^{2}}\right)_{p}\Add{.}
\Tag{(85)}
\]
This equation contains only quantities that can be directly
measured, and establishes a relation between the rate of
change of the coefficient of thermal expansion of the substance
with temperature (\ie\ the deviation from Gay-Lussac's
\PageSep{124}
law), and the rate of change of the specific heat with
pressure.

\Section{158.} By means of the relations furnished by the second
\index{Joule and Thomson's absolute temperature!(theory)}%
law we may also draw a further conclusion from Thomson
and Joule's experiments (\SecRef{70}), in which a gas was slowly
pressed through a tube plugged with cotton wool. The
interpretation in~\SecRef{70} was confined to their bearing on the
properties of perfect gases. It has been mentioned that the
characteristic feature of these experiments consists in giving
to a gas---without adding or withdrawing heat\footnotemark---an increase
\footnotetext{Whether this condition is actually fulfilled may be ascertained by
  measurements in the medium surrounding the tube through which the gas
  flows.}%
of volume, $V_{2} - V_{1}$, or $v_{2} - v_{1}$ per unit mass, while the
external work done per unit mass is represented by
\[
p_{1}v_{1} - p_{2}v_{2} = W.
\]
This expression vanishes in the case of perfect gases, since
then the temperature remains constant. In the case of
actual gases we may put
\begin{alignat*}{3}
p_{1} &= p,\quad & p_{2} &= p + \Delta p\quad && (\Delta p < 0) \\
v_{1} &= v, & v_{2} &= v + \Delta v && (\Delta v > 0)
\end{alignat*}
whence
\[
W = -\Delta(pv),
\]
and by the first law, since $Q = 0$,
\[
\Delta u = W + Q = -\Delta(pv).
\]
For the sake of simplicity we shall assume $\Delta p$~and $\Delta v$
to be small, and we may then write the above equation:
\[
\left(\frac{\dd u}{\dd \theta}\right)_{v} \Delta \theta + \left(\frac{\dd u}{\dd v}\right)_{\theta} \Delta v
  = -v\, \Delta p - p\, \Delta v
\]
or, by \Eq{(24)},~\Eq{(82)}, and~\Eq{(80)},
\[
\biggl\{c_{p} - \theta \left(\frac{\dd p}{\dd \theta}\right)_{v} \left(\frac{\dd v}{\dd \theta}\right)_{p}\biggr\} \Delta \theta
  + \theta \left(\frac{\dd p}{\dd \theta}\right)_{v} \Delta v = -v\, \Delta p,
\]
\PageSep{125}
and, by~\Eq{(6)}
\begin{align*}
c_{p}\, \Delta \theta
  &= -v\, \Delta p + \theta \left(\frac{\dd v}{\dd \theta}\right)_{p}
  · \biggl\{\left(\frac{\dd p}{\dd \theta}\right)_{v} \Delta \theta + \left(\frac{\dd p}{\dd v}\right)_{\theta} \Delta v\biggr\} \\
  &= -v\, \Delta p + \theta \left(\frac{\dd v}{\dd \theta}\right)_{p} \Delta p\Add{,} \displaybreak[0] \\
\therefore
\Delta \theta
  &= \frac{\theta \left(\dfrac{\dd v}{\dd \theta}\right)_{p} - v}{c_{p}}\, \Delta p\Add{.}
\Tag{(86)}
\end{align*}
By means of this simple equation, the change of temperature
($\Delta \theta$) of the gas in Thomson and Joule's experiments,
for a difference of pressure~$\Delta p$, may be found from its
specific heat,~$c_{p}$, and its deviation from Gay-Lussac's law.
If, under constant pressure, $v$~were proportional to~$\theta$, as in
Gay-Lussac's law, then, by equation~\Eq{(86)}, $\Delta \theta = 0$, as is
really the case for perfect gases.

\Section{159.} Thomson and Joule embraced the results of their
observations in the formula
\[
\Delta \theta = \frac{\alpha}{\theta^{2}}\, \Delta p,
\]
where $\alpha$~is a constant. If we express~$p$ in atmospheres, we
have, for air,
\[
\alpha = 0.276 × (273)^{2}.
\]
No doubt the formula is only approximate. Within the
region of its validity we get, from~\Typo{86}{\Eq{(86)}},
\[
\theta \left(\frac{\dd v}{\dd \theta}\right)_{p} - v
  = c_{p} \frac{\alpha}{\theta^{2}}
\Tag{(87)}
\]
and, differentiating with respect to~$\theta$,
\[
\theta \left(\frac{\dd^{2} v}{\dd \theta^{2}}\right)_{p}
  = \frac{\alpha}{\theta^{2}} \left(\frac{\dd c_{p}}{\dd \theta}\right)_{p} - \frac{2\alpha c_{p}}{\theta^{3}}
\]
whence, by the relation~\Eq{(85)},
\[
\left(\frac{\dd c_{p}}{\dd p}\right)_{\theta} + \frac{\alpha}{\theta^{2}} \left(\frac{\dd c_{p}}{\dd \theta}\right)_{p} - \frac{2\alpha c_{p}}{\theta^{3}} = 0.
\]
\PageSep{126}
The general solution of this differential equation is
\index{Equation!deduced from Thomson and Joule's experiments}%
\[
c_{p} = \theta^{2} · f(\theta^{3} - 3\alpha p),
\]
where $f$~denotes an arbitrary function of its argument,
$\theta^{3} - 3\alpha p$.

If we now assume that, for small values of~$p$, the gas, at
any temperature, approaches indefinitely near the ideal state,
then, when $p = 0$, $c_{p}$~becomes a constant $= c_{p}^{(0)}$ (for air, $c_{p}^{(0)} = 0.238~\Unit{ calorie}$).
Hence, generally,
\[
c_{p} = c_{p}^{(0)} \theta^{2} (\theta^{3} - 3\alpha p)^{-\efrac{2}{3}}\Add{,}
\]
or
\[
c_{p} = \frac{c_{p}^{(0)}}{\left(1 - \dfrac{3\alpha p}{\theta^{3}}\right)^{\efrac{2}{3}}}\Add{.}
\Tag{(88)}
\]
This expression for~$c_{p}$ will serve further to determine~$v$ in
terms of $\theta$~and~$p$. It follows from~\Eq{(87)} that
\[
\theta^{2}\, \frac{\dd}{\dd \theta} \left(\frac{v}{\theta}\right)_{p}
  = \frac{\alpha c_{p}}{\theta^{2}}
  = \frac{\alpha c_{p}^{(0)}}{(\theta^{3} - 3\alpha p)^{\efrac{2}{3}}},
\]
whence
\[
\frac{v}{\theta} = \alpha c_{p}^{(0)}
  \Bigint \frac{d\theta}{\theta^{4} \left(1 - \dfrac{3\alpha p}{\theta^{3}}\right)^{\efrac{2}{3}}},
\]
or
\[
v = \frac{c_{p}^{(0)} \theta}{3p} \left(\sqrt[3]{1 - \frac{3\alpha p}{\theta^{3}}} + \beta\right).
\Tag{(89)}
\]
This is the characteristic equation of the gas, and $\beta$, the
\index{Characteristic equation!deduced from Thomson and Joule's experiments}%
constant of integration, may be determined from its density
at $0°$~C. and atmospheric pressure. Equations \Eq{(88)}~and \Eq{(89)},
like Thomson and Joule's formula, are valid only within
certain limits. It is, however, of theoretical interest to see
how the different relations necessarily follow from one
another.

\Section{160.} A further, theoretically important application of
the second law is the determination of the absolute temperature
of a substance by a method independent of the
\PageSep{127}
\index{Joule and Thomson's absolute temperature|(}%
deviations of actual gases from the ideal state. In \SecRef{4} we
defined temperature by means of the gas thermometer, but
had to confine that definition to the cases in which the
readings of the different gas thermometers (hydrogen, air,
etc.)\ agree as nearly as the desired accuracy of the result
requires. For all other cases (including mean temperatures,
when a high degree of accuracy is desired) we postponed
the definition of absolute temperature. Equation~\Eq{(80)}
\index{Absolute temperature!deduced from Thomson and Joule's experiments|(}%
enables us to give an exact definition of absolute temperature,
entirely independent of the behaviour of special
substances.

Given the temperature readings,~$t$, of any arbitrary
thermometer (mercury-thermometer, or the scale deflection
of a thermo-element, or of a bolometer), our problem is to
reduce the thermometer to an absolute one, or to express
the absolute temperature~$\theta$ as a function of~$t$. We may by
direct measurement find how the behaviour of some appropriate
substance, \eg\ a gas, depends on~$t$ and either $v$~or~$p$.
Introducing, then, $t$~and $v$ as the independent variables in~\Eq{(80)}
instead of $\theta$~and~$v$, we obtain
\[
\left(\frac{\dd u}{\dd v}\right)_{t}
  = \theta \left(\frac{\dd p}{\dd t}\right)_{v} · \frac{dt}{d\theta} - p,
\]
where $\left(\dfrac{\dd u}{\dd v}\right)_{t}$,~$p$, and $\left(\dfrac{\dd p}{\dd t}\right)_{v}$ represent functions of $t$~and~$v$,
which can be experimentally determined. The equation
can then be integrated thus:
\[
\int \frac{d\theta}{\theta}
  = \Bigint \frac{\left(\dfrac{\dd p}{\dd t}\right)_{v} dt}{\left(\dfrac{\dd u}{\dd v}\right)_{t} + p}.
\]
If we further stipulate that at the freezing-point of water,
where $t = t_{0}$, $\theta = \theta_{0} = 273$, then,
\[
\log \frac{\theta}{\theta_{0}}
  = \Bigintlimits{t_{0}}{t} \frac{\left(\dfrac{\dd p}{\dd t}\right)_{v} dt}{\left(\dfrac{\dd u}{\dd v}\right)_{t} + p}.
\]
\PageSep{128}
This completely determines $\theta$~as a function of~$t$. It is
evident that the volume,~$v$, no longer enters into the
expression under the sign of integration.

\Section{161.} The numerator of this expression may be found
directly from the characteristic equation of the substance.
The denominator, however, depends on the amount of heat
which the substance absorbs during isothermal reversible
expansion. For, by~\Eq{(22)} of the first law, the ratio of the
heat absorbed during isothermal reversible expansion to the
change of volume is
\[
\left(\frac{q}{dv}\right)_{t} = \left(\frac{\dd u}{dv}\right)_{t} + p.
\]

\Section{162.} Instead of measuring the quantity of heat absorbed
during isothermal expansion, it may be more convenient,
for the determination of the absolute temperature, to experiment
on the changes of temperature of a slowly escaping
gas, according to the method of Thomson and Joule. If we
introduce~$t$ (of~\SecRef{160}) instead of~$\theta$ into equation~\Eq{(86)}, which
represents the theory of those experiments on the absolute
temperature scale, we have
\begin{align*}
\Delta \theta &= \frac{d\theta}{dt}\, \Delta t\Add{,} \\
\left(\frac{\dd v}{\dd \theta}\right)_{p}
  &= \left(\frac{\dd v}{\dd t}\right)_{p} · \frac{dt}{d\theta}, \\
c_{p} &= \left(\frac{q}{d\theta}\right)_{p}
  = \left(\frac{q}{dt}\right)_{p} · \frac{dt}{d\theta}
  = c_{p}'\, \frac{dt}{d\theta},
\end{align*}
where $c_{p}'$~is the specific heat at constant pressure, determined
by a $t$~thermometer. Consequently, by~\Eq{(86)},
\[
\Delta t = \frac{\theta \left(\dfrac{\dd v}{\dd t}\right)_{p} · \dfrac{dt}{d\theta} - v}{c_{p}'}\, \Delta p
\]
and again, by integration,
\[
\log \frac{\theta}{\theta_{0}}
  = \Bigintlimits{t_{0}}{t} \frac{\left(\dfrac{\delta v}{\delta t}\right)_{p} dt}{v + c_{p}'\, \dfrac{\Delta t}{\Delta p}} = J.
\Tag{(90)}
\]
\PageSep{129}
The expression to be integrated again contains quantities
which may be measured directly with comparative ease.

\Section{163.} The stipulation of \SecRef{160}, that, at the freezing-point
of water, $\theta = \theta_{0}= 273$, implies the knowledge of the
coefficient of expansion,~$\alpha$, of perfect gases. Strictly speaking,
however, all gases show at all temperatures deviations
from the behaviour of perfect gases, and disagree with one
another. To rid ourselves of any definite assumption about~$\alpha$,
we return to our original definition of temperature, viz.\
that the difference between the absolute temperature of
water boiling under atmospheric pressure~($\theta_{1}$), and that of
water freezing under the same pressure~($\theta_{0}$), shall be
\[
\theta_{1} - \theta_{0} = 100\Add{.}
\Tag{(91)}
\]

Now, if $t_{1}$~be the boiling-point of water, measured by
means of a $t$~thermometer, then, by~\Eq{(90)},
\[
\log \frac{\theta_{1}}{\theta_{0}}
  = \Bigintlimits{t_{0}}{t_{1}} \frac{\left(\dfrac{\dd v}{\dd t}\right)_{p} dt}{v + c_{p}'\, \dfrac{\Delta t}{\Delta p}} = J_{1}\Add{,}
\Tag{(92)}
\]
and, eliminating $\theta_{0}$~and~$\theta_{1}$ from \Eq{(90)},~\Eq{(91)}, and~\Eq{(92)}, we find
the absolute temperature:
\[
\theta = \frac{100 e^{J}}{e^{J_{1}} - 1}.
\Tag{(93)}
\]
From this we obtain the coefficient of thermal expansion of
a perfect gas, independently of any gas thermometer,
\[
\alpha = \frac{1}{\theta_{0}} = \frac{e^{J_{1}} - 1}{100}.
\Tag{(94)}
\]

Since, in both $J$~and $J_{1}$, the expression to be integrated
depends necessarily on $t$~only, it is sufficient for the calculation
of the value of the integral to experiment at different
temperatures under some simplifying condition, as, for
instance, always at the same pressure (atmospheric pressure).

\Section{164.} The formula may be still further simplified by
\PageSep{130}
using as thermometric substance in the $t$~thermometer the
same gas as that on which Thomson and Joule's experiments
are being performed. The coefficient of expansion,~$\alpha'$,
referred to temperature~$t$, is then a constant, and if, as is
usual, we put $t_{0} = 0$, and $t_{1} = 100$,
\[
v = v_{0} (1 + \alpha' t),
\]
$v_{0}$~being the specific volume at the melting-point of ice
under atmospheric pressure. Also
\[
\left(\frac{\dd v}{\dd t}\right)_{p} = \alpha' v_{0}\Typo{,}{.}
\]
Hence, by~\Eq{(90)},
\[
J = \Bigintlimits{0}{t} \frac{\alpha'\, dt}{1 + \alpha' t + \dfrac{c_{p}'}{v_{0}} · \dfrac{\Delta t}{\Delta p}},
\]
and, by~\Eq{(92)},
\[
J_{1} = \Bigintlimits{0}{\PadTo[l]{t}{100}} \frac{\alpha'\, dt}{1 + \alpha' t + \dfrac{c_{p}'}{v_{0}} · \dfrac{\Delta t}{\Delta p}}.
\]
In the case of an almost perfect gas (\eg\ air), $\Delta t$~is small,
and the term $\dfrac{c_{p}'}{v_{0}} · \dfrac{\Delta t}{\Delta p}$ acts merely as a correction term,
and, therefore, no great degree of accuracy is required in
the determination of $c_{p}'$~and~$v_{0}$. For a perfect gas we should
have $\Delta t = 0$, and, from the last two equations,
\[
J = \log (1 + \alpha' t),\quad
J_{1} = \log (1 + 100 \alpha');
\]
therefore, by~\Eq{(93)},
\[
\theta = t + \frac{1}{\alpha'},
\]
and, by~\Eq{(94)},
\[
\alpha = \frac{1}{\theta_{0}} = \alpha',
\]
as it should be.
\PageSep{131}

As soon as accurate measurement of even a single
substance has determined $\theta$~as a function of~$t$, the question
regarding the value of the absolute temperature may be
considered as solved for all cases.

The absolute temperature may be determined not only
\index{Absolute temperature!deduced from Thomson and Joule's experiments|)}%
by experiments on homogeneous substances, but also from
\index{Homogeneous!system|)}%
the theory of heterogeneous substances (\Typo{cf.}{\cf}~\SecRef{177}).
\index{Joule and Thomson's absolute temperature|)}%
\index{System!homogeneous|)}%
\PageSep{132}


\Chapter{II.}{System in Different States of Aggregation.}
\index{Aggregation, states of}%
\index{States of aggregation}%

\Section{165.} \First{We} shall discuss in this chapter the equilibrium of
a system which may consist of solid, liquid, and gaseous
portions. We assume that the state of each of these
portions is fully determined by mass, temperature, and
volume; or, in other words, that the system is formed of
but one independent constituent (\SecRef{198}). For this it is not
necessary that any portion of the system should be chemically
homogeneous. Indeed, the question with regard to
the chemical homogeneity cannot, in general, be completely
answered (\SecRef{92}). It is still very uncertain whether the
molecules of liquid water are the same as those of ice. In
fact, the anomalous properties of water in the neighbourhood
of its freezing-point make it probable that even in
the liquid state its molecules are of different kinds. The
decision of such questions has no bearing on the investigations
of this chapter. The system may even consist of
a mixture of substances in any proportion; that is, it may
be a solution or an alloy. What we assume is only this:
that the state of each of its homogeneous portions is quite
definite when the temperature~$\theta$ and the specific volume~$v$
are definitely given, and that, if the system consists of
different substances, their proportion is the same in all
portions of the system. We may now enunciate our problem
in the following manner:---

Let us imagine a substance of given total mass,~$M$,
enclosed in a receptacle of volume,~$V$, and the energy,~$U$,
added to it by heat-conduction. If the system be now
isolated and left to itself, $M$,~$V$, and~$U$ will remain constant,
while the entropy,~$\Phi$, will increase. We shall now
\PageSep{133}
investigate the state or states of equilibrium which the
system may assume, finding at the same time the conditions
of its stability or instability. This investigation may be
completely carried through by means of the proposition
expressed in equation~\Eq{(77)}, that of all the states that may
adiabatically arise from one another, the most stable state
of equilibrium is characterized by an absolute maximum of
the entropy. The entropy may in general, however, as
we shall see, assume several relative maxima, under the
given external conditions. Each maximum, which is not
the absolute one, will correspond to a more or less unstable
equilibrium. The system in a state of this kind
(\eg\ as supersaturated vapour) may occasionally, upon
appropriate, very slight disturbances, undergo a finite
change, and pass into another state of equilibrium, which
necessarily corresponds to a greater value of the entropy.

\Section{166.} We have now to find, first of all, the states in which
the entropy~$\Phi$ becomes a maximum. The most general assumption
regarding the state of the system is that it consists
of a solid, a liquid, and a gaseous portion. Denoting the
masses of these portions by $M_{1}$,~$M_{2}$,~$M_{3}$, but leaving open,
for the present, the question as to which particular portion
each suffix refers, we have for the entire mass of the system
$M_{1} + M_{2} + M_{3} = M$. All the quantities are positive, but
some may be zero. Further, since the state under discussion
is to be one of equilibrium, each portion of the system, also
when taken alone, must be in equilibrium, and therefore of
uniform temperature and density. To each of them, therefore,
we may apply the propositions which were deduced in
the preceding chapter for homogeneous substances. If
$v_{1}$,~$v_{2}$,~$v_{3}$, denote the specific volumes, the given volume of
the system is
\[
M_{1} v_{1} + M_{2} v_{2} + M_{3} v_{3} = V.
\]
Similarly, the given energy is
\[
M_{1} u_{1} + M_{2} u_{2} + M_{3} u_{3} = U,
\]
where $u_{1}$,~$u_{2}$,~$u_{3}$ denote the specific energies of the portions.
\PageSep{134}

These three equations represent the given \emph{external}
conditions.

\Section{167.} For the entropy of the system we have
\[
\Phi = M_{1} \phi_{1} + M_{2} \phi_{2} + M_{3} \phi_{3},
\]
$\phi_{1}$,~$\phi_{2}$,~$\phi_{3}$ being the specific entropies.

For an infinitesimal change of state this equation gives
\[
\delta \Phi = \tsum M_{1}\, \delta \phi_{1} + \tsum \phi_{1}\, \delta M_{1}.
\]
Since, by~\Eq{(61)}, we have, in general,
\[
\delta \phi = \frac{\delta u + p\, \delta v}{\theta}.
\]
we obtain
\[
\delta \Phi = \tsum \frac{M_{1}\, \delta u_{1}}{\theta_{1}}
  + \tsum \frac{M_{1}p_{1}\, \delta v_{1}}{\theta_{1}}
  + \tsum \phi_{1}\, \delta M_{1}\Add{.}
\Tag{(95)}
\]
These variations are not all independent of one another.
In fact, from the equations of the imposed (external) conditions,
it follows that
\[
\left.
\begin{aligned}
\tsum \delta M_{1} &= 0\Add{,} \\
\tsum \delta M_{1}\, \delta v_{1} + \tsum v_{1}\, \delta M_{1} &= 0\Add{,} \\
\tsum \delta M_{1}\, \delta u_{1} + \tsum u_{1}\, \delta M_{1} &= 0\Add{.} \\
\end{aligned}
\right\}
\Tag{(96)}
\]
With the help of these equations we must eliminate
from~\Eq{(95)} any three variations, in order that it may contain
only independent variations. If we substitute in~\Eq{(95)}, for
instance, the values for $\delta M_{2}$,~$\delta v_{2}$, and~$\delta u_{2}$ taken from~\Eq{(96)}, the
equation for~$\delta \Phi$ becomes
\[
\left.
\begin{aligned}
\delta \Phi
  &= \left(\frac{1}{\theta_{1}} - \frac{1}{\theta_{2}}\right) M_{1}\, \delta u_{1}
   - \left(\frac{1}{\theta_{2}} - \frac{1}{\theta_{3}}\right) M_{3}\, \delta u_{3} \\
  &+ \left(\frac{p_{1}}{\theta_{1}} - \frac{p_{2}}{\theta_{2}}\right) M_{1}\, \delta v_{1}
   - \left(\frac{p_{2}}{\theta_{2}} - \frac{p_{3}}{\theta_{3}}\right) M_{3}\, \delta v_{3} \\
  &+ \left(\phi_{1} - \phi_{2} - \frac{u_{1} - u_{2}}{\theta_{2}} - \frac{p_{2}(v_{1} - v_{2})}{\theta_{2}}\right) \delta M_{1} \\
  &- \left(\phi_{2} - \phi_{3} - \frac{u_{2} - u_{3}}{\theta_{2}} - \frac{p_{2}(v_{2} - v_{3})}{\theta_{2}}\right) \delta M_{3}\Add{.} \\
\end{aligned}
\right\}
\Tag{(97)}
\]
\PageSep{135}
Since the six variations occurring in this expression are
now independent of one another, it is necessary that each of
their six coefficients should vanish, in order that $\delta \Phi$~may be
zero for all changes of state. Therefore
\[
\left.
\begin{aligned}
\theta_{1} &= \theta_{2} = \theta_{3} (= \theta)\Add{,} \\
p_{1} &= p_{2} = p_{3}\Add{,} \\
\phi_{1} - \phi_{2} &= \frac{(u_{1} - u_{2}) + p_{1}(v_{1} - v_{2})}{\theta}\Add{,} \\
\phi_{2} - \phi_{3} &= \frac{(u_{2} - u_{3}) + p_{2}(v_{2} - v_{3})}{\theta}\Add{.} \\
\end{aligned}
\right\}
\Tag{(98)}
\]
These six equations represent necessary properties of any
state, which corresponds to a maximum value of the entropy,
\ie\ of any state of equilibrium. As the first four refer to
equality of temperature and pressure, the main interest
centres in the last two, which contain the thermodynamical
theory of fusion, evaporation, and sublimation.
\index{Evaporation!theory of}%

\Section{168.} These two equations may be considerably simplified
\index{Fusion, curve!theory of}%
\index{Sublimation, curve!theory of}%
\index{Thermodynamical theory!of fusion, vaporization, and sublimation}%
\index{Vaporization curve!theory of}%
by substituting the value of the specific entropy~$\phi$,
which, as well as $u$~and~$p$, is here considered as a function of
$\theta$~and~$v$. For, since \Eq{(61)}~gives, in general,
\[
d\phi = \frac{du + p\, dv}{\theta},
\]
we get, by integration,
\[
\phi_{1} - \phi_{2} = \int_{2}^{1} \frac{du + p\, dv}{\theta},
\]
where the upper limit of the integral is characterized by the
values $\theta_{1}$,~$v_{1}$, the lower by $\theta_{2}$,~$v_{2}$. The path of integration
is arbitrary, and does not influence the value of $\phi_{1} - \phi_{2}$.
Since, now, $\theta_{1} = \theta_{2} = \theta$ (by~\Typo{98}{\Eq{(98)}}), we may select an isothermal
path of integration ($\theta = \const$). This gives
\[
\phi_{1} - \phi_{2} = \frac{u_{1} - u_{2}}{\theta} + \frac{1}{\theta} \int_{v_{2}}^{v_{1}} p\, dv.
\]
\PageSep{136}
The integration is to be taken along an isotherm, since $p$~is
a known function of $\theta$~and $v$ determined by the characteristic
equation of the substance. Substituting the value
of $\phi_{1} - \phi_{2}$ in the equations~\Eq{(98)}, we have the relations:
\[
\left.
\begin{aligned}
\int_{v_{2}}^{v_{1}} p\, dv &= p_{1}(v_{1} - v_{2})\Add{,} \\
\int_{v_{3}}^{v_{2}} p\, dv &= p_{2}(v_{2} - v_{3})\Add{,} \\
\llap{\text{to which we add}\qquad\qquad}
p_{1} &= p_{2} = p_{3}\Add{.}
\end{aligned}
\right\}
\Tag{(99)}
\]
With the four unknowns $\theta$, $v_{1}$, $v_{2}$,~$v_{3}$, we have four equations
which the state of equilibrium must satisfy. The constants
\index{Equilibrium!conditions of}%
which occur in these equations depend obviously only on
the chemical nature of the substance, and in no way on the
given values of the mass,~$M$, the volume,~$V$, and the energy,~$U$,
of the system. The equations~\Eq{(99)} might therefore be
called the system's \emph{internal} or \emph{intrinsic} conditions of equilibrium,
while those of~\SecRef{166} represent the \emph{external} conditions
imposed on the system.

\Section{169.} Before discussing the values which the equations~\Eq{(99)}
\index{Condition of complete reversibility!of equilibrium}%
\index{External conditions of equilibrium}%
\index{Internal conditions of equilibrium}%
give to the unknowns, we shall investigate generally
whether, and under what condition, they lead to a maximum
value of the entropy and not to a minimum value. It is
necessary, for this purpose, to find the value of~$\delta^{2} \Phi$. If
this be negative for all virtual changes, then the state
considered is certainly one of maximum entropy.

From the expression for~$\delta \Phi$ \Eq{(97)} we obtain~$\delta^{2} \Phi$, which
may be greatly simplified with the help of the equations~\Eq{(98)}.
The equations of the imposed external conditions,
and the equations~\Eq{(96)} further simplify the result, and we
obtain, finally,
\[
\delta^{2} \Phi
  = -\tsum \frac{M_{1}\, \delta \phi_{1}\, \delta \theta_{1}}{\theta_{1}}
   + \tsum \frac{M_{1}\, \delta p_{1}\, \delta v_{1}}{\theta_{1}}.
\]
This may be written
\[
\theta\, \delta^{2} \Phi = -\tsum M_{1} (\delta \phi_{1}\, \delta \theta_{1} - \delta p_{1}\, \delta v_{1})\Add{.}
\]
\PageSep{137}

To reduce all variations to those of the independent
variables, $\theta$~and~$v$, we may write, according to~\Eq{(81)},
\begin{align*}
\delta \phi &= \left(\frac{\dd \phi}{\dd \theta}\right)_{v} \delta \theta + \left(\frac{\dd \phi}{\dd v}\right)_{\theta} \delta v
  = \frac{c_{v}}{\theta}\, \delta \theta + \left(\frac{\dd p}{\dd \theta}\right)_{v} \delta v \\
\intertext{and}
\delta p &= \left(\frac{\dd p}{\dd \theta}\right)_{v} \delta \theta + \left(\frac{\dd p}{\dd v}\right)_{\theta} \delta v, \\
\therefore
\theta\, \delta^{2} \Phi &= -\tsum M_{1} \left(\frac{(c_{v})_{1}}{\theta}\, \delta \theta_{1}^{2} - \left(\frac{\dd p_{1}}{\dd v}\right)_{\theta} \delta v_{1}^{2}\right).
\Tag{(100)}
\end{align*}
Obviously, if the quantities $(c_{v})_{1}$, $(c_{v})_{2}$, $(c_{v})_{3}$ be all positive,
and the quantities $\left(\dfrac{\dd p_{1}}{\dd v}\right)_{\theta}$\Add{,}~\dots all negative, $\delta^{2} \Phi$~is
negative in all cases, and $\Phi$~is really a maximum, and the
corresponding state is a state of equilibrium. Since $c_{v}$~is
the specific heat at constant volume, and therefore always
positive, the condition of equilibrium depends on whether
$\left(\dfrac{\dd p}{\dd v}\right)_{\theta}$~is negative for all three portions of the system or not.
In the latter case there is no equilibrium. Experience
immediately shows, however, that in any state of equilibrium
$\dfrac{\dd p}{\dd v}$~is negative, since the pressure, whether positive
or negative, and the volume always change in opposite
directions. A glance at the graphical representation of~$p$,
as an isothermal function of~$v$ (\Fig{1}, \SecRef{26}), shows that
there are certain states of the system in which $\dfrac{\dd p}{\dd v}$~is positive.
These, however, can never be states of equilibrium,
and are, therefore, not accessible to direct observation. If,
on the other hand, $\dfrac{\dd p}{\dd v}$~be negative, it is a state of equilibrium,
yet it need not be stable; for another state of
equilibrium may be found to exist which corresponds to
a greater value of the entropy.

We shall now discuss the values of the unknowns, $\theta$,~$v_{1}$,
$v_{2}$,~$v_{3}$, which represent solutions of the conditions of equilibrium~\Eq{(98)}.
Several such systems may be found. Thereafter,
we shall deal (beginning at \SecRef{189}) with the further
\PageSep{138}
question as to which of the different solutions in each case
represents the most stable equilibrium under the given
external conditions; \ie\ which one leads to the largest
value of the entropy of the system.

\Section{170.} \Topic{First Solution.}---If we put, in the first place,
\[
v_{1} = v_{2} = v_{3} (= v)
\]
all the equations~\Eq{(98)} are satisfied, for, since the temperature
is common to all three portions of the system, their
states become absolutely identical. The entire system is,
therefore, homogeneous. The state of the system is determined
\index{Homogeneous!substance}%
by the equations of~\SecRef{166}, which give the imposed
conditions. In this case they are
\begin{align*}
M_{1} + M_{2} + M_{3} &= M\Add{,} \\
v(M_{1} + M_{2} + M_{3}) &= V\Add{,} \\
u(M_{1} + M_{2} + M_{3}) &= U\Add{,} \\
\therefore
v = \frac{V}{M}\quad\text{and}\quad u &= \frac{U}{M}.
\end{align*}
From $v$~and~$u$, $\theta$~may be found, since $u$~was assumed to be
a known function of $\theta$~and~$v$.

This solution has always a definite meaning; but, as we
saw in equation~\Eq{(100)}, it represents a state of equilibrium
only when $\dfrac{\dd p}{\dd v}$~is negative. If this be the case, then, the
equilibrium is stable or unstable, according as under the
external conditions there exists a state of greater entropy
or not. This will be discussed later.

\Section{171.} \Topic{Second Solution.}---If, in the second case, we put
\[
v_{1} \gtrless v_{2},\quad v_{2} = v_{3},
\]
the states $2$~and $3$ coincide, and the equations~\Eq{(98)} reduce to
\[
\left.
\begin{aligned}
p_{1} &= p_{2}\Add{,} \\
\phi_{1} - \phi_{2} &= \frac{u_{1} - u_{2} + p_{1}(v_{1} - v_{2})}{\theta}\Add{,}
\end{aligned}
\right\}
\Tag{(101)}
\]
\PageSep{139}
or, instead of the second of these equations,
\[
\int_{v_{2}}^{v_{1}} p\, dv = p_{1} (v_{1} - v_{2})\Add{.}
\Tag{(102)}
\]
In this case two states of the system coexist; for instance,
the vapour and the liquid. The equations~\Eq{(101)} contain
three unknowns, $\theta$,~$v_{1}$,~$v_{2}$; and hence may serve to express $v_{1}$~and
$v_{2}$, consequently also the pressure $p_{1} = p_{2}$, and the
specific energies $u_{1}$~and~$u_{2}$, as definite functions of the
temperature~$\theta$. The internal state of two heterogeneous
portions of the same substance in contact with one another
is, therefore, completely determined by the temperature.
The temperature, as well as the masses of the two portions,
may be found from the imposed conditions (\SecRef{166}), which
are, in this case,
\[
\left.
\begin{aligned}
M_{1} + (M_{2} + M_{3}) &= M\Add{,} \\
M_{1} v_{1} + (M_{2} + M_{3}) v_{2} &= V\Add{,} \\
M_{1} u_{1} + (M_{2} + M_{3}) u_{2} &= U\Add{.} \\
\end{aligned}
\right\}
\Tag{(103)}
\]
These equations serve for the determination of the three
last unknowns, $\theta$,~$M_{1}$, and $M_{2} + M_{3}$. This completely
determines the physical state, for, in the case of the masses
$M_{2}$~and~$M_{3}$, it is obviously sufficient to know their sum.
Of course, the result can only bear a physical interpretation
if both $M_{1}$~and $M_{2} + M_{3}$ have positive values.

\Section{172.} An examination of equation~\Eq{(102)} shows that it
can be satisfied only if the pressure,~$p$, which is known to
have the same value ($p_{1} = p_{2}$) for both limits of the integral,
assume between the limits values which are partly larger
and partly smaller than~$p_{1}$. Some of these, then, must
correspond to unstable states (\SecRef{169}), since in certain places
$p$~and $v$ increase simultaneously $\left(\dfrac{\dd p}{\dd v} > 0\right)$. The equation
admits of a simple geometrical interpretation with the help
of the above-mentioned graphical representation of the
characteristic equation by isotherms (\Fig{1}, \SecRef{26}). For
the integral $\displaystyle\int_{2}^{1} p\, dv$ is represented by the area bounded by
\PageSep{140}
\index{Latent heat}%
the isotherm, the axis of abscissæ, and the ordinates at $v_{1}$~and
$v_{2}$, while the product $p_{1} (v_{1} - v_{2})$ is the rectangle formed
by the same ordinates ($p_{1} = p_{2}$), and the length ($v_{1} - v_{2}$).
We learn, therefore, from equation~\Eq{(102)} that in every
isotherm the pressure, under which two states of aggregation
of the substance may be kept in lasting contact, is represented
by the ordinate of the straight line parallel to the
axis of abscissæ, which intercepts equal areas on both sides
of the isotherm. Such a line is represented by~$ABC$ in
\Fig{1}. We are thus enabled to deduce directly from the
characteristic equation for homogeneous, stable and unstable,
states the functional relation between the pressure, the
density of the saturated vapour and of the liquid in contact
with it, and the temperature.

Taking Clausius' equation~\Eq{(12)} as an empirical expression
\index{Clausius' equation}%
\index{Equation!Clausius'}%
of the facts, we have, for the specific volume~$v_{1}$ of the
saturated vapour, and $v_{2}$~of the liquid in contact with it,
the two conditions
\[
\frac{R\theta}{v_{1} - a} - \frac{c}{\theta (v_{1} + b)^{2}}
  = \frac{R\theta}{v_{2} - a} - \frac{c}{\theta (v_{2} + b)^{2}},
\]
and, from~\Eq{(102)},
% [** TN: Re-breaking]
\begin{multline*}
R\theta \log \frac{v_{1} - a}{v_{2} - a} - \frac{c}{\theta} \left(\frac{1}{v_{2} + b} - \frac{1}{v_{1} + b}\right) \\
  = (v_{1} - v_{2}) \left(\frac{R\theta}{v_{1} - a} - \frac{c}{\theta (v_{1} + b)^{2}}\right).
\end{multline*}
By means of these $v_{1}$,~$v_{2}$ and $p_{1} = p_{2}$ may be expressed as
functions of~$\theta$, or, still more conveniently, $v_{1}$,~$v_{2}$, $p_{1}$, and~$\theta$ as
functions of some appropriately selected independent variable.

With Clausius' values of the constants for carbon dioxide
(\SecRef{25}), this calculation furnishes results which show a satisfactory
agreement with Andrews' observations. According
\index{Andrews}%
to Thiesen, however, Clausius' equation is by no means the
\index{Thiesen}%
general form of the characteristic equation.

\Section{173.} We shall now follow the interpretation of the
equation~\Eq{(101)} in other directions. If we put, for shortness,
\[
u - \theta \phi = f
\Tag{(104)}
\]
%[** TN: Next line set is small type in the original, as if squeezed in]
(free energy per unit mass, by equation~\Eq{(71)}),
\PageSep{141}
the equations~\Eq{(101)} become, simply,
\begin{align*}
p_{1} &= p_{2}\Add{,}
\Tag{(105)} \\
f_{2} - f_{1} &= p_{1} (v_{1} - v_{2})\Add{.}
\Tag{(106)}
\end{align*}
The function~$f$ satisfies the following simple conditions.

By~\Eq{(104)},
\begin{align*}
\left(\frac{\dd f}{\dd \theta}\right)_{v}
  &= \left(\frac{\dd u}{\dd \theta}\right)_{v} - \theta \left(\frac{\dd \phi}{\dd \theta}\right)_{v} - \phi.
\intertext{By~\Eq{(79a)},}
\left(\frac{\dd f}{\dd \theta}\right)_{v}
  &=  -\phi.
\Tag{(107)}
\intertext{Also, by~\Eq{(104)},}
\left(\frac{\dd f}{\dd v}\right)_{\theta}
  &= \left(\frac{\dd u}{\dd v}\right)_{\theta} - \theta \left(\frac{\dd \phi}{\dd v}\right)_{\theta}
\intertext{and, by \Eq{(80)} and \Eq{(81)},}
\left(\frac{\dd f}{\dd v}\right)_{\theta} &= -p.
\Tag{(108)}
\end{align*}

The conditions of equilibrium for two states of aggregation
in mutual contact hold for the three possible combinations
of the solid and liquid, liquid and gaseous,
gaseous and solid states. In order to fix our ideas, however,
we shall discuss that solution of those equations which
corresponds to the contact of a liquid with its vapour.
Denoting the vapour by the subscript~$1$, the liquid by~$2$, $v_{1}$~is
then the specific volume of the saturated vapour at the
temperature~$\theta$; $p_{1} = p_{2}$, its pressure; $v_{2}$~the specific volume
of the liquid with which it is in contact. All these
quantities, then, are functions of the temperature only,
which agrees with experience.

\Section{174.} Further theorems may be arrived at by the
differentiation of the conditions of equilibrium with respect
to~$\theta$. Since all variables now depend only on~$\theta$, we shall
use~$\dfrac{d}{d\theta}$ to indicate this total differentiation, while partial
differentiation with regard to~$\theta$ at constant~$v$ will be
expressed, as hitherto, by~$\dfrac{\dd}{\dd \theta}$.
\PageSep{142}

Equations \Eq{(105)} and \Eq{(106)}, thus differentiated, give
\[
\frac{dp_{1}}{d\theta} = \frac{dp_{2}}{d\theta}
\]
and
\[
\frac{df_{2}}{d\theta} - \frac{df_{1}}{d\theta}
  = (v_{1} - v_{2})\, \frac{dp_{1}}{d\theta} + p_{1}\left(\frac{dv_{1}}{d\theta} - \frac{dv_{2}}{d\theta}\right).
\]
But, by \Eq{(107)} and \Eq{(108)}, we have
\iffalse
\begin{align*}
\frac{df_{2}}{d\theta} - \frac{df_{1}}{d\theta}
  &\begin{aligned}[t]
     &= \left(\frac{\dd f_{2}}{\dd \theta}\right)_{v} + \left(\frac{\dd f_{2}}{\dd v}\right)_{\theta} · \frac{dv_{2}}{d\theta} \\
     &- \left(\frac{\dd f_{1}}{\dd \theta}\right)_{v} - \left(\frac{\dd f_{1}}{\dd v}\right)_{\theta} · \frac{dv_{1}}{d\theta}
   \end{aligned} \\
  &= -\phi_{2} - p_{2}\, \frac{dv_{2}}{d\theta} + \phi_{1} + p_{1}\, \frac{dv_{1}}{d\theta},
\end{align*}
\fi
\begin{align*}
\frac{df_{2}}{d\theta} - \frac{df_{1}}{d\theta}
  &= \left(\frac{\dd f_{2}}{\dd \theta}\right)_{v} \!\!+ \left(\frac{\dd f_{2}}{\dd v}\right)_{\theta} · \frac{dv_{2}}{d\theta}
   - \left(\frac{\dd f_{1}}{\dd \theta}\right)_{v} \!\!- \left(\frac{\dd f_{1}}{\dd v}\right)_{\theta} · \frac{dv_{1}}{d\theta} \\
  &= -\phi_{2} - p_{2}\, \frac{dv_{2}}{d\theta} + \phi_{1} + p_{1}\, \frac{dv_{1}}{d\theta},
\end{align*}
whence, by substitution,
\[
\phi_{1} - \phi_{2} = (v_{1} - v_{2})\, \frac{dp_{1}}{d\theta},
\]
or, finally, by~\Eq{(101)},
\[
(u_{1} - u_{2}) + p_{1} (v_{1} - v_{2}) = \theta (v_{1} - v_{2})\, \frac{dp_{1}}{d\theta}.
\Tag{(109)}
\]
Here the left-hand side of the equation, according to
equation~\Eq{(17)} of the first law, represents the heat of
vaporization,~$L$, of the liquid. It is the heat which must
be added to unit mass of the liquid, in order to completely
change it to vapour under the constant pressure of its
saturated vapour. For the corresponding change of energy
is $u_{1} - u_{2}$, and the external work performed, here negative,
amounts to
\begin{align*}
W &= -p_{1}(v_{1} - v_{2}) \\
\therefore
L &= u_{1} - u_{2} + p_{1}(v_{1} - v_{2}),
\Tag{(110)} \\
\intertext{whence}
L &= \theta (v_{1} - v_{2}).
\Tag{(111)}
\end{align*}
This equation, deduced by Clapeyron from Carnot's theory,
\index{Clapeyron}%
but first rigorously proved by Clausius, may be used
for the determination of the heat of vaporization at
any temperature, if we know the specific volumes of the
saturated vapour and the liquid, as well as the relation
between the pressure of the saturated vapour and the
\PageSep{143}
temperature. This formula has been verified by experiment
in a large number of cases.

\Section{175.} As an example, we shall calculate the heat of
\index{Heat, latent!approximation formula}%
vaporization of water at $100°$~C., \ie\ under atmospheric
pressure, from the following data:---
\settowidth{\TmpLen}{vapour at $100°$ in $\Unit{cm.}^{3}$, according to Wüllner.}%
\begin{align*}
\theta &= 273 + 100 = 373. \\
v_{1} &= 1658\quad\parbox[t]{\TmpLen}{(volume of $1~\Unit{gr.}$ of saturated water \\
vapour at $100°$ in $\Chg{\Unit{c.c.}}{\Unit{cm.}^{3}}$, according to Wüllner).} \\
v_{2} &= 1\quad\text{(volume of $1~\Unit{gr.}$ of water at $100°$ in $\Chg{\Unit{c.c.}}{\Unit{cm.}^{3}}$).}
\end{align*}
$\dfrac{dp_{1}}{d\theta}$~is found from the experiments of Regnault. Saturated
\index{Regnault}%
water vapour at $100°$~C. gave an increase of pressure of
$27.2~\Unit{mm.}$ of mercury for a rise of $1°$~C\@. In absolute units,
by~\SecRef{7},
\[
\frac{dp_{1}}{d\theta} = \frac{ 27.2}{760} × 1013650.
\]
Thus, the required latent heat of vaporization is
\index{Latent heat}%
\[
L = \frac{373 × 1657 × 27.2 × 1013650}{760 × 419 × 10^{5}} = 535~\Unit{cal.}
\]
By direct observation Regnault found the heat of vaporization
of water at $100°$~C. to be~$536$.

\Section{176.} As equation~\Eq{(110)} shows, part of the heat of
vaporization,~$L$, corresponds to an increase of energy, and
part to external work. To find the relation between these
two it is most convenient to find the ratio of the external
work to the latent heat of vaporization, viz.\
\[
\frac{p_{1} (v_{1} - v_{2})}{L} = \frac{p_{1}}{\theta\, \dfrac{dp_{1}}{d\theta}}.
\]
In the above case $p = 760~\Unit{mm.}$, $\theta = 373$, $\dfrac{dp}{d\theta} = 27.2~\Unit{mm.}$,
and therefore,
\[
\frac{p_{1} (v_{1} - v_{2})}{L} = \frac{760}{373 × 27.2} = 0.075.
\]
\PageSep{144}
This shows that the external work forms only a small
part of the value of the latent heat of vaporization.

\Section{177.} Equation~\Eq{(111)} also leads to a method of
calculating the absolute temperature~$\theta$, when the latent
heat of vaporization,~$L$, as well as the pressure and the
density of the saturated vapour and the liquid, have been
determined by experiment in terms of any scale of
temperature~$t$ (\SecRef{160}). We have
\begin{align*}
L &= \theta (v_{1} - v_{2})\, \frac{dp_{1}}{dt} · \frac{dt}{d\theta}, \\
\therefore
\log \theta &= \int \frac{v_{1} - v_{2}}{L} · \frac{dp_{1}}{dt}\, dt,
\end{align*}
and therefore $\theta$~may be determined as a function of~$t$. It
is obvious that any equation between measurable quantities,
deduced from the second law, may be utilized for a determination
of the absolute temperature. The question as to
which of those methods deserves preference is to be decided
by the degree of accuracy to be obtained in the actual
measurements.

\Section{178.} A simple approximation formula, which in many
cases gives good, though in some, only fair results, may be
obtained by neglecting in the equation~\Eq{(111)} the specific
volume of the liquid,~$v_{2}$, in comparison with that of the
vapour,~$v_{1}$, and assuming for the vapour the characteristic
equation of a perfect gas. Then, by~\Eq{(14)},
\[
v_{1} = \frac{R\theta}{mp_{1}},
\]
where $R$~is the absolute gas constant, and $m$~the molecular
weight of the vapour. Equation~\Eq{(111)} then becomes
\[
L = \frac{R}{m} · \frac{\theta^{2}}{p_{1}} · \frac{dp_{1}}{d\theta}.
\Tag{(112)}
\]
For water at $100°$~C. we have $R = 1.971$; $m = \ce{H2O} = 18$;
\PageSep{145}
$\theta = 373$; $p_{1} = 760~\Unit{mm.}$; $\dfrac{dp_{1}}{d\theta} = 27.2~\Unit{mm}$. Hence the
latent heat of vaporization is
\[
L = \frac{1.971 × 373^{2} × 27.2}{18 × 760}
  = 545~\Unit{cal.}
\]
This value is somewhat large (\SecRef{175}). The cause of this
lies in the fact that the volume of saturated water vapour
at~$100°$ is in reality smaller than that calculated from the
characteristic equation of a perfect gas of molecular weight~$18$.
But, for this very reason, accurate measurement of the
heat of vaporization may serve as a means of estimating
from the second law the amount by which the density of a
vapour deviates from the ideal value.

Another kind of approximation formula, valid within the
same limits, is found by substituting in~\Eq{(109)} the value of
the specific energy $u_{1} = c_{v} \theta + \const$, which, by~\Eq{(39)}, holds
for perfect gases. We may put the specific energy of the
liquid $u_{2} = c_{2} \theta + \const$, if we assume its specific heat,~$c_{2}$, to
be constant, and neglect the external work. It then follows
from~\Eq{(109)} that
\[
(c_{v} - c_{2}) \theta + \const + \frac{R \theta}{m}
  = \frac{R}{m} · \frac{\theta^{2}}{p_{1}} · \frac{dp_{1}}{d\theta}.
\]
If we multiply both sides by~$\dfrac{d\theta}{\theta^{2}}$, this equation may be
integrated, term by term, and we finally obtain with the
help of~\Eq{(33)}
\[
p_{1} = ae^{-\efrac{b}{\theta}} · \theta^{\efrac{m}{R} (c_{p} - c_{2})}
\]
where $a$~and $b$ are positive constants; $c_{p}$~and $c_{2}$ the specific
heats of the vapour and the liquid, at constant pressure. This
relation between the pressure of the saturated vapour and
the temperature is the more approximately true, the further
the temperature lies below the critical temperature of the
vapour.
\PageSep{146}

For mercury vapour, for example, according to a calculation
\index{Hertz, H.}%
by H.~Hertz, if $p_{1}$~be given in~$\Unit{mm.}$ of mercury,
\[
a = 3.915 × 10^{10};\quad
b = 7695;\quad
\frac{m}{R} (c_{p} - c_{2}) = - 0.847.
\]

\Section{179.} Equation~\Eq{(111)} is applicable to the processes of
fusion and sublimation in the same manner as to that of
evaporation. In the first case $L$~denotes the latent heat of
fusion of the substance, if the subscript~$1$ correspond to the
liquid state and $2$~to the solid state, and $p_{1}$~the melting
pressure, \ie\ the pressure under which the solid and the
liquid substance may be in contact and in equilibrium. The
melting pressure, therefore, just as the pressure of evaporation,
depends on the temperature only. Conversely, a
change of pressure produces a change in the melting point:
\[
\frac{d\theta}{dp_{1}} = \frac{\theta (v_{1} - v_{2})}{L}.
\Tag{(113)}
\]

For ice at $0°$~C. and under atmospheric pressure, we have
\settowidth{\TmpLen}{(heat of fusion of $1~\Unit{gr.}$}%
\begin{align*}
%[** TN: Re-breaking text on following line]
L &= 80 × 419 × 10^{5}\quad\parbox[t]{\TmpLen}{(heat of fusion of $1~\Unit{gr.}$ \\
of ice in C.G.S. units);} \\
\theta &= 273; \\
v_{1} &= 1.0\quad\text{(vol.\ of $1~\Unit{gr.}$ of water at $0°$~C. in~$\Chg{\Unit{c.c.}}{\Unit{cm.}^{3}}$);} \\
v_{2} &= 1.09\quad\text{(vol.\ of $1~\Unit{gr.}$ of ice at $0°$~C. in~$\Chg{\Unit{c.c.}}{\Unit{cm.}^{3}}$).}
\end{align*}
To obtain~$\dfrac{d\theta}{dp_{1}}$ in atmospheres we must multiply by $1,013,650$:
\[
\frac{d\theta}{dp_{1}}
  = -\frac{273 × 0.09 × 1013650}
          {80 × 419 × 10^{5}}
  = - 0.0074.
\Tag{(114)}
\]
On increasing the external pressure by $1$~atmosphere, the
melting point of ice will, therefore, be lowered by $0.0074°$~C.;
\index{Melting point of ice}%
\index{Melting point of ice!lowering of, by pressure}%
or, to lower the melting point of ice by $1°$~C., the pressure
must be increased by about $130$~atmospheres. This was first
verified by the measurements of W.~Thomson (Lord Kelvin).
\index{Thomson}%
Equation~\Eq{(113)} shows that, conversely, the melting point of
substances, which expand on melting, is raised by an increase
\PageSep{147}
of pressure. This has been qualitatively and quantitatively
verified by experiment.

\Section{180.} By means of the equations~\Eq{(101)} still further
\index{Second law of thermodynamics!test of}%
important properties of substances in different states may
be shown to depend on one another. From these, along
with~\Eq{(110)}, we obtain
\[
\frac{L}{\theta} = \phi_{1} - \phi_{2}.
\]
Differentiating this with respect to~$\theta$, we have
\begin{align*}
%[** TN: Re-breaking]
\frac{1}{\theta} · \frac{dL}{d\theta} - \frac{L}{\theta^{4}}
  &= \left(\frac{\dd \phi_{1}}{\dd \theta}\right)_{v}
  + \left(\frac{\dd \phi_{1}}{\dd v}\right)_{\theta} · \frac{dv_{1}}{d\theta} \\
  &- \left(\frac{\dd \phi_{2}}{\dd \theta}\right)_{v}
   - \left(\frac{\dd \phi_{2}}{\dd v}\right)_{\theta} · \frac{dv_{2}}{d\theta},
\intertext{or, by~\Eq{(81)},}
  &= \frac{(c_{v})_{1}}{\theta} + \left(\frac{\dd p_{1}}{\dd \theta}\right)_{v} · \frac{dv_{1}}{d\theta}
   - \frac{(c_{v})_{2}}{\theta} - \left(\frac{\dd p_{2}}{\dd \theta}\right)_{v} · \frac{dv_{2}}{d\theta}.
\end{align*}
We now introduce~$c_{p}$, the specific heat at constant pressure,
for~$c_{v}$, that at constant volume, and obtain by~\Eq{(82)}, on
multiplying by~$\theta$,
\begin{align*}
\frac{dL}{d\theta} - \frac{L}{\theta}
  &= (c_{p})_{1} - \theta \left(\frac{\dd p_{1}}{\dd \theta}\right)_{v} · \left(\frac{\dd v_{1}}{\dd \theta}\right)_{p}
  + \theta \left(\frac{\dd p_{1}}{\dd \theta}\right)_{v} · \frac{dv_{1}}{d\theta} \\
  &- (c_{p})_{2} + \theta \left(\frac{\dd p_{2}}{\dd \theta}\right)_{v} · \left(\frac{\dd v_{2}}{\dd \theta}\right)_{p}
  - \theta \left(\frac{\dd p_{2}}{\dd \theta}\right)_{v} · \frac{dv_{2}}{d\theta},
\end{align*}
or, since for both states, according to~\Eq{(6)},
\[
\left(\frac{\dd p}{\dd \theta}\right)_{v}
  = -\left(\frac{\dd v}{\dd \theta}\right)_{p} · \left(\frac{\dd p}{\dd v}\right)_{\theta},
\]
\begin{align*}
\frac{dL}{d\theta} - \frac{L}{\theta}
  &= (c_{p})_{1} - \theta \left(\frac{\dd v_{1}}{\dd \theta}\right)_{p} · \left[\left(\frac{\dd p_{1}}{\dd \theta}\right)_{v} + \left(\frac{\dd p_{1}}{\dd v}\right)_{\theta} · \frac{dv_{1}}{d\theta}\right] \\
  &- (c_{p})_{2} + \theta \left(\frac{\dd v_{2}}{\dd \theta}\right)_{p} · \left[\left(\frac{\dd p_{2}}{\dd \theta}\right)_{v} + \left(\frac{\dd p_{2}}{\dd v}\right)_{\theta} · \frac{dv_{2}}{d\theta}\right].
\end{align*}
Now, the expressions within the brackets are identical with
\[
\frac{dp_{1}}{d\theta} = \frac{dp_{2}}{d\theta} = \frac{L}{\theta (v_{1} - v_{2})}.
\]
\PageSep{148}
\index{Second law of thermodynamics!test of}%
This gives, finally,
\[
(c_{p})_{1} - (c_{p})_{2}
  = \frac{dL}{d\theta} - \frac{L}{\theta}
  + \frac{L}{v_{1} - v_{2}} · \left[\left(\frac{\dd v_{1}}{\dd \theta}\right)_{p} - \left(\frac{\dd v_{2}}{\dd \theta}\right)_{p}\right].
\Tag{(115)}
\]
This equation, which is rigorously true, again leads to a
test of the second law, since all the quantities in it may be
observed independently of one another.

\Section{181.} We shall again take as example saturated water
vapour at $100°$~C. (atmospheric pressure), and calculate
its specific heat at constant pressure. We have the following
\index{Specific heat!of steam}%
data:
\settowidth{\TmpLen}{decrease of the heat of vaporization with}%
\begin{align*}
(c_{p})_{2} &= 1.03\quad\text{($=$ spec.\ heat of liquid water at $100°$~C.);} \\
L &= 536; \\
\theta &= 373; \\
\frac{dL}{d\theta} &= -0.708\quad\parbox[t]{\TmpLen}{(decrease of the heat of vaporization with \\
increase of temperature, according to \\
Regnault's observations).}
\index{Regnault}%
\end{align*}
We determine~$v$, and $\left(\dfrac{\dd v_{1}}{\dd \theta}\right)_{p}$ from the observations of
Hirn, who found the volume of $1~\Unit{gr.}$ of steam, at $100°$~under
\index{Hirn}%
atmospheric pressure, to be $1650.4~\Chg{\Unit{c.c.}}{\Unit{cm.}^{3}}$; and at~$118.5°$,
$1740~\Chg{\Unit{c.c.}}{\Unit{cm.}^{3}}$, whence
\[
v_{1} = 1650.4;\quad
\left(\frac{\dd v_{1}}{\dd \theta}\right)_{p} = \frac{1740 - 1650.4}{18.5} = 4.843.
\]
Also
\[
v_{2} = 1.0;\quad
\left(\frac{\dd v_{2}}{\dd \theta}\right)_{p} = 0.001.
\]
These values, substituted in~\Eq{(115)}, give
\[
(c_{p})_{1} - (c_{p})_{2} = -0.56,
\]
or,
\[
(c_{p})_{1} = (c_{p})_{2} - 0.56 = 1.03 - 0.56 = 0.47.
\]
By direct measurement, Regnault found the mean specific
heat of steam under atmospheric pressure for temperatures
somewhat higher than $100°$~C. to be~$0.48$.
\PageSep{149}

\Section{182.} The relation~\Eq{(115)} may be simplified, but is inaccurate,
if we neglect the volume~$v_{2}$ of the liquid in comparison
with~$\Erratum{v}{v_{1}}$, that of the vapour, and apply to the vapour
the characteristic equation of a perfect gas.

Then
\begin{gather*}
v_{1} = \frac{R\theta}{mp_{1}},\displaybreak[0] \\
\left(\frac{\dd v_{1}}{\dd \theta}\right)_{p} = \frac{R}{mp_{1}},
\end{gather*}
and equation~\Eq{(115)} becomes, simply,
\[
(c_{p})_{1} - (c_{p})_{2} = \frac{dL}{d\theta}.
\]
In our example,
\begin{gather*}
(c_{p})_{1} - (c_{p})_{2} = -0.71 \\
(c_{p})_{1} = 1.03 - 0.71 = 0.32,
\end{gather*}
a value considerably too small.

\Section{183.} We shall now apply the relation~\Eq{(115)} to the
melting of ice at $0°$~C. and under atmospheric pressure.
The subscript~$1$ now refers to the liquid state, and $2$~to the
solid state. The relation between the latent heat of fusion
of ice and the temperature has probably never been measured.
It may, however, be calculated from~\Eq{(115)}, which gives
\[
\frac{dL}{d\theta}
  = (c_{p})_{1} - (c_{p})_{2} + \frac{L}{\theta}
  - \frac{L}{v_{1} - v_{2}} \left[\left(\frac{\dd v_{1}}{\dd \theta}\right)_{p} - \left(\frac{\dd v_{2}}{\dd \theta}\right)_{p}\right]
\]
in which
\settowidth{\TmpLen}{thermal coeff.\ of expansion\quad}%
% [** TN: Re-breaking]
\begin{align*}
(c_{p})_{1} &= 1\quad\text{(spec.\ heat of water at $0°$~C.);} \\
(c_{p})_{2} &= 0.505\quad\text{(spec.\ heat of ice at $0°$~C.);} \displaybreak[0] \\
L &= 80;\quad \theta = 273;\quad v_{1} = 1;\quad v_{2} = 1.09; \displaybreak[0] \\
\left(\frac{\dd v_{1}}{\dd \theta}\right)_{p}
  &= -0.00006\quad\parbox[t]{\TmpLen}{(therm.\ coeff.\ of expansion \\
    of water at $0°$~C.);} \\
\left(\frac{\dd v_{1}}{\dd \theta}\right)_{p}
  &= 0.00011\quad\parbox[t]{\TmpLen}{(thermal coeff.\ of expansion \\
  of ice at $0°$~C.).}
\end{align*}
\PageSep{150}
Hence, by the above equation,
\[
\frac{dL}{d\theta} = 0.64,
\]
\ie, if the melting point of ice be lowered $1°$~C. by an
appropriate increase of the external pressure, its heat of
fusion decreases by $0.64~\Unit{cal}$.

\Section{184.} It has been repeatedly mentioned in the early
chapters, that, besides the specific heat at constant pressure,
\index{Specific heat!of saturated vapour}%
or constant volume, any number of specific heats may be
defined according to the conditions under which the heating
takes place. Equation~\Eq{(23)} of the first law holds in each
case:
\[
c = \frac{du}{d\theta} + p\, \frac{dv}{d\theta}.
\]
In the case of saturated vapours special interest attaches
to the process of heating, which keeps them permanently in
a state of saturation. Denoting by~$h_{1}$ the specific heat of
the vapour corresponding to this process (Clausius called it
the specific heat of ``the saturated vapour''), we have
\[
h_{1} = \frac{du_{1}}{d\theta} + p_{1}\, \frac{dv_{1}}{d\theta}.
\Tag{(116)}
\]
No off-hand statement can be made with regard to the
value of~$h_{1}$; even its sign must in the mean time remain
uncertain. For, if during a rise of temperature of~$1°$ the
vapour is to remain just saturated, it must evidently be
compressed while being heated, since the specific volume of
the saturated vapour decreases as the temperature rises.
This compression, however, generates heat, and the question
is, whether the latter is so considerable that it must be in
part withdrawn by conduction, so as not to superheat the
vapour. Two cases may, therefore, arise: (1)~The heat of
compression may be considerable, and the withdrawal of
heat is necessary to maintain saturation at the higher
temperature, \ie\ $h_{1}$~is negative. (2)~The heat of compression
may be too slight to prevent the compressed vapour,
\PageSep{151}
without the addition of heat, from becoming supersaturated.
Then, $h_{1}$~has a positive value. Between the two there is a
limiting case ($h_{1} = 0$), where the heat of compression is
exactly sufficient to maintain saturation. In this case the
curve of the saturated vapour coincides with that of
adiabatic compression. Watt assumed this to be the case
for steam.

It is now easy to calculate~$h_{1}$ from the above formulæ.
Calling $h_{2}$~the corresponding specific heat of the liquid, we
have
\[
h_{2} = \frac{du_{2}}{d\theta} + p_{2}\, \frac{dv_{2}}{d\theta}.
\Tag{(117)}
\]
During heating, the liquid is kept constantly under the
pressure of its saturated vapour. Since the external
pressure, unless it amount to many atmospheres, has no
appreciable influence on the state of a liquid, the value of~$h_{2}$
practically coincides with that of the specific heat at
constant pressure,
\[
h_{2} = (c_{p})_{2}.
\Tag{(118)}
\]
Subtracting \Eq{(117)} from~\Eq{(116)}, we get
\[
h_{1} - h_{2} = \frac{d(u_{1} - u_{2})}{d\theta} + p_{1} \frac{d(v_{1} - v_{2})}{d\theta}.
\]
But \Eq{(110)}, differentiated with respect to~$\theta$, gives
\begin{align*}
\frac{dL}{d\theta} &= \frac{d(u_{1} - u_{2})}{d\theta} + p_{1} \frac{d(v_{1} - v_{2})}{d\theta} + (v_{1} - v_{2})\, \frac{dp_{1}}{d\theta}, \\
\therefore
h_{1} - h_{2} &= \frac{dL}{d\theta} - (v_{1} - v_{2})\, \frac{dp_{1}}{d\theta},
\end{align*}
or, by \Eq{(118)} and~\Eq{(111)},
\[
h_{1} = (c_{p})_{2} + \frac{dL}{d\theta} - \frac{L}{\theta}.
\]
For saturated water vapour at~$100°$, we have, as above,
\[
(c_{p})_{2} = 1.03;\quad
\frac{dL}{d\theta} = -0.71;\quad
L = 536;\quad
\theta = 373;
\]
\PageSep{152}
whence
\[
h_{1} = 1.03 - 0.71 - \tfrac{536}{373} = -1.12.
\]
Water vapour at $100°$~C. represents the first of the cases
described above, \ie\ saturated water vapour at~$100°$ is
superheated by adiabatic compression. Conversely, saturated
water vapour at~$100°$ becomes supersaturated by
adiabatic expansion. The influence of the heat of compression
(or expansion) is greater than the influence of the
increase (or decrease) of the density. Some other vapours
behave in the opposite way.

\Section{185.} It may happen that, for a given value of~$\theta$, the
\index{Critical point!temperature}%
values of $v_{1}$~and~$v_{2}$, which are fully determined by the
equation~\Eq{(101)}, become equal. Then the two states which are
in contact with one another are identical. Such a value of~$\theta$
is called a \emph{critical temperature} of the substance. From a
\index{Temperature!critical}%
purely mathematical point of view, every substance must
be supposed to have a critical temperature for each of the
three combinations, solid-liquid, liquid-gas, gas-solid. This
critical temperature, however, will not always be real. The
critical temperature~$\theta$ and the critical volume $v_{1} = v_{2}$, fully
determine the critical state. We may calculate it from the
equations~\Eq{(101)} by finding the condition that $v_{1} - v_{2}$ should
vanish. If we first assume $v_{1} - v_{2}$ to be very small, Taylor's
theorem then gives for any volume~$v$, lying between $v_{1}$
and~$v_{2}$,
\[
p = p_{2} + \left(\frac{\dd p}{\dd v}\right)_{2} (v - v_{2}) + \tfrac{1}{2} \left(\frac{\dd^{2} p}{\dd v^{2}}\right)_{2} (v - v_{2})^{2}\Add{,}
\Tag{(119)}
\]
and therefore the first equation~\Eq{(101)} becomes
% [** TN: Errata (three missing subscripts) applied to displays]
\[
p_{2} + \left(\frac{\dd p}{\dd v}\right)_{2} (v_{1} - v_{2}) + \tfrac{1}{2} \left(\frac{\dd^{2} p}{\dd v^{2}}\right)_{2} (v_{1} - v_{2})^{2} = p_{2},
\]
and equation~\Eq{(102)}, by the integration of~\Eq{(119)} with respect
to~$v$, gives
\[
p_{2} (v_{1} - v_{2})
  + \tfrac{1}{2} \left(\frac{\dd p}{\dd v}\right)_{2}\!\!\! (v_{1} - v_{2})^{2}
  + \tfrac{1}{2 · 3} \left(\frac{\dd^{2} p}{\dd v^{2}}\right)_{2}\!\!\! (v_{1} - v_{2})^{3}
  = p_{2} (v_{1} - v_{2}).
\]
\PageSep{153}
The last two equations give, as the conditions of the critical
state,
\[
\left(\frac{\dd p}{\dd v}\right)_{2} = 0,\quad\text{and}\quad
\left(\frac{\dd^{2} p}{\dd v^{2}}\right)_{2} = 0.
\]
These conditions agree with those found in~\SecRef{30}. They are
there geometrically illustrated by the curve of the critical
isotherm. In the critical state the compressibility is infinite;
so are also the thermal coefficient of expansion and the
specific heat at constant pressure; the heat of vaporization
is zero.

For all temperatures other than the critical one, the
values of $v_{1}$~and~$v_{2}$ are different. On one side of the critical
isotherm they have real, on the other imaginary values.
In this latter case our solution of the problem of equilibrium
no longer admits of a physical interpretation. Several
reasons may be given for assuming that not only for
evaporation but also for fusion in the case of many substances
there exists a real critical temperature at which the
solid and liquid states are identical (\SecRef{31} and~\SecRef{191}).

\Section{186.} \Topic{Third Solution.}---In the third place, we shall
assume that in the conditions of equilibrium~\Eq{(98)}
\[
v_{1} \gtrless v_{2} \gtrless v_{3}.
\]
We have then, without further simplification,
\[
\left.
\begin{aligned}
p_{1} &= p_{2} = p_{3}\Add{,} \\
\phi_{1} - \phi_{2} &= \frac{u_{1} - u_{2} + p_{1}(v_{1} - v_{2})}{\theta}\Add{,} \\
\phi_{2} - \phi_{3} &= \frac{u_{2} - u_{3} + p_{1}(v_{2} - v_{3})}{\theta}\Add{.}
\end{aligned}
\right\}
\Tag{(120)}
\]
These refer to a state in which the three states of aggregation
\index{Aggregation, states of!co-existence of states of}%
\index{Coexistence of states of aggregation}%
\index{States of aggregation!co-existence of}%
are simultaneously present. There are four equations,
and these assign definite values to the four unknowns $\theta$,~$v_{1}$,
$v_{2}$,~$v_{3}$. The coexistence of the three states of aggregation in
equilibrium is, therefore, possible only at a definite temperature,
and with definite densities; therefore, also, at a
\PageSep{154}
\index{Fundamental point (triple)!pressure}%
\index{Fundamental point (triple)!temperature}%
\index{Fundamental point (triple)!temperature of ice}%
definite pressure. We shall call this temperature the
\index{Pressure!fundamental}%
\index{Temperature!fundamental}%
\index{Temperature!fundamental!of ice}%
\emph{fundamental temperature}, and the corresponding pressure
the \emph{fundamental pressure} of the substance. According to
equations~\Eq{(120)}, the fundamental temperature is characterized
by the condition that at it the pressure of the saturated
vapour is equal to the pressure of fusion. It necessarily
follows, by addition of the last two equations, that this
pressure is also equal to the pressure of sublimation.

After the fundamental temperature and pressure have
been found, the external conditions of~\SecRef{166}---
\[
\left.
\begin{aligned}
M_{1} + (M_{2} + M_{3}) &= M\Add{,} \\
M_{1} v_{1} + M_{2} v_{2} + M_{3} v_{3} &= V\Add{,} \\
M_{1} u_{1} + M_{2} u_{2} + M_{3} u_{3} &= U\Add{,} \\
\end{aligned}
\right\}
\Tag{(121)}
\]
uniquely determine the masses of the three portions of
the substance. The solution, however, can be interpreted
physically only if $M_{1}$,~$M_{2}$, and~$M_{3}$ are positive.

\Section{187.} Let us determine, \eg, the fundamental state of
water. $0°$~C. is not its fundamental temperature, for at $0°$~C.
the maximum vapour pressure of water is $4.62~\Unit{mm.}$, but the
melting pressure of ice is $760~\Unit{mm}$. Now, the latter decreases
with rise of temperature, while the maximum vapour pressure
increases. A coincidence of the two is, therefore, to
be expected at a temperature somewhat higher than~$0°$~C\@.
According to equation~\Eq{(114)}, the melting point of ice rises
by $0.0074°$~C. approximately, when the pressure is lowered
from $760~\Unit{mm.}$ to $4.62~\Unit{mm}$. The fundamental temperature
of water is, then, approximately, $0.0074°$~C\@. At this
temperature the maximum vapour pressure of water nearly
coincides with the melting pressure of ice, and, therefore,
also with the maximum vapour pressure of ice. The
specific volumes of water in the three states are, therefore,
\[
v_{1} = 205,000;\quad
v_{2} = 1;\quad
v_{3} = 1.09.
\]
For all temperatures other than the fundamental temperature,
the pressure of vaporization, of fusion, and of
sublimation differ from one another.
\PageSep{155}

\Section{188.} We return once more to the intrinsic conditions
of equilibrium~\Eq{(101)} which hold for each of the three
combinations of two states of aggregation. The pressure~$p_{1}$,
and the specific volumes of the two portions of the
substance, in each case depend only on the temperature,
and are determined by~\Eq{(101)}. It is necessary, however, to
distinguish whether the saturated vapour is in contact
with the liquid or the solid, since in these two cases the
functions which express the pressure and specific volume
in terms of the temperature are quite different. The state
of the saturated vapour is determined only when there is
given, besides the temperature, the state of aggregation
with which it is in contact, whether it is in contact with the
liquid or solid. The same applies to the other two states of
aggregation. If we henceforth use the suffixes $1$,~$2$,~$3$, in
this order, to refer to the gaseous, liquid, and solid states,
we shall be obliged to use two of them when we refer to a
portion of the substance in a state of saturation. The first
of these will refer to the state of the portion considered,
the second to that of the portion with which it is in contact.
Both the symbols $v_{12}$~and $v_{13}$ thus denote the specific
volume of the saturated vapour,~$v_{12}$, in contact with the
liquid, and $v_{13}$~in contact with the solid. Similarly $v_{23}$~and
$v_{21}$,~$v_{31}$\Add{,} and~$v_{32}$ represent the specific volumes of the liquid,
and of the solid in a state of saturation. Each of these
six quantities is a definite function of the temperature
alone. The corresponding pressures are
\begin{center}
\TableFont
\begin{tabular}{ccc}
Of vaporization. & Of fusion. & Of sublimation. \\
$p_{12} = p_{21}$ &
$p_{23} = p_{32}$ &
$p_{31} = p_{13}$ \\
\end{tabular}
\end{center}
These are also functions of the temperature alone. Only
at the fundamental temperature do two of these pressures
become equal, and therefore equal to the third. If we
represent the relation between these three pressures and the
temperature by three curves, the temperatures as abscissæ
and the pressures as ordinates, these curves will meet in one
point, the \emph{fundamental point}, also called the \emph{triple point}.
\index{Fundamental point (triple)}%
\index{Point!triple}%
\index{Triple point}%
\PageSep{156}
The inclination of the curves to the abscissa is given by
the differential coefficients
\[
\frac{dp_{12}}{d\theta};\quad
\frac{dp_{23}}{d\theta};\quad
\frac{dp_{31}}{d\theta}.
\]
We have, therefore, according to equations~\Eq{(111)},
\begin{align*}
\frac{dp_{12}}{d\theta} &= \frac{L_{12}}{\theta (v_{1} - v_{2})}, \\
\frac{dp_{23}}{d\theta} &= \frac{L_{23}}{\theta (v_{2} - v_{3})}, \\
\frac{dp_{31}}{d\theta} &= \frac{L_{31}}{\theta (v_{3} - v_{1})}, \\
\end{align*}
where $v$~refers to the fundamental state, and, therefore,
requires only one suffix. We can thus find the direction of
each curve at the fundamental point if we know the heat
of vaporization, of fusion, and of sublimation.

Let us compare, for example, the curve of vaporization,~$p_{12}$,
of water, with its curve of sublimation,~$p_{13}$, near the
fundamental point, $0.0074°$~C\@. We have, in absolute units,
% [** TN: Re-breaking]
\settowidth{\TmpLen}{heat of vaporization of\quad}%
\begin{align*}
L_{12} &= 604 × 419 × 10^{5}\quad\parbox[t]{\TmpLen}{(heat of vaporization of \\
water at $0.0074°$~C.);} \\
L_{13} &= -L_{31} = (80 + 604) × 419 × 10^{5}\quad\parbox[t]{\TmpLen}{(heat of sublimation \\
of ice at $0.0074°$~C.);} \\
v_{1} &= 205000;\quad
v_{2} = 1;\quad
v_{3} = 1.09 \text{ (\SecRef{187})};\quad
\theta = 273.
\end{align*}
Hence
\begin{align*}
\frac{dp_{12}}{d\theta}
  &= \frac{604 × 419 × 10^{5} × 760}
          {273 × 205000 × 1013650} = 0.339, \\
\frac{dp_{31}}{d\theta}
  &= \frac{684 × 419 × 10^{5} × 760}
          {273 × 205000 × 1013650} = 0.384,
\end{align*}
in millimeters of mercury. The curve of the sublimation
pressure~$p_{13}$, is steeper at the fundamental point than the
curve of the vaporization pressure~$p_{12}$. For temperatures
above the fundamental one, therefore, $p_{13} > p_{12}$; for those
below it, $p_{12} > p_{13}$. Their difference is
\[
\frac{dp_{13}}{d\theta} - \frac{dp_{12}}{d\theta}
  = \frac{d(p_{13} - p_{12})}{d\theta}
  = 0.045\Add{.}
\]
\PageSep{157}
If, therefore, the maximum vapour pressure of water be
measured above the fundamental point, and of ice below it,
the curve of pressure will show an abrupt bend at the fundamental
point. This change of direction is measured by the
discontinuity of the differential coefficient. At $-1°$~C.,
($d\theta = -1$), we have, approximately,
\[
p_{13} - p_{12} = -0.045;
\]
\ie\ at $1°$~C. the maximum vapour pressure of ice is
$0.045~\Unit{mm.}$ less than that of water. This has been verified
by experiment. The existence of a sharp bend in the curve,
however, can only be inferred from theory.

\Section{189.} We have hitherto extended our investigation only
to the different admissible solutions of the equations which
express the intrinsic conditions of equilibrium, and have
deduced from them the properties of the states of equilibrium
to which they lead. We shall now consider the relative merit
of these solutions, \ie\ which of them represents the state of
greatest stability. For this purpose we resume our original
statement of the problem (\SecRef{165}), which is briefly as follows:

Given the total mass~$M$, the total volume~$V$, and the total
energy~$U$, it is required to find the state of most stable equilibrium,
\ie\ the state in which the total entropy of the system
is an absolute maximum. Instead of $V$~and~$U$, however, it is
often more convenient to introduce $v = \dfrac{V}{M}$, the mean specific
volume of the system, and $u = \dfrac{U}{M}$, the mean specific energy
of the system.

We have found that the conditions of equilibrium admit,
in general, of three kinds of solution, according as the
system is split into $1$,~$2$, or $3$~states of aggregation. When
we come to consider which of these three solutions deserves
the preference in a given case, we must remember that the
second and third can be interpreted physically only if the
values of the masses, as given by the equations \Eq{(103)} and~\Eq{(121)},
\PageSep{158}
are positive. This restricts the region of validity of
these two solutions. We shall first establish this region of
validity, and then prove that within its region the third
solution is always preferable to the other two, and, similarly,
the second is preferable to the first.

A geometrical representation may facilitate a general
survey of the problem. We shall take the mean specific
volume, $v = \dfrac{V}{M}$, and the mean specific energy, $u = \dfrac{U}{M}$, of the
system as the rectangular co-ordinate axes. The value
of~$M$ is here immaterial. Each point of this plane will,
\ifthenelse{\boolean{ForPrinting}}{%
  \Figure[\textwidth]{4}
}{%
  \Figure[4.25in]{4}
}
\index{Curves!of evaporation}%
\index{Curves!of fusion}%
\index{Curves!of sublimation}%
\index{Fusion, curve}%
\index{Sublimation, curve}%
\index{Vaporization curve}%
then, represent definite values of $u$~and~$v$. Our problem
is, therefore, to find the kind of stable equilibrium which
will correspond to any given point in this plane.

\Section{190.} Let us consider the region of validity of the third
\PageSep{159}
solution. The values of the masses given by the equations~\Eq{(121)}
are
\begin{multline*}
M_{1} : M_{2} : M_{3} : M \\
  = \left|\begin{array}{@{}ccc@{}}
  1 & 1 & 1 \\
  v & v_{2} & v_{3} \\
  u & u_{2} & u_{3} \\
  \end{array}\right|
: \left|\begin{array}{@{}ccc@{}}
  1 & 1 & 1 \\
  v & v_{3} & v_{1} \\
  u & u_{3} & u_{1} \\
  \end{array}\right|
: \left|\begin{array}{@{}ccc@{}}
  1 & 1 & 1 \\
  v & v_{1} & v_{2} \\
  u & u_{1} & u_{2} \\
  \end{array}\right|
: \left|\begin{array}{@{}ccc@{}}
  1 & 1 & 1 \\
  v_{1} & v_{2} & v_{3} \\
  u_{1} & u_{2} & u_{3} \\
  \end{array}\right|
\Tag{(121a)}
\end{multline*}
where $v_{1}$,~$v_{2}$,~$v_{3}$, $u_{1}$,~$u_{2}$,~$u_{3}$, refer, as hereafter, to the special
values which these quantities assume in the fundamental
state.

It is obvious from this that the values of $M_{1}$,~$M_{2}$,~$M_{3}$ can
be simultaneously positive only when the point $(v, u)$ lies
within the triangle formed by the points $(v_{1}, u_{1})$\Add{,} $(v_{2}, u_{2})$\Add{,} and
$(v_{3}, u_{3})$. The area of this triangle then represents the
region of validity of the third solution, and may be called
the \emph{fundamental triangle} of the substance. In \Fig{4} this
\index{Fundamental point (triple)!triangle}%
\index{Triangle, fundamental}%
triangle is represented by~\Eq{(123)}. The diagram is based on
a substance for which, as for water,
\[
v_{1} > v_{3} > v_{2}\quad\text{and}\quad u_{1} > u_{2} > u_{3}.
\]

\Section{191.} We shall now consider the region of validity of
the second solution contained in equations \Eq{(101)} and~\Eq{(103)}.
These equations furnish three sets of values for the three
possible combinations, and no preference can be given to
any one of these. If we consider first the combination of
liquid and vapour, the equations referred to become, under
our present notation,
\begin{gather*}
\left.
\begin{aligned}
\theta_{12} &= \theta_{21}\Add{,} \\
p_{12} &= p_{21}\Add{,} \\
\phi_{12} - \phi_{21} &= \frac{u_{12} - u_{21} + p_{12} (v_{12} - v_{21})}{\theta_{12}}\Add{;} \\
\end{aligned}
\right\}
\Tag{(122)}\displaybreak[0] \\
%
\left.
\begin{aligned}
M_{12} + M_{21} &= M\Add{,} \\
M_{12} v_{12} + M_{21} v_{21} &= V = Mv\Add{,} \\
M_{12} u_{12} + M_{21} u_{21} &= U = Mu\Add{.} \\
\end{aligned}
\right\}
\Tag{(123)}
\end{gather*}
In order to determine the area within which the point $(v, u)$
must lie so that $M_{12}$~and $M_{21}$ may both be positive, we
shall find the limits of that area, \ie\ the curves represented
\PageSep{160}
by the conditions $M_{12} = 0$, and $M_{21} = 0$. The introduction
of the latter (no liquid mass) gives $M_{12} = M$ and
\[
v = v_{12},\quad
u = u_{12}.
\Tag{(124)}
\]
Since $v_{12}$~and $u_{12}$ are functions of a single variable, the
conditions~\Eq{(124)} restrict the point $(v, u)$ to a curve, one of
the limits of the region of validity. The curve passes
through the vertex~$1$ of the fundamental triangle, because,
at the fundamental temperature, $v_{12} = v_{1}$, and $u_{12} = u_{1}$. To
follow the path of the curve it is necessary to find the
differential coefficient~$\dfrac{du_{12}}{dv_{12}}$. We, have
\[
\frac{du_{12}}{dv_{12}}
  = \left(\frac{\dd u}{\dd v}\right)_{12} + \left(\frac{\dd u}{\dd \theta}\right)_{12} \frac{d\theta_{12}}{dv_{12}}.
\]
The partial differential coefficients here refer to the
independent variables $\theta$~and~$v$. It follows from \Eq{(80)}~and
\Eq{(24)} that
\[
\frac{du_{12}}{dv_{12}}
  = \theta_{12} \left(\frac{\dd p}{\dd \theta}\right)_{12} - p_{12} + (c_{v})_{12}\, \frac{d\theta_{12}}{dv_{12}}.
\]
By means of this equation the path of the curve~\Eq{(124)}
may be experimentally plotted by taking~$\theta_{12}$, or~$v_{12}$, or some
other appropriate quantity as independent parameter.

Similarly, the condition $M_{12} = 0$ (no vapour) gives another
boundary of the region of validity, viz.\ the curve,
\[
v = v_{21},\quad
u = u_{21},
\]
which passes through the vertex~$2$ of the fundamental
triangle, and satisfies the differential equation
\[
\frac{du_{21}}{dv_{21}}
  = \theta_{21} \left(\frac{\dd p}{\dd \theta}\right)_{21} - p_{12} + (c_{v})_{21}\, \frac{d\theta_{12}}{dv_{21}},
\]
since $\theta_{21} = \theta_{12}$ and $p_{21} = p_{12}$.

The two limiting curves, however, are merely branches
of one curve, since they pass into one another at the critical
point ($v_{12} = v_{21}$) without forming an angle or cusp at that
\PageSep{161}
\index{Corresponding point.}%
\index{Curves!of evaporation}%
\index{Curves!of fusion}%
\index{Curves!of sublimation}%
\index{Fusion, curve}%
\index{Sublimation, curve}%
point, as a further discussion of the values of~$\dfrac{du_{12}}{dv_{12}}$ and
$\dfrac{du_{21}}{dv_{21}}$ will show. We may, therefore, include the two branches
under the name of the \emph{vaporization curve}. Every point
\index{Vaporization curve}%
$(v_{12}, u_{12})$ of one branch has a \emph{corresponding point} $(v_{21}, u_{21})$ on
the other, since these two represent the same temperature
$\theta_{12} = \theta_{21}$, and the same pressure $p_{12} = p_{21}$. This co-ordination
of points on the two branches is given by the equations~\Eq{(122)},
and has been indicated on our diagram (\Fig{4}) by
drawing some dotted lines joining corresponding points.
In this sense the vertices $1$~and~$2$ of the fundamental
triangle are corresponding points, and the critical point is
self-corresponding.

This vaporization curve bounds the region of validity
of that part of the second solution which refers to liquid in
contact with its vapour. Equation~\Eq{(123)} makes it obvious
that the region of validity lies within the concave side of
the curve. The curve has not been produced beyond the
vertices $1$~and~$2$ of the fundamental triangle, because we
shall see later, that the side~$12$ of that triangle bounds the
area within which this solution gives stable equilibrium.
There may be found, quite analogous to the vaporization
curve, also a \emph{fusion curve} the two branches of which are
represented by
\[
v = v_{23},\quad
u = u_{23},
\]
and
\[
v = v_{32},\quad
u = u_{32},
\]
and a \emph{sublimation curve} represented by
\[
v = v_{31},\quad
u = u_{31},
\]
and
\[
v = v_{13},\quad
u = u_{13},
\]
The former passes through the vertices $2$~and~$3$, the latter
through $3$~and~$1$, of the fundamental triangle. The region
of validity of the three parts of the second solution have been
marked $(12)$,~$(23)$, and $(31)$, respectively, in \Fig{4}. The
relations which have been specially deduced for the area~$(12)$
apply to $(23)$ and~$(31)$ as well only with a corresponding
\PageSep{162}
interchange of the suffixes. Some pairs of corresponding
points have again been joined by dotted lines. On the
fusion curve a critical point has been marked, on the
assumption that, with falling temperature, the latent heat of
ice decreases by $0.64$~calorie per degree Centigrade (\SecRef{183}).
If, as a rough approximation, we assume this same ratio to
hold for much lower temperatures, the latent heat of fusion
would be zero at about $-120°$~C., and this would be the
critical point of the fusion curve. The pressure here would
be about $17,000$~atmospheres, and water and ice would
become identical. We might imagine this to be the result
of a considerable increase in the viscosity of water and in
the plasticity of ice, as they both approach this state.

\Section{192.} Having thus fixed the region of validity for the
second solution, we find that for all points $(v, u)$ outside this
region only the first solution admits of physical interpretation.
It follows that for such points the first solution represents
the stable equilibrium. The areas where such is the case
have been marked $(1)$,~$(2)$, and~$(3)$ in our figure, to signify
the gaseous, liquid, and solid states respectively.

\Section{193.} We have now to consider the following question:
Which of the different states of equilibrium, that may
correspond to given values $M$,~$v$,~$u$ (or to a given point of
the figure), gives to the system the greatest value of the
entropy? Since each of the three solutions discussed leads
to a definite state of the system, we have for each given
system $(M, v, u)$ as many values of the entropy as there are
solutions applying to it. Denoting these by $\Phi$,~$\Phi'$, and~$\Phi''$, we
get for the first solution
\[
\Phi = M\phi\Add{;}
\Tag{(125)}
\]
for the second
\[
\Phi' = M\phi' = M_{12} \phi_{12} + M_{21} \phi_{21}
\Tag{(126)}
\]
(or a cyclic interchange of the suffixes $1$,~$2$,~$3$); for the
third
\[
\Phi'' = M\phi'' = M_{1} \phi_{1} + M_{2} \phi_{2} + M_{3} \phi_{3}\Add{.}
\Tag{(127)}
\]
\PageSep{163}
All these quantities are fully determined by the given
values of $M$,~$v$, and~$u$. Now, we can show that for any system
$(M, v, u)$ we have $\Phi'' > \Phi' > \Phi$, or $\phi'' > \phi' > \phi$, provided all
the partial masses are positive. It is more convenient to
deal with the mean specific entropies than with the entropies
themselves, because the former, being functions of $v$~and
$u$~alone, are quite independent of~$M$.

As a geometrical representation, we may imagine, on the
plane of our figure (\Fig{4}), perpendiculars erected at each
point $(v, u)$, proportional in length to the values of $\phi$,~$\phi'$,
and $\phi''$~respectively, at that point. The upper ends of these
perpendiculars will generate the three surfaces of entropy,
$\phi$,~$\phi'$, and~$\phi''$.

\Section{194.} We shall show that $\phi' - \phi$ is always positive, \ie\
that the surface of entropy,~$\phi'$, lies everywhere above
the surface~$\phi$.

While the value of~$\phi$ may be taken directly from~\Eq{(61)},
which contains the definition of the entropy for homogeneous
substances, $\phi'$~may be found from \Eq{(126)}, \Eq{(122)}, and~\Eq{(123)}, in
terms of $v$~and~$u$. The surface~$\phi'$ forms three sheets corresponding
to the three combinations of two states of aggregation.
We shall in the following refer to the combination
of vapour and liquid.

With regard to the relative position of the surfaces $\phi$ and~$\phi'$,
it is obvious that they have one curve in common, the
projection of which is the vaporization curve. At any
point on the vaporization curve we have $v = v_{12}$, $u = u_{12}$, and
for the first entropy surface, $\phi = \phi_{12}$; for the second we have,
from~\Eq{(123)},
\[
M_{21} = 0,\quad
M_{12} = M
\Tag{(128)}
\]
and, from~\Eq{(126)},
\[
\phi' = \phi_{12}.
\]
In fact, for all points of the vaporization curve, both
solutions coincide. The curve of intersection of the
surfaces $\phi$~and~$\phi'$ is represented by
\[
v = v_{12},\quad
u = u_{12},\quad
\phi = \phi_{12},
\]
\PageSep{164}
where $v$,~$u$, and~$\phi$ are the three rectangular co-ordinates of a
point in space. $v_{12}$,~$u_{12}$,~$\phi_{12}$ depend on a single variable
parameter, for example, the temperature, $\theta_{12} = \theta_{21}$. This
curve passes through the point $(v_{1}, u_{1}, \phi_{1})$, which has the
vertex~$1$ for its projection. A second branch of the same
curve is given by the equations
\[
v = v_{21},\quad
u = u_{21},\quad
\phi = \phi_{21},
\]
and these branches meet in a point whose projection is the
critical point. Each point of one branch has a corresponding
point on the other, since both correspond to the same
temperature, $\theta_{12} = \theta_{21}$, and the same pressure, $p_{12} = p_{21}$.
Thus, $(v_{1}, u_{1}, \phi_{1})$ and $(v_{2}, u_{2}, \phi_{2})$ are corresponding points.

It is further obvious that the surface~$\phi'$ is a ruled surface
\index{Developable surface}%
\index{Surface, developable}%
and is developable. The first may be shown by considering
any point with the co-ordinates
\[
v = \frac{\lambda v_{12} + \mu v_{21}}{\lambda + \mu};\quad
u = \frac{\lambda u_{12} + \mu u_{21}}{\lambda + \mu};\quad
\phi = \frac{\lambda \phi_{12} + \mu \phi_{21}}{\lambda + \mu};
\]
where $\lambda$~and~$\mu$ are arbitrary positive quantities. By giving
$\lambda$~and $\mu$ all positive values, we obtain all points of the
straight line joining the corresponding points $(v_{12}, u_{12}, \phi_{12})$
and $(v_{21}, u_{21}, \phi_{21})$. But this line lies on the surface~$\phi'$,
since all the above values of $(v, u, \phi)$ satisfy the equations
\Eq{(123)} and~\Eq{(126)} if we put $M_{12} = \lambda$ and $M_{21} = \mu$. The surface~$\phi'$,
then, is formed of the lines joining the corresponding
points on the curve in which the surfaces $\phi'$~and $\phi$ meet.
One of these is the line joining the points $(v_{1}, u_{1}, \phi_{1})$ and
$(v_{2}, u_{2}, \phi_{2})$, the projection of which is the side~$12$ of the
fundamental triangle. At the critical point, the line shrinks
to a point, and here the surface~$\phi$ ends. The other two
sheets of the surface are quite similar. One begins at the
line joining $(v_{2}, u_{2}, \phi_{2})$ and $(v_{3}, u_{3}, \phi_{3})$, the other at the line
joining $(v_{3}, u_{3}, \phi_{3})$ and $(v_{1}, u_{1}, \phi_{1})$.

The developability of the surface~$\phi'$ may best be inferred
from the following equation of a plane:---
\[
p_{12} (v - v_{12}) + (u - u_{12}) - \theta_{12} (\phi - \phi_{12}) = 0,
\]
\PageSep{165}
where $v$,~$u$,~$\phi$ are variable co-ordinates, while $p_{12}$,~$v_{12}$,
$u_{12}$, $\theta_{12}$,~$\phi_{12}$, depend, by~\Eq{(122)}, on one parameter, \eg~$\theta_{12}$.
This plane contains the point $(v_{12}, u_{12}, \phi_{12})$, and by the
equations~\Eq{(122)} the point $(v_{21}, u_{21}, \phi_{21})$, which are corresponding
points, and hence also the line joining them.
But it also, by~\Eq{(61)}, contains the neighbouring corresponding
points
\[
(v_{12} + dv_{12}, u_{12} + du_{12}, \phi_{12} + d\phi_{12})
\]
and
\[
(v_{21} + dv_{21}, u_{21} + du_{21}, \phi_{21} + d\phi_{21})\Add{,}
\]
hence also the line joining them. Therefore, two consecutive
generating lines are coplanar, which is the condition of
developability of a surface.

In order to determine the value of $\phi' - \phi$, we shall
find the change which this difference undergoes on passing
from a point $(v, u)$ to a neighbouring one $(v + \delta v, u + \delta u)$.
During this passage we shall keep $M = M_{12} + M_{21}$ constant.
This does not affect the generality of the result, since $\phi$~and
$\phi'$ are functions of $v$~and $u$ only. From~\Eq{(126)} we have
\[
M\, \delta \phi'
  = M_{12}\, \delta \phi_{12} + M_{21}\, \delta \phi_{21}
  + \phi_{12}\, \delta M_{12} + \phi_{21}\, \delta M_{21},
\]
and, by~\Eq{(61)},
\[
\delta \phi = \frac{\delta u + p\, \delta v}{\theta}.
\]
But, by~\Eq{(123)},
\[
\left.
\begin{aligned}
\delta M_{12} + \delta M_{21} &= 0\Add{,} \\
M_{12}\, \delta v_{12} + M_{21}\, \delta v_{21}
+ v_{12}\, \delta M_{12} + v_{21}\, \delta M_{21} &= M\, \delta v\Add{,} \\
M_{12}\, \delta u_{12} + M_{21}\, \delta u_{21}
+ u_{12}\, \delta M_{12} + u_{21}\, \delta M_{21} &= M\, \delta u\Add{.} \\
\end{aligned}
\right\}
\Tag{(129)}
\]
Whence, by~\Eq{(122)},
\begin{align*}
\delta \phi' &= \frac{\delta u + p_{12}\, \delta v}{\theta_{12}}
\Tag{(130)} \\
\intertext{and}
\delta (\phi' - \phi)
  &= \left(\frac{1}{\theta_{12}} - \frac{1}{\theta}\right) \delta u
   + \left(\frac{p_{12}}{\theta_{12}} - \frac{p}{\theta}\right) \delta v.
\Tag{(131)}
\end{align*}
\PageSep{166}

If we now examine the surfaces $\phi$~and $\phi'$ in the neighbourhood
of their curve of contact, it is evident from the
last equation that they touch one another along the whole
of this curve. For, at any point of the vaporization curve,
we have $v = v_{12}$ and $u = u_{12}$; therefore also
\[
\theta = \theta_{12},\quad\text{and}\quad p = p_{12}
\Tag{(132)}
\]
and hence, for the entire curve, $\delta (\phi' - \phi) = 0$.

To find the kind of contact between the two surfaces, we
form $\delta^{2} (\phi' - \phi)$ from~\Eq{(131)}, and apply it to the same points
of the curve of contact. In general,
\begin{align*}
\delta^{2} (\phi' - \phi)
  &= \delta u \left(\frac{\delta \theta}{\theta^{2}} - \frac{\delta \theta_{12}}{\theta_{12}^{2}}\right) \\
  &+ \delta v \left(\frac{\delta p_{12}}{\theta_{12}} - \frac{\delta p}{\theta} - \frac{p_{12}\, \delta \theta_{12}}{\theta_{12}^{2}} + \frac{p\, \delta \theta}{\theta^{2}}\right) \\
  &+ \delta^{2} u \left(\frac{1}{\theta_{12}} - \frac{1}{\theta}\right)
   + \delta^{2} v \left(\frac{p_{12}}{\theta_{12}} - \frac{p}{\theta}\right).
\end{align*}
According to~\Eq{(132)} we have, at the points of contact of
the surfaces,
\[
\theta^{2}\, \delta^{2} (\phi' - \phi)
  = \delta u\, (\delta \theta - \delta \theta_{12})
  + \delta v (\theta\, \delta p_{12} - \theta\, \delta p - p\, \delta \theta_{12} + p\,\delta \theta),
\]
or, by~\Eq{(61)},
\[
\theta\, \delta^{2} (\phi' - \phi)
  = (\delta \theta - \delta \theta_{12})\, \delta \phi + (\delta p_{12} - \delta p)\, \delta v.
\Tag{(133)}
\]
All these variations may be expressed in terms of $\delta \theta$~and~$\delta v$,
by putting
\begin{align*}
\delta \phi &= \frac{c_{v}}{\theta}\, \delta \theta + \frac{\dd p}{\dd \theta}\, \delta v\quad\text{\Chg{(by 81)}{by~\Eq{(81)}}}, \\
\delta p &= \frac{\dd p}{\dd \theta}\, \delta \theta + \frac{\dd p}{\dd v}\, \delta v, \\
\delta p_{12} &= \frac{dp_{12}}{d\theta_{12}}\, \delta \theta_{12}.
\end{align*}
We have now to express $\delta \theta_{12}$ in terms of $\delta \theta$~and~$\delta v$.
Equations~\Eq{(129)}, here simplified by~\Eq{(128)}, give
\[
\frac{\delta u_{12} - \delta u}{u_{12} - u_{21}}
  = \frac{\delta v_{12} - \delta v}{v_{12} - v_{21}}.
\]
\PageSep{167}
In these we put
\begin{align*}
\delta u_{12} &= \frac{du_{12}}{d\theta_{12}}\, \delta \theta_{12},\quad
\delta v_{12}  = \frac{dv_{12}}{d\theta_{12}}\, \delta \theta_{12}\Add{,}
\Tag{(134)} \\
\delta u &= c_{v}\, \delta \theta + \frac{\dd u}{\dd v}\, \delta v,
\end{align*}
and obtain
\[
\delta \theta_{12}
  = \frac{c_{v}\, \delta \theta + \left(\dfrac{\dd u}{\dd v} - \dfrac{u_{12} - u_{21}}{v_{12} - v_{21}}\right) \delta v}
         {\dfrac{du_{12}}{d\theta_{12}} - \dfrac{u_{12} - u_{21}}{v_{12} - v_{21}} \Add{·} \dfrac{dv_{12}}{d\theta_{12}}}.
\]
If we consider that, by~\Eq{(109)},
\begin{align*}
\frac{u_{12} - u_{21}}{v_{12} - v_{21}}
  &= \theta_{12}\, \frac{dp_{12}}{d\theta_{12}} - p_{12}
\Tag{(135)} \\
  &= \theta\, \frac{dp_{12}}{d\theta_{12}} - p,
\end{align*}
that, by~\Eq{(80)},
\[
\frac{\dd u}{\dd v} = \theta\, \frac{dp}{d\theta} - p
\]
and that
\begin{align*}
\frac{du_{12}}{d\theta_{12}}
  &= \left(\frac{\dd u}{\dd \theta}\right)_{12} + \left(\frac{\dd u}{\dd v}\right)_{12} · \frac{dv_{12}}{d\theta_{12}} \\
  &= c_{v} + \left(\theta\, \frac{\dd p}{\dd \theta} - p\right) \frac{dv_{12}}{d\theta_{12}}\Add{;}
\Tag{(136)}
\end{align*}
also that
\[
\frac{dp_{12}}{d\theta_{12}} = \frac{\dd p}{\dd \theta} + \frac{\dd p}{\dd v} · \frac{dv_{12}}{d\theta_{12}},
\]
we obtain
\[
\delta \theta_{12}
  = \frac{c_{v}\, \delta \theta - \theta\, \dfrac{\dd p}{\dd v} · \dfrac{dv_{12}}{d\theta_{12}}\, \delta v}{c_{v} - \theta\, \dfrac{\dd p}{\dd v} \left(\dfrac{dv_{12}}{d\theta_{12}}\right)^{2}}.
\]
Equation~\Eq{(133)}, with all variations expressed in terms of $\delta \theta$
and~$\delta v$, finally becomes
\[
\delta^{2} (\phi' - \phi)
  = -\frac{\dd p}{\dd v} · \frac{c_{v}}{\theta}
   · \frac{\left(\dfrac{dv_{12}}{d\theta_{12}}\, \delta \theta - \delta v\right)^{2}}{c_{v} - \theta\, \dfrac{\dd p}{\dd v} \left(\dfrac{dv_{12}}{d\theta_{12}}\right)^{2}}.
\]
This expression is essentially positive, since $c_{v}$~is positive
\PageSep{168}
on account of its physical meaning, and $\dfrac{\dd p}{\dd v}$~is negative for
any state of equilibrium (\SecRef{169}). There is a limiting case,
when
\[
\frac{dv_{12}}{d\theta_{12}}\, \delta \theta - \delta v = 0,
\]
for, then,
\[
\delta^{2} (\phi' - \phi) = 0.
\]
In this case the variation $(\delta \theta, \delta v)$ obviously takes place
along the curve of contact $(\theta_{12}, v_{12})$ of the surfaces. Here
we know that $\phi' = \phi$.

It follows that the surface~$\phi'$, in the vicinity of all points
of contact with~$\phi$, rises above the latter throughout, or that
$\phi' - \phi$ is everywhere~$> 0$. This proves that the second
solution of the conditions of equilibrium, within its region
of validity, \ie\ in the areas $(12)$,~$(23)$, and~$(31)$, always represents
the stable equilibrium.

\Section{195.} Similarly, it may be shown that the third solution,
within its region of validity, is preferable to the second
one. The quantities $v$~and $u$ being given, the value of the
mean specific entropy,~$\phi''$, corresponding to this solution is
uniquely determined by the equations \Eq{(127)} and~\Eq{(121)}. The
quantities $v_{1}$,~$v_{2}$,~$v_{3}$, $u_{1}$,~$u_{2}$,~$u_{3}$, and therefore also $\phi_{1}$,~$\phi_{2}$,~$\phi_{3}$,
have definite numerical values, given by equations~\Eq{(120)}.

In the first place, it is obvious that the surface~$\phi''$ is
the plane triangle formed by the points $(v_{1}, u_{1}, \phi_{1})$, $(v_{2}, u_{2}, \phi_{2})$,
and $(v_{3}, u_{3}, \phi_{3})$, the projection of which on the plane of
the figure is the fundamental triangle, since any point with
the co-ordinates
\begin{align*}
v &= \frac{\lambda v_{1} + \mu v_{2} + \nu v_{3}}{\lambda + \mu + \nu}, \\
u &= \frac{\lambda u_{1} + \mu u_{2} + \nu u_{3}}{\lambda + \mu + \nu}, \\
\phi &= \frac{\lambda \phi_{1} + \mu \phi_{2} + \nu \phi_{3}}{\lambda + \mu + \nu},
\end{align*}
($\lambda$,~$\mu$,~$\nu$ may have any positive values) satisfies the equations
\Eq{(121)} and~\Eq{(127)}. To show this, we need only put $M_{1} = \lambda$,
\PageSep{169}
$M_{2} = \mu$, $M_{3} = \nu$. This plane meets the three sheets of the
developable surface~$\phi'$ in the three lines joining the points
$(v_{1}, u_{1}, \phi_{1})$, $(v_{2}, u_{2}, \phi_{2})$, $(v_{3}, u_{3}, \phi_{3})$. In fact, by making
$\nu = 0$, \ie, by~\Eq{(121)}, $M_{3} = 0$, the third solution coincides
with the second; for, then,
\[
\left.
\begin{gathered}
M_{1} = M_{12};\quad
M_{2} = M_{21};\quad
v_{1} = v_{12};\quad
u_{1} = u_{12}; \\
v_{2} = v_{2l};\quad
\theta_{1} = \theta_{12};\quad\text{etc.}
\end{gathered}
\right\}
\Tag{(137)}
\]

If we also put $\mu = 0$, then we have $M_{2} = 0$, $v_{1} = v$,
$u_{1} = u$, which means the coincidence of all three surfaces,
$\phi''$,~$\phi'$, and~$\phi$.

In order to find the sign of $\phi'' - \phi'$, we again find
$\delta (\phi'' - \phi')$ in terms of $\delta u$~and~$\delta v$. Equation~\Eq{(127)} gives
\[
M\, \delta\phi''
  = \phi_{1}\, \delta M_{1} + \phi_{2}\, \delta M_{2} + \phi_{3}\, \delta M_{3}
\Tag{(138)}
\]
where, by~\Eq{(121)},
\begin{align*}
\delta M_{1} + \delta M_{2} + \delta M_{3} &= 0\Add{,} \\
v_{1}\, \delta M_{1} + v_{2}\, \delta M_{2} + v_{3}\, \delta M_{3} &= M\, \delta v\Add{,} \\
u_{1}\, \delta M_{1} + u_{2}\, \delta M_{2} + u_{3}\, \delta M_{3} &= M\, \delta u\Add{.}
\end{align*}
Multiplying the last of these by~$\dfrac{1}{\theta_{1}}$, the second by~$\dfrac{p_{1}}{\theta_{1}}$, and
adding to~\Eq{(138)}, we obtain, with the help of~\Eq{(120)},
\[
\delta \phi'' = \frac{\delta u + p_{1}\, \delta v}{\theta_{1}}.
\]
This, in combination with~\Eq{(130)}, gives
\[
\delta (\phi'' - \phi')
  = \left(\frac{1}{\theta_{1}} - \frac{1}{\theta_{12}}\right) \delta u
  + \left(\frac{p_{1}}{\theta_{1}} - \frac{p_{12}}{\theta_{12}}\right) \delta v,
\Tag{(139)}
\]
if the surface~$\phi'$ is represented by the sheet~$(12)$. This
equation shows that the surface~$\phi''$ is a tangent to the
sheet~$(12)$ along the line joining $(v_{1}, u_{1}, \phi_{1})$ and $(v_{2}, u_{2}, \phi_{2})$,
for all points of this line have $\theta_{1} = \theta_{12}$, $p_{1} = p_{12}$, so that
$\delta (\phi'' - \phi')$ vanishes. Thus, we find that the plane~$\phi''$ is a
tangent plane to the three sheets of the surface~$\phi'$. The
curves of contact are the three straight lines which form
\PageSep{170}
the sides of the plane triangle~$\phi''$. We have, from~\Eq{(139)},
for any point of contact
\[
\delta^{2} (\phi'' - \phi')
  = \frac{\delta \theta_{12}}{\theta_{1}^{2}}\, \delta u
  + \left(\frac{p_{1}\, \delta \theta_{12}}{\theta_{1}^{2}} - \frac{\delta p_{12}}{\theta_{1}}\right) \delta v,
\]
since $\theta_{1}$~and $p_{1}$ are absolute constants; or
\[
\theta_{1}^{2}\, \delta^{2} (\phi'' - \phi')
  = \left[\delta u - \left(\theta_{1}\, \frac{dp_{12}}{d\theta_{12}} - p_{1}\right) \delta v\right] \delta \theta_{12}.
\Tag{(140)}
\]
Now, by the elimination of $\delta M_{12}$ and~$\delta M_{21}$, it follows, from~\Eq{(129)},
that
\[
\frac{M_{12}\, \delta v_{12} + M_{21}\, \delta v_{21} - M\, \delta v}{v_{12} - v_{21}}
  = \frac{M_{12}\, \delta u_{12} + M_{21}\, \delta u_{21} - M\, \delta u}{u_{12} - u_{21}}
\]
or, by \Eq{(135)} and~\Eq{(134)},
% [** TN: Re-breaking]
\begin{multline*}
M \left[\delta u - \left(\theta_{1}\, \frac{dp_{12}}{d\theta_{12}} - p_{1}\right) \delta v\right] \\
  = \delta \theta_{12} \left[M_{12}\, \frac{du_{12}}{d\theta_{12}} + M_{21}\, \frac{du_{21}}{d\theta_{12}}\right. \\
  - \left.\left(\theta_{1}\, \frac{dp_{12}}{d\theta_{12}} - p_{1}\right) · \left(M_{12}\, \frac{dv_{12}}{d\theta_{12}} + M_{21}\, \frac{dv_{21}}{d\theta_{12}}\right)\right].
\end{multline*}
Substituting this expression in~\Eq{(140)}, and replacing $\dfrac{du_{12}}{d\theta_{12}}$
and $\dfrac{du_{21}}{d\theta_{21}}$ by their values~\Eq{(136)}, we obtain
\begin{multline*}
\delta^{2} (\phi'' - \phi')
  = \frac{\delta \theta_{12}^{2}}{M \theta_{1}^{2}} \left[M_{12} \left((c_{v})_{12} - \theta_{1} \left(\frac{\dd p}{\dd v}\right)_{12} · \left(\frac{dv_{12}}{d\theta_{12}}\right)^{2}\right)\right. \\
  + \left.M_{21} \left((c_{v})_{21} - \theta_{1} \left(\frac{\dd p}{\dd v}\right)_{21} · \left(\frac{dv_{21}}{d\theta_{12}}\right)^{2}\right)\right].
\end{multline*}
This quantity is essentially positive, since $M_{12}$,~$M_{21}$, as well
as~$c_{v}$, are always positive, and $\dfrac{\dd p}{\dd v}$~always negative for states
of equilibrium. There is a limiting case, when $\delta \theta_{12} = 0$,
\ie\ for a variation along the line of contact of the surfaces
$\phi''$~and~$\phi'$, as is obvious. It follows that the plane area~$\phi''$
rises everywhere above the surface~$\phi'$, and that $\phi'' - \phi'$ is
never negative. This proves that the third solution within
\PageSep{171}
its region of validity (the fundamental triangle of the substance)
represents stable equilibrium.

\Section{196.} We are now in a position to answer generally the
question proposed in \SecRef{165} regarding the stability of the
equilibrium.

The total mass~$M$, the volume~$V$, and the energy~$U$ of
a system being given, its corresponding state of stable
equilibrium is determined by the position of the point
$v = \dfrac{V}{M}$, $u = \dfrac{U}{M}$, in the plane of \Fig{4}.

If this point lie within one of the regions $(1)$,~$(2)$, or~$(3)$,
the system behaves as a homogeneous gas, liquid, or solid.
If it lie within $(12)$, $(23)$, or~$(31)$, the system splits into two
different states of aggregation, indicated by the numbers
used in the notation of the region. In this case, the common
temperature and the ratio of the two heterogeneous portions
are completely determined. According to the equation~\Eq{(123)},
the point $(v, u)$ lies on the straight line joining two
corresponding points of the limiting curve. If a straight
line be drawn through the given point $(v, u)$, cutting the
two branches of that curve in corresponding points, these
points give the properties of the two states of aggregation
into which the system splits. They have, of course, the
same temperature and pressure. The proportion of the
two masses, according to the equation~\Eq{(123)}, is given by
the ratio in which the point $(v, u)$ divides the line joining
the corresponding points.

If, finally, the point $(v, u)$ lie within the region of the
fundamental triangle~\Eq{(123)}, stable equilibrium is characterized
by a division of the system into a solid, a liquid,
and a gaseous portion at the fundamental temperature and
pressure. The masses of these three portions may then be
determined by the equations~\Eq{(121a)}. It will be seen that
their ratio is that of the three triangles, which the point~$(v, u)$
makes with the three sides of the fundamental
triangle.

The conditions of stable equilibrium of any substance
\PageSep{172}
can thus be found, provided its fundamental triangle, its
vaporization, fusion, and sublimation curves have been
drawn once for all. To obtain a better view of the different
relations, isothermal and isopiestic curves may be added to
the figure. These curves coincide in the regions $(12)$, $(23)$,
$(31)$, and form the straight lines joining corresponding
points on the limiting curves. On the other hand, the area~$(123)$
represents one singular isothermal and isopiestic (the
triple point). In this way we may find that ice cannot
exist in stable equilibrium at a higher temperature than the
fundamental temperature ($0.0074°$~C.), no matter how the
pressure may be reduced. Liquid water, on the other hand,
may, under suitable pressure, be brought to any temperature
without freezing or evaporating.

A question which may also be answered directly is the
following. Through what stages will a body pass if subjected
to a series of definite external changes? For instance,
the behaviour of a body of mass~$M$, when cooled or heated
at constant volume~$V$, may be known by observing the line
$v = \dfrac{V}{M}$ parallel to the axis of ordinates. The regions which
this line traverses show the states through which the body
passes, \eg\ whether the substance melts during the process,
or whether it sublimes, etc.
\PageSep{173}


\Chapter[Any Number of Independent Constituents.]{III.}{System of any Number of Independent Constituents.}
\index{Constituents, independent}%
\index{Independent constituents}%

\Section{197.} \First{We} proceed to investigate quite generally the equilibrium
of a system made up of distinct portions in contact
with one another. The system, contrary to that treated of in
the preceding chapter, may consist of any number of independent
constituents. Following Gibbs, we shall call each
\index{Gibbs}%
one of these portions, inasmuch as it may be considered
physically homogeneous (\SecRef{67}), a \emph{phase}. Thus, a quantity
\index{Phase!defined}%
of water partly gaseous, partly liquid, and partly solid,
forms a system of three phases. The number of phases as
well as the states of aggregation is quite arbitrary, although
we at once recognize the fact that a system in equilibrium
may consist of any number of solid and liquid phases, but
only one single \emph{gaseous} phase, for two different gases in
contact are never in equilibrium with one another.

\Section{198.} A system is characterized by the number of its
\emph{independent constituents},\footnote
  {Frequently termed \emph{components}.}
in addition to the number of its
phases. The main properties of the state of equilibrium
depend upon these. We define the number of independent
constituents as follows. First find the number of elements
contained in the system, and from these discard, as dependent
constituents, all those whose quantity is determined in
each phase by the remaining ones. The number of the
remaining elements will be the number of independent
constituents of the system. It is immaterial which of the
\PageSep{174}
constituents we regard as independent and which as dependent,
since we are here concerned with the number, and not
with the kind, of the independent constituents. The question
as to the number of the independent constituents has
nothing at all to do with the chemical constitution of the
substances in the different phases, in particular, with the
number of different kinds of molecules.

Thus, a quantity of water in any number of states forms
but one independent constituent, however many associations
and dissociations of \ce{H2O}~molecules may occur (it may be
a mixture of hydrogen and oxygen or ions), for the mass of
the oxygen in each phase is completely determined by that
of the hydrogen, and \textit{vice versâ}. Should, however, an excess
of oxygen or hydrogen be present in the vapour, we have
then two independent constituents.

An aqueous solution of sulphuric acid forms a system of
three chemical elements, \ce{S},~\ce{H}, and~\ce{O}, but contains only two
independent constituents, for, in each phase (\eg\ liquid,
vapour, solid) the mass of~\ce{O} depends on that of \ce{S}~and~\ce{H},
while the masses of \ce{S}~and~\ce{H} are not in each phase interdependent.
Whether the molecule \ce{H2SO4} dissociates in
any way, or whether hydrates are formed or not, does not
change the number of independent constituents of the
system.

\Section{199.} We denote the number of independent constituents
of a system by~$\alpha$. By our definition of this number we see,
at once, that each phase of a given system in equilibrium is
determined by the masses of each one of its $\alpha$~constituents,
the temperature~$\theta$, and the pressure~$p$. For the sake of
uniformity, we assume that each of the $\alpha$~independent constituents
actually occurs in each phase of the system in a
certain quantity, which, in special cases, may become
infinitely small. The selection of the temperature and the
pressure as independent variables, produces a change in the
form of the equations of the last chapter, where the temperature
and the specific volume were considered as the
independent variables. The substitution of the pressure
\PageSep{175}
for the volume is more convenient here, because the pressure
is the same for all phases in free contact, and it can in most
cases be more readily measured.

\Section{200.} We shall now consider the thermodynamical
equilibrium of a system, in which the total masses of the
a independent constituents $M_{1}$,~$M_{2}$,~\dots\Add{,} $M_{\alpha}$ are given. Of
the different forms of the condition of equilibrium it is best
to use that expressed by equation~\Eq{(79)}
\[
\delta \Psi = 0
\Tag{(141)}
\]
which holds, if $\theta$~and~$p$ remain constant, for any change
compatible with the given conditions. The function~$\Psi$ is
given in terms of the entropy~$\Phi$, the energy~$U$, and the
volume~$V$, by the equation
\[
\Psi = \Phi - \frac{U + pV}{\theta}.
\]

\Section{201.} Now, let $\beta$~be the number of phases in the system,
then $\Phi$,~$U$, and~$V$, and therefore also~$\Psi$, are sums of $\beta$~terms,
each of which refers to a single phase, \ie\ to a
physically homogeneous body:
\[
\Psi = \Psi' + \Psi'' + \dots +\Psi^{\beta}\Add{,}
\Tag{(142)}
\]
where the different phases are distinguished from one another
by dashes. For the first phase,
\[
\Psi' = \Phi' - \frac{U' + pV'}{\theta}.
\Tag{(143)}
\]
$\Phi'$,~$U'$,~$V'$\Add{,} and~$\Psi'$ are completely determined by $\theta$,~$p$, and
the masses $M_{1}$,~$M_{2}$,~\dots\Add{,} $M_{\alpha}$ of the independent constituents
in the phases. As to how they depend on the masses, all
we can at present say is, that, if all the masses were increased
in the same proportion (say doubled), each of these functions
would be increased in the same proportion. Since the
nature of the phase remains unchanged, the entropy, the
energy, and the volume change in the same proportion as
\PageSep{176}
the mass; hence, also, the function~$\Psi'$. In other words, $\Psi'$~is
a homogeneous function of the masses $M_{1}'$,~$M_{2}'$,~\dots\Add{,} $M_{\alpha}'$
of the first degree, but not necessarily linear.

To express this analytically, let us increase all the masses
in the same ratio $1 + \eps: 1$, where $\eps$~is very small. All
changes are then small; and for the corresponding change
of~$\Psi'$ we obtain
\begin{align*}
\Delta \Psi'
  &= \frac{\dd \Psi'}{\dd M_{1}}\, \Erratum{\Delta M_{1}}{\Delta M_{1}'} + \frac{\dd \Psi'}{\dd M_{2}}\, \Delta M_{2}' + \dots \\
  &= \frac{\dd \Psi'}{\dd M_{1}'}\, \eps M_{1}' + \frac{\dd \Psi'}{\dd M_{2}'}\, \eps M_{2}' + \dots\Add{.}
\end{align*}
But, by supposition,
\[
\Delta \Psi = \eps\Psi',
\]
and, therefore,
\index{Euler}%
\[
\Psi' = \frac{\dd \Psi'}{\dd M_{1}'}\, M_{1}' + \frac{\dd \Psi'}{\dd M_{2}'}\, M_{2}' + \dots + \frac{\dd \Psi'}{\dd M_{\alpha}'}\, M_{\alpha}'.
\Tag{(144)}
\]
{\Loosen Various forms may be given to this Eulerian equation
by further differentiation. The differential coefficients $\dfrac{\dd \Psi'}{\dd M_{1}''}$,
$\dfrac{\dd \Psi'}{\dd M_{2}''}$\Add{,}~\dots\ evidently depend on the constitution of the phase,
and not on its total mass, since a change of mass changes
both numerator and denominator in the same proportion.}

\Section{202.} By~\Eq{(142)}, the condition of equilibrium becomes
\index{Condition of complete reversibility!of equilibrium}%
\index{Equilibrium!conditions of}%
\[
\delta \Psi' + \delta \Psi'' + \dots \Add{+} \delta \Psi^{\beta} = 0\Add{,}
\Tag{(145)}
\]
or, since the temperature and pressure remain constant,
\begin{align*}
&\frac{\dd \Psi'}{\dd M_{1}'}\, \delta M_{1}'
  + \frac{\dd \Psi'}{\dd M_{2}'}\, \delta M_{2}'
  + \dots
  + \frac{\dd \Psi'}{\dd M_{\alpha}'}\, \delta M_{\alpha}' \\
%
+ & \frac{\dd \Psi''}{\dd M_{1}''}\, \delta M_{1}''
  + \frac{\dd \Psi''}{\dd M_{2}''}\, \delta M_{2}''
  + \dots
  + \frac{\dd \Psi''}{\dd M_{\alpha}''}\, \delta M_{\alpha}'' \\
+ & \dots \\
%
+ & \frac{\dd \Psi^{\beta}}{\dd M_{1}^{\beta}}\, \delta M_{1}^{\beta}
  + \frac{\dd \Psi^{\beta}}{\dd M_{2}^{\beta}}\, \delta M_{2}^{\beta}
  + \dots
  + \frac{\dd \Psi^{\beta}}{\dd M_{\alpha}^{\beta}}\, \delta M_{\alpha}^{\beta} = 0\Add{.}
\Tag{(146)}
\end{align*}
\PageSep{177}
If the variation of the masses were quite arbitrary, then the
equation could only be satisfied, if all the coefficients of the
variations were equal to~$0$. According to \SecRef{200}, however,
the following conditions exist between them,
\[
\left.
\begin{aligned}
M_{1} &= M_{1}' + M_{1}'' + \dots + M_{1}^{\beta}\Add{,} \\
M_{2} &= M_{2}' + M_{2}'' + \dots + M_{2}^{\beta}\Add{,} \\
  & \dots \\
M_{\alpha} &= M_{\alpha}' + M_{\alpha}'' + \dots + M_{\alpha}^{\beta}\Add{,}
\end{aligned}\right\}
\Tag{(147)}
\]
and, therefore, for any possible change of the system,
\[
\left.
\begin{aligned}
0 &= \delta M_{1}' + \delta M_{1}'' + \dots + \delta M_{1}^{\beta}\Add{,} \\
0 &= \delta M_{2}' + \delta M_{2}'' + \dots + \delta M_{2}^{\beta}\Add{,} \\
  & \dots \\
0 &= \delta M_{\alpha}' + \delta M_{\alpha}'' + \dots + \delta M_{\alpha}^{\beta}\Add{.}
\end{aligned}\right\}
\Tag{(148)}
\]
For the expression~\Eq{(146)} to vanish, the necessary and sufficient
condition is
\[
\left.
\begin{aligned}
\frac{\dd \Psi'}{\dd M_{1}'} &= \frac{\dd \Psi''}{\dd M_{1}''} = \dots = \frac{\dd \Psi^{\beta}}{\dd M_{1}^{\beta}}\Add{,} \\
\frac{\dd \Psi'}{\dd M_{2}'} &= \frac{\dd \Psi''}{\Erratum{}{\dd} M_{2}''} = \dots = \frac{\dd \Psi^{\beta}}{\dd M_{2}^{\beta}}\Add{,} \\
& \dots \\
\frac{\dd \Psi'}{\dd M_{\alpha}'} &= \frac{\dd \Psi''}{\dd M_{\alpha}''} = \dots = \frac{\dd \Psi^{\beta}}{\dd M_{\alpha}^{\beta}}\Add{.}
\end{aligned}\right\}
\Tag{(149)}
\]
There are for each independent constituent $(\beta - 1)$ equations,
which must be satisfied, and therefore for all the
$\alpha$~independent constituents $\alpha (\beta - 1)$ conditions. Each of
these equations refers to the transition from one phase into
another, and asserts that this particular transition does not
take place in nature. This condition depends, as it must,
on the internal constitution of the phase, and not on its
total mass. Since the equations in a single row with regard
to a particular constituent may be arranged in any order,
it follows that, if a phase be in equilibrium as regards a
\PageSep{178}
given constituent with two others, these two other phases
are in equilibrium with one another with regard to that
constituent (they coexist). This shows that, since any
system in equilibrium can have only one gaseous phase, two
coexisting phases must emit the same vapour. For, since
each phase is in equilibrium with the other, and also with
its own vapour with respect to all constituents, it must also
coexist with the vapour of the second phase. The coexistence
of solid and liquid phases may, therefore, be settled
by comparing their vapours.

\Section{203.} It is now easy to see how the state of equilibrium
\index{External conditions of equilibrium!variable}%
\index{Internal conditions of equilibrium!variable}%
\index{Variable, internal and external}%
of the system is determined, in general, by the given
external conditions~\Eq{(147)}, and the conditions of equilibrium~\Eq{(149)}.
There are a of the former and $\alpha(\beta - 1)$ of the latter,
a total of $\alpha\beta$~equations. On the other hand, the state of
the $\beta$~phases depends on $(\alpha\beta + 2)$ variables, viz.\ on the $\alpha\beta$~masses,
$M_{1}'$,~\dots\Add{,} $M_{\alpha}^{\beta}$, the temperature~$\theta$, and the pressure~$p$.
After all conditions have been satisfied, two variables still
remain undetermined. In general, the temperature and the
pressure may be arbitrarily chosen, but in special cases, as
will be shown presently, these are no longer arbitrary, and
in such cases two other variables, as the total energy and
the total volume of the system, are undetermined. By
disposing of the values of the arbitrary variables we completely
determine the state of the equilibrium.

\Section{204.} The $\alpha\beta + 2$ variables, which control the state of
the system, may be separated into those which merely
govern the composition of the phases (\emph{internal} variables),
and those which determine only the total masses of the
phases (\emph{external} variables). The number of the former is
$(\alpha - 1)\beta + 2$, for in each of the $\beta$~phases there are $\alpha - 1$
ratios between its $\alpha$~independent constituents, to which must
be added temperature and pressure. The number of the
external variables is~$\beta$, viz.\ the total masses of all the
phases.

We found that the $\alpha(\beta - 1)$ equations~\Eq{(149)} contain
\PageSep{179}
only internal variables, and, therefore, after these have been
satisfied, there remain
\[
\bigl[(\alpha - 1)\beta + 2\bigr] - \bigl[\alpha (\beta - 1)\bigr] = \alpha - \beta + 2
\]
of the internal variables, undetermined. This number
cannot be negative, for otherwise the number of the
internal variables of the system would not be sufficient for
\index{Non-variant system}%
\index{System!non-variant}%
the solution of the equations~\Eq{(149)}. It, therefore, follows
that
\[
\beta \leq \alpha + 2\Add{.}
\]
The number of the phases, therefore, cannot exceed the
number of the independent constituents by more than two;
or, a system of $\alpha$~independent constituents will contain at
most $(\alpha + 2)$ phases. In the limiting case, where $\beta = \alpha + 2$,
the number of the internal variables are just sufficient to
satisfy the internal conditions of equilibrium~\Eq{(149)}. Their
values in the state of equilibrium are completely determined
quite independently of the given external conditions.
Decreasing the number of phases by one increases the
number of the indeterminate internal variables by one.

This proposition, first propounded by Gibbs and universally
\index{Gibbs's phase rule}%
known as the \emph{phase rule}, has been amply verified,
\index{Phase!rule}%
especially by the experiments of Bakhuis Roozeboom.
\index{Roozeboom, Bakhuis}%

\Section{205.} We shall consider, first, the limiting case:
\[
\beta = \alpha + 2.
\]
(Non-variant systems.) Since all the internal variables are
completely determined, they form an \emph{$(n + 2)$-ple point}.
\index{Point!$(n + 2)$-ple}%
Change of the external conditions, as heating, compression,
further additions of the substances, alter the total masses of
the phases, but not their internal nature, including temperature
and pressure. This holds until the mass of some one
phase becomes zero, and therewith completely vanishes from
the system.

If $\alpha = 1$, then $\beta = 3$. A single constituent may split
\PageSep{180}
\index{System!univariant}%
\index{Univariant system}%
into three phases at most, forming a triple point. An
\index{Point!triple}%
\index{Point!quadruple}%
\index{Triple point}%
example of this is a substance existing in the three states
of aggregation, all in contact with one another. For water
it was shown in~\SecRef{187}, that at the triple point the temperature
is $0.0074°$~C., and the pressure $4.62~\Unit{mm.}$ of mercury.
The three phases need not, however, be different states of
aggregation. Sulphur, for instance, forms several modifications
\index{Sulphur!dioxide and water equilibrium}%
in the solid state. Each modification constitutes a
separate phase, and the proposition holds that two modifications
of a substance can coexist with a third phase of the
same substance, for example, its vapour, only at a definite
temperature and pressure.

A quadruple point is obtained when $\alpha = 2$. Thus, the
\index{Quadruple point}%
two independent constituents, \ce{SO2} (sulphur dioxide) and
\ce{H2O}, form the four coexisting phases: \ce{SO2}, \ce{7H2O} (solid), \ce{SO2}
dissolved in~\ce{H2O} (liquid), \ce{SO2} (liquid), \ce{SO2} (gaseous), at a
temperature of $12.1°$~C. and a pressure of $1770~\Unit{mm.}$ of mercury.
The question as to the formation of hydrates by \ce{SO2}
in aqueous solution does not influence the application of the
phase rule (see~\SecRef{198}).

Three independent constituents ($\alpha = 3$) lead to a \Erratum{quintiple}{quintuple}
\Erratum{\index{Point!quintiple}\index{Quintiple point}}{\index{Point!quintuple}\index{Quintuple point}}%
point. Thus \ce{Na2SO4}, \ce{MgSO4}, and \ce{H2O} give the double
salt \ce{Na2Mg(SO4)2 4H2O} (astrakanite), the crystals of the
two simple salts, aqueous solution, and water vapour, at a
temperature of $22°$~C. and a pressure of $19.6~\Unit{mm.}$ of mercury.

\Section{206.} We shall now take the case
\index{Heterogeneous system}%
\index{System!heterogeneous}%
\[
\beta = \alpha + 1,
\]
that is, $\alpha$~independent constituents form $\alpha + 1$ phases
(Univariant systems). The composition of all the phases is
then completely determined by a single variable, \eg\ the
temperature or the pressure. This case is generally called
\emph{perfect heterogeneous} equilibrium.

If $\alpha = 1$, then $\beta = 2$: one independent constituent in
two phases, \eg\ a liquid and its vapour. The pressure and
the density of the liquid and the vapour depend on the
temperature alone, as was pointed out in the last chapter.
\PageSep{181}
Evaporation involving chemical decomposition also belongs
to this class, since the system contains only one independent
constituent. The evaporation of solid \ce{NH4Cl}\Typo{.}{} is a case in
point. Unless there be present an excess of hydrochloric
acid or ammonia gas, there will be for each temperature a
quite definite dissociation pressure.

If $\alpha = 2$, then $\beta = 3$, for instance when the solution of a
salt is in contact with its vapour and with the solid salt, or
when two liquids that cannot be mixed in all proportions
(ether and water) are in contact with their common vapour.
Vapour pressure, density and concentration in each phase,
are here functions of the temperature alone.

\Section{207.} We often take the pressure instead of the temperature
as the variable which controls the phases in
perfect heterogeneous equilibrium; namely, in systems
which do not possess a gaseous phase, so-called \emph{condensed}
systems. Upon these the influence of the pressure is so
slight that, under ordinary circumstances, it may be considered
as given, and equal to that of the atmosphere. The
phase rule, therefore, gives rise to the following proposition:
\emph{A condensed system of a independent constituents forms $\alpha + 1$
\index{Condensed system}%
\index{System!condensed}%
phases at most, and is then completely determined, temperature
included.} The melting point of a substance, and the point
of transition from one allotropic modification to another, are
examples of $\alpha = 1$, $\beta = 2$. The point at which the cryohydrate
(ice and solid salt) separates out from the solution of
a salt, and also the point at which two liquid layers in
contact begin to precipitate a solid (\eg\ \ce{AsBr3}, and \ce{H2O})
are examples of $\alpha = 2$, $\beta = 3$. We have an example of
$\alpha = 3$, $\beta = 4$ when two salts, capable of forming a double
salt, are in contact with the solid simple salts, and also with
the double salt.

\Section{208.} If
\index{Divariant system}%
\index{System!divariant}%
\[
\beta = \alpha,
\]
then $\alpha$~independent constituents form $\alpha$~phases (Divariant
systems). The internal nature of all the phases depends
\PageSep{182}
on two variables, \eg\ on temperature and pressure. Any
\index{Temperature!critical solution}%
homogeneous substance furnishes an example of $\alpha = 1$. A
liquid solution of a salt in contact with its vapour is an
example of $\alpha = 2$. The temperature and the pressure determine
the concentration in the vapour as well as in the liquid.
The concentration of the liquid and either the temperature
or the pressure are frequently chosen as the independent
variables. In the first case, we say that a solution of given
concentration and given temperature emits a vapour of
definite composition and definite pressure; and in the second
case, that a solution of given concentration and given pressure
has a definite boiling point, and at this temperature a vapour
of definite composition may be distilled off.

Corresponding regularities hold when the second phase
is solid or liquid, as in the case of two liquids which do not
mix in all proportions. The internal nature of the two
phases, in our example the concentrations in the two layers
of the liquids, depends on two variables---pressure and
temperature. If, under special circumstances, the concentrations
become equal, a phenomenon is obtained which
is quite analogous to that of the critical point of a homogeneous
\index{Critical point!solution temperature}%
substance (critical solution temperature of two
liquids).

\Section{209.} Let us now consider briefly the case
\index{Isomorphous substance}%
\index{Substance, isomorphous}%
\[
\beta = \alpha - 1,
\]
where the number of phases is one less than the number of
the independent constituents, and the internal nature of
all phases depends on a third arbitrary variable, besides
temperature and pressure. Thus, $\alpha = 3$, $\beta = 2$ for an aqueous
solution of two isomorphous substances (potassium chlorate
\index{Potassium chlorate}%
and thallium chlorate) in contact with a mixed crystal. The
\index{Thallium chlorate}%
concentration of the solution under atmospheric pressure
and at a given temperature will vary according to the composition
of the mixed crystal. We cannot, therefore, speak
of a saturated solution of the two substances of definite
composition. However, should a second solid phase---for
\PageSep{183}
instance, a mixed crystal of different composition---separate
out, the internal nature of the system will be determined
by temperature and pressure alone. The experimental
investigation of the equilibrium of such systems may enable
us to decide whether a precipitate from a solution of two
salts forms one phase---for example, a mixed crystal of
changing concentration---or whether the two substances are
to be considered as two distinct phases in contact. If, at
a given temperature and pressure, the concentration of the
liquid in contact were quite definite, it would represent the
former case, and, if not, the latter.

\Section{210.} If the expressions for the functions $\Psi'$,~$\Psi''$,~\dots
for each phase were known, the equations~\Eq{(149)} would give
every detail regarding the state of the equilibrium. This,
however, is by no means the case, for, regarding the relations
between these functions and the masses of the constituents
in the individual phases, all we can, in general, assert is that
they are homogeneous functions of the first degree (\SecRef{201}).
We can, however, tell how they depend upon temperature
and pressure, since their differential coefficients with respect
to $\theta$~and~$p$ can be given. This point leads to far-reaching
conclusions concerning the variation of the equilibrium with
temperature and pressure.

Since, for the first phase, according to~\Eq{(143)},
\[
\Psi' = \Phi' - \frac{U' + pV'}{\theta},
\]
we have, for an infinitely small change,
\[
d\Psi' = d\Phi' - \frac{dU' + p\, dV' + V'\, dp}{\theta} + \frac{U' + pV'}{\theta^{2}}\, d\theta.
\]
Under the assumption that the change is produced only by
variations of $p$~and~$theta$, and not by that of the masses $M_{1}'$,
$M_{2}'$,~\dots\Add{,} $M_{\alpha}'$, equation~\Eq{(60)} gives
\[
d\Phi' = \frac{dU' + p\, dV'}{\theta},
\]
\PageSep{184}
and, therefore,
\[
d\Psi' = \frac{U' + pV'}{\theta^{2}}\, d\theta - \frac{V'\, dp}{\theta},
\]
whence
\[
\frac{\dd \Psi'}{\dd \theta} = \frac{U' + pV'}{\theta^{2}},
\quad\text{and}\quad
\frac{\dd \Psi'}{\dd p} = -\frac{V'}{\theta},
\]
and for the system, as the sum of all the phases,
\[
\frac{\dd \Psi}{\dd \theta} = \frac{U + pV}{\theta^{2}},
\quad\text{and}\quad
\frac{\dd \Psi}{\dd p} = -\frac{V}{\theta}.
\Tag{(150)}
\]

\Section{211.} These relations may be used to determine how
the equilibrium depends on the temperature and pressure.
For this purpose we shall distinguish between two different
kinds of infinitely small changes. The notation~$\delta$ will
refer, as hitherto, to a change of the masses $M_{1}'$, $M_{2}'$,~\dots\Add{,} $M_{\alpha}^{\beta}$,
consistent with the given external conditions, and, therefore,
consistent with the equations~\Eq{(148)}, temperature and pressure
being kept constant, \ie\ $\delta \theta = 0$ and $\delta p = 0$. The state, to
which this variation leads, need not be one of equilibrium,
and the equations~\Eq{(149)} need not, therefore, apply to it. The
notation~$d$, on the other hand, will refer to a change from
one state of equilibrium to another, only slightly different
from it. All external conditions, including temperature and
pressure, may be changed in any arbitrary manner.

The problem is now to find the conditions of equilibrium
of this second state, and to compare them with those of the
original state. Since the condition of equilibrium of the
first state is
\[
\delta \Psi = 0,
\]
the condition for the second state is
\[
\delta (\Psi + d\Psi) = 0,
\]
hence
\[
\delta\, d\Psi = 0\Add{.}
\Tag{(151)}
\]
\PageSep{185}
But
\[
d\Psi = \frac{\dd \Psi}{\dd \theta}\, d\theta + \frac{\dd \Psi}{\dd p}\, dp
  + \tsum^{\beta} \frac{\dd \Psi'}{\dd M_{1}'}\, dM_{1}' + \frac{\dd \Psi'}{\dd M_{2}'}\, \Erratum{dM_{2}}{dM_{2}'} + \dots
\]
where $\tsum$~denotes the summation over all the $\beta$~phases of
the system, while the summation over the $\alpha$~constituents of
a single phase is written out at length. This becomes, by~\Eq{(150)},
\[
d\Psi = \frac{U + pV}{\theta^{2}} - \frac{V}{\theta}\, dp
  + \tsum^{\beta} \frac{\dd \Psi'}{\dd M_{1}'}\, dM_{1}' + \frac{\dd \Psi'}{\dd M_{2}'}\, dM_{2}' + \dots\Add{.}
\]
The condition of equilibrium~\Eq{(151)} therefore becomes
\begin{multline*}
\frac{\delta U + p\, \delta V}{\theta^{2}}\, d\theta - \frac{\delta V}{\theta}\, dp
  + \tsum^{\beta} dM_{1}' \, \delta\frac{\dd \Psi'}{\dd M_{1}'} \\
  + dM_{2}' \, \delta\frac{\dd \Psi'}{\dd M_{2}'} + \dots = 0\Add{.}
\Tag{(152)}
\end{multline*}
All variations of $d\theta$, $dp$, $dM_{1}$, $dM_{2}'$,~\dots disappear because
$\delta \theta = 0$ and $\delta p = 0$, and because in the sum
\begin{align*}
& \frac{\dd \Psi'}{\dd M_{1}'}\, \delta dM_{1}' + \frac{\dd \Psi'}{\dd M_{2}'}\, \delta dM_{2}' + \dots \\
+ & \frac{\dd \Psi''}{\dd M_{1}''}\, \delta dM_{1}'' + \frac{\dd \Psi''}{\dd M_{2}''}\, \delta dM_{2}'' + \dots \\
+ & \dots \\
+& \frac{\dd \Psi^{\beta}}{\dd M_{1}^{\beta}}\, \delta dM_{1}^{\beta} + \frac{\dd \Psi^{\beta}}{\dd M_{2}^{\beta}}\, \delta dM_{2}^{\beta} + \dots
\end{align*}
each vertical column vanishes. Taking the first column for
example, we have, by~\Eq{(149)},
\[
\frac{\dd \Psi'}{\dd M_{1}'} = \frac{\dd \Psi''}{\dd M_{1}''} = \dots = \frac{\dd \Psi^{\beta}}{\dd M_{1}^{\beta}},
\]
\PageSep{186}
and also, by~\Eq{(148)},
\begin{multline*}
\delta dM_{1}' + \delta dM_{1}'' + \dots + \delta dM_{1}^{\beta} \\
  = d(\delta M_{1}' + \delta M_{1}'' + \dots + \delta M_{1}^{\beta}) = 0.
\end{multline*}
Furthermore, since, by the first law, $\delta U + p\, \delta V$ represents~$Q$,
the heat absorbed by the system during the virtual
change, the equation~\Eq{(152)} may also be written
\[
\frac{Q}{\Typo{\theta_{2}}{\theta^{2}}}\, d\theta - \frac{\delta V}{\theta}\, dp
  + \tsum^{\beta} dM_{1}'\, \delta \frac{\dd \Psi'}{\dd M_{1}'} + \Erratum{dM_{2}\, \delta \frac{\dd \Psi''}{\dd M_{2}'}}{dM_{2}'\, \delta \frac{\dd \Psi'}{\dd M_{2}'}} + \dots = 0.
\Tag{(153)}
\]
This equation shows how the equilibrium depends on the
temperature, and the pressure, and on the masses of the
independent constituents of the system. It shows, in
the first place, that the influence of the temperature
depends essentially on the heat effect which accompanies
a virtual change of state. If this be zero, the first term
vanishes, and a change of temperature does not disturb
the equilibrium. If $Q$~change sign, the influence of the
temperature is also reversed. It is quite similar with regard
to the influence of the pressure, which, in its turn, depends
essentially on the change of volume,~$\delta V$, produced by a
virtual isothermal and isopiestic change of state.

\Section{212.} We shall now apply the equation~\Eq{(153)} to several
special cases; first, to those of perfect heterogeneous equilibrium,
which are characterized (\SecRef{206}) by the relation
\[
\beta = \alpha + 1.
\]
The internal nature of all the phases, including the pressure,
is determined by the temperature alone. An isothermal,
infinitely slow compression, therefore, changes only the total
masses of the phases, but does not change either the composition
or the pressure. We shall choose a change of this
kind as the virtual change of state. In this special case it
leads to a new state of equilibrium. The internal nature of
all the phases, as well as the temperature and pressure,
\PageSep{187}
remain constant, and therefore the variations of the functions
$\dfrac{\dd \Psi'}{\dd M_{1}'}$, $\dfrac{\dd \Psi'}{\dd M_{2}'}$,~\dots are all equal to zero, since these quantities
depend only on the nature of the phases. The equation~\Eq{(153)}
therefore becomes
\[
\frac{dp}{d\theta} = \frac{Q}{\theta\, \delta V}.
\Tag{(154)}
\]
This means that the heat effect in a variation that leaves
the composition of all phases unchanged, divided by the
change of volume of the system and by the absolute temperature,
gives the rate of change of the equilibrium pressure
with the temperature. Where application of heat increases
the volume, as in the case of evaporation, the equilibrium
pressure increases with temperature; in the opposite case,
as in the melting of ice, it decreases with increase of
temperature.

\Section{213.} In the case of one independent constituent ($\alpha = 1$,
\index{Latent heat!from phase rule|(}%
and $\beta = 2$), equation~\Eq{(154)} leads immediately to the laws
discussed at length in the preceding chapter; namely, those
concerning the heat of vaporization, of fusion, and of sublimation.
If, for instance, the liquid form the first phase, the
vapour the second phase, and~$L$ denote the heat of vaporization
per unit mass, we have
\begin{align*}
Q &= L\, \delta M''\Add{,} \\
\delta V &= (v'' - v')\, \delta M''\Add{,}
\end{align*}
where $v'$~and $v''$ are the specific volumes of liquid and vapour,
and $\delta M''$~the mass of vapour formed during the isothermal
and isopiestic change of state. Hence, by~\Eq{(154)},
\[
L = \theta\, \frac{dp}{d\theta} (v'' - v'),
\]
which is identical with the equation~\Eq{(111)}.

This, of course, applies to chemical changes as well,
\PageSep{188}
\index{Ammonium carbamate, evaporation of}%
\index{Ammonium chloride, evaporation of}%
whenever the system under consideration contains one constituent
in two distinct phases; for example, to the vaporization
of ammonium chloride (first investigated with regard
to this law by Horstmann), which decomposes into hydrochloric
\index{Horstmann}%
acid and ammonia; or to the vaporization of
ammonium carbamate, which decomposes into ammonia and
carbon dioxide. Here $L$~of our last equation denotes the
heat of dissociation, and $p$~the dissociation pressure, which
depends only on the temperature.

\Section{214.} We shall also consider the perfect heterogeneous
equilibrium of two independent constituents ($\alpha = 2$, $\beta = 3$);
for example, water (suffix~$1$) and a salt (suffix~$2$) in three
phases; the first, an aqueous solution ($M_{1}'$~the mass of the
water, $M_{2}'$~that of the salt); the second, water vapour
(mass~$M_{1}''$); the third, solid salt (mass~$M_{2}'''$). For a virtual
change, therefore,
\[
\delta M_{1}' + \delta M_{1}'' = 0,\quad\text{and}\quad
\delta M_{2}' + \delta M_{2}''' = 0.
\]
According to the phase rule, the concentration of the
solution $\left(\dfrac{M_{2}'}{M_{1}'} = c\right)$, as well as the vapour pressure~($p$) is a
function of the temperature alone. By~\Eq{(154)}, the heat
absorbed ($\theta$,~$p$,~$c$ remaining constant) is
\[
Q = \theta · \frac{dp}{d\theta} · \delta V.
\Tag{(155)}
\]
Let the virtual change consist in the evaporation of a
\index{Evaporation!of ammonium carbamate}%
\index{Evaporation!of ammonium chloride}%
small quantity of water,
\[
\delta M_{1}'' = - \delta M_{1}'.
\]
Then, since the concentration also remains constant, the
quantity of salt
\[
\delta M_{2}''' = -\delta M_{2}' = -c\, \delta M_{1}' = c\, \delta M_{1}''
\]
is precipitated from the solution. All variations of mass
have here been expressed in terms of~$\delta M_{1}''$.
\PageSep{189}

The total volume of the system
\[
V = v' (M_{1}' + M_{2}') + v'' M_{1}'' + v''' M_{2}''',
\]
where $v'$,~$v''$, and~$v'''$ are the specific volumes of the phases,
is increased by
\begin{align*}
\delta V &= v' (\delta M_{1}' + \delta M_{2}') + v''\, \delta M_{1}'' + v'''\, \delta M_{2}'''\Add{,} \\
\delta V &= \bigl[(v'' + cv''') - (1 + c) v'\bigr]\, \delta M_{1}''.
\Tag{(156)}
\end{align*}

If $L$~be the quantity of heat that must be applied to
evaporate unit mass of water from the solution, and to precipitate
the corresponding quantity of salt, under constant
pressure, temperature, and concentration, then the equation~\Eq{(155)},
since
\[
Q = L\, \delta M_{1}'',
\]
becomes
\[
L = \theta\, \frac{dp}{d\theta} \bigl(v'' + cv''' - (1 + c)v'\bigr).
\]

A useful approximation is obtained by neglecting $v'$~and
$v'''$, the specific volumes of the liquid and solid, in comparison
with~$v''$, that of the vapour, and considering the
latter as a perfect gas. By~\Eq{(14)},
\[
v'' = \frac{R}{m} · \frac{\theta}{p}
\]
($R = \text{gas constant}$, $m = \text{the molecular weight of the vapour}$)
and we obtain
\[
L = \frac{R}{m} \theta^{2} · \frac{d \log p}{d\theta}.
\Tag{(157)}
\]

\Section{215.} Conversely, $L$~is at the same time the quantity of
heat given out when unit mass of water vapour combines, at
constant temperature and pressure, with the quantity of salt
necessary to form a saturated solution. This process may
be accomplished directly, or in two steps, viz.\ by condensing
unit mass of water vapour into pure water, and then dissolving
the salt in the water. According to the first law of
thermodynamics, since the initial and final states are the
\index{Latent heat!from phase rule|)}%
\PageSep{190}
same in both cases, the sum of the heat given out and the
\index{Heat!of solution}%
work done is the same.

In the first case the heat given out is~$L$, the work done,
$-p\, \dfrac{\delta V}{\delta M_{1}''}$; and the sum of these, by the approximation used
above, is
\[
\frac{R}{m} \theta^{2} · \frac{d \log p}{d\theta} - pv''.
\Tag{(158)}
\]

To calculate the same sum for the second case, we must
in the first place note that the vapour pressure of a solution
\index{Solution!heat of}%
is different from the vapour pressure of pure water at the
same temperature. It will, in fact, in no case be greater,
but smaller, otherwise the vapour would be supersaturated.
Denoting the vapour pressure of pure water at the temperature~$\theta$
by~$p_{0}$, then $p < p_{0}$.

We shall now bring, by isothermic compression, unit
mass of water vapour from pressure~$p$ and specific volume~$v''$
to pressure~$p_{0}$ and specific volume~$v_{0}''$, \ie\ to a state of saturation.
Work is thereby done on the substance, and heat
is given out. The sum of both, which gives the decrease of
the energy of the vapour, is zero, if we again assume that
the vapour behaves as a perfect gas, \ie\ that its energy
remains constant at constant temperature. If we then
condense the water vapour of volume~$v_{0}''$, at constant temperature~$\theta$
and constant pressure~$p_{0}$, into pure water, the sum
of the heat given out and work spent at this step is, by
equation~\Eq{(112)},
\[
\frac{R}{m} \theta^{2} · \frac{d \log p_{0}}{d\theta} - p_{0} v_{0}''.
\Tag{(159)}
\]
No appreciable external effects accompany the further
change of the liquid water from pressure~$p_{0}$ to pressure~$p$.

If, finally, we dissolve salt sufficient for saturation in the
newly formed unit of water, at constant temperature~$\theta$ and
constant pressure~$p$, the sum of the heat and work is simply
the heat of solution
\[
\lambda\Add{.}
\Tag{(160)}
\]
\PageSep{191}
By the first law, the sum of \Eq{(159)}~and \Eq{(160)} must be equal
to~\Eq{(158)},
\[
\frac{R}{m} \theta^{2} · \frac{\Erratum{d \log p}{d \log p_{0}}}{d\theta} - p_{0} v_{0}'' + \lambda
  = \frac{R}{m} \theta^{2} · \frac{d \log p}{d\theta} - pv'';
\]
or, since, by Boyle's law $p_{0} v_{0}'' = pv''$,
\[
\lambda = \frac{R}{m} \theta^{2} · \frac{d \log \dfrac{p}{p_{0}}}{d\theta}.
\Tag{(161)}
\]
This formula, first established by Kirchhoff, gives the heat
\index{Kirchhoff}%
evolved when salt sufficient for saturation is dissolved in
$1~\Unit{gr.}$ of pure water.

To express~$\lambda$ in calories, $R$~must be divided by the
mechanical equivalent of heat,~$J$\@. By~\Eq{(34)}, $\dfrac{R}{J} = 1.97$, and
since $m = 18$, we have
\[
\lambda = 0.11 \theta^{2} · \frac{d \log \dfrac{p}{p_{0}}}{d\theta}~\Unit{cal}.
\]
It is further worthy of notice that~$p$, the vapour pressure
of a saturated solution, is a function of the temperature
alone, since $c$,~the concentration of a saturated solution,
changes in a definite manner with the temperature.

The quantities neglected in this approximation may, if
necessary, be put in without any difficulty.

\Section{216.} We proceed now to the important case of two
independent constituents in two phases ($\alpha = 2$, $\beta = 2$).
We assume, for the present, that both constituents are contained
in both phases in appreciable quantity, having the
masses $M_{1}'$,~$M_{2}'$ in the first; $M_{1}''$,~$M_{2}''$, in the second phase.
The internal variables are the temperature, the pressure,
\PageSep{192}
and the concentrations of the second constituent in both
phases;
\[
c' = \frac{M_{2}'}{M_{1}'}
\quad\text{and}\quad
c'' = \frac{M_{2}''}{M_{1}''}.
\Tag{(162)}
\]
According to the phase rule, two of the variables, $\theta$,~$p$, $c'$,~$c''$,
are arbitrary.

Equation~\Eq{(153)} leads to the following law regarding the
shift of the equilibrium corresponding to any change of the
external conditions:
\begin{multline*}
\frac{Q}{\theta_{2}}\, d\theta - \frac{\delta V}{\theta}\, dp
  + dM_{1}'\, \delta \frac{\dd \Psi'}{\dd M_{1}'}
  + dM_{2}'\, \delta \frac{\dd \Psi'}{\dd M_{2}'} \\
  + dM_{1}''\, \delta \frac{\dd \Psi''}{\dd M_{1}''}
  + dM_{2}''\, \delta \frac{\dd \Psi''}{\dd M_{2}''} = 0.
\Tag{(163)}
\end{multline*}
Here, for the first phase,
\[
\left.
\begin{aligned}
\delta \frac{\dd \Psi'}{\dd M_{1}'}
  &= \frac{\dd^{2} \Psi'}{\dd M_{1}'^{2}}\, \delta M_{1}'
   + \frac{\dd^{2} \Psi'}{\dd M_{1}'\, \dd M_{2}'}\, \delta M_{2}'\Add{,} \\
\delta \frac{\dd \Psi'}{\dd M_{1}'}
  &= \frac{\dd^{2} \Psi'}{\dd M_{1}'\, \dd M_{2}'}\, \delta M_{1}'
   + \frac{\dd^{2} \Psi'}{\dd M_{2}'^{2}}\, \delta M_{2}'\Add{.}
\end{aligned}
\right\}
\Tag{(164)}
\]
Certain simple relations hold between the derived functions
of~$\Psi'$ with respect to $M_{1}'$~and~$M_{2}'$. For, since, by~\Eq{(144)},
\[
\Psi' = M_{1}'\, \frac{\dd \Psi'}{\dd M_{1}'} + M_{2}'\, \frac{\dd \Psi'}{\dd M_{2}'},
\]
partial differentiation with respect to $M_{1}'$~and $M_{2}'$ gives
\begin{align*}
0 &= M_{1}'\, \frac{\dd^{2} \Psi'}{\dd M_{1}'^{2}}
   + M_{2}'\, \frac{\dd^{2} \Psi'}{\dd M_{1}'\, \dd M_{2}'}, \\
0 &= M_{1}'\, \frac{\dd^{2} \Psi'}{\dd M_{1}'\, \dd M_{2}'}
   + M_{2}'\, \frac{\dd^{2} \Psi'}{\dd M_{2}'^{2}}.
\end{align*}
If we put, for shortness,
\[
M_{1}'\, \frac{\dd^{2} \Psi'}{\dd M_{1}'\, \dd M_{2}'} = \varphi',
\Tag{(165)}
\]
\PageSep{193}
a quantity depending only on the nature of the first phase,
% [** TN: Manually broke footnote]
on $\theta$,~$p$, and~$c'$, and not on the masses $M_{1}'$~and~$M_{2}'$ individually,\footnote
  {The general integral of $\Psi' = M_{1}'\, \dfrac{\dd \Psi'}{\dd M_{1}'} + M_{2}'\, \dfrac{\dd \Psi'}{\dd M_{2}'}$ \\ is $\Psi' = M_{2}' f\left(\dfrac{M_{1}'}{M_{2}'}\right)$.---\Tr.}
we have
\[
\left.
\begin{aligned}
\frac{\dd^{2} \Psi'}{\dd M_{1}'\, \dd M_{2}'} &= \frac{\varphi'}{M_{1}'}\Add{,} \\
\frac{\dd^{2} \Psi'}{\dd M_{1}'^{2}} &= -\frac{M_{2}'}{M_{1}'^{2}} · \varphi'\Add{,} \\
\frac{\dd^{2} \Psi'}{\dd M_{2}'^{2}} &= -\frac{\varphi'}{M_{2}'}\Add{.}
\end{aligned}
\right\}
\Tag{(166)}
\]
Analogous equations hold for the second phase if we put
\[
\varphi'' = M_{1}'' · \frac{\dd \Psi''}{\dd M_{1}''\, \dd M_{2}''}.
\]

\Section{217.} With respect to the quantities $\varphi'$~and~$\varphi''$ all we
can immediately settle is their sign. According to \SecRef{147}, $\Psi$~is
a maximum in stable equilibrium if only processes at
constant temperature and constant pressure be considered.
Hence
\[
\delta^{2} \Psi < 0.
\Tag{(167)}
\]
But
\[
\Psi = \Psi' + \Psi'',
\]
whence
\[
\delta \Psi
  = \frac{\dd \Psi'}{\dd M_{1}'}\, \delta M_{1}' + \frac{\dd \Psi'}{\dd M_{2}'}\, \delta M_{2}'
  + \frac{\dd \Psi''}{\dd M_{1}''}\, \delta M_{1}'' + \frac{\dd \Psi''}{\dd M_{2}''}\, \delta M_{2}''
\]
and
\begin{multline*}
\delta^{2} \Psi
  = \frac{\dd^{2} \Psi'}{\dd M_{1}'^{2}}\, \delta M_{1}'^{2}
  + 2 \frac{\dd^{2} \Psi'}{\dd M_{1}'\, \dd M_{2}'}\, \delta M_{1}'\, \delta M_{2}'
  + \frac{\dd^{2} \Psi'}{\dd M_{2}'^{2}}\, \delta M_{2}'^{2} \\
  + \frac{\dd^{2} \Psi''}{\dd M_{1}''^{2}}\, \delta M_{1}''^{2}
  + 2 \frac{\dd^{2} \Psi''}{\dd M_{1}''\, \dd M_{2}''}\, \delta M_{1}''\, \delta M_{2}''
  + \frac{\dd^{2} \Psi''}{\dd M_{2}''^{2}}\, \delta M_{2}''^{2}.
\end{multline*}
\PageSep{194}
If we introduce the quantities $\varphi'$~and~$\varphi''$, then
\[
\delta^{2} \Psi
  = -M_{2}' \varphi' \left(\frac{\delta M_{1}'}{M_{1}'} - \frac{\delta M_{2}'}{M_{2}'}\right)^{2}
    -M_{2}'' \varphi'' \left(\frac{\delta M_{1}''}{M_{1}''} - \frac{\delta M_{2}''}{M_{2}''}\right)^{2}.
\]
This relation shows that the inequality~\Eq{(167)} is satisfied,
and only then, if both $\varphi'$~and $\varphi''$ are positive.

\Section{218.} There are on the whole two kinds of changes
possible, according as the first or the second constituent
passes from the first to the second phase. We have, for the
first,
\[
\delta M_{1}' = -\delta M_{1}'';\quad
\delta M_{2}' = \delta M_{2}'' = 0;
\Tag{(168)}
\]
and for the second,
\[
\delta M_{1}' = \delta M_{1}'' = 0;\quad
\delta M_{2}' = -\delta M_{2}''.
\]
We shall distinguish~$Q$, the heat absorbed, and $\delta V$,
the change of volume, in these two cases by the suffixes $1$
and~$2$. In the first case, the law for the displacement of the
equilibrium, by \Eq{(163)}, \Eq{(164)}, \Eq{(168)}, \Eq{(166)}, and~\Eq{(162)}, reduces
to
\[
\frac{Q_{1}}{\theta_{2}}\, d\theta - \frac{\delta_{1}V}{\theta}\, dp
  - \delta M_{1}'' (\varphi'\, dc' - \varphi''\, dc'') = 0
\]
and, introducing for shortness the finite quantities
\[
L_{1} = \frac{Q_{1}}{\delta M_{1}''},\quad
s_{1} = \frac{\delta_{1}V}{\delta M_{1}''},
\Tag{(169)}
\]
\ie\ the ratios of the heat absorbed and of the change of
volume to the mass of the first constituent, which passes
from the first to the second phase, we have
\[
\frac{L_{1}}{\theta^{2}}\, d\theta - \frac{s_{1}}{\theta}\, dp - \varphi'\, dc' + \varphi''\, dc'' = 0.
\Tag{(170)}
\]
\PageSep{195}

Similarly, for the second constituent passing into the
second phase, we get
\[
\frac{L_{2}}{\theta^{2}}\, d\theta - \frac{s_{2}}{\theta}\, dp - \varphi'\, \frac{dc'}{c'} + \varphi''\, \frac{dc''}{c''} = 0.
\Tag{(171)}
\]
These are the two relations connecting the four differentials
$d\theta$, $dp$, $dc'$, $dc''$ in any displacement of the equilibrium.

\Section{219.} To show the application of these laws, let us
consider a mixture of two liquids (water and alcohol) in
two phases, the first a liquid, the second a vapour. The
phase rule leaves two of the variables $\theta$, $p$, $c'$, $c''$ arbitrary.
The pressure~$p$, and the concentration~$c''$ of the vapour, for
instance, are determined by the temperature and the concentration~$c'$
of the liquid mixture. Accordingly, for any
changes $d\theta$~and $dc'$ we have, from \Eq{(170)} and~\Eq{(171)},
\begin{align*}
dp &= \frac{\left(\dfrac{c''}{c'} - 1\right) \theta^{2} \varphi'\, dc' + (L_{1} + c'' L_{2})\, d\theta}{(s_{1} + c'' s_{2}) \theta},\displaybreak[0] \\
dc'' &= \frac{\left(\dfrac{1}{s_{1}} + \dfrac{1}{c's_{2}}\right) \varphi'\, dc' - \left(\dfrac{L_{1}}{s_{1}} - \dfrac{L_{2}}{s_{2}}\right) · \dfrac{d\theta}{\theta^{2}}}{\left(\dfrac{1}{s_{1}} + \dfrac{1}{c'' s_{2}}\right) \varphi''}.
\end{align*}
Of the many conclusions to be drawn from these equations,
we mention only the following:

Along an isotherm ($d\theta = 0$) the equations become
\begin{align*}
dp &= \frac{\left(\dfrac{c''}{c'} - 1\right) \theta \varphi'\, dc'}{s_{1} + c'' s_{2}},
\Tag{(172)}\displaybreak[0] \\
dc'' &= \frac{\left(\dfrac{1}{s_{1}} + \dfrac{1}{c's_{2}}\right)}{\left(\dfrac{1}{s_{1}} + \dfrac{1}{c'' s_{2}}\right)} · \frac{\varphi'}{\varphi''}\, dc'.
\Tag{(173)}
\end{align*}
The vapour pressure~$p$ may rise or fall with increasing
\PageSep{196}
concentration. When $p$~shows a maximum or minimum
value, as it does according to Konowalow for a $77 : 23$~mixture
\index{Konowalow}%
of propyl alcohol and water, then $\dfrac{dp}{dc'}$~vanishes, and, from
equation~\Eq{(172)}, $c' = c''$, \ie\ the percentage composition of
the liquid and the vapour is the same, or the liquid boils
at constant concentration. But if, along an isotherm, $p$~varies
with~$c'$, the concentration of the vapour will differ
from that of the liquid; in fact, the concentration of the
second constituent in the vapour will be more or less than
in the liquid ($c'' >$ or $< c'$), according as the vapour pressure~$p$
rises or falls with increasing concentration. This is an
immediate deduction from equation~\Eq{(172)} if we bear in
mind that $\varphi'$, $s_{1}$,~$s_{2}$, and~$c''$ are always positive.

The equation~\Eq{(173)} shows that along an isotherm the
concentration of both phases always changes in the same
sense.

\Section{220.} In the following applications we shall restrict
ourselves to the case in which the second constituent occurs
only in the first phase,
\[
c'' = 0,
\]
and
\[
\therefore
dc'' = 0\Add{.}
\Tag{(174)}
\]
The first constituent which occurs along with the second
in the first phase, and pure in the second, will be called
the \emph{solvent}; the second, the \emph{dissolved substance}. By~\Eq{(174)},
\index{Solvent}%
the equation~\Eq{(171)} is identically satisfied, and from~\Eq{(170)}
there remains
\[
\frac{L}{\theta^{2}}\, d\theta - \frac{s}{\theta}\, dp - \varphi\, dc = 0,
\Tag{(175)}
\]
if we omit suffixes and dashes for simplicity.

We shall take, first, a solution of a \Chg{nonvolatile}{non-volatile} salt in
contact with the vapour of the solvent, and investigate the
equation~\Eq{(175)} in three directions by keeping in turn the
concentration~$c$, the temperature~$\theta$, and the pressure~$p$
constant.
\PageSep{197}

\Section{221.} \Topic{Concentration Constant: $dc = 0$.}---The relation
\index{Boyle and Gay-Lussac}%
between the vapour pressure and the temperature is, by~\Eq{(175)},
\[
\left(\frac{\dd p}{\dd \theta}\right)_{c} = \frac{L}{\theta · s}.
\Tag{(176)}
\]
Here $L$~may be called briefly the heat of vaporization
of the solution. If, instead of regarding $L$~as the ratio of
two infinitely small quantities, we take it to be the heat of
vaporization per unit mass of the solvent, then the mass
of the solvent must be assumed so large that the concentration
is not appreciably altered by the evaporation of
unit mass. The quantity~$s$ may generally be put $= v$, the
specific volume of the vapour. Assuming, further, that the
laws of Boyle and \Typo{Gay Lussac}{Gay-Lussac} hold for the vapour, we get
\[
s = v = \frac{R}{m} · \frac{\theta}{p}
\Tag{(177)}
\]
and, by~\Eq{(176)},
\[
L = \frac{R}{m} \theta^{2} \left(\frac{\dd \log p}{\dd \theta}\right)_{c}.
\]
On the other hand, $L$~is also the quantity of heat given
out when unit mass of the vapour of the solvent combines
at constant temperature and pressure with a large quantity
of a solution of concentration~$c$. This process may be performed
directly, or unit mass of the vapour may be first
condensed to the pure solvent and then the solution diluted
with it.

In the first case the sum of the heat given out and the
work spent is
\[
L - pv = \frac{R}{m} \theta^{2} \left(\frac{\dd \log p}{\dd \theta}\right)_{c} - pv.
\]

In the second case, by the method used in \SecRef{215} we
obtain, as the sum of the heat given out and the work spent
during condensation and dilution,
\[
\frac{R}{m} \theta^{2} \frac{d \log p_{0}}{d\theta} - p_{0} v_{0} + \Delta,
\]
\PageSep{198}
where $p_{0}$~is the pressure, $v$~the specific volume of the vapour
of the solvent in contact with the pure liquid solvent, $\Delta$~the
heat of dilution of the solution, \ie\ the heat given out on
\index{Dilution!heat of}%
\index{Heat!of dilution}%
adding unit mass of the solvent to a large quantity of the
solution of concentration~$c$. Both the above expressions
being equal according to the first law, we obtain, on applying
Boyle's law,
\[
\Delta = \frac{R}{m} \theta^{2} \left(\frac{\dd \log \dfrac{p}{p_{0}}}{\dd \theta}\right)_{c},
\Tag{(178)}
\]
which is Kirchhoff's formula for the heat of dilution.
\index{Kirchhoff's formula}%

The quantities here neglected, by considering the vapour
a perfect gas, and its specific volume large in comparison
with that of the liquid, may readily be taken into account
when necessary.

The similarity of the expressions for~$\Delta$, the heat of
dilution, and for~$\lambda$, the heat of saturation~\Eq{(161)}, is only
external, since in this case the solution may be of any concentration,
and therefore may be differentiated with respect
to the temperature, $c$~being kept constant, while in~\Eq{(161)}
the concentration of a saturated solution changes with
temperature in a definite manner.

\Section{222.} Since $\Delta$~is \emph{small} for small values of~$c$ (dilute solutions,
\SecRef{97}), then, according to~\Eq{(178)}, the ratio of the vapour
pressure of a dilute solution of fixed concentration to the
vapour pressure of the pure solvent is practically independent
of the temperature (Babo's law).
\index{Babo's law}%
\index{Laws:!Babo's}%

\Section{223.} \Topic{Temperature Constant: $d\theta = 0$.}---The relation
between the vapour pressure~($p$) and the concentration~($c$)
of the solution is, according to~\Eq{(175)},
\[
\left(\frac{\dd p}{\dd c}\right)_{\theta} = -\frac{\theta\varphi}{s}.
\Tag{(179)}
\]
Neglecting the specific volume of the liquid in comparison
\PageSep{199}
\index{Lowering!of vapour pressure}%
\index{Vapour pressure, lowering of}%
with that of the vapour, and considering the latter a perfect
gas of molecular weight~$m$, equation~\Eq{(177)} gives
\[
\left(\frac{\dd p}{\dd c}\right)_{\theta} = -\frac{mp\theta}{R},
\]
or
\[
\left(\frac{\dd \log p}{\dd c}\right)_{\theta} = -\frac{m}{R}\, \varphi.
\]
Since $\varphi$~is always positive (\SecRef{217}), the vapour pressure must
decrease with increasing concentration. This proposition
furnishes a means of distinguishing between a solution
and an emulsion. In an emulsion the number of particles
suspended in the solution has no influence on the vapour
pressure.

So long as the quantity~$\varphi$ is undetermined, nothing
further can be stated with regard to the general relation
between the vapour pressure and the concentration.

\Section{224.} As we have $p = p_{0}$ when $c = 0$ (pure solvent), $p - p_{0}$
\index{Laws:!Wüllner's}%
\index{Wüllner's law}%
is small for small values of~$c$. We may, therefore, put
\[
\frac{\dd p}{\dd c} = \frac{p - p_{0}}{c - 0} = \frac{p - p_{0}}{c}.
\]
Hence, by~\Eq{(179)},
\[
p - p_{0} = \frac{c\theta\varphi}{s}
\Tag{(180)}
\]
and substituting for~$s$, as in~\Eq{(177)}, the specific volume of
the vapour, considered a perfect gas, we get
\[
\frac{p - p_{0}}{p} = \frac{cm\varphi}{R}.
\Tag{(181)}
\]
This means that the relative decrease of the vapour pressure
is proportional to the concentration of the solution (Wüllner's
law). For further particulars, see~\SecRef{270}.

\Section{225.} \Topic{Pressure Constant: $dp = 0$.}---The relation
\PageSep{200}
between the temperature (boiling point) and the concentration
is, by~\Eq{(175)},
\[
\left(\frac{\dd \theta}{\dd c}\right)_{p} = \frac{\theta^{2} \varphi}{L}.
\Tag{(182)}
\]
Since $\varphi$~is positive, the boiling point rises with increasing
concentration. By comparing this with the formula~\Eq{(179)}
for the decrease of the vapour pressure, we find that any
solution gives
\[
\left(\frac{\dd \theta}{\dd c}\right)_{p} : \left(\frac{\dd p}{\dd c}\right)_{\theta} = -\frac{\theta s}{L},
\]
\ie\ for an infinitely small increase of the concentration the
rise in the boiling point (at constant pressure) is to the
decrease of the vapour pressure (at constant temperature)
as the product of the absolute temperature and the specific
volume of the vapour is to the heat of vaporization of the
solution.

Remembering that this relation satisfies the identity
\[
\left(\frac{\dd \theta}{\dd c}\right)_{p} : \left(\frac{\dd p}{\dd c}\right)_{\theta} = -\left(\frac{\dd \theta}{\dd p}\right)_{c},
\]
we come immediately to the equation~\Eq{(176)}.

\Section{226.} Let $\theta_{0}$~be the boiling point of the pure solvent
\index{Boiling point, elevation of}%
\index{Elevation of boiling point}%
($c = 0$), then, for some values of~$c$, the difference between
$\theta$~and~$\theta_{0}$ will be small, and we may put
\[
\frac{\dd \theta}{\dd c} = \frac{\theta - \theta_{0}}{c - 0} = \frac{\theta - \theta_{0}}{c},
\]
whereby the equation becomes
\[
\theta - \theta_{0} = \frac{c\theta^{2} \varphi}{L}.
\Tag{(183)}
\]
This means that the elevation of the boiling point is proportional
to the concentration of the solution. For further
details, see~\SecRef{269}.
\PageSep{201}

\Section{227.} Let the second phase consist of the pure solvent
in the solid state instead of the gaseous state, as happens
in the freezing of an aqueous salt solution or in the precipitation
of salt from a saturated solution. In the latter
case, in conformity with the stipulations of~\SecRef{220}, the salt
will be regarded as the first constituent (the solvent), and
water as the second constituent (the dissolved substance).
The equation~\Eq{(175)} is then directly applicable, and may be
discussed in three different ways. We may ask how the
freezing point or the saturation point of a solution of definite
concentration changes with the pressure ($dc = 0$); or, how
the pressure must be changed, in order that a solution of
changing concentration may freeze or become saturated at
constant temperature ($d\theta = 0$); or, finally, how the freezing
point or the saturation point of a solution under given
pressure changes with the concentration ($dp = 0$). In the
last and most important case, if we denote the freezing
point or the saturation point as a function of the concentration
by~$\theta'$, to distinguish it from the boiling point, equation~\Eq{(175)}
gives
\[
\left(\frac{\dd \theta'}{\dd c}\right)_{p} = \frac{\theta^{2} \varphi}{L},
\]
$L$~being the heat absorbed when unit mass of the solvent
\index{Heat!of precipitation}%
\index{Heat!of solidification}%
\index{Precipitation, heat of}%
separates as a solid (ice, salt) from a large quantity of the
solution of concentration~$c$. Since $L$~is often negative, we
may put $L = -L'$ and call $L'$ the \emph{heat of solidification} of
\index{Solidification!heat of}%
the solution or the \emph{heat of precipitation} of the salt. We
have, then,
\[
\left(\frac{\dd \theta'}{\dd c}\right)_{p} = -\frac{\theta^{2} \varphi}{L'}.
\Tag{(184)}
\]
The heat of solidification~($L'$) of a salt solution is always
positive, hence the freezing point is lowered by an increase
of concentration~$c$. On the other hand, if the heat of precipitation~($L'$)
of a salt from a solution be positive, the
saturation point~$theta'$ is lowered by an increase of the mass
of water, or rises with an increase of the concentration of
\PageSep{202}
the salt. If $L'$~be negative, the saturation point is lowered
by an increase of the concentration of the salt. Should we
prefer to designate by~$c$, not the amount of water, but the
amount of salt in a saturated solution, then, according to
the definition of~$c$ in~\Eq{(162)} and of~$\varphi$ in~\Eq{(165)}, we should
have $\dfrac{1}{c}$ replacing~$c$ in~\Eq{(184)} and $c\Erratum{\phi}{\varphi}$~replacing~$\varphi$, and therefore
\[
\left(\frac{\dd \theta'}{\dd c}\right)_{p} = \frac{\theta^{2} \varphi}{cL'}.
\Tag{(185)}
\]
Here $c$~and $\Erratum{\phi}{\varphi}$ have the same meaning as in equation~\Eq{(184)},
which refers to the freezing point of a solution.

\Section{228.} {\Loosen Let $\theta_{0}'$~be the freezing point of the pure solvent
($c = 0$), then, for small values of~$c$, $\theta'$~will be nearly $= \theta_{0}'$,
and we may put}
\[
\frac{\dd \theta'}{\dd c}
  = \frac{\theta' - \theta_{0}'}{c - 0}
  = \frac{\theta' - \theta_{0}'}{c}.
\]
Equation~\Eq{(184)} then becomes
\[
\theta_{0}' - \theta' = \frac{c\theta^{2} \varphi}{L'}\Add{,}
\Tag{(186)}
\]
which means that the lowering of the freezing point is
\index{Lowering!of freezing point}%
proportional to the concentration. For further particulars,
see~\SecRef{271}.

\Section{229.} The positive quantity,~$\varphi$, which occurs in all these
formulæ, has a definite value for a solution of given $c$,~$\theta$,
and~$p$, and is independent of the nature of the second phase.
Our last equations, therefore, connect in a perfectly general
way the laws regarding the lowering of the vapour pressure,
the elevation of the boiling temperature, the depression of the
freezing point, and the change of the saturation point. Only
one of these phenomena need be experimentally investigated
in order to calculate~$\varphi$, and by means of the value thus
determined the others may be deduced for the same solution.
\PageSep{203}

We shall now consider a further case for which $\varphi$~is of
\index{Membranes, semipermeable}%
\index{Semipermeable membranes}%
fundamental importance, viz.\ the state of equilibrium which
ensues when the pure liquid solvent forms the second phase,
not in contact with a solution, for no equilibrium would
thus be possible, but separated from it by a membrane,
permeable to the solvent only. It is true that for no
solution can perfectly \emph{semipermeable} membranes of this character
be manufactured. In fact, the further development of
this theory (\SecRef{259}) will exclude them as a matter of principle,
for in every case the dissolved substance will also
diffuse through the membrane, though possibly at an
extremely slow rate. For the present it is sufficient that
we may, without violating a law of thermodynamics, assume
the velocity of diffusion of the dissolved substance as small
as we please in comparison with that of the solvent. This
assumption is justified by the fact that semipermeability
may be very closely approximated in the case of many
substances. The error committed in putting the rate of
diffusion of a salt through such a membrane equal to zero,
falls below all measurable limits. An exactly similar error
is made in assuming that a salt does not evaporate or
freeze from a solution, for, strictly speaking, this assumption
is not admissible (\SecRef{259}). The condition of equilibrium
of two phases separated by a semipermeable membrane is
contained in the general thermodynamical condition of
equilibrium~\Eq{(145)},
\[
\delta \Psi' + \delta \Psi'' = 0,
\Tag{(187)}
\]
which holds for virtual changes at constant temperature
and pressure in each phase. The only difference between
this case and free contact is, that the pressures in the two
phases may be different. Pressure always means hydrostatic
pressure as measured by a manometer. If, in the general
equation~\Eq{(76)}, we put
\[
W = -p'\, \delta V' - p''\, \delta V'',
\]
it immediately follows that \Eq{(187)}~is the condition of equilibrium.
The further conclusions from~\Eq{(187)} are completely
\PageSep{204}
analogous to those which are derived, when there is a free
surface of contact. Corresponding to~\Eq{(163)} we have for any
displacement of the equilibrium
\[
\frac{Q}{\theta^{2}}\, d\theta - \frac{\delta V'}{\theta}\, dp' - \frac{\delta V''}{\theta}\, dp''
  + dM_{1}'\, \delta \frac{\dd \Psi'}{\dd M_{1}'} + dM_{2}'\, \delta \frac{\dd \Psi'}{\dd M_{2}'} + \dots = 0.
\]
Since the constituent~$2$ occurs only in the first phase, we get,
instead of~\Eq{(175)},
\[
\frac{L}{\theta^{2}}\, d\theta - \frac{s'}{\theta}\, dp' - \frac{s''}{\theta}\, dp'' - \varphi\, dc = 0.
\Tag{(188)}
\]
Here, as in \SecRef{221}, $L$~is the ``\emph{heat of removal}'' of the solvent
from the solution, \ie\ the heat absorbed when, at constant
temperature and constant pressures $p'$~and~$p''$, unit mass of
the solvent passes through the semipermeable membrane
from a large quantity of the solution to the pure solvent.
The change of volume of the solution during this process
is~$s'$ (negative), that of the pure solvent~$s''$ (positive). In the
condition of equilibrium~\Eq{(188)}, three of the four variables
$\theta$, $p'$, $p''$, $c$ remain arbitrary, and the fourth is determined by
their values.

Consider the pressure~$p''$ in the pure solvent as given and
\index{Pressure!osmotic}%
constant, say one atmosphere, then $dp'' = 0$. If, further, we
put $d\theta = 0$ and $dc$~not equal to zero, we are then considering
solutions in which the concentration varies, but the temperature
and the pressure in the pure solvent remains the
same. Then, by~\Eq{(188)},
\[
\left(\frac{\dd p'}{\dd c}\right)_{\theta} = -\frac{\theta\varphi}{s'}.
\]
Since $\varphi > 0$, and $s' < 0$, $p'$~the pressure in the solution
increases with the concentration.

The difference of the pressures in the two phases,
$p' - p'' = P$, has been called the \emph{osmotic pressure} of the
\index{Osmotic pressure}%
solution. Since $p''$~has been assumed constant, we may write
\[
\left(\frac{\dd P}{\dd c}\right)_{\theta} = -\frac{\theta\varphi}{s'}.
\Tag{(189)}
\]
\PageSep{205}

Thus the laws of the osmotic pressure have also been
expressed in terms of~$\varphi$, which controls those of the depression
of the freezing point, the elevation of the boiling point, etc.
Since $\varphi$~is positive, the osmotic pressure increases with increasing
concentration, and also, since $p' - p''$ vanishes when
$c = 0$, the osmotic pressure is necessarily positive.

For small values of~$c$,
\[
\frac{\dd P}{\dd c} = \frac{P - 0}{c - 0} = \frac{P}{c},
\]
and $-s'$~is nearly equal to~$v$ the specific volume of the
solution. It therefore follows from~\Eq{(189)} that
\[
P = \frac{c\theta\varphi}{v}.
\Tag{(190)}
\]

A further discussion of this question will be found in~\SecRef{272}.

\Section{230.} In the preceding paragraphs we have expressed
the laws of equilibrium of several systems, that fulfil the
conditions of \SecRef{220}, in terms of a quantity~$\varphi$ which is characteristic
for the thermodynamical behaviour of a solution.
Starting from the two equations \Eq{(170)} and \Eq{(171)}, we find
that all the relations in question depend on $\varphi'$~and~$\varphi''$. A
better insight into the nature of these quantities is gained
by extending to the liquid state the idea of the molecule,
hitherto applied only to gases. This step is taken in
the next two chapters, and it appears that the manner in
which the idea applies is uniquely determined by the
propositions of thermodynamics, which have been given.

\Section{231.} Just as the conditions of equilibrium~\Eq{(170)} and
\Eq{(171)} were deduced for two independent constituents in two
phases from the general relation~\Eq{(153)}, so in the same
way an entirely analogous deduction may be made in the
general case.

We shall conclude this chapter by giving, briefly, the
\PageSep{206}
results for a system of $\alpha$~independent constituents in $\beta$~phases.

Denoting the concentrations of the independent constituents,
relative to one fixed constituent~$1$, by
\begin{align*}
\frac{M_{2}'}{M_{1}'} &= c_{2}'; &
\frac{M_{3}'}{M_{1}'} &= c_{3}'; &
\frac{M_{4}'}{M_{1}'} &= c_{4}'; \dots \\
%
\frac{M_{2}''}{M_{1}''} &= c_{2}''; &
\frac{M_{3}''}{M_{1}''} &= c_{3}''; &
\frac{M_{4}''}{M_{1}''} &= c_{4}''; \dots \\
\dots && \dots && \dots
\end{align*}
the condition that, by any infinitely small change of the
system: $d\theta$, $dp$, $dc_{2}'$, $dc_{3}'$, $dc_{4}'$,~\dots\Add{,} $dc_{2}''$, $dc_{3}''$, $dc_{4}''$,~\dots, the
equilibrium may remain stable with regard to the passage
of the constituent~$1$ from the phase denoted by one dash to
the phase denoted by two dashes is
\[
\frac{L_{1}}{\theta_{2}}\, d\theta - \frac{s_{1}}{\theta}\, dp
  + (\varphi_{2}''\, dc_{2}'' - \varphi_{2}'\, dc_{2}')
  + (\varphi_{3}''\, dc_{3}'' - \varphi_{3}'\, dc_{3}') + \dots = 0,
\]
where, analogous to~\Eq{(165)},
\begin{align*}
\varphi_{2}' &= M_{1}'\, \delta \frac{\dd^{2} \Psi'}{\dd M_{1}'\, \dd M_{2}'}; &
\varphi_{3}' &= M_{1}'\, \delta \frac{\dd^{2} \Psi'}{\dd M_{1}'\, \dd M_{3}'}; & \dots \\
%
\varphi_{2}'' &= M_{1}''\, \delta \frac{\dd^{2} \Psi''}{\dd M_{1}''\, \dd M_{2}''}; &
\varphi_{3}'' &= M_{1}''\, \delta \frac{\dd^{2} \Psi''}{\dd M_{1}''\, \dd M_{3}''}; & \dots \\
\end{align*}
and $L_{1}$,~$s_{1}$ denote the heat absorbed, and the increase of
volume of the system during the isothermal and isopiestic
transference of unit mass of constituent~$1$ from a large
quantity of the phase denoted by one dash to a large
quantity of the phase denoted by two dashes. The corresponding
conditions of equilibrium for any possible passage
of any constituent from any one phase to any other phase
may be established in the same way.
\PageSep{207}


\Chapter{IV.}{Gaseous System.}
\index{Gaseous system|(}%
\index{System!gaseous|(}%

\Section{232.} \First{The} relations, which have been deduced from the
general condition of equilibrium~\Eq{(79)} for the different properties
of thermodynamical equilibria, rest mainly on the
relations between the characteristic function~$\Psi$, the temperature,
and the pressure as given in the equations~\Eq{(150)}. It
will be impossible to completely answer all questions regarding
equilibrium until can be expressed in its functional
relation to the masses of the constituents in the different
phases. The introduction of the molecular weight serves
this purpose. Having already defined the molecular weight
of a chemically homogeneous gas as well as the number of
molecules of a mixture of gases by Avogadro's law, we shall
turn first to the investigation of a system consisting of one
gaseous phase.

The complete solution of the problem consists in expressing
$\Psi$~in terms of $\theta$,~$p$, and $n_{1}$,~$n_{2}$, $n_{3}$,~\dots, the number of
all the different kinds of molecules in the mixture.

Since we have, in general, by~\Eq{(75)},
\[
\Psi = \Phi - \frac{U + pV}{\theta},
\]
we are required to express the entropy~$\Phi$, the energy~$U$, and
the volume~$V$ as functions of the above independent variables.
This can be done, in general, on the assumption that the
mixture obeys the laws of perfect gases. Such a restriction
will not, in most cases, lead to appreciable errors. Even this
assumption may be set aside by special measurement of the
\PageSep{208}
quantities $\Phi$,~$U$, and~$V$, as is given later. For the present,
however, perfect gases will be assumed.

\Section{233.} The laws of Boyle, \Typo{Gay Lussac}{Gay-Lussac}, and Dalton determine
the volume of the mixture, for equation~\Eq{(16)} gives
\[
V = \frac{R\theta}{p} (n_{1} + n_{2} + \dots) = \frac{R\theta}{p} \tsum n_{1}.
\Tag{(191)}
\]
By the first law of thermodynamics, the energy~$U$ of a
mixture of gases is given by the energies of its constituents,
for, according to this law, the energy of the system remains
unchanged, no matter what internal changes take place,
provided there are no external effects. Experience shows
that when diffusion takes place between a number of gases
at constant temperature and pressure, neither does the
volume change, nor is heat absorbed or evolved. The energy
of the system, therefore, remains constant during the process.
Hence, the energy of a mixture of perfect gases is the sum
of the energies of the gases at the same temperature and
pressure. Now the energy~$U_{1}$ of $n_{1}$~molecules of a perfect
gas depends only on the temperature; it is, by~\Eq{(35)},
\[
U_{1} = \tsum n_{1} (c_{v_{1}}\theta + h_{1}),
\Tag{(192)}
\]
where $c_{v_{1}}$~is the molecular heat of the gas at constant volume,
and $h_{1}$~is a constant. Hence the total energy of the mixture
is
\[
U = \tsum n_{1} (c_{v_{1}}\theta + h_{1}).
\Tag{(193)}
\]

\Section{234.} We have now to determine the entropy~$\Phi$ as a
function of $\theta$,~$p$, and $n_{1}$,~$n_{2}$,~\dots the number of molecules.
$\Phi$,~in so far as it depends on $\theta$~and~$p$, may be calculated from
the equation~\Eq{(60)},
\[
d\Phi = \frac{dU + p\, dV}{\theta},
\]
\PageSep{209}
where the differentials correspond to variations of $\theta$~and~$p$,
but not of the number of molecules.

Now, by~\Eq{(193)},
\index{Energy!of gas mixture}%
\index{Entropy!of gas mixture|(}%
\index{Gas mixture!energy of}%
\index{Gas mixture!entropy of|(}%
\begin{align*}
dU &= \tsum n_{1} c_{v_{1}}\, d\theta, \\
\intertext{and, by~\Eq{(191)},}
dV &= R \tsum n_{1}\, d\left(\frac{\theta}{p}\right), \\
\therefore
d\Phi &= \tsum n_{1}\, \left(c_{v_{1}}\, \frac{d\theta}{\theta} + \frac{R\, d\theta}{\theta} - \frac{R\, dp}{p}\right),
\end{align*}
and, by integration,
\[
\Phi = \tsum n_{1} \left(c_{v_{1}} \log \theta + R \log \frac{\theta}{p}\right) + C.
\Tag{(194)}
\]
The constant of integration~$C$ is independent of $\theta$~and~$p$,
but may depend on the composition of the mixture, \ie\ on
the numbers $n_{1}$,~$n_{2}$, $n_{3}$,~\dots. The investigation of this relation
forms the most important part of our problem. The determination
of the constant is not, in this case, a matter of
definition. It can only be determined by applying the
second law of thermodynamics to a reversible process which
brings about a change in the composition of the mixture.
A reversible process produces a definite change of the entropy
which may be compared with the simultaneous changes of
the number of molecules, and thus the relation between the
entropy and the composition determined. If we select a
process devoid of external effects either in work or heat,
then the entropy remains constant during the whole process.
We cannot, however, use the process of diffusion, which leads
to the value of~$U$; for diffusion, as might be expected, and
as will be shown in~\SecRef{238}, is irreversible, and therefore leads
only to the conclusion that the entropy of the system is
thereby increased. There is, however, a reversible process
at our disposal, which will change the composition of the
\PageSep{210}
mixture, viz.\ the separation by a semipermeable membrane,
as introduced and established in~\SecRef{229}.

\Section{235.} Before we can apply a semipermeable membrane
to the purpose in hand, we must acquaint ourselves with
the nature of the thermodynamical equilibrium of a gas in
contact with both sides of a membrane permeable to it.
The membrane will act like a bounding wall to those gases
to which it is impermeable, and will, therefore, not introduce
any special conditions. Experience shows that a gas on both
sides of a membrane permeable to it is in equilibrium when
its partial pressures (\SecRef{18}) are the same on both sides, quite
independent of the other gases present. This proposition
is neither axiomatic nor a necessary consequence of the
preceding considerations, but it commends itself by its
simplicity, and has been confirmed without exception in
the few cases accessible to direct experiment.

A test of this kind may be established as follows:
Platinum foil at a white heat is permeable to hydrogen, but
impermeable to air. If a vessel having a platinum wall be
filled with pure hydrogen, and hermetically sealed, and the
platinum be then heated, the hydrogen must completely
diffuse out against atmospheric pressure. As the air cannot
enter, the vessel must finally become completely exhausted.\footnote
  {This inference was tested by me in the Physical Institute of the
  University of Munich in 1883, and was confirmed within the limits of
  experimental error as far as the actual deviation from ideal conditions might
  lead one to expect. As this experiment has not been published anywhere,
  I shall briefly describe it here. A glass tube of about $5~\Unit{mm.}$ internal diameter,
  blown out to a bulb at the middle, was provided with a stop-cock at one end.
  To the other end a platinum tube $10~\Unit{cm.}$ long was fastened, and closed at the
  end. The whole tube was exhausted by the mercury pump, filled with
  hydrogen at ordinary atmospheric pressure, and then closed. The closed
  end of the platinum portion was then heated in a horizontal position by a
  Bunsen burner. The connection between the glass and platinum tubes
  having been made by means of sealing-wax, had to be kept cool by a
  continuous current of water to prevent the softening of the wax. After
  four hours the tube was taken from the flame, cooled to the temperature of
  the room, and the stop-cock opened under mercury. The mercury rose
  rapidly, almost completely filling the tube, proving that the tube had been
  very nearly exhausted.}
\PageSep{211}

\Section{236.} We shall make use of the properties of semipermeable
membranes to separate in a reversible and simple
manner the constituents of a gas mixture. Let us consider
the following example:---

Let there be four pistons in a hollow cylinder, two of
them, $A$~and~$A'$, in fixed positions; two, $B$~and~$B'$, movable
in such a way that the distance~$BB'$ remains constant, and
equal to~$AA'$. This is indicated by the brackets in \Fig{5}.
\Figure{5}%
Further, let $A'$~(the bottom), and $B$~(the cover) be impermeable
to any gas, while $A$~is permeable only to one
gas~$(1)$, and $B'$~only to another one~$(2)$. The space above~$B$
is a vacuum.

At the beginning of the process the piston~$B$ is close to~$A$,
therefore $B'$~close to~$A'$, and the space between them
contains a mixture of the two gases ($1$~and~$2$). The connected
pistons $B$~and $B'$ are now very slowly raised. The
gas~$1$ will pass into the space opening up between $A$ and~$B$,
and the gas~$2$ into that between $A'$ and~$B'$. Complete
separation will have been effected when $B'$~is in contact with~$A$.
\PageSep{212}
We shall now calculate the external work of this process.
The pressure on the movable piston~$B$ consists only
of the pressure of the gas~$1$, upwards, since there is a vacuum
above~$B$; and on the other movable piston,~$B'$, there is only
the partial pressure of the same gas, which acts downwards.
According to the preceding paragraph both these pressures
are equal, and since the paths of $B$ and~$B'$ are also equal,
the total work done on the pistons is zero. If no heat be
absorbed or given out, as we shall further assume, the energy
of the system remains constant. But, by~\Eq{(193)}, the energy
of a mixture of gases depends, like that of pure gases, on
the temperature alone, so the temperature of the system
remains constant throughout.

Since this infinitely slow process is reversible, the
entropy in the initial and final states is the same, if there
are no external effects. Hence, the entropy of the mixture
is equal to the sum of the entropies which the two gases
would have, if at the same temperature each by itself occupied
the whole volume of the mixture. This proposition
may be easily extended to a mixture of any number of gases.
\emph{The entropy of a mixture of gases is the sum of the entropies
which the individual gases would have, if each at the same
temperature occupied a volume equal to the total volume of the
mixture.} This proposition was first established by Gibbs.
\index{Gibbs}%

\Section{237.} The entropy of a perfect gas of mass~$M$ and molecular
weight~$m$ was found to be~\Eq{(52)}
\[
M \left(\frac{c_{v}}{m} \log \theta + \frac{R}{m} \log v + \const\right),
\]
where $c_{v}$~is the molecular heat at constant volume, as in~\Eq{(192)}.
By the gas laws~\Eq{(14)}, the volume of unit mass is
\[
v = \frac{R}{m} · \frac{\theta}{p},
\]
whence the entropy is
\[
n (c_{v} \log \theta + R \log \frac{\theta}{p} + k),
\Tag{(195)}
\]
\PageSep{213}
where $n = \dfrac{M}{m}$, the number of molecules, and the constant~$k$
includes the term $\log \dfrac{R}{m}$. Hence, according to Gibbs's proposition,
the entropy of the mixture is
\[
\Phi = \tsum n_{1} (c_{v_{1}} \log \theta + R \log \frac{\theta}{p_{1}} + k_{1}),
\]
$p_{1}$~being the partial pressure of the first gas in the mixture.

Now, by~\Eq{(8)}, the pressure of the mixture is the sum of
the partial pressures, $\tsum p_{1} = p$, and, by \SecRef{40}, the partial
pressures are proportional to the number of molecules of
each gas,
\[
p_{1} : p_{2} : \dots = n_{1} : n_{2} : \dots\Add{.}
\]
Hence
\begin{align*}
p_{1} &= \frac{n_{1}}{n_{1} + n_{2} + \dots}\, p\Add{,} \\
p_{2} &= \frac{n_{2}}{n_{1} + n_{2} + \dots}\, p\Add{,} \\
& \dots\Add{,}
\end{align*}
or, if we introduce the \emph{concentrations} of the different gases
in the mixture,
\begin{align*}
c_{1} &= \frac{n_{1}}{n_{1} + n_{2} + \dots}; &
c_{2} &= \frac{n_{2}}{n_{1} + n_{2} + \dots}; \\
p_{1} &= c_{1} p; & p_{2} &= c_{2} p.
\Tag{(196)}
\end{align*}

Thus the expression for the entropy of a mixture as a
function of $\theta$,~$p$, and $n$~the number of molecules, finally
becomes
\[
\Phi = \tsum n_{1} (c_{v_{1}} \log \theta + R \log \frac{\theta}{pc_{1}} + k_{1}).
\Tag{(197)}
\]
Comparing this expression with the value of the entropy of
\PageSep{214}
a gas mixture given by~\Eq{(194)}, we see that the constant of
\index{Gas mixture!entropy of|)}%
integration which was left undetermined is
\[
C = \tsum n_{1} (k_{1} - R \log c_{1})\Add{.}
\Tag{(198)}
\]

\Section{238.} Knowing the value of the entropy of a gas
\index{Diffusion!increase of entropy by}%
\index{Diffusion!irreversible}%
\index{Entropy!increase of, by diffusion}%
\index{Irreversible diffusion}%
mixture, we may answer the question which we discussed
in~\SecRef{234}, whether and to what extent the entropy of a
system of gases is increased by diffusion. Let us take the
simplest case, that of two gases, the number of molecules
being $n_{1}$~and $n_{2}$, diffusing into one another under common
and constant pressure and temperature. Before diffusion
begins, the entropy of the system is the sum of the entropies
of the gases, by~\Eq{(195)},
\[
n_{1} (c_{v_{1}} \log \theta + R \log \frac{\theta}{\Typo{p}{p_{1}}} + k_{1})
  + n_{2} (c_{v_{2}} \log \theta + R \log \frac{\theta}{\Typo{p}{p_{2}}} + k_{2}).
\]
After diffusion it is, by~\Eq{(197)},
\[
n_{1} (\Erratum{c_{v_{2}}}{c_{v_{1}}} \log \theta + R \log \frac{\theta}{pc_{1}} + k_{1})
  + n_{2} (c_{v_{2}} \log \theta + R \log \frac{\theta}{pc_{2}} + k_{2}).
\]
Therefore, the change of the entropy of the system is,
by~\Eq{(196)},
\[
-n_{1} R \log c_{1} - n_{2} R \log c_{2}\Add{,}
\]
an essentially positive quantity. This shows that diffusion
is always irreversible.

It also appears that the increase of the entropy depends
solely on the number of the molecules $n_{1}$,~$n_{2}$, and not on
the nature---\eg\ the molecular weight, of the diffusing gases.
The increase of the entropy does not depend on whether the
gases are chemically alike or not. By making the two
gases the same, there is evidently no increase of the entropy,
\index{Entropy!of gas mixture|)}%
since no change of state ensues. It follows that the chemical
difference of two gases, or, in general, of two substances,
cannot be represented by a continuous variable; but that
\PageSep{215}
here we can speak only of a discontinuous relation, either
of equality or inequality. This fact involves a fundamental
distinction between chemical and physical properties, since
the latter may always be regarded as continuous.

\Section{239.} The values of the entropy~\Eq{(197)}, the energy~\Eq{(193)},
and the volume~\Eq{(191)}, substituted in~\Eq{(75)}, give the
function~$\Psi$,
\[
\Psi = \tsum n_{1} (c_{v_{1}} \log \theta + R \log \frac{\theta}{pc_{1}} + k_{1} - c_{v_{1}} - \frac{h_{1}}{\theta} - R);
\]
or, putting the quantity, which depends on $p$~and~$\theta$, and
not on the number of molecules,
\begin{gather*}
c_{v_{1}} \log \theta - \frac{h_{1}}{\theta} + R \log \frac{\theta}{p} + k_{1} - c_{v_{1}} - R = \varphi_{1},
\Tag{(199)} \\
\Psi = \tsum n_{1} (\varphi_{1} - R \log c_{1})\Add{.}
\end{gather*}

\Section{240.} This enables us to establish the condition of
\index{Condition of complete reversibility!of a gas mixture|(}%
equilibrium. If in a gas mixture a chemical change, which
\index{Equilibrium!of gas mixture}%
changes the number of molecules $n_{1}$,~$n_{2}$,~\dots by $\delta n_{1}$,~$\delta n_{2}$,~\dots
be possible, then such a change will not take place if the
condition of equilibrium~\Eq{(79)} be fulfilled, \ie\ if, when $\delta \theta = 0$
and $\delta p = 0$,
\[
\delta \Psi = 0,
\]
or
\[
\tsum (\varphi_{1} - R \log c_{1})\, \delta n_{1}
  + \tsum n_{1}\, \delta(\varphi_{1} - R \log c_{1}) = 0.
\Tag{(200)}
\]
The quantities $\varphi_{1}$, $\varphi_{2}$\Add{,}~\dots depend on $\theta$~and $p$~only, therefore
\[
\delta \varphi_{1} = \delta \varphi_{2} = \dots = 0.
\]
Further,
\[
n_{1}\, \delta \log c_{1} + n_{2}\, \delta \log c_{2} + \dots
  = \frac{n_{1}}{c_{1}}\, \delta c_{1} + \frac{n_{2}}{c_{2}}\, \delta c_{2} + \dots
\]
and, by~\Eq{(196)},
\[
= (n_{1} + n_{2} + \dots)(\delta c_{1} + \delta c_{2} + \dots) = 0,
\]
since
\[
c_{1} + c_{2} + \dots = 1.
\]
\PageSep{216}
The condition of equilibrium, therefore, reduces to
\[
\tsum (\varphi_{1} - R \log c_{1})\, \delta n_{1} = 0.
\]
Since this equation does not involve the absolute values of
the variations~$\delta n_{1}$, but only their ratios, we may put
\[
\delta n_{1} : \delta n_{2} : \dots = \nu_{1} : \nu_{2} : \dots
\Tag{(201)}
\]
and take $\nu_{1}$,~$\nu_{2}$\Add{,}~\dots to denote the number of molecules
simultaneously passing into the mass of each constituent.
They are simple integers, positive or negative, according as
the gas in question is forming, or is being used up in the
formation of others. The condition of equilibrium now
becomes
\[
\tsum (\varphi_{1} - R \log c_{1}) \nu_{1} = 0,
\]
or
\[
\nu_{1} \log c_{1} + \nu_{2} \log c_{2} + \dots
  = \frac{\nu_{1} \varphi_{1} + \nu_{2} \varphi_{2} + \dots}{R}.
\]
The right-hand side of the equation depends only on temperature
and pressure~\Eq{(199)}. The equation gives a definite
relation between the concentrations of the different kinds of
molecules for given temperature and pressure.

\Section{241.} We shall now substitute the values of $\varphi_{1}$,~$\varphi_{2}$,~\dots\Add{.}
If, for shortness, we put the constants
\begin{align*}
\frac{\tsum \nu_{1} (k_{1} - c_{v_{1}} - R)}{R} &= \log a\Add{,} \\
\frac{\tsum \nu_{1} h_{1}}{R} = b\Add{,}
\Tag{(202)} \\
\frac{\tsum \nu_{1} c_{v_{1}}}{R} = c\Add{,}
\Tag{(203)} \\
\end{align*}
\PageSep{217}
then
\[
\nu_{1} \log c_{1} + \nu_{2} \log c_{2} + \dots
  = \log a + (\nu_{1} + \nu_{2} + \dots) \log \frac{\theta}{p} - \frac{b}{\theta} + c \log \theta,
\]
or
\begin{align*}
c_{1}^{\nu_{1}} c_{2}^{\nu_{2}} \dots &= a\left(\frac{\theta}{p}\right)^{\nu_{1} + \nu_{2} + \dots} e^{-\efrac{b}{\theta}} \theta^{c}\Add{,} \\
\prod c_{1}^{\nu_{1}} &= a\left(\frac{\theta}{p}\right)^{\tsum \nu_{1}} e^{-\efrac{b}{\theta}} \theta^{c}\Add{.}
\end{align*}

\Section{242.} This condition may be further simplified by
making use of the experimental fact (\SecRef{50}) that the atomic
heat of an element remains unchanged in its combinations.
By equation~\Eq{(203)} $Rc$~is the change of the sum of the
molecular heats of all the molecules of the system during
the reaction. The sum of the molecular heats, however,
being the sum of the atomic heats, remains unchanged,
hence $c = 0$, and the equation becomes
\[
\prod c_{1}^{\nu_{1}} = a e^{-\efrac{b}{\theta}} \left(\frac{\theta}{p}\right)^{\tsum \nu_{1}}.
\]

\Section{243.} According to this equation the influence of the
pressure on the equilibrium depends entirely on the number~$\tsum \nu_{1}$,
\index{Equilibrium!of gas mixture}%
which gives the degree to which the total number of
molecules, therefore also the volume of the mixture, is
increased by the reaction considered. Where the volume
remains unchanged, as, \eg, in the dissociation of hydriodic
acid, considered below, the equilibrium is independent of
the pressure.

The influence of the temperature depends further on the
constant~$b$, which is closely connected with the heat effect
of the reaction. For, by the first law,
\[
Q = \delta U + p\, \delta V,
\]
which, by \Eq{(193)} and~\Eq{(191)}, $\theta$~and $p$ being constant, becomes
\[
Q = \tsum (c_{v_{1}} \theta + h_{1} + R\theta)\, \delta n_{1}.
\]
\index{Condition of complete reversibility!of a gas mixture|)}%
\PageSep{218}
If we refer the heat effect to the finite numbers~$\nu$, instead
of the infinitely small numbers~$\delta n$, then the heat absorbed is:
\[
L = \tsum (c_{v_{1}} \theta + h_{1} + R\theta) \nu_{1},
\]
and by \Eq{(202)} and~\Eq{(203)}, again putting $c = 0$,
\[
L = Rb + R\theta \tsum \nu_{1},
\]
or,
\[
L = 1.97 (b + \theta \tsum \nu_{1})~\Unit{cal.}
\]
The term containing~$b$ refers to the heat spent in the increase
of the internal energy; the term containing~$\theta$ to that spent
in external work.

\Section{244.} Before proceeding to numerical applications, we
shall enumerate the principal equations.

Suppose that in a gaseous system
\[
n_{1}, m_{1};\quad
n_{2}, m_{2};\quad
n_{3}, m_{3};\ \dots
\]
($n$~the number of molecules, $m$~the molecular weight) any
chemical change be possible, in which the simultaneous
changes of the number of molecules are
\[
\delta n_{1} : \delta n_{2} : \delta n_{3} \dots = \nu_{1} : \nu_{2} : \nu_{3} : \dots
\]
($\nu$~simple, positive or negative integers) then there will be
equilibrium, if the concentrations
\[
c_{1} = \frac{n_{1}}{n_{1} + n_{2} + \Add{\dots}};\quad
c_{2} = \frac{n_{2}}{n_{1} + n_{2} + \Add{\dots}};\ \dots
\]
satisfy the condition
\begin{align*}
c_{1}^{\nu_{1}} c_{2}^{\nu_{2}} c_{3}^{\nu_{3}} \dots
  &= ae^{-\efrac{b}{\theta}} \left(\frac{\theta}{p}\right)^{\nu_{1} + \nu_{2} + \nu_{3} \dots}\Add{,} \\
\prod c_{1}^{\nu_{1}}
  &= ae^{-\efrac{b}{\theta}} \left(\frac{\theta}{p}\right)^{\tsum \nu_{1}}\Add{.}
\Tag{(204)}
\end{align*}
\PageSep{219}
The heat absorbed during the change at constant temperature
and pressure is
\[
L = 1.97 \{b + (\nu_{1} + \nu_{2} + \dots) \theta\} = 1.97 (b + \theta \tsum \nu_{1})~\Unit{cal.}\Add{,}
\Tag{(205)}
\]
while the change of volume is
\[
s = R(\nu_{1} + \nu_{2} + \dots) \frac{\theta}{p} = R\, \frac{\theta}{p} \tsum \nu_{1}.
\Tag{(206)}
\]

\Section{245.} \Topic{Dissociation of Hydriodic Acid.}---Since hydriodic
\index{Dissociation!of hydriodic acid}%
\index{Hydriodic acid, dissociation of}%
acid gas splits partly into hydrogen and iodine vapour, the
system is represented by three kinds of molecules:
\[
\ce{$n_{1}$ HI};\quad
\ce{$n_{2}$ H2};\quad
\ce{$n_{3}$ I2};
\]
The concentrations are:
\[
c_{1} = \frac{n_{1}}{n_{1} + n_{2} + n_{3}};\quad
c_{2} = \frac{n_{2}}{n_{1} + n_{2} + n_{3}};\quad
c_{3} = \frac{n_{3}}{n_{1} + n_{2} + n_{3}}.
\]
The reaction consists in the transformation of two molecules
of~\ce{HI} into one of~\ce{H2} and one of~\ce{I2}:
\[
\nu_{1} = -2;\quad
\nu_{2} = 1;\quad
\nu_{3} = 1.
\]
By~\Eq{(204)}, therefore, in the state of equilibrium,
\[
c_{1}^{-2} c_{2}^{1} c_{3}^{1} = ae^{-\efrac{b}{\theta}}\Add{,}
\]
or
\[
\frac{c_{2}c_{3}}{c_{1}^{2}} = \frac{n_{2}n_{3}}{n_{1}^{2}} = ae^{-\efrac{b}{\theta}}\Add{.}
\Tag{(207)}
\]
Since the total number of atoms of hydrogen $(n_{1} + 2n_{2})$
and of iodine $(n_{1} + 2n_{3})$ in the system are supposed to be
known, equation~\Eq{(207)} is sufficient for the determination of
the three quantities, $n_{1}$,~$n_{2}$, and~$n_{3}$, at any given temperature.
The pressure has in this case no influence on the equilibrium.
Any two measurements of the degree of dissociation
are sufficient for the calculation of $a$~and~$b$. From Bodenstein's
\index{Bodenstein}%
measurements we have for
\[
\theta = 273 + 448 = 712;\quad
\frac{c_{2}c_{3}}{c_{1}^{2}} = 0.01984;
\]
\PageSep{220}
and for
\[
\theta = 273 + 350 = 623;\quad
\frac{c_{2}c_{3}}{c_{1}^{2}} = 0.01494.
\]
Hence, by~\Eq{(207)},
\[
a = 0.120;\quad
b = 1300.
\]
Thus the equilibrium of any mixture of hydriodic acid,
hydrogen, and iodine vapour at any temperature, even
when the hydrogen and the iodine are not present in
equivalent quantities, is determined by~\Eq{(207)}. Equation~\Eq{(205)}
gives the heat of dissociation of two molecules of
\index{Dissociation!of iodine vapour}%
\index{Iodine vapour, dissociation of}%
hydriodic acid into a molecule of hydrogen and a molecule
of iodine vapour:
\[
L = 1.971 × 1300 = 2560~\Unit{cal.}
\]

\Section{246.} \Topic{Dissociation of Iodine Vapour.}---At high temperatures
iodine vapour appreciably decomposes, leading to
a system of two kinds of molecules:
\[
\ce{$n_{1}$ I2};\quad
\ce{$n_{2}$ I}.
\]
The concentrations are
\[
c_{1} = \frac{n_{1}}{n_{1} + n_{2}};\quad
c_{2} = \frac{n_{2}}{n_{1} + n_{2}}.
\]
The reaction consists in a splitting of the molecule~\ce{I2} into
two molecules~\ce{I},
\[
\therefore
\nu_{1} = - 1;\quad
\nu_{2} = 2;
\]
and in equilibrium, by~\Eq{(204)},
\[
c_{1}^{-1} c_{2}^{2} = \frac{n_{2}^{2}}{n_{1} (n_{1} + n_{2})}
  = a' e^{-\efrac{b'}{\theta}} · \frac{\theta}{p}\Add{.}
\Tag{(208)}
\]
$a'$~and $b'$ may be calculated from data given by Fr.~Meier
and Crafts. When $p = 728~\Unit{mm.}$ of mercury,
\index{Crafts}%
\index{Meier, Fr.}%
\begin{align*}
\frac{n_{2}}{2n_{1} + n_{2}} &= 0.145\quad\text{when}\quad \theta = 273 + 940 = 1213, \\
\llap{\text{and\qquad\qquad}} &= 0.662\quad\text{when}\quad \theta = 273 + 1390 = 1663.
\end{align*}
\PageSep{221}
This gives, if $p$~be measured in millimeters of mercury,
\[
a' = 9375;\quad
b' = 14690;
\]
from which the equilibrium of dissociation may be determined
\index{Dissociation!graded}%
for any temperature and pressure.

The heat of dissociation of a molecule of iodine is,
by~\Eq{(205)},
\[
L = 1.97 (14690 + \theta) = 28900 + 1.97\theta~\Unit{cal.}
\]
It will be seen that at such temperatures the external work,
on which the second term depends, has an appreciable
influence. At $1500°$~C. ($\theta = 1773$) it amounts to $3500~\Unit{cal.}$,
making the heat of dissociation
\[
L = 32400~\Unit{cal.}
\]

\Section{247.} \Topic{Graded Dissociation.}---Since, by equation~\Eq{(208)},
\index{Graded dissociation}%
the concentration~$c_{2}$ of the monatomic iodine molecules does
not vanish even at low temperatures, the decomposition of
the iodine vapour should be taken into account in determining
the dissociation of hydriodic acid. This will have
practically no influence on the results of \SecRef{245}, but nevertheless
we shall give the more rigorous solution on account
of the theoretical interest which attaches to it.

There are now four kinds of molecules in the system:
\[
\ce{$n_{1}$ HI};\quad
\ce{$n_{2}$ H2};\quad
\ce{$n_{3}$ I2};\quad
\ce{$n_{4}$ I}.
\]
Two kinds of chemical changes are possible:
\begin{align*}
\text{(1)}\quad
\nu_{1} &= -2;\quad \nu_{2} = 1;\quad \nu_{3} = 1;\quad \nu_{4} = 0;\quad\text{and} \\
\text{(2)}\quad
\nu_{1}' &= 0;\quad \nu_{2}' = 0;\quad \nu_{3}' = -1;\quad \nu_{4}' = 2.
\end{align*}
There will be equilibrium for each of these, if, by~\Eq{(204)},
\[
\text{(1)}\quad
c_{1}^{\nu_{1}} c_{2}^{\nu_{2}} c_{3}^{\nu_{3}} c_{4}^{\nu_{4}}
  = \frac{c_{2}c_{3}}{c_{1}^{2}}
  = \frac{n_{2}n_{3}}{n_{1}^{2}}
  = ae^{-\efrac{b}{\theta}},
\]
\PageSep{222}
and
\[
\text{(2)}\quad
c_{1}^{\nu_{1}'} c_{2}^{\nu_{2}'} c_{3}^{\nu_{3}'} c_{4}^{\nu_{4}'}
  = \frac{c_{4}^{2}}{c_{3}}
  = \frac{n_{4}^{2}}{n_{3} (n_{1} + n_{2} + n_{3} + n_{4})}
  = a' e^{-\efrac{b'}{\theta}} · \frac{\theta}{p}.
\]
The constants $a$,~$b$, $a'$,~$b'$ have the values calculated above.
The total number of hydrogen atoms $(n_{1} + 2n_{2})$ and of
iodine atoms $(n_{1} + 2n_{3} + n_{4})$ being known, we have four
equations for the complete determination of the four
quantities $n_{1}$, $n_{2}$, $n_{3}$,~$n_{4}$.

\Section{248.} The general equation of equilibrium~\Eq{(204)} also
\index{Dilute solutions!thermodynamical theory of}%
\index{Thermodynamical theory!of dilute solutions}%
shows that at finite temperatures and pressures none of the
concentrations,~$c$, can ever vanish; in other words, that the
dissociation can never be complete, nor can it completely
vanish. There is always present a finite, though perhaps
a very small number of all possible kinds of molecules.
Thus, in water vapour at any temperature at least a trace
of oxygen and hydrogen must be present (see also \SecRef{259}).
In a great number of phenomena, however, these quantities
are too small to be of any importance.
\index{Gaseous system|)}%
\index{System!gaseous|)}%
\PageSep{223}


\Chapter{V.}{Dilute Solutions.}
\index{Dilute solutions|(}%
\index{Solutions, dilute|(}%

\Section{249.} To determine~$\Psi$ as a function of the temperature~$\theta$,
the pressure~$p$, and the number~$n$ of the different kinds of
molecules in a system of any number of constituents and
any number of phases, we may use the method of the preceding
chapter. It is necessary first to find by suitable
measurements the volume~$V$, and the internal energy~$U$ of
each single phase, and then calculate the entropy~$\Phi$ from
the definition~\Eq{(60)}. A simple summation extending over
all the phases gives, by~\Eq{(75)}, the function~$\Psi$ for the whole
system. On account of incomplete experimental data, however,
the calculation of~$\Psi$ can be performed, besides for a
gaseous phase, only for a \emph{dilute solution}, \ie\ for a phase in
which one kind of molecule far outnumbers all the others in
the phase. We shall in future call this kind of molecule
the \emph{solvent}, the other kinds the \emph{dissolved substances}. This
differs from the definition of~\SecRef{220}. If $n_{0}$~be the number
of molecules of the solvent, $n_{1}$,~$n_{2}$, $n_{3}$,~\dots the number of
molecules of the dissolved substances, then the solution
may be considered dilute if $n_{0}$~be large in comparison with
each of the numbers $n_{1}$,~$n_{2}$, $n_{3}$\Add{,}~\dots. The state of aggregation
of the substance is of no importance, it may be solid,
liquid, or gaseous.

\Section{250.} We shall now determine by the above method
the energy~$U$ and the volume~$V$ of a dilute solution. The
important simplification, to which this definition of a dilute
solution leads, rests on the mathematical theorem, that a
finite, continuous, and differentiable function of several
\PageSep{224}
variables, which have very small values, is necessarily a
\emph{linear} function of these variables. This determines $U$~and
$V$ as functions of $n_{0}$,~$n_{1}$, $n_{2}$,~\dots\Add{.} Physically speaking, this
means that the properties of a dilute solution, besides
depending on the interactions between the molecules of the
solvent, necessarily depend only on the interactions between
the molecules of the solvent and the molecules of the dissolved
substances, but not on the interactions of the dissolved
substances among themselves, for these are small quantities
of a higher order.

\Section{251.} The quotient~$\dfrac{U}{n_{0}}$, \ie\ the internal energy divided
\index{Energy!of dilute solution}%
by the number of molecules of the solvent, remains unchanged
if the numbers, $n_{0}$,~$n_{1}$, $n_{2}$\Add{,}~\dots be varied in the same
proportion; for, by \SecRef{201}, $U$~is a homogeneous function of
the number of molecules $n_{0}$,~$n_{1}$, $n_{2}$,~\dots, of the first degree.
$\dfrac{U}{n_{0}}$~is, therefore, a function of the ratios $\dfrac{n_{1}}{n_{0}}$, $\dfrac{n_{2}}{n_{0}}$,~\dots, and also a
linear function, since these ratios are small, and the function
is supposed to be differentiable. The function is, therefore,
of the form
\[
\frac{U}{n_{0}} = u_{0} + u_{1}\, \frac{n_{1}}{n_{0}} + u_{2}\, \frac{n_{2}}{n_{0}} + \dots\Add{,}
\]
where, $u_{0}$, $u_{1}$, $u_{2}$ are quantities depending, not on the number
of molecules, but only on the temperature~$\theta$, the pressure~$p$,
and the nature of the molecules. In fact, $u_{0}$~depends only
on the nature of the solvent, since the energy reduces to
$n_{0} u_{0}$, when $n_{1} = 0 = n_{2} = \dots$, and $u_{1}$~only on the nature of
the first dissolved substance and the solvent, and so on.
$u_{0}$,~therefore, corresponds to the interactions between the
molecules of the solvent, $u_{1}$~to those between the solvent
and the first dissolved substance, and so on. This contains
a refutation of an objection, which is often raised against the
modern theory of dilute solutions, that it treats dilute solutions
\index{Dilute solutions!energy of}%
simply as gases, and takes no account of the influence
of the solvent.
\PageSep{225}

\Section{252.} If the dilution is not sufficient to warrant the use
\index{Dilute solutions!volume of}%
\index{Entropy!of dilute solution}%
of this very simple form of the function~$U$, a more accurate
relation may be obtained by expanding Taylor's series still
further,
\[
\frac{U}{n_{0}}
  = u_{0} + u_{1}\, \frac{n_{1}}{n_{0}} + \dots
  + u_{11} \left(\frac{n_{1}}{n_{0}}\right)^{2}
  + 2u_{12}\, \frac{n_{1}}{n_{0}} · \frac{n_{2}}{n_{0}}
  + u_{22} \left(\frac{n_{2}}{n_{0}}\right)^{2}
  + \dots\Add{.}
\]
The coefficients $u_{11}$,~$u_{12}$, $u_{22}$,~\dots refer to the influence of the
interactions of the dissolved molecules with one another.
This, in fact, is the only practicable way of obtaining a
rational thermodynamical theory of solutions of any concentration.

\Section{253.} We shall here keep to the simple form, and write
\[
\left.
\begin{aligned}
U &= n_{0} u_{0} + n_{1} u_{1} + n_{2} u_{2} + \dots\Add{,} \\
\llap{\text{and\qquad\qquad}}
V &= n_{0} v_{0} + n_{1} v_{1} + n_{2} v_{2} + \dots\Add{.}
\end{aligned}
\right\}
\Tag{(209)}
\]
How far these equations correspond to the facts may be
determined by the inferences to which they lead. If we
dilute the solution still further by adding one molecule of
the solvent in the same state of aggregation as the solution,
keeping meanwhile the temperature~$\theta$ and the pressure~$p$
constant, the corresponding change of volume and the heat
effect may be calculated from the above equations. One
molecule of the pure solvent, at the same temperature and
pressure, has the volume~$v_{0}$ and the energy~$u_{0}$. After dilution,
the volume of the solution becomes
\[
V' = (n_{0} + 1) v_{0} + n_{1} v_{1} + n_{2} v_{2} + \dots
\]
and the energy
\[
U' = (n_{0} + 1) u_{0} + n_{1} u_{1} + n_{2} u_{2} + \dots\Add{.}
\]
The increase of volume brought about by the dilution is
therefore
\[
V' - (V + v_{0}),
\]
\ie\ the increase of volume is zero. The heat absorbed is, by
the first law~\Eq{(47)},
\[
U' - (U + u_{0}) + p \bigl\{V' - (V + v_{0})\bigr\}.
\]
\PageSep{226}
This also vanishes. These inferences presuppose that the
number of molecules of the dissolved substances remain
unchanged, \ie\ that no chemical changes (\eg\ changes of
the degree of dissociation) are produced by the dilution. If
such were the case, the number of molecules of the dissolved
substances would have values in the equations for $U'$~and~$V'$
different from those in the equations for $U$~and~$V$, and therefore
would not disappear on subtraction. We may therefore
enunciate the following proposition: \emph{Further dilution of a
dilute solution, if no chemical changes accompany the process,
produces neither an appreciable change of volume nor an
appreciable heat effect}; or, in other words, \emph{any change of
volume or any heat effect produced by further dilution of a
dilute solution is due to chemical transformations among the
molecules of the dissolved substances}.

\Section{254.} We now turn to the calculation of the entropy~$\Phi$
\index{Dilute solutions!entropy of}%
of a dilute solution. If the number of molecules $n_{0}$,~$n_{1}$,
$n_{2}$,~\dots be constant, we have, by~\Eq{(60)},
\[
d\Phi = \frac{dU + p\, dV}{\theta},
\]
and, by~\Eq{(209)},
\[
d\Phi = n_{0}\, \frac{du_{0} + p\, dv_{0}}{\theta}
  + n_{1}\, \frac{du_{1} + p\, dv_{1}}{\theta}
  + n_{2}\, \frac{du_{2} + p\, dv_{2}}{\theta} + \dots\Add{.}
\]
Since $u$~and $v$ are functions of $\theta$~and $p$~only, and not of~$n$,
each of the coefficients of $n_{0}$,~$n_{1}$, $n_{2}$,~\dots, must be a perfect
differential, \ie\ there must be certain functions~$\phi$, depending
only on $\theta$~and~$p$, such that
\[
\left.
\begin{aligned}
d\phi_{0} &= \frac{du_{0} + p\, dv_{0}}{\theta}\Add{,} \\
d\phi_{1} &= \frac{du_{1} + p\, dv_{1}}{\theta}\Add{,} \\
d\phi_{2} &= \frac{du_{2} + p\, dv_{2}}{\theta}\Add{.}
\end{aligned}
\right\}
\Tag{(210)}
\]
\PageSep{227}
We have, then,
\[
\Phi = n_{0} \phi_{0} + n_{1} \phi_{1} + n_{2} \phi_{2} + \dots + C,
\Tag{(211)}
\]
where the integration constant~$C$ cannot depend on $\theta$~and~$p$,
but may be a function of the number of molecules. $C$~may
be determined as a function of $n_{0}$,~$n_{1}$, $n_{2}$,~\dots for a particular
temperature and pressure, and this will be the general expression
for~$C$ at any temperature and pressure.

We shall now determine~$C$ as a function of~$n$ taking
the particular case of high temperature and small pressure.
By increasing the temperature and diminishing the
pressure, the solution, whatever may have been its original
state of aggregation, will pass completely into the gaseous
state. Chemical changes will certainly take place at the
same time, \ie\ the number of molecules~$n$ will change. But
we shall assume that the process takes place in such a way
as to leave the number of the different kinds of molecules
unaltered, because $C$~remains constant only in this case.
Only an ideal process can accomplish this, since it passes
through unstable states. There is, however, no objection to
its use for our present purpose, since the above expression
for~$\Phi$ holds not only for stable states of equilibrium, but for
all states characterized by quite arbitrary values of $\theta$,~$p$, $n_{0}$,~$n_{1}$,
$n_{2}$,~\dots\Add{.} Stable equilibrium is a special case, satisfying
\index{Cantor@Cantor, Hr.|indexnote}%
\Typo{\index{Planck}}{\index{Planck@Planck|indexnote}}%
a further condition to be established below.\footnote
  {Hr.~Cantor maintains (Ann.\ d.\ Phys., \textbf{10}, p.~205, 1903) that it is not permissible
  to suppose that the gaseous state may be reached in this way. ``It
  must be proved that this represents a possible state of the substance, that it
  may be at least a momentary state. But no theoretical proof of this has
  been advanced, and direct experience does not at all justify such an assumption.''
  In reply, it has first of all to be pointed out that the possibility of
  varying the temperature and the pressure, keeping the number of molecules
  constant, merely depends on the fact that the number of molecules together
  with the temperature and the pressure form the independent variables which
  are necessary for the unique determination of the state of solution under
  consideration. The variables are not subject to any limitations, except that
  the number of atoms must remain on the whole unchanged. This does not
  concern us here, and is chemically self-evident. Therefore, from a general
  thermodynamical point of view, nothing stands in the way of letting the
  pressure diminish and the temperature rise in any way, keeping the number
  of molecules constant, if the formation of a new phase is prevented. When
  this is recognized, it requires only the hypothesis that by continuing this
  process the ideal gaseous state is finally reached---a supposition which scarcely
  any one can object to, and which Hr.~Cantor does not, at least directly, contradict
  (Planck, Ann.\ d.\ Phys.\ \textbf{10}, p.~436, 1903).---\Tr.}
\PageSep{228}

At a sufficiently high temperature, and a sufficiently
low pressure, any gaseous system possesses so small a
density, that it may be regarded as a mixture of perfect
gases (\SecRef{21}, and \SecRef{43}). We have, therefore, by~\Eq{(194)}, bearing
in mind that here the first kind of molecule is denoted
by the suffix~$0$,
\begin{align*}
\Phi &= n_{0} \left(c_{v_{0}} \log \theta + R \log \frac{\theta}{p}\right) \\
     &+ n_{1} \left(c_{v_{1}} \log \theta + R \log \frac{\theta}{p}\right)
  + \dots + C.
\Tag{(212)}
\end{align*}
The constant~$C$ is independent of $\theta$~and~$p$, and has the
value given in~\Eq{(198)}. On comparing this with~\Eq{(211)}, it is
seen that the expression for~$\Phi$ can pass from~\Eq{(211)} into~\Eq{(212)}
by mere change of temperature and pressure, only if the
constant~$C$ is the same in both expressions, \ie\ if, by~\Eq{(198)},
\[
C = n_{0} (k_{0} - R \log c_{0}) + n_{1} (k_{1} - R \log c_{1}) + \dots\Add{.}
\]
Here $k_{0}$,~$k_{1}$, $k_{2}$,~\dots are constants, and the concentrations are
\[
c_{0} = \frac{n_{0}}{n_{0} + n_{1} + n_{2} + \dots};\quad
c_{1} = \frac{n_{1}}{n_{0} + n_{1} + n_{2} + \dots}.
\]
By~\Eq{(211)}, the entropy of a dilute solution becomes
\[
\Phi = n_{0} (\phi_{0} + k_{0} - R \log c_{0})
     + n_{1} (\phi_{1} + k_{1} - R \log c_{1}) + \dots\Add{.}
\Tag{(213)}
\]
If we put, for shortness, the quantities which depend only
on $\theta$~and~$p$,
\[
\left.
\begin{aligned}
\phi_{0} + k_{0} - \frac{u_{0} + pv_{0}}{\theta} &= \varphi_{0}\Add{,} \\
\phi_{1} + k_{1} - \frac{u_{1} + pv_{1}}{\theta} &= \varphi_{1}\Add{,} \\
\phi_{2} + k_{2} - \frac{u_{2} + pv_{2}}{\theta} &= \varphi_{2}\Add{,} \\
\end{aligned}
\right\}
\Tag{(214)}
\]
\PageSep{229}
we have, finally, from \Eq{(75)}, \Eq{(213)} and~\Eq{(209)},
\begin{align*}
\Psi = n_{0} (\varphi_{0} - R \log c_{0})
     &+ n_{1} (\varphi_{1} - R \log c_{1}) \\
     &+ n_{2} (\varphi_{2} - R \log c_{2})
      + \dots\Add{.}
\Tag{(215)}
\end{align*}
This equation determines the thermodynamical properties of
a dilute solution.

\Section{255.} We may now proceed to establish the conditions
of equilibrium of a system consisting of several phases. As
hitherto, the different kinds of molecules in the phase will
be denoted by suffixes, and the different phases by dashes.
For the sake of simplicity the first phase will be left without
a dash. The entire system is then represented by
\begin{gather*}
n_{0}\ m_{0}, n_{1}\ m_{1}, n_{2}\ m_{2}, \dots \mid
n_{0}'\ m_{0}', n_{1}'\ m_{1}', n_{2}'\ m_{2}', \dots \\
\mid n_{0}''\ m_{0}'', n_{1}''\ m_{1}'', n_{2}''\ m_{2}'', \dots \mid
\Tag{(216)}
\end{gather*}
The number of molecules \Erratum{are}{is} denoted by~$n$, and the
molecular weights by~$m$, and the individual phases are
separated by vertical lines. In the general formula we
signify the summation over the different kinds of molecules
of one and the same phase by writing the individual terms
of the summation; the summation over the different phases,
on the other hand, by the symbol~$\tsum$.

In order to enable us to apply the derived formulæ, we
shall assume that each phase is either a mixture of perfect
gases or a dilute solution. The latter designation will be
applied to phases containing only one kind of molecule, \eg\
a chemically homogeneous solid precipitate from an aqueous
solution. One kind of molecule represents the special case
of a dilute solution in which the concentrations of all the
dissolved substances are zero.

\Section{256.} Suppose now that an isothermal isopiestic change
be possible, corresponding to a simultaneous variation
$\delta n_{0}$,~$\delta n_{1}$, $\delta n_{2}$,~\dots\Add{,} $\delta n_{0}'$,~$\delta n_{1}'$, $\delta n_{2}'$,~\dots... of the number of molecules
$n_{0}$,~$n_{1}$, $n_{2}$,~\dots\Add{,} $n_{0}'$,~$n_{1}'$, $n_{2}'$,~\dots; then, by~\Eq{(79)}, this change
\PageSep{230}
will not take place, if at constant temperature and pressure
\[
\delta \Psi = 0
\]
or, by~\Eq{(215)}, if
\begin{multline*}
\tsum (\varphi_{0} - R \log c_{0})\, \delta n_{0}
   + (\varphi_{1} - R \log c_{1})\, \delta n_{1}
   + (\varphi_{2} - R \log c_{2})\, \delta n_{2} \\
\begin{aligned}
  &+ \dots
   + \tsum n_{0}\, \delta(\varphi_{0} - R \log c_{0})
   + n_{1}\, \delta(\varphi_{1} - R \log c_{1}) \\
  &+ n_{2}\, \delta(\varphi_{2} - R \log c_{2})
   + \dots = 0\Add{.}
\end{aligned}
\end{multline*}
The summation~$\tsum$ extends over all the phases of the
system. The second series is identically equal to zero for
the same reason as was given in connection with equation~\Eq{(200)}.
If we again introduce the simple integral ratio
\begin{multline*}
\delta n_{0} : \delta n_{1} : \delta n_{2} : \dots :
\delta n_{0}' : \delta n_{1}' : \delta n_{2}' : \dots \\
  = \nu_{0} : \nu_{1} : \nu_{2} : \dots : \nu_{0}' : \nu_{1}' : \nu_{2}' : \dots
\Tag{(217)}
\end{multline*}
then the equation of equilibrium becomes
% [** TN: Re-breaking]
%\begin{multline*}
\[
\tsum (\varphi_{0} - R \log c_{0})\, \nu_{0}
   \Erratum{-}{+} (\varphi_{1} - R \log c_{1})\, \nu_{1} \\
   + (\varphi_{2} - R \log c_{2})\, \nu_{2} + \dots = 0
\]
%\end{multline*}
or
\begin{align*}
\tsum \nu_{0} \log c_{0} + \nu_{1} \log c_{1} + \nu_{2} \log c_{2} + \dots
  &= \frac{1}{R} \tsum \nu_{0} \varphi_{0} + \nu_{1} \varphi_{1} + \dots \\
  &= \log K.
\Tag{(218)}
\end{align*}
$K$~like $\varphi_{0}$, $\varphi_{1}$, $\varphi_{2}$, is independent of the number of molecules~$n$.

\Section{257.} The definition of~$K$ gives its functional relation
to $\theta$~and~$p$.
\begin{align*}
\frac{\dd \log K}{\dd \theta}
  &= \frac{1}{R} \tsum \nu_{0}\, \frac{\dd \varphi_{0}}{\dd \theta} + \nu_{1}\, \frac{\dd \varphi_{1}}{\dd \theta} +\nu_{2}\, \frac{\dd \varphi_{2}}{\dd \theta} + \dots\Add{,} \\
\frac{\dd \log K}{\dd p}
  &= \frac{1}{R} \tsum \nu_{0}\, \frac{\dd \varphi_{0}}{\dd p} + \nu_{1}\, \frac{\dd \varphi_{1}}{\dd p} +\nu_{2}\, \frac{\dd \varphi_{2}}{\dd p} + \dots\Add{.}
\end{align*}
\PageSep{231}
Now, by~\Eq{(214)}, we have for an infinitely small change
of $\theta$~and~$p$
\[
d\varphi_{0} = d\phi_{0} - \frac{du_{0} + p\, dv_{0} + v_{0}\, dp}{\theta} + \frac{u_{0} + pv_{0}}{\theta^{2}}\, d\theta
\]
and therefore, by~\Eq{(210)},
\[
d\varphi_{0} = \frac{u_{0} + pv_{0}}{\theta^{2}}\, d\theta - \frac{v_{0}\, dp}{\theta}.
\]
From this it follows that
\[
\frac{\dd \varphi_{0}}{\dd \theta} = \frac{u_{0} + pv_{0}}{\theta^{2}};\quad
\frac{\dd \varphi_{0}}{\dd p} = -\frac{v_{0}}{\theta}.
\]
Similarly
\[
\frac{\dd \varphi_{1}}{\dd \theta} = \frac{u_{1} + pv_{1}}{\theta^{2}};\quad
\frac{\dd \varphi_{1}}{\dd p} = -\frac{v_{1}}{\theta}.
\]
Hence
\begin{align*}
\frac{\dd \log K}{\dd \theta}
  &= \frac{1}{R\theta^{2}} \tsum (\nu_{0} u_{0} + \nu_{1} u_{1} + \dots) + p(\nu_{0} v_{0} + \nu_{1} v_{1} + \dots), \\
\frac{\dd \log K}{\dd p}
  &= -\frac{1}{R\theta} \tsum \nu_{0} v_{0} + \nu_{1} v_{1} + \dots\Add{.}
\end{align*}
Denoting by~$s$ the increase of volume of the system, and
by~$L$ the heat absorbed, when the change corresponding to~\Eq{(217)}
takes place at constant temperature and pressure,
then, by~\Eq{(209)},
\[
s = \tsum \nu_{0} v_{0} + \nu_{1} v_{1} + \nu_{2} v_{2} + \dots
\]
and, by the first law of thermodynamics,
\[
L = \tsum (\nu_{0} u_{0} + \nu_{1} u_{1} + \dots) + p(\nu_{0} v_{0} + \nu_{1} v_{1} + \dots);
\]
therefore
\[
\frac{\dd \log K}{\dd \theta} = \frac{L}{R\theta^{2}}
\Tag{(219)}
\]
\PageSep{232}
and
\[
\frac{\dd \log K}{\dd p} = -\frac{s}{R\theta}\Add{.}
\Tag{(220)}
\]
The influence of the temperature on~$K$, and therewith
on the condition of equilibrium towards a certain chemical
reaction, is controlled by the heat effect of that reaction,
and the influence of the pressure is controlled by the corresponding
change of volume of the system. If the reaction
take place without the absorption or evolution of heat, the
temperature has no influence on the equilibrium. If it
produce no change of volume the pressure has no influence.
The former equations \Eq{(205)} and \Eq{(206)} are particular cases
of \Eq{(219)} and \Eq{(220)}, as may be seen by substituting for
$\log K$ the special value obtained from \Eq{(218)} and~\Eq{(204)}:
\[
\log K = \log a - \frac{b}{\theta} + (\nu_{1} + \nu_{2} + \dots) \log \frac{\theta}{p}.
\]

\Section{258.} By means of equation~\Eq{(218)} a condition of equilibrium
may be established for each possible change in a
given system subject to chemical change. Of course, $K$~will
have a different value in each case. This corresponds
to the requirements of Gibbs's phase rule, which is general
\index{Gibbs}%
\index{Gibbs's phase rule}%
in its application (\SecRef{204}). The number of the different
kinds of molecules in the system must be distinguished
from the number of the independent constituents (\SecRef{198}).
Only the latter determines the number and nature of the
phases; while the number of the different kinds of molecules
plays no part whatever in the application of the phase rule.
If another kind of molecule be introduced the number of
the variables increases, to be sure, but so does the number
of the possible reactions, and therewith, the number of the
conditions of equilibrium by the same amount, so that the
number of independent variables is quite independent
thereof.

\Section{259.} \Erratum{Equations}{Equation}~\Eq{(218)} shows further that, generally
speaking, all kinds of molecules possible in the system
\PageSep{233}
will be present in finite numbers in every phase; for instance,
molecules of~\ce{H2O} must occur in any precipitate from an
aqueous solution. Even solid bodies in contact must
partially dissolve in one another, if sufficient time be given.
The quantity~$K$, which determines the equilibrium, possesses,
according to the definition~\Eq{(218)}, a definite, in general, a finite
value for each possible chemical change, and none of the
concentrations~$c$ can, therefore, vanish so long as the
temperature and the pressure remain finite. This principle,
based entirely on thermodynamical considerations, has
already served to explain certain facts, \eg\ the impossibility
of removing the last traces of impurity from gases, liquids,
and even solids. It also follows from it that absolutely semipermeable
membranes are non-existent, for the substance
of any membrane would, in time, become saturated with
the molecules of all the various kinds of substances in
contact with one side of it, and thus give up each kind of
substance to the other side.

On the other hand, this view greatly complicates the
calculation of the thermodynamical properties of a solution,
since, in order to make no mistake, it is necessary to assume
from the start the existence in every phase of all kinds of
molecules possible from the given constituents. We must
not neglect any kind of molecule until we have ascertained
by a particular experiment that its quantity is inappreciable.
Many cases of apparent discrepancy between theory and
experiment may probably be explained in this way.

We shall now discuss some of the most important particular
cases. They have been arranged, in the first place, according
to the number of the independent constituents of the
system; in the second, according to the number of the
phases.

\Section{260.} \Topic{One Independent Constituent in One Phase.}---According
to the phase rule, the nature of the phase depends
on two variables, \eg\ on the temperature and the pressure.
The phase may contain any number of different kinds of
molecules. Water, for instance, will contain simple, double,
\PageSep{234}
and multiple \ce{H2O}-molecules; molecules of hydrogen and
oxygen, \ce{H2} and~\ce{O2}; electrically charged ions \ce{H+},~\ce{HO-} and~\ce{O^{-\,-}},
etc., in finite quantities. The electrical charges of the
ions do not play any important part in thermodynamics, so
long as there is no direct conflict between the electrical
and the thermodynamical forces. This happens when and
only when the thermodynamical conditions of equilibrium
call for such a distribution of the ions in the different phases
of the system as would lead, on account of the constant
charges of the ions, to free electricity in any phase. The
electrical forces strongly oppose such a distribution, and the
resulting deviation from the pure thermodynamical equilibrium
is, however, compensated by differences of potential
between the phases. A general view of these electromolecular
phenomena may be got by generalizing the expressions for
the entropy and the energy of the system by the addition
of electrical terms. We shall restrict our discussion to
states which do not involve electrical phenomena, and need
not consider the charges of the ions, which we may treat
like other molecules.

In the case mentioned above, then, the concentrations of
all kinds of molecules are determined by $\theta$~and~$p$. The
calculation of the concentrations has succeeded so far only
in the case of the \ce{H+}~and \ce{OH-} ions (the number of the
\ce{O^{-\,-}}~ions is negligible), in fact, among other methods, by the
measurement of the electrical conductivity of the solution,
which depends only on the ions. Kohlrausch and Heydweiller
\index{Heydweiller}%
\index{Kohlrausch}%
found the degree of dissociation of water, \ie\ the ratio of
\index{Dissociation!of water}%
\index{Water, dissociation of}%
the mass of water split into \ce{H+}~and \ce{OH-} ions to the total
mass of water to be, at $18°$~C.,
\[
14.3 × 10^{-10}.
\]
This number represents the ratio of the number of dissociated
molecules to the total number of molecules. We
may determine by thermodynamics the change of the dissociation
with temperature.
\PageSep{235}

The condition of equilibrium will now be established.
The system is, by~\Eq{(216)},
\[
\ce{$n_{2}$ H2O};\quad
\ce{$n_{1}$ H+};\quad
\ce{$n_{2}$ OH-}.
\]
Let the total number of molecules be
\[
n = n_{0} + n_{1} + n_{2},
\]
the concentrations are, therefore,
\[
c_{0} = \frac{n_{0}}{n};\quad
c_{1} = \frac{n_{1}}{n};\quad
c_{2} = \frac{n_{2}}{n}.
\]
The chemical reaction in question,
\[
\nu_{0} : \nu_{1} : \nu_{2} = \delta n_{0} : \delta n_{1} : \delta n_{2},
\]
consists in the dissociation of one \ce{H2O}~molecule into \ce{H+} and
\ce{OH-}.
\[
\nu_{0} = 1;\quad
\nu_{1} = 1;\quad
\nu_{2} = 1;
\]
and therefore, by~\Eq{(218)}, in the state of equilibrium
\[
-\log c_{0} + \log c_{1} + \log c_{2} = K,
\]
or, since $c_{1} = c_{2}$, and $c_{0} = 1$ nearly,
\[
2 \log c_{1} = \log K.
\]
This gives, by~\Eq{(219)}, the relation between the concentration
and the temperature:
\[
2\, \frac{\dd \log c_{1}}{\dd \theta} = \frac{1}{R} · \frac{L}{\theta^{2}}.
\Tag{(221)}
\]
According to Arrhenius, $L$,~the heat necessary for the dissociation
\index{Arrhenius}%
of one molecule of~\ce{H2O} into \ce{H+}~and \ce{OH-}, is equal to
the heat of neutralization of a strong monobasic acid and
base in dilute aqueous solution. J.~Thomsen's experiments
\index{Thomsen, J.}%
give for mean temperatures:
\[
L = \frac{4045000}{\theta}~\Unit{cal.}
\]
\PageSep{236}
On reducing calories to C.G.S. units, we get
\[
\frac{\dd \log c_{1}}{\dd \theta} = \frac{1}{2 × 1.971} × \frac{4045000}{\theta^{3}}.
\]
On integrating, we have
\begin{align*}
\log c_{1}
  &= -\frac{4045000}{7.884} · \frac{1}{\theta^{2}}
   = -\frac{513000}{\theta^{2}} + \const \\
c_{1} &= C e^{-\efrac{513000}{\theta^{2}}}.
\end{align*}
The value of the constant~$C$ is found from the degree of
dissociation at $18°$~C. ($\theta = 291$);
\begin{align*}
c_{l} = c_{2} &= 14.3 × 10^{-10}\Add{,} \\
\therefore
C &= 6.1 × 10^{-7}\Add{.}
\end{align*}
Hence the degree of dissociation for any temperature is,
\[
c_{1} = 6.1 e^{-\efrac{513000}{\theta^{2}}} × 10^{-7}.
\]
This agrees well with the electrical conductivity of pure
\index{Conductivity of water, electrical}%
\index{Electrical conductivity of water}%
water when measured at different temperatures. Only at
the absolute zero of temperature does the dissociation, and
with it the conductivity, vanish. On the other hand, it
does not increase indefinitely with temperature, but reaches
a maximum value~$C$.

\Section{261.} \Topic{One Independent Constituent in Two or
Three Phases.}---The main features of these cases have
already been discussed in Chapter~II., \SSecRef{205} to~\SecNum{207}, and~\SecRef{213}.

\Section{262.} \Topic{Two Independent Constituents in One Phase\Add{.}}\\
---(A substance dissolved in a homogeneous solvent).
According to the phase rule, one other variable besides the
pressure and the temperature is arbitrary, \eg\ the number
of the molecules dissolved in $1$~litre of the solution, a
quantity which may be directly measured. The values of
\PageSep{237}
these three variables determine the concentrations of all
kinds of molecules, whether they have their origin in dissociation,
association, formation of hydrates, or hydrolysis
of the dissolved molecules. Let us consider the simple
case of a binary electrolyte, \eg\ acetic acid in water. The
\index{Acetic acid}%
\index{Binary electrolyte}%
\index{Electrolyte, binary}%
system is represented by
\[
\ce{$n_{0}$ H2O},\quad
\ce{$n_{1}$ CH3 . COOH},\quad
\ce{$n_{2}$ H+},\quad
\ce{$n_{3}$ CH3- . COO}.
\]
The total number of molecules,
\[
n = n_{0} + n_{1} + n_{2} + n_{3},
\]
is only slightly greater than~$n_{0}$. The concentrations are
\[
c_{0} = \frac{n_{0}}{n};\quad
c_{1} = \frac{n_{1}}{n};\quad
c_{2} = \frac{n_{2}}{n};\quad
c_{3} = \frac{n_{3}}{n}.
\]
The reaction to be considered is represented by
\[
\nu_{0} : \nu_{1} : \nu_{2} : \nu_{3}
  = \delta n_{0} : \delta n_{1} : \delta n_{2} : \delta n_{3},
\]
and consists in the dissociation of one molecule of \ce{CH3 . COOH}
into its two ions.
\[
\nu_{0} = 0;\quad
\nu_{1} = -1;\quad
\nu_{2} = 1;\quad
\nu_{3} = 1.
\]
Therefore, in equilibrium,
\[
-\log c_{1} + \log c_{2} + \log c_{3} = \log K;
\]
or, since $c_{2} = c_{3}$,
\[
\frac{c_{2}^{2}}{c_{1}} = K\Add{.}
\Tag{(222)}
\]
Now, we may regard the sum
\[
c_{1} + c_{2} = c
\]
as known, since the total number ($n_{1} + n_{2}$) of the undissociated
and the dissociated molecules of the acid, and the total
number of water molecules, which may be put $= n$, are
\PageSep{238}
measured directly. Hence $c_{1}$~and $c_{2}$ may be calculated from
the last two equations.
\begin{align*}
\frac{c_{1}}{c} &= \frac{n_{1}}{n_{1} + n_{2}}
  = 1 - \frac{K}{2c} \left(\sqrt{1 + \frac{4c}{K}} - 1\right); \\
\frac{c_{2}}{c} &= \frac{n_{2}}{n_{1} + n_{2}}
  = \frac{K}{2c} \left(\sqrt{1 + \frac{4c}{K}} - 1\right).
\end{align*}
With increasing dilution (decreasing~$c$), the ratio~$\dfrac{c_{2}}{c}$ increases
\index{Dilution!law of, of binary electrolytes}%
in a definite manner approaching the value~$1$, \ie\ complete
dissociation. This also gives for the electrical conductivity
\index{Dissociation!of \ce{H2SO4}}%
\index{Laws:!Ostwald's}%
\index{Ostwald's law}%
\index{Sulphuric acid, dissociation of}%
of a solution of given concentration Ostwald's so-called \emph{law
of dilution of binary electrolytes},\footnote
  {$K = \dfrac{\lambda_{v}}{\lambda_{\infty} (\lambda_{\infty} - \lambda_{v})^{v}}$,
  where $\lambda_{v}$~is the molecular conductivity at dilution~$v$; $\lambda_{\infty}$~the molecular conductivity
  at infinite dilution; and $v$~the molecular volume of the electrolyte.---\Tr.}
which has been experimentally
verified in numerous cases. In a manner quite
similar to that of~\SecRef{260}, the heat effect of the dissociation
shows how the degree of dissociation depends
on the temperature. Conversely, as was first shown by
Arrhenius, the heat of dissociation may be calculated from
\index{Arrhenius}%
the rate of change of the dissociation with temperature.

\Section{263.} Usually, however, in a solution, not one, but a
large number of reactions will be possible. Accordingly,
the complete system contains many kinds of molecules.
As another example, we shall discuss the case of an electrolyte
capable of splitting into ions in several ways, viz.\ an
aqueous solution of sulphuric acid. The system is represented
by
\[
\ce{$n_{0}$ H2O},\quad
\ce{$n_{1}$ H2SO4},\quad
\ce{$n_{2}$ H+},\quad
\ce{$n_{3}$ HSO4-},\quad
\ce{$n_{4}$ SO4^{-\,-}}.
\]
The total number of molecules is
\[
n = n_{0} + n_{1} + n_{2} + n_{3} + n_{4}\quad\text{(nearly equal to~$n_{0}$).}
\]
\PageSep{239}
The concentrations are
\[
c_{0} = \frac{n_{0}}{n};\quad
c_{1} = \frac{n_{1}}{n};\quad
c_{2} = \frac{n_{2}}{n};\quad
c_{3} = \frac{n_{2}}{n};\quad
c_{4} = \frac{n_{3}}{n}.
\]
Here two different kinds of reactions
\[
\nu_{0} : \nu_{1} : \nu_{2} : \nu_{3} : \nu_{4}
  = \delta n_{0} : \delta n_{1} : \delta n_{2} : \delta n_{3} : \delta n_{4}
\]
must be considered; first, the dissociation of one molecule
of \ce{H2SO4} into \ce{H+} and \ce{HSO4-}:
\[
\nu_{0} = 0;\quad
\nu_{1} = -1;\quad
\nu_{2} = 1;\quad
\nu_{3} = 1;\quad
\nu_{4} = 0;
\]
second, the dissociation of the ion \ce{HSO4-} into \ce{H+} and \ce{SO4^{-\,-}}\Add{:}
\[
\nu_{0} = 0;\quad
\nu_{1} = 0;\quad
\nu_{2} = 1;\quad
\nu_{3} = -1;\quad
\nu_{4} = 1.
\]
Hence, by~\Eq{(218)}, there are two conditions of equilibrium:
\begin{align*}
-\log c_{1} + \log c_{2} + \log c_{3} &= \log K \\
\intertext{and}
\log c_{2} - \log c_{3} + \log c_{4} &= \log K';
\end{align*}
or
\[
\frac{c_{2} c_{3}}{c_{1}} = K
\]
and
\[
\frac{c_{2} c_{4}}{c_{3}} = K'.
\]

This further condition must be added, viz.\ that the total
number of \ce{SO4} radicals ($n_{1} + n_{2} + n_{3}$) must be equal to half
the number of \ce{H}~atoms ($2n_{1} + n_{2} + n_{3}$); otherwise the
system would contain more than two independent constituents.
This condition is
\[
2c_{4} + c_{3} = c_{2}.
\]
Finally, the quantity of sulphuric acid in the solution is
supposed to be given:
\[
c_{1} + c_{3} + c_{4} = c.
\]
\PageSep{240}
The last four equations determine $c_{1}$,~$c_{2}$, $c_{3}$,~$c_{4}$, and hence the
state of equilibrium is found.

For a more accurate determination it would be necessary
to consider still other kinds of molecules. Every one of
these introduces a new variable, but also a new possible
reaction, and therefore a new condition of equilibrium, so
that the state of equilibrium remains uniquely determined.

\Section{264.} \Topic{Two Independent Constituents in Two Phases.}\\
---The state of equilibrium, by the phase rule, depends on
two variables, \eg\ temperature and pressure. The wide range
of cases in point makes a subdivision desirable, according as
only one phase contains both constituents in appreciable
quantity, or both phases contains both constituents.

Let us first take the simpler case, where one (first) phase
contains both constituents, and the other (second) phase
contains only one single constituent. Strictly speaking
this never occurs (by~\Chg{259}{\SecRef{259}}), but in many cases it is a sufficient
approximation to the actual facts. The application
of the general condition of equilibrium~\Eq{(218)} to this case
leads to different laws, according as the constituent in the
second phase plays the part of dissolved substance or solvent
(\SecRef{249}) in the first phase. We shall therefore divide this
case into two further subdivisions.

\Section{265.} \Topic{The Pure Substance in the Second Phase
forms the Dissolved Body in the First.}---An example of
this is the absorption of a gas, \eg\ carbon dioxide in a liquid
of comparatively small vapour pressure. The system is
represented by
\[
\ce{$n$ H2O},\quad
\ce{$n_{1}$ CO2} \mid
\ce{$n_{0}'$ CO2}.
\]
The concentrations of the different kinds of molecule of
the system in the two phases are
\[
c_{0} = \frac{n_{0}}{n_{0} + n_{1}};\quad
c_{1} = \frac{n_{1}}{n_{0} + n_{1}};\quad
c_{0}' = \frac{n_{0}'}{n_{0}'} = 1.
\]
\PageSep{241}
The reaction
\[
\nu_{0} : \nu_{1} : \nu_{0}' = \delta n_{0} : \delta n_{1} : \delta n_{0}'
\]
consists in the evaporation of one molecule of carbon dioxide
from the solution, therefore,
\[
\nu_{0} = 0,\quad
\nu_{1} = -1,\quad
\nu_{0}' = 1.
\]
The condition of equilibrium
\[
\nu_{0} \log c_{0} + \nu_{1} \log c_{1} + \nu_{0}' \log c_{0}' = \log K,
\]
is, therefore,
\[
-\log c_{1} = \log K,
\Tag{(223)}
\]
or, at a given temperature and pressure (for these determine~$K$),
$c_{1}$~the concentration of the gas in the solution is
determined. The change of concentration with pressure
and temperature is found by substituting \Eq{(223)} in \Eq{(219)}
and~\Eq{(220)}:
\begin{align*}
\frac{\dd \log c_{1}}{\dd p} &= \frac{1}{R} · \frac{s}{\theta}\Add{,}
\Tag{(224)} \\
\frac{\dd \log c_{1}}{\dd \theta} &= -\frac{1}{R} · \frac{L}{\Erratum{\theta_{2}}{\theta^{2}}}\Add{.}
\Tag{(225)}
\end{align*}
$s$~is the increase of volume of the system, $L$~the heat
absorbed during isothermal-isopiestic evaporation of one
gram molecule of~\ce{CO2}. Since $s$~represents nearly the
volume of one gram molecule of \Erratum{carbonic}{carbon} dioxide gas, we
may, by~\Eq{(16)}, put
\[
s = \frac{R\theta}{p},
\]
and equation~\Eq{(224)} gives
\[
\frac{\dd \log c_{1}}{\dd p} = \frac{1}{p}.
\]
\PageSep{242}
On integrating, we have
\[
\log c_{1} = \log p + \const
\]
or
\[
c_{1} = Cp
\Tag{(226)}
\]
\ie\ \emph{the concentration of the dissolved gas is proportional to the
pressure of the free gas on the solution} (Henry's law). The
\index{Henry's law}%
\index{Laws:!Henry's}%
factor~$C$, which is a measure of the solubility of the gas, still
depends on the temperature, since \Eq{(225)} and \Eq{(226)} give
\[
\frac{\dd \log C}{\dd \theta} = -\frac{1}{R} · \frac{L}{\Erratum{\theta_{2}}{\theta^{2}}}\Add{.}
\]
If, therefore, heat is absorbed during the evaporation of the
gas from the solution, $L$~is positive, and the solubility
decreases with increase of temperature. Conversely, from
the variation of~$C$ with temperature, the heat effect produced
by the absorption may be calculated;
\[
L = -\frac{R \theta^{2}}{C} · \frac{\dd C}{\dd \theta}.
\]
According to the experiments of Naccari and Pagliani,
\index{Naccari}%
\index{Pagliani}%
the solubility of carbon dioxide in water at~$20°$ ($\theta = 293$),
(expressed in a unit which need not be discussed here),
is~$0.8928$, its temperature coefficient $-0.02483$; therefore,
by~\Eq{(34)},
\[
L = \frac{1.971 × 293^{2} × 0.02483}{0.8928} = 4700~\Unit{cal.}
\]
Thomsen found the heat effect of the absorption of one gram
molecule of carbon dioxide to be $5880~\Unit{cal}$. The error
(according to Nernst) lies mainly in the determination of
\index{Nernst}%
the coefficient of solubility. Of the heat effect, the amount
\[
R\theta = 1.97 × 293 = 586~\Unit{cal.}
\]
corresponds, by~\Eq{(48)}, to external work.

\Section{266.} A further example is the saturation of a liquid
\PageSep{243}
with an almost insoluble salt; \eg\ succinic acid in water.
\index{Succinic acid}%
The system is represented by
\[
\ce{$n_{0}$ H2O},\quad
n_{1}\ \begin{array}{@{}l@{}}
\ce{CH2} - \ce{COOH} \\
| \\
\ce{CH2} - \ce{COOH}, \\
\end{array}
\quad\left|\quad
n_{0}'\ \begin{array}{@{}l@{}}
\ce{CH2} - \ce{COOH} \\
| \\
\ce{CH2} - \ce{COOH}, \\
\end{array}
\right.
\]
if the slight dissociation of the acid in water be neglected.
The calculation of the condition of equilibrium gives, as in~\SecRef{223},
\[
-\log c_{1} = \log K,
\]
$c_{1}$~is determined by temperature and pressure. Further, by~\Eq{(219)},
\[
L = -R\theta^{2} \frac{\dd \log c_{1}}{\dd \theta}\Add{.}
\Tag{(227)}
\]
Van't Hoff was the first to calculate~$L$ by means of this
equation from the solubility of succinic acid at $0°$~C. ($2.88$)
and at $8.5°$~C. ($4.22$)
\[
\frac{\dd \log c_{1}}{\dd \theta} = \frac{\log_{e} 4.22 - \log_{e} 2.88}{8.5}
  = 0.04494.
\]
This gives, for $\theta = 273$, $L = -1.971 × 273^{2} × 0.4494
= 6600~\Unit{cals.}$; \ie\ on the precipitation of one \Erratum{molecule}{gram molecule}
of the solid from the solution, $6600~\Unit{cals.}$ are given out.
Berthelot found the heat of solution to be $6700~\Unit{cals}$.
\index{Berthelot}%

If $L$~be regarded as independent of the temperature,
which is permissible in many cases as a first approximation,
the equation~\Eq{(227)} may be integrated with respect to~$\theta$,
giving
\[
\log c_{1} = \frac{L}{R\theta} + \const
\]

\Section{267.} The relation~\Eq{(227)} becomes inapplicable if the
salt in solution undergoes an appreciable chemical transformation,
\eg\ dissociation. For then, besides the ordinary
\PageSep{244}
molecules of the salt, the products of the dissociation are
present in the solution; for example, in the system of
water and silver acetate,
\index{Acetate!silver}%
\index{Silver!acetate}%
\index{Silver!nitrate}%
\[
\ce{$n_{0}$ H2O},\
\ce{$n_{1}$ CH3COOAg},\
\ce{$n_{2}$ Ag+},\
\ce{$n_{3}$ CH3- . COO} \mid
\ce{$n_{0}'$ CH3COOAg}.
\]
The total number of molecules in the solution:
\[
n = n_{0} + n_{1} + n_{2} + n_{3}\quad\text{(nearly $ = n_{0}$).}
\]
The concentrations of the different molecules in both phases
are
\[
c_{0} = \frac{n_{0}}{n};\quad
c_{1} = \frac{n_{1}}{n};\quad
c_{2} = \frac{n_{2}}{n};\quad
c_{3} = \frac{n_{3}}{n};\quad
c_{0}' = \frac{n_{0}'}{n_{0}'} = 1.
\]
The reactions,
\[
\nu_{0} : \nu_{1} : \nu_{2} : \nu_{3} : \nu_{0}' = \delta n_{0} : \delta n_{1} : \delta n_{2} : \delta n_{3} : \delta n_{0}',
\]
are:

(1) The precipitation of a molecule of the salt from the
solution:
\[
\nu_{0} = 0,\quad
\nu_{1} = -1,\quad
\nu_{2} = 0,\quad
\nu_{3} = 0,\quad
\nu_{0}' = 1.
\]

(2) The dissociation of a molecule of silver acetate:
\[
\nu_{0} = 0,\quad
\nu_{1} = -1,\quad
\nu_{2} = 1,\quad
\nu_{3} = 1,\quad
\nu_{0}' = 0.
\]

Accordingly, the two conditions of equilibrium are:

(1)\qquad $-\log c_{1} = \log K$\Add{,}

(2)\qquad $-\log c_{1} + \log c_{2} + \log c_{3} = \log K'$; \\
or, since $c_{2} = c_{3}$,
\[
\frac{c_{2}^{2}}{c_{1}} = K'.
\]
At given temperature and pressure, therefore, there is in
the saturated solution of a salt a definite number of undissociated
molecules; and the concentration~($c_{2}$) of the
dissociated molecules may be derived from that of the
\PageSep{245}
undissociated~($c_{1}$) by the law of dissociation of an electrolyte,
as given in~\Eq{(222)}.

Now, since by measuring the solubility the value of
$c_{1} + c_{2}$, and by measuring the electrical conductivity the
value of~$c_{2}$, may be found, the quantities $K$~and $K'$ can be
calculated for any temperature. Their dependence on temperature,
by~\Eq{(219)}, serves as a measure of the heat effect of the
precipitation of an undissociated molecule from the solution,
and of the dissociation of a dissolved molecule. Jahn has thus
\index{Jahn}%
given a method of calculating the actual heat of solution of
a salt, from measurements of the solubility of the salt and
of the conductivity of saturated solutions at different temperatures;
\ie\ the heat effect which takes place when one
gram molecule of the solid salt is dissolved, and the fraction
$\dfrac{c_{2}}{c_{1} + c_{2}}$ is dissociated into its ions, as is actually the case in
the process of solution.

\Section{268.} \Topic{The Pure Substance occurring in the Second
Phase forms the Solvent in the First Phase.}---This case
is realized when the pure solvent in any state of aggregation
is separated out from a solution of another state of aggregation,
\eg\ by freezing, evaporation, fusion, and sublimation.
The type of such a system is
\[
n_{0}\ m_{0},\quad
n_{1}\ m_{1},\quad
n_{2}\ m_{2},\quad
n_{3}\ m_{3},\ \dots \mid
n_{0}'\ m_{0}'.
\]
The question whether the solvent has the same molecular
weight in both phases, or not, is left open. The total
number of molecules in the solution is
\[
n = n_{0} + n_{1} + n_{2} + n_{3} + \dots\quad\text{(nearly $= n_{0}$).}
\]
The concentrations are
\[
c_{0} = \frac{n_{0}}{n};\quad
c_{1} = \frac{n_{1}}{n};\quad
c_{2} = \frac{n_{2}}{n};\ \dots
c_{0}' = \frac{n_{0}'}{n_{0}'} = 1.
\]
A possible transformation,
\[
\nu_{0} : \nu_{1} \Add{:} \dots : \nu_{0}' = \delta n_{0} : \delta n_{1} \Add{:} \dots : \delta n_{0}',
\]
\PageSep{246}
is the passage of a molecule of the solvent from the first
phase to the second phase, \ie\
\[
\nu_{0} = -1;\quad
\nu_{1} = 0;\quad
\nu_{2} = 0;\ \dots
\nu_{0}' = \frac{m_{0}}{m_{0}'}.
\Tag{(228)}
\]
Equilibrium demands, by~\Eq{(218)}, that
\[
-\log c_{0} + \frac{m_{0}}{m_{0}'} \log c_{0}' = \log K,
\]
and, therefore, on substituting the above values of $c_{0}$~and~$c_{0}'$,
\[
\log \frac{n}{n_{0}} = \log K.
\]
But
\[
\frac{n}{n_{0}} = 1 + \frac{n_{1} + n_{2} + n_{3} + \dots}{n_{0}},
\]
and, therefore, since the fraction on the right is very small,
\[
\frac{n_{1} + n_{2} + n_{3} + \dots}{n_{0}} = \log K.
\Tag{(229)}
\]
By the general definition~\Eq{(218)}, we have
\[
\log K = \frac{1}{R} (\nu_{0} \varphi_{0} + \nu_{1} \varphi_{1} + \nu_{2} \varphi_{2} + \dots + \nu_{0}' \varphi_{0}'),
\]
and, therefore, on substituting the values of~$\nu$ from~\Eq{(228)},
\[
\frac{n_{1} + n_{2} + n_{3} + \dots}{n_{0}} = \frac{1}{R} \left(\frac{m_{0}}{m_{0}'}\, \varphi_{0}' - \varphi_{0}\right)\Add{.}
\Tag{(230)}
\]
This expression shows that $\log K$ also has a small value.

Suppose for the moment that $\log K = 0$, \ie\ that the
pure solvent takes the place of the solution
\[
n_{1} + n_{2} + \dots = 0,
\]
then, by~\Eq{(230)},
\[
\frac{\varphi_{0}}{m_{0}} = \frac{\varphi_{0}'}{m_{0}'}.
\]
\PageSep{247}
Since $\varphi_{0}$~and $\varphi_{0}'$ depend only on $\theta$,~$p$ and the nature of the
solvent, and not on the dissolved substances, the above
equation asserts a definite relation between temperature and
pressure, which is, in fact, the condition which $\theta$~and~$p$ must
fulfil, in order that the two states of aggregation of the pure
solvent may exist in contact. On substituting the values of
$\varphi_{0}$~and $\varphi_{0}'$ from~\Eq{(214)}, we return immediately to the condition
of equilibrium~\Eq{(101)} which we deduced in the second chapter.
The pressure (vapour pressure) may be taken as depending
on the temperature, or the temperature (boiling point, melting
point) as depending on the pressure.

Returning now to the general case expressed in equation~\Eq{(230)},
we find that the solution of foreign molecules,
$n_{1}$,~$n_{2}$, $n_{3}$,~\dots affects the functional relation between $\theta$~and~$p$,
which holds for the pure solvent. The deviation, in fact,
depends only on the total number of dissolved molecules,
and not on their nature. To find its amount in measurable
quantities, we may introduce either~$p_{0}$, the pressure which
would exist in the system at the given temperature~$\theta$, if
there were no dissolved molecules (lowering of the vapour
pressure), or the temperature~$\theta_{0}$ which would exist at the
given pressure~$p$, if there were no dissolved molecules
(elevation of the boiling point, depression of the freezing
point). If we take the second alternative, $\theta - \theta_{0}$ will be
very small, and we may, therefore, put
\[
\log K = \frac{\dd \log K}{\dd \theta}\; (\theta - \theta_{0}),
\]
or, by~\Eq{(219)},
\[
\log K = \frac{1}{R} · \frac{L}{\theta^{2}} (\theta - \theta_{0}),
\]
and
\[
\therefore
\frac{n_{1} + n_{2} + n_{3} + \dots}{n_{0}}
  = \frac{L}{R\theta^{2}} (\theta - \theta_{0}),
\]
or
\[
\theta - \theta_{0} = \frac{R\theta^{2}}{n_{0} L} (n_{1} + n_{2} + n_{3} + \dots)\Add{.}
\Tag{(231)}
\]
By this formula the elevation of the boiling point may
be calculated directly from the number of the dissolved
molecules, the temperature, and the heat of vaporization.
\PageSep{248}
Since $L$~refers to the evaporation of one gram molecule of the
liquid, the product~$n_{0} L$ depends only on the mass, and not
on the molecular weight~($m_{0}$) of the liquid solvent. If $L$~is
to be expressed in calories, we must put $R = 1.97$ (by~\Chg{34}{\Eq{(34)}}).
For instance, for one litre of water under atmospheric
pressure,
\[
n_{0} L = 1000 × 536~\Unit{cal.}\quad\text{(approximately)},\quad
\theta = 373,
\]
and, therefore, the elevation of the boiling point is
\begin{align*}
\theta - \theta_{0}
  &= \frac{1.97 × 373^{2}}{1000 × 536} (n_{1} + n_{2} + \dots) \\
  &= 0.51 (n_{1} + n_{2} + \dots)°~\Unit{C.}
\end{align*}

\Section{269.} Let us now compare equation~\Eq{(231)} with the
relation~\Eq{(183)}, also referring to the elevation of the boiling
point, but deduced from more general principles independent
of any molecular theory. The equation is
\[
\theta - \theta_{0} = \frac{c \theta^{2} \varphi}{L}\Add{.}
\Tag{(232)}
\]
Here $c$~denotes the ratio of the mass~$M_{2}$ of the dissolved
non-volatile substance to the mass~$M_{1}$ of the solvent. In
the present notation,
\[
c = \frac{n_{1} m_{1} + n_{2} m_{2} + \dots}{n_{0} m_{0}}.
\Tag{(233)}
\]
$L$,~in~\Eq{(232)}, is the heat of vaporization per unit mass of the
solvent; therefore, in the present notation,
\[
\frac{L}{m_{0}}\Add{.}
\Tag{(234)}
\]
The equation~\Eq{(232)}, therefore, becomes
\[
\theta - \theta_{0} = \frac{(n_{1} m_{1} + n_{2} m_{2} + \dots) \theta^{2} \varphi}{n_{0} L}.
\]
\PageSep{249}
Comparison with~\Eq{(231)} shows that the two theories will
agree perfectly only if
\[
\varphi = \frac{R(n_{1} + n_{2} + \dots)}{n_{1} m_{1} + n_{2} m_{2} + \dots}.
\Tag{(235)}
\]
The molecular theory here set forth specializes the previous
more general theory in such a way as to assign the particular
value~\Eq{(235)} to the quantity~$\varphi$, formerly defined by~\Eq{(165)}.

\Section{270.} The quantity~$\varphi$ was found to be of importance for
a whole series of other properties of solutions besides the
elevation of the boiling point. These relations may at once
be specialized for dilute solutions by substituting the value
of~$c\varphi$ from \Eq{(233)} and~\Eq{(235)},
\[
c\varphi = \frac{R(n_{1} + n_{2} + n_{3} + \dots)}{n_{0} m_{0}},
\Tag{(236)}
\]
and for $L$~and~$s$, by~\Eq{(234)}, the values
\[
\frac{L}{m_{0}}\quad\text{and}\quad \frac{s}{m_{0}}\Add{.}
\Tag{(237)}
\]
In this way, for the lowering of the vapour pressure of dilute
solutions, we deduce, from~\Eq{(180)},
\[
p_{0} - p = \frac{R\theta}{n_{0}s} (n_{1} + n_{2} + n_{3} + \dots).
\Tag{(238)}
\]
If the vapour of the solvent form a perfect gas, and the
specific volume of the solution be negligible in comparison
with that of the vapour, then $s$~(the change of volume of
the system produced by the evaporation of a gram molecule
of the liquid) is equal to the volume of the vapour formed.
By~\Eq{(228)},
\[
s = R\, \frac{m_{0}}{m_{0}'} · \frac{\theta}{p},
\]
\PageSep{250}
therefore, by~\Eq{(238)},
\[
p_{0} - p = \frac{m_{0}'p (n_{1} + n_{2} + \dots)}{n_{0} m_{0}},
\]
or, the relative lowering of the vapour pressure,
\index{Lowering!of vapour pressure}%
\[
\frac{p_{0} - p}{p} = (n_{1} + n_{2} + n_{3} + \dots)\, \frac{m_{0}'}{n_{0} m_{0}}.
\]
This relation is frequently stated thus:---\emph{The relative lowering
of the vapour pressure of a solution is equal to the ratio
of the number of the dissolved molecules $(n_{1} + n_{2} + n_{3} + \dots)$
to the number of the molecules of the solvent~$(n_{0})$, or, what is
the same thing in dilute solutions, to the total number of the
molecules of the solution.} This proposition holds only, as is
evident, if $m_{0} = m_{0}'$, \ie\ if the molecules of the solvent possess
the same molecular weight in the vapour as in the
liquid. This, however, is not generally true, as, for example,
in the case of water. It may be well therefore to emphasize
this fact, that nothing concerning the molecular weight of
the solvent can be inferred from the relative lowering of the
vapour pressure, any more than from its boiling point, freezing
point, or osmotic pressure. Measurements of this kind
will not, under any circumstances, lead to anything but the
total number ($n_{1} + n_{2} + \dots$) of the dissolved molecules.
Thus, in the last equation the product~$n_{0} m_{0}$ is immediately
determined by the mass of the liquid solvent, and the mole-weight,~$m_{0}'$,
of the vapour by its density.

\Section{271.} For the depression of the freezing point of a
\index{Depression of freezing point}%
\index{Freezing point!depression of}%
dilute solution, it follows from \Eq{(186)}, \Eq{(236)}, and~\Eq{(237)}, that
\[
\theta_{0}' - \theta' = \frac{R \theta^{2}}{n_{0} L'} (n_{1} + n_{2} + n_{3} + \dots),
\]
$L'$~being the heat of solidification of a gram molecule of the
solvent. The product,~$n_{0} L'$, is given by the mass of the
solvent; it is independent of its molecular weight. To
express~$L'$ in calories we must put $R = 1.97$ (by~\Chg{34}{\Eq{(34)}}).
\PageSep{251}

Take water as an example: For $1$~litre of water under
atmospheric pressure, $n_{0} L' = 1000 × 80~\Unit{cal.}$ approximately.
\index{Pressure!osmotic}%
$\theta_{0}' = 273$, and therefore the depression of the freezing point is
\[
\theta_{0}' - \theta'
  = \frac{1.97 × 273^{2}}{1000 × 80} (n_{1} + n_{2} + \dots)
  = 1.84 (n_{1} + n_{2} + \dots)°~\Unit{C.}
\]

\Section{272.} Finally, for the osmotic pressure~$P$ we have, from
\index{Osmotic pressure}%
\Eq{(190)},
\[
P = \frac{R\theta}{n_{0} m_{0} v} (n_{1} + n_{2} + n_{3} + \dots),
\]
$v$~is the specific volume of the solution, and therefore the
product~$n_{0} m_{0} v$ is approximately its whole volume~$V$.
Hence
\[
P = \frac{R\theta}{V} (n_{1} + n_{2} + n_{3} + \dots),
\]
an expression identical with the characteristic equation of a
mixture of perfect gases with the number of molecules,
$n_{1}$,~$n_{2}$, $n_{3}$,~\dots\Add{.}

\Section{273.} Each of the theorems deduced in the preceding
paragraphs contains a method of determining the total
number of the dissolved molecules in a dilute solution.
Should the number calculated from such a measurement
disagree with the number calculated from the percentage
composition of the solution on the assumption of normal
molecules, some chemical change of the dissolved molecules
must have taken place by dissociation, association, hydrolysis,
or the like. This inference is of great importance in the
determination of the chemical nature of dilute solutions.
The number and nature of the different kinds of molecules
are uniquely determined by the total number of molecules
only in quite special cases, viz.\ when the dissolved
substance undergoes a chemical change only in one way.
In this case the total mass of the dissolved substance
and the total number of molecules formed by it in the
\PageSep{252}
solution are sufficient for the calculation of the number
of all the different kinds of molecules present. This
case is exceptional, however, for we have seen (\SecRef{259})
that all the molecules a substance is capable of forming
necessarily occur in the solution in finite quantities. As
soon as two reactions (\eg\ \ce{H2SO4 $=$ 2H+ $+$ SO4^{-\,-}} and
\ce{H2SO4 $=$ H+ $+$ HSO4-}) must be considered, the analysis of
the equilibrium remains indeterminate, since there are more
unknown quantities than determining equations. For this
reason there is no direct connection between the depression
of freezing point, the elevation of the boiling point, etc., on
the one hand, and electrical conductivity on the other. For
the one set of quantities depends on the total number of
the dissolved molecules, charged or uncharged, while the
other depends on the number and nature of molecules
charged with electricity (ions), which cannot, in general,
be calculated from the former. Conversely, a disagreement
between the depression of the freezing point as calculated
from the conductivity, and as observed, is not in itself an
objection to the theory, but rather to the assumptions made
in the calculation concerning the kinds of molecules present.

Raoult was the first to establish rigorously by experiment
the relation between the depression of the freezing
point and the number of the molecules of the dissolved
substance; and van't Hoff gave a thermodynamical explanation
and generalization of it by means of his theory of
osmotic pressure. Application to electrolytes was rendered
possible by Arrhenius' theory of electrolytic dissociation.
\index{Arrhenius'!theory of electrolytic dissociation}%
\index{Dissociation!Arrhenius' theory of electrolytic}%
\index{Electrolytic dissociation!Arrhenius' theory of}%
Thermodynamics has led quite independently, by the method
here described, to the necessity of postulating chemical
changes of the dissolved substances in dilute solutions.

\Section{274.} \Topic{Each Phase contains both Constituents in
Appreciable Quantity.}---The most important case is the
evaporation of a liquid solution, in which not only the
solvent, but also the dissolved substance is volatile. The
general equation of equilibrium~\Eq{(218)}, being applicable to
\PageSep{253}
mixtures of perfect gases whether the mixture may be
supposed dilute or not, holds with corresponding approximation
for a vapour of any composition. The liquid, on the
other hand, must be assumed to be a dilute solution.

In general, all kinds of molecules will be present in
\index{Van't Hoff's laws}%
both phases, and therefore the system is represented by
\[
n_{0}\ m_{0},\quad
n_{1}\ m_{1},\quad
n_{2}\ m_{2},\ \dots \mid
n_{0}'\ m_{0},\quad
n_{1}'\ m_{1},\quad
n_{2}'\ m_{2},\ \dots\Add{.}
\]
The molecules have the same molecular weight in both
phases. The total number of molecules in the liquid is
\[
n = n_{0} + n_{1} + n_{2} + \dots\quad\text{(nearly $= n_{0}$)},
\]
in the vapour
\[
n' = n_{0}' + n_{1}' + n_{2}' + \dots\Add{.}
\]
The concentrations of the different kinds of molecules are,
in the liquid,
\[
c_{0} = \frac{n_{0}}{n};\quad
c_{1} = \frac{n_{1}}{n};\quad
c_{2} = \frac{n_{2}}{n};\ \dots
\]
in the vapour,
\[
c_{0}' = \frac{n_{0}'}{n'};\quad
c_{1}' = \frac{n_{1}'}{n'};\quad
c_{2}' = \frac{n_{2}'}{n'};\ \dots\Add{.}
\]
The reaction
\begin{multline*}
\nu_{0} : \nu_{1} : \nu_{2} : \dots : \nu_{0}' : \nu_{1}' : \nu_{2}' : \dots \\
  = \delta n_{0} : \delta n_{1} : \delta n_{2} : \dots : \delta n_{0}' : \delta n_{1}' : \delta n_{2}' : \dots
\end{multline*}
consists in the evaporation of a molecule of the first kind,
and therefore
\[
\nu_{0} = 0,\
\nu_{1} = -1,\
\nu_{2} = 0,\ \dots\quad
\nu_{0}' = 0,\
\nu_{1}' = 1,\
\nu_{2}' = 0,\ \dots\Add{.}
\]
The equation of equilibrium becomes
\[
-\log c_{1} + \log c_{1}' = \log K,
\]
or
\[
\frac{c_{1}'}{c_{1}} = K.
\]
\PageSep{254}
\emph{For every kind of molecule, which possesses the saute molecular
weight in both phases, there is a constant ratio of distribution,
which is independent of the presence of other molecules}
(Nernst's law of distribution).
\index{Distribution law (Nernst's)}%
\index{Laws:!Nernst's}%
\index{Nernst}%
\index{Nernst's law of distribution}%

If, on the other hand, a molecule of the solvent
evaporate, we have,
\[
\nu_{0} = -1,\
\nu_{1} = 0,\
\nu_{2} = 0,\ \dots\quad
\nu_{0}' = 1,\
\nu_{1}' = 0,\
\nu_{2}' = 0\Add{,}\ \dots;
\]
and the equation of equilibrium becomes
\[
-\log c_{0} + \log c_{0}' = \log K,
\]
where
\begin{align*}
-\log c_{0}
  &= \log \frac{n}{n_{0}}
   = \log \left(1 + \frac{n_{1} + n_{2} + \dots}{n_{0}}\right)
   = \frac{n_{1} + n_{2} + \dots}{n_{0}} \\
  &= c_{1} + c_{2} + \dots
\Tag{(239)}
\end{align*}
\[
\therefore
c_{1} + c_{2} + \dots + \log c_{0}' = \log K,
\Tag{(240)}
\]
where $c_{1}$,~$c_{2}$,~\dots, the concentrations of the molecules dissolved
in the liquid, have small values. Two cases must be
considered.

Either, the molecules in~$m_{0}$ in the vapour form only a small
or at most a moderate portion of the number of the vapour
molecules. Then the small numbers $c_{1}$,~$c_{2}$,~\dots, may be
neglected in comparison with the logarithm, and therefore
\[
\log c_{0}' = \log K.
\]
This asserts that the concentration of the molecules of
the solvent in the vapour does not depend on the composition
of the solution. An example of this is the evaporation
of a dilute solution, when the solvent is not very volatile,
\eg\ alcohol in water. The partial pressure of the solvent
(water) in the vapour is not at all dependent on the concentration
of the solution, but is equal to that of the pure
solvent.

Or, the molecules~$m_{0}$ in the vapour far outnumber all
the other molecules, as, \eg, when alcohol is the solvent in
\PageSep{255}
the liquid phase, water the dissolved substance. The concentrations
$c_{1}$,~$c_{2}$,~\dots must not be neglected, and, as in~\Eq{(239)},
\[
\log c_{0}' = -(c_{1}' + c_{2}' + \dots);
\]
equation~\Eq{(240)} therefore becomes
\[
(c_{1} + c_{2} + \dots) - (c_{1}' + c_{2}' + \dots) = \log K.
\]

This relation contains an extension of van't Hoff's laws
\index{Laws:!Van't Hoff's}%
concerning the elevation of the boiling point, the diminution
of the vapour pressure, etc., and asserts that \emph{when the
substance dissolved in the liquid also passes in part into the
vapour, the elevation of boiling point or the diminution of
the vapour pressure depends no longer on the concentrations
of the molecules dissolved in the liquid, but on the \Emph{difference}
of their concentrations in the liquid and in the vapour}.
If this difference be zero, the distillate being of the same
composition as the liquid, the elevation of the boiling point
and the diminution of the vapour pressure vanish. This
conclusion has already been reached from a more general
point of view (\SecRef{219}). If the concentration of the dissolved
substance in the vapour be larger than that in the liquid,
as may happen in the evaporation of an aqueous solution of
alcohol, the boiling point falls, while the vapour pressure
rises.

Exactly analogous theorems may, of course, be deduced
for other states of aggregation. Thus, the more general
statement of the law concerning the freezing point would
be: \emph{If both the solvent and the dissolved substance of a dilute
solution solidify in such a way as to form another dilute
solution, the depression of the freezing point is not proportional
to the concentrations of the dissolved substances in the liquid,
but to the \Emph{difference} of the concentrations of the dissolved
substances in the liquid and solid phases, and changes sign
with this difference.} The solidification of some alloys is an
example.

While these laws govern the distribution of the molecules
in both phases, the equilibrium within each phase
\PageSep{256}
obeys the laws, which were deduced in \SecRef{262}, etc. We
again meet with the laws of dissociation, association, etc.
(Nernst).

\Section{275.} \Topic{Three Independent Constituents in one Phase.} \\
---Two dissolved substances in a dilute solution will not affect
one another unless they have certain kinds of molecules in
common, for there is no transformation possible, and therefore
no special condition of equilibrium to fulfil. If two
dilute solutions of totally different electrolytes in the same
solvent be mixed, each solution will behave as if it had been
diluted with a corresponding quantity of the pure solvent.
The degree of the dissociation will rise to correspond to the
greater dilution.

It is different when both electrolytes have an ion in
common, as, for example, acetic acid and sodium acetate.
\index{Acetate!sodium}%
In this case, before mixing there are two systems:
\[
\ce{$n_{0}$ H2O},\quad
\ce{$n_{1}$ CH3 . COOH},\quad
\ce{$n_{2}$ H+},\quad
\ce{$n_{3}$ CH3- . COO},
\]
and
\[
\ce{$n_{0}'$ H2O},\quad
\ce{$n_{1}'$ CH3 . COONa},\quad
\ce{$n_{2}'$ Na+},\quad
\ce{$n_{3}'$ CH3- . COO}.
\]
As in~\Eq{(222)}, for the first solution,
\[
\frac{c_{2}^{2}}{c_{1}} = K,\quad\text{or}\quad
\frac{n_{2}^{2}}{n_{1} n_{0}} = K,
\Tag{(241)}
\]
for the second,
\[
\frac{c_{2}'^{2}}{c_{1}'} = K',\quad\text{or}\quad
\frac{n_{2}'^{2}}{n_{1}' n_{0}'} = K'.
\Tag{(242)}
\]
After mixing the two, we have the system
\begin{multline*}
\ce{$\bar{n}_{0}$ H2O},\quad
\ce{$\bar{n}_{1}$ CH3 . COOH},\quad
\ce{$\bar{n}_{2}$ CH3 . COONa}, \\
\ce{$\bar{n}_{3}$ H+},\quad
\ce{$\bar{n}_{4}$ Na+},\quad
\ce{$\bar{n}_{5}$ CH3- . COO},
\end{multline*}
where, necessarily,
\settowidth{\TmpLen}{\null\qquad$=$ number of $-$~ions)}%
\[
\left.
\begin{aligned}
\bar{n}_{0} &= n_{0} + n_{0}'\quad\text{(number of \ce{H2O} molecules)\Add{,}} \\
\bar{n}_{2} + \bar{n}_{4} &= n_{1}' + n_{2}'\quad\text{(number of \ce{Na} atoms)\Add{,}} \\
\bar{n}_{1} + \bar{n}_{3} &= n_{1} + n_{2}\quad\text{(number of \ce{H} atoms)\Add{,}} \\
\bar{n}_{3} + \bar{n}_{4} &= \bar{n}_{5}\quad\parbox[t]{\TmpLen}{(number of $+$~ions \\ \null\qquad$=$ number of $-$~ions)\Add{.}}
\end{aligned}
\right\}
\Tag{(243)}
\]
\PageSep{257}
The total number of molecules in the system is
\[
\bar{n} = \bar{n}_{0} + \bar{n}_{1} + \bar{n}_{2} + \bar{n}_{3} + \bar{n}_{4} + \bar{n}_{5}\quad\text{(nearly $= \bar{n}_{0}$).}
\]
The concentrations are
\[
\bar{c}_{0} = \frac{\bar{n}_{0}}{\bar{n}};\
\bar{c}_{1} = \frac{\bar{n}_{1}}{\bar{n}};\
\bar{c}_{2} = \frac{\bar{n}_{2}}{\bar{n}};\
\bar{c}_{3} = \frac{\bar{n}_{3}}{\bar{n}};\
\bar{c}_{4} = \frac{\bar{n}_{4}}{\bar{n}};\
\bar{c}_{5} = \frac{\bar{n}_{5}}{\bar{n}}.
\]
In the system there are two different reactions,
\[
\nu_{0} : \nu_{1} : \nu_{2} : \nu_{3} : \nu_{4} : \nu_{5}
  = \delta \bar{n}_{0} : \delta \bar{n}_{1} : \delta \bar{n}_{2} : \delta \bar{n}_{3} : \delta \bar{n}_{4} : \delta \bar{n}_{5},
\]
possible; first, the dissociation of one molecule of acetic acid,
\[
\nu_{0} = 0,\quad
\nu_{1} = -1,\quad
\nu_{2} = 0,\quad
\nu_{3} = 1,\quad
\nu_{4} = 0,\quad
\nu_{5} = 1,
\]
and therefore the condition of equilibrium is, by~\Eq{(218)},
\[
-\log \bar{c}_{1} + \log \bar{c}_{3} + \log \bar{c}_{5} = \log K,
\]
or
\[
\frac{\bar{c}_{3} \bar{c}_{5}}{\bar{c}_{1}} = K,\quad\text{or}\quad
\frac{\bar{n}_{3} · \bar{n}_{5}}{\bar{n}_{1} · \bar{n}_{0}}
  = \frac{\bar{n}_{3} · \bar{n}_{5}}{\bar{n}_{1} (n_{0} + n_{0}')} = K;
\Tag{(244)}
\]
second, the dissociation of a molecule of sodium acetate,
\[
\nu_{0} = 0,\quad
\nu_{1} = 0,\quad
\nu_{2} = -1,\quad
\nu_{3} = 0,\quad
\nu_{4} = 1,\quad
\nu_{5} = 1,
\]
whence, for equilibrium,
\[
-\log \bar{c}_{2} + \log \bar{c}_{4} + \log \bar{c}_{5} = \log K',
\quad\text{or}\quad
\frac{\bar{c}_{4} \bar{c}_{5}}{\bar{c}_{2}} = K',
\]
or
\[
\frac{\bar{n}_{4} · \bar{n}_{5}}{\bar{n}_{2} · \bar{n}_{0}}
  = \frac{\bar{n}_{4} · \bar{n}_{5}}{\bar{n}_{2} (n_{0} + n_{0}')} = K'.
\Tag{(245)}
\]
The quantities $K$~and $K'$ are the same as those in \Eq{(241)}
and~\Eq{(242)}. They depend, besides on $\theta$~and~$p$, only on the
nature of the reaction, and not on the concentrations, nor
on other possible reactions. By the conditions of equilibrium
\Eq{(244)} and~\Eq{(245)}, together with the four equations~\Eq{(243)}, the
values of the six quantities $\bar{n}_{0}$, $\bar{n}_{1}$,~\dots\Add{,} $\bar{n}_{5}$ are uniquely
\PageSep{258}
determined, if the original solutions and also the number of
molecules $n_{0}$, $n_{1}$,~\dots and $n_{0}'$, $n_{1}'$\Add{,}~\dots be given.

\Section{276.} The condition that the two solutions should be
\index{Solution!isohydric}%
\emph{isohydric}, \ie\ that their degree of dissociation should remain
unchanged on mixing them, is evidently expressed by the
two equations
\[
\bar{n}_{1} = n_{1},\quad\text{and}\quad
\bar{n}_{2} = n_{1}',
\]
\ie\ the number of undissociated molecules of both acetic
acid and sodium acetate must be the same in the original
solutions as in the mixture. It immediately follows, by~\Eq{(243)},
that
\[
\bar{n}_{3} = n_{2},\quad
\bar{n}_{4} = n_{2}',\quad
\bar{n}_{5} = n_{2} + n_{2}'.
\]
These values, substituted in \Eq{(244)} and~\Eq{(245)}, and combined
with \Eq{(241)} and~\Eq{(242)}, give
\begin{align*}
\frac{n_{2} (n_{2} + n_{2}')}{n_{1} (n_{0} + n_{0}')}
  &= K = \frac{n_{2}^{2}}{n_{1} n_{0}}, \\
\frac{n_{2}' (n_{2} + n_{2}')}{n_{1}' (n_{0} + n_{0}')}
  &= K' = \frac{n_{2}'^{2}}{n_{1}' n_{0}'},
\end{align*}
whence the single condition of isohydric solutions is
\index{Isohydric solutions}%
\[
\frac{n_{2}}{n_{0}} = \frac{n_{2}'}{n_{0}'},\quad\text{or}\quad
c_{2} = c_{2}'\ (= c_{3} = c_{3}'),
\]
or, \emph{the two solutions are isohydric if the concentration of
the common ion \emph{\ce{CH3- COO}} is the same in both}. This proposition
was enunciated by Arrhenius, who verified it by
\index{Arrhenius}%
numerous experiments. In all cases where this condition
is not realized, chemical changes must take place on mixing
the solutions, either dissociation or association. The direction
and amount of these changes may be estimated by imagining
the dissolved substances separate, and the entire solvent
distributed over the two so as to form isohydric solutions.
\PageSep{259}
If, for instance, both solutions are originally normal ($1$~gram
molecule in $1$~litre of solution), they will not be isohydric,
since sodium acetate in normal solution is more strongly
dissociated, and has, therefore, a greater concentration of
\ce{CH3- . COO}-ions, than acetic acid. In order to distribute the
solvent so that the concentration of the common ion
\ce{CH3- . COO} may be the same in both solutions, some water
must be withdrawn from the less dissociated electrolyte
(acetic acid), and added to the more strongly dissociated
(\ce{Na}-acetate). For, though it is true that with decreasing
dilution the dissociation of the acid becomes less, the concentration
of free ions increases, as \Typo{(262)}{\SecRef{262}}~shows, because the
ions are now compressed into a smaller quantity of water.
Conversely, the dissociation of the sodium acetate increases
on the addition of water, but the concentration of the free
ions decreases, because they are distributed over a larger
quantity of water. In this way the concentration of the
common ion \ce{CH3- . COO} may be made the same in both
solutions, and then their degree of dissociation will not be
changed by mixing. This is also the state ultimately
reached by the two normal solutions, when mixed. It
follows, then, that when two equally diluted solutions of
binary electrolytes are mixed, the dissociation of the more
weakly dissociated recedes, while that of the more strongly
dissociated increases still further.

\Section{277.} \Topic{Three Independent Constituents in Two
Phases.}---We shall first discuss the simple case, where the
second phase contains only one constituent in appreciable
quantity. A solution of an almost insoluble salt in a liquid,
to which a small quantity of a third substance has been
added, forms an example of this case. Let us consider an
aqueous solution of silver bromate and silver nitrate. This
\index{Silver!bromate}%
\index{Silver!nitrate}%
two-phase system is represented by
\begin{multline*}
\ce{$n_{0}$ H2O},\quad
\ce{$n_{1}$ AgBrO3},\quad
\ce{$n_{2}$ AgNO3}, \\
\ce{$n_{3}$ Ag+},\quad
\ce{$n_{4}$ BrO3-},\quad
\ce{$n_{5}$ NO3-} \mid
\ce{$n_{0}'$ AgBrO3}.
\end{multline*}
\PageSep{260}
The concentrations are
\[
c_{0} = \frac{n_{0}}{n};\quad
c_{1} = \frac{n_{1}}{n};\quad
c_{2} = \frac{n_{2}}{n};\ \dots;\quad
c_{0}' = \frac{n_{0}'}{n_{0}'},
\]
where
\[
n = n_{0} + n_{1} + n_{2} + n_{3} + n_{4} + n_{5}\quad\text{(nearly $= n_{0}$).}
\]
Of the possible reactions,
\begin{multline*}
\nu_{0} : \nu_{1} : \nu_{2} : \nu_{3} : \nu_{4} : \nu_{5} : \nu_{0}' \\
  = \delta n_{0} : \delta n_{1} : \delta n_{2} : \delta n_{3} : \delta n_{4} : \delta n_{5} : \delta n_{0}',
\end{multline*}
we shall first consider the passage of one molecule of
\ce{AgBrO3} from the solution, viz.\
\[
\nu_{0} = 0,\quad
\nu_{1} = -1,\quad
\nu_{2} = 0,\ \dots\Add{,}\quad
\nu_{0}' = 1.
\]
The condition of equilibrium is, therefore,
\[
-\log c_{1} + \log c_{0}' = \log K
\]
or
\[
c_{1} = \frac{1}{K}.
\Tag{(246)}
\]
The concentration of the undissociated molecules of silver
bromate in the saturated solution depends entirely on the
temperature and the pressure.

We may now consider the dissociation of a molecule of
\ce{AgBrO3} into its two ions.
\[
\nu_{0} = 0,\
\nu_{1} = -1,\
\nu_{2} = 0,\
\nu_{3} = 1,\
\nu_{4} = 1,\
\nu_{5} = 0,\
\nu_{0}' = 0,
\]
and, therefore,
\begin{gather*}
-\log c_{1} + \log c_{3} + \log c_{4} = \log K', \\
\frac{c_{3} c_{4}}{c_{1}} = K',
\end{gather*}
or, by~\Eq{(246)},
\[
c_{3} c_{4} = \frac{K'}{K},
\Tag{(247)}
\]
\ie\ the product of the concentrations of the \ce{Ag+}~and \ce{BrO3-}
ions depends only on temperature and pressure. The concentration
of the \ce{Ag+}-ions is inversely proportional to the
\PageSep{261}
concentration of the \ce{BrO3-}-ions. Since the addition of silver
nitrate increases the number of the \ce{Ag+}-ions, it diminishes
the number of the \ce{BrO3-}-ions, and thereby the solubility of
the bromate, which is evidently measured by the sum $c_{1} + c_{4}$.

We shall, finally, consider the dissociation of a molecule
of~\ce{AgNO3} into its ions.
\[
\nu_{0} = 0,\
\nu_{1} = 0,\
\nu_{2} = -1,\
\nu_{3} = 1,\
\nu_{4} = 0,\
\nu_{5} = 1,\
\nu_{0}' = 0,
\]
whence, by~\Eq{(218)},
\[
\frac{c_{3} c_{5}}{c_{2}} = K''.
\Tag{(248)}
\]
To equations \Eq{(246)}, \Eq{(247)}, and~\Eq{(248)}, must be added, as a
fourth, the condition
\[
c_{3} = c_{4} + c_{5},
\]
and, as a fifth, the value of $c_{2} + c_{5}$, given by the quantity of
the nitrate added, so that the five unknown quantities, $c_{1}$,~$c_{2}$,
$c_{3}$, $c_{4}$,~$c_{5}$, are uniquely determined.

The theory of such influences on solubility was first
established by Nernst, and has been experimentally verified
\index{Nernst}%
by him, and more recently by Noyes.
\index{Noyes}%

\Section{278.} The more general case, where each of the two
phases contains all three constituents, is realized in the
distribution of a salt between two solvents, which are themselves
soluble to a small extent in one another (\eg\ water
and ether). The equilibrium is completely determined by
a combination of the conditions holding for the transition of
molecules from one phase to another with those holding for
the chemical reactions of the molecules within one and the
same phase. The former set of conditions may be summed
up in Nernst's law of distribution (\SecRef{274}). It assigns to
each kind of molecule in the two phases a constant ratio of
distribution, which is independent of the presence of other
dissolved molecules. The second set is the conditions of
\PageSep{262}
the coexistence of three independent constituents in one
phase (\SecRef{275}), to which must be added Arrhenius' theory of
\index{Arrhenius'!theory of isohydric solutions}%
isohydric solutions.
\index{Isohydric solutions!Arrhenius' theory of}%

\Section{279.} The same method applies to four or more independent
constituents combined into one or several phases.
The notation of the system is given in each case by~\Eq{(216)},
and any possible reaction of the system may be reduced to
the form~\Eq{(217)}, which corresponds to the condition of equilibrium~\Eq{(218)}.
All the conditions of equilibrium, together
with the given conditions of the system, give the number of
equations which the phase rule prescribes for the determination
of the state of equilibrium.

When chemical interchanges between the different substances
in solution are possible, as, \eg, in a solution of dissociating
salts and acids with common ions, the term \emph{degree
of dissociation} has no meaning, for the ions may be combined
arbitrarily into dissociated molecules. For instance, in the
solution
\begin{multline*}
\ce{$n_{0}$ H2O},\quad
\ce{$n_{1}$ NaCl},\quad
\ce{$n_{2}$ KCl},\quad
\ce{$n_{3}$ NaNO3}, \\
\ce{$n_{4}$ KNO3},\quad
\ce{$n_{5}$ Na+},\quad
\ce{$n_{6}$ K+},\quad
\ce{$n_{7}$ Cl-},\quad
\ce{$n_{8}$ NO3-}
\end{multline*}
we cannot tell which of the \ce{Na+}-ions should be regarded as
belonging to~\ce{NaCl}, and which to~\ce{NaNO3}. In such cases
the only course is to characterize the state by the concentrations
of the dissolved molecules.

The above system consists of water and four salts, but,
besides the solvent, only three are independent constituents,
for the quantities of the \ce{Na}, the~\ce{K}, and the~\ce{Cl} determine
that of the~\ce{NO3}. Accordingly, by \SecRef{204} ($\alpha = 4$, $\beta = 1$) all
the concentrations are completely determined at given
temperature and pressure by three of them. This is independent
of other kinds of molecules, and other reactions,
which, as is likely, may have to be considered in establishing
the conditions of equilibrium.

\Section{280.} If in a system of any number of independent
\PageSep{263}
constituents in any number of phases, the condition of
equilibrium~\Eq{(218)} is not satisfied, \ie\ if for any virtual
isothermal-isopiestic change
\[
\tsum \nu_{0} \log c_{0} + \nu_{1} \log c_{1} + \nu_{2} \log c_{2} + \dots \gtrless \log K,
\]
then the direction of the change which will actually take
place in nature is given by the condition $d\Psi > 0$ (\SecRef{147}).
If we now denote by $\nu_{0}$,~$\nu_{1}$, $\nu_{2}$,~\dots, simple whole numbers,
which are not only proportional to, but also of the same
sign as the actual changes which take place, then we have,
by~\Eq{(215)},
\[
\tsum \nu_{0} \log c_{0} + \nu_{1} \log c_{1} + \nu_{2} \log c_{2} + \dots < \log K,
\]
for the direction of any actual isothermal isopiestic change,
whether it be a chemical change inside any single phase, or
the passage of molecules between the different phases. The
constant~$K$ is defined by~\Eq{(218)}.

To find the connection between the difference of the
expressions on the right and left and the time of the reaction
is immediately suggested, and, in fact, a general law for the
velocity of an irreversible isothermal isopiestic process may
be thus deduced. We shall not, however, enter further into
these considerations in this book.
\index{Dilute solutions|)}%
\index{Solutions, dilute|)}%
\PageSep{264}


\BackMatter

\Appendix{Catalogue}
\index{Catalogue}%

Of the Author's Publications on Thermodynamics, excluding the
applications to Electricity, with a reference to the paragraphs
of this book, which deal with the same point.

\selectlanguage{german}
\Bibitem
``\Chg{Ueber}{Über} den zweiten Hauptsatz der mechanischen Wärmetheorïe.
  Inaugural-dissertation.'' München. Th.\ Ackermann. S.~1--61.
  1879. (\SecRefs{106}{136}.)

\Bibitem
``Gleichgewichtszustände isotroper Körper in verschieden Temperaturen,
  Habitationsschrift.'' München. Th.\ Ackermann.
  S.~1--68. 1880. (\SecRefs{153}{187}.)

\Bibitem
``Die Theorie des Sättigungsgesetzes.'' Wied.\ Ann.~13. S.~535--543.
  1881. (\SecRef{172}.)

\Bibitem
``Verdampfen, Schmelzen, und Sublimiren.'' Wied.\ Ann.~15.
  S.~446--475. 1882. (\SecRefs{188}{196}.)

\Bibitem
``\Chg{Ueber}{Über} das thermodynamische Gleichgewicht von Gasgemengen.''
  Wied.\ Ann.~19. S.~358--378. 1883. (\SecRefs{232}{248}.)

\Bibitem
``Das Princip der Erhaltung der Energie.'' Leipzig. B.~G.
  Teubner. S.~1--247. 1887. (\SecRefs{55}{105}.)

\Bibitem
``\Chg{Ueber}{Über} das Princip der Vermehrung der Entropie.'' Erste
  Abhandlung. Gesetze des Verlaufs von Reaktionen, die
  nach constanten Gewichts-verhältnissen vor sich gehen.
  Wied.\ Ann.~30. S.~562--582. 1887. (\SecRefs{206}{212}.)

\Bibitem
``\Chg{Ueber}{Über} das Princip der Vermehrung der Entropie.'' Zweite
  Abhandlung. Gesetze der Dissociation gasförmiger Verbindungen.
  Wied.\ Ann.~31. S.~189--203. 1887. (\SecRefs{232}{248}.)

\Bibitem
``\Chg{Ueber}{Über} das Princip der Vermehrung der Entropie.'' Dritte
  Abhandlung. Gesetze des Eintritts beliebiger thermodynamischer
  und chemischer Reaktionen. Wied.\ Ann.~32.
  S.~462--503. 1887. (\SecRefs{232}{279}.)

\Bibitem
``\Chg{Ueber}{Über} die Molekulare Constitution verdünnter Lösungen.''
\PageSep{265}
Zeitschr.\ f.~phys.\ Chem.~1. S.~577--582. 1887. (\SSecRef{271},
\SecNum{273}.)

\Bibitem
``Das chemische Gleichgewicht in verdünnten Lösungen.'' Wied.\
  Ann.~34. S.~139--154. 1888. (\SecRef{262}~f., \SecRefs{268}{273}.)

\Bibitem
{\Loosen ``\Chg{Ueber}{Über} die Hypothese der Dissociation der Salze in sehr verdünnten
  Lösungen.'' Zeitschr.\ f.~phys.\ Chem.~2. S.~343.
  1888. (\SecRef{271}.)}

\Bibitem
``\Chg{Ueber}{Über} die Dampfspannung von verdünnten Lösungen flüchtiger
  Stoffe.'' Zeitchr.\ f.~phys.\ Chem.~2. S.~405--414. 1888.
  (\SecRef{274}).

\Bibitem
``\Chg{Ueber}{Über} den osmotischen Druck.'' Zeitschr.\ f.~phys.\ Chem.~6.
  S.~187--189. 1890. (\SSecRef{229}, \SecNum{272}.)

\Bibitem
``Allgemeines zur neuren Entwicklung der Wärmetheorie.''
  Zeitschr.\ f.~phys.\ Chem.~8. S.~647--656. 1891. (\SecRef{136}.)

\Bibitem
``Bemerkung über das Carnot-Clausius'sche Princip.'' Wied.\
  Ann.~46. S.~162--166. 1892. (\SecRef{134}.)

\Bibitem
``Erwiderung auf einen von Herrn Arrhenius erhobenen Einwand.''
  Zeitschr.\ f.~phys.\ Chem.~9. S.~636~f. 1892. (\SecRef{253}.)

\Bibitem
``Der Kern des zweiten Hauptsatzes der Wärmetheorie.'' Zeitschr.\
  f.~d.\ phys.\ und chem.\ Untemcht~6. S.~217--221. 1893.
  (\SecRefs{106}{115}.)

\Bibitem
``Grundriss der allgemeinen Thermochemie.'' Breslau. E.\
  Trewendt. S.~1--140. 1893. (\SecRefs{1}{66}, \SecRefs[]{92}{152}, \SecRefs[]{197}{279}.)

\Bibitem
``Gegen die neuere Energetik.'' Wied.\ Ann.~57. S.~72--78.
  1896. (\SecRefs{108}{113}.)
\PageSep{266}
% [** TN: Blank Page]
\PageSep{267}

\selectlanguage{UKenglish}
\printindex

% [** TN: Index text]
\iffalse

INDEX

The numbers refer to pages.


% A

Abnormal vapour densities 30

Absolute temperature 6
  deduced from Thomson and Joule's experiments 127-131

Acetate
  silver 244
  sodium 256

Acetic acid 237

Adiabatic process 59, 109

Affinity of hydrogen for oxygen 112

Aggregation, states of 69, 132
  co-existence of states of 153

Air, composition of 11

Ammonium carbamate, evaporation of 188

Ammonium chloride, evaporation of 188

Andrews 14, 140

Apt, Dr.\ Richard 14

Arrhenius 235, 238, 258

Arrhenius'
  theory of electrolytic dissociation 252
  theory of isohydric solutions 262

Atmospheric pressure 4

Atom, definition of 25

Atomic heat 34

Avogadro's law 25, 27


% B

Babo's law 198

Barus 20

Berthelot 71-73, 243

Berthelot's principle 113

Binary electrolyte 237

Bodenstein 219

Boiling point, elevation of 200

Boyle and Gay-Lussac 197

Boyle's law 5, 57


% C

Calorie
  laboratory 33
  large 33
  mean 33
  small 33
  zero 33

Calorimetric bomb 71

Cantor@Cantor, Hr.|indexnote#Cantor 227

Carbon, combustion of 74

Carbon dioxide
  Van der Waals' constants for 14
  isotherms of 15

Carnot's
  theory 36
  cycle 62, 100

Catalogue 264

Characteristic constant 11

Characteristic equation 5, 6, 11
  deduced from Thomson and Joule's experiments 126

Clapeyron 142

Clausius 87

Clausius' equation 14, 140
  form of second law 96
  notation 55
  statement of first and second laws 101

Coefficient
  of compressibility 8
  of elasticity 7
  of expansion 7
  of pressure 7

Coexistence of states of aggregation 153

Combustion, of carbon 74
  influence of temperature on 76

Condensed system 181

Condition of complete reversibility
%\PageSep{268}
  of a process 91
  of equilibrium 115, 136, 176
  of a gas mixture 215-217

Conductivity of water, electrical 236

Conservation of energy 38, 40

Constituents, independent 173

Corresponding point. 161

Crafts 220

Critical point 17
  of \ce{CO2}#CO 19
  pressure 17
  solution temperature 182
  specific volume 17
  temperature 17, 152

Curves
  of evaporation 158, 161
  of fusion 158, 161
  of sublimation 158, 161

Cycle of operations 44


% D

Dalton's law 10, 20

Davy 36

Decrease of free energy by dilution 112

Deductions from second law of thermodynamics 105

Density
  specific 7
  abnormal vapour 30

Depression of freezing point 250

Developable surface 164

Deviation from perfect gases 13, 123

Difference of specific heats 121

Diffusion 9
  increase of entropy by 214
  irreversible 214

Dilute solutions 223-263
  energy of 224
  entropy of 226
  thermodynamical theory of 222
  volume of 225

Dilution
  decrease of free energy by 112
  heat of 198
  infinite 70
  law of, of binary electrolytes 238

Direction of natural process 108

Dissipation of energy. 101

Dissociation
  graded 221
  of \ce{H2SO4}#SO 238
  of hydriodic acid 219
  of iodine vapour 220
  of water 234
  Arrhenius' theory of electrolytic 252

Distribution law (Nernst's) 254

Divariant system 181

Duhem@Duhem|indexnote#Duhem 115

Dulong and Petit's law 34

Dyne 4


% E

Elasticity, coefficient of 7

Electrical conductivity of water 236

Electrolyte, binary 237

Electrolytic dissociation
  Arrhenius' theory of 252

Elevation of boiling point 200

Endothermal process 37

Energetics 79, 84

Energy
  change of 43
  conservation of 38, 40
  definition of 39
  dissipation of 101
  free 110
    of perfect gas 113
  internal 47
    of perfect gas 57
  latent 110
  of a solution 70
  of dilute solution 224
  of gas mixture 209
  potential 45
  total 110
  zero 44

Entropy
  definition 97
  diminution of 93
  increase of, by diffusion 214
  maximum value 117
  of a gas 89
  of a system of gases 92
  of dilute solution 225
  of gas mixture 209-214
  principle of increase of 100
  specific 119

Equation
  characteristic 5, 6, 11
  deduced from Thomson and Joule's experiments 126
  Clausius' 14, 140
  Van der Waals' 13

Equilibrium
  thermal 2
  conditions of 115, 136, 176
  of gas mixture 215, 217

Equivalent weight 23

Equivalents, number of 23

Euler 176

Evaporation
  of ammonium carbamate 188
  of ammonium chloride 188
  theory of 135

Exothermal process 37

Expansion, coefficient of 7

External conditions of equilibrium 136
  effect 39
  variable 178
  work in complete cycle 54
  work in reversible process 51, 52
%\PageSep{269}


% F

Favre 74

First law of thermodynamics 38, 42, 46

Free energy 110
  change of, with temperature. 113
  decrease of, by dilution 112
  minimum value of 117
  of a perfect gas 113

Freezing point
  depression of 250

Function $\Psi$ 114

Fundamental point (triple) 155
  pressure 154
  temperature 154
  temperature of ice 154
  triangle 159

Fusion, curve 158, 161
  theory of 135


% G

Gas
  constant 27
  thermometer 3
  volume 27

Gas mixture 9
  energy of 209
  entropy of 209-214
  volume 28

Gases, perfect 5, 57

Gaseous system 207-222

Gay-Lussac 24, 57

Gay-Lussac's law, deviations from 123

Gibbs 73, 173, 212, 232

Gibbs's phase rule 179, 232

Graded dissociation 221

Gram-calorie, mechanical equivalent of 41


% H

Heat
  absorbed 53
  atomic 34
  capacity 33
  conception of 1
  molecular 34
    of perfect gases 58
  of combustion 75, 76
  of dilution 198
  of formation of \ce{CO2}, of \ce{CS2}, of \ce{CH4}#formation 75
  of fusion 37
  of neutralization 73
  of precipitation 201
  of solidification 201
  of solution 190
  of sublimation 37
  of vaporization 37
  quantity 32
  specific (definition) 33
  total 36
  unit 32

Heat and work, analogy between 53

Heat effect 37
  at constant pressure 71
  in thermochemistry 68
  of dilution of \ce{H2SO4}#SO 70

Heat function at constant pressure 73

Heat, latent
  theory 110
  approximation formula 143

Heating
  at constant pressure 56
  at constant volume 56

Henry's law 242

Hertz, H. 146

Heterogeneous system 180

Heydweiller 234

Hirn 148

Homogeneous
  substance 138
  system 119-131

Horstmann 188

Hydriodic acid, dissociation of 219

Hydrobromamylene 30

Hydrogen, affinity of, for oxygen 112

Hydrogen peroxide 74


% I

Independent constituents 173

Inertia resistance 116

Infinite dilution 70

Infinitely slow
  compression 50
  process 49-51

Inflection
  point of 17

Influence
  of pressure on specific heat 123
  of temperature on combustion 76

Internal conditions of equilibrium 136
  variable 178

Internal energy 47
  of perfect gas 48

Iodine vapour, dissociation of 220

Irreversible diffusion 214

Isobaric change 7

Isochoric change 7

Isohydric solutions 258
  Arrhenius' theory of 262

Isomorphous substance 182

Isopiestic change 7

Isopycnic change 7

Isothermal processes 110
%\PageSep{270}

Isothermal-isopiestic process 114

Isotherms of \ce{CO2}#CO 15

Isotropic bodies 3


% J

Jahn 245

Joule 36

Joule's experiments 40-42, 47

Joule and Thomson's absolute temperature 127-131
  experiments 48, 57
  (theory) 124


% K

Kirchhoff 191

Kirchhoff's formula 198

Kohlrausch 234

Konowalow 196

Krigar@Krigar-Manzel|indexnote#Krigar 99

Kundt 122


% L

Latent energy 110

Latent heat 37, 140, 143
  from phase rule 187-189

Laws:
  Avogadro's 25, 57
  Babo's 198
  Boyle's 5, 57
  Dalton's 10, 20
  Dulong and Petit's 34
  Gay-Lussac's 6, 24, 57
  Henry's 242
  Mariotte's 5
  Nernst's 254
  Neumann's (Regnault) 35
  Ostwald's 238
  Van't Hoff's 255
  Wüllner's 199

Laws of thermodynamics@Laws of thermodynamics|see{\textit{First} and \text{Second}}#Thermodynamics 0

Lead sulphide 68

Liquefaction pressure 20

Lowering
  of freezing point 202
  of vapour pressure 199, 250


% M

Mariotte's law 5

Maximum value
  of entropy 117
  of free energy 117
  of $\Psi$ 118

Maximum work 111

Maxwell 87

Mechanical equivalent
  of a gram-calorie 41
  of heat 40
    in absolute units 42

Meier, Fr. 220

Melting point of ice 146
  lowering of, by pressure 146

Membranes, semipermeable 29, 203

Meyer, Robert 62

Mixture of gases 9

Mixtures 20

Molecular heat 34
  of perfect gases 58

Molecular weight 22
  apparent 28

Molecules, number of 25


% N

Naccari 242

Natural process, direction of 108

Nernst 242, 254, 261

Nernst's law of distribution 254

Neumann, F. 35

Neutralization, heat of 73

Nitrogen
  oxides 23
  peroxide 30

Non-variant system 179

Noyes 261


% O

Osmotic pressure 204, 251

Ostwald's law 238

Oxides of nitrogen 23


% P

Pagliani 242

Partial pressures 10

Perfect gases 5, 57
  system 44

Phase
  defined 173
  rule 179

Phosphorus pentachloride 30

Planck 228

Point
  $(n + 2)$-ple#-ple 179
  of inflection 17
  triple 155, 180
  quadruple 180
  quintuple 180

Porous plug experiments 48

Potassium chlorate 182

Potential, energy 45
  thermodynamic@thermodynamic|indexnote#thermodynamic 115
%\PageSep{271}

Precipitation, heat of 201

Pressure coefficient 7
  of mercury 9

Pressure
  fundamental 154
  osmotic 204, 251
  of liquefaction 20

Principle of Berthelot 113

Process
  adiabatic 109
  endothermal 37
  isothermal 110
  isothermal-isopiestic 114
  exothermal 37

Processes
  periodic 83
  reversible and irreversible 82


% Q

Quadruple point 180

Quantity of heat 32

Quintuple point 180


% R

Ratio of specific heats 59, 122

Regnault 58, 143, 148

Resistance inertia 116

Reversibility of a process
  complete
    condition of 94

Roozeboom, Bakhuis 179

Rumford 36


% S

Saturation point 16

Second law of thermodynamics
  introduction 77
  proof 86
  possible limitations 103
  deductions 105
  test of 147, 148

Semipermeable membranes 29, 203

Silbermann 74

Silver
  acetate 244
  bromate 259
  nitrate 244, 259

Singular values 37

Sodium
  carbonate 73
  hydrate 73

Solidification
  heat of 201
  pressure 20

Solution
  heat of 190
  isohydric 258

Solutions, dilute 223-263

Solvent 196

Sound, velocity of 61

Specific density 7

Specific entropy 119

Specific heat 33
  at constant pressure 56, 59, 120
  at constant volume 56, 59, 120
  influence of temperature on, at constant pressure 123
  of saturated vapour 150
  of steam 148

Specific heats
  difference of 121
  ratio of 59, 122

Spring 20

States of aggregation 69, 132
  co-existence of 153

Stohmann 71

Sublimation, curve 158, 161
  theory of 135

Substance, isomorphous 182

Succinic acid 243

Sulphur 31
  dioxide and water equilibrium 180

Sulphuric acid, dissociation of 238

Surface, developable 164

System
  condensed 181
  divariant 181
  gaseous 207-222
  heterogeneous 180
  homogeneous 119-131
  non-variant 179
  perfect 44
  univariant 180


% T

Temperature
  absolute 6
  critical 17, 152
  critical solution 182
  definition of 2, 3
  fundamental 154
    of ice 154

Thallium chlorate 182

Theoretical regions 19

Thermal equilibrium 2

Thermochemical symbols 68

Thermodynamic potential 115

Thermodynamical theory
  of dilute solutions 222
  of fusion, vaporization, and sublimation 135

Thermometer, gas 3

Thiesen 140

Thomsen, J. 68, 73, 235

Thomson 36, 87, 146

Thomson@Thomson|see{Joule and Thomson}#Thomson

Transformability of heat into work 80

Triangle, fundamental 159

Triple point 155, 180
%\PageSep{272}


% U

Unit of heat 32

Univariant system 180


% V

Van der Waals'
  constants for \ce{CO2}#CO 14
  equation 13

Van't Hoff's laws 253

Vaporization curve 158, 161
  theory of 135

Vapour densities, abnormal 30

Vapour pressure, lowering of 199

Variable, internal and external 178


% W

Warburg 122

Water, dissociation of 234

Weights molecular and equivalent 22, 23

Work
  and heat, analogy between 53
  external, in reversible process 51
  maximum 111

Wüllner's law 199


% Z

Zero
  calorie 33
  energy 44
  state 92

%[** TN End of index]
\fi

% PRINTED BY WILLIAM CLOWES AND SONS, LIMITED, LONDON AND BECCLES.

%%%%%%%%%%%%%%%%%%%%%%%%% GUTENBERG LICENSE %%%%%%%%%%%%%%%%%%%%%%%%%%
\PGLicense
\begin{PGtext}
End of Project Gutenberg's Treatise on Thermodynamics, by Max Planck

*** END OF THIS PROJECT GUTENBERG EBOOK TREATISE ON THERMODYNAMICS ***

***** This file should be named 50880-t.tex or 50880-pdf.zip *****
This and all associated files of various formats will be found in:
        http://www.gutenberg.org/5/0/8/8/50880/

Produced by Andrew D. Hwang
Updated editions will replace the previous one--the old editions will
be renamed.

Creating the works from print editions not protected by U.S. copyright
law means that no one owns a United States copyright in these works,
so the Foundation (and you!) can copy and distribute it in the United
States without permission and without paying copyright
royalties. Special rules, set forth in the General Terms of Use part
of this license, apply to copying and distributing Project
Gutenberg-tm electronic works to protect the PROJECT GUTENBERG-tm
concept and trademark. Project Gutenberg is a registered trademark,
and may not be used if you charge for the eBooks, unless you receive
specific permission. If you do not charge anything for copies of this
eBook, complying with the rules is very easy. You may use this eBook
for nearly any purpose such as creation of derivative works, reports,
performances and research. They may be modified and printed and given
away--you may do practically ANYTHING in the United States with eBooks
not protected by U.S. copyright law. Redistribution is subject to the
trademark license, especially commercial redistribution.

START: FULL LICENSE

THE FULL PROJECT GUTENBERG LICENSE
PLEASE READ THIS BEFORE YOU DISTRIBUTE OR USE THIS WORK

To protect the Project Gutenberg-tm mission of promoting the free
distribution of electronic works, by using or distributing this work
(or any other work associated in any way with the phrase "Project
Gutenberg"), you agree to comply with all the terms of the Full
Project Gutenberg-tm License available with this file or online at
www.gutenberg.org/license.

Section 1. General Terms of Use and Redistributing Project
Gutenberg-tm electronic works

1.A. By reading or using any part of this Project Gutenberg-tm
electronic work, you indicate that you have read, understand, agree to
and accept all the terms of this license and intellectual property
(trademark/copyright) agreement. If you do not agree to abide by all
the terms of this agreement, you must cease using and return or
destroy all copies of Project Gutenberg-tm electronic works in your
possession. If you paid a fee for obtaining a copy of or access to a
Project Gutenberg-tm electronic work and you do not agree to be bound
by the terms of this agreement, you may obtain a refund from the
person or entity to whom you paid the fee as set forth in paragraph
1.E.8.

1.B. "Project Gutenberg" is a registered trademark. It may only be
used on or associated in any way with an electronic work by people who
agree to be bound by the terms of this agreement. There are a few
things that you can do with most Project Gutenberg-tm electronic works
even without complying with the full terms of this agreement. See
paragraph 1.C below. There are a lot of things you can do with Project
Gutenberg-tm electronic works if you follow the terms of this
agreement and help preserve free future access to Project Gutenberg-tm
electronic works. See paragraph 1.E below.

1.C. The Project Gutenberg Literary Archive Foundation ("the
Foundation" or PGLAF), owns a compilation copyright in the collection
of Project Gutenberg-tm electronic works. Nearly all the individual
works in the collection are in the public domain in the United
States. If an individual work is unprotected by copyright law in the
United States and you are located in the United States, we do not
claim a right to prevent you from copying, distributing, performing,
displaying or creating derivative works based on the work as long as
all references to Project Gutenberg are removed. Of course, we hope
that you will support the Project Gutenberg-tm mission of promoting
free access to electronic works by freely sharing Project Gutenberg-tm
works in compliance with the terms of this agreement for keeping the
Project Gutenberg-tm name associated with the work. You can easily
comply with the terms of this agreement by keeping this work in the
same format with its attached full Project Gutenberg-tm License when
you share it without charge with others.

1.D. The copyright laws of the place where you are located also govern
what you can do with this work. Copyright laws in most countries are
in a constant state of change. If you are outside the United States,
check the laws of your country in addition to the terms of this
agreement before downloading, copying, displaying, performing,
distributing or creating derivative works based on this work or any
other Project Gutenberg-tm work. The Foundation makes no
representations concerning the copyright status of any work in any
country outside the United States.

1.E. Unless you have removed all references to Project Gutenberg:

1.E.1. The following sentence, with active links to, or other
immediate access to, the full Project Gutenberg-tm License must appear
prominently whenever any copy of a Project Gutenberg-tm work (any work
on which the phrase "Project Gutenberg" appears, or with which the
phrase "Project Gutenberg" is associated) is accessed, displayed,
performed, viewed, copied or distributed:

  This eBook is for the use of anyone anywhere in the United States and
  most other parts of the world at no cost and with almost no
  restrictions whatsoever. You may copy it, give it away or re-use it
  under the terms of the Project Gutenberg License included with this
  eBook or online at www.gutenberg.org. If you are not located in the
  United States, you'll have to check the laws of the country where you
  are located before using this ebook.

1.E.2. If an individual Project Gutenberg-tm electronic work is
derived from texts not protected by U.S. copyright law (does not
contain a notice indicating that it is posted with permission of the
copyright holder), the work can be copied and distributed to anyone in
the United States without paying any fees or charges. If you are
redistributing or providing access to a work with the phrase "Project
Gutenberg" associated with or appearing on the work, you must comply
either with the requirements of paragraphs 1.E.1 through 1.E.7 or
obtain permission for the use of the work and the Project Gutenberg-tm
trademark as set forth in paragraphs 1.E.8 or 1.E.9.

1.E.3. If an individual Project Gutenberg-tm electronic work is posted
with the permission of the copyright holder, your use and distribution
must comply with both paragraphs 1.E.1 through 1.E.7 and any
additional terms imposed by the copyright holder. Additional terms
will be linked to the Project Gutenberg-tm License for all works
posted with the permission of the copyright holder found at the
beginning of this work.

1.E.4. Do not unlink or detach or remove the full Project Gutenberg-tm
License terms from this work, or any files containing a part of this
work or any other work associated with Project Gutenberg-tm.

1.E.5. Do not copy, display, perform, distribute or redistribute this
electronic work, or any part of this electronic work, without
prominently displaying the sentence set forth in paragraph 1.E.1 with
active links or immediate access to the full terms of the Project
Gutenberg-tm License.

1.E.6. You may convert to and distribute this work in any binary,
compressed, marked up, nonproprietary or proprietary form, including
any word processing or hypertext form. However, if you provide access
to or distribute copies of a Project Gutenberg-tm work in a format
other than "Plain Vanilla ASCII" or other format used in the official
version posted on the official Project Gutenberg-tm web site
(www.gutenberg.org), you must, at no additional cost, fee or expense
to the user, provide a copy, a means of exporting a copy, or a means
of obtaining a copy upon request, of the work in its original "Plain
Vanilla ASCII" or other form. Any alternate format must include the
full Project Gutenberg-tm License as specified in paragraph 1.E.1.

1.E.7. Do not charge a fee for access to, viewing, displaying,
performing, copying or distributing any Project Gutenberg-tm works
unless you comply with paragraph 1.E.8 or 1.E.9.

1.E.8. You may charge a reasonable fee for copies of or providing
access to or distributing Project Gutenberg-tm electronic works
provided that

* You pay a royalty fee of 20% of the gross profits you derive from
  the use of Project Gutenberg-tm works calculated using the method
  you already use to calculate your applicable taxes. The fee is owed
  to the owner of the Project Gutenberg-tm trademark, but he has
  agreed to donate royalties under this paragraph to the Project
  Gutenberg Literary Archive Foundation. Royalty payments must be paid
  within 60 days following each date on which you prepare (or are
  legally required to prepare) your periodic tax returns. Royalty
  payments should be clearly marked as such and sent to the Project
  Gutenberg Literary Archive Foundation at the address specified in
  Section 4, "Information about donations to the Project Gutenberg
  Literary Archive Foundation."

* You provide a full refund of any money paid by a user who notifies
  you in writing (or by e-mail) within 30 days of receipt that s/he
  does not agree to the terms of the full Project Gutenberg-tm
  License. You must require such a user to return or destroy all
  copies of the works possessed in a physical medium and discontinue
  all use of and all access to other copies of Project Gutenberg-tm
  works.

* You provide, in accordance with paragraph 1.F.3, a full refund of
  any money paid for a work or a replacement copy, if a defect in the
  electronic work is discovered and reported to you within 90 days of
  receipt of the work.

* You comply with all other terms of this agreement for free
  distribution of Project Gutenberg-tm works.

1.E.9. If you wish to charge a fee or distribute a Project
Gutenberg-tm electronic work or group of works on different terms than
are set forth in this agreement, you must obtain permission in writing
from both the Project Gutenberg Literary Archive Foundation and The
Project Gutenberg Trademark LLC, the owner of the Project Gutenberg-tm
trademark. Contact the Foundation as set forth in Section 3 below.

1.F.

1.F.1. Project Gutenberg volunteers and employees expend considerable
effort to identify, do copyright research on, transcribe and proofread
works not protected by U.S. copyright law in creating the Project
Gutenberg-tm collection. Despite these efforts, Project Gutenberg-tm
electronic works, and the medium on which they may be stored, may
contain "Defects," such as, but not limited to, incomplete, inaccurate
or corrupt data, transcription errors, a copyright or other
intellectual property infringement, a defective or damaged disk or
other medium, a computer virus, or computer codes that damage or
cannot be read by your equipment.

1.F.2. LIMITED WARRANTY, DISCLAIMER OF DAMAGES - Except for the "Right
of Replacement or Refund" described in paragraph 1.F.3, the Project
Gutenberg Literary Archive Foundation, the owner of the Project
Gutenberg-tm trademark, and any other party distributing a Project
Gutenberg-tm electronic work under this agreement, disclaim all
liability to you for damages, costs and expenses, including legal
fees. YOU AGREE THAT YOU HAVE NO REMEDIES FOR NEGLIGENCE, STRICT
LIABILITY, BREACH OF WARRANTY OR BREACH OF CONTRACT EXCEPT THOSE
PROVIDED IN PARAGRAPH 1.F.3. YOU AGREE THAT THE FOUNDATION, THE
TRADEMARK OWNER, AND ANY DISTRIBUTOR UNDER THIS AGREEMENT WILL NOT BE
LIABLE TO YOU FOR ACTUAL, DIRECT, INDIRECT, CONSEQUENTIAL, PUNITIVE OR
INCIDENTAL DAMAGES EVEN IF YOU GIVE NOTICE OF THE POSSIBILITY OF SUCH
DAMAGE.

1.F.3. LIMITED RIGHT OF REPLACEMENT OR REFUND - If you discover a
defect in this electronic work within 90 days of receiving it, you can
receive a refund of the money (if any) you paid for it by sending a
written explanation to the person you received the work from. If you
received the work on a physical medium, you must return the medium
with your written explanation. The person or entity that provided you
with the defective work may elect to provide a replacement copy in
lieu of a refund. If you received the work electronically, the person
or entity providing it to you may choose to give you a second
opportunity to receive the work electronically in lieu of a refund. If
the second copy is also defective, you may demand a refund in writing
without further opportunities to fix the problem.

1.F.4. Except for the limited right of replacement or refund set forth
in paragraph 1.F.3, this work is provided to you 'AS-IS', WITH NO
OTHER WARRANTIES OF ANY KIND, EXPRESS OR IMPLIED, INCLUDING BUT NOT
LIMITED TO WARRANTIES OF MERCHANTABILITY OR FITNESS FOR ANY PURPOSE.

1.F.5. Some states do not allow disclaimers of certain implied
warranties or the exclusion or limitation of certain types of
damages. If any disclaimer or limitation set forth in this agreement
violates the law of the state applicable to this agreement, the
agreement shall be interpreted to make the maximum disclaimer or
limitation permitted by the applicable state law. The invalidity or
unenforceability of any provision of this agreement shall not void the
remaining provisions.

1.F.6. INDEMNITY - You agree to indemnify and hold the Foundation, the
trademark owner, any agent or employee of the Foundation, anyone
providing copies of Project Gutenberg-tm electronic works in
accordance with this agreement, and any volunteers associated with the
production, promotion and distribution of Project Gutenberg-tm
electronic works, harmless from all liability, costs and expenses,
including legal fees, that arise directly or indirectly from any of
the following which you do or cause to occur: (a) distribution of this
or any Project Gutenberg-tm work, (b) alteration, modification, or
additions or deletions to any Project Gutenberg-tm work, and (c) any
Defect you cause.

Section 2. Information about the Mission of Project Gutenberg-tm

Project Gutenberg-tm is synonymous with the free distribution of
electronic works in formats readable by the widest variety of
computers including obsolete, old, middle-aged and new computers. It
exists because of the efforts of hundreds of volunteers and donations
from people in all walks of life.

Volunteers and financial support to provide volunteers with the
assistance they need are critical to reaching Project Gutenberg-tm's
goals and ensuring that the Project Gutenberg-tm collection will
remain freely available for generations to come. In 2001, the Project
Gutenberg Literary Archive Foundation was created to provide a secure
and permanent future for Project Gutenberg-tm and future
generations. To learn more about the Project Gutenberg Literary
Archive Foundation and how your efforts and donations can help, see
Sections 3 and 4 and the Foundation information page at
www.gutenberg.org



Section 3. Information about the Project Gutenberg Literary Archive Foundation

The Project Gutenberg Literary Archive Foundation is a non profit
501(c)(3) educational corporation organized under the laws of the
state of Mississippi and granted tax exempt status by the Internal
Revenue Service. The Foundation's EIN or federal tax identification
number is 64-6221541. Contributions to the Project Gutenberg Literary
Archive Foundation are tax deductible to the full extent permitted by
U.S. federal laws and your state's laws.

The Foundation's principal office is in Fairbanks, Alaska, with the
mailing address: PO Box 750175, Fairbanks, AK 99775, but its
volunteers and employees are scattered throughout numerous
locations. Its business office is located at 809 North 1500 West, Salt
Lake City, UT 84116, (801) 596-1887. Email contact links and up to
date contact information can be found at the Foundation's web site and
official page at www.gutenberg.org/contact

For additional contact information:

    Dr. Gregory B. Newby
    Chief Executive and Director
    gbnewby@pglaf.org

Section 4. Information about Donations to the Project Gutenberg
Literary Archive Foundation

Project Gutenberg-tm depends upon and cannot survive without wide
spread public support and donations to carry out its mission of
increasing the number of public domain and licensed works that can be
freely distributed in machine readable form accessible by the widest
array of equipment including outdated equipment. Many small donations
($1 to $5,000) are particularly important to maintaining tax exempt
status with the IRS.

The Foundation is committed to complying with the laws regulating
charities and charitable donations in all 50 states of the United
States. Compliance requirements are not uniform and it takes a
considerable effort, much paperwork and many fees to meet and keep up
with these requirements. We do not solicit donations in locations
where we have not received written confirmation of compliance. To SEND
DONATIONS or determine the status of compliance for any particular
state visit www.gutenberg.org/donate

While we cannot and do not solicit contributions from states where we
have not met the solicitation requirements, we know of no prohibition
against accepting unsolicited donations from donors in such states who
approach us with offers to donate.

International donations are gratefully accepted, but we cannot make
any statements concerning tax treatment of donations received from
outside the United States. U.S. laws alone swamp our small staff.

Please check the Project Gutenberg Web pages for current donation
methods and addresses. Donations are accepted in a number of other
ways including checks, online payments and credit card donations. To
donate, please visit: www.gutenberg.org/donate

Section 5. General Information About Project Gutenberg-tm electronic works.

Professor Michael S. Hart was the originator of the Project
Gutenberg-tm concept of a library of electronic works that could be
freely shared with anyone. For forty years, he produced and
distributed Project Gutenberg-tm eBooks with only a loose network of
volunteer support.

Project Gutenberg-tm eBooks are often created from several printed
editions, all of which are confirmed as not protected by copyright in
the U.S. unless a copyright notice is included. Thus, we do not
necessarily keep eBooks in compliance with any particular paper
edition.

Most people start at our Web site which has the main PG search
facility: www.gutenberg.org

This Web site includes information about Project Gutenberg-tm,
including how to make donations to the Project Gutenberg Literary
Archive Foundation, how to help produce our new eBooks, and how to
subscribe to our email newsletter to hear about new eBooks.
\end{PGtext}

% %%%%%%%%%%%%%%%%%%%%%%%%%%%%%%%%%%%%%%%%%%%%%%%%%%%%%%%%%%%%%%%%%%%%%%% %
%                                                                         %
% End of Project Gutenberg's Treatise on Thermodynamics, by Max Planck    %
%                                                                         %
% *** END OF THIS PROJECT GUTENBERG EBOOK TREATISE ON THERMODYNAMICS ***  %
%                                                                         %
% ***** This file should be named 50880-t.tex or 50880-pdf.zip *****      %
% This and all associated files of various formats will be found in:      %
%         http://www.gutenberg.org/5/0/8/8/50880/                         %
%                                                                         %
% %%%%%%%%%%%%%%%%%%%%%%%%%%%%%%%%%%%%%%%%%%%%%%%%%%%%%%%%%%%%%%%%%%%%%%% %

\end{document}
###
@ControlwordReplace = (
  ['\\BookHead', 'Treatise on Thermodynamics'],
  ['\\cf', 'cf.'],
  ['\\Cf', 'Cf.'],
  ['\\ie', 'i.e.'],
  ['\\ie', 'i.e.'],
  ['\\eg', 'e.g.']
  );

@ControlwordArguments = (
  ['\\TPage', 1, 0, '', '', 1, 1, '', ''],
  ['\\BookMark', 1, 0, '', '', 1, 0, '', ''],
  ['\\Graphic', 1, 0, '', '', 1, 0, '', ''],
  ['\\Signature', 1, 1, '', ' ', 1, 1, '', ' ', 1, 1, '', ' ', 1, 1, '', ''],
  ['\\Figure', 0, 0, '', '', 1, 0, '<FIGURE>', ''],
  ['\\ce', 1, 0, '<CHEM>', ''],
  ['\\First', 1, 1, '', ''],
  ['\\Part', 1, 1, 'Part ', ' ', 1, 1, '', ''],
  ['\\Chapter', 0, 0, '', '', 1, 1, 'Chapter ', ' ', 1, 1, '', ''],
  ['\\Section', 1, 1, '§ ', ''],
  ['\\SecRef', 1, 1, '§', ''],
  ['\\SSecRef', 1, 1, '§§', ''],
  ['\\SecRefs', 0, 0, '', '', 1, 1, '§§', '--', 1, 1, '', ''],
  ['\\SecNum', 1, 1, '', ''],
  ['\\Fig', 1, 1, 'Fig. ', ''],
  ['\\Eq', 1, 1, '', ''],
  ['\\Erratum', 1, 0, '', '', 1, 1, '', ''],
  ['\\Typo', 1, 0, '', '', 1, 1, '', ''],
  ['\\Add', 1, 1, '', ''],
  ['\\Chg', 1, 0, '', '', 1, 1, '', '']
  );
$PageSeparator = qr/^\\PageSep/;
$CustomClean = 'print "\\nCustom cleaning in progress...";
my $cline = 0;
 while ($cline <= $#file) {
   $file[$cline] =~ s/--------[^\n]*\n//; # strip page separators
   $cline++
 }
 print "done\\n";';
###
This is pdfTeX, Version 3.1415926-2.5-1.40.14 (TeX Live 2013/Debian) (format=pdflatex 2015.9.16)  8 JAN 2016 19:13
entering extended mode
 %&-line parsing enabled.
**50880-t.tex
(./50880-t.tex
LaTeX2e <2011/06/27>
Babel <3.9h> and hyphenation patterns for 78 languages loaded.
(/usr/share/texlive/texmf-dist/tex/latex/base/book.cls
Document Class: book 2007/10/19 v1.4h Standard LaTeX document class
(/usr/share/texlive/texmf-dist/tex/latex/base/bk12.clo
File: bk12.clo 2007/10/19 v1.4h Standard LaTeX file (size option)
)
\c@part=\count79
\c@chapter=\count80
\c@section=\count81
\c@subsection=\count82
\c@subsubsection=\count83
\c@paragraph=\count84
\c@subparagraph=\count85
\c@figure=\count86
\c@table=\count87
\abovecaptionskip=\skip41
\belowcaptionskip=\skip42
\bibindent=\dimen102
) (/usr/share/texlive/texmf-dist/tex/latex/base/inputenc.sty
Package: inputenc 2008/03/30 v1.1d Input encoding file
\inpenc@prehook=\toks14
\inpenc@posthook=\toks15
(/usr/share/texlive/texmf-dist/tex/latex/base/latin1.def
File: latin1.def 2008/03/30 v1.1d Input encoding file
)) (/usr/share/texlive/texmf-dist/tex/generic/babel/babel.sty
Package: babel 2013/12/03 3.9h The Babel package
(/usr/share/texlive/texmf-dist/tex/generic/babel-german/germanb.ldf
Language: germanb 2013/12/13 v2.7 German support for babel (traditional orthogr
aphy)
(/usr/share/texlive/texmf-dist/tex/generic/babel/babel.def
File: babel.def 2013/12/03 3.9h Babel common definitions
\babel@savecnt=\count88
\U@D=\dimen103
)
\l@austrian = a dialect from \language\l@german 
Package babel Info: Making " an active character on input line 107.
) (/usr/share/texlive/texmf-dist/tex/generic/babel-english/english.ldf
Language: english 2012/08/20 v3.3p English support from the babel system
\l@canadian = a dialect from \language\l@american 
\l@australian = a dialect from \language\l@british 
\l@newzealand = a dialect from \language\l@british 
)) (/usr/share/texlive/texmf-dist/tex/latex/base/ifthen.sty
Package: ifthen 2001/05/26 v1.1c Standard LaTeX ifthen package (DPC)
) (/usr/share/texlive/texmf-dist/tex/latex/amsmath/amsmath.sty
Package: amsmath 2013/01/14 v2.14 AMS math features
\@mathmargin=\skip43
For additional information on amsmath, use the `?' option.
(/usr/share/texlive/texmf-dist/tex/latex/amsmath/amstext.sty
Package: amstext 2000/06/29 v2.01
(/usr/share/texlive/texmf-dist/tex/latex/amsmath/amsgen.sty
File: amsgen.sty 1999/11/30 v2.0
\@emptytoks=\toks16
\ex@=\dimen104
)) (/usr/share/texlive/texmf-dist/tex/latex/amsmath/amsbsy.sty
Package: amsbsy 1999/11/29 v1.2d
\pmbraise@=\dimen105
) (/usr/share/texlive/texmf-dist/tex/latex/amsmath/amsopn.sty
Package: amsopn 1999/12/14 v2.01 operator names
)
\inf@bad=\count89
LaTeX Info: Redefining \frac on input line 210.
\uproot@=\count90
\leftroot@=\count91
LaTeX Info: Redefining \overline on input line 306.
\classnum@=\count92
\DOTSCASE@=\count93
LaTeX Info: Redefining \ldots on input line 378.
LaTeX Info: Redefining \dots on input line 381.
LaTeX Info: Redefining \cdots on input line 466.
\Mathstrutbox@=\box26
\strutbox@=\box27
\big@size=\dimen106
LaTeX Font Info:    Redeclaring font encoding OML on input line 566.
LaTeX Font Info:    Redeclaring font encoding OMS on input line 567.
\macc@depth=\count94
\c@MaxMatrixCols=\count95
\dotsspace@=\muskip10
\c@parentequation=\count96
\dspbrk@lvl=\count97
\tag@help=\toks17
\row@=\count98
\column@=\count99
\maxfields@=\count100
\andhelp@=\toks18
\eqnshift@=\dimen107
\alignsep@=\dimen108
\tagshift@=\dimen109
\tagwidth@=\dimen110
\totwidth@=\dimen111
\lineht@=\dimen112
\@envbody=\toks19
\multlinegap=\skip44
\multlinetaggap=\skip45
\mathdisplay@stack=\toks20
LaTeX Info: Redefining \[ on input line 2665.
LaTeX Info: Redefining \] on input line 2666.
) (/usr/share/texlive/texmf-dist/tex/latex/amsfonts/amssymb.sty
Package: amssymb 2013/01/14 v3.01 AMS font symbols
(/usr/share/texlive/texmf-dist/tex/latex/amsfonts/amsfonts.sty
Package: amsfonts 2013/01/14 v3.01 Basic AMSFonts support
\symAMSa=\mathgroup4
\symAMSb=\mathgroup5
LaTeX Font Info:    Overwriting math alphabet `\mathfrak' in version `bold'
(Font)                  U/euf/m/n --> U/euf/b/n on input line 106.
)) (/usr/share/texlive/texmf-dist/tex/latex/tools/array.sty
Package: array 2008/09/09 v2.4c Tabular extension package (FMi)
\col@sep=\dimen113
\extrarowheight=\dimen114
\NC@list=\toks21
\extratabsurround=\skip46
\backup@length=\skip47
) (/usr/share/texlive/texmf-dist/tex/latex/base/alltt.sty
Package: alltt 1997/06/16 v2.0g defines alltt environment
) (/usr/share/texlive/texmf-dist/tex/latex/footmisc/footmisc.sty
Package: footmisc 2011/06/06 v5.5b a miscellany of footnote facilities
\FN@temptoken=\toks22
\footnotemargin=\dimen115
\c@pp@next@reset=\count101
\c@@fnserial=\count102
Package footmisc Info: Declaring symbol style bringhurst on input line 855.
Package footmisc Info: Declaring symbol style chicago on input line 863.
Package footmisc Info: Declaring symbol style wiley on input line 872.
Package footmisc Info: Declaring symbol style lamport-robust on input line 883.

Package footmisc Info: Declaring symbol style lamport* on input line 903.
Package footmisc Info: Declaring symbol style lamport*-robust on input line 924
.
) (/usr/share/texlive/texmf-dist/tex/latex/mhchem/mhchem.sty
Package: mhchem 2014/02/01 v3.15 for typesetting chemical formulae
(/usr/share/texlive/texmf-dist/tex/latex/tools/calc.sty
Package: calc 2007/08/22 v4.3 Infix arithmetic (KKT,FJ)
\calc@Acount=\count103
\calc@Bcount=\count104
\calc@Adimen=\dimen116
\calc@Bdimen=\dimen117
\calc@Askip=\skip48
\calc@Bskip=\skip49
LaTeX Info: Redefining \setlength on input line 76.
LaTeX Info: Redefining \addtolength on input line 77.
\calc@Ccount=\count105
\calc@Cskip=\skip50
) (/usr/share/texlive/texmf-dist/tex/latex/oberdiek/twoopt.sty
Package: twoopt 2008/08/11 v1.5 Definitions with two optional arguments (HO)
) (/usr/share/texlive/texmf-dist/tex/latex/graphics/keyval.sty
Package: keyval 1999/03/16 v1.13 key=value parser (DPC)
\KV@toks@=\toks23
) (/usr/share/texlive/texmf-dist/tex/latex/graphics/graphics.sty
Package: graphics 2009/02/05 v1.0o Standard LaTeX Graphics (DPC,SPQR)
(/usr/share/texlive/texmf-dist/tex/latex/graphics/trig.sty
Package: trig 1999/03/16 v1.09 sin cos tan (DPC)
) (/usr/share/texlive/texmf-dist/tex/latex/latexconfig/graphics.cfg
File: graphics.cfg 2010/04/23 v1.9 graphics configuration of TeX Live
)
Package graphics Info: Driver file: pdftex.def on input line 91.
(/usr/share/texlive/texmf-dist/tex/latex/pdftex-def/pdftex.def
File: pdftex.def 2011/05/27 v0.06d Graphics/color for pdfTeX
(/usr/share/texlive/texmf-dist/tex/generic/oberdiek/infwarerr.sty
Package: infwarerr 2010/04/08 v1.3 Providing info/warning/error messages (HO)
) (/usr/share/texlive/texmf-dist/tex/generic/oberdiek/ltxcmds.sty
Package: ltxcmds 2011/11/09 v1.22 LaTeX kernel commands for general use (HO)
)
\Gread@gobject=\count106
)) (/usr/share/texlive/texmf-dist/tex/generic/oberdiek/pdftexcmds.sty
Package: pdftexcmds 2011/11/29 v0.20 Utility functions of pdfTeX for LuaTeX (HO
)
(/usr/share/texlive/texmf-dist/tex/generic/oberdiek/ifluatex.sty
Package: ifluatex 2010/03/01 v1.3 Provides the ifluatex switch (HO)
Package ifluatex Info: LuaTeX not detected.
) (/usr/share/texlive/texmf-dist/tex/generic/oberdiek/ifpdf.sty
Package: ifpdf 2011/01/30 v2.3 Provides the ifpdf switch (HO)
Package ifpdf Info: pdfTeX in PDF mode is detected.
)
Package pdftexcmds Info: LuaTeX not detected.
Package pdftexcmds Info: \pdf@primitive is available.
Package pdftexcmds Info: \pdf@ifprimitive is available.
Package pdftexcmds Info: \pdfdraftmode found.
) (/usr/share/texlive/texmf-dist/tex/latex/l3kernel/expl3.sty (/usr/share/texli
ve/texmf-dist/tex/latex/l3kernel/l3names.sty (/usr/share/texlive/texmf-dist/tex
/latex/l3kernel/l3bootstrap.sty
Package: l3bootstrap 2014/01/04 v4640 L3 Experimental bootstrap code
)
Package: l3names 2014/01/04 v4640 L3 Namespace for primitives
) (/usr/share/texlive/texmf-dist/tex/latex/etex-pkg/etex.sty
Package: etex 1998/03/26 v2.0 eTeX basic definition package (PEB)
\et@xins=\count107
)
Package: expl3 2014/01/07 v4646 L3 Experimental code bundle wrapper
(/usr/share/texlive/texmf-dist/tex/latex/l3kernel/l3basics.sty
Package: l3basics 2014/01/04 v4642 L3 Basic definitions
) (/usr/share/texlive/texmf-dist/tex/latex/l3kernel/l3expan.sty
Package: l3expan 2014/01/04 v4642 L3 Argument expansion
) (/usr/share/texlive/texmf-dist/tex/latex/l3kernel/l3tl.sty
Package: l3tl 2013/12/27 v4625 L3 Token lists
) (/usr/share/texlive/texmf-dist/tex/latex/l3kernel/l3seq.sty
Package: l3seq 2013/12/14 v4623 L3 Sequences and stacks
) (/usr/share/texlive/texmf-dist/tex/latex/l3kernel/l3int.sty
Package: l3int 2013/08/02 v4583 L3 Integers
\c_max_int=\count108
\l_tmpa_int=\count109
\l_tmpb_int=\count110
\g_tmpa_int=\count111
\g_tmpb_int=\count112
) (/usr/share/texlive/texmf-dist/tex/latex/l3kernel/l3quark.sty
Package: l3quark 2013/12/14 v4623 L3 Quarks
) (/usr/share/texlive/texmf-dist/tex/latex/l3kernel/l3prg.sty
Package: l3prg 2014/01/04 v4642 L3 Control structures
\g__prg_map_int=\count113
) (/usr/share/texlive/texmf-dist/tex/latex/l3kernel/l3clist.sty
Package: l3clist 2013/07/28 v4581 L3 Comma separated lists
) (/usr/share/texlive/texmf-dist/tex/latex/l3kernel/l3token.sty
Package: l3token 2013/08/25 v4587 L3 Experimental token manipulation
) (/usr/share/texlive/texmf-dist/tex/latex/l3kernel/l3prop.sty
Package: l3prop 2013/12/14 v4623 L3 Property lists
) (/usr/share/texlive/texmf-dist/tex/latex/l3kernel/l3msg.sty
Package: l3msg 2013/07/28 v4581 L3 Messages
) (/usr/share/texlive/texmf-dist/tex/latex/l3kernel/l3file.sty
Package: l3file 2013/10/13 v4596 L3 File and I/O operations
\l_iow_line_count_int=\count114
\l__iow_target_count_int=\count115
\l__iow_current_line_int=\count116
\l__iow_current_word_int=\count117
\l__iow_current_indentation_int=\count118
) (/usr/share/texlive/texmf-dist/tex/latex/l3kernel/l3skip.sty
Package: l3skip 2013/07/28 v4581 L3 Dimensions and skips
\c_zero_dim=\dimen118
\c_max_dim=\dimen119
\l_tmpa_dim=\dimen120
\l_tmpb_dim=\dimen121
\g_tmpa_dim=\dimen122
\g_tmpb_dim=\dimen123
\c_zero_skip=\skip51
\c_max_skip=\skip52
\l_tmpa_skip=\skip53
\l_tmpb_skip=\skip54
\g_tmpa_skip=\skip55
\g_tmpb_skip=\skip56
\c_zero_muskip=\muskip11
\c_max_muskip=\muskip12
\l_tmpa_muskip=\muskip13
\l_tmpb_muskip=\muskip14
\g_tmpa_muskip=\muskip15
\g_tmpb_muskip=\muskip16
) (/usr/share/texlive/texmf-dist/tex/latex/l3kernel/l3keys.sty
Package: l3keys 2013/12/08 v4614 L3 Experimental key-value interfaces
\g__keyval_level_int=\count119
\l_keys_choice_int=\count120
) (/usr/share/texlive/texmf-dist/tex/latex/l3kernel/l3fp.sty
Package: l3fp 2014/01/04 v4642 L3 Floating points
\c__fp_leading_shift_int=\count121
\c__fp_middle_shift_int=\count122
\c__fp_trailing_shift_int=\count123
\c__fp_big_leading_shift_int=\count124
\c__fp_big_middle_shift_int=\count125
\c__fp_big_trailing_shift_int=\count126
\c__fp_Bigg_leading_shift_int=\count127
\c__fp_Bigg_middle_shift_int=\count128
\c__fp_Bigg_trailing_shift_int=\count129
) (/usr/share/texlive/texmf-dist/tex/latex/l3kernel/l3box.sty
Package: l3box 2013/07/28 v4581 L3 Experimental boxes
\c_empty_box=\box28
\l_tmpa_box=\box29
\l_tmpb_box=\box30
\g_tmpa_box=\box31
\g_tmpb_box=\box32
) (/usr/share/texlive/texmf-dist/tex/latex/l3kernel/l3coffins.sty
Package: l3coffins 2013/12/14 v4624 L3 Coffin code layer
\l__coffin_internal_box=\box33
\l__coffin_internal_dim=\dimen124
\l__coffin_offset_x_dim=\dimen125
\l__coffin_offset_y_dim=\dimen126
\l__coffin_x_dim=\dimen127
\l__coffin_y_dim=\dimen128
\l__coffin_x_prime_dim=\dimen129
\l__coffin_y_prime_dim=\dimen130
\c_empty_coffin=\box34
\l__coffin_aligned_coffin=\box35
\l__coffin_aligned_internal_coffin=\box36
\l_tmpa_coffin=\box37
\l_tmpb_coffin=\box38
\l__coffin_display_coffin=\box39
\l__coffin_display_coord_coffin=\box40
\l__coffin_display_pole_coffin=\box41
\l__coffin_display_offset_dim=\dimen131
\l__coffin_display_x_dim=\dimen132
\l__coffin_display_y_dim=\dimen133
) (/usr/share/texlive/texmf-dist/tex/latex/l3kernel/l3color.sty
Package: l3color 2012/08/29 v4156 L3 Experimental color support
) (/usr/share/texlive/texmf-dist/tex/latex/l3kernel/l3luatex.sty
Package: l3luatex 2013/07/28 v4581 L3 Experimental LuaTeX-specific functions
\g__cctab_allocate_int=\count130
\g__cctab_stack_int=\count131
) (/usr/share/texlive/texmf-dist/tex/latex/l3kernel/l3candidates.sty
Package: l3candidates 2014/01/06 v4643 L3 Experimental additions to l3kernel
\l__box_top_dim=\dimen134
\l__box_bottom_dim=\dimen135
\l__box_left_dim=\dimen136
\l__box_right_dim=\dimen137
\l__box_top_new_dim=\dimen138
\l__box_bottom_new_dim=\dimen139
\l__box_left_new_dim=\dimen140
\l__box_right_new_dim=\dimen141
\l__box_internal_box=\box42
\l__coffin_bounding_shift_dim=\dimen142
\l__coffin_left_corner_dim=\dimen143
\l__coffin_right_corner_dim=\dimen144
\l__coffin_bottom_corner_dim=\dimen145
\l__coffin_top_corner_dim=\dimen146
\l__coffin_scaled_total_height_dim=\dimen147
\l__coffin_scaled_width_dim=\dimen148
))
\mhchem@ce@result=\toks24
\mhchem@ce@part=\toks25
\mhchem@arrow@deployType=\toks26
\mhchem@arrow@params=\toks27
\mhchem@arrowlength@pgf=\skip57
\mhchem@bondwidth=\skip58
\mhchem@bondheight=\skip59
\mhchem@smallbondwidth@tmpA=\skip60
\mhchem@smallbondwidth@tmpB=\skip61
\mhchem@smallbondwidth=\skip62
\__mhchem_option_version_int=\count132
\mhchem@prepostscript@tmp@i=\skip63
\mhchem@prepostscript@tmp@ii=\skip64
\mhchem@mathbox@tmp@i=\skip65
\mhchem@mathbox@tmp@ii=\skip66
\mhchem@minispace@tmp=\skip67
\mhchem@option@minussidebearingleft=\skip68
\mhchem@option@minussidebearingright=\skip69
) (/usr/share/texlive/texmf-dist/tex/latex/tools/multicol.sty
Package: multicol 2011/06/27 v1.7a multicolumn formatting (FMi)
\c@tracingmulticols=\count133
\mult@box=\box43
\multicol@leftmargin=\dimen149
\c@unbalance=\count134
\c@collectmore=\count135
\doublecol@number=\count136
\multicoltolerance=\count137
\multicolpretolerance=\count138
\full@width=\dimen150
\page@free=\dimen151
\premulticols=\dimen152
\postmulticols=\dimen153
\multicolsep=\skip70
\multicolbaselineskip=\skip71
\partial@page=\box44
\last@line=\box45
\mult@rightbox=\box46
\mult@grightbox=\box47
\mult@gfirstbox=\box48
\mult@firstbox=\box49
\@tempa=\box50
\@tempa=\box51
\@tempa=\box52
\@tempa=\box53
\@tempa=\box54
\@tempa=\box55
\@tempa=\box56
\@tempa=\box57
\@tempa=\box58
\@tempa=\box59
\@tempa=\box60
\@tempa=\box61
\@tempa=\box62
\@tempa=\box63
\@tempa=\box64
\@tempa=\box65
\@tempa=\box66
\c@columnbadness=\count139
\c@finalcolumnbadness=\count140
\last@try=\dimen154
\multicolovershoot=\dimen155
\multicolundershoot=\dimen156
\mult@nat@firstbox=\box67
\colbreak@box=\box68
\multicol@sort@counter=\count141
) (/usr/share/texlive/texmf-dist/tex/latex/base/makeidx.sty
Package: makeidx 2000/03/29 v1.0m Standard LaTeX package
) (/usr/share/texlive/texmf-dist/tex/latex/graphics/graphicx.sty
Package: graphicx 1999/02/16 v1.0f Enhanced LaTeX Graphics (DPC,SPQR)
\Gin@req@height=\dimen157
\Gin@req@width=\dimen158
) (/usr/share/texlive/texmf-dist/tex/latex/fancyhdr/fancyhdr.sty
\fancy@headwidth=\skip72
\f@ncyO@elh=\skip73
\f@ncyO@erh=\skip74
\f@ncyO@olh=\skip75
\f@ncyO@orh=\skip76
\f@ncyO@elf=\skip77
\f@ncyO@erf=\skip78
\f@ncyO@olf=\skip79
\f@ncyO@orf=\skip80
) (/usr/share/texlive/texmf-dist/tex/latex/geometry/geometry.sty
Package: geometry 2010/09/12 v5.6 Page Geometry
(/usr/share/texlive/texmf-dist/tex/generic/oberdiek/ifvtex.sty
Package: ifvtex 2010/03/01 v1.5 Detect VTeX and its facilities (HO)
Package ifvtex Info: VTeX not detected.
) (/usr/share/texlive/texmf-dist/tex/generic/ifxetex/ifxetex.sty
Package: ifxetex 2010/09/12 v0.6 Provides ifxetex conditional
)
\Gm@cnth=\count142
\Gm@cntv=\count143
\c@Gm@tempcnt=\count144
\Gm@bindingoffset=\dimen159
\Gm@wd@mp=\dimen160
\Gm@odd@mp=\dimen161
\Gm@even@mp=\dimen162
\Gm@layoutwidth=\dimen163
\Gm@layoutheight=\dimen164
\Gm@layouthoffset=\dimen165
\Gm@layoutvoffset=\dimen166
\Gm@dimlist=\toks28
) (/usr/share/texlive/texmf-dist/tex/latex/hyperref/hyperref.sty
Package: hyperref 2012/11/06 v6.83m Hypertext links for LaTeX
(/usr/share/texlive/texmf-dist/tex/generic/oberdiek/hobsub-hyperref.sty
Package: hobsub-hyperref 2012/05/28 v1.13 Bundle oberdiek, subset hyperref (HO)

(/usr/share/texlive/texmf-dist/tex/generic/oberdiek/hobsub-generic.sty
Package: hobsub-generic 2012/05/28 v1.13 Bundle oberdiek, subset generic (HO)
Package: hobsub 2012/05/28 v1.13 Construct package bundles (HO)
Package hobsub Info: Skipping package `infwarerr' (already loaded).
Package hobsub Info: Skipping package `ltxcmds' (already loaded).
Package hobsub Info: Skipping package `ifluatex' (already loaded).
Package hobsub Info: Skipping package `ifvtex' (already loaded).
Package: intcalc 2007/09/27 v1.1 Expandable calculations with integers (HO)
Package hobsub Info: Skipping package `ifpdf' (already loaded).
Package: etexcmds 2011/02/16 v1.5 Avoid name clashes with e-TeX commands (HO)
Package etexcmds Info: Could not find \expanded.
(etexcmds)             That can mean that you are not using pdfTeX 1.50 or
(etexcmds)             that some package has redefined \expanded.
(etexcmds)             In the latter case, load this package earlier.
Package: kvsetkeys 2012/04/25 v1.16 Key value parser (HO)
Package: kvdefinekeys 2011/04/07 v1.3 Define keys (HO)
Package hobsub Info: Skipping package `pdftexcmds' (already loaded).
Package: pdfescape 2011/11/25 v1.13 Implements pdfTeX's escape features (HO)
Package: bigintcalc 2012/04/08 v1.3 Expandable calculations on big integers (HO
)
Package: bitset 2011/01/30 v1.1 Handle bit-vector datatype (HO)
Package: uniquecounter 2011/01/30 v1.2 Provide unlimited unique counter (HO)
)
Package hobsub Info: Skipping package `hobsub' (already loaded).
Package: letltxmacro 2010/09/02 v1.4 Let assignment for LaTeX macros (HO)
Package: hopatch 2012/05/28 v1.2 Wrapper for package hooks (HO)
Package: xcolor-patch 2011/01/30 xcolor patch
Package: atveryend 2011/06/30 v1.8 Hooks at the very end of document (HO)
Package atveryend Info: \enddocument detected (standard20110627).
Package: atbegshi 2011/10/05 v1.16 At begin shipout hook (HO)
Package: refcount 2011/10/16 v3.4 Data extraction from label references (HO)
Package: hycolor 2011/01/30 v1.7 Color options for hyperref/bookmark (HO)
) (/usr/share/texlive/texmf-dist/tex/latex/oberdiek/auxhook.sty
Package: auxhook 2011/03/04 v1.3 Hooks for auxiliary files (HO)
) (/usr/share/texlive/texmf-dist/tex/latex/oberdiek/kvoptions.sty
Package: kvoptions 2011/06/30 v3.11 Key value format for package options (HO)
)
\@linkdim=\dimen167
\Hy@linkcounter=\count145
\Hy@pagecounter=\count146
(/usr/share/texlive/texmf-dist/tex/latex/hyperref/pd1enc.def
File: pd1enc.def 2012/11/06 v6.83m Hyperref: PDFDocEncoding definition (HO)
)
\Hy@SavedSpaceFactor=\count147
(/usr/share/texlive/texmf-dist/tex/latex/latexconfig/hyperref.cfg
File: hyperref.cfg 2002/06/06 v1.2 hyperref configuration of TeXLive
)
Package hyperref Info: Option `hyperfootnotes' set `false' on input line 4319.
Package hyperref Info: Option `bookmarks' set `true' on input line 4319.
Package hyperref Info: Option `linktocpage' set `false' on input line 4319.
Package hyperref Info: Option `pdfdisplaydoctitle' set `true' on input line 431
9.
Package hyperref Info: Option `pdfpagelabels' set `true' on input line 4319.
Package hyperref Info: Option `bookmarksopen' set `true' on input line 4319.
Package hyperref Info: Option `colorlinks' set `true' on input line 4319.
Package hyperref Info: Hyper figures OFF on input line 4443.
Package hyperref Info: Link nesting OFF on input line 4448.
Package hyperref Info: Hyper index ON on input line 4451.
Package hyperref Info: Plain pages OFF on input line 4458.
Package hyperref Info: Backreferencing OFF on input line 4463.
Package hyperref Info: Implicit mode ON; LaTeX internals redefined.
Package hyperref Info: Bookmarks ON on input line 4688.
\c@Hy@tempcnt=\count148
(/usr/share/texlive/texmf-dist/tex/latex/url/url.sty
\Urlmuskip=\muskip17
Package: url 2013/09/16  ver 3.4  Verb mode for urls, etc.
)
LaTeX Info: Redefining \url on input line 5041.
\XeTeXLinkMargin=\dimen168
\Fld@menulength=\count149
\Field@Width=\dimen169
\Fld@charsize=\dimen170
Package hyperref Info: Hyper figures OFF on input line 6295.
Package hyperref Info: Link nesting OFF on input line 6300.
Package hyperref Info: Hyper index ON on input line 6303.
Package hyperref Info: backreferencing OFF on input line 6310.
Package hyperref Info: Link coloring ON on input line 6313.
Package hyperref Info: Link coloring with OCG OFF on input line 6320.
Package hyperref Info: PDF/A mode OFF on input line 6325.
LaTeX Info: Redefining \ref on input line 6365.
LaTeX Info: Redefining \pageref on input line 6369.
\Hy@abspage=\count150
\c@Item=\count151
)

Package hyperref Message: Driver: hpdftex.

(/usr/share/texlive/texmf-dist/tex/latex/hyperref/hpdftex.def
File: hpdftex.def 2012/11/06 v6.83m Hyperref driver for pdfTeX
\Fld@listcount=\count152
\c@bookmark@seq@number=\count153
(/usr/share/texlive/texmf-dist/tex/latex/oberdiek/rerunfilecheck.sty
Package: rerunfilecheck 2011/04/15 v1.7 Rerun checks for auxiliary files (HO)
Package uniquecounter Info: New unique counter `rerunfilecheck' on input line 2
82.
)
\Hy@SectionHShift=\skip81
)
\TmpLen=\skip82
\@indexfile=\write3
\openout3 = `50880-t.idx'.

Writing index file 50880-t.idx
\c@ChapNo=\count154
(./50880-t.aux)
\openout1 = `50880-t.aux'.

LaTeX Font Info:    Checking defaults for OML/cmm/m/it on input line 590.
LaTeX Font Info:    ... okay on input line 590.
LaTeX Font Info:    Checking defaults for T1/cmr/m/n on input line 590.
LaTeX Font Info:    ... okay on input line 590.
LaTeX Font Info:    Checking defaults for OT1/cmr/m/n on input line 590.
LaTeX Font Info:    ... okay on input line 590.
LaTeX Font Info:    Checking defaults for OMS/cmsy/m/n on input line 590.
LaTeX Font Info:    ... okay on input line 590.
LaTeX Font Info:    Checking defaults for OMX/cmex/m/n on input line 590.
LaTeX Font Info:    ... okay on input line 590.
LaTeX Font Info:    Checking defaults for U/cmr/m/n on input line 590.
LaTeX Font Info:    ... okay on input line 590.
LaTeX Font Info:    Checking defaults for PD1/pdf/m/n on input line 590.
LaTeX Font Info:    ... okay on input line 590.
(/usr/share/texlive/texmf-dist/tex/context/base/supp-pdf.mkii
[Loading MPS to PDF converter (version 2006.09.02).]
\scratchcounter=\count155
\scratchdimen=\dimen171
\scratchbox=\box69
\nofMPsegments=\count156
\nofMParguments=\count157
\everyMPshowfont=\toks29
\MPscratchCnt=\count158
\MPscratchDim=\dimen172
\MPnumerator=\count159
\makeMPintoPDFobject=\count160
\everyMPtoPDFconversion=\toks30
)
*geometry* driver: auto-detecting
*geometry* detected driver: pdftex
*geometry* verbose mode - [ preamble ] result:
* driver: pdftex
* paper: <default>
* layout: <same size as paper>
* layoutoffset:(h,v)=(0.0pt,0.0pt)
* hratio: 1:1
* modes: includehead includefoot twoside 
* h-part:(L,W,R)=(9.03374pt, 325.215pt, 9.03375pt)
* v-part:(T,H,B)=(0.54498pt, 486.46004pt, 0.81747pt)
* \paperwidth=343.28249pt
* \paperheight=487.8225pt
* \textwidth=325.215pt
* \textheight=424.58624pt
* \oddsidemargin=-63.23625pt
* \evensidemargin=-63.23624pt
* \topmargin=-71.725pt
* \headheight=12.0pt
* \headsep=19.8738pt
* \topskip=12.0pt
* \footskip=30.0pt
* \marginparwidth=98.0pt
* \marginparsep=7.0pt
* \columnsep=10.0pt
* \skip\footins=10.8pt plus 4.0pt minus 2.0pt
* \hoffset=0.0pt
* \voffset=0.0pt
* \mag=1000
* \@twocolumnfalse
* \@twosidetrue
* \@mparswitchtrue
* \@reversemarginfalse
* (1in=72.27pt=25.4mm, 1cm=28.453pt)

\AtBeginShipoutBox=\box70
(/usr/share/texlive/texmf-dist/tex/latex/graphics/color.sty
Package: color 2005/11/14 v1.0j Standard LaTeX Color (DPC)
LaTeX Info: Redefining \color on input line 76.
(/usr/share/texlive/texmf-dist/tex/latex/latexconfig/color.cfg
File: color.cfg 2007/01/18 v1.5 color configuration of teTeX/TeXLive
)
Package color Info: Driver file: pdftex.def on input line 130.
)
Package hyperref Info: Link coloring ON on input line 590.
(/usr/share/texlive/texmf-dist/tex/latex/hyperref/nameref.sty
Package: nameref 2012/10/27 v2.43 Cross-referencing by name of section
(/usr/share/texlive/texmf-dist/tex/generic/oberdiek/gettitlestring.sty
Package: gettitlestring 2010/12/03 v1.4 Cleanup title references (HO)
)
\c@section@level=\count161
)
LaTeX Info: Redefining \ref on input line 590.
LaTeX Info: Redefining \pageref on input line 590.
LaTeX Info: Redefining \nameref on input line 590.
(./50880-t.out) (./50880-t.out)
\@outlinefile=\write4
\openout4 = `50880-t.out'.


Overfull \hbox (10.53983pt too wide) in paragraph at lines 604--604
[]\OT1/cmtt/m/n/8 to check the laws of the country where you are located before
 using this ebook.[] 
 []

LaTeX Font Info:    Try loading font information for U+msa on input line 620.
(/usr/share/texlive/texmf-dist/tex/latex/amsfonts/umsa.fd
File: umsa.fd 2013/01/14 v3.01 AMS symbols A
)
LaTeX Font Info:    Try loading font information for U+msb on input line 620.
(/usr/share/texlive/texmf-dist/tex/latex/amsfonts/umsb.fd
File: umsb.fd 2013/01/14 v3.01 AMS symbols B
) [1

{/var/lib/texmf/fonts/map/pdftex/updmap/pdftex.map}] [2] <./images/device.pdf, 
id=138, 159.71669pt x 135.8676pt>
File: ./images/device.pdf Graphic file (type pdf)
<use ./images/device.pdf>
Package pdftex.def Info: ./images/device.pdf used on input line 676.
(pdftex.def)             Requested size: 108.405pt x 92.2208pt.
[1


 <./images/device.pdf>] [2

] [3

] [4] [5] [6] (./50880-t.toc [7


])
\tf@toc=\write5
\openout5 = `50880-t.toc'.

[8]
LaTeX Font Info:    Try loading font information for OMS+cmr on input line 983.

(/usr/share/texlive/texmf-dist/tex/latex/base/omscmr.fd
File: omscmr.fd 1999/05/25 v2.5h Standard LaTeX font definitions
)
LaTeX Font Info:    Font shape `OMS/cmr/bx/n' in size <12> not available
(Font)              Font shape `OMS/cmsy/b/n' tried instead on input line 983.
LaTeX Font Info:    Font shape `OMS/cmr/m/n' in size <8> not available
(Font)              Font shape `OMS/cmsy/m/n' tried instead on input line 1016.

LaTeX Font Info:    Font shape `OMS/cmr/m/n' in size <7> not available
(Font)              Font shape `OMS/cmsy/m/n' tried instead on input line 1019.

[1



] [2]
LaTeX Font Info:    Font shape `OMS/cmr/m/n' in size <12> not available
(Font)              Font shape `OMS/cmsy/m/n' tried instead on input line 1083.

[3] [4] [5] [6] [7] [8] [9] [10] [11] [12] [13] [14] [15] [16] <./images/fig1.p
df, id=371, 340.27126pt x 585.18625pt>
File: ./images/fig1.pdf Graphic file (type pdf)
<use ./images/fig1.pdf>
Package pdftex.def Info: ./images/fig1.pdf used on input line 1622.
(pdftex.def)             Requested size: 252.94499pt x 435.0044pt.

LaTeX Warning: Float too large for page by 10.41815pt on input line 1622.

[17] [18 <./images/fig1.pdf>] [19] [20] [21] [22] [23] [24] [25] [26

] [27] [28] [29] [30] [31] [32] [33] [34] [35] [36] [37] [38

] [39] [40] [41] [42] [43] [44] [45

] [46] [47] [48] [49] [50] [51] [52] [53] [54

] [55] [56] [57] [58] [59] [60] <./images/fig2.pdf, id=662, 236.885pt x 263.986
25pt>
File: ./images/fig2.pdf Graphic file (type pdf)
<use ./images/fig2.pdf>
Package pdftex.def Info: ./images/fig2.pdf used on input line 3129.
(pdftex.def)             Requested size: 236.88441pt x 263.9856pt.
[61] [62 <./images/fig2.pdf>] [63] [64] [65] [66] [67]
Overfull \hbox (2.41357pt too wide) in paragraph at lines 3382--3387
\OT1/cmr/m/n/12 Assuming that only the laws of Boyle, Gay-Lussac, and []Avogadr
o
 []

[68] [69] [70] [71] [72] [73] [74] <./images/fig3.pdf, id=821, 242.9075pt x 264
.99pt>
File: ./images/fig3.pdf Graphic file (type pdf)
<use ./images/fig3.pdf>
Package pdftex.def Info: ./images/fig3.pdf used on input line 3665.
(pdftex.def)             Requested size: 242.90689pt x 264.98935pt.
[75] [76 <./images/fig3.pdf>] [77] [78] [79] [80

] [81] [82] [83] [84] [85] [86] [87] [88] [89] [90] [91] [92

] [93] [94] [95] [96] [97] [98] [99] [100] [101] [102

] [103] [104] [105] [106] [107] [108] [109] [110] [111] [112] [113] [114] [115]
[116] [117]
LaTeX Font Info:    Font shape `OMS/cmr/m/n' in size <10> not available
(Font)              Font shape `OMS/cmsy/m/n' tried instead on input line 5246.

[118] [119] [120] [121] [122] [123] [124] [125

] [126] [127] [128] [129] [130] [131] [132] [133] [134] [135] [136] [137] [138]
[139] [140] [141] [142

] [143] [144] [145] [146] [147] [148] [149] [150] [151] [152] [153] [154] [155]
[156] [157] [158] [159

] [160] [161] [162] [163] [164] [165] [166] [167] [168] [169] [170] [171] [172]
[173] [174] [175] [176] [177] [178] [179] [180] [181] [182] [183] [184] [185] [
186] [187] [188] [189]
Overfull \hbox (5.85457pt too wide) in paragraph at lines 7833--7833
[] 
 []

[190] [191] <./images/fig4.pdf, id=1826, 386.44376pt x 370.38374pt>
File: ./images/fig4.pdf Graphic file (type pdf)
<use ./images/fig4.pdf>
Package pdftex.def Info: ./images/fig4.pdf used on input line 7908.
(pdftex.def)             Requested size: 307.14749pt x 294.38585pt.
[192] [193 <./images/fig4.pdf>] [194] [195] [196] [197] [198] [199] [200] [201]
[202] [203] [204] [205] [206] [207] [208] [209] [210] [211] [212

] [213] [214] [215] [216] [217] [218] [219] [220] [221] [222] [223] [224] [225]
[226] [227] [228] [229] [230] [231] [232] [233] [234] [235] [236] [237] [238] [
239] [240] [241] [242] [243] [244] [245] [246] [247] [248] [249] [250] [251] [2
52] [253

] [254] [255] [256] <./images/fig5.pdf, id=2464, 201.75375pt x 254.9525pt>
File: ./images/fig5.pdf Graphic file (type pdf)
<use ./images/fig5.pdf>
Package pdftex.def Info: ./images/fig5.pdf used on input line 10304.
(pdftex.def)             Requested size: 201.75325pt x 254.95186pt.
[257] [258 <./images/fig5.pdf>] [259] [260] [261] [262] [263] [264] [265] [266]
[267] [268] [269] [270] [271] [272

] [273] [274] [275] [276] [277] [278] [279] [280] [281] [282] [283] [284] [285]
[286] [287] [288] [289] [290] [291]
Overfull \hbox (1.07619pt too wide) in paragraph at lines 11644--11650
[]\OMS/cmsy/b/n/12 x\OT1/cmr/bx/n/12 264. []Two In-de-pend-ent Con-stitu-ents i
n Two Phases. 
 []

[292] [293] [294] [295]
Overfull \hbox (5.88911pt too wide) detected at line 11837
[]\OML/cmm/m/it/12 ;  [];  [];  [] \OMS/cmsy/m/n/12 j []\OML/cmm/m/it/12 :
 []

[296] [297] [298] [299] [300] [301] [302] [303] [304] [305] [306] [307] [308] [
309] [310] [311] [312] [313] [314] [315] [316] [317] [318] [319] [320]
Package babel Info: Redefining german shorthand "f
(babel)             in language  on input line 12881.
Package babel Info: Redefining german shorthand "|
(babel)             in language  on input line 12881.
Package babel Info: Redefining german shorthand "~
(babel)             in language  on input line 12881.
Package babel Info: Redefining german shorthand "f
(babel)             in language  on input line 12881.
Package babel Info: Redefining german shorthand "|
(babel)             in language  on input line 12881.
Package babel Info: Redefining german shorthand "~
(babel)             in language  on input line 12881.
[321



]
Package babel Info: Redefining german shorthand "f
(babel)             in language  on input line 12933.
Package babel Info: Redefining german shorthand "|
(babel)             in language  on input line 12933.
Package babel Info: Redefining german shorthand "~
(babel)             in language  on input line 12933.
Package babel Info: Redefining german shorthand "f
(babel)             in language  on input line 12933.
Package babel Info: Redefining german shorthand "|
(babel)             in language  on input line 12933.
Package babel Info: Redefining german shorthand "~
(babel)             in language  on input line 12933.
[322] (./50880-t.ind [323] [324

] [325] [326] [327] [328] [329] [330] [331] [332] [333]) [1


] [2] [3] [4] [5] [6]
Overfull \hbox (6.28976pt too wide) in paragraph at lines 14091--14091
[]\OT1/cmtt/m/n/8 Section 3. Information about the Project Gutenberg Literary A
rchive Foundation[] 
 []

[7] [8]
Package atveryend Info: Empty hook `BeforeClearDocument' on input line 14184.
[9]
Package atveryend Info: Empty hook `AfterLastShipout' on input line 14184.
(./50880-t.aux)
Package atveryend Info: Executing hook `AtVeryEndDocument' on input line 14184.


 *File List*
    book.cls    2007/10/19 v1.4h Standard LaTeX document class
    bk12.clo    2007/10/19 v1.4h Standard LaTeX file (size option)
inputenc.sty    2008/03/30 v1.1d Input encoding file
  latin1.def    2008/03/30 v1.1d Input encoding file
   babel.sty    2013/12/03 3.9h The Babel package
 germanb.ldf    2013/12/13 v2.7 German support for babel (traditional orthograp
hy)
 english.ldf    2012/08/20 v3.3p English support from the babel system
  ifthen.sty    2001/05/26 v1.1c Standard LaTeX ifthen package (DPC)
 amsmath.sty    2013/01/14 v2.14 AMS math features
 amstext.sty    2000/06/29 v2.01
  amsgen.sty    1999/11/30 v2.0
  amsbsy.sty    1999/11/29 v1.2d
  amsopn.sty    1999/12/14 v2.01 operator names
 amssymb.sty    2013/01/14 v3.01 AMS font symbols
amsfonts.sty    2013/01/14 v3.01 Basic AMSFonts support
   array.sty    2008/09/09 v2.4c Tabular extension package (FMi)
   alltt.sty    1997/06/16 v2.0g defines alltt environment
footmisc.sty    2011/06/06 v5.5b a miscellany of footnote facilities
  mhchem.sty    2014/02/01 v3.15 for typesetting chemical formulae
    calc.sty    2007/08/22 v4.3 Infix arithmetic (KKT,FJ)
  twoopt.sty    2008/08/11 v1.5 Definitions with two optional arguments (HO)
  keyval.sty    1999/03/16 v1.13 key=value parser (DPC)
graphics.sty    2009/02/05 v1.0o Standard LaTeX Graphics (DPC,SPQR)
    trig.sty    1999/03/16 v1.09 sin cos tan (DPC)
graphics.cfg    2010/04/23 v1.9 graphics configuration of TeX Live
  pdftex.def    2011/05/27 v0.06d Graphics/color for pdfTeX
infwarerr.sty    2010/04/08 v1.3 Providing info/warning/error messages (HO)
 ltxcmds.sty    2011/11/09 v1.22 LaTeX kernel commands for general use (HO)
pdftexcmds.sty    2011/11/29 v0.20 Utility functions of pdfTeX for LuaTeX (HO)
ifluatex.sty    2010/03/01 v1.3 Provides the ifluatex switch (HO)
   ifpdf.sty    2011/01/30 v2.3 Provides the ifpdf switch (HO)
   expl3.sty    2014/01/07 v4646 L3 Experimental code bundle wrapper
 l3names.sty    2014/01/04 v4640 L3 Namespace for primitives
l3bootstrap.sty    2014/01/04 v4640 L3 Experimental bootstrap code
    etex.sty    1998/03/26 v2.0 eTeX basic definition package (PEB)
l3basics.sty    2014/01/04 v4642 L3 Basic definitions
 l3expan.sty    2014/01/04 v4642 L3 Argument expansion
    l3tl.sty    2013/12/27 v4625 L3 Token lists
   l3seq.sty    2013/12/14 v4623 L3 Sequences and stacks
   l3int.sty    2013/08/02 v4583 L3 Integers
 l3quark.sty    2013/12/14 v4623 L3 Quarks
   l3prg.sty    2014/01/04 v4642 L3 Control structures
 l3clist.sty    2013/07/28 v4581 L3 Comma separated lists
 l3token.sty    2013/08/25 v4587 L3 Experimental token manipulation
  l3prop.sty    2013/12/14 v4623 L3 Property lists
   l3msg.sty    2013/07/28 v4581 L3 Messages
  l3file.sty    2013/10/13 v4596 L3 File and I/O operations
  l3skip.sty    2013/07/28 v4581 L3 Dimensions and skips
  l3keys.sty    2013/12/08 v4614 L3 Experimental key-value interfaces
    l3fp.sty    2014/01/04 v4642 L3 Floating points
   l3box.sty    2013/07/28 v4581 L3 Experimental boxes
l3coffins.sty    2013/12/14 v4624 L3 Coffin code layer
 l3color.sty    2012/08/29 v4156 L3 Experimental color support
l3luatex.sty    2013/07/28 v4581 L3 Experimental LuaTeX-specific functions
l3candidates.sty    2014/01/06 v4643 L3 Experimental additions to l3kernel
multicol.sty    2011/06/27 v1.7a multicolumn formatting (FMi)
 makeidx.sty    2000/03/29 v1.0m Standard LaTeX package
graphicx.sty    1999/02/16 v1.0f Enhanced LaTeX Graphics (DPC,SPQR)
fancyhdr.sty    
geometry.sty    2010/09/12 v5.6 Page Geometry
  ifvtex.sty    2010/03/01 v1.5 Detect VTeX and its facilities (HO)
 ifxetex.sty    2010/09/12 v0.6 Provides ifxetex conditional
hyperref.sty    2012/11/06 v6.83m Hypertext links for LaTeX
hobsub-hyperref.sty    2012/05/28 v1.13 Bundle oberdiek, subset hyperref (HO)
hobsub-generic.sty    2012/05/28 v1.13 Bundle oberdiek, subset generic (HO)
  hobsub.sty    2012/05/28 v1.13 Construct package bundles (HO)
 intcalc.sty    2007/09/27 v1.1 Expandable calculations with integers (HO)
etexcmds.sty    2011/02/16 v1.5 Avoid name clashes with e-TeX commands (HO)
kvsetkeys.sty    2012/04/25 v1.16 Key value parser (HO)
kvdefinekeys.sty    2011/04/07 v1.3 Define keys (HO)
pdfescape.sty    2011/11/25 v1.13 Implements pdfTeX's escape features (HO)
bigintcalc.sty    2012/04/08 v1.3 Expandable calculations on big integers (HO)
  bitset.sty    2011/01/30 v1.1 Handle bit-vector datatype (HO)
uniquecounter.sty    2011/01/30 v1.2 Provide unlimited unique counter (HO)
letltxmacro.sty    2010/09/02 v1.4 Let assignment for LaTeX macros (HO)
 hopatch.sty    2012/05/28 v1.2 Wrapper for package hooks (HO)
xcolor-patch.sty    2011/01/30 xcolor patch
atveryend.sty    2011/06/30 v1.8 Hooks at the very end of document (HO)
atbegshi.sty    2011/10/05 v1.16 At begin shipout hook (HO)
refcount.sty    2011/10/16 v3.4 Data extraction from label references (HO)
 hycolor.sty    2011/01/30 v1.7 Color options for hyperref/bookmark (HO)
 auxhook.sty    2011/03/04 v1.3 Hooks for auxiliary files (HO)
kvoptions.sty    2011/06/30 v3.11 Key value format for package options (HO)
  pd1enc.def    2012/11/06 v6.83m Hyperref: PDFDocEncoding definition (HO)
hyperref.cfg    2002/06/06 v1.2 hyperref configuration of TeXLive
     url.sty    2013/09/16  ver 3.4  Verb mode for urls, etc.
 hpdftex.def    2012/11/06 v6.83m Hyperref driver for pdfTeX
rerunfilecheck.sty    2011/04/15 v1.7 Rerun checks for auxiliary files (HO)
supp-pdf.mkii
   color.sty    2005/11/14 v1.0j Standard LaTeX Color (DPC)
   color.cfg    2007/01/18 v1.5 color configuration of teTeX/TeXLive
 nameref.sty    2012/10/27 v2.43 Cross-referencing by name of section
gettitlestring.sty    2010/12/03 v1.4 Cleanup title references (HO)
 50880-t.out
 50880-t.out
    umsa.fd    2013/01/14 v3.01 AMS symbols A
    umsb.fd    2013/01/14 v3.01 AMS symbols B
./images/device.pdf
  omscmr.fd    1999/05/25 v2.5h Standard LaTeX font definitions
./images/fig1.pdf
./images/fig2.pdf
./images/fig3.pdf
./images/fig4.pdf
./images/fig5.pdf
 50880-t.ind
 ***********

Package atveryend Info: Executing hook `AtEndAfterFileList' on input line 14184
.
Package rerunfilecheck Info: File `50880-t.out' has not changed.
(rerunfilecheck)             Checksum: B36BC6583614C4B13F36B6DFCFD8827A;2043.
Package atveryend Info: Empty hook `AtVeryVeryEnd' on input line 14184.
 ) 
Here is how much of TeX's memory you used:
 14542 strings out of 493304
 235562 string characters out of 6139871
 345829 words of memory out of 5000000
 16775 multiletter control sequences out of 15000+600000
 17567 words of font info for 67 fonts, out of 8000000 for 9000
 957 hyphenation exceptions out of 8191
 38i,20n,46p,10391b,1066s stack positions out of 5000i,500n,10000p,200000b,80000s
</usr/share/texlive/texmf-dist/fonts/type1/public/amsfonts/cm/cmbsy10.pfb></u
sr/share/texlive/texmf-dist/fonts/type1/public/amsfonts/cm/cmbx10.pfb></usr/sha
re/texlive/texmf-dist/fonts/type1/public/amsfonts/cm/cmbx12.pfb></usr/share/tex
live/texmf-dist/fonts/type1/public/amsfonts/cm/cmcsc10.pfb></usr/share/texlive/
texmf-dist/fonts/type1/public/amsfonts/cm/cmex10.pfb></usr/share/texlive/texmf-
dist/fonts/type1/public/amsfonts/cm/cmmi10.pfb></usr/share/texlive/texmf-dist/f
onts/type1/public/amsfonts/cm/cmmi12.pfb></usr/share/texlive/texmf-dist/fonts/t
ype1/public/amsfonts/cm/cmmi6.pfb></usr/share/texlive/texmf-dist/fonts/type1/pu
blic/amsfonts/cm/cmmi7.pfb></usr/share/texlive/texmf-dist/fonts/type1/public/am
sfonts/cm/cmmi8.pfb></usr/share/texlive/texmf-dist/fonts/type1/public/amsfonts/
cm/cmr10.pfb></usr/share/texlive/texmf-dist/fonts/type1/public/amsfonts/cm/cmr1
2.pfb></usr/share/texlive/texmf-dist/fonts/type1/public/amsfonts/cm/cmr17.pfb><
/usr/share/texlive/texmf-dist/fonts/type1/public/amsfonts/cm/cmr6.pfb></usr/sha
re/texlive/texmf-dist/fonts/type1/public/amsfonts/cm/cmr7.pfb></usr/share/texli
ve/texmf-dist/fonts/type1/public/amsfonts/cm/cmr8.pfb></usr/share/texlive/texmf
-dist/fonts/type1/public/amsfonts/cm/cmsy10.pfb></usr/share/texlive/texmf-dist/
fonts/type1/public/amsfonts/cm/cmsy6.pfb></usr/share/texlive/texmf-dist/fonts/t
ype1/public/amsfonts/cm/cmsy7.pfb></usr/share/texlive/texmf-dist/fonts/type1/pu
blic/amsfonts/cm/cmsy8.pfb></usr/share/texlive/texmf-dist/fonts/type1/public/am
sfonts/cm/cmti10.pfb></usr/share/texlive/texmf-dist/fonts/type1/public/amsfonts
/cm/cmti12.pfb></usr/share/texlive/texmf-dist/fonts/type1/public/amsfonts/cm/cm
tt10.pfb></usr/share/texlive/texmf-dist/fonts/type1/public/amsfonts/cm/cmtt8.pf
b></usr/share/texlive/texmf-dist/fonts/type1/public/amsfonts/symbols/msam10.pfb
>
Output written on 50880-t.pdf (352 pages, 1106477 bytes).
PDF statistics:
 4010 PDF objects out of 4296 (max. 8388607)
 3564 compressed objects within 36 object streams
 1011 named destinations out of 1200 (max. 500000)
 263 words of extra memory for PDF output out of 10000 (max. 10000000)

