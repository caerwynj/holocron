%The Project Gutenberg EBook of Relativity: The Special and General Theory
%by Albert Einstein
%(#1 in our series by Albert Einstein)
%
%Note: 58 image files are part of this eBook.  They include tables,
%equations and figures that could not be represented well as plain text.
%
%Copyright laws are changing all over the world. Be sure to check the
%copyright laws for your country before downloading or redistributing
%this or any other Project Gutenberg eBook.
%
%This header should be the first thing seen when viewing this Project
%Gutenberg file.  Please do not remove it.  Do not change or edit the
%header without written permission.
%
%Please read the ``legal small print," and other information about the
%eBook and Project Gutenberg at the bottom of this file.  Included is
%important information about your specific rights and restrictions in
%how the file may be used.  You can also find out about how to make a
%donation to Project Gutenberg, and how to get involved.
%
%
%**Welcome To The World of Free Plain Vanilla Electronic Texts**
%
%**eBooks Readable By Both Humans and By Computers, Since 1971**
%
%*****These eBooks Were Prepared By Thousands of Volunteers!*****
%
%
%Title: Relativity: The Special and General Theory
%
%Author: Albert Einstein
%
%Release Date: February, 2004  [EBook #5001]
%[Yes, we are more than one year ahead of schedule]
%[This file was first posted on April 1, 2002]
%
%Edition: 10
%
%Language: English
%
%Character set encoding: ASCII
%
%*** START OF THE PROJECT GUTENBERG EBOOK, RELATIVITY ***
%
%
%
%
%ALBERT EINSTEIN REFERENCE ARCHIVE
%
%RELATIVITY: THE SPECIAL AND GENERAL THEORY
%
%BY ALBERT EINSTEIN
%
%
%Written: 1916 (this revised edition: 1924)
%Source: Relativity: The Special and General Theory (1920)
%Publisher: Methuen & Co Ltd
%First Published: December, 1916
%Translated: Robert W. Lawson (Authorised translation)
%Transcription/Markup: Brian Basgen <brian@marxists.org>
%Transcription to text: Gregory B. Newby <gbnewby@petascale.org>
%Typeset into LaTeX: Robert Bradshaw <rwb00@myrealbox.com>
%Copyleft: Einstein Reference Archive (marxists.org) 1999, 2002.
%Permission is granted to copy and/or distribute this document under
%the terms of the GNU Free Documentation License (end of this file)
%The Einstein Reference Archive is online at:
%http://www.marxists.org/reference/archive/einstein/index.htm



%\input gutenberg-simple.tex
\documentclass[11pt]{report}

\renewcommand{\labelenumi}{\alph{enumi}.}
\renewcommand{\thesection}{\alph{section}.}
%\renewcommand{\thesection}{}


\begin{document}

%\gtitle{Relativity: The Special and General Theory}
\title{Relativity: The Special and General Theory}
\date{1916}

%\gauthor{Albert Einstein}

\author{Albert Einstein}

%\frontmatter 

\maketitle

\tableofcontents

\newpage

%CONTENTS

%Preface

%Part I: The Special Theory of Relativity

%01. Physical Meaning of Geometrical Propositions
%02. The System of Co-ordinates
%03. Space and Time in Classical Mechanics
%04. The Galileian System of Co-ordinates
%05. The Principle of Relativity (in the Restricted Sense)
%06. The Theorem of the Addition of Velocities employed in
%Classical Mechanics
%07. The Apparent Incompatability of the Law of Propagation of
%Light with the Principle of Relativity
%08. On the Idea of Time in Physics
%09. The Relativity of Simultaneity
%10. On the Relativity of the Conception of Distance
%11. The Lorentz Transformation
%12. The Behaviour of Measuring-Rods and Clocks in Motion
%13. Theorem of the Addition of Velocities. The Experiment of Fizeau
%14. The Hueristic Value of the Theory of Relativity
%15. General Results of the Theory
%16. Expereince and the Special Theory of Relativity
%17. Minkowski's Four-dimensial Space


%Part II: The General Theory of Relativity

%18. Special and General Principle of Relativity
%19. The Gravitational Field
%20. The Equality of Inertial and Gravitational Mass as an Argument
%for the General Postulate of Relativity
%21. In What Respects are the Foundations of Classical Mechanics
%and of the Special Theory of Relativity Unsatisfactory?
%22. A Few Inferences from the General Principle of Relativity
%23. Behaviour of Clocks and Measuring-Rods on a Rotating Body of
%Reference
%24. Euclidean and non-Euclidean Continuum
%25. Gaussian Co-ordinates
%26. The Space-Time Continuum of the Speical Theory of Relativity
%Considered as a Euclidean Continuum
%27. The Space-Time Continuum of the General Theory of Relativity
%is Not a Euclidean Continuum
%28. Exact Formulation of the General Principle of Relativity
%29. The Solution of the Problem of Gravitation on the Basis of the
%General Principle of Relativity


%Part III: Considerations on the Universe as a Whole

%30. Cosmological Difficulties of Netwon's Theory
%31. The Possibility of a ``Finite" and yet ``Unbounded" Universe
%32. The Structure of Space According to the General Theory of
%Relativity


%Appendices:

%01. Simple Derivation of the Lorentz Transformation (sup. ch. 11)
%02. Minkowski's Four-Dimensional Space ("World") (sup. ch 17)
%03. The Experimental Confirmation of the General Theory of Relativity
%04. The Structure of Space According to the General Theory of
%Relativity (sup. ch 32)
%05. Relativity and the Problem of Space

%Note: The fifth Appendix was added by Einstein at the time of the
%fifteenth re-printing of this book; and as a result is still under
%copyright restrictions so cannot be added without the permission of
%the publisher.


%\chapter{Preface}

 (December, 1916)
 
 ~

The present book is intended, as far as possible, to give an exact
insight into the theory of Relativity to those readers who, from a
general scientific and philosophical point of view, are interested in
the theory, but who are not conversant with the mathematical apparatus
of theoretical physics. The work presumes a standard of education
corresponding to that of a university matriculation examination, and,
despite the shortness of the book, a fair amount of patience and force
of will on the part of the reader. The author has spared himself no
pains in his endeavour to present the main ideas in the simplest and
most intelligible form, and on the whole, in the sequence and
connection in which they actually originated. In the interest of
clearness, it appeared to me inevitable that I should repeat myself
frequently, without paying the slightest attention to the elegance of
the presentation. I adhered scrupulously to the precept of that
brilliant theoretical physicist L. Boltzmann, according to whom
matters of elegance ought to be left to the tailor and to the cobbler.
I make no pretence of having withheld from the reader difficulties
which are inherent to the subject. On the other hand, I have purposely
treated the empirical physical foundations of the theory in a
"step-motherly" fashion, so that readers unfamiliar with physics may
not feel like the wanderer who was unable to see the forest for the
trees. May the book bring some one a few happy hours of suggestive
thought!

~

December, 1916

A. EINSTEIN

%\mainmatter 

\part{The Special Theory of Relativity}

\chapter{Physical Meaning of Geometrical Propositions}

In your schooldays most of you who read this book made acquaintance
with the noble building of Euclid's geometry, and you remember---perhaps
with more respect than love---the magnificent structure, on
the lofty staircase of which you were chased about for uncounted hours
by conscientious teachers. By reason of our past experience, you would
certainly regard everyone with disdain who should pronounce even the
most out-of-the-way proposition of this science to be untrue. But
perhaps this feeling of proud certainty would leave you immediately if
some one were to ask you: ``What, then, do you mean by the assertion
that these propositions are true?" Let us proceed to give this
question a little consideration.

Geometry sets out form certain conceptions such as ``plane,'' ``point,"
and ``straight line," with which we are able to associate more or less
definite ideas, and from certain simple propositions (axioms) which,
in virtue of these ideas, we are inclined to accept as ``true." Then,
on the basis of a logical process, the justification of which we feel
ourselves compelled to admit, all remaining propositions are shown to
follow from those axioms, {\it i.e.} they are proven. A proposition is then
correct (``true") when it has been derived in the recognised manner
from the axioms. The question of ``truth" of the individual geometrical
propositions is thus reduced to one of the ``truth" of the axioms. Now
it has long been known that the last question is not only unanswerable
by the methods of geometry, but that it is in itself entirely without
meaning. We cannot ask whether it is true that only one straight line
goes through two points. We can only say that Euclidean geometry deals
with things called ``straight lines," to each of which is ascribed the
property of being uniquely determined by two points situated on it.
The concept ``true" does not tally with the assertions of pure
geometry, because by the word ``true" we are eventually in the habit of
designating always the correspondence with a ``real" object; geometry,
however, is not concerned with the relation of the ideas involved in
it to objects of experience, but only with the logical connection of
these ideas among themselves.

It is not difficult to understand why, in spite of this, we feel
constrained to call the propositions of geometry ``true." Geometrical
ideas correspond to more or less exact objects in nature, and these
last are undoubtedly the exclusive cause of the genesis of those
ideas. Geometry ought to refrain from such a course, in order to give
to its structure the largest possible logical unity. The practice, for
example, of seeing in a ``distance" two marked positions on a
practically rigid body is something which is lodged deeply in our
habit of thought. We are accustomed further to regard three points as
being situated on a straight line, if their apparent positions can be
made to coincide for observation with one eye, under suitable choice
of our place of observation.

If, in pursuance of our habit of thought, we now supplement the
propositions of Euclidean geometry by the single proposition that two
points on a practically rigid body always correspond to the same
distance (line-interval), independently of any changes in position to
which we may subject the body, the propositions of Euclidean geometry
then resolve themselves into propositions on the possible relative
position of practically rigid bodies.\footnotemark\ Geometry which has been
supplemented in this way is then to be treated as a branch of physics.
We can now legitimately ask as to the ``truth" of geometrical
propositions interpreted in this way, since we are justified in asking
whether these propositions are satisfied for those real things we have
associated with the geometrical ideas. In less exact terms we can
express this by saying that by the ``truth" of a geometrical
proposition in this sense we understand its validity for a
construction with rule and compasses.

Of course the conviction of the ``truth" of geometrical propositions in
this sense is founded exclusively on rather incomplete experience. For
the present we shall assume the ``truth" of the geometrical
propositions, then at a later stage (in the general theory of
relativity) we shall see that this ``truth" is limited, and we shall
consider the extent of its limitation.


%  Notes

\footnotetext[1]{It follows that a natural object is associated also with a
straight line. Three points A, B and C on a rigid body thus lie in a
straight line when the points A and C being given, B is chosen such
that the sum of the distances AB and BC is as short as possible. This
incomplete suggestion will suffice for the present purpose.}



\chapter{The System of Co-ordinates}


On the basis of the physical interpretation of distance which has been
indicated, we are also in a position to establish the distance between
two points on a rigid body by means of measurements. For this purpose
we require a ``distance'' (rod S) which is to be used once and for
all, and which we employ as a standard measure. If, now, A and B are
two points on a rigid body, we can construct the line joining them
according to the rules of geometry; then, starting from A, we can
mark off the distance S time after time until we reach B. The number
of these operations required is the numerical measure of the distance
AB. This is the basis of all measurement of length.\footnotemark

Every description of the scene of an event or of the position of an
object in space is based on the specification of the point on a rigid
body (body of reference) with which that event or object coincides.
This applies not only to scientific description, but also to everyday
life. If I analyse the place specification ``Times Square, New York,"\footnotemark
I arrive at the following result. The earth is the rigid body
to which the specification of place refers; ``Times Square, New York,"
is a well-defined point, to which a name has been assigned, and with
which the event coincides in space.\footnotemark

This primitive method of place specification deals only with places on
the surface of rigid bodies, and is dependent on the existence of
points on this surface which are distinguishable from each other. But
we can free ourselves from both of these limitations without altering
the nature of our specification of position. If, for instance, a cloud
is hovering over Times Square, then we can determine its position
relative to the surface of the earth by erecting a pole
perpendicularly on the Square, so that it reaches the cloud. The
length of the pole measured with the standard measuring-rod, combined
with the specification of the position of the foot of the pole,
supplies us with a complete place specification. On the basis of this
illustration, we are able to see the manner in which a refinement of
the conception of position has been developed.

\begin{enumerate}
\item We imagine the rigid body, to which the place specification is
referred, supplemented in such a manner that the object whose position
we require is reached by. the completed rigid body.

\item In locating the position of the object, we make use of a number
(here the length of the pole measured with the measuring-rod) instead
of designated points of reference.

\item We speak of the height of the cloud even when the pole which
reaches the cloud has not been erected. By means of optical
observations of the cloud from different positions on the ground, and
taking into account the properties of the propagation of light, we
determine the length of the pole we should have required in order to
reach the cloud.
\end{enumerate}

From this consideration we see that it will be advantageous if, in the
description of position, it should be possible by means of numerical
measures to make ourselves independent of the existence of marked
positions (possessing names) on the rigid body of reference. In the
physics of measurement this is attained by the application of the
Cartesian system of co-ordinates.

This consists of three plane surfaces perpendicular to each other and
rigidly attached to a rigid body. Referred to a system of
co-ordinates, the scene of any event will be determined (for the main
part) by the specification of the lengths of the three perpendiculars
or co-ordinates $(x, y, z)$ which can be dropped from the scene of the
event to those three plane surfaces. The lengths of these three
perpendiculars can be determined by a series of manipulations with
rigid measuring-rods performed according to the rules and methods laid
down by Euclidean geometry.

In practice, the rigid surfaces which constitute the system of
co-ordinates are generally not available; furthermore, the magnitudes
of the co-ordinates are not actually determined by constructions with
rigid rods, but by indirect means. If the results of physics and
astronomy are to maintain their clearness, the physical meaning of
specifications of position must always be sought in accordance with
the above considerations. \footnotemark

We thus obtain the following result: Every description of events in
space involves the use of a rigid body to which such events have to be
referred. The resulting relationship takes for granted that the laws
of Euclidean geometry hold for ``distances;" the ``distance" being
represented physically by means of the convention of two marks on a
rigid body.


%  Notes

\footnotetext[1]{Here we have assumed that there is nothing left over {\it i.e.} that
the measurement gives a whole number. This difficulty is got over by
the use of divided measuring-rods, the introduction of which does not
demand any fundamentally new method.}

\footnotetext[2]{Einstein used ``Potsdamer Platz, Berlin" in the original text.
In the authorised translation this was supplemented with ``Tranfalgar
Square, London". We have changed this to ``Times Square, New York", as
this is the most well known/identifiable location to English speakers
in the present day. [Note by the janitor.]}

\footnotetext[3]{It is not necessary here to investigate further the significance
of the expression ``coincidence in space." This conception is
sufficiently obvious to ensure that differences of opinion are
scarcely likely to arise as to its applicability in practice.}

\footnotetext[4]{A refinement and modification of these views does not become
necessary until we come to deal with the general theory of relativity,
treated in the second part of this book.}



\chapter{Space and Time in Classical Mechanics}


The purpose of mechanics is to describe how bodies change their
position in space with ``time." I should load my conscience with grave
sins against the sacred spirit of lucidity were I to formulate the
aims of mechanics in this way, without serious reflection and detailed
explanations. Let us proceed to disclose these sins.

It is not clear what is to be understood here by ``position" and
"space." I stand at the window of a railway carriage which is
travelling uniformly, and drop a stone on the embankment, without
throwing it. Then, disregarding the influence of the air resistance, I
see the stone descend in a straight line. A pedestrian who observes
the misdeed from the footpath notices that the stone falls to earth in
a parabolic curve. I now ask: Do the ``positions" traversed by the
stone lie ``in reality" on a straight line or on a parabola? Moreover,
what is meant here by motion ``in space"? From the considerations of
the previous section the answer is self-evident. In the first place we
entirely shun the vague word ``space," of which, we must honestly
acknowledge, we cannot form the slightest conception, and we replace
it by ``motion relative to a practically rigid body of reference." The
positions relative to the body of reference (railway carriage or
embankment) have already been defined in detail in the preceding
section. If instead of ``body of reference'' we insert ``system of
co-ordinates," which is a useful idea for mathematical description, we
are in a position to say: The stone traverses a straight line
relative to a system of co-ordinates rigidly attached to the carriage,
but relative to a system of co-ordinates rigidly attached to the
ground (embankment) it describes a parabola. With the aid of this
example it is clearly seen that there is no such thing as an
independently existing trajectory (lit. ``path-curve"\footnotemark), but only
a trajectory relative to a particular body of reference.

In order to have a complete description of the motion, we must specify
how the body alters its position with time; {\it i.e.} for every point on
the trajectory it must be stated at what time the body is situated
there. These data must be supplemented by such a definition of time
that, in virtue of this definition, these time-values can be regarded
essentially as magnitudes (results of measurements) capable of
observation. If we take our stand on the ground of classical
mechanics, we can satisfy this requirement for our illustration in the
following manner. We imagine two clocks of identical construction;
the man at the railway-carriage window is holding one of them, and the
man on the footpath the other. Each of the observers determines the
position on his own reference-body occupied by the stone at each tick
of the clock he is holding in his hand. In this connection we have not
taken account of the inaccuracy involved by the finiteness of the
velocity of propagation of light. With this and with a second
difficulty prevailing here we shall have to deal in detail later.


%  Notes

\footnotetext[1]{That is, a curve along which the body moves.}


\chapter{The Galilean System of Co-ordinates}


As is well known, the fundamental law of the mechanics of
Galilei-Newton, which is known as the law of inertia, can be stated
thus: A body removed sufficiently far from other bodies continues in a
state of rest or of uniform motion in a straight line. This law not
only says something about the motion of the bodies, but it also
indicates the reference-bodies or systems of coordinates, permissible
in mechanics, which can be used in mechanical description. The visible
fixed stars are bodies for which the law of inertia certainly holds to
a high degree of approximation. Now if we use a system of co-ordinates
which is rigidly attached to the earth, then, relative to this system,
every fixed star describes a circle of immense radius in the course of
an astronomical day, a result which is opposed to the statement of the
law of inertia. So that if we adhere to this law we must refer these
motions only to systems of coordinates relative to which the fixed
stars do not move in a circle. A system of co-ordinates of which the
state of motion is such that the law of inertia holds relative to it
is called a ``Galileian system of co-ordinates." The laws of the
mechanics of Galflei-Newton can be regarded as valid only for a
Galileian system of co-ordinates.


\chapter{The Principle of Relativity in the Restricted Sense}

In order to attain the greatest possible clearness, let us return to
our example of the railway carriage supposed to be travelling
uniformly. We call its motion a uniform translation (``uniform" because
it is of constant velocity and direction, ``translation'' because
although the carriage changes its position relative to the embankment
yet it does not rotate in so doing). Let us imagine a raven flying
through the air in such a manner that its motion, as observed from the
embankment, is uniform and in a straight line. If we were to observe
the flying raven from the moving railway carriage. we should find that
the motion of the raven would be one of different velocity and
direction, but that it would still be uniform and in a straight line.
Expressed in an abstract manner we may say: If a mass m is moving
uniformly in a straight line with respect to a co-ordinate system $K$,
then it will also be moving uniformly and in a straight line relative
to a second co-ordinate system $K'$ provided that the latter is
executing a uniform translatory motion with respect to $K$. In
accordance with the discussion contained in the preceding section, it
follows that:

If $K$ is a Galileian co-ordinate system. then every other co-ordinate
system $K'$ is a Galileian one, when, in relation to $K$, it is in a
condition of uniform motion of translation. Relative to $K'$ the
mechanical laws of Galilei-Newton hold good exactly as they do with
respect to $K$.

We advance a step farther in our generalisation when we express the
tenet thus: If, relative to $K$, $K'$ is a uniformly moving co-ordinate
system devoid of rotation, then natural phenomena run their course
with respect to $K'$ according to exactly the same general laws as with
respect to $K$. This statement is called the \emph{principle of relativity} (in
the restricted sense).

As long as one was convinced that all natural phenomena were capable
of representation with the help of classical mechanics, there was no
need to doubt the validity of this principle of relativity. But in
view of the more recent development of electrodynamics and optics it
became more and more evident that classical mechanics affords an
insufficient foundation for the physical description of all natural
phenomena. At this juncture the question of the validity of the
principle of relativity became ripe for discussion, and it did not
appear impossible that the answer to this question might be in the
negative.

Nevertheless, there are two general facts which at the outset speak
very much in favour of the validity of the principle of relativity.
Even though classical mechanics does not supply us with a sufficiently
broad basis for the theoretical presentation of all physical
phenomena, still we must grant it a considerable measure of ``truth,"
since it supplies us with the actual motions of the heavenly bodies
with a delicacy of detail little short of wonderful. The principle of
relativity must therefore apply with great accuracy in the domain of
mechanics. But that a principle of such broad generality should hold
with such exactness in one domain of phenomena, and yet should be
invalid for another, is a priori not very probable.

We now proceed to the second argument, to which, moreover, we shall
return later. If the principle of relativity (in the restricted sense)
does not hold, then the Galileian co-ordinate systems $K$, $K'$, $K''$, etc.,
which are moving uniformly relative to each other, will not be
equivalent for the description of natural phenomena. In this case we
should be constrained to believe that natural laws are capable of
being formulated in a particularly simple manner, and of course only
on condition that, from amongst all possible Galileian co-ordinate
systems, we should have chosen \emph{one} ($K_0$) of a particular state of
motion as our body of reference. We should then be justified (because
of its merits for the description of natural phenomena) in calling
this system ``absolutely at rest," and all other Galileian systems $K$ 
``in motion." If, for instance, our embankment were the system $K_0$ then
our railway carriage would be a system $K$, relative to which less
simple laws would hold than with respect to $K_0$. This diminished
simplicity would be due to the fact that the carriage $K$ would be in
motion ({\it i.e.} ``really") with respect to $K_0$. In the general laws of
nature which have been formulated with reference to $K$, the magnitude
and direction of the velocity of the carriage would necessarily play a
part. We should expect, for instance, that the note emitted by an
organpipe placed with its axis parallel to the direction of travel
would be different from that emitted if the axis of the pipe were
placed perpendicular to this direction.

Now in virtue of its motion in an orbit round the sun, our earth is
comparable with a railway carriage travelling with a velocity of about
30 kilometres per second. If the principle of relativity were not
valid we should therefore expect that the direction of motion of the
earth at any moment would enter into the laws of nature, and also that
physical systems in their behaviour would be dependent on the
orientation in space with respect to the earth. For owing to the
alteration in direction of the velocity of revolution of the earth in
the course of a year, the earth cannot be at rest relative to the
hypothetical system $K_0$ throughout the whole year. However, the most
careful observations have never revealed such anisotropic properties
in terrestrial physical space, {\it i.e.} a physical non-equivalence of
different directions. This is very powerful argument in favour of the
principle of relativity.



\chapter{The Theorem of the Addition of Velocities Employed in 
Classical Mechanics}


Let us suppose our old friend the railway carriage to be travelling
along the rails with a constant velocity $v$, and that a man traverses
the length of the carriage in the direction of travel with a velocity
$w$. How quickly or, in other words, with what velocity $W$ does the man
advance relative to the embankment during the process? The only
possible answer seems to result from the following consideration: If
the man were to stand still for a second, he would advance relative to
the embankment through a distance $v$ equal numerically to the velocity
of the carriage. As a consequence of his walking, however, he
traverses an additional distance $w$ relative to the carriage, and hence
also relative to the embankment, in this second, the distance w being
numerically equal to the velocity with which he is walking. Thus in
total be covers the distance $W=v+w$ relative to the embankment in the
second considered. We shall see later that this result, which
expresses the theorem of the addition of velocities employed in
classical mechanics, cannot be maintained; in other words, the law
that we have just written down does not hold in reality. For the time
being, however, we shall assume its correctness.



\chapter{The Apparent Incompatability of the Law of Propagation of Light 
with the Principle of Relativity}


There is hardly a simpler law in physics than that according to which
light is propagated in empty space. Every child at school knows, or
believes he knows, that this propagation takes place in straight lines
with a velocity $c= 300,000$ km./sec. At all events we know with great
exactness that this velocity is the same for all colours, because if
this were not the case, the minimum of emission would not be observed
simultaneously for different colours during the eclipse of a fixed
star by its dark neighbour. By means of similar considerations based
on observations of double stars, the Dutch astronomer De Sitter was
also able to show that the velocity of propagation of light cannot
depend on the velocity of motion of the body emitting the light. The
assumption that this velocity of propagation is dependent on the
direction ``in space" is in itself improbable.

In short, let us assume that the simple law of the constancy of the
velocity of light $c$ (in vacuum) is justifiably believed by the child
at school. Who would imagine that this simple law has plunged the
conscientiously thoughtful physicist into the greatest intellectual
difficulties? Let us consider how these difficulties arise.

Of course we must refer the process of the propagation of light (and
indeed every other process) to a rigid reference-body (co-ordinate
system). As such a system let us again choose our embankment. We shall
imagine the air above it to have been removed. If a ray of light be
sent along the embankment, we see from the above that the tip of the
ray will be transmitted with the velocity $c$ relative to the
embankment. Now let us suppose that our railway carriage is again
travelling along the railway lines with the velocity $v$, and that its
direction is the same as that of the ray of light, but its velocity of
course much less. Let us inquire about the velocity of propagation of
the ray of light relative to the carriage. It is obvious that we can
here apply the consideration of the previous section, since the ray of
light plays the part of the man walking along relatively to the
carriage. The velocity $w$ of the man relative to the embankment is here
replaced by the velocity of light relative to the embankment. $w$ is the
required velocity of light with respect to the carriage, and we have

                               $$w = c-v.$$

The velocity of propagation ot a ray of light relative to the carriage
thus comes cut smaller than $c$.

But this result comes into conflict with the principle of relativity
set forth in Section V. For, like every other general law of
nature, the law of the transmission of light in vacuo [in vacuum]
must, according to the principle of relativity, be the same for the
railway carriage as reference-body as when the rails are the body of
reference. But, from our above consideration, this would appear to be
impossible. If every ray of light is propagated relative to the
embankment with the velocity $c$, then for this reason it would appear
that another law of propagation of light must necessarily hold with
respect to the carriage---a result contradictory to the principle of
relativity.

In view of this dilemma there appears to be nothing else for it than
to abandon either the principle of relativity or the simple law of the
propagation of light in vacuo. Those of you who have carefully
followed the preceding discussion are almost sure to expect that we
should retain the principle of relativity, which appeals so
convincingly to the intellect because it is so natural and simple. The
law of the propagation of light in vacuo would then have to be
replaced by a more complicated law conformable to the principle of
relativity. The development of theoretical physics shows, however,
that we cannot pursue this course. The epoch-making theoretical
investigations of H. A. Lorentz on the electrodynamical and optical
phenomena connected with moving bodies show that experience in this
domain leads conclusively to a theory of electromagnetic phenomena, of
which the law of the constancy of the velocity of light in vacuo is a
necessary consequence. Prominent theoretical physicists were theref
ore more inclined to reject the principle of relativity, in spite of
the fact that no empirical data had been found which were
contradictory to this principle.

At this juncture the theory of relativity entered the arena. As a
result of an analysis of the physical conceptions of time and space,
it became evident that \emph{in realily there is not the least
incompatibilitiy between the principle of relativity and the law of
propagation of light}, and that by systematically holding fast to both
these laws a logically rigid theory could be arrived at. This theory
has been called the \emph{special theory of relativity} to distinguish it
from the extended theory, with which we shall deal later. In the
following pages we shall present the fundamental ideas of the special
theory of relativity.


\chapter{On the Idea of Time in Physics}

Lightning has struck the rails on our railway embankment at two places
A and B far distant from each other. I make the additional assertion
that these two lightning flashes occurred simultaneously. If I ask you
whether there is sense in this statement, you will answer my question
with a decided ``Yes." But if I now approach you with the request to
explain to me the sense of the statement more precisely, you find
after some consideration that the answer to this question is not so
easy as it appears at first sight.

After some time perhaps the following answer would occur to you: ``The
significance of the statement is clear in itself and needs no further
explanation; of course it would require some consideration if I were
to be commissioned to determine by observations whether in the actual
case the two events took place simultaneously or not." I cannot be
satisfied with this answer for the following reason. Supposing that as
a result of ingenious considerations an able meteorologist were to
discover that the lightning must always strike the places A and B
simultaneously, then we should be faced with the task of testing
whether or not this theoretical result is in accordance with the
reality. We encounter the same difficulty with all physical statements
in which the conception ``simultaneous'' plays a part. The concept
does not exist for the physicist until he has the possibility of
discovering whether or not it is fulfilled in an actual case. We thus
require a definition of simultaneity such that this definition
supplies us with the method by means of which, in the present case, he
can decide by experiment whether or not both the lightning strokes
occurred simultaneously. As long as this requirement is not satisfied,
I allow myself to be deceived as a physicist (and of course the same
applies if I am not a physicist), when I imagine that I am able to
attach a meaning to the statement of simultaneity. (I would ask the
reader not to proceed farther until he is fully convinced on this
point.)

After thinking the matter over for some time you then offer the
following suggestion with which to test simultaneity. By measuring
along the rails, the connecting line AB should be measured up and an
observer placed at the mid-point M of the distance AB. This observer
should be supplied with an arrangement ({\it e.g.} two mirrors inclined at
$90^\circ$) which allows him visually to observe both places A and B at the
same time. If the observer perceives the two flashes of lightning at
the same time, then they are simultaneous.

I am very pleased with this suggestion, but for all that I cannot
regard the matter as quite settled, because I feel constrained to
raise the following objection:

"Your definition would certainly be right, if only I knew that the
light by means of which the observer at M perceives the lightning
flashes travels along the length A~$\longrightarrow$~M with the same velocity as
along the length B~$\longrightarrow$~M. But an examination of this supposition
would only be possible if we already had at our disposal the means of
measuring time. It would thus appear as though we were moving here in
a logical circle."

After further consideration you cast a somewhat disdainful glance at
me---and rightly so---and you declare:

 ``I maintain my previous definition nevertheless, because in reality it
assumes absolutely nothing about light. There is only one demand to be
made of the definition of simultaneity, namely, that in every real
case it must supply us with an empirical decision as to whether or not
the conception that has to be defined is fulfilled. That my definition
satisfies this demand is indisputable. That light requires the same
time to traverse the path A~$\longrightarrow$~M as for the path B~$\longrightarrow$~M is in
reality neither a supposition nor a hypothesis about the physical
nature of light, but a stipulation which I can make of my own freewill
in order to arrive at a definition of simultaneity."

It is clear that this definition can be used to give an exact meaning
not only to \emph{two} events, but to as many events as we care to choose,
and independently of the positions of the scenes of the events with
respect to the body of reference\footnotemark[1] (here the railway embankment).
We are thus led also to a definition of ``time" in physics. For this
purpose we suppose that clocks of identical construction are placed at
the points A, B, and C of the railway line (co-ordinate system) and
that they are set in such a manner that the positions of their
pointers are simultaneously (in the above sense) the same. Under these
conditions we understand by the ``time" of an event the reading
(position of the hands) of that one of these clocks which is in the
immediate vicinity (in space) of the event. In this manner a
time-value is associated with every event which is essentially capable
of observation.

This stipulation contains a further physical hypothesis, the validity
of which will hardly be doubted without empirical evidence to the
contrary. It has been assumed that all these clocks go \emph{at the same
rate} if they are of identical construction. Stated more exactly: When
two clocks arranged at rest in different places of a reference-body
are set in such a manner that a \emph{particular} position of the pointers of
the one clock is \emph{simultaneous} (in the above sense) with the same
position, of the pointers of the other clock, then identical ``settings''
are always simultaneous (in the sense of the above
definition).


%  Notes

\footnotetext[1]{We suppose further, that, when three events A, B, and C occur in
different places in such a manner that A is simultaneous with B and B
is simultaneous with C (simultaneous in the sense of the above
definition), then the criterion for the simultaneity of the pair of
events A, C is also satisfied. This assumption is a physical
hypothesis about the the of propagation of light: it must certainly be
fulfilled if we are to maintain the law of the constancy of the
velocity of light in vacuo.}



\chapter{The Relativity of Simulatneity}


Up to now our considerations have been referred to a particular body
of reference, which we have styled a ``railway embankment." We suppose
a very long train travelling along the rails with the constant
velocity v and in the direction indicated in Fig. \ref{fig:1}. People travelling
in this train will with a vantage view the train as a rigid
reference-body (co-ordinate system); they regard all events in
reference to the train. Then every event which takes place along the
line also takes place at a particular point of the train. Also the
definition of simultaneity can be given relative to the train in
exactly the same way as with respect to the embankment. As a natural
consequence, however, the following question arises:

%                       Fig. 01:
%
%   v --->         M' ----->      v --->   Train
%   -------|------------|------------|----/
% ---------|------------|------------|-------
%          A            M            B   Embankment
%
%

\begin{figure}[hbtp]

\centering
\caption{}
\label{fig:1}

\begin{picture}(250,100)(0,0)
\thicklines
\put(0,40){\line(1,0){250}}
\put(230,30){Embankment}
\put(15,50){\line(1,0){210}}
\put(225,50){\line(1,1){10}}
\put(235,60){Train}
\put(40,35){\line(0,1){20}}
\put(37,23){A}
\put(125,48){\line(0,1){7}}
\put(125,35){\line(0,1){7}}
\put(121,23){M}
\put(210,35){\line(0,1){20}}
\put(207,23){B}
\thinlines
\put(15,60){$v$}
\put(22,62){\vector(1,0){25}}
\put(195,60){$v$}
\put(202,62){\vector(1,0){25}}
\put(105,65){M$'$}
\put(122,67){\vector(1,0){25}}
\end{picture}

\end{figure}

Are two events ({\it e.g.} the two strokes of lightning A and B) which are
simultaneous \emph{with reference to the railway embankment} also
simultaneous \emph{relatively to the train}? We shall show directly that the
answer must be in the negative.

When we say that the lightning strokes A and B are simultaneous with
respect to be embankment, we mean: the rays of light emitted at the
places A and B, where the lightning occurs, meet each other at the
mid-point M of the length A $\longrightarrow$ B of the embankment. But the events
A and B also correspond to positions A and B on the train. Let M$'$ be
the mid-point of the distance A $\longrightarrow$ B on the travelling train. Just
when the flashes (as judged from the embankment) of lightning occur,
this point M$'$ naturally coincides with the point M but it moves
towards the right in the diagram with the velocity v of the train. If
an observer sitting in the position M$'$ in the train did not possess
this velocity, then he would remain permanently at M, and the light
rays emitted by the flashes of lightning A and B would reach him
simultaneously, {\it i.e.} they would meet just where he is situated. Now in
reality (considered with reference to the railway embankment) he is
hastening towards the beam of light coming from B, whilst he is riding
on ahead of the beam of light coming from A. Hence the observer will
see the beam of light emitted from B earlier than he will see that
emitted from A. Observers who take the railway train as their
reference-body must therefore come to the conclusion that the
lightning flash B took place earlier than the lightning flash A. We
thus arrive at the important result:

Events which are simultaneous with reference to the embankment are not
simultaneous with respect to the train, and vice versa (relativity of
simultaneity). Every reference-body (co-ordinate system) has its own
particular time; unless we are told the reference-body to which the
statement of time refers, there is no meaning in a statement of the
time of an event.

Now before the advent of the theory of relativity it had always
tacitly been assumed in physics that the statement of time had an
absolute significance, {\it i.e.} that it is independent of the state of
motion of the body of reference. But we have just seen that this
assumption is incompatible with the most natural definition of
simultaneity; if we discard this assumption, then the conflict between
the law of the propagation of light in vacuo and the principle of
relativity (developed in Section 6) disappears.

We were led to that conflict by the considerations of Section 6,
which are now no longer tenable. In that section we concluded that the
man in the carriage, who traverses the distance $w$ \emph{per second} relative
to the carriage, traverses the same distance also with respect to the
embankment \emph{in each second} of time. But, according to the foregoing
considerations, the time required by a particular occurrence with
respect to the carriage must not be considered equal to the duration
of the same occurrence as judged from the embankment (as
reference-body). Hence it cannot be contended that the man in walking
travels the distance $w$ relative to the railway line in a time which is
equal to one second as judged from the embankment.

Moreover, the considerations of Section 6 are based on yet a second
assumption, which, in the light of a strict consideration, appears to
be arbitrary, although it was always tacitly made even before the
introduction of the theory of relativity.



\chapter{On the Relativity of the Conception of Distance}


Let us consider two particular points on the train\footnotemark travelling
along the embankment with the velocity $v$, and inquire as to their
distance apart. We already know that it is necessary to have a body of
reference for the measurement of a distance, with respect to which
body the distance can be measured up. It is the simplest plan to use
the train itself as reference-body (co-ordinate system). An observer
in the train measures the interval by marking off his measuring-rod in
a straight line ({\it e.g.} along the floor of the carriage) as many times
as is necessary to take him from the one marked point to the other.
Then the number which tells us how often the rod has to be laid down
is the required distance.

It is a different matter when the distance has to be judged from the
railway line. Here the following method suggests itself. If we call
A$'$ and B$'$ the two points on the train whose distance apart is
required, then both of these points are moving with the velocity $v$
along the embankment. In the first place we require to determine the
points A and B of the embankment which are just being passed by the
two points A$'$ and B$'$ at a particular time $t$---judged from the
embankment. These points A and B of the embankment can be determined
by applying the definition of time given in Section 8. The distance
between these points A and B is then measured by repeated application
of thee measuring-rod along the embankment.

A priori it is by no means certain that this last measurement will
supply us with the same result as the first. Thus the length of the
train as measured from the embankment may be different from that
obtained by measuring in the train itself. This circumstance leads us
to a second objection which must be raised against the apparently
obvious consideration of Section 6. Namely, if the man in the
carriage covers the distance $w$ in a unit of time---\emph{measured from the
train},---then this distance--\emph{as measured from the embankment}---is
not necessarily also equal to $w$.


%  Notes

\footnotetext{{\it e.g.} the middle of the first and of the hundredth carriage.}


\chapter{The Lorentz Transformation}


The results of the last three sections show that the apparent
incompatibility of the law of propagation of light with the principle
of relativity (Section 7) has been derived by means of a
consideration which borrowed two unjustifiable hypotheses from
classical mechanics; these are as follows:

\begin{enumerate}
\item The time-interval (time) between two events is independent of the
condition of motion of the body of reference.
\item The space-interval (distance) between two points of a rigid body
is independent of the condition of motion of the body of reference.
\end{enumerate}

If we drop these hypotheses, then the dilemma of Section 7
disappears, because the theorem of the addition of velocities derived
in Section 6 becomes invalid. The possibility presents itself that
the law of the propagation of light in vacuo may be compatible with
the principle of relativity, and the question arises: How have we to
modify the considerations of Section 6 in order to remove the
apparent disagreement between these two fundamental results of
experience? This question leads to a general one. In the discussion of
Section 6 we have to do with places and times relative both to the
train and to the embankment. How are we to find the place and time of
an event in relation to the train, when we know the place and time of
the event with respect to the railway embankment? Is there a
thinkable answer to this question of such a nature that the law of
transmission of light in vacuo does not contradict the principle of
relativity? In other words: Can we conceive of a relation between
place and time of the individual events relative to both
reference-bodies, such that every ray of light possesses the velocity
of transmission $c$ relative to the embankment and relative to the train?
This question leads to a quite definite positive answer, and to a
perfectly definite transformation law for the space-time magnitudes of
an event when changing over from one body of reference to another.

% Figure 2

%          z'
%          | ---> 
%  z       |    y'  
%  |       |   / --->
%  |    y  |  /  v
%  |   /   | / --->
%  |  /    |/______________x'
%  | /     K'
%  |/______________x
%  K

\begin{figure}[hbtp]

\centering
\caption{}
\label{fig:2}

\begin{picture}(200,220)(0,0)
\thicklines
\put(15,10){$K$}
\put(20,20){\line(1,0){125}}
\put(149,17){$x$}
\put(20,20){\line(0,1){125}}
\put(17,150){$z$}
\put(20,20){\line(1,2){40}}
\put(55,105){$y$}

\put(85,25){$K'$}
\put(90,35){\line(1,0){125}}
\put(219,32){$x'$}
\put(90,35){\line(0,1){125}}
\put(87,165){$z'$}
\put(90,35){\line(1,2){40}}
\put(125,120){$y'$}

\thinlines
\put(95,155){\vector(1,0){35}}
\put(135,110){\vector(1,0){35}}
\put(110,40){\vector(1,0){35}}
\end{picture}

\end{figure}

Before we deal with this, we shall introduce the following incidental
consideration. Up to the present we have only considered events taking
place along the embankment, which had mathematically to assume the
function of a straight line. In the manner indicated in Section 2
we can imagine this reference-body supplemented laterally and in a
vertical direction by means of a framework of rods, so that an event
which takes place anywhere can be localised with reference to this
framework. Similarly, we can imagine the train travelling with
the velocity $v$ to be continued across the whole of space, so that
every event, no matter how far off it may be, could also be localised
with respect to the second framework. Without committing any
fundamental error, we can disregard the fact that in reality these
frameworks would continually interfere with each other, owing to the
impenetrability of solid bodies. In every such framework we imagine
three surfaces perpendicular to each other marked out, and designated
as ``co-ordinate planes" (``co-ordinate system"). A co-ordinate
system $K$ then corresponds to the embankment, and a co-ordinate system
$K'$ to the train. An event, wherever it may have taken place, would be
fixed in space with respect to $K$ by the three perpendiculars $x, y, z$
on the co-ordinate planes, and with regard to time by a time value $t$.
Relative to $K'$, the same event would be fixed in respect of space and
time by corresponding values $x', y', z', t'$, which of course are not
identical with $x, y, z, t$. It has already been set forth in detail how
these magnitudes are to be regarded as results of physical
measurements.

Obviously our problem can be exactly formulated in the following
manner. What are the values $x', y', z', t'$, of an event with respect
to $K'$, when the magnitudes $x, y, z, t$, of the same event with respect
to $K$ are given? The relations must be so chosen that the law of the
transmission of light in vacuo is satisfied for one and the same ray
of light (and of course for every ray) with respect to $K$ and $K'$. For
the relative orientation in space of the co-ordinate systems indicated
in the diagram (Fig \ref{fig:2}), this problem is solved by means of the
equations:

\begin{eqnarray*} 
x' &=& \frac{x-vt}{\sqrt{I-\frac{v^2}{c^2}}} \\
y' &=& y \\
z' &=& z \\
t' &=& \frac{t-\frac{v}{c^2}x}{\sqrt{I-\frac{v^2}{c^2}}} \\
\end{eqnarray*}

\noindent This system of equations is known as the ``Lorentz transformation."\footnotemark

If in place of the law of transmission of light we had taken as our
basis the tacit assumptions of the older mechanics as to the absolute
character of times and lengths, then instead of the above we should
have obtained the following equations:

\begin{eqnarray*} 
x' &=& x - vt \\
y' &=& y \\
z' &=& z \\
t' &=& t \\
\end{eqnarray*}

\noindent This system of equations is often termed the ``Galilei
transformation." The Galilei transformation can be obtained from the
Lorentz transformation by substituting an infinitely large value for
the velocity of light $c$ in the latter transformation.

Aided by the following illustration, we can readily see that, in
accordance with the Lorentz transformation, the law of the
transmission of light in vacuo is satisfied both for the
reference-body $K$ and for the reference-body $K'$. A light-signal is sent
along the positive $x$-axis, and this light-stimulus advances in
accordance with the equation

                               $$x = ct,$$

\noindent {\it i.e.} with the velocity $c$. According to the equations of the Lorentz
transformation, this simple relation between $x$ and $t$ involves a
relation between $x'$ and $t'$. In point of fact, if we substitute for $x$
the value $ct$ in the first and fourth equations of the Lorentz
transformation, we obtain:

\begin{eqnarray*} 
x' &=& \frac{(c-v)t}{\sqrt{I-\frac{v^2}{c^2}}} \\
t' &=& \frac{(I-\frac{v}{c})t}{\sqrt{I-\frac{v^2}{c^2}}}
\end{eqnarray*}

\noindent from which, by division, the expression

                               $$x' = ct'$$

\noindent immediately follows. If referred to the system $K'$, the propagation of
light takes place according to this equation. We thus see that the
velocity of transmission relative to the reference-body $K'$ is also
equal to $c$. The same result is obtained for rays of light advancing in
any other direction whatsoever. Of cause this is not surprising, since
the equations of the Lorentz transformation were derived conformably
to this point of view.


%  Notes

\footnotetext{A simple derivation of the Lorentz transformation is given in
Appendix I.}



\chapter{The Behaviour of Measuring-Rods and Clocks in Motion}


Place a metre-rod in the $x'$-axis of $K'$ in such a manner that one end
(the beginning) coincides with the point $x'=0$ whilst the other end
(the end of the rod) coincides with the point $x'=I$. What is the length
of the metre-rod relatively to the system $K$? In order to learn this,
we need only ask where the beginning of the rod and the end of the rod
lie with respect to $K$ at a particular time $t$ of the system $K$. By means
of the first equation of the Lorentz transformation the values of
these two points at the time $t = 0$ can be shown to be

\begin{eqnarray*} 
x_{\mbox{(begining of rod)}} &=& 0 \overline{\sqrt{I-\frac{v^2}{c^2}}} \\
x_{\mbox{(end of rod)}} &=& 1 \overline{\sqrt{I-\frac{v^2}{c^2}}}
\end{eqnarray*}
~

\noindent the distance between the points being $\sqrt{I-v^2/c^2}$.

But the metre-rod is moving with the velocity v relative to K. It
therefore follows that the length of a rigid metre-rod moving in the
direction of its length with a velocity $v$ is $\sqrt{I-v^2/c^2}$ of a metre.

The rigid rod is thus shorter when in motion than when at rest, and
the more quickly it is moving, the shorter is the rod. For the
velocity $v=c$ we should have $\sqrt{I-v^2/c^2} = 0$,
and for stiII greater velocities the square-root becomes imaginary.
From this we conclude that in the theory of relativity the velocity $c$
plays the part of a limiting velocity, which can neither be reached
nor exceeded by any real body.

Of course this feature of the velocity $c$ as a limiting velocity also
clearly follows from the equations of the Lorentz transformation, for
these became meaningless if we choose values of $v$ greater than $c$.

If, on the contrary, we had considered a metre-rod at rest in the
$x$-axis with respect to $K$, then we should have found that the length of
the rod as judged from $K'$ would have been $\sqrt{I-v^2/c^2}$;
this is quite in accordance with the principle of relativity which
forms the basis of our considerations.

\emph{A Priori} it is quite clear that we must be able to learn something
about the physical behaviour of measuring-rods and clocks from the
equations of transformation, for the magnitudes $z, y, x, t$, are
nothing more nor less than the results of measurements obtainable by
means of measuring-rods and clocks. If we had based our considerations
on the Galileian transformation we should not have obtained a
contraction of the rod as a consequence of its motion.

Let us now consider a seconds-clock which is permanently situated at
the origin ($x'=0$) of $K'$. $t'=0$ and $t'=I$ are two successive ticks of
this clock. The first and fourth equations of the Lorentz
transformation give for these two ticks:

$$t = 0$$

\noindent and

$$t' = \frac{I}{\sqrt{I-\frac{v^2}{c^2}}}$$
~

As judged from $K$, the clock is moving with the velocity $v$; as judged
from this reference-body, the time which elapses between two strokes
of the clock is not one second, but

$$\frac{I}{\sqrt{I-\frac{v^2}{c^2}}}$$
~

\noindent seconds, {\it i.e.} a somewhat larger time. As a consequence of its motion
the clock goes more slowly than when at rest. Here also the velocity $c$
plays the part of an unattainable limiting velocity.



\chapter{Theorem of the Addition of Velocities.\\
The Experiment of Fizeau}


Now in practice we can move clocks and measuring-rods only with
velocities that are small compared with the velocity of light; hence
we shall hardly be able to compare the results of the previous section
directly with the reality. But, on the other hand, these results must
strike you as being very singular, and for that reason I shall now
draw another conclusion from the theory, one which can easily be
derived from the foregoing considerations, and which has been most
elegantly confirmed by experiment.

In Section 6 we derived the theorem of the addition of velocities
in one direction in the form which also results from the hypotheses of
classical mechanics---This theorem can also be deduced readily horn the
Galilei transformation (Section 11). In place of the man walking
inside the carriage, we introduce a point moving relatively to the
co-ordinate system $K'$ in accordance with the equation

$$x' = wt'.$$
~

By means of the first and fourth equations of the Galilei
transformation we can express $x'$ and $t'$ in terms of $x$ and $t$, and we
then obtain

$$x = (v + w)t.$$
~

This equation expresses nothing else than the law of motion of the
point with reference to the system $K$ (of the man with reference to the
embankment). We denote this velocity by the symbol $W$, and we then
obtain, as in Section 6,
\begin{equation}
W=v+w
\label{eqnA}
\end{equation}

But we can carry out this consideration just as well on the basis of
the theory of relativity. In the equation
\begin{equation}
x'=wt'
\label{eqnB}
\end{equation}

\noindent we must then express $x'$and $t'$ in terms of $x$ and $t$, making use of the
first and fourth equations of the Lorentz transformation. Instead of
the equation \ref{eqnA} we then obtain the equation

$$W = \frac{v+w}{I+\frac{vw}{c^2}}$$
~

\noindent which corresponds to the theorem of addition for velocities in one
direction according to the theory of relativity. The question now
arises as to which of these two theorems is the better in accord with
experience. On this point we axe enlightened by a most important
experiment which the brilliant physicist Fizeau performed more than
half a century ago, and which has been repeated since then by some of
the best experimental physicists, so that there can be no doubt about
its result. The experiment is concerned with the following question.
Light travels in a motionless liquid with a particular velocity $w$. How
quickly does it travel in the direction of the arrow in the tube T
(see the accompanying diagram, Figure \ref{fig:3}) when the liquid above
mentioned is flowing through the tube with a velocity $v$?

% Figure 3
%
%                       T
%                     /
%  --------------------------------------
%        v --------->
%  --------------------------------------
% 


\begin{figure}[hbtp]

\centering
\caption{}
\label{fig:3}

\begin{picture}(200,75)(0,0)
\thicklines
\put(0,15){\line(1,0){200}}
\put(0,35){\line(1,0){200}}
\put(100,35){\line(1,3){5}}
\put(107,52){T}

\thinlines
\put(40,25){\vector(1,0){50}}
\put(60,26){$v$}
\end{picture}

\end{figure}


In accordance with the principle of relativity we shall certainly have
to take for granted that the propagation of light always takes place
with the same velocity w \emph{with respect to the liquid}, whether the
latter is in motion with reference to other bodies or not. The
velocity of light relative to the liquid and the velocity of the
latter relative to the tube are thus known, and we require the
velocity of light relative to the tube.

It is clear that we have the problem of Section 6 again before us. The
tube plays the part of the railway embankment or of the co-ordinate
system $K$, the liquid plays the part of the carriage or of the
co-ordinate system $K'$, and finally, the light plays the part of the
man walking along the carriage, or of the moving point in the present
section. If we denote the velocity of the light relative to the tube
by $W$, then this is given by the equation \ref{eqnA} or \ref{eqnB}, according as the
Galilei transformation or the Lorentz transformation corresponds to
the facts. Experiment\footnotemark decides in favour of equation \ref{eqnB} derived
from the theory of relativity, and the agreement is, indeed, very
exact. According to recent and most excellent measurements by Zeeman,
the influence of the velocity of flow $v$ on the propagation of light is
represented by formula \ref{eqnB} to within one per cent.

Nevertheless we must now draw attention to the fact that a theory of
this phenomenon was given by H. A. Lorentz long before the statement
of the theory of relativity. This theory was of a purely
electrodynamical nature, and was obtained by the use of particular
hypotheses as to the electromagnetic structure of matter. This
circumstance, however, does not in the least diminish the
conclusiveness of the experiment as a crucial test in favour of the
theory of relativity, for the electrodynamics of Maxwell-Lorentz, on
which the original theory was based, in no way opposes the theory of
relativity. Rather has the latter been developed trom electrodynamics
as an astoundingly simple combination and generalisation of the
hypotheses, formerly independent of each other, on which
electrodynamics was built.


%  Notes

\footnotetext{Fizeau found $W=w+v\left(I-\frac{I}{n^2}\right)$, where $n=\frac{c}{w}$
is the index of refraction of the liquid. On the other hand, owing to
the smallness of $\frac{vw}{c^2}$ as compared with $I$,
we can replace (B) in the first place by $W=(w+v)\left(I-\frac{vw}{c^2}\right)$, or to the same order
of approximation by
$w+v\left(I-\frac{I}{n^2}\right)$, which agrees with Fizeau's result.}



\chapter{The Heuristic Value of the Theory of Relativity}


Our train of thought in the foregoing pages can be epitomised in the
following manner. Experience has led to the conviction that, on the
one hand, the principle of relativity holds true and that on the other
hand the velocity of transmission of light in vacuo has to be
considered equal to a constant $c$. By uniting these two postulates we
obtained the law of transformation for the rectangular co-ordinates $x,
y, z$ and the time $t$ of the events which constitute the processes of
nature. In this connection we did not obtain the Galilei
transformation, but, differing from classical mechanics, the \emph{Lorentz
transformation}.

The law of transmission of light, the acceptance of which is justified
by our actual knowledge, played an important part in this process of
thought. Once in possession of the Lorentz transformation, however, we
can combine this with the principle of relativity, and sum up the
theory thus:

Every general law of nature must be so constituted that it is
transformed into a law of exactly the same form when, instead of the
space-time variables $x, y, z, t$ of the original coordinate system $K$,
we introduce new space-time variables $x', y', z', t'$ of a co-ordinate
system $K'$. In this connection the relation between the ordinary and
the accented magnitudes is given by the Lorentz transformation. Or in
brief: General laws of nature are co-variant with respect to Lorentz
transformations.

This is a definite mathematical condition that the theory of
relativity demands of a natural law, and in virtue of this, the theory
becomes a valuable heuristic aid in the search for general laws of
nature. If a general law of nature were to be found which did not
satisfy this condition, then at least one of the two fundamental
assumptions of the theory would have been disproved. Let us now
examine what general results the latter theory has hitherto evinced.



\chapter{General Results of the Theory}


It is clear from our previous considerations that the (special) theory
of relativity has grown out of electrodynamics and optics. In these
fields it has not appreciably altered the predictions of theory, but
it has considerably simplified the theoretical structure, {\it i.e.} the
derivation of laws, and---what is incomparably more important---it
has considerably reduced the number of independent hypothese forming
the basis of theory. The special theory of relativity has rendered the
Maxwell-Lorentz theory so plausible, that the latter would have been
generally accepted by physicists even if experiment had decided less
unequivocally in its favour.

Classical mechanics required to be modified before it could come into
line with the demands of the special theory of relativity. For the
main part, however, this modification affects only the laws for rapid
motions, in which the velocities of matter $v$ are not very small as
compared with the velocity of light. We have experience of such rapid
motions only in the case of electrons and ions; for other motions the
variations from the laws of classical mechanics are too small to make
themselves evident in practice. We shall not consider the motion of
stars until we come to speak of the general theory of relativity. In
accordance with the theory of relativity the kinetic energy of a
material point of mass m is no longer given by the well-known
expression

$$m\frac{v^2}{2}$$

\noindent but by the expression

$$\frac{mc^2}{\sqrt{I-\frac{v^2}{c^2}}}$$
~

This expression approaches infinity as the velocity $v$ approaches the
velocity of light $c$. The velocity must therefore always remain less
than $c$, however great may be the energies used to produce the
acceleration. If we develop the expression for the kinetic energy in
the form of a series, we obtain

$$mc^2 + m\frac{v^2}{2} + \frac{3}{8} m \frac{v^4}{c^2} + \cdots$$
~

When $v^2/c^2$ is small compared with unity, the third of these terms is
always small in comparison with the second,
which last is alone considered in classical mechanics. The first term
$mc^2$ does not contain the velocity, and requires no consideration if
we are only dealing with the question as to how the energy of a
point-mass; depends on the velocity. We shall speak of its essential
significance later.

The most important result of a general character to which the special
theory of relativity has led is concerned with the conception of mass.
Before the advent of relativity, physics recognised two conservation
laws of fundamental importance, namely, the law of the canservation of
energy and the law of the conservation of mass these two fundamental
laws appeared to be quite independent of each other. By means of the
theory of relativity they have been united into one law. We shall now
briefly consider how this unification came about, and what meaning is
to be attached to it.

The principle of relativity requires that the law of the concervation
of energy should hold not only with reference to a co-ordinate system
$K$, but also with respect to every co-ordinate system $K'$ which is in a
state of uniform motion of translation relative to $K$, or, briefly,
relative to every ``Galileian'' system of co-ordinates. In contrast to
classical mechanics; the Lorentz transformation is the deciding factor
in the transition from one such system to another.

By means of comparatively simple considerations we are led to draw the
following conclusion from these premises, in conjunction with the
fundamental equations of the electrodynamics of Maxwell: A body moving
with the velocity $v$, which absorbs\footnotemark\ an amount of energy $E_0$ in
the form of radiation without suffering an alteration in velocity in
the process, has, as a consequence, its energy increased by an amount

$$\frac{E_0}{\sqrt{I-\frac{v^2}{c^2}}}$$
~

In consideration of the expression given above for the kinetic energy
of the body, the required energy of the body comes out to be

$$\frac{\left(m+\frac{E_0}{c^2}\right)c^2}{\sqrt{I-\frac{v^2}{c^2}}}$$
~

\noindent Thus the body has the same energy as a body of mass

$$\left(m+\frac{E_0}{c^2}\right)$$
~

\noindent moving with the velocity $v$. Hence we can say: If a body takes up an
amount of energy $E_0$, then its inertial mass increases by an amount

$$\frac{E_0}{c^2}$$
~

\noindent the inertial mass of a body is not a constant but varies according to
the change in the energy of the body. The inertial mass of a system of
bodies can even be regarded as a measure of its energy. The law of the
conservation of the mass of a system becomes identical with the law of
the conservation of energy, and is only valid provided that the system
neither takes up nor sends out energy. Writing the expression for the
energy in the form

$$\frac{mc^2+E_0}{\sqrt{I-\frac{v^2}{c^2}}}$$
~

\noindent we see that the term $mc^2$, which has hitherto attracted our attention,
is nothing else than the energy possessed by the body\footnotemark\ before it
absorbed the energy $E_0$.

A direct comparison of this relation with experiment is not possible
at the present time (1920; see\footnotemark\ Note, p. 48), owing to the fact that
the changes in energy E[0] to which we can Subject a system are not
large enough to make themselves perceptible as a change in the
inertial mass of the system.

$$\frac{E_0}{c^2}$$
~

\noindent is too small in comparison with the mass $m$, which was present before
the alteration of the energy. It is owing to this circumstance that
classical mechanics was able to establish successfully the
conservation of mass as a law of independent validity.

Let me add a final remark of a fundamental nature. The success of the
Faraday-Maxwell interpretation of electromagnetic action at a distance
resulted in physicists becoming convinced that there are no such
things as instantaneous actions at a distance (not involving an
intermediary medium) of the type of Newton's law of gravitation.
According to the theory of relativity, action at a distance with the
velocity of light always takes the place of instantaneous action at a
distance or of action at a distance with an infinite velocity of
transmission. This is connected with the fact that the velocity c
plays a fundamental role in this theory. In Part II we shall see in
what way this result becomes modified in the general theory of
relativity.


%  Notes

\footnotetext[1]{$E_0$ is the energy taken up, as judged from a co-ordinate system
moving with the body.}

\footnotetext[2]{As judged from a co-ordinate system moving with the body.}

\footnotetext[3]{The equation $E = mc^2$ has been thoroughly proved time and
again since this time.}



\chapter{Experience and the Special Theory of Relativity}


To what extent is the special theory of relativity supported by
experience?  This question is not easily answered for the reason
already mentioned in connection with the fundamental experiment of
Fizeau. The special theory of relativity has crystallised out from the
Maxwell-Lorentz theory of electromagnetic phenomena. Thus all facts of
experience which support the electromagnetic theory also support the
theory of relativity. As being of particular importance, I mention
here the fact that the theory of relativity enables us to predict the
effects produced on the light reaching us from the fixed stars. These
results are obtained in an exceedingly simple manner, and the effects
indicated, which are due to the relative motion of the earth with
reference to those fixed stars are found to be in accord with
experience. We refer to the yearly movement of the apparent position
of the fixed stars resulting from the motion of the earth round the
sun (aberration), and to the influence of the radial components of the
relative motions of the fixed stars with respect to the earth on the
colour of the light reaching us from them. The latter effect manifests
itself in a slight displacement of the spectral lines of the light
transmitted to us from a fixed star, as compared with the position of
the same spectral lines when they are produced by a terrestrial source
of light (Doppler principle). The experimental arguments in favour of
the Maxwell-Lorentz theory, which are at the same time arguments in
favour of the theory of relativity, are too numerous to be set forth
here. In reality they limit the theoretical possibilities to such an
extent, that no other theory than that of Maxwell and Lorentz has been
able to hold its own when tested by experience.

But there are two classes of experimental facts hitherto obtained
which can be represented in the Maxwell-Lorentz theory only by the
introduction of an auxiliary hypothesis, which in itself---{\it i.e.}
without making use of the theory of relativity---appears extraneous.

It is known that cathode rays and the so-called $\beta$-rays emitted by
radioactive substances consist of negatively electrified particles
(electrons) of very small inertia and large velocity. By examining the
deflection of these rays under the influence of electric and magnetic
fields, we can study the law of motion of these particles very
exactly.

In the theoretical treatment of these electrons, we are faced with the
difficulty that electrodynamic theory of itself is unable to give an
account of their nature. For since electrical masses of one sign repel
each other, the negative electrical masses constituting the electron
would necessarily be scattered under the influence of their mutual
repulsions, unless there are forces of another kind operating between
them, the nature of which has hitherto remained obscure to us.\footnotemark\   If
we now assume that the relative distances between the electrical
masses constituting the electron remain unchanged during the motion of
the electron (rigid connection in the sense of classical mechanics),
we arrive at a law of motion of the electron which does not agree with
experience. Guided by purely formal points of view, H. A. Lorentz was
the first to introduce the hypothesis that the form of the electron
experiences a contraction in the direction of motion in consequence of
that motion. the contracted length being proportional to the
expression

$$\overline{\sqrt{I-\frac{v^2}{c^2}}}.$$

This, hypothesis, which is not justifiable by any electrodynamical
facts, supplies us then with that particular law of motion which has
been confirmed with great precision in recent years.

The theory of relativity leads to the same law of motion, without
requiring any special hypothesis whatsoever as to the structure and
the behaviour of the electron. We arrived at a similar conclusion in
Section 13 in connection with the experiment of Fizeau, the result
of which is foretold by the theory of relativity without the necessity
of drawing on hypotheses as to the physical nature of the liquid.

The second class of facts to which we have alluded has reference to
the question whether or not the motion of the earth in space can be
made perceptible in terrestrial experiments. We have already remarked
in Section 5 that all attempts of this nature led to a negative
result. Before the theory of relativity was put forward, it was
difficult to become reconciled to this negative result, for reasons
now to be discussed. The inherited prejudices about time and space did
not allow any doubt to arise as to the prime importance of the
Galileian transformation for changing over from one body of reference
to another. Now assuming that the Maxwell-Lorentz equations hold for a
reference-body $K$, we then find that they do not hold for a
reference-body $K'$ moving uniformly with respect to $K$, if we assume
that the relations of the Galileian transformstion exist between the
co-ordinates of $K$ and $K'$. It thus appears that, of all Galileian
co-ordinate systems, one ($K$) corresponding to a particular state of
motion is physically unique. This result was interpreted physically by
regarding $K$ as at rest with respect to a hypothetical �ther of space.
On the other hand, all coordinate systems $K'$ moving relatively to $K$
were to be regarded as in motion with respect to the �ther. To this
motion of $K'$ against the {\ae}ther (``{\ae}ther-drift'' relative to $K'$) were
attributed the more complicated laws which were supposed to hold
relative to $K'$. Strictly speaking, such an {\ae}ther-drift ought also to
be assumed relative to the earth, and for a long time the efforts of
physicists were devoted to attempts to detect the existence of an
{\ae}ther-drift at the earth's surface.

In one of the most notable of these attempts Michelson devised a
method which appears as though it must be decisive. Imagine two
mirrors so arranged on a rigid body that the reflecting surfaces face
each other. A ray of light requires a perfectly definite time T to
pass from one mirror to the other and back again, if the whole system
be at rest with respect to the �ther. It is found by calculation,
however, that a slightly different time $T'$ is required for this
process, if the body, together with the mirrors, be moving relatively
to the {\ae}ther. And yet another point: it is shown by calculation that
for a given velocity v with reference to the {\ae}ther, this time $T'$ is
different when the body is moving perpendicularly to the planes of the
mirrors from that resulting when the motion is parallel to these
planes. Although the estimated difference between these two times is
exceedingly small, Michelson and Morley performed an experiment
involving interference in which this difference should have been
clearly detectable. But the experiment gave a negative result---a
fact very perplexing to physicists. Lorentz and FitzGerald rescued the
theory from this difficulty by assuming that the motion of the body
relative to the �ther produces a contraction of the body in the
direction of motion, the amount of contraction being just sufficient
to compensate for the differeace in time mentioned above. Comparison
with the discussion in Section 11 shows that also from the
standpoint of the theory of relativity this solution of the difficulty
was the right one. But on the basis of the theory of relativity the
method of interpretation is incomparably more satisfactory. According
to this theory there is no such thing as a ``specially favoured''
(unique) co-ordinate system to occasion the introduction of the
�ther-idea, and hence there can be no �ther-drift, nor any experiment
with which to demonstrate it. Here the contraction of moving bodies
follows from the two fundamental principles of the theory, without the
introduction of particular hypotheses; and as the prime factor
involved in this contraction we find, not the motion in itself, to
which we cannot attach any meaning, but the motion with respect to the
body of reference chosen in the particular case in point. Thus for a
co-ordinate system moving with the earth the mirror system of
Michelson and Morley is not shortened, but it is shortened for a
co-ordinate system which is at rest relatively to the sun.


%  Notes

\footnotetext{The general theory of relativity renders it likely that the
electrical masses of an electron are held together by gravitational
forces.}



\chapter{Minkowski's Four-Dimensional Space}


The non-mathematician is seized by a mysterious shuddering when he
hears of ``four-dimensional" things, by a feeling not unlike that
awakened by thoughts of the occult. And yet there is no more
common-place statement than that the world in which we live is a
four-dimensional space-time continuum.

Space is a three-dimensional continuum. By this we mean that it is
possible to describe the position of a point (at rest) by means of
three numbers (co-ordinales) $x, y, z$, and that there is an indefinite
number of points in the neighbourhood of this one, the position of
which can be described by co-ordinates such as $x_1, y_1, z_1$, which
may be as near as we choose to the respective values of the
co-ordinates $x, y, z$, of the first point. In virtue of the latter
property we speak of a ``continuum," and owing to the fact that there
are three co-ordinates we speak of it as being ``three-dimensional."

Similarly, the world of physical phenomena which was briefly called 
 ``world" by Minkowski is naturally four dimensional in the space-time
sense. For it is composed of individual events, each of which is
described by four numbers, namely, three space co-ordinates $x, y, z$,
and a time co-ordinate, the time value $t$. The ``world" is in this sense
also a continuum; for to every event there are as many ``neighbouring"
events (realised or at least thinkable) as we care to choose, the
co-ordinates $x_1, y_1, z_1, t_1$ of which differ by an indefinitely
small amount from those of the event $x, y, z, t$ originally considered.
That we have not been accustomed to regard the world in this sense as
a four-dimensional continuum is due to the fact that in physics,
before the advent of the theory of relativity, time played a different
and more independent role, as compared with the space coordinates. It
is for this reason that we have been in the habit of treating time as
an independent continuum. As a matter of fact, according to classical
mechanics, time is absolute, {\it i.e.} it is independent of the position
and the condition of motion of the system of co-ordinates. We see this
expressed in the last equation of the Galileian transformation ($t' =
t$)

The four-dimensional mode of consideration of the ``world" is natural
on the theory of relativity, since according to this theory time is
robbed of its independence. This is shown by the fourth equation of
the Lorentz transformation:

$$t' = \frac{t-\frac{v}{c^2}x}{\sqrt{I-\frac{v^2}{c^2}}}$$
~

Moreover, according to this equation the time difference $\Delta t'$ of two
events with respect to $K'$ does not in general vanish, even when the
time difference $\Delta t'$ of the same events with reference to $K$ vanishes.
Pure ``space-distance'' of two events with respect to $K$ results in 
 ``time-distance'' of the same events with respect to $K'$. But the
discovery of Minkowski, which was of importance for the formal
development of the theory of relativity, does not lie here. It is to
be found rather in the fact of his recognition that the
four-dimensional space-time continuum of the theory of relativity, in
its most essential formal properties, shows a pronounced relationship
to the three-dimensional continuum of Euclidean geometrical
space.\footnotemark \ In order to give due prominence to this relationship,
however, we must replace the usual time co-ordinate $t$ by an imaginary
magnitude $\sqrt{-I}ct$ proportional to it. Under these conditions, the
natural laws satisfying the demands of the (special) theory of
relativity assume mathematical forms, in which the time co-ordinate
plays exactly the same role as the three space co-ordinates. Formally,
these four co-ordinates correspond exactly to the three space
co-ordinates in Euclidean geometry. It must be clear even to the
non-mathematician that, as a consequence of this purely formal
addition to our knowledge, the theory perforce gained clearness in no
mean measure.

These inadequate remarks can give the reader only a vague notion of
the important idea contributed by Minkowski. Without it the general
theory of relativity, of which the fundamental ideas are developed in
the following pages, would perhaps have got no farther than its long
clothes. Minkowski's work is doubtless difficult of access to anyone
inexperienced in mathematics, but since it is not necessary to have a
very exact grasp of this work in order to understand the fundamental
ideas of either the special or the general theory of relativity, I
shall leave it here at present, and revert to it only towards the end
of Part 2.


%  Notes

\footnotetext{Cf. the somewhat more detailed discussion in Appendix II.}




%PART II

\part{Then General Theory of Relativity}

\chapter{Special and General Principle of Relativity}

The basal principle, which was the pivot of all our previous
considerations, was the special principle of relativity, \emph{{\it i.e.}} the
principle of the physical relativity of all \emph{uniform} motion. Let as
once more analyse its meaning carefully.

It was at all times clear that, from the point of view of the idea it
conveys to us, every motion must be considered only as a relative
motion. Returning to the illustration we have frequently used of the
embankment and the railway carriage, we can express the fact of the
motion here taking place in the following two forms, both of which are
equally justifiable:

\begin{enumerate}
\item The carriage is in motion relative to the embankment,
\item The embankment is in motion relative to the carriage.
\end{enumerate}

In (a) the embankment, in (b) the carriage, serves as the body of
reference in our statement of the motion taking place. If it is simply
a question of detecting or of describing the motion involved, it is in
principle immaterial to what reference-body we refer the motion. As
already mentioned, this is self-evident, but it must not be confused
with the much more comprehensive statement called ``the principle of
relativity," which we have taken as the basis of our investigations.

The principle we have made use of not only maintains that we may
equally well choose the carriage or the embankment as our
reference-body for the description of any event (for this, too, is
self-evident). Our principle rather asserts what follows: If we
formulate the general laws of nature as they are obtained from
experience, by making use of

\begin{enumerate}
\item the embankment as reference-body,
\item the railway carriage as reference-body,
\end{enumerate}

\noindent then these general laws of nature ({\it e.g.} the laws of mechanics or the
law of the propagation of light in vacuo) have exactly the same form
in both cases. This can also be expressed as follows: For the
physical description of natural processes, neither of the reference
bodies $K$, $K'$ is unique (lit. ``specially marked out'') as compared
with the other. Unlike the first, this latter statement need not of
necessity hold a priori; it is not contained in the conceptions of 
``motion" and ``reference-body'' and derivable from them; only
experience can decide as to its correctness or incorrectness.

Up to the present, however, we have by no means maintained the
equivalence of all bodies of reference $K$ in connection with the
formulation of natural laws. Our course was more on the following
Iines. In the first place, we started out from the assumption that
there exists a reference-body $K$, whose condition of motion is such
that the Galileian law holds with respect to it: A particle left to
itself and sufficiently far removed from all other particles moves
uniformly in a straight line. With reference to $K$ (Galileian
reference-body) the laws of nature were to be as simple as possible.
But in addition to $K$, all bodies of reference $K'$ should be given
preference in this sense, and they should be exactly equivalent to $K$
for the formulation of natural laws, provided that they are in a state
of uniform rectilinear and non-rotary motion with respect to $K$; all
these bodies of reference are to be regarded as Galileian
reference-bodies. The validity of the principle of relativity was
assumed only for these reference-bodies, but not for others ({\it e.g.}
those possessing motion of a different kind). In this sense we speak
of the special principle of relativity, or special theory of
relativity.

In contrast to this we wish to understand by the ``general principle of
relativity" the following statement: All bodies of reference $K$, $K'$,
etc., are equivalent for the description of natural phenomena
(formulation of the general laws of nature), whatever may be their
state of motion. But before proceeding farther, it ought to be pointed
out that this formulation must be replaced later by a more abstract
one, for reasons which will become evident at a later stage.

Since the introduction of the special principle of relativity has been
justified, every intellect which strives after generalisation must
feel the temptation to venture the step towards the general principle
of relativity. But a simple and apparently quite reliable
consideration seems to suggest that, for the present at any rate,
there is little hope of success in such an attempt; Let us imagine
ourselves transferred to our old friend the railway carriage, which is
travelling at a uniform rate. As long as it is moving unifromly, the
occupant of the carriage is not sensible of its motion, and it is for
this reason that he can without reluctance interpret the facts of the
case as indicating that the carriage is at rest, but the embankment in
motion. Moreover, according to the special principle of relativity,
this interpretation is quite justified also from a physical point of
view.

If the motion of the carriage is now changed into a non-uniform
motion, as for instance by a powerful application of the brakes, then
the occupant of the carriage experiences a correspondingly powerful
jerk forwards. The retarded motion is manifested in the mechanical
behaviour of bodies relative to the person in the railway carriage.
The mechanical behaviour is different from that of the case previously
considered, and for this reason it would appear to be impossible that
the same mechanical laws hold relatively to the non-uniformly moving
carriage, as hold with reference to the carriage when at rest or in
uniform motion. At all events it is clear that the Galileian law does
not hold with respect to the non-uniformly moving carriage. Because of
this, we feel compelled at the present juncture to grant a kind of
absolute physical reality to non-uniform motion, in opposition to the
general principle of relatvity. But in what follows we shall soon see
that this conclusion cannot be maintained.


\chapter{The Gravitational Field}

``f we pick up a stone and then let it go, why does it fall to the
ground?" The usual answer to this question is: ``Because it is
attracted by the earth." Modern physics formulates the answer rather
differently for the following reason. As a result of the more careful
study of electromagnetic phenomena, we have come to regard action at a
distance as a process impossible without the intervention of some
intermediary medium. If, for instance, a magnet attracts a piece of
iron, we cannot be content to regard this as meaning that the magnet
acts directly on the iron through the intermediate empty space, but we
are constrained to imagine---after the manner of Faraday---that the
magnet always calls into being something physically real in the space
around it, that something being what we call a ``magnetic field." In
its turn this magnetic field operates on the piece of iron, so that
the latter strives to move towards the magnet. We shall not discuss
here the justification for this incidental conception, which is indeed
a somewhat arbitrary one. We shall only mention that with its aid
electromagnetic phenomena can be theoretically represented much more
satisfactorily than without it, and this applies particularly to the
transmission of electromagnetic waves. The effects of gravitation also
are regarded in an analogous manner.

The action of the earth on the stone takes place indirectly. The earth
produces in its surrounding a gravitational field, which acts on the
stone and produces its motion of fall. As we know from experience, the
intensity of the action on a body dimishes according to a quite
definite law, as we proceed farther and farther away from the earth.
From our point of view this means: The law governing the properties
of the gravitational field in space must be a perfectly definite one,
in order correctly to represent the diminution of gravitational action
with the distance from operative bodies. It is something like this:
The body ({\it e.g.} the earth) produces a field in its immediate
neighbourhood directly; the intensity and direction of the field at
points farther removed from the body are thence determined by the law
which governs the properties in space of the gravitational fields
themselves.

In contrast to electric and magnetic fields, the gravitational field
exhibits a most remarkable property, which is of fundamental
importance for what follows. Bodies which are moving under the sole
influence of a gravitational field receive an acceleration, which does
not in the least depend either on the material or on the physical
state of the body. For instance, a piece of lead and a piece of wood
fall in exactly the same manner in a gravitational field (in vacuo),
when they start off from rest or with the same initial velocity. This
law, which holds most accurately, can be expressed in a different form
in the light of the following consideration.

According to Newton's law of motion, we have

\begin{center}
(Force) = (inertial mass) $\times$ (acceleration),
\end{center}

\noindent where the ``inertial mass" is a characteristic constant of the
accelerated body. If now gravitation is the cause of the acceleration,
we then have

\begin{center}
(Force) = (gravitational mass) $\times$ (intensity of the gravitational
field),
\end{center}

\noindent where the ``gravitational mass" is likewise a characteristic constant
for the body. From these two relations follows:

$$\mbox{(acceleration)} = \frac{\mbox{gravitational mass}}{\mbox{inertial mass}}
 \times \mbox{intensity of the gravitational field}$$
~

If now, as we find from experience, the acceleration is to be
independent of the nature and the condition of the body and always the
same for a given gravitational field, then the ratio of the
gravitational to the inertial mass must likewise be the same for all
bodies. By a suitable choice of units we can thus make this ratio
equal to unity. We then have the following law: The gravitational mass
of a body is equal to its inertial law.

It is true that this important law had hitherto been recorded in
mechanics, but it had not been interpreted. A satisfactory
interpretation can be obtained only if we recognise the following fact:
The same quality of a body manifests itself according to
circumstances as ``inertia'' or as ``weight'' (lit. ``heaviness''). In
the following section we shall show to what extent this is actually
the case, and how this question is connected with the general
postulate of relativity.



\chapter{The Equality of Inertial and Gravitational Mass
as an Argument for the General Postule of Relativity}

We imagine a large portion of empty space, so far removed from stars
and other appreciable masses, that we have before us approximately the
conditions required by the fundamental law of Galilei. It is then
possible to choose a Galileian reference-body for this part of space
(world), relative to which points at rest remain at rest and points in
motion continue permanently in uniform rectilinear motion. As
reference-body let us imagine a spacious chest resembling a room with
an observer inside who is equipped with apparatus. Gravitation
naturally does not exist for this observer. He must fasten himself
with strings to the floor, otherwise the slightest impact against the
floor will cause him to rise slowly towards the ceiling of the room.

To the middle of the lid of the chest is fixed externally a hook with
rope attached, and now a ``being'' (what kind of a being is immaterial
to us) begins pulling at this with a constant force. The chest
together with the observer then begin to move ``upwards" with a
uniformly accelerated motion. In course of time their velocity will
reach unheard-of values---provided that we are viewing all this from
another reference-body which is not being pulled with a rope.

But how does the man in the chest regard the Process? The
acceleration of the chest will be transmitted to him by the reaction
of the floor of the chest. He must therefore take up this pressure by
means of his legs if he does not wish to be laid out full length on
the floor. He is then standing in the chest in exactly the same way as
anyone stands in a room of a home on our earth. If he releases a body
which he previously had in his land, the accelertion of the chest will
no longer be transmitted to this body, and for this reason the body
will approach the floor of the chest with an accelerated relative
motion. The observer will further convince himself that the
acceleration of the body towards the floor of the chest is always of
the same magnitude, whatever kind of body he may happen to use for the
experiment.

Relying on his knowledge of the gravitational field (as it was
discussed in the preceding section), the man in the chest will thus
come to the conclusion that he and the chest are in a gravitational
field which is constant with regard to time. Of course he will be
puzzled for a moment as to why the chest does not fall in this
gravitational field. just then, however, he discovers the hook in the
middle of the lid of the chest and the rope which is attached to it,
and he consequently comes to the conclusion that the chest is
suspended at rest in the gravitational field.

Ought we to smile at the man and say that he errs in his conclusion?
I do not believe we ought to if we wish to remain consistent; we must
rather admit that his mode of grasping the situation violates neither
reason nor known mechanical laws. Even though it is being accelerated
with respect to the ``Galileian space" first considered, we can
nevertheless regard the chest as being at rest. We have thus good
grounds for extending the principle of relativity to include bodies of
reference which are accelerated with respect to each other, and as a
result we have gained a powerful argument for a generalised postulate
of relativity.

We must note carefully that the possibility of this mode of
interpretation rests on the fundamental property of the gravitational
field of giving all bodies the same acceleration, or, what comes to
the same thing, on the law of the equality of inertial and
gravitational mass. If this natural law did not exist, the man in the
accelerated chest would not be able to interpret the behaviour of the
bodies around him on the supposition of a gravitational field, and he
would not be justified on the grounds of experience in supposing his
reference-body to be ``at rest."

Suppose that the man in the chest fixes a rope to the inner side of
the lid, and that he attaches a body to the free end of the rope. The
result of this will be to strech the rope so that it will hang 
``vertically'' downwards. If we ask for an opinion of the cause of
tension in the rope, the man in the chest will say: ``The suspended
body experiences a downward force in the gravitational field, and this
is neutralised by the tension of the rope; what determines the
magnitude of the tension of the rope is the gravitational mass of the
suspended body." On the other hand, an observer who is poised freely
in space will interpret the condition of things thus: ``The rope must
perforce take part in the accelerated motion of the chest, and it
transmits this motion to the body attached to it. The tension of the
rope is just large enough to effect the acceleration of the body. That
which determines the magnitude of the tension of the rope is the
inertial mass of the body." Guided by this example, we see that our
extension of the principle of relativity implies the necessity of the
law of the equality of inertial and gravitational mass. Thus we have
obtained a physical interpretation of this law.

From our consideration of the accelerated chest we see that a general
theory of relativity must yield important results on the laws of
gravitation. In point of fact, the systematic pursuit of the general
idea of relativity has supplied the laws satisfied by the
gravitational field. Before proceeding farther, however, I must warn
the reader against a misconception suggested by these considerations.
A gravitational field exists for the man in the chest, despite the
fact that there was no such field for the co-ordinate system first
chosen. Now we might easily suppose that the existence of a
gravitational field is always only an apparent one. We might also
think that, regardless of the kind of gravitational field which may be
present, we could always choose another reference-body such that no
gravitational field exists with reference to it. This is by no means
true for all gravitational fields, but only for those of quite special
form. It is, for instance, impossible to choose a body of reference
such that, as judged from it, the gravitational field of the earth (in
its entirety) vanishes.

We can now appreciate why that argument is not convincing, which we
brought forward against the general principle of relativity at theend
of Section 18. It is certainly true that the observer in the
railway carriage experiences a jerk forwards as a result of the
application of the brake, and that he recognises, in this the
non-uniformity of motion (retardation) of the carriage. But he is
compelled by nobody to refer this jerk to a ``real ``acceleration
(retardation) of the carriage. He might also interpret his experience
thus: ``My body of reference (the carriage) remains permanently at
rest. With reference to it, however, there exists (during the period
of application of the brakes) a gravitational field which is directed
forwards and which is variable with respect to time. Under the
influence of this field, the embankment together with the earth moves
non-uniformly in such a manner that their original velocity in the
backwards direction is continuously reduced."



\chapter{In What Respects Are the Foundations of Classical Mechanics and of the
Special Theory of Relativity Unsatisfactory?}


We have already stated several times that classical mechanics starts
out from the following law: Material particles sufficiently far
removed from other material particles continue to move uniformly in a
straight line or continue in a state of rest. We have also repeatedly
emphasised that this fundamental law can only be valid for bodies of
reference $K$ which possess certain unique states of motion, and which
are in uniform translational motion relative to each other. Relative
to other reference-bodies $K$ the law is not valid. Both in classical
mechanics and in the special theory of relativity we therefore
differentiate between reference-bodies $K$ relative to which the
recognised ``laws of nature'' can be said to hold, and
reference-bodies $K$ relative to which these laws do not hold.

But no person whose mode of thought is logical can rest satisfied with
this condition of things. He asks: ``How does it come that certain
reference-bodies (or their states of motion) are given priority over
other reference-bodies (or their states of motion)? What is the
reason for this Preference?'' In order to show clearly what I mean by
this question, I shall make use of a comparison.

I am standing in front of a gas range. Standing alongside of each
other on the range are two pans so much alike that one may be mistaken
for the other. Both are half full of water. I notice that steam is
being emitted continuously from the one pan, but not from the other. I
am surprised at this, even if I have never seen either a gas range or
a pan before. But if I now notice a luminous something of bluish
colour under the first pan but not under the other, I cease to be
astonished, even if I have never before seen a gas flame. For I can
only say that this bluish something will cause the emission of the
steam, or at least possibly it may do so. If, however, I notice the
bluish something in neither case, and if I observe that the one
continuously emits steam whilst the other does not, then I shall
remain astonished and dissatisfied until I have discovered some
circumstance to which I can attribute the different behaviour of the
two pans.

Analogously, I seek in vain for a real something in classical
mechanics (or in the special theory of relativity) to which I can
attribute the different behaviour of bodies considered with respect to
the reference systems $K$ and $K$.\footnotemark\  Newton saw this objection and
attempted to invalidate it, but without success. But E. Mach recognsed
it most clearly of all, and because of this objection he claimed that
mechanics must be placed on a new basis. It can only be got rid of by
means of a physics which is conformable to the general principle of
relativity, since the equations of such a theory hold for every body
of reference, whatever may be its state of motion.


%  Notes

\footnotetext{The objection is of importance more especially when the state of
motion of the reference-body is of such a nature that it does not
require any external agency for its maintenance, {\it e.g.} in the case when
the reference-body is rotating uniformly.}



\chapter{A Few Inferences from the General Principle of Relativity}

The considerations of Section 20 show that the general principle of
relativity puts us in a position to derive properties of the
gravitational field in a purely theoretical manner. Let us suppose,
for instance, that we know the space-time ``course'' for any natural
process whatsoever, as regards the manner in which it takes place in
the Galileian domain relative to a Galileian body of reference $K$. By
means of purely theoretical operations ({\it i.e.} simply by calculation) we
are then able to find how this known natural process appears, as seen
from a reference-body $K'$ which is accelerated relatively to $K$. But
since a gravitational field exists with respect to this new body of
reference $K$, our consideration also teaches us how the gravitational
field influences the process studied.

For example, we learn that a body which is in a state of uniform
rectilinear motion with respect to $K$ (in accordance with the law of
Galilei) is executing an accelerated and in general curvilinear motion
with respect to the accelerated reference-body $K'$ (chest). This
acceleration or curvature corresponds to the influence on the moving
body of the gravitational field prevailing relatively to $K$. It is
known that a gravitational field influences the movement of bodies in
this way, so that our consideration supplies us with nothing
essentially new.

However, we obtain a new result of fundamental importance when we
carry out the analogous consideration for a ray of light. With respect
to the Galileian reference-body $K$, such a ray of light is transmitted
rectilinearly with the velocity $c$. It can easily be shown that the
path of the same ray of light is no longer a straight line when we
consider it with reference to the accelerated chest (reference-body
$K'$). From this we conclude, that, in general, rays of light are
propagated curvilinearly in gravitational fields. In two respects this
result is of great importance.

In the first place, it can be compared with the reality. Although a
detailed examination of the question shows that the curvature of light
rays required by the general theory of relativity is only exceedingly
small for the gravitational fields at our disposal in practice, its
estimated magnitude for light rays passing the sun at grazing
incidence is nevertheless 1.7 seconds of arc. This ought to manifest
itself in the following way. As seen from the earth, certain fixed
stars appear to be in the neighbourhood of the sun, and are thus
capable of observation during a total eclipse of the sun. At such
times, these stars ought to appear to be displaced outwards from the
sun by an amount indicated above, as compared with their apparent
position in the sky when the sun is situated at another part of the
heavens. The examination of the correctness or otherwise of this
deduction is a problem of the greatest importance, the early solution
of which is to be expected of astronomers.\footnotemark

In the second place our result shows that, according to the general
theory of relativity, the law of the constancy of the velocity of
light in vacuo, which constitutes one of the two fundamental
assumptions in the special theory of relativity and to which we have
already frequently referred, cannot claim any unlimited validity. A
curvature of rays of light can only take place when the velocity of
propagation of light varies with position. Now we might think that as
a consequence of this, the special theory of relativity and with it
the whole theory of relativity would be laid in the dust. But in
reality this is not the case. We can only conclude that the special
theory of relativity cannot claim an unlinlited domain of validity;
its results hold only so long as we are able to disregard the
influences of gravitational fields on the phenomena ({\it e.g.} of light).

Since it has often been contended by opponents of the theory of
relativity that the special theory of relativity is overthrown by the
general theory of relativity, it is perhaps advisable to make the
facts of the case clearer by means of an appropriate comparison.
Before the development of electrodynamics the laws of electrostatics
were looked upon as the laws of electricity. At the present time we
know that electric fields can be derived correctly from electrostatic
considerations only for the case, which is never strictly realised, in
which the electrical masses are quite at rest relatively to each
other, and to the co-ordinate system. Should we be justified in saying
that for this reason electrostatics is overthrown by the
field-equations of Maxwell in electrodynamics? Not in the least.
Electrostatics is contained in electrodynamics as a limiting case;
the laws of the latter lead directly to those of the former for the
case in which the fields are invariable with regard to time. No fairer
destiny could be allotted to any physical theory, than that it should
of itself point out the way to the introduction of a more
comprehensive theory, in which it lives on as a limiting case.

In the example of the transmission of light just dealt with, we have
seen that the general theory of relativity enables us to derive
theoretically the influence of a gravitational field on the course of
natural processes, the Iaws of which are already known when a
gravitational field is absent. But the most attractive problem, to the
solution of which the general theory of relativity supplies the key,
concerns the investigation of the laws satisfied by the gravitational
field itself. Let us consider this for a moment.

We are acquainted with space-time domains which behave (approximately)
in a ``Galileian'' fashion under suitable choice of reference-body,
{\it i.e.} domains in which gravitational fields are absent. If we now refer
such a domain to a reference-body $K'$ possessing any kind of motion,
then relative to $K'$ there exists a gravitational field which is
variable with respect to space and time.\footnotemark\  The character of this
field will of course depend on the motion chosen for $K'$. According to
the general theory of relativity, the general law of the gravitational
field must be satisfied for all gravitational fields obtainable in
this way. Even though by no means all gravitationial fields can be
produced in this way, yet we may entertain the hope that the general
law of gravitation will be derivable from such gravitational fields of
a special kind. This hope has been realised in the most beautiful
manner. But between the clear vision of this goal and its actual
realisation it was necessary to surmount a serious difficulty, and as
this lies deep at the root of things, I dare not withhold it from the
reader. We require to extend our ideas of the space-time continuum
still farther.


%  Notes

\footnotetext[1]{By means of the star photographs of two expeditions equipped by
a Joint Committee of the Royal and Royal Astronomical Societies, the
existence of the deflection of light demanded by theory was first
confirmed during the solar eclipse of 29th May, 1919. (Cf. Appendix
III.)}

\footnotetext{This follows from a generalisation of the discussion in
Section 20}


\chapter{Behaviour of Clocks and Measuring-Rods on a Rotating Body of Reference}

Hitherto I have purposely refrained from speaking about the physical
interpretation of space- and time-data in the case of the general
theory of relativity. As a consequence, I am guilty of a certain
slovenliness of treatment, which, as we know from the special theory
of relativity, is far from being unimportant and pardonable. It is now
high time that we remedy this defect; but I would mention at the
outset, that this matter lays no small claims on the patience and on
the power of abstraction of the reader.

We start off again from quite special cases, which we have frequently
used before. Let us consider a space time domain in which no
gravitational field exists relative to a reference-body $K$ whose state
of motion has been suitably chosen. $K$ is then a Galileian
reference-body as regards the domain considered, and the results of
the special theory of relativity hold relative to $K$. Let us supposse
the same domain referred to a second body of reference $K'$, which is
rotating uniformly with respect to $K$. In order to fix our ideas, we
shall imagine $K'$ to be in the form of a plane circular disc, which
rotates uniformly in its own plane about its centre. An observer who
is sitting eccentrically on the disc $K'$ is sensible of a force which
acts outwards in a radial direction, and which would be interpreted as
an effect of inertia (centrifugal force) by an observer who was at
rest with respect to the original reference-body $K$. But the observer
on the disc may regard his disc as a reference-body which is ``at rest'';
on the basis of the general principle of relativity he is
justified in doing this. The force acting on himself, and in fact on
all other bodies which are at rest relative to the disc, he regards as
the effect of a gravitational field. Nevertheless, the
space-distribution of this gravitational field is of a kind that would
not be possible on Newton's theory of gravitation.\footnotemark\ But since the
observer believes in the general theory of relativity, this does not
disturb him; he is quite in the right when he believes that a general
law of gravitation can be formulated---a law which not only explains
the motion of the stars correctly, but also the field of force
experienced by himself.

The observer performs experiments on his circular disc with clocks and
measuring-rods. In doing so, it is his intention to arrive at exact
definitions for the signification of time- and space-data with
reference to the circular disc $K'$, these definitions being based on
his observations. What will be his experience in this enterprise?

To start with, he places one of two identically constructed clocks at
the centre of the circular disc, and the other on the edge of the
disc, so that they are at rest relative to it. We now ask ourselves
whether both clocks go at the same rate from the standpoint of the
non-rotating Galileian reference-body $K$. As judged from this body, the
clock at the centre of the disc has no velocity, whereas the clock at
the edge of the disc is in motion relative to $K$ in consequence of the
rotation. According to a result obtained in Section 12, it follows
that the latter clock goes at a rate permanently slower than that of
the clock at the centre of the circular disc, {\it i.e.} as observed from $K$.
It is obvious that the same effect would be noted by an observer whom
we will imagine sitting alongside his clock at the centre of the
circular disc. Thus on our circular disc, or, to make the case more
general, in every gravitational field, a clock will go more quickly or
less quickly, according to the position in which the clock is situated
(at rest). For this reason it is not possible to obtain a reasonable
definition of time with the aid of clocks which are arranged at rest
with respect to the body of reference. A similar difficulty presents
itself when we attempt to apply our earlier definition of simultaneity
in such a case, but I do not wish to go any farther into this
question.

Moreover, at this stage the definition of the space co-ordinates also
presents insurmountable difficulties. If the observer applies his
standard measuring-rod (a rod which is short as compared with the
radius of the disc) tangentially to the edge of the disc, then, as
judged from the Galileian system, the length of this rod will be less
than I, since, according to Section 12, moving bodies suffer a
shortening in the direction of the motion. On the other hand, the
measaring-rod will not experience a shortening in length, as judged
from $K$, if it is applied to the disc in the direction of the radius.
If, then, the observer first measures the circumference of the disc
with his measuring-rod and then the diameter of the disc, on dividing
the one by the other, he will not obtain as quotient the familiar
number $\pi$ = 3.14 . . ., but a larger number,\footnotemark\ whereas of course,
for a disc which is at rest with respect to $K$, this operation would
yield $\pi$ exactly. This proves that the propositions of Euclidean
geometry cannot hold exactly on the rotating disc, nor in general in a
gravitational field, at least if we attribute the length I to the rod
in all positions and in every orientation. Hence the idea of a
straight line also loses its meaning. We are therefore not in a
position to define exactly the co-ordinates $x, y, z$ relative to the
disc by means of the method used in discussing the special theory, and
as long as the co-ordinates and times of events have not been
defined, we cannot assign an exact meaning to the natural laws in
which these occur.

Thus all our previous conclusions based on general relativity would
appear to be called in question. In reality we must make a subtle
detour in order to be able to apply the postulate of general
relativity exactly. I shall prepare the reader for this in the
following paragraphs.


%  Notes

\footnotetext[1]{The field disappears at the centre of the disc and increases
proportionally to the distance from the centre as we proceed outwards.}

\footnotetext[2]{Throughout this consideration we have to use the Galileian
(non-rotating) system $K$ as reference-body, since we may only assume
the validity of the results of the special theory of relativity
relative to $K$ (relative to $K'$ a gravitational field prevails).}


\chapter{Euclidean and Non-Euclidean Continuum}



The surface of a marble table is spread out in front of me. I can get
from any one point on this table to any other point by passing
continuously from one point to a ``neighbouring'' one, and repeating
this process a (large) number of times, or, in other words, by going
from point to point without executing ``jumps." I am sure the reader
will appreciate with sufficient clearness what I mean here by 
``neighbouring'' and by ``jumps'' (if he is not too pedantic). We
express this property of the surface by describing the latter as a
continuum.

Let us now imagine that a large number of little rods of equal length
have been made, their lengths being small compared with the dimensions
of the marble slab. When I say they are of equal length, I mean that
one can be laid on any other without the ends overlapping. We next lay
four of these little rods on the marble slab so that they constitute a
quadrilateral figure (a square), the diagonals of which are equally
long. To ensure the equality of the diagonals, we make use of a little
testing-rod. To this square we add similar ones, each of which has one
rod in common with the first. We proceed in like manner with each of
these squares until finally the whole marble slab is laid out with
squares. The arrangement is such, that each side of a square belongs
to two squares and each corner to four squares.

It is a veritable wonder that we can carry out this business without
getting into the greatest difficulties. We only need to think of the
following. If at any moment three squares meet at a corner, then two
sides of the fourth square are already laid, and, as a consequence,
the arrangement of the remaining two sides of the square is already
completely determined. But I am now no longer able to adjust the
quadrilateral so that its diagonals may be equal. If they are equal of
their own accord, then this is an especial favour of the marble slab
and of the little rods, about which I can only be thankfully
surprised. We must experience many such surprises if the construction
is to be successful.

If everything has really gone smoothly, then I say that the points of
the marble slab constitute a Euclidean continuum with respect to the
little rod, which has been used as a ``distance'' (line-interval). By
choosing one corner of a square as ``origin" I can characterise every
other corner of a square with reference to this origin by means of two
numbers. I only need state how many rods I must pass over when,
starting from the origin, I proceed towards the ``right'' and then
 ``upwards," in order to arrive at the corner of the square under
consideration. These two numbers are then the ``Cartesian co-ordinates"
of this corner with reference to the ``Cartesian co-ordinate system"
which is determined by the arrangement of little rods.

By making use of the following modification of this abstract
experiment, we recognise that there must also be cases in which the
experiment would be unsuccessful. We shall suppose that the rods 
``expand'' by in amount proportional to the increase of temperature. We
heat the central part of the marble slab, but not the periphery, in
which case two of our little rods can still be brought into
coincidence at every position on the table. But our construction of
squares must necessarily come into disorder during the heating,
because the little rods on the central region of the table expand,
whereas those on the outer part do not.

With reference to our little rods---defined as unit lengths---the
marble slab is no longer a Euclidean continuum, and we are also no
longer in the position of defining Cartesian co-ordinates directly
with their aid, since the above construction can no longer be carried
out. But since there are other things which are not influenced in a
similar manner to the little rods (or perhaps not at all) by the
temperature of the table, it is possible quite naturally to maintain
the point of view that the marble slab is a ``Euclidean continuum."
This can be done in a satisfactory manner by making a more subtle
stipulation about the measurement or the comparison of lengths.

But if rods of every kind ({\it i.e.} of every material) were to behave in
the same way as regards the influence of temperature when they are on
the variably heated marble slab, and if we had no other means of
detecting the effect of temperature than the geometrical behaviour of
our rods in experiments analogous to the one described above, then our
best plan would be to assign the distance one to two points on the
slab, provided that the ends of one of our rods could be made to
coincide with these two points; for how else should we define the
distance without our proceeding being in the highest measure grossly
arbitrary? The method of Cartesian coordinates must then be
discarded, and replaced by another which does not assume the validity
of Euclidean geometry for rigid bodies.\footnotemark\  The reader will notice
that the situation depicted here corresponds to the one brought about
by the general postitlate of relativity (Section 23).


%  Notes

\footnotetext{Mathematicians have been confronted with our problem in the
following form. If we are given a surface ({\it e.g.} an ellipsoid) in
Euclidean three-dimensional space, then there exists for this surface
a two-dimensional geometry, just as much as for a plane surface. Gauss
undertook the task of treating this two-dimensional geometry from
first principles, without making use of the fact that the surface
belongs to a Euclidean continuum of three dimensions. If we imagine
constructions to be made with rigid rods in the surface (similar to
that above with the marble slab), we should find that different laws
hold for these from those resulting on the basis of Euclidean plane
geometry. The surface is not a Euclidean continuum with respect to the
rods, and we cannot define Cartesian co-ordinates in the surface.
Gauss indicated the principles according to which we can treat the
geometrical relationships in the surface, and thus pointed out the way
to the method of Riemman of treating multi-dimensional, non-Euclidean
continuum. Thus it is that mathematicians long ago solved the formal
problems to which we are led by the general postulate of relativity.}



\chapter{Gaussian Co-Ordinates}

% Figure 4
%
%         __ u=1
%       _/
%  ____/    ___ u=2
%    / \___/    ___ u=3
% __/_  _/ \___/
%  /  \/__  _/  \_
%__/__/   \/__    \
% /   \__/    \_   v=3
%/   /  /\__    \
%   /  /    \   v=2
%           v=1

\begin{figure}[bthp]

\centering
\caption{}
\label{fig:4}

%Created by jPicEdt 1.x
%Standard LaTeX format (emulated lines)
%Thu Aug 25 17:20:05 PDT 2005
\unitlength 1mm
\begin{picture}(75.00,55.00)(0,12)

\qbezier(20.00,20.00)(35.00,50.00)(65.00,50.00)
\qbezier(15.00,25.00)(25.00,55.00)(45.00,60.00)
\qbezier(30.00,20.00)(45.00,40.00)(70.00,40.00)

\qbezier(50.00,15.00)(40.00,30.00)(15.00,35.00)
\qbezier(55.00,25.00)(45.00,40.00)(20.00,45.00)
\qbezier(65.00,30.00)(55.00,50.00)(25.00,55.00)

\put(34.00,16.00){\makebox(0,0)[cc]{P}}

\put(34.00,19.00){\vector(1,3){2}}
\put(36.50,27.50){\circle*{1.50}}

\put(54.00,12.500){\makebox(0,0)[cc]{$v=1$}}
\put(59.00,22.50){\makebox(0,0)[cc]{$v=2$}}
\put(69.00,27.50){\makebox(0,0)[cc]{$v=3$}}
\put(52.00,61.00){\makebox(0,0)[cc]{$u=1$}}
\put(72.00,51.00){\makebox(0,0)[cc]{$u=2$}}
\put(77.00,41.00){\makebox(0,0)[cc]{$u=3$}}

\end{picture}

\end{figure}


According to Gauss, this combined analytical and geometrical mode of
handling the problem can be arrived at in the following way. We
imagine a system of arbitrary curves (see Fig. \ref{fig:4}) drawn on the surface
of the table. These we designate as $u$-curves, and we indicate each of
them by means of a number. The curves $u=1$, $u=2$ and $u=3$ are drawn in
the diagram. Between the curves $u=1$ and $u=2$ we must imagine an
infinitely large number to be drawn, all of which correspond to real
numbers lying between 1 and 2. fig. 04 We have then a system of
u-curves, and this ``infinitely dense" system covers the whole surface
of the table. These u-curves must not intersect each other, and
through each point of the surface one and only one curve must pass.
Thus a perfectly definite value of u belongs to every point on the
surface of the marble slab. In like manner we imagine a system of
v-curves drawn on the surface. These satisfy the same conditions as
the u-curves, they are provided with numbers in a corresponding
manner, and they may likewise be of arbitrary shape. It follows that a
value of u and a value of v belong to every point on the surface of
the table. We call these two numbers the co-ordinates of the surface
of the table (Gaussian co-ordinates). For example, the point $P$ in the
diagram has the Gaussian co-ordinates $u=3$, $v=1$. Two neighbouring
points $P$ and $P_1$ on the surface then correspond to the co-ordinates
\begin{eqnarray*} 
P: & u ~~,~~v \\
P': & u + du , v + dv
\end{eqnarray*} 
where $du$ and $dv$ signify very small numbers. In a similar manner we may
indicate the distance (line-interval) between $P$ and $P_1$, as measured
with a little rod, by means of the very small number $ds$. Then
according to Gauss we have

                $$ds_2 = g_{11}du^2 + 2g_{12}dudv = g_{22}dv^2$$

\noindent where $g_{11}, g_{12}, g_{22}$, are magnitudes which depend in a perfectly
definite way on $u$ and $v$. The magnitudes $g_{11}$, $g_{12}$ and $g_{22}$,
determine the behaviour of the rods relative to the $u$-curves and
$v$-curves, and thus also relative to the surface of the table. For the
case in which the points of the surface considered form a Euclidean
continuum with reference to the measuring-rods, but only in this case,
it is possible to draw the $u$-curves and $v$-curves and to attach numbers
to them, in such a manner, that we simply have:

                           $$ds^2 = du^2 + dv^2$$


Under these conditions, the $u$-curves and $v$-curves are straight lines
in the sense of Euclidean geometry, and they are perpendicular to each
other. Here the Gaussian coordinates are simply Cartesian ones. It is
clear that Gauss co-ordinates are nothing more than an association of
two sets of numbers with the points of the surface considered, of such
a nature that numerical values differing very slightly from each other
are associated with neighbouring points ``in space."

So far, these considerations hold for a continuum of two dimensions.
But the Gaussian method can be applied also to a continuum of three,
four or more dimensions. If, for instance, a continuum of four
dimensions be supposed available, we may represent it in the following
way. With every point of the continuum, we associate arbitrarily four
numbers, $x_1, x_2, x_3, x_4$, which are known as ``co-ordinates."
Adjacent points correspond to adjacent values of the coordinates. If a
distance $ds$ is associated with the adjacent points $P$ and $P_1$, this
distance being measurable and well defined from a physical point of
view, then the following formula holds:

$$ds^2 = g_{11}dx_1^2 + 2g_{12}dx_1dx_2 . . . . g_{44}dx_4^2$$

\noindent where the magnitudes g[11], etc., have values which vary with the
position in the continuum. Only when the continuum is a Euclidean one
is it possible to associate the co-ordinates $x_1 \ldots x_4$. with the
points of the continuum so that we have simply

$$ds2 = dx_1^2 + dx_2^2 + dx_3^2 + dx_4^2$$

In this case relations hold in the four-dimensional continuum which
are analogous to those holding in our three-dimensional measurements.

However, the Gauss treatment for $ds^2$ which we have given above is not
always possible. It is only possible when sufficiently small regions
of the continuum under consideration may be regarded as Euclidean
continua. For example, this obviously holds in the case of the marble
slab of the table and local variation of temperature. The temperature
is practically constant for a small part of the slab, and thus the
geometrical behaviour of the rods is almost as it ought to be
according to the rules of Euclidean geometry. Hence the imperfections
of the construction of squares in the previous section do not show
themselves clearly until this construction is extended over a
considerable portion of the surface of the table.

We can sum this up as follows: Gauss invented a method for the
mathematical treatment of continua in general, in which 
``size-relations''`(``distances'' between neighbouring points) are
defined. To every point of a continuum are assigned as many numbers
(Gaussian coordinates) as the continuum has dimensions. This is done
in such a way, that only one meaning can be attached to the
assignment, and that numbers (Gaussian coordinates) which differ by an
indefinitely small amount are assigned to adjacent points. The
Gaussian coordinate system is a logical generalisation of the
Cartesian co-ordinate system. It is also applicable to non-Euclidean
continua, but only when, with respect to the defined ``size'' or
``distance,'' small parts of the continuum under consideration behave
more nearly like a Euclidean system, the smaller the part of the
continuum under our notice.



\chapter{The Space-Time Continuum of the Speical Theory of Relativity Considered as a
Euclidean Continuum}


We are now in a position to formulate more exactly the idea of
Minkowski, which was only vaguely indicated in Section 17. In
accordance with the special theory of relativity, certain co-ordinate
systems are given preference for the description of the
four-dimensional, space-time continuum. We called these ``Galileian
co-ordinate systems." For these systems, the four co-ordinates $x, y,
z, t$, which determine an event or---in other words, a point of the
four-dimensional continuum---are defined physically in a simple
manner, as set forth in detail in the first part of this book. For the
transition from one Galileian system to another, which is moving
uniformly with reference to the first, the equations of the Lorentz
transformation are valid. These last form the basis for the derivation
of deductions from the special theory of relativity, and in themselves
they are nothing more than the expression of the universal validity of
the law of transmission of light for all Galileian systems of
reference.

Minkowski found that the Lorentz transformations satisfy the following
simple conditions. Let us consider two neighbouring events, the
relative position of which in the four-dimensional continuum is given
with respect to a Galileian reference-body $K$ by the space co-ordinate
differences $dx, dy, dz$ and the time-difference $dt$. With reference to a
second Galileian system we shall suppose that the corresponding
differences for these two events are $dx', dy', dz', dt'$. Then these
magnitudes always fulfil the condition\footnotemark.

     $$dx^2 + dy^2 + dz^2 - c^2dt^2 = dx' 2 + dy' 2 + dz' 2 - c^2dt'^2$$

The validity of the Lorentz transformation follows from this
condition. We can express this as follows: The magnitude

                   $$ds^2 = dx^2 + dy^2 + dz^2 - c^2dt^2$$

\noindent which belongs to two adjacent points of the four-dimensional
space-time continuum, has the same value for all selected (Galileian)
reference-bodies. If we replace $x, y, z$, $\sqrt{-I} \cdot ct$ , by $x_1,
x_2, x_3, x_4$, we also obtaill the result that

             $$ds^2 = dx_1^2 + dx_2^2 + dx_3^2 + dx_4^2$$

\noindent is independent of the choice of the body of reference. We call the
magnitude ds the ``distance'' apart of the two events or
four-dimensional points.

Thus, if we choose as time-variable the imaginary variable $\sqrt{-I} \cdot ct$
instead of the real quantity $t$, we can regard the space-time
contintium---accordance with the special theory of relativity---as a
``Euclidean'' four-dimensional continuum, a result which follows from
the considerations of the preceding section.


%  Notes

\footnotetext{Cf. Appendixes I and 2. The relations which are derived
there for the co-ordlnates themselves are valid also for co-ordinate
differences, and thus also for co-ordinate differentials (indefinitely
small differences).}



\chapter{The Space-Time Continuum of the General Theory of Relativity is Not a 
Euclidean Continuum}


In the first part of this book we were able to make use of space-time
co-ordinates which allowed of a simple and direct physical
interpretation, and which, according to Section 26, can be regarded
as four-dimensional Cartesian co-ordinates. This was possible on the
basis of the law of the constancy of the velocity of tight. But
according to Section 21 the general theory of relativity cannot
retain this law. On the contrary, we arrived at the result that
according to this latter theory the velocity of light must always
depend on the co-ordinates when a gravitational field is present. In
connection with a specific illustration in Section 23, we found
that the presence of a gravitational field invalidates the definition
of the coordinates and the ifine, which led us to our objective in the
special theory of relativity.

In view of the resuIts of these considerations we are led to the
conviction that, according to the general principle of relativity, the
space-time continuum cannot be regarded as a Euclidean one, but that
here we have the general case, corresponding to the marble slab with
local variations of temperature, and with which we made acquaintance
as an example of a two-dimensional continuum. Just as it was there
impossible to construct a Cartesian co-ordinate system from equal
rods, so here it is impossible to build up a system (reference-body)
from rigid bodies and clocks, which shall be of such a nature that
measuring-rods and clocks, arranged rigidly with respect to one
another, shaIll indicate position and time directly. Such was the
essence of the difficulty with which we were confronted in Section
23.

But the considerations of Sections 25 and 26 show us the way to
surmount this difficulty. We refer the fourdimensional space-time
continuum in an arbitrary manner to Gauss co-ordinates. We assign to
every point of the continuum (event) four numbers, $x_1, x_2, x_3,
x_4$ (co-ordinates), which have not the least direct physical
significance, but only serve the purpose of numbering the points of
the continuum in a definite but arbitrary manner. This arrangement
does not even need to be of such a kind that we must regard $x_1,
x_2, x_3$, as ``space" co-ordinates and $x_4$, as a ``time'' 
co-ordinate.

The reader may think that such a description of the world would be
quite inadequate. What does it mean to assign to an event the
particular co-ordinates $x_1, x_2, x_3, x_4$, if in themselves these
co-ordinates have no significance? More careful consideration shows,
however, that this anxiety is unfounded. Let us consider, for
instance, a material point with any kind of motion. If this point had
only a momentary existence without duration, then it would to
described in space-time by a single system of values $x_1, x_2, x_3,
x_4$. Thus its permanent existence must be characterised by an
infinitely large number of such systems of values, the co-ordinate
values of which are so close together as to give continuity;
corresponding to the material point, we thus have a (uni-dimensional)
line in the four-dimensional continuum. In the same way, any such
lines in our continuum correspond to many points in motion. The only
statements having regard to these points which can claim a physical
existence are in reality the statements about their encounters. In our
mathematical treatment, such an encounter is expressed in the fact
that the two lines which represent the motions of the points in
question have a particular system of co-ordinate values, $x_1, x_2,
x_3, x_4$, in common. After mature consideration the reader will
doubtless admit that in reality such encounters constitute the only
actual evidence of a time-space nature with which we meet in physical
statements.

When we were describing the motion of a material point relative to a
body of reference, we stated nothing more than the encounters of this
point with particular points of the reference-body. We can also
determine the corresponding values of the time by the observation of
encounters of the body with clocks, in conjunction with the
observation of the encounter of the hands of clocks with particular
points on the dials. It is just the same in the case of
space-measurements by means of measuring-rods, as a litttle
consideration will show.

The following statements hold generally: Every physical description
resolves itself into a number of statements, each of which refers to
the space-time coincidence of two events A and B. In terms of Gaussian
co-ordinates, every such statement is expressed by the agreement of
their four co-ordinates $x_1, x_2, x_3, x_4$. Thus in reality, the
description of the time-space continuum by means of Gauss co-ordinates
completely replaces the description with the aid of a body of
reference, without suffering from the defects of the latter mode of
description; it is not tied down to the Euclidean character of the
continuum which has to be represented.



\chapter{Exact Formulation of the General Principle of Relativity}


We are now in a position to replace the pro. visional formulation of
the general principle of relativity given in Section 18 by an exact
formulation. The form there used, ``All bodies of reference $K, K^1,$
etc., are equivalent for the description of natural phenomena
(formulation of the general laws of nature), whatever may be their
state of motion," cannot be maintained, because the use of rigid
reference-bodies, in the sense of the method followed in the special
theory of relativity, is in general not possible in space-time
description. The Gauss co-ordinate system has to take the place of the
body of reference. The following statement corresponds to the
fundamental idea of the general principle of relativity: ``All Gaussian
co-ordinate systems are essentially equivalent for the formulation of
the general laws of nature."

We can state this general principle of relativity in still another
form, which renders it yet more clearly intelligible than it is when
in the form of the natural extension of the special principle of
relativity. According to the special theory of relativity, the
equations which express the general laws of nature pass over into
equations of the same form when, by making use of the Lorentz
transformation, we replace the space-time variables $x, y, z, t$, of a
(Galileian) reference-body $K$ by the space-time variables $x^1, y^1, z^1,
t^1$, of a new reference-body $K^1$. According to the general theory of
relativity, on the other hand, by application of arbitrary
substitutions of the Gauss variables $x_1, x_2, x_3, x_4$, the
equations must pass over into equations of the same form; for every
transformation (not only the Lorentz transformation) corresponds to
the transition of one Gauss co-ordinate system into another.

If we desire to adhere to our ``old-time" three-dimensional view of
things, then we can characterise the development which is being
undergone by the fundamental idea of the general theory of relativity
as follows: The special theory of relativity has reference to
Galileian domains, {\it i.e.} to those in which no gravitational field
exists. In this connection a Galileian reference-body serves as body
of reference, {\it i.e.} a rigid body the state of motion of which is so
chosen that the Galileian law of the uniform rectilinear motion of
``isolated" material points holds relatively to it.

Certain considerations suggest that we should refer the same Galileian
domains to non-Galileian reference-bodies also. A gravitational field
of a special kind is then present with respect to these bodies (cf.
Sections 20 and 23).

In gravitational fields there are no such things as rigid bodies with
Euclidean properties; thus the fictitious rigid body of reference is
of no avail in the general theory of relativity. The motion of clocks
is also influenced by gravitational fields, and in such a way that a
physical definition of time which is made directly with the aid of
clocks has by no means the same degree of plausibility as in the
special theory of relativity.

For this reason non-rigid reference-bodies are used, which are as a
whole not only moving in any way whatsoever, but which also suffer
alterations in form {\it ad lib.} during their motion. Clocks, for which the
law of motion is of any kind, however irregular, serve for the
definition of time. We have to imagine each of these clocks fixed at a
point on the non-rigid reference-body. These clocks satisfy only the
one condition, that the ``readings" which are observed simultaneously
on adjacent clocks (in space) differ from each other by an
indefinitely small amount. This non-rigid reference-body, which might
appropriately be termed a ``reference-mollusc", is in the main
equivalent to a Gaussian four-dimensional co-ordinate system chosen
arbitrarily. That which gives the ``mollusc" a certain
comprehensibility as compared with the Gauss co-ordinate system is the
(really unjustified) formal retention of the separate existence of the
space co-ordinates as opposed to the time co-ordinate. Every point on
the mollusc is treated as a space-point, and every material point
which is at rest relatively to it as at rest, so long as the mollusc
is considered as reference-body. The general principle of relativity
requires that all these molluscs can be used as reference-bodies with
equal right and equal success in the formulation of the general laws
of nature; the laws themselves must be quite independent of the choice
of mollusc.

The great power possessed by the general principle of relativity lies
in the comprehensive limitation which is imposed on the laws of nature
in consequence of what we have seen above.



\chapter{The Solution of the Problem of Gravitation on the Basis of the General
Principle of Relativity}


If the reader has followed all our previous considerations, he will
have no further difficulty in understanding the methods leading to the
solution of the problem of gravitation.

We start off on a consideration of a Galileian domain, {\it i.e.} a domain
in which there is no gravitational field relative to the Galileian
reference-body $K$. The behaviour of measuring-rods and clocks with
reference to K is known from the special theory of relativity,
likewise the behaviour of ``isolated" material points; the latter move
uniformly and in straight lines.

Now let us refer this domain to a random Gauss coordinate system or to
a ``mollusc" as reference-body $K^1$. Then with respect to $K^1$ there is a
gravitational field $G$ (of a particular kind). We learn the behavior
of measuring-rods and clocks and also of freely-moving material points
with reference to $K^1$ simply by mathematical transformation. We
interpret this behaviour as the behaviour of measuring-rods, docks and
material points tinder the influence of the gravitational field $G$.
Hereupon we introduce a hypothesis: that the influence of the
gravitational field on measuringrods, clocks and freely-moving
material points continues to take place according to the same laws,
even in the case where the prevailing gravitational field is not
derivable from the Galfleian special care, simply by means of a
transformation of co-ordinates.

The next step is to investigate the space-time behaviour of the
gravitational field $G$, which was derived from the Galileian special
case simply by transformation of the coordinates. This behaviour is
formulated in a law, which is always valid, no matter how the
reference-body (mollusc) used in the description may be chosen.

This law is not yet the general law of the gravitational field, since
the gravitational field under consideration is of a special kind. In
order to find out the general law-of-field of gravitation we still
require to obtain a generalisation of the law as found above. This can
be obtained without caprice, however, by taking into consideration the
following demands:

\begin{enumerate}
\item The required generalisation must likewise satisfy the general
postulate of relativity.
\item If there is any matter in the domain under consideration, only its
inertial mass, and thus according to Section 15 only its energy is
of importance for its etfect in exciting a field.
\item Gravitational field and matter together must satisfy the law of
the conservation of energy (and of impulse).
\end{enumerate}

Finally, the general principle of relativity permits us to determine
the influence of the gravitational field on the course of all those
processes which take place according to known laws when a
gravitational field is absent {\it i.e.} which have already been fitted into
the frame of the special theory of relativity. In this connection we
proceed in principle according to the method which has already been
explained for measuring-rods, clocks and freely moving material
points.

The theory of gravitation derived in this way from the general
postulate of relativity excels not only in its beauty; nor in
removing the defect attaching to classical mechanics which was brought
to light in Section 21; nor in interpreting the empirical law of
the equality of inertial and gravitational mass; but it has also
already explained a result of observation in astronomy, against which
classical mechanics is powerless.

If we confine the application of the theory to the case where the
gravitational fields can be regarded as being weak, and in which all
masses move with respect to the coordinate system with velocities
which are small compared with the velocity of light, we then obtain as
a first approximation the Newtonian theory. Thus the latter theory is
obtained here without any particular assumption, whereas Newton had to
introduce the hypothesis that the force of attraction between mutually
attracting material points is inversely proportional to the square of
the distance between them. If we increase the accuracy of the
calculation, deviations from the theory of Newton make their
appearance, practically all of which must nevertheless escape the test
of observation owing to their smallness.

We must draw attention here to one of these deviations. According to
Newton's theory, a planet moves round the sun in an ellipse, which
would permanently maintain its position with respect to the fixed
stars, if we could disregard the motion of the fixed stars themselves
and the action of the other planets under consideration. Thus, if we
correct the observed motion of the planets for these two influences,
and if Newton's theory be strictly correct, we ought to obtain for the
orbit of the planet an ellipse, which is fixed with reference to the
fixed stars. This deduction, which can be tested with great accuracy,
has been confirmed for all the planets save one, with the precision
that is capable of being obtained by the delicacy of observation
attainable at the present time. The sole exception is Mercury, the
planet which lies nearest the sun. Since the time of Leverrier, it has
been known that the ellipse corresponding to the orbit of Mercury,
after it has been corrected for the influences mentioned above, is not
stationary with respect to the fixed stars, but that it rotates
exceedingly slowly in the plane of the orbit and in the sense of the
orbital motion. The value obtained for this rotary movement of the
orbital ellipse was 43 seconds of arc per century, an amount ensured
to be correct to within a few seconds of arc. This effect can be
explained by means of classical mechanics only on the assumption of
hypotheses which have little probability, and which were devised
solely for this purponse.

On the basis of the general theory of relativity, it is found that the
ellipse of every planet round the sun must necessarily rotate in the
manner indicated above; that for all the planets, with the exception
of Mercury, this rotation is too small to be detected with the
delicacy of observation possible at the present time; but that in the
case of Mercury it must amount to 43 seconds of arc per century, a
result which is strictly in agreement with observation.

Apart from this one, it has hitherto been possible to make only two
deductions from the theory which admit of being tested by observation,
to wit, the curvature of light rays by the gravitational field of the
sun\footnotemark[1], and a displacement of the spectral lines of light reaching
us from large stars, as compared with the corresponding lines for
light produced in an analogous manner terrestrially ({\it i.e.} by the same
kind of atom)\footnotemark[2].  These two deductions from the theory have both
been confirmed.


%  Notes

\footnotetext[1]{First observed by Eddington and others in 1919. (Cf. Appendix
III, pp. 126-129).}

\footnotetext[2]{Established by Adams in 1924. (Cf. p. 132)}




%PART III

\part{Considerations on the Universe as a Whole}


\chapter{Cosmological Difficulties of Newton's Theory}


Part from the difficulty discussed in Section 21, there is a second
fundamental difficulty attending classical celestial mechanics, which,
to the best of my knowledge, was first discussed in detail by the
astronomer Seeliger. If we ponder over the question as to how the
universe, considered as a whole, is to be regarded, the first answer
that suggests itself to us is surely this: As regards space (and time)
the universe is infinite. There are stars everywhere, so that the
density of matter, although very variable in detail, is nevertheless
on the average everywhere the same. In other words: However far we
might travel through space, we should find everywhere an attenuated
swarm of fixed stars of approrimately the same kind and density.

This view is not in harmony with the theory of Newton. The latter
theory rather requires that the universe should have a kind of centre
in which the density of the stars is a maximum, and that as we proceed
outwards from this centre the group-density of the stars should
diminish, until finally, at great distances, it is succeeded by an
infinite region of emptiness. The stellar universe ought to be a
finite island in the infinite ocean of space\footnotemark.

This conception is in itself not very satisfactory. It is still less
satisfactory because it leads to the result that the light emitted by
the stars and also individual stars of the stellar system are
perpetually passing out into infinite space, never to return, and
without ever again coming into interaction with other objects of
nature. Such a finite material universe would be destined to become
gradually but systematically impoverished.

In order to escape this dilemma, Seeliger suggested a modification of
Newton's law, in which he assumes that for great distances the force
of attraction between two masses diminishes more rapidly than would
result from the inverse square law. In this way it is possible for the
mean density of matter to be constant everywhere, even to infinity,
without infinitely large gravitational fields being produced. We thus
free ourselves from the distasteful conception that the material
universe ought to possess something of the nature of a centre. Of
course we purchase our emancipation from the fundamental difficulties
mentioned, at the cost of a modification and complication of Newton's
law which has neither empirical nor theoretical foundation. We can
imagine innumerable laws which would serve the same purpose, without
our being able to state a reason why one of them is to be preferred to
the others; for any one of these laws would be founded just as little
on more general theoretical principles as is the law of Newton.


%  Notes

\footnotetext[1]{Proof---According to the theory of Newton, the number of  ``lines
of force" which come from infinity and terminate in a mass $m$ is
proportional to the mass $m$. If, on the average, the Mass density $p_0$
is constant throughout tithe universe, then a sphere of volume $V$ will
enclose the average mass $p_0V$. Thus the number of lines of force
passing through the surface $F$ of the sphere into its interior is
proportional to $p_0 V$. For unit area of the surface of the sphere the
number of lines of force which enters the sphere is thus proportional
to $p_0 V/F$ or to $p_0R$. Hence the intensity of the field at the
surface would ultimately become infinite with increasing radius $R$ of
the sphere, which is impossible.}



\chapter{The Possibility of a ``Finite" and yet ``Unbounded" Universe}


But speculations on the structure of the universe also move in quite
another direction. The development of non-Euclidean geometry led to
the recognition of the fact, that we can cast doubt on the
infiniteness of our space without coming into conflict with the laws
of thought or with experience (Riemann, Helmholtz). These questions
have already been treated in detail and with unsurpassable lucidity by
Helmholtz and Poincar\'{e}, whereas I can only touch on them briefly here.

In the first place, we imagine an existence in two dimensional space.
Flat beings with flat implements, and in particular flat rigid
measuring-rods, are free to move in a plane. For them nothing exists
outside of this plane: that which they observe to happen to themselves
and to their flat ``things'' is the all-inclusive reality of their
plane. In particular, the constructions of plane Euclidean geometry
can be carried out by means of the rods {\it e.g.} the lattice construction,
considered in Section 24. In contrast to ours, the universe of
these beings is two-dimensional; but, like ours, it extends to
infinity. In their universe there is room for an infinite number of
identical squares made up of rods, {\it i.e.} its volume (surface) is
infinite. If these beings say their universe is ``plane," there is
sense in the statement, because they mean that they can perform the
constructions of plane Euclidean geometry with their rods. In this
connection the individual rods always represent the same distance,
independently of their position.

Let us consider now a second two-dimensional existence, but this time
on a spherical surface instead of on a plane. The flat beings with
their measuring-rods and other objects fit exactly on this surface and
they are unable to leave it. Their whole universe of observation
extends exclusively over the surface of the sphere. Are these beings
able to regard the geometry of their universe as being plane geometry
and their rods withal as the realisation of ``distance''? They cannot
do this. For if they attempt to realise a straight line, they will
obtain a curve, which we ``three-dimensional beings'' designate as a
great circle, {\it i.e.} a self-contained line of definite finite length,
which can be measured up by means of a measuring-rod. Similarly, this
universe has a finite area that can be compared with the area, of a
square constructed with rods. The great charm resulting from this
consideration lies in the recognition of the fact that the universe of
these beings is finite and yet has no limits.

But the spherical-surface beings do not need to go on a world-tour in
order to perceive that they are not living in a Euclidean universe.
They can convince themselves of this on every part of their ``world,"
provided they do not use too small a piece of it. Starting from a
point, they draw ``straight lines'' (arcs of circles as judged in
three dimensional space) of equal length in all directions. They will
call the line joining the free ends of these lines a ``circle." For a
plane surface, the ratio of the circumference of a circle to its
diameter, both lengths being measured with the same rod, is, according
to Euclidean geometry of the plane, equal to a constant value $\pi$, which
is independent of the diameter of the circle. On their spherical
surface our flat beings would find for this ratio the value

                        $$\pi \frac{\sin \frac{r}{R}}{\frac{r}{R}}$$
{\it i.e.} a smaller value than $\pi$, the difference being the more
considerable, the greater is the radius of the circle in comparison
with the radius $R$ of the ``world-sphere." By means of this relation
the spherical beings can determine the radius of their universe 
(``world''), even when only a relatively small part of their worldsphere
is available for their measurements. But if this part is very small
indeed, they will no longer be able to demonstrate that they are on a
spherical ``world'' and not on a Euclidean plane, for a small part of
a spherical surface differs only slightly from a piece of a plane of
the same size.

Thus if the spherical surface beings are living on a planet of which
the solar system occupies only a negligibly small part of the
spherical universe, they have no means of determining whether they are
living in a finite or in an infinite universe, because the ``piece of
universe'' to which they have access is in both cases practically
plane, or Euclidean. It follows directly from this discussion, that
for our sphere-beings the circumference of a circle first increases
with the radius until the ``circumference of the universe'' is
reached, and that it thenceforward gradually decreases to zero for
still further increasing values of the radius. During this process the
area of the circle continues to increase more and more, until finally
it becomes equal to the total area of the whole ``world-sphere."

Perhaps the reader will wonder why we have placed our ``beings ``on a
sphere rather than on another closed surface. But this choice has its
justification in the fact that, of all closed surfaces, the sphere is
unique in possessing the property that all points on it are
equivalent. I admit that the ratio of the circumference $c$ of a circle
to its radius $r$ depends on $r$, but for a given value of $r$ it is the
same for all points of the ``worldsphere''; in other words, the ``
world-sphere'' is a ``surface of constant curvature."

To this two-dimensional sphere-universe there is a three-dimensional
analogy, namely, the three-dimensional spherical space which was
discovered by Riemann. its points are likewise all equivalent. It
possesses a finite volume, which is determined by its ``radius"
($2\pi^2R^3$). Is it possible to imagine a spherical space? To imagine a
space means nothing else than that we imagine an epitome of our 
``space'' experience, {\it i.e.} of experience that we can have in the
movement of ``rigid'' bodies. In this sense we can imagine a spherical
space.

Suppose we draw lines or stretch strings in all directions from a
point, and mark off from each of these the distance r with a
measuring-rod. All the free end-points of these lengths lie on a
spherical surface. We can specially measure up the area ($F$) of this
surface by means of a square made up of measuring-rods. If the
universe is Euclidean, then $F = 4\pi R^2$; if it is spherical, then $F$ is
always less than $4\pi R^2$. With increasing values of $r$, $F$ increases from
zero up to a maximum value which is determined by the ``world-radius,"
but for still further increasing values of $r$, the area gradually
diminishes to zero. At first, the straight lines which radiate from
the starting point diverge farther and farther from one another, but
later they approach each other, and finally they run together again at
a ``counter-point" to the starting point. Under such conditions they
have traversed the whole spherical space. It is easily seen that the
three-dimensional spherical space is quite analogous to the
two-dimensional spherical surface. It is finite ({\it i.e.} of finite
volume), and has no bounds.

It may be mentioned that there is yet another kind of curved space: 
``elliptical space." It can be regarded as a curved space in which the
two ``counter-points'' are identical (indistinguishable from each
other). An elliptical universe can thus be considered to some extent
as a curved universe possessing central symmetry.

It follows from what has been said, that closed spaces without limits
are conceivable. From amongst these, the spherical space (and the
elliptical) excels in its simplicity, since all points on it are
equivalent. As a result of this discussion, a most interesting
question arises for astronomers and physicists, and that is whether
the universe in which we live is infinite, or whether it is finite in
the manner of the spherical universe. Our experience is far from being
sufficient to enable us to answer this question. But the general
theory of relativity permits of our answering it with a moduate degree
of certainty, and in this connection the difficulty mentioned in
Section 30 finds its solution.



\chapter{The Structure of Space According to the General Theory of Relativity}


According to the general theory of relativity, the geometrical
properties of space are not independent, but they are determined by
matter. Thus we can draw conclusions about the geometrical structure
of the universe only if we base our considerations on the state of the
matter as being something that is known. We know from experience that,
for a suitably chosen co-ordinate system, the velocities of the stars
are small as compared with the velocity of transmission of light. We
can thus as a rough approximation arrive at a conclusion as to the
nature of the universe as a whole, if we treat the matter as being at
rest.

We already know from our previous discussion that the behaviour of
measuring-rods and clocks is influenced by gravitational fields, {\it i.e.}
by the distribution of matter. This in itself is sufficient to exclude
the possibility of the exact validity of Euclidean geometry in our
universe. But it is conceivable that our universe differs only
slightly from a Euclidean one, and this notion seems all the more
probable, since calculations show that the metrics of surrounding
space is influenced only to an exceedingly small extent by masses even
of the magnitude of our sun. We might imagine that, as regards
geometry, our universe behaves analogously to a surface which is
irregularly curved in its individual parts, but which nowhere departs
appreciably from a plane: something like the rippled surface of a
lake. Such a universe might fittingly be called a quasi-Euclidean
universe. As regards its space it would be infinite. But calculation
shows that in a quasi-Euclidean universe the average density of matter
would necessarily be {\it nil}. Thus such a universe could not be inhabited
by matter everywhere; it would present to us that unsatisfactory
picture which we portrayed in Section 30.

If we are to have in the universe an average density of matter which
differs from zero, however small may be that difference, then the
universe cannot be quasi-Euclidean. On the contrary, the results of
calculation indicate that if matter be distributed uniformly, the
universe would necessarily be spherical (or elliptical). Since in
reality the detailed distribution of matter is not uniform, the real
universe will deviate in individual parts from the spherical, {\it i.e.} the
universe will be quasi-spherical. But it will be necessarily finite.
In fact, the theory supplies us with a simple connection\footnotemark  between
the space-expanse of the universe and the average density of matter in
it.


%  Notes

\footnotetext{For the radius R of the universe we obtain the equation

                        $$R^2=\frac{2}{\kappa  p}$$

The use of the C.G.S. system in this equation gives $2/k = 1^.08 \cdot 10^{27}$;
$p$ is the average density of the matter and $k$ is a constant connected
with the Newtonian constant of gravitation.}



%APPENDIX I

\appendix

\chapter{Simple Derivation of the Lorentz Transformation
(Supplementary to Section 11)}


For the relative orientation of the co-ordinate systems indicated in
Fig. 2, the x-axes of both systems pernumently coincide. In the
present case we can divide the problem into parts by considering first
only events which are localised on the $x$-axis. Any such event is
represented with respect to the co-ordinate system $K$ by the abscissa $x$
and the time $t$, and with respect to the system $K^1$ by the abscissa $x'$
and the time $t'$. We require to find $x'$ and $t'$ when $x$ and $t$ are given.

A light-signal, which is proceeding along the positive axis of $x$, is
transmitted according to the equation

                                $$x = ct$$
or
\begin{equation}
\label{eqn:a1}
                                   x - ct = 0
\end{equation}
 
Since the same light-signal has to be transmitted relative to $K^1$ with
the velocity $c$, the propagation relative to the system $K^1$ will be
represented by the analogous formula

\begin{equation}
\label{eqn:a2}
                                   x' - ct' = 0
\end{equation}

Those space-time points (events) which satisfy (\ref{eqn:a1}) must also satisfy
(\ref{eqn:a2}). Obviously this will be the case when the relation

\begin{equation}
\label{eqn:a3}
          (x' - ct') = \lambda (x - ct)
\end{equation}

\noindent is fulfilled in general, where $\lambda$ indicates a constant; for, according
to (\ref{eqn:a3}), the disappearance of $(x - ct)$ involves the disappearance of
$(x' - ct')$.

If we apply quite similar considerations to light rays which are being
transmitted along the negative x-axis, we obtain the condition

\begin{equation}
\label{eqn:a4}
           (x' + ct') = \mu (x + ct)
\end{equation}

By adding (or subtracting) equations (\ref{eqn:a3}) and (\ref{eqn:a4}), and introducing for
convenience the constants $a$ and $b$ in place of the constants $\lambda$ and $\mu$,
where

                        $$a = \frac{\lambda+\mu}{2}$$

\noindent and

                        $$a = \frac{\lambda-\mu}{2}$$  % ??

\noindent we obtain the equations

\begin{equation}
\label{eqn:a5}
           \left. \begin{array}{rcl} x' &=& ax-bct \\ ct' &=& act-bx \end{array} \right\} 
\end{equation}

We should thus have the solution of our problem, if the constants $a$
and $b$ were known. These result from the following discussion.

For the origin of $K^1$ we have permanently $x' = 0$, and hence according
to the first of the equations (\ref{eqn:a5})

                        $$x = \frac{bc}{a}t$$

If we call $v$ the velocity with which the origin of $K^1$ is moving
relative to $K$, we then have

\begin{equation}
\label{eqn:a6}
                        v=\frac{bc}{a}
\end{equation}

The same value $v$ can be obtained from equations (\ref{eqn:a5}), if we calculate
the velocity of another point of $K^1$ relative to $K$, or the velocity
(directed towards the negative $x$-axis) of a point of $K$ with respect to
$K'$. In short, we can designate $v$ as the relative velocity of the two
systems.

Furthermore, the principle of relativity teaches us that, as judged
from $K$, the length of a unit measuring-rod which is at rest with
reference to $K^1$ must be exactly the same as the length, as judged from
$K'$, of a unit measuring-rod which is at rest relative to $K$. In order
to see how the points of the $x$-axis appear as viewed from $K$, we only
require to take a ``snapshot'' of $K^1$ from $K$; this means that we have
to insert a particular value of $t$ (time of $K$), {\it e.g.} $t = 0$. For this
value of $t$ we then obtain from the first of the equations (5)

                               $$x' = ax$$

Two points of the $x'$-axis which are separated by the distance $\Delta x' = I$
when measured in the $K^1$ system are thus separated in our instantaneous
photograph by the distance

\begin{equation}
\label{eqn:a7}
                        \Delta x = \frac{I}{a}
\end{equation}

\noindent But if the snapshot be taken from $K'(t' = 0)$, and if we eliminate $t$
from the equations (\ref{eqn:a5}), taking into account the expression (\ref{eqn:a6}), we
obtain

                        $$x' = a \left( I - \frac{v^2}{c^2} \right) x$$

\noindent From this we conclude that two points on the $x$-axis separated by the
distance $I$ (relative to $K$) will be represented on our snapshot by the
distance

                        $$\Delta x' = a \left( I - \frac{v^2}{c^2} \right) \quad . \quad . \quad . \quad \mbox{(7a)}$$

But from what has been said, the two snapshots must be identical;
hence $\Delta x$ in (7) must be equal to $\Delta x'$ in (7a), so that we obtain

                        $$a = \frac{I}{I-\frac{v^2}{c^2}} \quad . \quad . \quad . \quad \mbox{(7b)} $$

The equations (\ref{eqn:a6}) and (7b) determine the constants $a$ and $b$. By
inserting the values of these constants in (\ref{eqn:a5}), we obtain the first
and the fourth of the equations given in Section 11.

\begin{equation}
\label{eqn:a8}
           \left. \begin{array}{rcl} 
           x' &=& \frac{x-vt}{\sqrt{I-\frac{v^2}{c^2}}} \\ 
           ~ \\
           t' &=& \frac{t-\frac{v}{c^2}x}{\sqrt{I-\frac{v^2}{c^2}}} \end{array} \right\}  
\end{equation}

Thus we have obtained the Lorentz transformation for events on the
$x$-axis. It satisfies the condition

         $$x'^2 - c^2t'^2 = x^2 - c^2t^2 \quad . \quad . \quad . \quad \mbox{(8a)} $$

The extension of this result, to include events which take place
outside the $x$-axis, is obtained by retaining equations (\ref{eqn:a8}) and
supplementing them by the relations

\begin{equation}
\label{eqn:a9}
           \left. \begin{array}{rcl} y' &=& y \\ z' &=& z \end{array} \right\}
\end{equation}

In this way we satisfy the postulate of the constancy of the velocity
of light in vacuo for rays of light of arbitrary direction, both for
the system $K$ and for the system $K'$. This may be shown in the following
manner.

We suppose a light-signal sent out from the origin of $K$ at the time $t
= 0$. It will be propagated according to the equation

                        $$r = \sqrt{x^2+y^2+z^2} = ct$$

\noindent or, if we square this equation, according to the equation

\begin{equation}
\label{eqn:a10}
          x^2 + y^2 + z^2 = c^2t^2 = 0
\end{equation}

It is required by the law of propagation of light, in conjunction with
the postulate of relativity, that the transmission of the signal in
question should take place---as judged from $K^1$---in accordance with
the corresponding formula

                               $$r' = ct'$$
                               
\noindent or,

       $$x'^2 + y'^2 + z'^2 - c^2t'^2 = 0  \quad . \quad . \quad . \quad \mbox{(10a)} $$

In order that equation (10a) may be a consequence of equation (\ref{eqn:a10}), we
must have

\begin{equation}
\label{eqn:a11}
             x'^2 + y'^2 + z'^2 - c^2t'^2 = \sigma (x^2 + y^2 + z^2 - c^2t^2)
\end{equation}


Since equation (8a) must hold for points on the $x$-axis, we thus have $\sigma
= I$. It is easily seen that the Lorentz transformation really
satisfies equation (\ref{eqn:a11}) for $\sigma = I$; for (\ref{eqn:a11}) is a consequence of (8a)
and (\ref{eqn:a9}), and hence also of (\ref{eqn:a8}) and (\ref{eqn:a9}). We have thus derived the
Lorentz transformation.

The Lorentz transformation represented by  (\ref{eqn:a8}) and (\ref{eqn:a9}) still requires
to be generalised. Obviously it is immaterial whether the axes of $K^1$
be chosen so that they are spatially parallel to those of $K$. It is
also not essential that the velocity of translation of $K^1$ with respect
to $K$ should be in the direction of the $x$-axis. A simple consideration
shows that we are able to construct the Lorentz transformation in this
general sense from two kinds of transformations, {\it viz.} from Lorentz
transformations in the special sense and from purely spatial
transformations. which corresponds to the replacement of the
rectangular co-ordinate system by a new system with its axes pointing
in other directions.

Mathematically, we can characterise the generalised Lorentz
transformation thus:

It expresses $x', y', x', t'$, in terms of linear homogeneous functions
of $x, y, x, t$, of such a kind that the relation

     $$x'^2 + y'^2 + z'^2 - c^2t'^2 = x^2 + y^2 + z^2 - c^2t^2  \quad . \quad . \quad . \quad \mbox{(11a)} $$

\noindent is satisficd identically. That is to say: If we substitute their
expressions in $x, y, x, t$, in place of $x', y', x', t'$, on the
left-hand side, then the left-hand side of (11a) agrees with the
right-hand side.



% APPENDIX II

\chapter{MINKOWSKI'S FOUR-DIMENSIONAL SPACE (``WORLD")
(SUPPLEMENTARY TO SECTION 17)}


We can characterise the Lorentz transformation still more simply if we
introduce the imaginary $\sqrt{-I} \cdot ct$ in place of $t$, as time-variable. If, in
accordance with this, we insert
\begin{eqnarray*}
                              x_1 & = & x \\
                              x_2 & = & y \\
                              x_3 & = & z \\
                              x_4 & = & \sqrt{-I} \cdot ct
\end{eqnarray*}
and similarly for the accented system $K^1$, then the condition which is
identically satisfied by the transformation can be expressed thus:

$${x'_1}^2 + {x'}_2^2 + {x'}_3^2 + {x'}_4^2 = x_1^2 + x_2^2 + x_3^2 + x_4^2   \quad . \quad . \quad . \quad \mbox{(12)}.$$

\noindent That is, by the afore-mentioned choice of ``coordinates," (11a) [see
the end of Appendix II] is transformed into this equation.

We see from (12) that the imaginary time co-ordinate $x_4$, enters into
the condition of transformation in exactly the same way as the space
co-ordinates $x_1, x_2, x_3$. It is due to this fact that, according
to the theory of relativity, the ``time'' $x_4$, enters into natural
laws in the same form as the space co ordinates $x_1, x_2, x_3$.

A four-dimensional continuum described by the ``co-ordinates" $x_1,
x_2, x_3, x_4$, was called ``world" by Minkowski, who also termed a
point-event a ``world-point." From a ``happening'' in three-dimensional
space, physics becomes, as it were, an ``existence ``in the
four-dimensional ``world."

This four-dimensional ``world'' bears a close similarity to the
three-dimensional ``space'' of (Euclidean) analytical geometry. If we
introduce into the latter a new Cartesian co-ordinate system ($x'_1,
x'_2, x'_3$) with the same origin, then $x'_1, x'_2, x'_3$, are
linear homogeneous functions of $x_1, x_2, x_3$ which identically
satisfy the equation

        $${x'}_1^2 + {x'}_2^2 + {x'}_3^2 = x_1^2 + x_2^2 + x_3^2$$

The analogy with (12) is a complete one. We can regard Minkowski's ``world'' 
in a formal manner as a four-dimensional Euclidean space (with
an imaginary time coordinate); the Lorentz transformation corresponds
to a ``rotation'' of the co-ordinate system in the four-dimensional 
``world."



%APPENDIX III

\chapter{The Experimental Confirmation of the General Theory of Relativity}


From a systematic theoretical point of view, we may imagine the
process of evolution of an empirical science to be a continuous
process of induction. Theories are evolved and are expressed in short
compass as statements of a large number of individual observations in
the form of empirical laws, from which the general laws can be
ascertained by comparison. Regarded in this way, the development of a
science bears some resemblance to the compilation of a classified
catalogue. It is, as it were, a purely empirical enterprise.

But this point of view by no means embraces the whole of the actual
process; for it slurs over the important part played by intuition and
deductive thought in the development of an exact science. As soon as a
science has emerged from its initial stages, theoretical advances are
no longer achieved merely by a process of arrangement. Guided by
empirical data, the investigator rather develops a system of thought
which, in general, is built up logically from a small number of
fundamental assumptions, the so-called axioms. We call such a system
of thought a {\it theory}. The theory finds the justification for its
existence in the fact that it correlates a large number of single
observations, and it is just here that the ``truth'' of the theory
lies.

Corresponding to the same complex of empirical data, there may be
several theories, which differ from one another to a considerable
extent. But as regards the deductions from the theories which are
capable of being tested, the agreement between the theories may be so
complete that it becomes difficult to find any deductions in which the
two theories differ from each other. As an example, a case of general
interest is available in the province of biology, in the Darwinian
theory of the development of species by selection in the struggle for
existence, and in the theory of development which is based on the
hypothesis of the hereditary transmission of acquired characters.

We have another instance of far-reaching agreement between the
deductions from two theories in Newtonian mechanics on the one hand,
and the general theory of relativity on the other. This agreement goes
so far, that up to the preseat we have been able to find only a few
deductions from the general theory of relativity which are capable of
investigation, and to which the physics of pre-relativity days does
not also lead, and this despite the profound difference in the
fundamental assumptions of the two theories. In what follows, we shall
again consider these important deductions, and we shall also discuss
the empirical evidence appertaining to them which has hitherto been
obtained.

\section{Motion of the Perihelion of Mercury}

According to Newtonian mechanics and Newton's law of gravitation, a
planet which is revolving round the sun would describe an ellipse
round the latter, or, more correctly, round the common centre of
gravity of the sun and the planet. In such a system, the sun, or the
common centre of gravity, lies in one of the foci of the orbital
ellipse in such a manner that, in the course of a planet-year, the
distance sun-planet grows from a minimum to a maximum, and then
decreases again to a minimum. If instead of Newton's law we insert a
somewhat different law of attraction into the calculation, we find
that, according to this new law, the motion would still take place in
such a manner that the distance sun-planet exhibits periodic
variations; but in this case the angle described by the line joining
sun and planet during such a period (from perihelion--closest
proximity to the sun--to perihelion) would differ from $360^\circ$. The line
of the orbit would not then be a closed one but in the course of time
it would fill up an annular part of the orbital plane, viz. between
the circle of least and the circle of greatest distance of the planet
from the sun.

According also to the general theory of relativity, which differs of
course from the theory of Newton, a small variation from the
Newton-Kepler motion of a planet in its orbit should take place, and
in such away, that the angle described by the radius sun-planet
between one perhelion and the next should exceed that corresponding to
one complete revolution by an amount given by

                        $$+ \frac{24\pi^3a^2}{T^2e^2(I-e^2)}$$

\noindent (N.B. -- One complete revolution corresponds to the angle $2\pi$ in the
absolute angular measure customary in physics, and the above
expression giver the amount by which the radius sun-planet exceeds
this angle during the interval between one perihelion and the next.)
In this expression $a$ represents the major semi-axis of the ellipse, $e$
its eccentricity, $c$ the velocity of light, and $T$ the period of
revolution of the planet. Our result may also be stated as follows:
According to the general theory of relativity, the major axis of the
ellipse rotates round the sun in the same sense as the orbital motion
of the planet. Theory requires that this rotation should amount to 43
seconds of arc per century for the planet Mercury, but for the other
Planets of our solar system its magnitude should be so small that it
would necessarily escape detection.\footnotemark

In point of fact, astronomers have found that the theory of Newton
does not suffice to calculate the observed motion of Mercury with an
exactness corresponding to that of the delicacy of observation
attainable at the present time. After taking account of all the
disturbing influences exerted on Mercury by the remaining planets, it
was found (Leverrier: 1859; and Newcomb: 1895) that an unexplained
perihelial movement of the orbit of Mercury remained over, the amount
of which does not differ sensibly from the above mentioned +43 seconds
of arc per century. The uncertainty of the empirical result amounts to
a few seconds only.

\section{Deflection of Light by a Gravitational Field}

In Section 22 it has been already mentioned that according to the
general theory of relativity, a ray of light will experience a
curvature of its path when passing through a gravitational field, this
curvature being similar to that experienced by the path of a body
which is projected through a gravitational field. As a result of this
theory, we should expect that a ray of light which is passing close to
a heavenly body would be deviated towards the latter. For a ray of
light which passes the sun at a distance of $\Delta$ sun-radii from its
centre, the angle of deflection (a) should amount to

                        $$a = \frac{1.7 \mbox{seconds of arc}}{\Delta}$$

It may be added that, according to the theory, half of Figure 05 this
deflection is produced by the Newtonian field of attraction of the
sun, and the other half by the geometrical modification (``curvature") 
of space caused by the sun.

\begin{figure}[hbtp]

\centering
\caption{}
\label{fig:5}

%              / D1
%             /
%       /    /
%       /   /
%      /   /
%      /D /
% S( )/--/
%     / /
% D1 / / D2
%    //
%    /
%  _/
% 


\begin{picture}(110,250)(0,30)
\thicklines
\put(38,138){\circle{15}}
\put(22,135){S}

\multiput(5,45)(15,15){2}{\line(1,1){10}}
\multiput(30,70)(5,20){6}{\line(1,4){3}}
\multiput(30,70)(10,20){4}{\line(1,2){5}}
\multiput(70,150)(5,20){6}{\line(1,4){3}}

\put(40,90){\vector(1,2){5}}
\put(35,90){\vector(1,4){3}}
\put(90,230){\vector(1,4){3}}

\put(15,100){$D_1$}
\put(50,90){$D_2$}
\put(100,230){$D_1$}

\put(45,135){\line(3,-1){15}}
\put(50,137){$\Delta$}

\end{picture}

\end{figure}


This result admits of an experimental test by means of the
photographic registration of stars during a total eclipse of the sun.
The only reason why we must wait for a total eclipse is because at
every other time the atmosphere is so strongly illuminated by the
light from the sun that the stars situated near the sun's disc are
invisible. The predicted effect can be seen clearly from the
accompanying diagram. If the sun (S) were not present, a star which is
practically infinitely distant would be seen in the direction $D_1$, as
observed front the earth. But as a consequence of the deflection of
light from the star by the sun, the star will be seen in the direction
$D_2$, {\it i.e.} at a somewhat greater distance from the centre of the sun
than corresponds to its real position.

In practice, the question is tested in the following way. The stars in
the neighborhood of the sun are photographed during a solar eclipse.
In addition, a second photograph of the same stars is taken when the
sun is situated at another position in the sky, {\it i.e.} a few months
earlier or later. As compared whh the standard photograph, the
positions of the stars on the eclipse-photograph ought to appear
displaced radially outwards (away from the centre of the sun) by an
amount corresponding to the angle a.

We are indebted to the [British] Royal Society and to the Royal
Astronomical Society for the investigation of this important
deduction. Undaunted by the [first world] war and by difficulties of
both a material and a psychological nature aroused by the war, these
societies equipped two expeditions---to Sobral (Brazil), and to the
island of Principe (West Africa)---and sent several of Britain's most
celebrated astronomers (Eddington, Cottingham, Crommelin, Davidson),
in order to obtain photographs of the solar eclipse of 29th May, 1919.
The relative discrepancies to be expected between the stellar
photographs obtained during the eclipse and the comparison photographs
amounted to a few hundredths of a millimetre only. Thus great accuracy
was necessary in making the adjustments required for the taking of the
photographs, and in their subsequent measurement.

The results of the measurements confirmed the theory in a thoroughly
satisfactory manner. The rectangular components of the observed and of
the calculated deviations of the stars (in seconds of arc) are set
forth in the following table of results:

%                      Table 01:
$$
\begin{array}{r|rr|rr}
\mbox{Number of the Star} & \mbox{First} & \mbox{Co-ordinate~~} & \mbox{Second} & \mbox{Co-ordinate~~}  \\
\hline
 & \mbox{Observed} & \mbox{Calculated} & \mbox{Observed} & \mbox{Calculated} \\
11 & -0'19 & -0'22 & +0'16 & +0'02 \\
5 & +0'29 & +0'31 & -0'46 & -0'43 \\
4 & +0'11 & +0'10 & +0'83 & +0'73 \\
3 & +0'22 & +0'12 & +1'00 & +0'87 \\
6 & +0'10 & +0'04 & +0'57 & +0'40 \\
10 & -0'08 & +0'09 & +0'35 & +0'32 \\
2 & +'095 & +0'85 & -0'27 & -0'09
\end{array}
$$

\section{Displacement of Spectral Lines Towards the Red}

In Section 23 it has been shown that in a system $K^1$ which is in
rotation with regard to a Galileian system $K$, clocks of identical
construction, and which are considered at rest with respect to the
rotating reference-body, go at rates which are dependent on the
positions of the clocks. We shall now examine this dependence
quantitatively. A clock, which is situated at a distance $r$ from the
centre of the disc, has a velocity relative to $K$ which is given by

                                $$V = \omega r$$

\noindent where $\omega$ represents the angular velocity of rotation of the disc $K^1$
with respect to $K$. If $v_0$, represents the number of ticks of the
clock per unit time (``rate'' of the clock) relative to $K$ when the
clock is at rest, then the ``rate'' of the clock ($v$) when it is moving
relative to $K$ with a velocity $V$, but at rest with respect to the disc,
will, in accordance with Section 12, be given by

                        $$v = v_2\sqrt{I-\frac{v^2}{c^2}}$$

\noindent or with sufficient accuracy by

                        $$v = v_0 \left( I-\frac{1}{2} \frac{v^2}{c^2} \right)$$

\noindent This expression may also be stated in the following form:

                        $$v = v_0 \left( I-\frac{1}{c^2} \frac{\omega^2r^2}{2} \right)$$

If we represent the difference of potential of the centrifugal force
between the position of the clock and the centre of the disc by $\phi$,
{\it i.e.} the work, considered negatively, which must be performed on the
unit of mass against the centrifugal force in order to transport it
from the position of the clock on the rotating disc to the centre of
the disc, then we have

                        $$\phi = \frac{\omega^2r^2}{2}$$

\noindent From this it follows that

                        $$v = v_0 \left( I + \frac{\phi}{c^2} \right)$$

In the first place, we see from this expression that two clocks of
identical construction will go at different rates when situated at
different distances from the centre of the disc. This result is aiso
valid from the standpoint of an observer who is rotating with the
disc.

Now, as judged from the disc, the latter is in a gravititional field
of potential $\phi$, hence the result we have obtained will hold quite
generally for gravitational fields. Furthermore, we can regard an atom
which is emitting spectral lines as a clock, so that the following
statement will hold:

{\it An atom absorbs or emits light of a frequency which is dependent on
the potential of the gravitational field in which it is situated.}

The frequency of an atom situated on the surface of a heavenly body
will be somewhat less than the frequency of an atom of the same
element which is situated in free space (or on the surface of a
smaller celestial body).

Now $\phi = - K (M/r)$, where $K$ is Newton's constant of gravitation, and $M$
is the mass of the heavenly body. Thus a displacement towards the red
ought to take place for spectral lines produced at the surface of
stars as compared with the spectral lines of the same element produced
at the surface of the earth, the amount of this displacement being

                        $$\frac{v_0-v}{v_0}  = \frac{K}{c^2} \frac{M}{r}$$

For the sun, the displacement towards the red predicted by theory
amounts to about two millionths of the wave-length. A trustworthy
calculation is not possible in the case of the stars, because in
general neither the mass $M$ nor the radius $r$ are known.

It is an open question whether or not this effect exists, and at the
present time (1920) astronomers are working with great zeal towards
the solution. Owing to the smallness of the effect in the case of the
sun, it is difficult to form an opinion as to its existence. Whereas
Grebe and Bachem (Bonn), as a result of their own measurements and
those of Evershed and Schwarzschild on the cyanogen bands, have placed
the existence of the effect almost beyond doubt, while other
investigators, particularly St. John, have been led to the opposite
opinion in consequence of their measurements.

Mean displacements of lines towards the less refrangible end of the
spectrum are certainly revealed by statistical investigations of the
fixed stars; but up to the present the examination of the available
data does not allow of any definite decision being arrived at, as to
whether or not these displacements are to be referred in reality to
the effect of gravitation. The results of observation have been
collected together, and discussed in detail from the standpoint of the
question which has been engaging our attention here, in a paper by E.
Freundlich entitled ``Zur Pr�fung der allgemeinen
Relativit\"ats-Theorie" ({\it Die Naturwissenschaften}, 1919, No. 35,
p. 520: Julius Springer, Berlin).

At all events, a definite decision will be reached during the next few
years. If the displacement of spectral lines towards the red by the
gravitational potential does not exist, then the general theory of
relativity will be untenable. On the other hand, if the cause of the
displacement of spectral lines be definitely traced to the
gravitational potential, then the study of this displacement will
furnish us with important information as to the mass of the heavenly
bodies. \footnotemark


%  Notes

\footnotetext[1]{Especially since the next planet Venus has an orbit that is
almost an exact circle, which makes it more difficult to locate the
perihelion with precision.}

\footnotetext[2]{The displacentent of spectral lines towards the red end of the
spectrum was definitely established by Adams in 1924, by observations
on the dense companion of Sirius, for which the effect is about thirty
times greater than for the Sun. R.W.L. -- translator}



%APPENDIX IV
\chapter{The Structure of Space According to the General Theory of Relativity
(Supplementary to Section 32)}

Since the publication of the first edition of this little book, our
knowledge about the structure of space in the large (``cosmological
problem'') has had an important development, which ought to be
mentioned even in a popular presentation of the subject.

My original considerations on the subject were based on two
hypotheses:

\begin{enumerate}
\item There exists an average density of matter in the whole of space
which is everywhere the same and different from zero.

\item The magnitude (``radius'') of space is independent of time.
\end{enumerate}

Both these hypotheses proved to be consistent, according to the
general theory of relativity, but only after a hypothetical term was
added to the field equations, a term which was not required by the
theory as such nor did it seem natural from a theoretical point of
view (``cosmological term of the field equations'').

Hypothesis (2) appeared unavoidable to me at the time, since I thought
that one would get into bottomless speculations if one departed from
it.

However, already in the 'twenties, the Russian mathematician Friedman
showed that a different hypothesis was natural from a purely
theoretical point of view. He realized that it was possible to
preserve hypothesis (1) without introducing the less natural
cosmological term into the field equations of gravitation, if one was
ready to drop hypothesis (2). Namely, the original field equations
admit a solution in which the ``world radius'' depends on time
(expanding space). In that sense one can say, according to Friedman,
that the theory demands an expansion of space.

A few years later Hubble showed, by a special investigation of the
extra-galactic nebulae (``milky ways''), that the spectral lines
emitted showed a red shift which increased regularly with the distance
of the nebulae. This can be interpreted in regard to our present
knowledge only in the sense of Doppler's principle, as an expansive
motion of the system of stars in the large---as required, according
to Friedman, by the field equations of gravitation. Hubble's discovery
can, therefore, be considered to some extent as a confirmation of the
theory.

There does arise, however, a strange difficulty. The interpretation of
the galactic line-shift discovered by Hubble as an expansion (which
can hardly be doubted from a theoretical point of view), leads to an
origin of this expansion which lies ``only'' about $10^9$ years ago,
while physical astronomy makes it appear likely that the development
of individual stars and systems of stars takes considerably longer. It
is in no way known how this incongruity is to be overcome.

I further want to rernark that the theory of expanding space, together
with the empirical data of astronomy, permit no decision to be reached
about the finite or infinite character of (three-dimensional) space,
while the original ``static'' hypothesis of space yielded the closure
(finiteness) of space.

\newpage

~\\
$K$ = co-ordinate system \\
$x, y$ = two-dimensional co-ordinates \\
$x, y, z$ = three-dimensional co-ordinates \\
$x, y, z, t$ = four-dimensional co-ordinates \\

~\\
$t$ = time \\
$I$ = distance \\
$v$ = velocity \\

~\\
$F$ = force \\
$G$ = gravitational field


%
%                GNU Free Documentation License
%                   Version 1.1, March 2000

% Copyright (C) 2000  Free Software Foundation, Inc.
%     59 Temple Place, Suite 330, Boston, MA  02111-1307  USA
% Everyone is permitted to copy and distribute verbatim copies
% of this license document, but changing it is not allowed.

%
%0. PREAMBLE

%The purpose of this License is to make a manual, textbook, or other
%written document ``free" in the sense of freedom: to assure everyone
%the effective freedom to copy and redistribute it, with or without
%modifying it, either commercially or noncommercially.  Secondarily,
%this License preserves for the author and publisher a way to get
%credit for their work, while not being considered responsible for
%modifications made by others.

%This License is a kind of ``copyleft", which means that derivative
%works of the document must themselves be free in the same sense.  It
%complements the GNU General Public License, which is a copyleft
%license designed for free software.

%We have designed this License in order to use it for manuals for free
%software, because free software needs free documentation: a free
%program should come with manuals providing the same freedoms that the
%software does.  But this License is not limited to software manuals;
%it can be used for any textual work, regardless of subject matter or
%whether it is published as a printed book.  We recommend this License
%principally for works whose purpose is instruction or reference.

%
%1. APPLICABILITY AND DEFINITIONS

%This License applies to any manual or other work that contains a
%notice placed by the copyright holder saying it can be distributed
%under the terms of this License.  The ``Document", below, refers to any
%such manual or work.  Any member of the public is a licensee, and is
%addressed as ``you".

%A ``Modified Version" of the Document means any work containing the
%Document or a portion of it, either copied verbatim, or with
%modifications and/or translated into another language.

%A ``Secondary Section" is a named appendix or a front-matter section of
%the Document that deals exclusively with the relationship of the
%publishers or authors of the Document to the Document's overall subject
%(or to related matters) and contains nothing that could fall directly
%within that overall subject.  (For example, if the Document is in part a
%textbook of mathematics, a Secondary Section may not explain any
%mathematics.)  The relationship could be a matter of historical
%connection with the subject or with related matters, or of legal,
%commercial, philosophical, ethical or political position regarding
%them.

%The ``Invariant Sections" are certain Secondary Sections whose titles
%are designated, as being those of Invariant Sections, in the notice
%that says that the Document is released under this License.

%The ``Cover Texts" are certain short passages of text that are listed,
%as Front-Cover Texts or Back-Cover Texts, in the notice that says that
%the Document is released under this License.

%A ``Transparent" copy of the Document means a machine-readable copy,
%represented in a format whose specification is available to the
%general public, whose contents can be viewed and edited directly and
%straightforwardly with generic text editors or (for images composed of
%pixels) generic paint programs or (for drawings) some widely available
%drawing editor, and that is suitable for input to text formatters or
%for automatic translation to a variety of formats suitable for input
%to text formatters.  A copy made in an otherwise Transparent file
%format whose markup has been designed to thwart or discourage
%subsequent modification by readers is not Transparent.  A copy that is
%not ``Transparent" is called ``Opaque".

%Examples of suitable formats for Transparent copies include plain
%ASCII without markup, Texinfo input format, LaTeX input format, SGML
%or XML using a publicly available DTD, and standard-conforming simple
%HTML designed for human modification.  Opaque formats include
%PostScript, PDF, proprietary formats that can be read and edited only
%by proprietary word processors, SGML or XML for which the DTD and/or
%processing tools are not generally available, and the
%machine-generated HTML produced by some word processors for output
%purposes only.

%The ``Title Page" means, for a printed book, the title page itself,
%plus such following pages as are needed to hold, legibly, the material
%this License requires to appear in the title page.  For works in
%formats which do not have any title page as such, ``Title Page" means
%the text near the most prominent appearance of the work's title,
%preceding the beginning of the body of the text.

%
%2. VERBATIM COPYING

%You may copy and distribute the Document in any medium, either
%commercially or noncommercially, provided that this License, the
%copyright notices, and the license notice saying this License applies
%to the Document are reproduced in all copies, and that you add no other
%conditions whatsoever to those of this License.  You may not use
%technical measures to obstruct or control the reading or further
%copying of the copies you make or distribute.  However, you may accept
%compensation in exchange for copies.  If you distribute a large enough
%number of copies you must also follow the conditions in section 3.

%You may also lend copies, under the same conditions stated above, and
%you may publicly display copies.

%
%3. COPYING IN QUANTITY

%If you publish printed copies of the Document numbering more than 100,
%and the Document's license notice requires Cover Texts, you must enclose
%the copies in covers that carry, clearly and legibly, all these Cover
%Texts: Front-Cover Texts on the front cover, and Back-Cover Texts on
%the back cover.  Both covers must also clearly and legibly identify
%you as the publisher of these copies.  The front cover must present
%the full title with all words of the title equally prominent and
%visible.  You may add other material on the covers in addition.
%Copying with changes limited to the covers, as long as they preserve
%the title of the Document and satisfy these conditions, can be treated
%as verbatim copying in other respects.

%If the required texts for either cover are too voluminous to fit
%legibly, you should put the first ones listed (as many as fit
%reasonably) on the actual cover, and continue the rest onto adjacent
%pages.

%If you publish or distribute Opaque copies of the Document numbering
%more than 100, you must either include a machine-readable Transparent
%copy along with each Opaque copy, or state in or with each Opaque copy
%a publicly-accessible computer-network location containing a complete
%Transparent copy of the Document, free of added material, which the
%general network-using public has access to download anonymously at no
%charge using public-standard network protocols.  If you use the latter
%option, you must take reasonably prudent steps, when you begin
%distribution of Opaque copies in quantity, to ensure that this
%Transparent copy will remain thus accessible at the stated location
%until at least one year after the last time you distribute an Opaque
%copy (directly or through your agents or retailers) of that edition to
%the public.

%It is requested, but not required, that you contact the authors of the
%Document well before redistributing any large number of copies, to give
%them a chance to provide you with an updated version of the Document.

%
%4. MODIFICATIONS

%You may copy and distribute a Modified Version of the Document under
%the conditions of sections 2 and 3 above, provided that you release
%the Modified Version under precisely this License, with the Modified
%Version filling the role of the Document, thus licensing distribution
%and modification of the Modified Version to whoever possesses a copy
%of it.  In addition, you must do these things in the Modified Version:

%A. Use in the Title Page (and on the covers, if any) a title distinct
%   from that of the Document, and from those of previous versions
%   (which should, if there were any, be listed in the History section
%   of the Document).  You may use the same title as a previous version
%   if the original publisher of that version gives permission.
%B. List on the Title Page, as authors, one or more persons or entities
%   responsible for authorship of the modifications in the Modified
%   Version, together with at least five of the principal authors of the
%   Document (all of its principal authors, if it has less than five).
%C. State on the Title page the name of the publisher of the
%   Modified Version, as the publisher.
%D. Preserve all the copyright notices of the Document.
%E. Add an appropriate copyright notice for your modifications
%   adjacent to the other copyright notices.
%F. Include, immediately after the copyright notices, a license notice
%   giving the public permission to use the Modified Version under the
%   terms of this License, in the form shown in the Addendum below.
%G. Preserve in that license notice the full lists of Invariant Sections
%   and required Cover Texts given in the Document's license notice.
%H. Include an unaltered copy of this License.
%I. Preserve the section entitled ``History", and its title, and add to
%   it an item stating at least the title, year, new authors, and
%   publisher of the Modified Version as given on the Title Page.  If
%   there is no section entitled ``History" in the Document, create one
%   stating the title, year, authors, and publisher of the Document as
%   given on its Title Page, then add an item describing the Modified
%   Version as stated in the previous sentence.
%J. Preserve the network location, if any, given in the Document for
%   public access to a Transparent copy of the Document, and likewise
%   the network locations given in the Document for previous versions
%   it was based on.  These may be placed in the ``History" section.
%   You may omit a network location for a work that was published at
%   least four years before the Document itself, or if the original
%   publisher of the version it refers to gives permission.
%K. In any section entitled ``Acknowledgements" or ``Dedications",
%   preserve the section's title, and preserve in the section all the
%   substance and tone of each of the contributor acknowledgements
%   and/or dedications given therein.
%L. Preserve all the Invariant Sections of the Document,
%   unaltered in their text and in their titles.  Section numbers
%   or the equivalent are not considered part of the section titles.
%M. Delete any section entitled ``Endorsements".  Such a section
%   may not be included in the Modified Version.
%N. Do not retitle any existing section as ``Endorsements"
%   or to conflict in title with any Invariant Section.

%If the Modified Version includes new front-matter sections or
%appendices that qualify as Secondary Sections and contain no material
%copied from the Document, you may at your option designate some or all
%of these sections as invariant.  To do this, add their titles to the
%list of Invariant Sections in the Modified Version's license notice.
%These titles must be distinct from any other section titles.

%You may add a section entitled ``Endorsements", provided it contains
%nothing but endorsements of your Modified Version by various
%parties--for example, statements of peer review or that the text has
%been approved by an organization as the authoritative definition of a
%standard.

%You may add a passage of up to five words as a Front-Cover Text, and a
%passage of up to 25 words as a Back-Cover Text, to the end of the list
%of Cover Texts in the Modified Version.  Only one passage of
%Front-Cover Text and one of Back-Cover Text may be added by (or
%through arrangements made by) any one entity.  If the Document already
%includes a cover text for the same cover, previously added by you or
%by arrangement made by the same entity you are acting on behalf of,
%you may not add another; but you may replace the old one, on explicit
%permission from the previous publisher that added the old one.

%The author(s) and publisher(s) of the Document do not by this License
%give permission to use their names for publicity for or to assert or
%imply endorsement of any Modified Version.

%
%5. COMBINING DOCUMENTS

%You may combine the Document with other documents released under this
%License, under the terms defined in section 4 above for modified
%versions, provided that you include in the combination all of the
%Invariant Sections of all of the original documents, unmodified, and
%list them all as Invariant Sections of your combined work in its
%license notice.

%The combined work need only contain one copy of this License, and
%multiple identical Invariant Sections may be replaced with a single
%copy.  If there are multiple Invariant Sections with the same name but
%different contents, make the title of each such section unique by
%adding at the end of it, in parentheses, the name of the original
%author or publisher of that section if known, or else a unique number.
%Make the same adjustment to the section titles in the list of
%Invariant Sections in the license notice of the combined work.

%In the combination, you must combine any sections entitled ``History"
%in the various original documents, forming one section entitled
%"History"; likewise combine any sections entitled ``Acknowledgements",
%and any sections entitled ``Dedications".  You must delete all sections
%entitled ``Endorsements."

%
%6. COLLECTIONS OF DOCUMENTS

%You may make a collection consisting of the Document and other documents
%released under this License, and replace the individual copies of this
%License in the various documents with a single copy that is included in
%the collection, provided that you follow the rules of this License for
%verbatim copying of each of the documents in all other respects.

%You may extract a single document from such a collection, and distribute
%it individually under this License, provided you insert a copy of this
%License into the extracted document, and follow this License in all
%other respects regarding verbatim copying of that document.

%
%7. AGGREGATION WITH INDEPENDENT WORKS

%A compilation of the Document or its derivatives with other separate
%and independent documents or works, in or on a volume of a storage or
%distribution medium, does not as a whole count as a Modified Version
%of the Document, provided no compilation copyright is claimed for the
%compilation.  Such a compilation is called an ``aggregate", and this
%License does not apply to the other self-contained works thus compiled
%with the Document, on account of their being thus compiled, if they
%are not themselves derivative works of the Document.

%If the Cover Text requirement of section 3 is applicable to these
%copies of the Document, then if the Document is less than one quarter
%of the entire aggregate, the Document's Cover Texts may be placed on
%covers that surround only the Document within the aggregate.
%Otherwise they must appear on covers around the whole aggregate.

%
%8. TRANSLATION

%Translation is considered a kind of modification, so you may
%distribute translations of the Document under the terms of section 4.
%Replacing Invariant Sections with translations requires special
%permission from their copyright holders, but you may include
%translations of some or all Invariant Sections in addition to the
%original versions of these Invariant Sections.  You may include a
%translation of this License provided that you also include the
%original English version of this License.  In case of a disagreement
%between the translation and the original English version of this
%License, the original English version will prevail.

%
%9. TERMINATION

%You may not copy, modify, sublicense, or distribute the Document except
%as expressly provided for under this License.  Any other attempt to
%copy, modify, sublicense or distribute the Document is void, and will
%automatically terminate your rights under this License.  However,
%parties who have received copies, or rights, from you under this
%License will not have their licenses terminated so long as such
%parties remain in full compliance.

%
%10. FUTURE REVISIONS OF THIS LICENSE

%The Free Software Foundation may publish new, revised versions
%of the GNU Free Documentation License from time to time.  Such new
%versions will be similar in spirit to the present version, but may
%differ in detail to address new problems or concerns.  See
%http://www.gnu.org/copyleft/.

%Each version of the License is given a distinguishing version number.
%If the Document specifies that a particular numbered version of this
%License ``or any later version" applies to it, you have the option of
%following the terms and conditions either of that specified version or
%of any later version that has been published (not as a draft) by the
%Free Software Foundation.  If the Document does not specify a version
%number of this License, you may choose any version ever published (not
%as a draft) by the Free Software Foundation.

%
%ADDENDUM: How to use this License for your documents

%To use this License in a document you have written, include a copy of
%the License in the document and put the following copyright and
%license notices just after the title page:

%      Copyright (c)  YEAR  YOUR NAME.
%      Permission is granted to copy, distribute and/or modify this document
%      under the terms of the GNU Free Documentation License, Version 1.1
%      or any later version published by the Free Software Foundation;
%      with the Invariant Sections being LIST THEIR TITLES, with the
%      Front-Cover Texts being LIST, and with the Back-Cover Texts being LIST.
%      A copy of the license is included in the section entitled ``GNU
%      Free Documentation License".

%If you have no Invariant Sections, write ``with no Invariant Sections"
%instead of saying which ones are invariant.  If you have no
%Front-Cover Texts, write ``no Front-Cover Texts" instead of
%"Front-Cover Texts being LIST"; likewise for Back-Cover Texts.

%If your document contains nontrivial examples of program code, we
%recommend releasing these examples in parallel under your choice of
%free software license, such as the GNU General Public License,
%to permit their use in free software.

%

%

%*** END OF THE PROJECT GUTENBERG EBOOK, RELATIVITY ***

%This file should be named relat10.tex

%Corrected EDITIONS of our eBooks get a new NUMBER, relat11.txt
%VERSIONS based on separate sources get new LETTER, relat10a.txt

%Project Gutenberg eBooks are often created from several printed
%editions, all of which are confirmed as Public Domain in the US
%unless a copyright notice is included.  Thus, we usually do not
%keep eBooks in compliance with any particular paper edition.

%We are now trying to release all our eBooks one year in advance
%of the official release dates, leaving time for better editing.
%Please be encouraged to tell us about any error or corrections,
%even years after the official publication date.

%Please note neither this listing nor its contents are final til
%midnight of the last day of the month of any such announcement.
%The official release date of all Project Gutenberg eBooks is at
%Midnight, Central Time, of the last day of the stated month.  A
%preliminary version may often be posted for suggestion, comment
%and editing by those who wish to do so.

%Most people start at our Web sites at:
%http://gutenberg.net or
%http://promo.net/pg

%These Web sites include award-winning information about Project
%Gutenberg, including how to donate, how to help produce our new
%eBooks, and how to subscribe to our email newsletter (free!).

%
%Those of you who want to download any eBook before announcement
%can get to them as follows, and just download by date.  This is
%also a good way to get them instantly upon announcement, as the
%indexes our cataloguers produce obviously take a while after an
%announcement goes out in the Project Gutenberg Newsletter.

%http://www.ibiblio.org/gutenberg/etext03 or
%ftp://ftp.ibiblio.org/pub/docs/books/gutenberg/etext03

%Or /etext02, 01, 00, 99, 98, 97, 96, 95, 94, 93, 92, 92, 91 or 90

%Just search by the first five letters of the filename you want,
%as it appears in our Newsletters.

%
%Information about Project Gutenberg (one page)

%We produce about two million dollars for each hour we work.  The
%time it takes us, a rather conservative estimate, is fifty hours
%to get any eBook selected, entered, proofread, edited, copyright
%searched and analyzed, the copyright letters written, etc.   Our
%projected audience is one hundred million readers.  If the value
%per text is nominally estimated at one dollar then we produce \$2
%million dollars per hour in 2002 as we release over 100 new text
%files per month:  1240 more eBooks in 2001 for a total of 4000+
%We are already on our way to trying for 2000 more eBooks in 2002
%If they reach just 1-2\% of the world's population then the total
%will reach over half a trillion eBooks given away by year's end.

%The Goal of Project Gutenberg is to Give Away 1 Trillion eBooks!
%This is ten thousand titles each to one hundred million readers,
%which is only about 4\% of the present number of computer users.

%Here is the briefest record of our progress (* means estimated):

%eBooks Year Month

%    1  1971 July
%   10  1991 January
%  100  1994 January
% 1000  1997 August
% 1500  1998 October
% 2000  1999 December
% 2500  2000 December
% 3000  2001 November
% 4000  2001 October/November
% 6000  2002 December*
% 9000  2003 November*
%10000  2004 January*

%
%The Project Gutenberg Literary Archive Foundation has been created
%to secure a future for Project Gutenberg into the next millennium.

%We need your donations more than ever!

%As of February, 2002, contributions are being solicited from people
%and organizations in: Alabama, Alaska, Arkansas, Connecticut,
%Delaware, District of Columbia, Florida, Georgia, Hawaii, Illinois,
%Indiana, Iowa, Kansas, Kentucky, Louisiana, Maine, Massachusetts,
%Michigan, Mississippi, Missouri, Montana, Nebraska, Nevada, New
%Hampshire, New Jersey, New Mexico, New York, North Carolina, Ohio,
%Oklahoma, Oregon, Pennsylvania, Rhode Island, South Carolina, South
%Dakota, Tennessee, Texas, Utah, Vermont, Virginia, Washington, West
%Virginia, Wisconsin, and Wyoming.

%We have filed in all 50 states now, but these are the only ones
%that have responded.

%As the requirements for other states are met, additions to this list
%will be made and fund raising will begin in the additional states.
%Please feel free to ask to check the status of your state.

%In answer to various questions we have received on this:

%We are constantly working on finishing the paperwork to legally
%request donations in all 50 states.  If your state is not listed and
%you would like to know if we have added it since the list you have,
%just ask.

%While we cannot solicit donations from people in states where we are
%not yet registered, we know of no prohibition against accepting
%donations from donors in these states who approach us with an offer to
%donate.

%International donations are accepted, but we don't know ANYTHING about
%how to make them tax-deductible, or even if they CAN be made
%deductible, and don't have the staff to handle it even if there are
%ways.

%Donations by check or money order may be sent to:

%Project Gutenberg Literary Archive Foundation
%PMB 113
%1739 University Ave.
%Oxford, MS 38655-4109

%Contact us if you want to arrange for a wire transfer or payment
%method other than by check or money order.

%The Project Gutenberg Literary Archive Foundation has been approved by
%the US Internal Revenue Service as a 501(c)(3) organization with EIN
%[Employee Identification Number] 64-622154.  Donations are
%tax-deductible to the maximum extent permitted by law.  As fund-raising
%requirements for other states are met, additions to this list will be
%made and fund-raising will begin in the additional states.

%We need your donations more than ever!

%You can get up to date donation information online at:

%http://www.gutenberg.net/donation.html

%
%***

%If you can't reach Project Gutenberg,
%you can always email directly to:

%Michael S. Hart <hart@pobox.com>

%Prof. Hart will answer or forward your message.

%We would prefer to send you information by email.

%
%**The Legal Small Print**

%
%(Three Pages)

%***START**THE SMALL PRINT!**FOR PUBLIC DOMAIN EBOOKS**START***
%Why is this ``Small Print!" statement here? You know: lawyers.
%They tell us you might sue us if there is something wrong with
%your copy of this eBook, even if you got it for free from
%someone other than us, and even if what's wrong is not our
%fault. So, among other things, this ``Small Print!" statement
%disclaims most of our liability to you. It also tells you how
%you may distribute copies of this eBook if you want to.

%*BEFORE!* YOU USE OR READ THIS EBOOK
%By using or reading any part of this PROJECT GUTENBERG-tm
%eBook, you indicate that you understand, agree to and accept
%this ``Small Print!" statement. If you do not, you can receive
%a refund of the money (if any) you paid for this eBook by
%sending a request within 30 days of receiving it to the person
%you got it from. If you received this eBook on a physical
%medium (such as a disk), you must return it with your request.

%ABOUT PROJECT GUTENBERG-TM EBOOKS
%This PROJECT GUTENBERG-tm eBook, like most PROJECT GUTENBERG-tm eBooks,
%is a ``public domain" work distributed by Professor Michael S. Hart
%through the Project Gutenberg Association (the ``Project").
%Among other things, this means that no one owns a United States copyright
%on or for this work, so the Project (and you!) can copy and
%distribute it in the United States without permission and
%without paying copyright royalties. Special rules, set forth
%below, apply if you wish to copy and distribute this eBook
%under the ``PROJECT GUTENBERG" trademark.

%Please do not use the ``PROJECT GUTENBERG" trademark to market
%any commercial products without permission.

%To create these eBooks, the Project expends considerable
%efforts to identify, transcribe and proofread public domain
%works. Despite these efforts, the Project's eBooks and any
%medium they may be on may contain ``Defects". Among other
%things, Defects may take the form of incomplete, inaccurate or
%corrupt data, transcription errors, a copyright or other
%intellectual property infringement, a defective or damaged
%disk or other eBook medium, a computer virus, or computer
%codes that damage or cannot be read by your equipment.

%LIMITED WARRANTY; DISCLAIMER OF DAMAGES
%But for the ``Right of Replacement or Refund" described below,
%[1] Michael Hart and the Foundation (and any other party you may
%receive this eBook from as a PROJECT GUTENBERG-tm eBook) disclaims
%all liability to you for damages, costs and expenses, including
%legal fees, and [2] YOU HAVE NO REMEDIES FOR NEGLIGENCE OR
%UNDER STRICT LIABILITY, OR FOR BREACH OF WARRANTY OR CONTRACT,
%INCLUDING BUT NOT LIMITED TO INDIRECT, CONSEQUENTIAL, PUNITIVE
%OR INCIDENTAL DAMAGES, EVEN IF YOU GIVE NOTICE OF THE
%POSSIBILITY OF SUCH DAMAGES.

%If you discover a Defect in this eBook within 90 days of
%receiving it, you can receive a refund of the money (if any)
%you paid for it by sending an explanatory note within that
%time to the person you received it from. If you received it
%on a physical medium, you must return it with your note, and
%such person may choose to alternatively give you a replacement
%copy. If you received it electronically, such person may
%choose to alternatively give you a second opportunity to
%receive it electronically.

%THIS EBOOK IS OTHERWISE PROVIDED TO YOU ``AS-IS". NO OTHER
%WARRANTIES OF ANY KIND, EXPRESS OR IMPLIED, ARE MADE TO YOU AS
%TO THE EBOOK OR ANY MEDIUM IT MAY BE ON, INCLUDING BUT NOT
%LIMITED TO WARRANTIES OF MERCHANTABILITY OR FITNESS FOR A
%PARTICULAR PURPOSE.

%Some states do not allow disclaimers of implied warranties or
%the exclusion or limitation of consequential damages, so the
%above disclaimers and exclusions may not apply to you, and you
%may have other legal rights.

%INDEMNITY
%You will indemnify and hold Michael Hart, the Foundation,
%and its trustees and agents, and any volunteers associated
%with the production and distribution of Project Gutenberg-tm
%texts harmless, from all liability, cost and expense, including
%legal fees, that arise directly or indirectly from any of the
%following that you do or cause:  [1] distribution of this eBook,
%[2] alteration, modification, or addition to the eBook,
%or [3] any Defect.

%DISTRIBUTION UNDER ``PROJECT GUTENBERG-tm"
%You may distribute copies of this eBook electronically, or by
%disk, book or any other medium if you either delete this
%"Small Print!" and all other references to Project Gutenberg,
%or:

%[1]  Only give exact copies of it.  Among other things, this
%     requires that you do not remove, alter or modify the
%     eBook or this ``small print!" statement.  You may however,
%     if you wish, distribute this eBook in machine readable
%     binary, compressed, mark-up, or proprietary form,
%     including any form resulting from conversion by word
%     processing or hypertext software, but only so long as
%     *EITHER*:

%     [*]  The eBook, when displayed, is clearly readable, and
%          does *not* contain characters other than those
%          intended by the author of the work, although tilde
%          (~), asterisk (*) and underline (\_) characters may
%          be used to convey punctuation intended by the
%          author, and additional characters may be used to
%          indicate hypertext links; OR

%     [*]  The eBook may be readily converted by the reader at
%          no expense into plain ASCII, EBCDIC or equivalent
%          form by the program that displays the eBook (as is
%          the case, for instance, with most word processors);
%          OR

%     [*]  You provide, or agree to also provide on request at
%          no additional cost, fee or expense, a copy of the
%          eBook in its original plain ASCII form (or in EBCDIC
%          or other equivalent proprietary form).

%[2]  Honor the eBook refund and replacement provisions of this
%   ``Small Print!" statement.

%[3]  Pay a trademark license fee to the Foundation of 20/% of the
%     gross profits you derive calculated using the method you
%     already use to calculate your applicable taxes.  If you
%     don't derive profits, no royalty is due.  Royalties are
%     payable to ``Project Gutenberg Literary Archive Foundation"
%     the 60 days following each date you prepare (or were
%     legally required to prepare) your annual (or equivalent
%     periodic) tax return.  Please contact us beforehand to
%     let us know your plans and to work out the details.

%WHAT IF YOU *WANT* TO SEND MONEY EVEN IF YOU DON'T HAVE TO?
%Project Gutenberg is dedicated to increasing the number of
%public domain and licensed works that can be freely distributed
%in machine readable form.

%The Project gratefully accepts contributions of money, time,
%public domain materials, or royalty free copyright licenses.
%Money should be paid to the:
%"Project Gutenberg Literary Archive Foundation."

%If you are interested in contributing scanning equipment or
%software or other items, please contact Michael Hart at:
%hart@pobox.com

%[Portions of this eBook's header and trailer may be reprinted only
%when distributed free of all fees.  Copyright (C) 2001, 2002 by
%Michael S. Hart.  Project Gutenberg is a TradeMark and may not be
%used in any sales of Project Gutenberg eBooks or other materials be
%they hardware or software or any other related product without
%express permission.]

%*END THE SMALL PRINT! FOR PUBLIC DOMAIN EBOOKS*Ver.02/11/02*END*

\end{document}